\documentclass{report}
\usepackage{amsmath,amssymb}
\setlength{\parindent}{0mm}
\setlength{\parskip}{1em}
\begin{document}
\begin{center}

\rule{6in}{1pt} \
{\large
J.G.Schmidt
\\ {\bf
On the Use of Algebraic Multigrid inside a Non-Linear Finite Element
Method for Maxillo-Facial Surgery Simulations
}}

C+C Research Labs, NEC Europe Ltd. \\
Rathausallee 10 \\
53757 Sankt Augustin \\
Germany 
\\ {\tt
schmidt@ccrl-nece.de
}
\\
G.Berti,
J.Fingberg 
\end{center}



In this paper we will present a computational tool chain for simulating
the outcome of the ostheogenetic distraction process of a
maxillo-facial surgery.

The main tool of this tool chain is a parallel Finite Element code for
non-linear, viscoelastic elasticity (FEBiNA), that has been developed
at the C+C Labs.

The Finite Element meshes used for this maxillo-facial surgery
simulation are based on patient specific tomography data and their
geometry is highly complicated. In addition to that the material
parameters for bony structures, like skull and teeth, differ by several
orders of magnitude from those for the soft tissues of the human head.
Finally, the latter materials are nearly incompressible.

Due to those difficulties, the systems of linear equations that have to
be solved during the simulation process are severly maleconditioned.
Our tests show that given our hardware environment, only multigrid
methods can handle those systems within a reasonable solution time.

Nevertheless, the algebraic multigrid solvers we are using do not
always show satisfying behaviour. We will show some numerical
experiments and point out the arising problems we encountered during
the linear solution phase.

By doing so, we hope to nurture a fruitful discussion on the use of
algebraic multigrid methods for these kinds of problems. 

\end{document}
