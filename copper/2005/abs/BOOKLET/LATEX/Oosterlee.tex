\documentclass{report}
\usepackage{amsmath,amssymb}
\setlength{\parindent}{0mm}
\setlength{\parskip}{1em}
\begin{document}
\begin{center}

\rule{6in}{1pt} \
{\large
Cornelis W. Oosterlee 
\\ {\bf
Multigrid for a segregated version of the poroelasticity system
}}

Delft University of Technology, DIAM \\
Mekelweg 4, 2628 CD Delft \\
the Netherlands
\\ {\tt
c.w.oosterlee@ewi.tudelft.nl
}
\\
Fransisco J. Gaspar,  Francisco J. Lisbona
\end{center}

Poroelasticity has a wide range of applications in biology, filtration,
tissue engineering and soil sciences. It represents a model for
problems where an elastic porous solid is saturated by a viscous fluid.
The poroelasticity equations were derived by Biot in 1941, studying the
consolidation of soils. In this talk we will present an efficient
solution method for the poroelasticity system. A staggered grid
discretization is adopted from incompressible flow and modified for the
poroelasticity system. Standard finite elements (or finite differences)
do not lead to stable solutions without additional stabilization. The
staggered grid discretization leads to a natural stable discretization.
In contrast to our previous work, where we provided multigrid
solvers for the whole system of poroelasticity equations, here we first
split the system into scalar equations. This splitting can be
interpreted as a segregated solution approach, similarly to
pressure-correction methods in incompressible fluid flow simulation. In
fact, we just reverse loops as compared to our previous work: The
distributive smoothing method from now acts as the "outer loop" in
the solution process. The method can also be seen as a form of
preconditioning of the original poroelasticity system. Next to a right-
we also present the left-preconditioned system and corresponding
results. In the segregated framework we need to solve for scalar
equations only, thus enabling the solution of three-dimensional
problems on relatively fine grids. Highly efficient multigrid schemes
are used to solve the resulting scalar equations. Next to numerical
experiments and analysis results we will present some theoretical
convergence results for the approach adopted. 

\end{document}
