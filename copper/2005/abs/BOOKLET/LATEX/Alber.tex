\documentclass{report}
\usepackage{amsmath,amssymb}
\setlength{\parindent}{0mm}
\setlength{\parskip}{1em}
\begin{document}
\begin{center}

\rule{6in}{1pt} \
{\large David Alber \\
{\bf Modifying CLJP Coarse Grid Selection to Attain Lower
Complexities}}

Siebel Center for Computer Science \\ University of Illinois at
Urbana-Champaign \\ 201 North Goodwin Avenue \\ Urbana IL 61801
\\
{\tt alber@uiuc.edu}
\end{center}

To build an efficient parallel algebraic multigrid (AMG) solver both
the setup phase and the solve phase need to be implemented with
parallel algorithms.  Parallelizing the latter is straightforward
because of the similarities between the AMG solve phase and geometric
multigrid.  However, parallelizing the original Ruge-St\"{u}ben (RS) setup
phase presents difficulties.  The RS coarse grid selection algorithm
uses a breadth first search for its first pass which makes the
algorithm inherently sequential.  Parallel coarse grid selection
algorithms had to be designed in order to make a parallel setup phase.

Of those algorithms which select coarse grids for use with RS
interpolation, Cleary-Luby-Jones-Plassmann (CLJP) is unique because it
produces the same coarse grid regardless of the number of processors on
which the coarsening algorithm is run.  Additionally, CLJP is able to
coarsen the grid to a single node.  Both of these properties have
positive implications in the design of a parallel setup phase.
Unfortunately, CLJP tends to select coarse grids which lead to high
operator complexities on many problems and especially high complexities
on problems in three dimensions.

This talk discusses methods to modify CLJP in order to produce grid
hierarchies which lead to lower operator complexities.  In particular,
methods to modify the maximal independent set algorithm on which CLJP
is based are examined.  By modifying this algorithm, CLJP can be made
to produce results with more evenly distributed coarse nodes.
Experimental results show that these changes lead to lower operator
complexities.  The eventual goal is to create CLJP-based algorithms
with performance close to that of the Falgout hybrid algorithm while
retaining the positive aspects of the original CLJP.


\end{document}
