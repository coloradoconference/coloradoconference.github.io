\documentclass{report}
\usepackage{amsmath,amssymb}
\setlength{\parindent}{0mm}
\setlength{\parskip}{1em}
\begin{document}
\begin{center}

\rule{6in}{1pt} \
{\large
Dimitri J. Mavriplis
\\ {\bf
Multigrid solution of the lattice Boltzmann equation
}}


Department of Mechanical Engineering \\
University of Wyoming \\
Laramie WY 82071 
\\ {\tt
mavripl@uwyo.edu
}
\end{center}

The last decade has seen rapid progress in the theoretical
understanding, algorithmic development, and use of lattice Boltzmann
methods (LBM). As opposed to traditional computational fluid dynamics
approaches, which compute macroscopic fluid dynamic properties by
discretizing the continuum governing equations, the independent
variables in a lattice Boltzmann approach consist of particle
distribution functions in phase space, from which macroscopic continuum
fluid properties can be recovered. Historically, LBM methods have been
derived from lattice gas automata (LGA) methods. LGA methods model
macroscopic fluid motion through the evolution of a set of discrete
particles on a regular lattice (cartesian grid). Many of the
deficiencies of LGA methods in reproducing accurate macroscopic
behavior were resolved by the LBM approach, by neglecting individual
particle motion, and adopting a particle distribution function
approach. However, LBM methods have retained this conceptual link to
particle methods such as LGA, which has inhibited the use of more
sophisticated numerical techniques.

In this work, we first describe the lattice Boltzmann method as a
particular finite-difference discretization in space and time of the
Boltzmann equation. The LBM time-step is then shown to be equivalent to
a first-order explicit time step (forward Euler), which is also
equivalent to a Jacobi iteration for the steady-state form of the LBM
equations. It is then shown how this iteration strategy may be used as
a solver for the steady-state lattice Boltzmann equation, or as a
solver for an implicit time-integration strategy. Finally, it is shown
how these problems may be solved more efficiently using the Jacobi
iteration as a smoother in a multigrid scheme for the steady-state
lattice Boltzmann equation. This requires under-relaxation of the
iteration scheme to achieve good high-frequency damping properties, and
a careful matching of the LBM discretizations on the coarse grid
levels, in order to ensure consistent coarse and fine grid problems.
The final result is a multigrid solver which can make use of a
pre-existing LBM routine in a modular fashion, by invoking the LBM
routine on each grid level. Convergence rates which are independent of
the mesh resolution are demonstrated for the driven cavity problem on a
cartesian grid, although the convergence rates degrade with increasing
Reynolds number.

[1] D.~J.~Mavriplis, {\em Multigrid Solution of the Steady-State Lattice
Boltzmann Equation}, paper delivered at International Conference for
Mesoscopic Methods in Engineering and Science (ICMMES), Braunschweig,
Germany, July 2004. To appear in Computers and Fluids, 2005. 

\end{document}
