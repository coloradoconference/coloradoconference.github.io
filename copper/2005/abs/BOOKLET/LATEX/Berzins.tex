\documentclass{report}
\usepackage{amsmath,amssymb}
\setlength{\parindent}{0mm}
\setlength{\parskip}{1em}
\begin{document}
\begin{center}

\rule{6in}{1pt} \
{\large
Martin Bedrzins
\\ {\bf
Efficient parallelisation of a multigrid multilevel integration EHL
solver
}}

SCI Institute, Utah
\\
Salt Lake City UT 84112
\\ {\tt
mb@sci.utah.edu
}
\\
Christopher Goodyer
\end{center}

The numerical solution of large scale elastohydrodynamic lubrication
(EHL) problems is only computationally realistic on fine meshes by
making using using multilevel techniques.  In this work we show how the
parallelisation of both multigrid and multilevel multi-integration for
these problems may be accomplished without damaging solution quality.
A parallel performance model of the implemented algorithm is described
and analysed using the Isoefficiency and Isomemory metrics for
distributed memory architectures.  Results are shown with good
speed-ups and excellent scalability.

Parallelisation of scientific engineering codes, such as the EHL code
considerd here, has proved to be particularly useful whenever either
results are needed quickly or the memory requirements are too large to
be handled in serial. In the case of solvers for the important
engineering problem of elastohydrodynamic lubrication both these
situtaions can arise. The EHL regime occurs in journal bearings and
gears, where, under severe loads in the presence of a lubricant, there
may be a very large pressure exerted on a very small area, often up to
3 GPa.  This causes the shape of the contacting surfaces to deform and
flatten out at the centre of the contact.  There are also significant
changes in the behaviour of the lubricant in this area, for example it
may take on glass-like properties.

The computational challenge in solving such problems is considerable.
The equations to be solved consist of a nonlinear differential equation
which is elliptic/hyperbolic and defined in terms of pressure film
thickness values and a coupled integral equation which defines the film
thickness in terms of all the spatial pressures.  The efficient serial
solution of these problems is achieved by using a multigrid solver for
the differential equation coupled to a multilevel multi-integration
method for the filmthickness calculation. Although the time dependent
partial differential and integral equations apply only in one or two
space dimensions, they have a dense sparsity pattern and are highly
nonlinear.  Full details of both the EHL problem and the serial
solution methods used are described in the book by Venner and Lubrecht
and with details specific to the discussion here given by the thesis of
Goodyer.

One  of the EHL problems of current interest is to calculate the
frictional characteristics of measured surface roughness profiles.
This has been successfully undertaken for one dimensional line contact
cases.  Tackling the more realistic 2-D case has been recognised as one
of the immediate challenges in tribology. In order to do this spatial
meshes of $10^6 \times 10^6$ points may be needed. This means that
$10^{12}$ dense nonlinear equations need to be solved. This challenge
is beyond a single workstation at present and requires the use of
parallel computers.

In order to describe the parallel solution techniques the numerical
problem to be solved and the serial algorithm will first be described.
The multigrid and multilevel techniques used will be highlighted, along
with the reasons why they make effective parallelisation such a
communication intensive process.  The parallel approaches we have taken
are then explained and a careful performance model constructed using
the isomemory and isoefficiency metrics.  This analysis  will show how
a demanding numerical problem, which is both highly intensive in terms
of communication, and requires global knowledge, has been successfully
parallelised.  Use of MPI has meant this implementation is portable
between both shared and distributed memory architectures.
Communication costs have been limited through use of non-blocking local
directives, and the memory requirements per process have been
significantly reduced.  The computationl results show the overall
speed-up of the code is excellent, especially on higher grid
resolutions.  The scalability has been shown to be similarly impressive
with comparable results when increasing the problem size and number of
processors whilst utilising the same coarsest multi-level
multi-integration level.  The paper concludes by considering future
directions in terms of solving still larger problems.



\end{document}
