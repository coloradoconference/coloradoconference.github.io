\documentclass{report}
\usepackage{amsmath,amssymb}
\setlength{\parindent}{0mm}
\setlength{\parskip}{1em}
\begin{document}
\begin{center}

\rule{6in}{1pt} \
{\large
Bert Seynaeve
\\ {\bf
Fourier-mode analysis of a multigrid method for PDEs with random
parameters}
}

Department of Computer Science \\
K.U.Leuven; Celestijnenlaan 200A \\
B-3001 Heverlee, Belgium
\\ {\tt
bert.seynaeve@cs.kuleuven.ac.be
}
\\
Stefan Vandewalle,
Bart Nicola\"{i}
\end{center}

We consider the numerical solution of elliptic and parabolic partial
differential equations with stochastic coefficients. Such equations
appear, e.g., in reliability problems. Various approaches exist for
dealing with such `uncertainty propagation' models: Monte Carlo
methods, perturbation techniques, variance propagation, etc. Here, we
deal with the stochastic finite element method (SFEM) [1]. This method
transforms a system of PDEs with stochastic parameters into a
stochastic linear system by means of a finite element Galerkin
discretization. The stochastic solution vector of this system is
approximated by a linear combination of deterministic vectors; the
coefficients are orthogonal polynomials in the random variables. Unlike
commonly used methods such as the perturbation method, the SFEM gives a
result that contains all stochastic characteristics of the solution. It
also improves Monte Carlo methods significantly because sampling can be
done after solving the system of PDEs.

In order to solve the discretized stochastic system that appears in the
SFEM, stochastic versions of iterative methods can be applied, and
their convergence can be accelerated by implementing them in a
multigrid context [2]. In the work we present here, the convergence
properties of these stochastic iterative methods and multigrid methods
are investigated theoretically: deterministic (local) Fourier-mode
analysis techniques are extended to the stochastic case by taking the
eigenstructure into account of a matrix that depends on the random
structure of the problem. This is equivalent to choosing an alternative
set of polynomial basis functions in the random variables, which
results in a decoupling of the original stochastic problem into a very
large number of deterministic problems of the same type. The
theoretical convergence rates that we obtain predict the results of
numerical experiments very well, and the results of the analysis can
also be used to design optimal stochastic multigrid algorithms.

[1] R.G. Ghanem and P.D. Spanos.
{\em Stochastic finite elements: a
spectral approach}. Springer-Verlag, New York, 1991.

[2] O.P. Le Ma\^{i}tre et.al.
{\em A multigrid solver for
two-dimensional stochastic diffusion equations}, Computer Methods in
Applied Mechanics and Engineering, Vol. 192, Iss. 41-42, 2003,
pp.4723--4744.

\end{document}
