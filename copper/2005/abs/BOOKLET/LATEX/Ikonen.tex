\documentclass{report}
\usepackage{amsmath,amssymb}
\setlength{\parindent}{0mm}
\setlength{\parskip}{1em}
\begin{document}
\begin{center}

\rule{6in}{1pt} \
{\large
Samuli Ikonen
\\ {\bf
Multigrid Methods for Pricing American Options under Stochastic
Volatility
}}

Department of Mathematical Information Technology \\
University of Jyv\"{a}skyl\"{a} \\
Finland
\\ {\tt
samikon@cc.jyu.fi
}
\\
Jari Toivanen
\end{center}

We study numerical methods for pricing American put options with
Heston's stochastic volatility model. This model leads to a two
dimensional parabolic partial differential equation with an early
exercise constraint. We perform the space discretization using a finite
difference method with a seven point stencil. Implicit time
discretizations lead to a sequence of linear complementarity problems
(LCPs).

We consider two approaches employing multigrid methods. The first
approach uses an operator splitting method [4]. The idea is to decouple
the system of linear equations and the early exercise constraint into
separate fractional time steps. In the first fractional step, a
convection diffusion type problem with a second-order cross derivative
is solve. In a multigrid method we use an alternating direction
smoother proposed by Oosterlee in [5]. In the second fractional step, a
simple update is performed so that the solution satisfies the early
exercise constraint.

The second approach is to solve the LCPs using a multigrid based on a
projected full approximation scheme (PFAS) proposed by Brandt and Cryer
in [1]. The papers [3,5] consider such multigrids for pricing American
options. We study the use of the Brennan and Schwartz algorithm [2] in
the line smoothing in these multigrids.

[1] A.~Brandt, C.W.~Cryer,
{\em Multigrid Algorithms for the Solution of
Linear Complementarity Problems Arising from Free Boundary Problems},
SIAM Journal on Scientific and Statistical Computing, 4(1983), 655--684.

[2] M.J.~Brennan, E.S.~Schwartz,
{\em The Valuation of American Put Options},
Journal of Finance, 32(1977), 449--462.

[3] N.~Clarke, K.~Parrott,
{\em Multigrid for American Option Pricing with
Stochastic Volatility}, Applied Mathematical Finance, 6(1999), 177--195.

[4] S.~Ikonen, J.~Toivanen,
{\em Operator Splitting Methods for Pricing
American Options with Stochastic Volatility}, Report B11/2004,
Department of Mathematical Information Technology, University of
Jyv\"{a}skyl\"{a}, 2004.

[5] C.W.~Oosterlee,
{\em On Multigrid for Linear Complementarity Problems
with Application to American-style Options}, Electronic Transactions on
Numerical Analysis, 15(2003), 165--185.

\end{document}
