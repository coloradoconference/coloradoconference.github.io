\documentclass{report}
\usepackage{amsmath,amssymb}
\setlength{\parindent}{0mm}
\setlength{\parskip}{1em}
\begin{document}
\begin{center}

\rule{6in}{1pt} \
{\large
Cristian R. Nastase
\\ {\bf
High-Order Spectral $hp$-Multigrid Methods on Unstructured Grids
}}


Department of Mechanical Engineering \\
University of Wyoming \\
1000 E. University Ave. \\
Laramie WY 82071-3295 
\\ {\tt
nastase@uwyo.edu
}
\\
Dimitri J. Mavriplis
\end{center}

The development of optimal, or near optimal solution strategies for
higher-order discretizations, including steady-state solutions
methodologies, and implicit time integration strategies, remains one of
the key determining factors in devising higher-order methods which are
not just competitive but superior to lower-order methods in overall
accuracy and efficiency. The goal of this work is to investigate and
develop a fast and robust algorithm for the solution of high-order
accurate Discontinuous Galerkin discretizations of non-linear systems
of conservation laws on unstructured grids. Herein we extend the
spectral multigrid approach described in [1] to the two-dimensional
steady-state Euler equations, and couple the spectral $p$-multigrid
approach with a more traditional agglomeration $h$-multigrid method for
unstructured meshes. The investigation of efficient smoothers to be
used at each level of the multigrid algorithm is also pursued, and
comparisons between linear and non-linear solver strategies are made as
well [2]. The overall goal is the development of a solution algorithm
which delivers convergence rates which are independent of ``$p$" (the
order of accuracy of the discretization) and independent of ``$h$" (the
degree of mesh resolution), while minimizing the cost of each
iteration.

The computational domain is partitioned into an ensemble of
non-overlapping unstructured elements and within each element the
solution is approximated by a truncated polynomial expansion. The
semi-discrete formulation employs a local discontinuous Galerkin
formulation in spatial variables within each element. Thus, the
solution approximation is local, discontinuous, and doubled valued on
each elemental interface. Monotone numerical fluxes are used to resolve
the discontinuity, providing the means of communication between
adjacent elements and specification of the boundary conditions. An
approximate Riemann solver is used to compute the flux at inter-element
boundaries. The discrete form of the local discontinuous Galerkin
formulation is defined by the particular choice of the set of basis
functions. Hereby a set of hierarchical basis functions is used in
order to simplify our subsequent spectral multigrid implementation. The
full description of the basis function set is given in [3]. The
resulting set of equations is solved in the modal space and the
integrals are evaluated by Gaussian quadrature rules.

The spectral $p$-multigrid approach is based on the same concepts as a
traditional $h$-multigrid method, but makes use of ``coarser'' levels
which are constructed by reducing the order of accuracy of the
discretization, rather than using physically coarser grids with fewer
elements. Furthermore, the formulation of the interpolation operators,
between fine and coarse grid levels, is greatly simplified when a
hierarchical basis set is employed for the solution approximation. The
main advantage is due to the fact that the lower order basis functions
are a subset of the higher order basis (i.e. hierarchical) and the
restriction and prolongation operators become simple projection
operators into a lower and higher order space, respectively [4]. For
pure $p$-multigrid methods, the recursive application of lower order
discretizations ends with the $p=0$ discretization on the same grid as
the fine level problem. For relatively fine meshes, the (exact)
solution of this $p=0$ problem at each multigrid cycle can become
expensive, and may impede the $h$-independence property of the multigrid
strategy. The $p=0$ problem can either be solved approximately by
employing the same number of smoothing cycles on this level as on the
finer $p$-levels, or it can be solved more accurately by performing a
larger number of smoothing cycles at each visit to this coarsest level.
In either case, the convergence efficiency will be compromised, either
due to inadequate coarse level convergence, or due to excessive coarse
level solution cost. An alternative is to employ an $h$-multigrid
procedure to solve the coarse level problem at each multigrid cycle. In
this scenario, the $p$-multigrid scheme reverts to an agglomeration
multigrid scheme once the $p=0$ level has been reached, making use of a
complete sequence of physically coarser agglomerated grids, thus the
designation $hp$-multigrid. First-order accurate (p=0) agglomeration
multigrid methods for unstructured meshes are well established [5] and
deliver near optimal convergence rates. Therefore, this procedure has
the potential of resulting in a truly $h$- and $p$-independent solution
strategy for high-order accurate discontinuous Galerkin
discretizations.

At each $h$- or $p$-multigrid level an element-Jacobi type scheme is used
as a smoother. The element-Jacobi scheme can be viewed as an
approximate Newton scheme where the full Jacobian matrix is replaced by
the block diagonal entries representing the coupling between all modes
within each element, [D], thus neglecting the coupling between
neighboring element modes, which arises through the inter-element flux
evaluations. The [D] blocks represent small dense matrices associated
with each grid element. A second variant of this solver is denoted as
the ``linearized" element-Jacobi method. In this approach, the full
Jacobian matrix is retained, but is decomposed into block diagonal [D]
and off-diagonal [O] components. Note that the linearized element
Jacobi scheme involves a dual iteration strategy, where each nth outer
non-linear iteration entails ``k" inner linear iterations. The
advantage of this formulation is that the non-linear residual and the
Jacobian entries, [D] and [O], are held constant during the linear
iterations. This can significantly reduce the required computational
time per cycle for expensive non-linear residual constructions. Because
this scheme represents an exact linearization of the element-Jacobi
scheme both approaches can be expected to converge asymptotically at
the same rates per cycle [6]. The convergence can be further
accelerated by using a Gauss-Seidel strategy where the off-diagonal
matrices are split into lower, [L], and upper, [U] contributions (i.e.
[O]=[L]+[U]). This last solver variant, again, involves a dual
iteration strategy, but follows an ordered sweep across the elements
using latest available neighboring information in the Gauss-Seidel
sense. In this work, we employ a frontal sweep along the elements which
begins near the inner boundary and proceeds toward the outer boundary,
using the numbering assigned to the grid elements from an advancing
front mesh generation technique.

The resulting $hp$-multigrid scheme demonstrates $p$-independent and nearly
$h$-independent convergence rates. The coupling of $p$- and $h$-multigrid
procedures, through the use of agglomerated coarse levels for
unstructured meshes, increases the overall solution efficiency compared
to a $p$-alone multigrid procedure, and the benefits of the $hp$-multigrid
approach can be expected to increase for finer meshes. The multigrid
procedure can itself be applied as a non-linear solver (FAS), or as a
linear solver (CGC) for a Newton scheme applied to the non-linear
problem. The linear multigrid approach demonstrates superior overall
efficiency, provided a suitable linear iteration termination strategy
is employed. Results are presented for compressible flow over a bump in
a channel and for flow over a four element airfoil in two dimensions.


[1] Helenbrook, B. and Mavriplis, D. J. and Atkins, H.,
{\em Analysis of
``p''-Multigrid for Continuous and Discontinuous Finite Element
Discretizations}, Proceedings of the 16th AIAA Computational Fluid
Dynamics Conference, 2003, AIAA Paper 2003-3989.

[2] Nastase, C.R. and Mavriplis, D.J.,
{\em High-Order Discontinuous
Galerkin Methods using a Spectral Multigrid Approach}, Proceedings of
the 43rd AIAA Aerospace Sciences Meeting and Exhibit, 2005, AIAA Paper
2005-1268.

[3] Solin, P. and Segeth, P. and Zel, I.D.,
{\em High-Order Finite Element Methods}, Chapman and Hall, 2003.

[4] Fidkowski, K.J. and Darmofal, D.L.,
{\em Development of a
Higher-Order Solver for Aerodynamic Applications}, Proceedings of the
42nd AIAA Aerospace Sciences Meeting and Exhibit, 2004, AIAA Paper
2004-0436.

[5] Mavriplis, D. J. and Venkatakrishnan, V.,
{\em Agglomeration Multigrid for Two Dimensional Viscous Flows},
Computers and Fluids, 24(5):553--570, 1995.

[6] Mavriplis, D.J.,
{\em An Assessment of Linear versus Non-Linear
Multigrid Methods for Unstructured Mesh Solvers},
J.~Comput.~Phys., 175:302--325, 2002. 

\end{document}
