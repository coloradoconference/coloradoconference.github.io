\documentclass{report}
\usepackage{amsmath,amssymb}
\setlength{\parindent}{0mm}
\setlength{\parskip}{1em}
\begin{document}
\begin{center}

\rule{6in}{1pt} \
{\large
Marian Brezina 
\\ {\bf
Application of the Adaptive Smoothed Aggregtion to Problems with
Nonsmooth Kernels
}}

Dept. of Applied Mathematics
\\
University of Colorado
\\
Boulder CO 80309-0526 
\\ {\tt
mbrezina@math.cudenver.edu
}
\\
J.~Brannick,
R.~Falgout,
S.~MacLachlan,
T.~Manteuffel,
S.~McCormick,
J.~Ruge
\end{center}

The success of multigrid methods relies on complementarity between the
relaxation processs and the coarse-grid correction. Multigrid methods
are usually designed based on the assumption that the error that needs
to be eliminated by the coarse-grid correction possesses smoothness.

We discuss several applications of practical interest where these
assumptions are violated, so standard approaches cannot be successfully
applied. We describe the recently developed extension of the smoothed
aggregation method, designed to identify the critical error components
and to incorporate them into the coarse space design. The presented
numerical experiments demonstrate that the method can be used to
achieve good convergence. 

\end{document}
