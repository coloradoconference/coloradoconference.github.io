\documentclass{report}
\usepackage{amsmath,amssymb}
\setlength{\parindent}{0mm}
\setlength{\parskip}{1em}
\begin{document}
\begin{center}

\rule{6in}{1pt} \
{\large
Krzysztof J. Fidkowski
\\ {\bf
$p$-Multigrid Solution of High-Order Discontinuous Galerkin
Discretizations of the Euler and Compressible Navier-Stokes Equations
}}

Massachusetts Institute of Technology \\
77 Mass. Ave \\
MIT 37-442 \\
Cambridge MA 02139
\\ {\tt
kfid@mit.edu
}
\\
David L. Darmofal
\end{center}

In this talk, we focus on a $p$-multigrid solution algorithm for
high-order discontinuous Galerkin (DG) finite element discretizations
of the Euler and compressible Navier-Stokes equations. $p$-Multigrid, or
multi-order, solution strategies have been studied by other authors,
including Helenbrook, Mavriplis, and most recently Bassi and Rebay.
Common features of this method for high-order DG include ease of
implementation and order-independent convergence rates. A key aspect of
our $p$-multigrid implementation is the use of an element-line Jacobi
smoother instead of the standard element-Jacobi. The element-line
Jacobi smoother consists of solving implicitly on lines of elements
formed using coupling based on a $p=0$ discretization of the scalar
convection-diffusion equation. This choice of elemental coupling allows
for the removal of stiffness associated with strong convection or
regions of high grid anisotropy frequently required in viscous layers.
A line creation algorithm is presented for general unstructured meshes,
showing how unique lines can be obtained in two and three dimensions
for a given elemental coupling.

Using Fourier analysis for scalar convection-diffusion, we demonstrate
that the higher-order DG discretizations can be stably marched for all
orders with element Jacobi and element-line Jacobi schemes without the
use of multi-stage iterations. $p$-Multigrid is then applied with the
element-line smoother to inviscid and viscous test cases, in two and
three dimensions. Results demonstrate optimal order of accuracy of the
discretization, as well as $p$-independent multigrid convergence rates.
$h$-dependence is observed, although it is not found to be strong for
many practical problems. Finally, for the smooth problems considered,
$p$-refinement outperforms $h$-refinement in terms of the time required to
reach a desired high accuracy level.

\end{document}
