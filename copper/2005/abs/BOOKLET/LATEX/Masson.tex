\documentclass{report}
\usepackage{amsmath,amssymb}
\setlength{\parindent}{0mm}
\setlength{\parskip}{1em}
\begin{document}
\begin{center}

\rule{6in}{1pt} \
{\large
Masson Roland
\\ {\bf
Block Preconditioners with Algebraic Multigrid Block Solve in
Stratigraphic Modeling for oil exploration
}}

Institut Francais du Petrole \\
1 et 4 Avenue Bois Preau, 92000 Rueil \\
Malmaison, France
\\ {\tt
roland.masson@ifp.fr 
}
\\
Gervais Veronique
\end{center}

Stratigraphic models simulate the erosion and sedimentation of
sedimentary basins at geological time scales given the sea level
variations, the tectonics displacements of the basement, and the
sediments fluxes at the boundary of the basin. We consider in this talk
a sediments transport model coupling three main processes: a gravity
driven transport of the sediments for which the fluxes are proportional
to the gradient of the topography, a weather limited transport taking
into account the disymmetry between erosion and sedimentation, and a
fluvial transport model for which the sediments fluxes are proportional
to the water discharge. The main variables of the problem are the
sediment thickness, a flux limitor, and the $L$ sediments concentrations
in basic lithologies such as sand or shale or carbonates. Such model is
applied in oil exploration for a better prediction of potential
reservoirs location. The model is derived writing the mass conservation
of each lithology leading to a system of mixed parabolic hyperbolic
type. It is discretized by a finite volume scheme in space and a fully
implicit time integration, leading to the solution at each time step of
a non linear systems of $L+2$ variables on the 2D mesh.

After Newton type
linearization, we are left with the solution of an ill conditioned
linear system with sharp jumps in the diffusion coefficients and
coupling $L+2$ variables of mixed types. These linear systems are solved
using an iterative solver and a block approach for the preconditioner
in order to separate the different variables. In a first step, a
mixture equation is obtained by linear combinations of the rows that
should concentrate the ellipticity of the system and as much as
possible decouple the first two variables (sediment thickness and flux
limitor) from the $L$ concentrations variables. Then the overall system
is solved using a block Gauss Seidel preconditioner. The First block
(sediment thickness and flux limitor variables) is preconditioned
either by a direct sparse solver or by an ILU0 incomplete
factorization, or by a vcycle of an algebraic multigrid solver (AMG1R5
from Ruge and Stuben) on the sediment thickness sub-block only in order
to avoid non diagonal dominance. The remaining blocks are solve using a
gauss seidel sweep in topographical order leading to an exact inversion
of these blocks. These block preconditioners are compared with a sparse
direct solver and an ILU0 incomplete factorization on the overall
system. The comparison is made in terms of CPU time, and scalability
with respect to the mesh size and the jump of the diffusion
coefficients on two real test cases: the Paris basin and a Rift test
case with water discharge. It shows that the block approach combined
with a multigrid preconditioning of the the sediment thickness
sub-block is a very efficient method, nearly scalable for this problem.


\end{document}
