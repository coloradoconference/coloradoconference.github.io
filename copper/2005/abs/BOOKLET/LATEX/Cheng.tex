\documentclass{report}
\usepackage{amsmath,amssymb}
\setlength{\parindent}{0mm}
\setlength{\parskip}{1em}
\begin{document}
\begin{center}

\rule{6in}{1pt} \
{\large
Zhen Cheng
\\ {\bf
Effective adaptive multigrid for strongly anisotropic problems with
Krylov smoothers
}}

Department of Computer Science
\\
University of Illinois
\\
Urbana IL 61801
\\ {\tt
zcheng@uiuc.edu
}
\\
Eric de Sturler
\end{center}

We consider the parallel solution of strongly anisotropic diffusion
problems on adaptive grids by multigrid methods. The standard multigrid
methods with pointwise relaxation and standard coarsening have problems
because smoothers are not effective. To improve multigrid, the common
ways are to use semicoarsening or line smoothers. However, neither of
these techniques is suitable for parallel adaptive grids environment.
Moreover, both techniques require that the anisotropy aligned with the
grid, which may not be the case for realistic problems. In this talk,
we will discuss a robust and easy-to-parallelize smoother for multigrid
which can deal with strongly anisotropic problems on adaptive grids
effectively. This method remains effective even if the anisotropy is
not aligned with the grid.

We use multilevel adaptive technique (MLAT). The main idea of MLAT is
to perform smoothing sweeps only on locally refined grids, and use the
full approximation scheme (FAS) to generate the error correction cycle.
We propose to use Krylov subspace methods as smoothers. Our numerical
experiments show that they reduce oscillatory error components
effectively. Therefore with such smoothers, multigrid achieves fast
convergence rate. Moreover, parallelizing Krylov methods is
straightforward since only matrix vector products and vector inner
products need communication.

Convergence rate analysis for multigrid on adaptive grids can be
simplified to analysis on uniform grids, since with properly chosen
interpolation and prolongation operators, convergence rate on adaptive
grids is almost identical to that on uniform grids. Standard analytic
tools such as Local Fourier Analysis (LFA) fail for Krylov smoothers
because they require the smoothing operators to be linear. In addition,
Krylov methods are not ``strict'' smoothers. Their smoothing effect for
high frequency modes may deteriorate when large smooth error components
are present. In our work, we use a slightly different approach. Assume
the coarse grid correction operator (including interpolation and
restriction) satisfies certain requirements, we derive the
level-independent upper bound for convergence rate of Krylov methods.
This rate is used to estimate multigrid convergence rate. This explains
why level-independent convergence rate of multigrid is achieved. The
numerical experiments verify our statement. Our approach of
quantitative analysis can be applied to more general problems and may
be useful for other multigrid practitioners.

This work is part of IBEAM project which is sponsored under a Round III
Grand Challenge Cooperative Agreement with NASA's Computational
Technologies Project. 

\end{document}
