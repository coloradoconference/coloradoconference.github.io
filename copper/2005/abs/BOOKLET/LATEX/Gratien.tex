\documentclass{report}
\usepackage{amsmath,amssymb}
\setlength{\parindent}{0mm}
\setlength{\parskip}{1em}
\begin{document}
\begin{center}

\rule{6in}{1pt} \
{\large
Jean-Marc Gratien
\\ {\bf
Using parallel algebraic multi-grid preconditioner in an industrial
reservoir simulation software
}}

Institut Fran\c{c}ais du P\'{e}trole
\\
1 et 4 avenue Bois Pr\'{e}au
\\
92852 Rueil-Malmaison Cedex
\\
France
\\ {\tt
j-marc.gratien@ifp.fr
}
\\
Pierre Bonneau,
Rolland Masson,
Phillippe Quandalle
\end{center}

As full field reservoir simulations require a large amounts of
computing resources, the trend is to use parallel computing to overcome
hardware limitations. This paper presents the linear solvers and the
preconditioning techniques applied to solve very large scale problems
in the reservoir simulation software developed at IFP, software which
has been specially designed for Linux clusters. We discuss on the
choice of different efficient preconditioners. We report the
scalability of the parallel simulator and numerical stability of
underlying algorithms calculated from a test campaign on either
synthetical or real industrial study case.

The reservoir simulator is based on a system of partial differential
equations with algebraic closure laws discretized with a finite volume
scheme in space and a fully implicit or semi-implicit time integration.
After a Newton type linearization of the non linear system, we are
left, at each Newton iteration, with the solution of a large, ill
conditioned linear system coupling several unknowns of mixed parabolic
and hyperbolic type. Progress of computer performances now allows to
better take into account physical phenomenon like strong and very
strong rock heterogeneities and anisotropies in reservoir modeling so
that the complexity of reservoir simulations keeps on increasing with
larger meshes, and more difficult problems. This in turn urges the need
to design more efficient and more scalable preconditioned iterative
solvers, adapted to the new parallel architectures, in order to fully
benefit from the computers performances on such complex cases.

Due to the high condition number of the linear system, the convergence
of the Newton algorithm can be very sensitive to the choice of the
preconditioner, to the linear system stopping criteria and even to the
number of processors in parallel implementations. In such cases, the
accuracy of the solver is critical to ensure parallel performance as
the cumulative number of steps (solver resolution, Newton loops, time
loops) can be very dependent on the number of domains and as the cost
of a run depends on this cumulative number of steps. This shows the
importance of choosing a stable numerical scheme, good solver options
and robust parallel preconditionner to have a numerically stable
simulation. This difficulty emphasises the special importance of using
very stable numerical schemes during parallel simulation. Solver
options may have important influence on convergence criteria, and the
choice of a good parallel preconditioner can help to have stable
convergence iteration numbers.

In the standard sequential simulator, linear systems are efficiently
solved by the Bi Conjugated Gradient Stabilised with an incomplete ILU0
preconditionner. This preconditioner, well known to be efficient for
standard case is not naturally parallel as its algorithm is recursive.

We have developed a new parallel ILU0 preconditionner which turns to be
a good scalable preconditionner. However, it can encounter difficulties
on complex industrial cases with very complex geometry and physical
models. ILU0 is not scalable with respect to the size of the mesh and
the jumps of the permeability field. Multigrid methods are known to be
scalable for scalar convection diffusion problems but it is also known
that they are not adapted to systems coupling unknowns of mixed types.
Our approach is to define a pressure block that will concentrate most
of the bad conditioning and ellipticity of the system, and for which we
shall used a multigrid preconditioner. The definition of this block is
obtained by linear combination of the rows and columns in order to
reduce the coupling with the other variables (compositions and
concentrations). For the remaining variables and equations we use a
block Gauss Seidel approach. In some cases, this approach does not
ensure a sufficient coupling of the variables, and it needs to be
combined mutiplicatively with a global ILU0 preconditioner.

In reservoir modelling, the linear systems couple a pressure unknown of
elliptic or parabolic type to concentrations or saturations unknowns of
hyperbolic type. In some industrial cases, especially when there are a
lot of heterogeneities and anisotropies, we have noticed that the
variations of saturation are very sensitive to pressure gradient so
that having a robust preconditionner for the pressure is very
important. As a consequence, Algebraic Multigrid Methods have to be
considered as good candidates for preconditioning. Thus, we have
developed 2 AMG based preconditioners . The first approach is a Block
Aggregation AMG. It aims at taking advantage of the natural block
structure of the system when the unknowns pressure and saturation are
grouped cell by cell. This strategy turns out to be effective when a
block ILU0 smoother is used in the multigrid cycle. The second approach
called ``Two level AMG'' combines an algebraic multigrid method on
pressure unknowns with a more traditional parallel method on other
variables (ILU0, Block ILU0, polynomial). This results in a new method
which is robust, parallel and efficient

We have tested these methods (Parallel ILU0, Two level AMG and Block
Aggregation AMG) and studied their influence on the performance and on
the numerical stability of the reservoir simulator for different study
case. We discussed the results obtained on a synthetical study case,
the Tenth SPE Comparative Solution Project, Model 2 and on some real
industrial customer study cases.

\end{document}
