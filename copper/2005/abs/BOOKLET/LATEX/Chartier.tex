\documentclass{report}
\usepackage{amsmath,amssymb}
\setlength{\parindent}{0mm}
\setlength{\parskip}{1em}
\def\mathbi#1{\textbf{\em #1}}
\begin{document}
\begin{center}

\rule{6in}{1pt} \
{\large
Tim Chartier
\\ {\bf
Algebraic Multigrid Schemes
}}

Department of Mathematics
\\
Davidson College
\\
Davidson NC 28035-6908
\\ {\tt
tichartier@davidson.edu
}
\\
Edmond Chow
\end{center}

Considerable efforts in recent multigrid research have concentrated on
algebraic multigrid schemes. A vital aspect of this work is uncovering
algebraically smooth error modes in order to construct effective
multigrid components. Many existing algebraic multigrid algorithms rely
on assumptions regarding the nature of algebraic smoothness. For
example, a common assumption is that smooth error is essentially
constant along `strong connections'. Performance can degrade as smooth
error for a problem differs from this assumption. Through tests on the
homogeneous problem
($\mathbi{Ax} = \mathbf{0}$)
adaptive multigrid methods expose
algebraically smooth error.

The method presented in this talk uses relaxation and subcycling on
complementary grids as an evaluative tool in correcting multigrid
cycling. Each complementary grid is constructed with the intent of
dampening a subset of the basis of algebraically smooth error. The
particular implementation of this framework manages smooth error in a
manner analogous to spectral AMGe. Numerical results will be included.


\end{document}
