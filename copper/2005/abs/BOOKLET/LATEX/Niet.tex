\documentclass{report}
\usepackage{amsmath,amssymb}
\setlength{\parindent}{0mm}
\setlength{\parskip}{1em}
\begin{document}
\begin{center}

\rule{6in}{1pt} \
{\large
Arie de Niet
\\ {\bf
Multilevel Preconditioners in Thermohaline Ocean Circulation
}}

Research Institute for Mathematics and Computing Science
\\
University of Groningen
\\
P.O.Box 800 \\
9700 AV Groningen \\
The Netherlands
\\ {\tt
a.c.de.niet@math.rug.nl
}
\\
Fred Wubs
\end{center}

The numerical continuation of steady states in thermohaline ocean
circulation requires the solution of large systems. The (linearized)
equations describing the thermohaline ocean circulation (for a detailed
description of these equations see [5]) can be splitted in equations
for velocities in longitudinal and latitudinal direction ($x_{uv}$),
velocity in vertical direction ($x_w$), pressure ($x_p$) and salinity and
temperature ($x_{ST}$). The equations are coupled in the following way,
$$ \left( \begin{array}{cccc}
A_{uv} & 0 & G_{uv} & 0 \\
0 & 0 & G_w & B_{ST} \\
D_{uv} & D_w & 0 & 0 \\
B_{uv} & B_w & 0 & A_{ST}
\end{array}\right)
\left( \begin{array}{c}
	x_{uv} \\ x_{w} \\ x_{p} \\ x_{ST}
	\end{array}\right)
= \left( \begin{array}{c}
	b_{uv} \\ b_{w} \\ b_{p} \\ b_{ST}
	\end{array}\right)
$$
where $G*$ represents the discrete gradient operator and $D*$ the discrete
divergence operator. We assume that $G*T=D*$, i.e. the discrete operators
are each others transpose. Furthermore, $A_{ST}$ is a convection-diffusion
matrix, $A_{uv}$ is also a convection-diffusion matrix, but includes a
Coriolis force. Respectively the four equations describe conservation
of momentum (in longitudinal and latitudinal direction), the
hydrostatic pressure, conservation of mass and conservation of salt and
heat.

Due to the size of the systems they are solved via a Krylov subspace
method using a preconditioner to speed up the convergence. In general
there are two ways to obtain a preconditioner for the matrix in the
equation. One could try to exploit the structure of the system or apply
a general black-box preconditioning method like an incomplete LU
factorization. We will do both and compare the results.

In the current code for numerical simulation of thermohaline ocean
circulation (THCM, see [5], the systems are solved with the MRILU [1]
preconditioner. MRILU is a multilevel ILU method containing ideas
similar to algebraic multigrid methods. However MRILU doesn't need any
smoothing steps, because of sophisticated dropping- and lumping
criteria. In THCM MRILU is applied to the clustered equations, that is
the six unknowns ($u$,$v$,$w$,$p$,$T$,$S$),
that belong to one and the same cell,
are treated as one unknown. The method is able to construct a good
factorization and the Krylov subspace method converges fast.
Unfortunately the preconditioner requires a lot of construction time
and memory, which becomes a problem if the size of the problem is
increased.

In order to reduce the memory usage, we developed an alternative
preconditioner that exploits the structure. The ingredients are: a
splitting of the pressure in a depth-averaged pressure and a component
perpendicular to that; a non-symmetric reordering of the matrix;
dropping the block $B_{ST}$. This together gives a block-Gauss-Seidel
preconditioner, that requires some matrix-vector products and the
solution of three relatively easy systems: (i) a depth-averaged saddle
point system based on $A_{uv}$, $G_{uv}$ and $D_{uv}$, (ii) a
convection-diffusion-Coriolis equation involving $A_{uv}$ and (iii) a
convection-diffusion equation involving $A_{ST}$. The last two problems can
be handled very well by MRILU. For the saddle point system we can apply
any of the preconditioners from literature [3,4] or the ones we
developed ourselves based on artificial compressibility and
grad-div-stabilization [2] respectively.

A comparison of the structured preconditioner with MRILU applied to the
subsystems and MRILU applied to the clustered matrix at once shows that
the required amount of memory is reduced with almost a factor 10. We
have to pay for that in an increase of the number of iterations in the
Krylov subspace method, but this factor is at most 2. Overall we see a
serious speed up.

On the conference we will describe the block-Gauss-Seidel
preconditioner and show the results of a comparison using both MRILU
and an algebraic multigrid method as a subsolver in the preconditioner.

[1]
E. F. F. Botta and F. W. Wubs.
{\em Matrix renumbering ILU: an effective algebraic multilevel ILU
preconditioner for sparse matrices}.
SIAM J.~Matrix~Anal.~Appl., {\bf 20(4)}:1007--1026 (electronic),
1999.
Sparse and structured matrices and their applications
(Coeur d'Alene, ID, 1996).

[2]
Arie C. de Niet and Fred W. Wubs.
{\em Two preconditioners for the saddle point
equation}.
In Proceedings of the European Congres on
Computational Methods in Applied Sciences
and Engineering (ECCOMAS,
Jyv\"{a}skyl\"{a}, 2004).
See http://www.mit.jyu.fi/eccomas2004/.

[3]
Howard C. Elman, David J.
Silvester, and Andrew J. Wathen.
{\em Performance and analysis of
saddle point preconditioners
for the discrete steady-state
Navier-Stokes equations}.
Numer. Math., {\bf 90(4)}:665a--688, 2002.

[4]
C. Vuik and A. Saghir.
{\em The Krylov accelerated SIMPLE(R) method for incompressible flow}.
Report 02-01, Delft University of Technology, Department of Applied
Mathematical Analysis, Delft, 2002.

[5]
Weijer Wilbert, Henk A.  Dijkstra, Hakan Oksuzoglu, Fred W.  Wubs, and
Arie C. de Niet.
{\em A fully-implicit model of the global ocean circulation}.
J.~Comp.~Phys., {\bf 192}:452--470, 2003.

\end{document}
