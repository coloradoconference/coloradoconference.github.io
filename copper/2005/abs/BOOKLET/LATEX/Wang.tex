\documentclass{report}
\usepackage{amsmath,amssymb}
\setlength{\parindent}{0mm}
\setlength{\parskip}{1em}
\begin{document}
\begin{center}

\rule{6in}{1pt} \
{\large
Li Wang
\\ {\bf
Implicit Solution of High-Order Accurate Discontinuous Galerkin
Discretizations of the Unsteady Wave Equation Using Spectral Multigrid
}}


Department of Mechanical Engineering \\
University of Wyoming \\
1000 E. University Ave. \\
Laramie, WY 82071-3295
\\ {\tt
wangli@uwyo.edu
}
\\
Dimitri J. Mavriplis
\end{center}

The use of high-order accurate spatial discretizations for simulating
steady and unsteady fluid flows has become more widespread over the
last decade, For unsteady simulations using high-order spatial
discretizations with a wide disparity between temporal and spatial
scales, higher-order implicit time-integration approaches are
desirable. However, such approaches require the efficient solution of a
non-linear problem at each time-step in order to remain competitive. In
this work, we investigate the use of spectral multigrid methods for
driving implicit time-integration schemes of high-order discontinuous
Galerkin discretizations, using the linear two-dimensional wave
equation as a model problem.

The method of lines is employed, in which the wave equation is first
discretized in space, resulting in a large set of coupled ordinary
differential equations, which are then discretized and integrated in
time. The Discontinuous Galerkin (DG) method represents a spatial
discretization approach based on a finite-element method, which makes
use of element based basis functions which are discontinuous across
element interfaces [1]. In this approach, the computational domain is
partitioned into an ensemble of non-overlapping elements and within
each element the solution is approximately by a truncated polynomial
expansion. The solution is thus determined by the modal coefficients of
the expansion in terms of the basis functions within each element.
Because the resulting solution representation is discontinuous across
element interfaces, an upwind numerical flux function is used to
resolve the discontinuity at element interfaces. The current
implementation uses a set of hierarchical basis functions on triangles
[2], enabling solutions from 1st order ($p=0$), up to 4th order
($p=3$) spatial accuracy.

In order to solve the time-dependent problem, the resulting spatially
discretized equations must be integrated in time. Although explicit
time-integration schemes have been widely used for DG discretizations,
in this work we concentrate on the use of implicit time-integration
schemes, which are not restricted by the stability limit of explicit
methods, and are more suitable for stiff problems. The implicit
time-integration schemes employed in this work range from first to
fourth-order accurate in time, including both first and second order
accurate multistep backward difference formulas(BDF1, BDF2) and a
fourth-order accurate implicit multistage Runge-Kutta scheme. The use
of implicit Runge-Kutta schemes represents an attempt to balance the
spatial and temporal orders of accuracy. Moreover, even for
second-order accurate finite-volume schemes, fourth-order implicit
Runge-Kutta schemes have been found to outperform BDF2 schemes for
engineering accuracy levels [3].

At each time-step, implicit time-integration methods require the
solution of one or more non-linear problems. Efficient non-linear
solvers are required for this task in order to result in an overall
competitive approach. Our approach consists of using a spectral
multigrid strategy [4,5] for solving the implicit system at each time
step. The spectral or $p$-multigrid approach consists of a multigrid
method where the coarser levels are constructed by reducing the order
of accuracy ($p$-coarsening) while keeping the spatial grid resolution
fixed (as opposed to $h$-coarsening). Thus, for a
$p=3$ (fourth-order
accurate) spatial discretization, three coarser multigrid levels are
employed, consisting of $p=2$, $p=1$, and $p=0$ at the coarsest level. At
each $p$-multigrid level, an element-Jacobi scheme is used as a smoother.
The element Jacobi smoother can be viewed as an approximate Newton
method, where only the Jacobian entries corresponding to the modal
coupling within an element are retained, and all other entries are
discarded, resulting in a block diagonal matrix, which is easily
inverted using Gaussian elimination at the block level. When used as a
single grid solver, this smoother is shown to produce $p$-independent
convergence rates, with strong $h$-dependence. When used as a smoother
within the $p$-multigrid scheme, both $h$ and $p$ independent convergence
rates are obtained.

In the paper, we show both $p$ and $h$ independent convergence rates for
the implicit systems arising from the various time discretizations
using the element Jacobi driven spectral multigrid solver. The overall
efficiency and accuracy of the various time-integration schemes are
compared by examining the error as a function of time for various time
step sizes, where the design accuracy of the respective
time-integration schemes is demonstrated. Further work will focus on
extending this approach to the unsteady Euler and Navier-Stokes
equations.

References

[1] T.~C.~Warburton, I.~Lomtev, Y.~Du, S.J.~Sherwin and
G.E.~Karniadakis,
{\em Galerkin and Discontinuous Galerkin Spectral/$hp$ Methods},
Comput.~Methods Appl.~Mech.~Engrg. 175(1999), 343--359.

[2] F.~Graham, J.~Carey and T.~Oden,
{\em Finite Elements A Second Course},
Vol.2, 1983, 89~95

[3] G.~Jothiprasad , D.~J.~Mavriplis and D.~Caughey,
{\em Higher-Order
Time-Integration Schemes for the Unsteady Navier-Stokes Equations on
Unstructured Meshes}, Journal of Computational Physics, Vol 191, Issue
2, pp.542--566, November 2003

[4] B.~Helenbrook, D.~J.~Mavriplis and H.~Atkins,
{\em Analysis of
``$p$''-Multigrid for Continuous and Discontinuous Finite Element
Discretizations}, Proceedings of the 16th AIAA Computational Fluid
Dynamics Conference, 2003, AIAA Paper 2003-3989,

[5] C.~R.~Nastase and D.~J.~Mavriplis,
{\em High-Order Discontinuous
Galerkin Methods using a Spectral Multigrid Approach}, AIAA Paper
2005-1268, January 2005.

\end{document}
