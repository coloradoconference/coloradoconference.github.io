\documentclass{report}
\usepackage{amsmath,amssymb}
\setlength{\parindent}{0mm}
\setlength{\parskip}{1em}
\begin{document}
\begin{center}

\rule{6in}{1pt} \
{\large
Tzanio V. Kolev
\\ {\bf
Experiments with Adaptive Element Agglomeration Algebraic Multigrid for
H(div) and H(curl).
}}

Center for Applied Scientific Computing \\
Lawrence Livermore National Laboratory \\
Box 808, L-560 \\ Livermore CA 94551
\\ {\tt
tzanio@llnl.gov
}
\\
Panayot S. Vassilevski
\end{center}

In this talk we combine the element agglomeration AMGe (from [2]) and
the adaptive AMG (from [1]). The former method is used to generate an
initial V-cycle that coarsens only the nullspace of the respective
H(div) or H(curl) form, whereas the second method is used to gradually
augment the current coarse grids and interpolation matrices. The
numerical tests indicate that 3 to 5 adaptive cycles are sufficient in
order to achieve an efficient AMG solver. A main tool in the adaptation
process is the hierarchical construction of the modified interpolation
matrices, see [3], which is based on solving local constrained
minimization problems, fitting one ``algebraically smooth'' vector at a
time.

[1] M.~Brezina, R.~Falgout, S.~MacLachlan, T.~Manteuffel, S.~McCormick,
and J.~Ruge, {\em Adaptive Algebraic Multigrid Methods},
SIAM J.~Sci.~Comp., 2004, submitted.

[2] J.~Jones and P.~Vassilevski, {\em AMGe Based on Element
Agglomerations}, SIAM J.~Sci.~Comp. {\bf 23} (2001), pp.109--133.

[3] P.~Vassilevski and L.~Zikatanov,
{\em Multiple Vector Preserving
Interpolation Mappings in Algebraic Multigrid},
Lawrence Livermore
National Laboratory Technical Report UCRL-JRNL-208036, November 2004.


\end{document}
