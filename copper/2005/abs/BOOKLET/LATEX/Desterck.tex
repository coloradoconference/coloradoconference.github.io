\documentclass{report}
\usepackage{amsmath,amssymb}
\setlength{\parindent}{0mm}
\setlength{\parskip}{1em}
\begin{document}
\begin{center}

\rule{6in}{1pt} \
{\large
Hans De Sterck
\\ {\bf
Study of Aggressive Coarsening and Multipass Interpolation in Algebraic
Multigrid
}}

Department of Applied Mathematics
\\ University of Waterloo \\
Ontario, Canada
\\ {\tt
hdesterck@uwaterloo.ca
}
\\
Ulrike Meier Yang
\end{center}

Algebraic multigrid is a very efficent algorithm for solving large
linear systems on unstructured grids. Use of coarsening schemes such as
parallel variants of the standard coarsening algorithm by Ruge and
Stueben [1] or CLJP coarsening [2], a method based on parallel maximal
independent set algorithms, can lead to high complexities with regard
to memory usage as well as computation time, which adversely affect
scalability.
In recent work [3], we have proposed two new parallel AMG coarsening
schemes, that are based on solely enforcing a maximum independent set
property, resulting in sparser coarse grids. The new coarsening
techniques remedy memory and execution time complexity growth for
various large three-dimensional (3D) problems. If used within AMG as a
preconditioner for Krylov subspace methods, the resulting iterative
methods tend to converge fast. For some difficult problems, however,
these methods still produce complexities that are too high, or don't
converge well enough, and further remedies in terms of coarsening and
interpolation need to be found in order to obtain scalable methods.
In this paper we describe the combination of these various coarsening
methods with ``aggressive coarsening'' techniques, which require long
range ``multipass'' interpolation [4], and empirically study complexity
and convergence properties of the resulting iterative methods. The
resulting AMG methods, implemented in the hypre solver library [5], are
applied to first-order system least-squares (FOSLS) discretizations of
elliptic PDE systems. Parallel scalability of the combined FOSLS-AMG
method is investigated for large-scale three-dimensional applications.

*This work was performed under the auspices of the U.S. Department of
Energy by University of California Lawrence Livermore National
Laboratory under contract number W-7405-Eng-48 and subcontract number
B545391.

[1] J. Ruge, K. Stueben, Algebraic multigrid (AMG), in: S. McCormick,
ed., Multigrid Methods, vol. 3 of Frontires in Applied Mathematics
(SIAM, 1987) 73--130.

[2] V. E. Henson, U. M. Yang,
{\em BoomerAMG: a Parallel Algebraic Multigrid
Solver and Preconditioner}, Applied Numerical Mathematics,
{\bf 41} (2002) 155--177.

[3] H. De Sterck, U. M. Yang, and J. J. Heys,
{\em Reducing Complexity in
Parallel Algebraic Multigrid Preconditioners}, submitted to SIAM Journal
on Matrix Analysis and Applications, 2004.

[4] K. Stueben, {\em Algebraic multigrid (AMG): an introduction with
applications}, in: U. Trottenberg, C. Osterlee and A. Schueller, eds.,
Multigrid (Academic Press, 2000) 413--532.

[5] R. Falgout, U. M. Yang, {\em hypre: a Library of High Performance
Preconditioners}, in Computational Science - ICCS 2002 Part III, P.
Sloot, C. Tan, J. Dongarra and A. Hoekstra, eds., Lecture Notes in
Computer Science, {\bf 2331} (Springer, 2002) 632--641.

\end{document}
