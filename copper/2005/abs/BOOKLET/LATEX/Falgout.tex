\documentclass{report}
\usepackage{amsmath,amssymb}
\setlength{\parindent}{0mm}
\setlength{\parskip}{1em}
\begin{document}
\begin{center}

\rule{6in}{1pt} \
{\large
Robert D. Falgout
\\ {\bf
Sharpening the Predictive Properties of Compatible Relaxation
}}

Lawrence Livermore National Laboratory
\\
P.O. Box 808, L-561
\\
Livermore CA 94551 
\\ {\tt
rfalgout@llnl.gov
}
\end{center}

The notion of compatible relaxation (CR) was introduced by Brandt in
[1] as a modified relaxation scheme that keeps the coarse-level
variables invariant. Brandt stated that the convergence rate of CR is a
general measure for the quality of the set of coarse variables. A
supporting theory for these ideas was presented in [2], from which we
developed a CR-based algebraic coarsening algorithm for use in
algebraic multigrid (AMG) methods. In [3], a new sharp convergence
theory was developed for AMG. The form of this new theory bears a
striking resemblance to its predecessor and suggests the possibility of
improving the CR measure.

In this talk, we will use the relationship between these two theories
to motivate a new approach for CR, one that has the potential of being
a sharper measure of coarse grid quality and a better predictor of AMG
convergence (one specific version of this method was suggested by Livne
[4]). We will discuss the theoretical properties of the new method,
provide some numerical results, and discuss open questions and future
directions.

This work was performed under the auspices of the U.S. Department of
Energy by University of California Lawrence Livermore National
Laboratory under contract No. W-7405-Eng-48.

[1] A. Brandt,
{\em General highly accurate algebraic coarsening}, Electronic
Transactions on Numerical Analysis, {\bf 10} (2000), pp.1--20.

[2] R.~D.~Falgout and P.~S.~Vassilevski, {\em On Generalizing the AMG
Framework}, SIAM J.~Numer.~Anal., {\bf 42} (2004), pp.1669--1693.

[3] R.~D.~Falgout, P. S. Vassilevski, L. T. Zikatanov,
{\em On Two-Grid Convergence Estimates},
Numer. Linear Algebra Appl., to appear.

[4] O.~E.~Livne,
{\em Coarsening by Compatible Relaxation},
Numer.~Linear Algebra Appl., {\bf 11} (2004), pp.205--227. 

\end{document}
