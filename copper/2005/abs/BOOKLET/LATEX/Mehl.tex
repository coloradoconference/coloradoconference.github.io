\documentclass{report}
\usepackage{amsmath,amssymb}
\setlength{\parindent}{0mm}
\setlength{\parskip}{1em}
\begin{document}
\begin{center}

\rule{6in}{1pt} \
{\large
Miriam Mehl
\\ {\bf
A cache-oblivious self-adaptive full multigrid method
}}

Institut f\"{u}r Informatik, TU M\"{u}nchen \\
Boltzmannstr. 3 \\
85748 Garching, Germany
\\ {\tt
mehl@in.tum.de
}
\\
Nadine Dieminger,
Christoph Zenger
\end{center}

In many implementations of modern solvers for partial differential
equations, the use of multigrid methods in particular in combination
with dynamically adaptive grids causes a non-negligible loss of
efficiency in data access and storage usage: As multilevel data on
adaptively refined grids are typically organized in trees and stored
with the help of pointer structures, the evaluation of difference
stencils, restrictions and interpolations generate jumps within the
physical memory space and, thus, prevent an efficient usage of
cache-hierarchies and prefetching strategies. In addition, the storage
of pointers to neighbours, fathers and sons of each grid cell is
required. Our algorithm overcomes these inefficiencies by a
linearization of all data with the help of space-trees and
space-filling curves. Space-filling curves are well known as an
efficient tool for parallelization strategies on space-tree grids as
they define a linear ordering of all grid cells and, thus, allow a
simple partitioning of data. Space-tree grids are structured grids but,
at the same time, permit arbitrary local refinement (The only exception
are unisotropic refinements. The realization of such refinements is
subject of our current studies.).

In our algorithm, we associate the degrees of freedom of the unknown
function(s) to the vertices of the grid cells and - as a first step
towards cache-optimality - we replace the common pointwise operator
evaluation - that is the complete computation of the operator value at
a grid point necessitating access to data of neighbouring points and,
thus, causing a part of the jumps within the physical memory space. We
use an elementwise operator evaluation instead: We decompose the
pointwise difference stencils in parts only incorporating vertex data
of the actual cell and get the overall operator value by accumulation
over all cells involved. From the algorithmic point of view, the
consequence is a cellwise instead of pointwise processing of the
space-tree grid during the iterations of our solver.

The second and crucial step on our way to cache-optimality is to
construct suitable data structures with minimal storage overhead and
strictly local access properties. If we use the discrete iterates of
the Peano-curve, a self-similar recursively defined space-filling
curve, to define a top-down-depth-first processing order of the grid
cells (on all levels), we can derive a storage concept with a fixed,
small and constant number of stacks as the only kind of data structures
[1,2,3,4]: Due to the dimension-recursivity of the Peano-curve, this
traversion order leads to a to-and-fro-processing of vertex data on
certain sub-manifolds of the domain which exactly fits the properties
of stacks: Stacks are very simple data structures allowing only the two
basic operations push and pop (push puts data on top of a pile and pop
takes data from the top of a pile). Due to this, the locality of data
access is inherently very high which makes data access very fast - even
faster than the common access of non-hierarchical data stored in
matrices - and, in particular, reduces cache misses considerably. In
addition, we can define locally deterministic rules for pushing and
poping data to/from the set of stacks. Thus, also the storage overhead
for administrational information is minimal. Each cell only has to
carry one bit for geometrical and one bit for refinement information.

Dynamical adaptivity can be realized in a very natural way in this
concept by definition of data sources and sinks at interface to the
input and output stacks of a solver iteration. Based on the algorithmic
concepts described above, we implemented an adaptive Full Multigrid
method for the three-dimensional Poisson equation with several
adaptivity criteria like shape of the domain, error estimators for the
global error and dual approaches to handle local accuracy requirements.
In addition, the Multigrid method was combined with a tau-extrapolation
scheme to achieve a fourth order discretization [5,6]. Thus, we
considerably enhanced the numerical efficiency without loosing
cache-efficiency compared to a simple non-adaptive one-level solver. In
all case studies, the number of actual cache misses is only about ten
percent higher than the unavoidable minimum of cache misses caused by
the first loading of data points to the cache.

[1] F.~G\"{u}nther, M.~Mehl, M.~P\"{o}gl, Ch.~Zenger,
{\em A cache-aware
algorithm for PDEs on hierarchical data structures},
Conference Proceedings PARA '04, Kopenhagen, June 2004,
LNCS, Springer, submitted.F.

[2] F.~G\"{u}nther, M.~Mehl, M.~P\"{o}gl, Ch.~Zenger,
{\em A cache-aware
algorithm for PDEs on hierarchical data structures based on
space-filling curves}, SIAM Journal on Scientific Computing,
in review.

[3] F.~G\"{u}nther,
{\em A Cache-Optimal Implementation of the
Finite-Element-Method} (German: Eine cache-optimale
Implementierung der Finite-Elemente-Methode), Doctoral
Thesis, Institut f\"{u}r Informatik, TU M\"{u}nchen, 2004.

[4] M.~P\"{o}gl,
{\em Development of a Cache-Optimal 3D
Finite-Element-Method for Big Problems} (German:
Entwicklung eines cache-optimalen 3D
Finite-Element-Verfahrens f\"{u}r groe Probleme),
Doctoral Thesis, Institut f\"{u}r Informatik, TU
M\"{u}nchen, 2004.

[5] A.~Krahnke,
{\em Adaptive Higher Order Methods on Cache-Optimal
Datastructures for Three-Dimensional Problems} (German:
Adaptive Verfahren hherer Ordnung auf cache-optimalen
Datenstrukturen f\"{u}r dreidimensionale Probleme),
Doctoral Thesis, Institut f\"{u}r Informatik, TU
M\"{u}nchen, 2005.

[6] N.~Dieminger,
{\em Criteria for the Selfadaption of
Cache-Efficient Multigrid Algorithms} (German: Kriterien
f\"{u}r die Selbstadaption cache-effizienter
Mehrgitteralgorithmen), Diploma Thesis, Institut f\"{u}r
Informatik, TU M\"{u}nchen, 2005.


\end{document}
