\documentclass{report}
\usepackage{amsmath,amssymb}
\setlength{\parindent}{0mm}
\setlength{\parskip}{1em}
\begin{document}
\begin{center}

\rule{6in}{1pt} \
{\large
Scott MacLachlan
\\ {\bf
Fully Adaptive AMG
}}

Department of Applied Mathematics \\
University of Colorado \\
Boulder CO 80309-0526
\\ {\tt
maclachl@colorado.edu
}
\end{center}

Numerical simulation of physical processes is often constrained by our
ability to solve the complex linear systems at the core of the
computation. Classical geometric and algebraic multigrid methods rely
on (implicit) assumptions about the character of these matrices in
order to develop appropriately complementary coarse-grid correction
processes for a given relaxation scheme. The aim of the adaptive
multigrid framework is to reduce the restrictions imposed by such
assumptions, thus allowing for efficient black-box multigrid solution
of a wider class of problems.

There are, however, many challenges in altogether removing the reliance
on assumptions about the errors left after relaxation. In this talk, we
discuss work to date on a fully adaptive AMG algorithm that chooses all
components of the coarse-grid correction based on automated analysis of
the performance of relaxation. Fundamental measures of the need for and
quality of a coarse-grid correction will be discussed, along with
related techniques for choosing coarse grids and interpolation
operators. We will also discuss the (lack of) computability of these
ideal measures, and suggest cost-efficient alternatives.

This research has been performed in collaboration with James Brannick,
Marian Brezina, Tom Manteuffel, Steve McCormick, and John Ruge at
CU-Boulder. It has been supported by an NSF SciDAC grant (TOPS), as
well as Lawrence Livermore and Los Alamos National Laboratories.

\end{document}
