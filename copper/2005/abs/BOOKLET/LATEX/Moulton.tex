\documentclass{report}
\usepackage{amsmath,amssymb}
\setlength{\parindent}{0mm}
\setlength{\parskip}{1em}
\begin{document}
\begin{center}

\rule{6in}{1pt} \
{\large
J. David Moulton
\\ {\bf
Multilevel Upscaling: Multigrid's Lost Twin
}}

MS B284 \\
Los Alamos National Laboratory \\
Los Alamos NM 87545
\\ {\tt
moulton@lanl.gov
}
\\
Scott P. MacLachlan
\end{center}

In many applications, homogenization (or upscaling) techniques are
necessary to develop computationally feasible models on scales coarser
than the variation of the coefficients of the continuum model. The
accuracy of such techniques depends dramatically on assumptions that
underlie the particular upscaling methodology used. For example,
decoupling of fine- and coarse-scale effects in the underlying medium
may utilize artificial internal boundary conditions on sub-cell
problems. Such assumptions, however, may be at odds with the true,
fine-scale solution, yielding coarse-scale errors that may be
unbounded.

In this work, we present an efficient multilevel upscaling (MLUPS)
procedure for single-phase saturated flow through porous media.
Coarse-scale models are explicitly created from a given fine-scale
model through the application of standard operator-induced variational
coarsening techniques. Such coarsenings, which originated with robust
multigrid solvers, have been shown to accurately capture the influence
of fine-scale heterogeneity over the complete hierarchy of resulting
coarse-scale models. Moreover, implicit in this hierarchy is the
construction of interpolation operators that provide a natural and
complete multiscale basis for the fine-scale problem. Thus, this new
multilevel upscaling methodology is similar to the Multiscale Finite
Element Method (MSFEM) and, indeed, we show that it attains similar
accuracy on a variety of problems. While MSFEM is based on a two-scale
approach, MLUPS generates a complete hierarchy of coarse-scale models,
resulting in a speed-up factor of approximately 15. In addition, we
demonstrate that this new upscaling methodology can use both structured
coarsening algorithms, such as Dendy's BoxMG, and fully algebraic
algorithms, such as Ruge's AMG.

\end{document}
