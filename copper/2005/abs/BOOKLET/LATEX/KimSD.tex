\documentclass{report}
\usepackage{amsmath,amssymb}
\setlength{\parindent}{0mm}
\setlength{\parskip}{1em}
\begin{document}
\begin{center}

\rule{6in}{1pt} \
{\large
Sang Dong Kim
\\ {\bf
A Preconditioner on High-order Finite Element Methods
}}

Dept
\\
Inst
\\
Addr
\\ {\tt
email
}
\end{center}

Even if the high-order finite element method has many advantages for
solving a uniformly self adjoint elliptic operator such as Lu := -
 A u + c0u in  = [-1,1]  [-1,1] with boundary conditions ( L = D(L)
N(L)) u = 0 on D(L), n  A u = 0 on N(L), one may have a difficulty
controlling condition numbers occurred from spectral element
discretizations which makes it uneasy to use iterative methods.  In
order to alleviate such a situation, we take a lower order finite
element preconditioner operator corresponding to Bv := -  u + b0u in
with boundary conditions (B = D(B)  N(B)) v = 0 on D(B), n  v = 0 on
N(B).  Let {k}N be the standard Legendre-Gauss-Lobatto (=:LGL) points
in [-1,1].  By translations from I to a jth subinterval Ij := [xj ,xj]
we denote {k}N as the kth- LGL points in each subinterval Ij for j =
1,2, ,M.  Let PN be the subspace of C[-1,1] which consists of piecewise
polynomials with support Ij = [xj ,xj] whose

degree is less than or equal to N.  For the space PN, we choose a
piecewise Lagrange polynomial basis functions denoted as {jk(x)}
supported in Ij for j = 1, ,M.  Let VN be the space of all piecewise
Lagrange linear functions k(x).  Define an

interpolation operator IN : C[-1,1]  PN(I) such that (INv)() = v(), v
C[-1,1].  First, we set up the following relations for v  VN

where two positive constants c and C do not independent of the mesh
size hj = element stiffness matrices corresponding to L and B
respectively.  Then we will show the preconditioned system (BN)- LN

xj - xj ^h ^h -1 and the degree N of piecewise polynomial.  Let (LN)
and BN be finite

^h 1 h ^

has positive eigenvalues which are independent of the mesh size hj = xj
- xj the degree N of piecewise polynomial.

-1 and

This work was supported by KOSEF R02-2004-000-10109-0 Department of
Mathematics Education, Kyungpook National University, Taegu 702-701,
Korea

\end{document}
