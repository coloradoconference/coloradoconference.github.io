\documentclass{report}
\usepackage{amsmath,amssymb}
\setlength{\parindent}{0mm}
\setlength{\parskip}{1em}
\begin{document}
\begin{center}

\rule{6in}{1pt} \
{\large
Alfonso Limon
\\ {\bf
A Multilevel Adaptive Solver for the Density-Gradient Equation
}}

School of Mathematical Sciences
\\
710 N. College Avenue
\\
Claremont CA 91711
\\ {\tt
alfonso.limon@cgu.edu
}
\\
Hedley Morris
\end{center}

Continuing advances in the miniaturization of integrated circuits have
imposed new challenges to designers of semiconductor devices, as
traditional circuit analysis tools are no longer applicable. In this
study, we focus on the gate region of MOSFET devices with an oxide
thickness of order 4-6 nanometers. For oxide layers of this width,
quantum effects start to become noticeable and the standard equations
of semiconductor physics require quantum corrections. The
Density-Gradient equation [1], [2] is a means of calculating
approximate quantum corrections to existing formulae without solving
the full Poisson-Schodinger system.

We use a one-dimensional
approximation to reduce the D-G equations to a set of singularly
perturbed ODEs [3]. These equations have interior layer solutions,
which compromise the numerical treatment of the problem if care is not
taken to resolve the large gradients in the solution within the layer
correctly. Several approaches have been proposed to resolve the
boundary layer region including nonlinear discretization schemes [4],
[5]. However, these methods have difficulty resolving interior layers
[6]; therefore, we propose a multilevel adaptive scheme to solve the
model equations. The method is akin to multigrid, but utilizes
high-order inter-grid operators, which preserve a nested space
structure throughout the various resolution levels, similar to the
methods described by Goedecker [7] and Warming \& Beam [8]. This
multilevel method has several advantages over the previous
discretization schemes, the most important being that it adapts
dynamically to interior layers without a priori knowledge of the
location or geometry of the layer.

[1] Ancona M.G. 1990. Macroscopic description of quantum-mechanical
tunneling. Phys. Rev. B 42: 1222.

[2] Ancona M.G. 1998. Simulation of quantum confinement effects in
Ultra-Thin-oxide MOS structures. J. Tech. CAD (11).

[3] Ward, J. F., Odeh M., and Cohen D. S., Asymptotic methods for metal
oxide semiconductor field affect transistor modeling. SIAM J. Appl.
Math 50 1099-1125 (1990).

[4] Ancona M.G. and Biegel B.A. 2000. Nonlinear discretization scheme
for the density-gradient equations. In: Proc. SISPAD~R00, p. 196
(2000).

[5] Wettstein A., Schenk A., and Fichtner W. 2001. Quantum
device-simulation with the density-gradient model on unstructured
grids. IEEE Trans. Elec. Dev. 48: 279.

[6] Wang X. and Tang T.-W. 2002. Discretization scheme for
Density-Gradient Equation. Journal of Computational Electronics 1,
389-392.

[7] Goedecker S., Ivanov O., ~SSolution of Partial Differential
Equations Using Wavelets~T, Comp. In Phys, 12, 548 (1998) .

[8] Warming R., Beam R, ~SDiscrete Multiresolution Analysis Using
Hermite Interpolating: Biorthogonal Multiwavelets,~T SIAM J. Sci.
Comput. 22, 4, 1269-1377 (2000).

\end{document}
