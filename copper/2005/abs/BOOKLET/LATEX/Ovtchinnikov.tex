\documentclass{report}
\usepackage{amsmath,amssymb}
\setlength{\parindent}{0mm}
\setlength{\parskip}{1em}
\begin{document}
\begin{center}

\rule{6in}{1pt} \
{\large
Serguei Ovtchinnikov
\\ {\bf
A Fully Coupled Implicit Method for A Magnetohydrodynamics Problem
}}

Department of Computer Science \\
University of Colorado \\
Boulder CO 80309-0430
\\ {\tt
serguei.ovtchinnikov@colorado.edu
}
\\
Xiao-Chuan Cai,
Florin Dobrian,
David Keyes
\end{center}

In this talk we discuss a parallel fully implicit Newton-Krylov-Schwarz
algorithm for the numerical solution of the unsteady magnetic
reconnection problem described by the system of the reduced
magnetohydrodynamics (MHD) equations in two-dimensional space. In the
MHD formalism plasma is treated as conducting fluid and behaves
according to the fluid dynamics equations, coupled with the system of
Maxwell’s equations. One of the intrinsic features of MHD is
the formation of singular current density sheets, which is believed to
be linked to the reconnecting of magnetic fields. A robust solver is
required for handling the high nonlinearity associated with the
simulation of magnetic reconnection phenomena. The near singular
behavior of the solution of the system often limits the usable time
step size required by explicit schemes thus making implicit methods
potentially more attractive. We employ a stream function approach to
enforce the divergence-free conditions on the magnetic and velocity
fields, and solve the resulting fully coupled current-vorticity system
of equations with a fully implicit time integration using Newton-Krylov
techniques with an one-level additive Schwarz preconditioning. In this
work we study the parallel convergence of the implicit algorithm and
compare our results with those obtained by an explicit method.

\end{document}
