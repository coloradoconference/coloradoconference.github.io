\documentclass{report}
\usepackage{amsmath,amssymb}
\setlength{\parindent}{0mm}
\setlength{\parskip}{1em}
\begin{document}
\begin{center}

\rule{6in}{1pt} \
{\large
Mohit Tandon
\\ {\bf
AMR for Turbulent Buoyant Plumes
}}

380 INSCC \\
University of Utah \\
155 S 1452 E \\
Salt Lake City UT 84112
\\ {\tt
mohit@crsim.utah.edu
}
\\
Rajesh Rawat,
Philip J Smith,
Andrew M Wissink,
Brian Gunney
\end{center}

Numerical simulation of turbulent buoyancy driven plumes is
characterized by two distinct length and time scales. Buoyancy induces
large scale mixing of air and other species, forming a plume which can
persist as an organized structure over large length scales. These large
scale vortical structures have a puffing frequency that is inversely
proportional to the square root of the length scales of these
structures. The small scales tend to reduce the buoyant force by local
mixing phenomena. Global refinement to capture the details at smaller
length scales is not practical in many applications. Local mesh
refinement provides an efficient alternative to global refinement.
Motivation to simulate buoyant plumes and capture the salient flow
features in the regions where small scale mixing occurs led us to
develop a code with the capability of Adaptive Mesh Refinement (AMR).
The coupled AMR, Large Eddy Simulation (LES) procedure allows us to
capture the information at the scales of the large structures with a
higher fidelity and also the details which require a sub grid scale LES
model.

Our approach uses a filtered form for the variable density
incompressible flow equations that conserves both mass and momentum. We
use a collocated grid for our computations. The projection formulation
used avoids any velocity – pressure decoupling. The method
is based on a projection formulation for momentum equations in which we
first solve the advection – diffusion equations to predict
intermediate velocities. We then project the velocities after
interpolating them on to the faces to enforce the continuity
constraint. This projection method successfully handles
``large density'' variations of ten to one observed
in buoyant plume applications. This approach is implemented in SAMRAI
(Structured Adaptive Mesh Refinement Application Infrastructure), an
AMR framework from Lawrence Livermore National Laboratory (LLNL)
developed for structured hierarchical grids. For solving the pressure
poisson equation on a composite mesh, we use Fast Adaptive Composite
(FAC) solver in SAMRAI, which is a multilevel solver. The FAC solver
interfaces with HYPRE [developed at LLNL] linear solvers on the
coarsest level. The time integration algorithm is a recursive procedure
for each level of refinement. For this study, the criteria used for
grid refinement is the gradient of the mass fraction of a specie.

Verification and Validation of our method is done to assess its
accuracy and reliability. Verification tests are carried out by using
the analytical solutions for Eulers equation and Navier Stokes
equations on a periodic box. These test examples demonstrate the
accuracy and convergence properties of the algorithm. Validation of the
code is carried out by demonstrating the performance of the method on a
more relevant problem, in which a jet of light density fluid like
Helium mixes in ambient air in the computational domain.

\end{document}
