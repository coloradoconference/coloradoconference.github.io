\documentclass{report}
\usepackage{amsmath,amssymb}
\setlength{\parindent}{0mm}
\setlength{\parskip}{1em}
\begin{document}
\begin{center}

\rule{6in}{1pt} \
{\large
Michael W. Gee
\\ {\bf
Nonlinear Nearly Matrix-Free Algebraic Multigrid for Solid Mechanics
}}

Sandia National Laboratories
\\
PO Box 5800, MS 1110
\\
Albuquerque  NM  87185-1110
\\ {\tt
mwgee@sandia.gov
}
\\
Ray S. Tuminaro
\end{center}

The increasing demands of large-scale complex solid mechanics
simulations are placing greater emphasis on the challenges associated
with the efficient solution of the set of nonlinear equations. Here,
the solution of large systems of equations with material and geometric
nonlinearities in a parallel framework is addressed.


Instead of applying widely used Newton- or Newton-Krylov type methods
that involve the derivation of a stiffness matrix and a sequence of
linear solves, the presented work details the implementation of a
(nearly) matrix-free nonlinear algebraic multigrid algorithm applied to
solve this set of nonlinear equations.


As a basic iterative method, a nonlinear conjugate gradient algorithm
(nlnCG) using the Polak-Ribiere formula and a secant method for the
stepsize is applied. It has the advantage of being a completely
matrix-free method eliminating the need to form a stiffness matrix.
Preconditioned nonlinear CG is then applied as a smoother/coarse solver
in a classical full approximation scheme (FAS) nonlinear multigrid
cycle. Transfer operators are constructed by an aggregation approach
operating on the graph of the fine level problem. The preconditioners
to the nlnCG on all levels are obtained by multicolor finite
differencing and are chosen to be either simple Jacobi or a direct
solve on the coarsest level.
For Jacobi-preconditioned nonlinear CG, only the main diagonal of a
Jacobian matrix needs to be formed involving a distance-1 graph
coloring algorithm and an inexpensive modified colored finite
difference scheme.


The algorithm is implemented within Sandia National Laboratories'
freely available parallel `Trilinos' linear algebra framework and makes
use of its smoothed aggregation multigrid library `ML' and its
nonlinear solver library `NOX'. The outline of the algorithm and
implementation are given together with examples demonstrating the
advantages of this new approach. Several variants of the algorithm will
be discussed and compared. 

\end{document}
