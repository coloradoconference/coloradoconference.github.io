\documentclass{report}
\usepackage{amsmath,amssymb}
\setlength{\parindent}{0mm}
\setlength{\parskip}{1em}
\begin{document}
\begin{center}

\rule{6in}{1pt} \
{\large
Jonas Persson
\\ {\bf
Space-Time Adaptive Finite Difference Method for European Multi-Asset
Options
}}

Box 337 \\
SE-75105 Uppsala \\
Sweden
\\ {\tt
jonas.persson@it.uu.se
}
\\
Per L\"{o}tstedt,
Lina von Sydow,
Johan Tysk


\end{center}

We are interested in the numerical solution of the multi-dimensional
Black-Scholes equation to determine the arbitrage free price
$F(t,x)$ of
an option. We will consider European call basket options on several
underlying assets (e.g. stocks). This problem can e.g. be solved by
finding a numerical solution to a multi-dimensional PDE. This way to
determine the arbitrage free price was introduced independently by $F$.
Black and M. Scholes (who gave name to the PDE) and R.C. Merton in
1973.

In a previous paper an adaptive finite difference method was developed
with full control of the local discretization errors. The method was
shown to be very efficient. In this paper we develop a space-time
adaptive FD-method with control of the global error.

The Black-Scholes equation is discretized by second order accurate
finite difference stencils on a Cartesian grid. An error equation is
derived for the global error $E(t,x)$ in the solution. The driving right
hand side in the error equation is the discretization error in the
original PDE. This error is estimated a posteriori and the grid is
adapted so that the Cartesian structure of the grid is maintained and
the time step is adapted in every time step. The step sizes are chosen
so that a linear functional of the solution error at maturity of the
option satisfies an accuracy constraint. This means that the integral
over the error, $E(x,t)$ times a function $g(x)$ is smaller than some
positive constant.

The time step is adjusted to comply with the bound on the local error
and the space grid is changed at a few pre-specified time points. The
weights for the local error bounds in each time interval are solutions
of the adjoint equation of the multidimensional Black-Scholes PDE. The
growth of the error in the intervals between the grid adaptations is
estimated a priori by the maximum principle for parabolic equations. In
the same manner estimates of the solution of the adjoint equation is
bounded.

The advantage with our procedure is that we obtain estimates of the
numerical errors and we have an algorithm to choose the computational
grid so that bounds like the functional explained above on the errors
are satisfied also for multi-dimensional equations. For more dimensions
than five (or so), the solution with finite difference approximations
on a grid suffers severely from the `curse of dimensionality' with an
exponential growth in the number of grid points and other alternatives
must be considered. 

\end{document}
