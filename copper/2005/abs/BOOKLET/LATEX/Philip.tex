\documentclass{report}
\usepackage{amsmath,amssymb}
\setlength{\parindent}{0mm}
\setlength{\parskip}{1em}
\begin{document}
\begin{center}

\rule{6in}{1pt} \
{\large
Bobby Philip
\\ {\bf
Performance of FAC Preconditioners for Multi-Material Equilibrium
Radiation Diffusion on Adaptively Refined Grids
}}

MS B256, CCS-3 \\
Los Alamos National Laboratory \\
Los Alamos NM 87545
\\ {\tt
bphilip@lanl.gov
}
\\
Michael Pernice
\end{center}

Radiation transport plays an important role in numerous fields of
study, including astrophysics, laser fusion, and combustion
applications such as modeling of coal-fired power generation systems
and wildfire spread. A diffusion approximation provides a reasonably
accurate description of penetration of radiation from a hot source to
a cold medium in materials with short mean free paths. This
approximation features a nonlinear conduction coefficient that leads
to formation of a sharply defined thermal front, or Marshak wave, in
which the solution can vary several orders of magnitude over a very
short distance. Resolving these localized features with adaptive mesh
refinement (AMR) concentrates computational effort by increasing
spatial resolution only locally. Previously we have demonstrated the
effectiveness of combining AMR with implicit time integration to solve
these highly nonlinear time-dependent problems. The key to this
approach has been the use of effective multilevel preconditioners that
exploit the hierarchical structure of AMR grids.

Our previous work used the Fast Adaptive Grid (FAC) method which is
multiplicative in nature. While extremely robust FAC does impose
sequential processing of levels in an AMR hierarchy. The additive
variants of FAC, namely AFAC and AFACx, provide the opportunity to
overlap communication and computation. However, little is known about
their performance as preconditioners for difficult problems. We report
on efforts to solve multimaterial equilibrium radiation diffusion
problems using structured AMR and the Newton-Krylov method
preconditioned by FAC, AFAC,or AFACx. We describe our FAC, AFAC, and
AFACx solvers and report on their performance.

This work was performed under the auspices of the U.S. Department of
Energy by Los Alamos National Laboratory under contract W-7405-ENG-36.
Los Alamos National Laboratory does not endorse the viewpoint of a
publication or guarantee its technical correctness. LAUR 05-0750.

\end{document}
