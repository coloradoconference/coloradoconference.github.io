\documentclass{report}
\usepackage{amsmath,amssymb}
\setlength{\parindent}{0mm}
\setlength{\parskip}{1em}
\begin{document}
\begin{center}

\rule{6in}{1pt} \
{\large
Stefan Vandewalle
\\ {\bf
Analysis of a two-level time-parallel time-integration method
for ordinary and partial differential equations
}}


Katholieke Universiteit Leuven \\
Department of Computer Science \\
Celestijnenlaan 200A \\
B-3001 Leuven, Belgium
\\ {\tt
stefan.vandewalle@cs.kuleuven.ac.be
}
\\
Martin Gander
\end{center}

During the last twenty years several algorithms have been suggested
for solving time dependent problems parallel in time. In such
algorithms parts of the solution later in time are approximated
simultaneously to parts of the solution earlier in time. A very recent
method was presented in 2001 by Lions, Maday and Turinici, who called
their algorithm the parareal algorithm [1]. The name parareal was
chosen for the iterative algorithm to indicate that it is well suited
for parallel real time computations of evolution problems whose
solution can not be obtained in real time using one processor only.

The method is not meant as a method to be used on a one processor
computer. One iteration of the method costs already as much as the
sequential solution of the entire problem, when used on one processor
only. If however several processors are used, then the algorithm can
lead to an approximate solution in less time than the time needed to
compute the solution sequentially; hence parallel speedup is possible
with the parareal algorithm.

The parareal algorithm has received a lot of attention over the past
few years and extensive experiments have been done for fluid and
structure problems [2,3,4]. In this talk, we will show that the
parareal algorithm can be reformulated as a two-level space-time
multigrid method with an aggressive semi-coarsening in the
time-dimension. The method can also be seen as a multiple shooting
method with a coarse grid Jacobian approximation. These equivalences
have opened up new paths for the convergence analysis of the
algorithm, which is the topic of the second part of this talk.

First, we will show a sharp linear, and a new superlinear convergence
result for the parareal algorithm applied to ordinary differential
equations. We then use Fourier analysis to derive convergence results
for the parareal algorithm applied to partial differential equations.
We show that the algorithm converges superlinearly on bounded time
intervals, both for parabolic and hyperbolic problems. On long time
intervals the algorithm converges linearly for parabolic PDEs. For
hyperbolic problems however there is no such convergence estimate on
long time intervals.

[1] Lions, Maday, and Turinici,
{\em A "parareal" in time discretization of
PDEs}, C.R. Acad.Sci. Paris, t.332, pp.661--668, 2001.

[2] Farhat, and Chandesris,
{\em Time-decomposed parallel time-integrators:
theory and feasibility studies for fluid, structure, and
fluid-structure applications}, Int. J. Numer. Meth. Eng.,
58(9):1397--1434.

[3] Fisher, Hecht, and Maday,
{\em A parareal in time semi-implicit
approximation of the Navier-Stokes equations}, in Proceedings of the
15th International Domain Decomposition Conference, Kornhuber et
al.(eds), pp.433--440, Springer LNCSE, 2004.

[4] Garrido, Espedal, and Fladmark,
{\em An algorithm for time
parallelization applied to reservoir simulation}, in Proceedings of the
15th International Domain Decomposition Conference, Kornhuber et
al.(eds), pp.469--476, Springer LNCSE, 2004.

\end{document}
