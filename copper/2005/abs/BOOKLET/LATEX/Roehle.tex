\documentclass{report}
\usepackage{amsmath,amssymb}
\setlength{\parindent}{0mm}
\setlength{\parskip}{1em}
\begin{document}
\begin{center}

\rule{6in}{1pt} \
{\large
Oliver R\"{o}hrle
\\ {\bf
Modeling Jaw and Teeth Mechanics
}}

Bioengineering Institute \\
The University of Auckland \\
Private Bag 92019 \\
Auckland, New Zealand
\\ {\tt
o.rohrle@auckland.ac.nz
}
\\

Iain Anderson,
Andrew Pullan
\end{center}


Abstract

To get a better understanding of forces acting on teeth and the
temporo-mandibular joint (TMJ) during the mastication process, we are
developing accurate finite element models for parts of the human
musculo-skeletal system. While the ultimate goal is to obtain models
guided by active electrical muscle activation, we focus in a first step
on large displacement deformations of the muscles of mastication. The
deformation of these muscles is dictated by their material properties
and the movement of the mandible. A motion tracking system is used to
track the movements of the mandible during a chewing cycle. This allows
us to prescribe the displacements at the attachment points of the
muscles to the bone.

Three dimensional high-order (cubic Hermite) elements are used to
construct accurate meshes for the anatomically based geometries. Taking
into account the complexity (and desired accuracy) of the geometry, the
high-order discretizations, and the kinematics and kinetic of the
mandible, it is clear that fast, efficient and accurate iterative
solvers, such as an (algebraic) multigrid method, are inevitable. In
this talk, we discuss the applications, introduce the model creation
for bone, teeth, and muscles, and present first numerical results.

This research is funded by the Foundation for Research in Science and
Technology (FRST) under contract number UOAX0406. 

\end{document}
