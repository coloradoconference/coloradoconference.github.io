\documentclass{report}
\usepackage{amsmath,amssymb}
\setlength{\parindent}{0mm}
\setlength{\parskip}{1em}
\begin{document}
\begin{center}

\rule{6in}{1pt} \
{\large
Tim Boonen
\\ {\bf
A new prolongator for multigrid for the curlcurl equation
}}

Computer Science Department
\\
Katholieke Universiteit Leuven
\\ {\tt
tim.boonen@cs.kuleuven.ac.be
}
\\
Stefan Vandewalle
\end{center}

We consider algebraic multigrid methods for the numerical solution of
curlcurl systems in computational electromagnetics. Existing
prolongation schemes for the curlcurl equation are often based on the
Reitzinger-Sch\"{o}berl prolongator presented in [1]. There, a
piecewise constant edge prolongator $P_e$ is derived from a piecewise
constant nodal prolongator $P_n$, such that the commutation property is
satisfied. This prolongation scheme can be improved by applying ideas
of smoothed aggregation multigrid to it ([2], [3]).

We will present an alternative prolongation scheme that takes as a
starting point an arbitrary partition of unity nodal prolongator $P_n$.
We will show that it is possible to associate with the set of coarse nodal
elements (the columns of $P_n$) a set of coarse edge elements. The link
between both sets is the analytical formula for edge elements on
triangular/tetrahedral meshes

$$ E_{ij} = N_i \nabla(N_j) - N_j \nabla(N_i)  $$

We will show that in the coarse setting, this formula has an exact
discrete counterpart, which satisfies the commutation property. This
prolongation schema contains the Reitzinger-Sch\"{o}berl edge
prolongator as the special case for a piecewise constant nodal
prolongator $P_n$.


[1] S. Reitzinger, J. Sch\"{o}berl, \emph{An algebraic multigrid method
for finite element discretizations with edge elements}, Numer. Linear
Algebra Appl. 2002, {\bf 9}, pp.223--238.

[2] P.B.Bochev, C.J.Garasi, J.J.Hu, A.C.Robinson, R.S.Tuminaro,
\emph{An
improved algebraic multigrid method for solving Maxwell's equations},
Siam Journal on Scientific Computing, {\bf 25}, pp.623--642, 2003.

[3] J.Hu, R.Tuminaro, P.Bochev, C.Garasi, A.Robinson, \emph{Toward an
$h$-independent Algebraic Multigrid Method for Maxwell's Equations},
to appear in SIAM Journal on Scientific Computing, 2005. 

\end{document}
