\documentclass{report}
\usepackage{amsmath,amssymb}
\setlength{\parindent}{0mm}
\setlength{\parskip}{1em}
\begin{document}
\begin{center}

\rule{6in}{1pt} \
{\large
James Brannick
\\ {\bf
Adaptive Algebraic Multigrid Preconditioners in Quantum Chromodynamics
}}

Dept. of Applied Mathematics
\\
University of Colorado
\\
Boulder CO 80309-0526
\\ {\tt
james.brannick@colorado.edu
}
\\
Marian Brezina, David Keyes, Oren Livne,
Irene Livshits, Scott MacLachlan, Tom Manteuffel,
Steve McCormick, John Ruge, Ludmil Zikatanov
\end{center}


Standard algebraic multigrid methods assume explicit knowledge of
so-called algebraically-smooth or near-kernel components, which loosely
speaking are errors that give relatively small residuals.  Tyically,
these methods automatically generate a sequence of coarse problems
under the assumption that the near-kernel is locally constant.  The
difficulty in applying algebraic multigrid to lattice QCD is that the
near-kernel components can be far from constant, often exhibiting
little or no apparent smoothness. In fact, the local character of these
components appears to be random, depending on the randomness of the
so-called "gauge" group. Hence, no apriori knowledge of the local
character of the near-kernel is readily available.

This talk proposes adaptive algebraic multigrid (AMG) preconditioners
suitable for the linear systems arising in lattice QCD.  These methods
recover good convergence properties in situations where explicit
knowledge of the near-kernel components may not be available.  This is
accomplished using the method itself to determine near-kernel
components automatically, by applying it carefully to the homogeneous
matrix equation, $Ax=0$.  The coarsening process is modified to use and
improve the computed components. Preliminary results with model 2D QCD
problems suggest that this approach yields optimal multigrid-like
performance that is uniform in matrix dimension and gauge-group
randomness.


\end{document}
