\documentclass{report}
\usepackage{amsmath,amssymb}
\setlength{\parindent}{0mm}
\setlength{\parskip}{1em}
\begin{document}
\begin{center}

\rule{6in}{1pt} \
{\large
Craig C. Douglas
\\ {\bf
Dynamic Data-Driven Application Simulations (DDDAS)
}}

Departments of Computer Science and Mechanical Engineering
\\
University of Kentucky, Lexington KY
\\
and
\\
Department of Computer Science \\ Yale University, New Haven CT
\\ {\tt
douglas-craig@cs.yale.edu
}
\\
Yalchin Effendiev
\end{center}

DDDAS is a new paradigm in which data dynamically controls almost all
aspects of long term simulations. Rather than run many simulations
using static data as initial conditions, a very small number of
simulations are run with additional data injected as it becomes
available. Most candidate problems for the DDDAS paradigm involve
solving a nonlinear time dependent partial differential equation of the
form
$F(x+Dx(t))=0$ by iteratively choosing a new approximate
solution $x$ based on the time dependent perturbation
$Dx(t)$.

In practice, the data streaming in may have errors and therefore may
not be completely accurate or reliable (for example, in reservoir data
sets, a 15\% error in the data is common). As a result, perhaps one does
not need to solve the nonlinear equation precisely at each step. This
can expedite the execution.

At each iterative step, the following three issues may need to be
addressed:
\begin{enumerate}
\item Incomplete solves of a sequence of related models must be
understood.
\item The effects of perturbations, either in the data and/or the
model, need to be resolved and kept within acceptable limits.
\item Nontraditional convergence issues have to be understood and
resolved.
\end{enumerate}

Consequently, there will be a high premium on developing quick
approximate direction choices, such as, lower rank updates and
continuation methods, and understanding their behavior are
important issues. Fault tolerant algorithms have a premium.

The dynamic data is used to determine
\begin{enumerate}
\item Whether or not a warm restart is necessary due to
unacceptable errors building up in parts of the domain.
\item If a rollback in time is required.
\item If the simulation is running with acceptable errors.
\end{enumerate}

Ideally, there does not have to be a human in the
control loop throughout a simulation.

Using the data appropriately lets the physical and
mathematical models, the discretization, and the
scales of interesting parts of the computations
become parameters that can be changed during the
course of the simulation. In addition, error
propagation is of particular interest in nonlinear
time dependent simulations.

DDDAS offers interesting computational and
mathematically unsolved problems, such as, how do you
analyze the properties of a generalized PDE when you
do not know either where or what the local boundary
conditions are at any given moment in the simulation
in advance?

\end{document}
