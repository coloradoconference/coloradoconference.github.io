\documentclass{report}

\setlength{\parskip}{2mm}
\setlength{\parindent}{0mm}
\setlength{\textwidth}{180mm}
\setlength{\oddsidemargin}{-10mm}
\setlength{\evensidemargin}{-10mm}
\setlength{\textheight}{242mm}
\setlength{\topmargin}{-18mm}

\newcommand{\boldx}{\mbox{\bf x}}

\begin{document}
\begin{center}
{\Large \bf Copper Mountain Conference on Multigrid Methods} \\
{\large \bf March 30 -- April 4, 2003} \\
{\huge \bf Abstracts}
\end{center}


\begin{center}
\rule{6in}{1pt}\\
{\large 1. \rule{0mm}{1.5em}Mark Adams \\
{\bf Algebraic Multigrid Methods for Constrained Systems \\
of Linear Equations }}
\end{center}


\begin{center}
Mark Adams \\
Sandia National Laboratories \\
PO Box 969. \\
Livermore CA 94551-9217
\end{center}


Constrained systems of linear equations (KKT systems) arise in many applications from contact in solid mechanics and incompressible flow, to problems in mathematical optimization for PDE systems.
Several approaches have been developed for solving these systems, given a solver for the primal part of the system: 1) constraint reduction, 2) projection methods, and 3) Uzawa/augmented Lagrange.
All of these methods have advantages and disadvantages.
Uzawa methods are robust and scalable though they require several approximate solves, as they iterate on the Shur complement.
Constraint reduction is effective for certain types of constraints, but is limited in its applicability.
Projection methods are attractive as they handle general constraints and are simple to use with an existing solver or preconditioner, but they are generally not scalable nor are they robust when used with powerful preconditioners, such as multigrid.


Algebraic multigrid is a popular method for solving the equations that arise from discretized PDEs, the primal part of KKT systems, but the application of multigrid to the entire KKT system has not been investigated to a great degree.
V.H.
Schulz has recently applied structured geometric multigrid to optimization problems, S.P.
Vanka has developed structured geometric multigrid techniques for incompressible flow problems, and a commercial product (TaskFlow) has developed an algebraic (aggregation) multigrid method for unstructured incompressible flow problems.
Multigrid methods are attractive for constrained systems as they have the potential to be both highly optimal like constraint reduction and general like projection methods by maintaining tight coupling of the constraints in the preconditioning process while not requiring that the primal part of the system be augmented by regularization as in Uzawa methods or a Shur complement as in constraint reduction.


This talk discusses the application of algebraic multigrid techniques to discretized PDEs with constraints.
We describe a framework for developing algebraic multigrid methods for KKT systems and discuss coarsening techniques for the constraint equations.
We cover several approaches to constructing multigrid smoothers and present examples of application of some of these ideas to contact problems in solid mechanics.




\begin{center}
\rule{6in}{1pt}\\
{\large 2. \rule{0mm}{1.5em}David Alber \\
{\bf Local Fourier Analysis and Code Performance of \\
Structured Multigrid Solvers for Systems of PDEs \\
}}
\end{center}


\begin{center}
David Alber \\
3315 DCL, Department of Computer Science \\
University of Illinois at Urbana-Champaign \\
1304 W.  Springfield Ave., Urbana IL 61801 \\
Jim Jones (Center for Applied Scientific Computing, LLNL)
\end{center}


Our discussion concerns SysPFMG, a structured parallel multigrid solver for systems of PDEs.
SysPFMG uses semi-coarsening and operator-based interpolation schemes like those in Dendy's black-box multigrid.
The performance of the solver will be tested on model PDE systems problems, including an elasticity problem.
The performance results will be compared with the expected performance predicted by local Fourier analysis.


This work was performed under the auspices of the U.S.
Department of Energy by University of California Lawrence Livermore National Laboratory under Contract W-7405-Eng-48.




\begin{center}
\rule{6in}{1pt}\\
{\large 3. \rule{0mm}{1.5em}Peter Arbenz, Roman Geus \\
{\bf Multilevel preconditioners for solving eigenvalue \\
problems occuring in the design of resonant cavities \\
}}
\end{center}


\begin{center} Peter Arbenz \\
Institute of Computational Science, Swiss Federal Institute of Technology \\
CH-8092 Zurich, Switzerland \\
Roman Geus \\
Institute of Computational Science, Swiss Federal Institute of Technology \\
CH-8092 Zurich, Switzerland
\end{center}


We investigate eigensolvers for computing a few of the smallest eigenvalues of a constrained generalized eigenvalue problem
$$ A \boldx = \lambda M \boldx, ~ ~ C^T \boldx = \mbox{\bf 0} $$
resulting from the finite element discretization of the time independent Maxwell equation.
We found the Jacobi-Davidson algorithm (JD) and the locally optimal block preconditioned conjugate gradient (LOBPCG) method to be the most effective factorization-free algorithms for solving this eigenvalue problem provided that they are complemented by a powerful preconditioner.
We compare various combinations of hierarchical basis and AMG preconditioners that improve the convergence rate of the eigensolvers without consuming memory space excessively.
We present numerical results of very large eigenvalue problems originating from the design of resonant cavities of particle accelerators that are not reasonably solvable without the above techniques.




\begin{center}
\rule{6in}{1pt}\\
{\large 4. \rule{0mm}{1.5em}Merico E. Argentati, Andrew V. Knyazev \\
{\bf Implementation of a Scalable Preconditioned Eigenvalue \\
Solver Using Hypre }}
\end{center}


Merico E.  Argentati \\
Center for Computational Mathematics \\
University of Colorado at Denver \\
Campus Box 170, P.O.  Box 173364 \\
Denver CO 80217-3364 \\
Andrew V. Knyazev


The goal of this project is to develop a Scalable
``Preconditioned Eigenvalue Solver''
for the solution of partial eigenvalue problems for large sparse symmetric matrices on massively parallel computers,
taking advantage of advances in the Scalable Linear Solvers project,
in particular in multigrid technology and in incomplete factorizations (ILU) developed under the HYPRE project,
at the Lawrence Livermore National Laboratory,
Center for Applied Scientific Computing (LLNL-CASC).
The solver allows the utilization of HYPRE preconditioners for symmetric eigenvalue problems.
In this talk we discuss the implementation approach for a flexible
``matrix free'' parallel algorithm,
and the capabilities of the developed software.
We also discuss performance on a set of test problems.


The base iterative method that has been implemented is the locally
optimal block preconditioned conjugate gradient (LOBPCG) method
described in: A.~V.~Knyazev,
{\em Toward the Optimal Preconditioned
Eigensolver: Locally Optimal Block Preconditioned Conjugate Gradient
Method}, SIAM Journal on Scientific Computing {\bf 23} (2001), no.2,
pp.517--541.  The LOBPCG solver finds one or more of the smallest
eigenvalues and the corresponding eigenvectors of a symmetric
matrix.


The code is written in MPI based C-language and uses HYPRE and LAPACK libraries.
It has been tested with HYPRE version 1.6.0.
The user interface to the solver is implemented using HYPRE style object oriented function calls.
The matrix-vector multiply and the preconditioned solve are done through user supplied functions.
This approach provides significant flexibility.
The implementation illustrates that this method can successfully and efficiently use parallel libraries.


The following HYPRE preconditioners have been tested, AMG-PCG, DS-PCG, ParaSails-PCG, Schwarz-PCG and Euclid-PCG,
in the eigenvalue solver.
Partition of processors is determined by user input consisting of an initial array of parallel vectors.
The code has been mainly developed and tested on a beowulf cluster at CU Denver .
This system includes 36 nodes, 2 processors per node, PIII 933MHz processors, 2GB memory per node running Linux Redhat, and a 7.2SCI Dolpin interconnect.
Lobpcg has also been tested on several LLNL clusters using Compaq and IBM hardware, running Unix and/or Linux.


Keywords: Eigensolvers, parallel preconditioning, sparse matrices, parallel computing, conjugate gradients, additive Schwarz preconditioner.




\begin{center}
\rule{6in}{1pt}\\
{\large 5. \rule{0mm}{1.5em}Travis Austin \\
{\bf Multigrid for grad-div dominated systems discretized \\
with $H^1$-conforming FEs }}
\end{center}

\begin{center} Travis Austin \\
MS B284, T-7, LANL \\ Los Alamos NM 87545
\end{center}


The displacement formulation of linear elasticity yields a system of equations in which the grad-div operator is the dominant term in the incompressible limit.
A similar operator also arises when using least-squares techniques to pose the linear Boltzmann equation as a minimization problem.
When using a standard FE discretization (using linear FEs) and a standard multigrid method (using pointwise smoothing) for either system, non-robust convergence results will be observed in the multigrid method as the grad-div operator becomes dominant.
The prescription to overcoming this insufficient performance is to use a FE space with a sizable divergence-free subspace and the proper relaxation operator.
In this talk, we investigate using different orders of polynomials to discretize the problem and a group smoothing approach to reduce divergence-free error in the multigrid routine.



\begin{center}
\rule{6in}{1pt}\\
{\large 6. \rule{0mm}{1.5em}Constantin Bacuta \\
{\bf Partition of Unity Method for Stokes Problem \\
}}
\end{center}

\begin{center}
Constantin Bacuta \\
328 McAllister Bldg., University Park \\
State College PA 16802 \\
Jinchao Xu \end{center}


We propose a finite element method for overlapping or nonmatching grids for the Stokes Problem based on the partition of unity method.
We prove that the discrete inf-sup condition holds with a constant independent of the overlapping size of the subdomains.
The results are valid for multiple subdomains and any spatial dimension.




\begin{center}
\rule{6in}{1pt}\\
{\large 7. \rule{0mm}{1.5em}Ben Bergen \\
{\bf Hierarchical Hybrid Grids: A Framework for Efficient \\
Multigrid on High Performance Architectures }}

Ben Bergen \\
Universit\"{a}t Erlangen-N\"{u}rnberg \\
Lehrstuhl fur Informatik 10 (Systemsimulation)
\end{center}


For many scientific and engineering applications it is often desirable to use unstructured grids to represent complex geometries.
Unfortunately the data structures required to represent discretizations on such grids typically result in extremely inefficient performance on current high-performance architectures.
Here we introduce a grid framework using patch-wise, regular refinement that retains the flexibility of unstructured grids while achieving performance comparable to that seen with purely structured grids.
This approach leads to a grid hierarchy suitable for use with geometric multigrid methods thus combining asymptotically optimal algorithms with extremely efficient data structures to obtain a powerful technique for large scale simulations.




\begin{center}
\rule{6in}{1pt}\\
{\large 8. \rule{0mm}{1.5em}Markus Berndt \\
{\bf Multilevel accelerated optimization for problems \\
in grid generation }}
\end{center}

\begin{center} Markus Berndt \\
Mail Stop B284, T-7 \\
Los Alamos National Laboratory, Los Alamos NM 87545 \\
Mikhail Shashkov
\end{center}


The quality of numerical simulations of processes that are modeled by partial differential equations strongly depends on the quality of the mesh that is used for their discretization.
This quality is affected, for example, by mesh smoothness, or discretization error.
To improve the mesh, a functional that is in general nonlinear must be minimized (for example, the $L^2$ approximation error on the mesh).
This minimization is constrained by the validity of the mesh, since no mesh folding is allowed.
Classical techniques, such as nonlinear CG, perform very poorly on this class of minimization problems.
We will introduce a new minimization technique that utilizes the underlying geometry of the problem.
By coarsening the mesh successively, in a multilevel-like fashion, minimizing appropriate coarse grid quality measures, and interpolating finer meshes from coarser ones, a more rapid movement of fine mesh points results, and the overall convergence of the minimization procedure is accelerated.


\begin{center}
\rule{6in}{1pt}\\
{\large 9. \rule{0mm}{1.5em}J. Bordner \\
{\bf MGMPI: A Parallel Multigrid Software Library}}
\end{center}

\begin{center} J. Bordner \\
University of California, San Diego \\
9500 Gilman Drive Dept. 0424, La Jolla CA 92093-0424
\end{center}


MGMPI (\verb9http://cosmos.ucsd.edu/mgmpi9)
is a parallel multigrid software library under development at the Laboratory for Computational Astrophysics, currently located at the University of California, San Diego.
MGMPI is written using a combination of C++ and Fortran-77, and is designed to solve 3D linear elliptic PDEs discretized on logically Cartesian grids.
Parallelism is obtained by partitioning grids into subregions, and using MPI for communication between these subregions.


Current multigrid components include V-, W-, and F-cycle methods, operator-based prolongation and restriction, coarse-grid problem generation using harmonic averaging or Galerkin coarsening, and weighted point Jacobi and point red-black Gauss Seidel smoothers.
Coarse-grid problems can be solved using one of a variety of Krylov subspace solvers, including CG and BiCG-STAB, or by applying any of the available smoothers.


Future components we hope to implement soon include semi-coarsening, and alternating line- and plane-smoothers.
We also intend to include fast computational kernels that exploit properties of multi-level memory hierarchies, and to optimize parallel performance.




\begin{center}
\rule{6in}{1pt}\\
{\large 10. \rule{0mm}{1.5em}Achi Brandt \\
{\bf Multilevel Structure and Multigrid Calculation \\
of Eigenbases }}
\end{center}

\begin{center} Achi Brandt \\
Dept. of Computer Science and Applied Mathematics \\
The Weizmann Institute of Science, Rehovot 76100, Israel
\end{center}


The eigenfunctions of a differential operator (or a discretized
differential operator, or a matrix of a similar type) need not each
be separately represented.  A compact multilevel collective structure
can be organized, in which many eigenfunctions have the same
fine-level representation; they are distinguished only in the way
in which that representation is mollified, where the mollification
is a point-by-pont multiplication by a smooth function which is
represented on the next coarser grid.  At that coarser level, many
of the mollification functions can again use a common representation,
etc., so at increasingly coarser levels more and more eigensets
are separated out, each becoming progressively more specific.


This makes it possible to calculate $N$ eigenfunctions in just
$O(N \log N)$ computer operations, using only
$O(N \log N)$ storage.  By comparison,
a usual multigrid eigensolver would require at least
$O(N^3)$ operations and $O(N^2)$ storage.
Moreover the multiscale
eigenbasis (MEB) structure allows to expand any given function in
terms of the $N$ eigenfunctions in again just
$O(N \log N)$ operations.
This is a vast generalization of the Fast Fourier Transform (FFT),
which achieves the same complexity order (with a smaller prefactor),
but is limited to the eigenbasis of differential operators with
constant coefficients, periodic boundary conditions and uniform
discretization.


The multiplicative Galerkin mode of coarsening the eigenproblem
avoids certain flaws in classical multigrid eigensolvers: it yields
coarse equations that have themselves the form of an eigenproblem,
whose eigenvectors correspond one-to-one to fine eigenvectors;
orthogonalizations need only be performed at coarse levels; and
arbitrarily coarse grids can be employed even in calculating high
eigenfunctions.  This mode is particularly natural and useful for
algebraic multigrid (AMG), since it involves construction of the
intergrid interpolation operators.


The MEB structure also allows very efficient eigen calculations in
unbounded domains for operators that have asymptotically constant
or periodic coefficients, which is the usual situation in many
applications, including molecular and condensed-matter electronic
structure calculations.


The work is a collaboration in progress with Drs.  Oren Livne and
Ira Livshits.




\begin{center}
\rule{6in}{1pt}\\
{\large 11. \rule{0mm}{1.5em}M. Brezina \\
{\bf Adaptive Smoothed Aggregation Method and Applications \\
}}
\end{center}

\begin{center}
M.~Brezina \\
University of Colorado, 526 UCB, Boulder CO 80309-0526 \\
T.~Betlach, R.~Falgout, S.~MacLachlan,
T.~Manteuffel, S.~McCormick, J.~Ruge
\end{center}


Substantial effort has been focused over the last two decades on developing multilevel iterative methods capable of solving the large linear systems encountered in engineering practice.
These systems often arise from discretizing partial differential equations over unstructured meshes, and the particular parameters or geometry of the physical problem being discretized may be unavailable to the solver.
Algebraic multilevel methods have been of particular interest in this context because of their promises of optimal performance without the need for explicit knowledge of the problem geometry.
These methods construct a hierarchy of coarse problems based on the linear system itself and on certain assumptions about the low-energy components of the error.


For smoothed aggregation methods applied to discretizations of
elliptic problems, these assumptions typically consist of knowledge
of the near-nullspace of the weak form.  We describe a recently
developed extension of the smoothed aggregation method in which
good convergence properties are achieved in situations where explicit
knowledge of the near-nullspace components may be unavailable.
This extension is accomplished by using the method itself to
determine near-nullspace components when none are provided.  The
coarsening process is modified to use and improve the computed
components.


Numerical experiments include test problems arising in the LANL's
hydrocode SAGE, featuring cell-centered discretization, automatic
mesh refinement and coefficient discontinuities.




\begin{center}
\rule{6in}{1pt}\\
{\large 12. \rule{0mm}{1.5em}Zhiqiang Cai \\
{\bf Least-Squares Methods for Elasticity and Incompressible \\
Newtonian Fluid Flow }}
\end{center}

\begin{center} Zhiqiang Cai \\
Department of Mathematics, Purdue University \\
150 N. University Street, West Lafayette IN 47907-2067
\end{center}


The primitive physical equations for elastic problems and incompressible
Newtonian fluid flows are first-order partial differential systems:
the stress-displacement and the stress-velocity-pressure formulations,
respectively.  Traditionally, these first-order systems are converted
into second-order partial differential systems with less variables
through differentiation and elimination.  Computations are then
based on these well-known second-order systems.  Although substantial
progress in numerical methods and in computations has been achieved,
these problems may still be difficult and expensive to solve.  In
this talk, we will first derive the stress-velocity formulation
for incompressible Newtonian fluid flows since the pressure can be
represented in terms of the normal stresses.  We will then introduce
a least-squares method applied to the first-order system: the
stress-displacement (velocity) formulation.  The principal attractions
of our least-squares method include freedom in the choice of finite
element spaces, fast MULTIGRID solver for the resulting algebraic
equations, a practical and sharp a posteriori error indicator for
adaptive mesh refinements at no additional cost, and the same
approaches of spatial discretization and fast solution solver for
both solid and fluid problems.  Numerical results for a benchmark
test problem of planar elasticity will be presented.




\begin{center}
\rule{6in}{1pt}\\
{\large 13. \rule{0mm}{1.5em}Tim Chartier \\
{\bf Adaptive multigrid via subcycling on complementary \\
grids }}
\end{center}

\begin{center} Tim Chartier \\
Department of Mathematics, University of Washington \\
Box 354350, Seattle WA 98195 \\
Edmond Chow
\end{center}


Considerable efforts in recent multigrid research have concentrated
on algebraic multigrid schemes.  A vital aspect of this work is
uncovering algebraically smooth error components in order to
construct effective multigrid components.  Adapative or self-correcting
multigrid schemes expose algebraically smooth error, analyze the
effectiveness of the resulting multigrid algorithm and adjust the
cycling as needed in order to improve the rate toward convergence.
This talk will discuss an adaptive multigrid method that uses
relaxation and subcycling on complementary grids as an evaluative
tool in correcting MG cycling.  The particular implementation of
this idea manages smooth error in a manner analogous to spectral
AMGe.  Numerical results will be included.




\begin{center}
\rule{6in}{1pt}\\
{\large 14. \rule{0mm}{1.5em}Gregory Dardyk \\
{\bf A Multigrid Approach to Two-Dimensional Phase \\
Unwrapping }}
\end{center}


\begin{center}
Gregory Dardyk \\
Department of Computer Science, Technion 32000, Haifa, Israel \\
Irad Yavneh
\end{center}


The two-dimensional phase unwrapping problem is studied.
Using the minimum $L^p$-norm approach,
we apply three different nonlinear multigrid algorithms for
reconstructing surfaces from their ``wrapped''
values --- two classical approaches and a novel Multilevel
Nonlinear Method (MNM).
The methods prove to be efficient even for difficult problems
with noisy and discontinuous original images.
The new method, MNM, exhibits the fastest convergence of the three for all the problems,
given an appropriate choice of a damping parameter.
Proposed methods for choosing this parameter automatically are mentioned.


\newpage	%% HEY

\begin{center}
% \rule{6in}{1pt}\\
{\large 15. \rule{0mm}{1.5em}Boris Diskin \\
{\bf On Efficient Multigrid Methods for Flows with \\
Small Particles. }}
\end{center}

\begin{center}
Boris Diskin \\
National Institute of Aerospace, 144 Research Drive, Hampton VA 23666 \\
Vasyl M.~Harik \\
Swales Materials Group, Mail Stop 186A, NASA Langley Research Center, Hampton VA 23681
\end{center}


Multiscale modeling of materials requires simulations of multiple
levels of structural hierarchy.  The computational efficiency of
numerical methods becomes a critical factor for simulating large
physical systems with highly desperate length scales.  A flow with
small particles represents a particularly difficult problem for
efficient multigrid solution.  These particles are often so small,
occupying just a few points on the fine grid, that they cannot be
properly incorporated into a coarse-grid formulation.  Efficiency,
i.e., convergence rate, of classical multigrid solvers deteriorates
significantly because of the poor accuracy of coarse-grid
approximations.

In this talk, I am planning to overview and
evaluate several existing multigrid approaches to solving problems
with small particles.  A new multigrid method based on a modified
local Galerkin coarsening scheme will be presented.  Numerical
tests confirm recovering the optimal (Laplace-like) efficiency in
solution of practically interesting problems.




\begin{center}
\rule{6in}{1pt}\\
{\large 16. \rule{0mm}{1.5em}Craig C. Douglas \\
{\bf Acceleration Techniques for the Spectral Element \\
Ocean Model Methodology }}
\end{center}

\begin{center}
Craig C.  Douglas \\
Department of Computer Science, University of Kentucky \\
325 McVey Hall - CCS, Lexington KY 40506-0045 \\
Gundolf Haase \\
Johannes Kepler University Linz \\
Institute for Analysis and Computational Mathematics \\
Department of Computational Mathematics and Optimization \\
Altenberger Strasse 69, A-4040 Linz, Austria
\end{center}


A mathematical model and computational results are presented for
wind driven ocean modeling based on the Spectral Ocean Element
Method (SEOM).  The method features advanced algorithms, based on
h-p type finite element methods, allowing accurate representation
of complex coastline and oceanic bathymetry, variable lateral
resolution, and high order solution of the three dimensional oceanic
equations of motion.  SEOM is robust, accurate over a many year
simulation, and scales extremely well on a wide variety of parallel
computers including traditional supercomputers and clusters.
Acceleration techniques will be defined that lead to significant
speedups.




\begin{center}
\rule{6in}{1pt}\\
{\large 17. \rule{0mm}{1.5em}Miguel Dumett \\
{\bf FAS interpolation and smoother components: A \\
comparative study }}
\end{center}

\begin{center}
Miguel Dumett \\
Center for Applied Scientific Computing,
Lawrence Livermore National Laboratory \\
P.O. Box 808, L-551, Livermore CA 94551 \\
(925)422-8435 (Office) \\
Carol Woodward
\end{center}


We study the performance of some FAS components in a multigrid V-cycle for nonlinear anisotropic reaction diffusion equations.
We investigate the efficiency of using operator-dependent grid transfer operators and compare the performance of full system and point smoothers.
Since interpolation operators and coarse grids are solution dependent, we study the issue of when to refresh both.
These studies utilize non-Galerkin coarse equations.

\newpage	%% HEY

\begin{center}
% \rule{6in}{1pt}\\
{\large 18. \rule{0mm}{1.5em}Robert D. Falgout \\
{\bf On Generalizing the AMG Framework }}
\end{center}

\begin{center}
Robert D.  Falgout \\
Lawrence Livermore National Laboratory \\
P.O.  Box 808, L-561, Livermore CA 94551 \\
Panayot S.  Vassilevski
\end{center}


In recent years, much work has been done to increase the robustness
of Algebraic Multigrid (AMG) methods.  The classical AMG method of
Ruge and St\"{u}ben [5] used heuristics based on properties of
M-matrices.  Although this algorithm works remarkably well for a
wide variety of problems [3], the M-matrix assumption still limits
its applicability.  To address this, a new class of algorithms was
developed based on multigrid theory: AMGe [1], element-free AMGe
[4], and spectral AMGe [2].  All of these algorithms assume a basic
framework in their construction.  Namely, they assume that relaxation
is a simple pointwise method, then they build the coarse-grid
correction step to eliminate the so-called {\em algebraically
smooth}  error left over by the relaxation process.  In the AMGe
methods above, this is done with the help of a {\em measure} (or
weak-approximation property) that defines the approximation property
that interpolation must satisfy in order to achieve uniform
convergence assuming a pointwise relaxation.

In this talk, we introduce a new measure for constructing algebraic
multigrid methods.  The purpose of this new measure is to generalize
the AMG framework to include problems such as Maxwell's Equations,
which has a particularly large null-space when discretized using
the common N\'{e}d\'{e}lec finite elements.  In the old framework,
it is necessary to take all $O(N)$ of these null-space components
to the coarse grid, which results in a non-optimal method.  This
problem can be overcome by using something other than pointwise
relaxation to damp a subspace of these near-null-space components
on the fine grid.  Examples include overlapping block relaxation
or the so-called Hiptmair smoother (Brandt's distributive relaxation).
The measure proposed here takes into account the smoothing process
and changes the AMGe heuristic in a subtle but important way.  The
hope is that this new measure will allow us to develop AMG methods
that can handle difficult problems such as Maxwell's equations or
Helmholtz.

This work was performed under the auspices of the U.S.  Department
of Energy by University of California Lawrence Livermore National
Laboratory under contract No.  W-7405-Eng-48.  UCRL-JC-150807-ABS


[1] M.~Brezina, A.~J.~Cleary, R.~D.~Falgout,
V.~E.~Henson, J.~E.~Jones, T.~A.~Manteuffel,
S.~F.~McCormick, and J.~W.~Ruge,
{\em Algebraic multigrid based on element interpolation (AMGe)},
SIAM J.~Sci.~Comput. {\bf 22} (2000), pp.1570--1592.

[2] T.~Chartier, R.~D.~Falgout, V.~E.~Henson, J.~E.~Jones,
T.~A.~Manteuffel, S.~F.~McCormick, J.~W.~Ruge, and P.~S.~Vassilevski,
{\em Spectral AMGe}, SIAM J.~Sci.~Comput. (to appear).

[3] A.~J.~Cleary, R.~D.~Falgout, V.~E.~Henson, J.~E.~Jones,
T.~A.~Manteuffel, S.~F.~McCormick, G.~N.~Miranda, and J.~W.~Ruge,
{\em Robustness and scalability of algebraic multigrid},
SIAM J.~Sci.~Comput., {\bf 21} (2000), pp.1886--1908.

[4] V.~E.~Henson and P.~S.~Vassilevski,
{\em Element-free AMGe: general algorithms for computing
interpolation weights in AMG},
SIAM J.~Sci.~Comput. {\bf 23} (2001), pp.629--650.

[5] J.~W.~Ruge and K.~St\"{u}ben,
{\em Algebraic multigrid (AMG)}, in Multigrid Methods,
S.~F.~McCormick, ed., vol.~3 of
{\em Frontiers in Applied Mathematics},
SIAM, Philadelphia PA, 1987, pp.73--130.




\begin{center}
\rule{6in}{1pt}\\
{\large 19. \rule{0mm}{1.5em}Stefan Feigh \\
{\bf Geometric Multigrid Method for Electro- and Magnetostatic \\
Field Simulations Using the Conformal Finite \\
Integration Technique }}
\end{center}

\begin{center}
Stefan Feigh \\
Darmstadt University of Technology \\
Computational Electromagnetics Laboratory \\
Schlossgartenstrasse 8, 64289 Darmstadt, Germany \\
Markus Clemens, Thomas Weiland
\end{center}


A geometrical multigrid algorithm for vector based formulations for electromagnetic field problems discretized by the Conformal Finite Integration Technique is proposed.
The transfer operators and the coarse grid operators are constructed for a hierarchy of non-nested grids.
A validation of the presented approach is achieved for electro-static and magneto-static test problems for which a discrete Poisson and a discrete Curl-Curl equation have to be solved respectively.
Experimental results for the asymptotic complexity and for the convergence characteristics in case of discontinuous material coefficients are presented.


\begin{center}
\rule{6in}{1pt}\\
{\large 20. \rule{0mm}{1.5em}Michael Flanagan \\
{\bf Order N reconstructor algorithms for Adaptive \\
Optics. }}
\end{center}

\begin{center}
Michael Flanagan \\
43 E.  Lanikaula St, Hilo HI 96720
\end{center}


We apply multigrid methods for the fast solution of large scale problems in Adaptive Optics.
Adaptive Optics (AO) is a technique for removing the blurring effects in optical systems.
It has important uses in medical imaging as well as astronomical telescopes.
In astonomy, AO senses incoming light that has been distorted--due to atmospheric turbulence--and corrects the distortions using a deformable mirror.
Of great practical importance is the ability to evaluate the linear map from sensor measurements to mirror deformations (called the reconstructor matrix) within the 5-10 millisecond time scale of the turbulence.
For 30+ meter telescopes now being developed, the reconstructor matrix will have tens or even hundreds of thousands of unknowns.
In this talk we will describe the special structure of the reconstructor matrix, and we will present a multigrid algorithm for the fast evaluation of this matrix.
Classically, least squares reconstruction algorithms are used.
We present the model with regularization.
We will discuss issues such as the sensor/actuator geometries that affect the performance of the multigrid algorithm.
Incorporating Kolmogorov statistics into the least squares reconstruction as a regularizing term will be described.




\begin{center}
\rule{6in}{1pt}\\
{\large 21. \rule{0mm}{1.5em}Rima Gandlin, Achi Brandt \\
{\bf Two Multigrid Algorithms for Inverse Problem \\
in Electrical Impedance Tomography }}
\end{center}

\begin{center}
Rima Gandlin \\
Faculty of Mathematics and Computer Science \\
The Weizmann Institute of Science \\
POB 26, Rehovot 76100, ISRAEL \\
Achi Brandt
\end{center}


A {\em direct} partial differential problem involves an interior
differential equation and a set of initial and/or boundary conditions
which stably determines a unique solution.  An {\em  inverse}
problem is one in which the differential equation and/or initial
and/or boundary conditions are not fully given and instead the
results of a set of solution observations (measurements) are known.
They may contain errors, and even without errors the problem is
usually ill-posed: the known data may be approximated by widely
different solutions, but the ill-posedness of an inverse problem
does not necessarily imply more expensive solution process.  On
the contrary: once the nature of the ill-posedness has been generally
understood, to solve an inverse problem may even be much less
expensive than to solve a corresponding direct problem.  I am going
to demonstrate this methodological point on an example of inverse
problem for Electrical Impedance Tomography (EIT).


An EIT device for medical use consists of a set of electrodes
attached to the chest of a patient.  A small {\em known} current
is passed between two driver electrodes.  In each measurement a
current is passed through a different electrode pair, while the
voltage drops at {\em all} the electrodes are recorded.  The
collected data are used in order to compute the conductivity
distribution in a part of the patient's chest and then to display
it on a screen in order to detect anomalies, such as tumors.  The
electrical potential satisfies the equation $\cdot(s\cdot u)=0$,
where $s$ is an electrical conductivity.  The set of measurements
gives ideally (in the limit of many small electrodes and as many
measurements) the {\em Neumann to Dirichlet} mapping: the Dirichlet
$u$ boundary condition resulting from any Neumann $\cdot
u/\cdot\mbox{\boldmath{\emph{n}}}$ condition.  The inverse problem
is to evaluate s from this mapping.  The conductivity depends on
the EIT data in a very weak way.  Therefore the inverse problem of
EIT is ill-posed.


There exist some works on numerical methods for the relevant
problems, but their number is rather sparse and even those papers
do not consider the question of numerical efficiency, despite its
importance for applications.  In this specific inverse EIT problem,
employing local Fourier decompositions, we have shown that all
conductivity-function components of wavelength $l$ are ill-posed
at distances $r \gg l$ from the boundary.  Hence there is no need
to use at such distances fine solution grids, since all we can know
about the solution can be calculated with grids whose meshsizes
increase proportionality to $r$.


Two multigrid solvers for this problem are presented.  The {\bf
first multigrid solver} uses the Tikhonov regularization technique.
The inverse problem is reformulated as a variational minimization
problem.  The resulting Euler equations form a PDE system (for $u$,
$s$ and Lagrange multipliers), which makes the problem suitable in
principle for an effective numerical solution by multigrid methods.
The multigrid solver starts with large and then procceds to
progressively smaller regularization parameters.  Many features of
classical multigrid were adapted to this particular problem (such
as intergrid communications, boundary condition treatment and coarse
grid solution).


In the case of a large regularization parameter, numerical results
demonstrate a good convergence of the developed solver, but the
obtained solution is too smeared and doesn't approximate the real
conductivity too well.  At small regularization the final approximation
is much better, especially near the boundary.  However, in this
case the system is no longer elliptic and for efficiency the
multigrid solver must use more sophisticated relaxation scheme,
which effectively decomposes the system into its scalar factors.
Although the multigrid cycles assymptotically slow down, the final
approximation to the {\em conductivity} is obtained by just one
multigrid cycle per grid refinement even for the smallest regularization
for which solution still exists.


In the {\bf second multigrid solver} a regularization in the
classical sense is replaced by {\em a careful choice of grids}.
In order to avoid introducing ill-posed components into the current
approximation, changes to the solution far from the measurements
are calculated only on coarse enough grids (with meshsizes at least
comparable to the distance from the measurements), which only define
large-scale {\em averages }of the solution.  On the other hand,
near the boundary of measurements we introduce both high and low
oscillatory components to the solution.  For this we use semi-coarsening
in the direction parallel to the boundary of measurements and
full-coarsening in the perpendicular direction.


The algorithm is based on three main principles: (1) due to the
ill-posedness, the computational work on the grid of meshsize $h$
should only be done near the boundary, at distances $O(h)$ from
it; (2) similar changes introduced into $s$ at the different $O(h)$
distances from the boundary of measurements cause {\em very different}
changes in $u$ on this boundary for smooth Fourier components in
$y$ a full coarsening (both in $x$ and $y$) does not resolve these
differences, hence we must use {\em a semi coarsening} in $y$
direction (i.e., without coarsening the meshsize in the $x$
direction); (3) it is necessary to start always from the {\em
better-posed} changes; in each cycle (FMG, V-cycle, W-cycle,
semi-cycle) we introduce changes to the current approximation first
on the coarsest (or semi-coarsest) grid, where those changes are
well-posed, and only after that changes from the finer grids.


Numerical {\bf results} show that the second algorithm approximates
$s$ {\em  better} than the regularized method with its many artificial
parameters (the regularization coefficients, which should change
over the domain), for less work (no Lagrange multipliers).  The
accuracy of both algorithm, in particular near the boundary of
measurements, is nearly the same.  But the second algorithm does
not use any regularization in a classical sense, and as a result
significant space and time saving is achieved.  In this algorithm
there is no need to solve a large system of discretized differential
equations and boundary conditions; also, no Lagrange-multiplier
equations need to be treated.




\begin{center}
\rule{6in}{1pt}\\
{\large 22. \rule{0mm}{1.5em}Roland Glowinski, Jari Toivanen \\
{\bf Solution of Non-Equilibrium Radiation Diffusion \\
Problems using Multigrid }}
\end{center}

\begin{center}
Roland Glowinski, Jari Toivanen \\
Department of Mathematics, University of Houston
\end{center}


The gray approximation for thermal non-equilibrium radiation diffusion problems yields a time dependent non-linear system of equations coupling the radiation energy and the material temperature.
A flux-limiting term is added to the diffusion coefficient for the radiation energy.
Implicit time stepping schemes lead to the solution of non-linear systems of equations.

A Newton-Krylov-method is employed in the solution of arising non-linear problems.
For GMRES iterations, we study preconditioners based on multigrids methods.
The first approach is to apply a geometric multigrid method directly to the coupled linearized problems.

Another approach is to construct an operator splitting which treats the transport phenomenon and the equilibration coupling in separate substeps.
A similar approach was proposed by Mousseau, Knoll and Rider, 2000.

We demonstrate the effectiveness of the proposed approaches by solving one-dimensional and two-dimensional model problems.



\begin{center}
\rule{6in}{1pt}\\
{\large 23. \rule{0mm}{1.5em}Jayadeep Gopalakrishnan \\
{\bf Preconditioning hybridized mixed methods }}
\end{center}

\begin{center}
Jayadeep Gopalakrishnan \\
Department of Mathematics \\
University of Florida, Gainesville FL 32611-8105
\end{center}

Hybridization of mixed methods for the Dirichlet problem by
introduction of Lagrange multipliers is the preferred way of
implementing the mixed method for compelling theoretical and
practical reasons.  We introduce a new variational characterization
of the Lagrange multiplier equation arising from hybridization.
This result has several applications, but its applications in
constructing preconditioners will be our main focus.  We will begin
by examining the conditioning of this equation when no preconditioner
is used.  Then we will establish certain spectral equivalences
which allow construction of a Schwarz preconditioner for the Lagrange
multiplier equation.  Although preconditioners for the lowest order
case of the hybridized Raviart-Thomas method have been constructed
previously by exploiting its connection with a nonconforming method,
our approach is different in that we use the new variational
characterization.  This allows us to precondition even the higher
order cases of these methods.  Among other applications of the
characterization result, is a previously unsuspected relationship
between the Raviart-Thomas and the Brezzi-Douglas-Marini methods,
e.g., we show that when these methods are used to approximate
harmonic solutions, they yield identical Lagrange multipliers.


[1] B.~Cockburn and J.~Gopalakrishnan,
{\em A characterization of hybridized mixed methods for the Dirichlet problem}
(preprint).

[2] J.~Gopalakrishnan,
{\em A Schwarz preconditioner for a hybridized mixed method}
(preprint).



\begin{center}
\rule{6in}{1pt}\\
{\large 24. \rule{0mm}{1.5em}Sue Goudy \\
{\bf SMG on SMP Clusters: Performance Issues }}
\end{center}

\begin{center}
Sue Goudy \\
Department of Computer Science, New Mexico Institute of Mining and Technology, Socorro NM 87801 \\
Lorie Liebrock, Steve Schaffer
\end{center}


The symmetric multiprocessor (SMP) appears as a unit in systems from desktop computers to massively parallel systems.
The focus of this paper is the performance of a particular algorithm, two-dimensional semicoarsening multigrid, on a cluster of SMPs.
Complexity models for hybrid parallelization of a portion of this algorithm are derived.
We examine system parameters, such as the capability for simultaneous message-passing in a symmetric multiprocessor, that can affect the performance of hybrid code.
Complexity estimates are tested for a variety of decomposition strategies, line solve methods, and problem sizes.
Results from the Intel Teraflops supercomputer and from a Beowulf cluster of dual-processor Xeons are presented.
An expanded abstract in PostScript form has been sent to MGNet.


\newpage		%% HEY

\begin{center}
% \rule{6in}{1pt}\\
{\large 25. \rule{0mm}{1.5em}Gundolf Haase \\
{\bf Acceleration and Parallelization of Algebraic \\
Multigrid }}
\end{center}

\begin{center}
Gundolf Haase \\
Johannes Kepler University Linz, Institute for Computational Mathematics, Altenberger Str.69, A-4040 Linz, Austria \\
Stefan Reitzinger
\end{center}


Although algebraic multigrid methods (AMG) possess optimal properties wrt.
memory requirements and the time for solving, a commercial user could be dissatisfied with the computational performance in comparison to highly optimized standard solvers.
A lot of performance can be gained by designing data structures with respect to state-of-the-art computer architectures, parallelization and redesign of numerical algorithms.
The parallelization needs some modifications in the coarsening process such that the inter grid transfer operators fulfill a certain condition on the pattern of the interpolation/restriction.
This guarantees that the parallel AMG is only a simple modification of the sequential AMG.
The presented parallelization strategy for AMG results in very good speedups.
Discretized differential equations have to be solved several thousand times inside the solution process of an inverse problem.
We got for this special application of AMG a significant gain in CPU time (factor 4 and more) due to additional acceleration of our code PEBBLES by simultaneous handling of several data sets, cache aware programming and by merging of multigrid subroutines.
Together with a parallelization, the solution time of the original code was accelerated from 8 days to 5 hours on a 12 processor parallel computer.




\begin{center}
\rule{6in}{1pt}\\
{\large 26. \rule{0mm}{1.5em}Jeffrey J. Heys \\
{\bf  First-Order System Least Squares (FOSLS) and \\
AMG for Fluid-Elastic Problems }}
\end{center}


\begin{center}
Jeffrey J. Heys \\
Thomas A. Manteuffel, Stephen F.  McCormick
\end{center}


Numerous mathematical models for the mechanical coupling between
a moving fluid and an elastic solid have been developed.  The models
are inherently nonlinear because the shape of the Eulerian fluid
domain is not known {\em a priori} -- it is at least partially
determined by the displacement of the elastic solid.  Different
iteration techniques have been developed to solve the nonlinear
system of equations.  For example, one approach is to iteratively
solve the fluid equations on a fixed domain, apply the fluid stresses
from the fluid solution to an elastic solid, and remap the fluid
domain based on the displacement of the elastic solid.  At the
other extreme, the full system of equations (fluid, elastic, and
mapping/meshing) can be solved simultaneously using a Newton
iteration.  There are advantages and disadvantages to each approach,
and the choices effect the performance of the solver (AMG) and the
accuracy of the solution differently.

The performance of different iteration techniques will be presented
for a model of a linear elastic solid coupled to a Newtonian fluid
using a FOSLS formulation.  In this approach, the system of non-linear
partial differential equations is recast as a linearized first-order
system of equations, and the solution is found using least squares
minimization principles.  A finite element discretization of the
FOSLS formulation results in a SPD matrix problem that is solved
using AMG.  With AMG, only a single Newton iteration is required
for each refinement beyond the course grid to achieve an accurate
discrete solution on the fine grid.  However, the convergence rate
of the AMG cycles depends on the iteration technique.


\begin{center}
\rule{6in}{1pt}\\
{\large 27. \rule{0mm}{1.5em}Feng-Nan Hwang \\
{\bf  A Parallel Nonlinear Additive Schwarz Preconditioned \\
Inexact Newton Algorithm for Incompressible Navier-Stokes \\
Equations}}
\end{center}

\begin{center}
Feng-Nan Hwang \\
Department of Applied Mathematics, University of Colorado, Boulder CO 80309
\\ Xiao-Chuan Cai \\
Department of Computer Science, University of Colorado, Boulder CO 80309
\end{center}


A nonlinear additive Schwarz preconditioned inexact Newton method
(ASPIN) was introduced recently for solving large sparse nonlinear
systems of equations obtained from the discretization of nonlinear
partial differential equations, and the method has proved numerically
to be more robust than the traditional inexact Newton methods,
especially for problems with unbalanced nonlinearities.  In this talk,
we discuss some recent development of ASPIN for solving the steady
state incompressible Navier-Stokes equations in the velocity-pressure
formulation.  The sparse nonlinear system is obtained by using a
Galerkin least squares finite element discretization on two dimensional
unstructured meshes.  The key idea of ASPIN is that we find the
solution of the original system $F(u)=0$ by solving a nonlinearly
preconditioned system $G(u)=0$ that has the same solution as the original
system, but with more balanced nonlinearities.  Our numerical results
show that ASPIN is more robust than the traditional inexact Newton
method when the Reynolds number is high and when the number of
processors is large.  In this talk we present some results obtained on
parallel computers for high Reynolds number flows and compare our
approach with some inexact Newton method with different forcing terms.



\begin{center}
\rule{6in}{1pt}\\
{\large 28. \rule{0mm}{1.5em}Ana H. Iontcheva, Panayot S. Vassilevski \\
{\bf Nonlinear AMGe with Coarsening Away from the \\
Contact Boundary for the Signorini's Problem}}

Ana H.  Iontcheva \\
Institute for Scientific Computing Research \\
UC Lawrence Livermore National Laboratory \\
P.O.  Box 808, L-419, Livermore CA 94551 \\
Panayot S.  Vassilevski \\
Center for Applied Scientific Computing \\
UC Lawrence Livermore National Laboratory \\
P.  O.  Box 808, L-560, Livermore CA 94551
\end{center}


The finite element discretization of the Signorini's problem leads to inequality constrained minimization problem.
In this talk we present a nonlinear element based algebraic multigrid method with special coarsening away from the contact boundary for the solution of this problem.
As a smoothing procedure we use the Projected Gauss-Seidel algorithm and for the coarse grid solver --- a modification of the Dostal's algorithm.
The performance of the resulting method is illustrated by numerical experiments.


This work was performed under the auspices of the U.S.
Department of Energy by the University of California, Lawrence Livermore National Laboratory



\begin{center}
\rule{6in}{1pt}\\
{\large 29. \rule{0mm}{1.5em}Vladimir Ivanov, Joe Maruszewski \\
{\bf Development of Selective AMG Solver and Comparison \\
of its Performance with Aggregative AMG }}
\end{center}

\begin{center}
Vladimir Ivanov and Joe Maruszewski \\
Fluent Inc., 10 Cavendish Court, Lebanon NH 03766
\end{center}


Selective multigrid has higher order of interpolation and prolongation
operators in comparison with standard aggregative multigrid.  These
properties provide better convergence rate for selective multigrid
but come at price of a longer setup phase.  A question what solver
is faster can't be answered without direct comparison of both
solvers under a specified convergence criteria.

The selective multigrid has been implemented following J.W.~Ruge
and K.~Stueben's works.  The standard coarsening scheme was used
and the direct and standard interpolations were tried.  The standard
interpolation is more exact but more expensive in time and memory.

The convergence and running time of the selective multigrid were
compared with that of the Weiss et~al.'s aggregative multigrid.
Both solvers were applied to a test problem based on a diffusivity
equation and to a few matrices built from a pressure equation for
some CFD problems.  Their performance is analyzed based on reduction
of residuals by one and six order of magnitude.

[1] Stueben, K.,  {\em An Introduction to Algebraic Multigrid},
pp.~413--532 in Trottenberg U., Oosterlee C.W., and Schuller A.,
Multigrid, Academic Press, 2001.

[2] Weiss, J.M, Maruszewski, J.P, Smith, W.A.,
{\em Implicit Solution of the Navier-Stokes Equations on Unstructured Meshes},
13th AIAA CFD Conference, June 1997, Paper 97-2103.



\begin{center}
\rule{6in}{1pt}\\
{\large 30. \rule{0mm}{1.5em}Tzanio V. Kolev, Joseph E. Pasciak, Panayot S. Vassilevski \\
{\bf Algebraic construction of mortar finite element \\
spaces with application to parallel AMGe }}
\end{center}

\begin{center}
Tzanio V.~Kolev, Joseph E.~Pasciak, \\
Department of Mathematics \\
Texas A\&M University \\
College Station TX 77843-3368 \\
and Panayot S.~Vassilevski \\
Center for Applied Scientific Computing \\
UC Lawrence Livermore National Laboratory \\
P.O.~Box 808, L-560, Livermore CA 94551
\end{center}


Finite element problems posed on large unstructured grids arise naturally in simulations, where only a very fine discretization of the domain is available.
In order to achieve performance comparable with multilevel methods for the geometrically refined case, one can use an algebraic solver based on sequence of coarsened meshes.


In this talk we present a possible parallelization of one such algorithm - the agglomeration based algebraic multigrid for finite element problems (AMGe).
The method starts with a partitioning of the original domain into subdomains with a generally unstructured finite element mesh on each subdomain.
The agglomeration based AMGe is then applied independently in each subdomain.
It needs access to the local stiffness matrices which are (variationally) constructed after coarsening.
Note that even if one starts with a conforming fine grid, independent coarsening generally leads to non-matching grids on the coarser levels.
We use an element-based dual basis mortar finite element method in order to set up global problems on each level.
Since the considered AMGe produces abstract elements and faces defined as lists of nodes, the mortar multiplier spaces are also constructed in a purely algebraic way.
This construction requires inversion of the local mass matrices on each interface boundary shared between two subdomains, which is possible because of the way AMGe agglomerates the faces.
Finally, a multigrid-preconditioned solver is applied to the resulting sequence of (non-nested) spaces.
The talk will conclude with set of numerical results illustrating the computational behavior of this new algebraic multigrid algorithm.


This work was performed under the auspices of the U.S.
Department of Energy by the University of California, Lawrence Livermore National Laboratory under contract No.
W-7405-Eng-48.




\begin{center}
\rule{6in}{1pt}\\
{\large 31. \rule{0mm}{1.5em}Johannes K. Kraus \\
{\bf Algebraic Multilevel Preconditioning Based on \\
Element Agglomeration }}
\end{center}

\begin{center}
Johannes K.~Kraus \\
University of Leoben \\
Franz Josef Strasse 18, A-8700 Leoben, Austria
\end{center}


We consider an algebraic multilevel preconditioning method for SPD matrices resulting from finite element discretization of elliptic PDEs.
The method is based on element agglomeration, and, in particular, designed for non-M matrices.
Granted that the element matrices at the fine-grid level are given, we further assume that we have access to some algorithm that performs a reasonable agglomeration of fine-grid elements at any given level.
The coarse-grid element matrices are simply Schur complements computed from the locally assembled fine-grid element matrices, i.e., agglomerate matrices.
Hence, these matrices can be assembled to a global approximate Schur complement.
The elimination of fine-degrees of freedom in the agglomerate matrices is done without neglecting any fill-in.
This offers the opportunity to construct a new kind of incomplete LU factorization of the pivot matrix at every level, which is done by means of a slightly modified assembling process.
Based on these components an algebraic multilevel preconditioner can be defined for more general SPD matrices.
The method can also be applied to systems of PDEs.
A numerical analysis shows its efficiency and robustness.




\begin{center}
\rule{6in}{1pt}\\
{\large 32. \rule{0mm}{1.5em}Young-Ju Lee \\
{\bf A Sharp Convergence Estimate on the Method of \\
Subspace Correction for Singular Systems }}
\end{center}

\begin{center}
Young-Ju Lee \\
Department of Mathematics, Eberly College of Science \\
The Pennsylvania State University, 218 McAllister Building, University Park PA 16802-6401 \\
Jinbiao Wu, Jinchao Xu, Ludmil Zikatanov
\end{center}


We shall present a sharp result on the convergence rate of the method of successive subspace corrections for singular system of equations.
The result has been obtained for variational problems on a Hilbert space.
Various identities for the convergence rate are presented especially for multigrid method and domain decomposition method.
Some illustrations of theory are also given for problems posed on a finite dimensional Hilbert space.




\begin{center}
\rule{6in}{1pt}\\
{\large 33. \rule{0mm}{1.5em}Koung-Hee Leem \\
{\bf Algebraic Multigrid (AMG) for indefinite linear \\
systems from meshfree dscretizations }}
\end{center}

\begin{center}
Koung-Hee Leem \\
Department of Mathematics, 14 MacLean Hall \\
University of Iowa, Iowa City IA 52242-1419 \\
S.~Oliveira, D.~E.~Stewart
\end{center}


Meshfree discretizations construct approximate solutions to partial differential equation based on particles, not on meshes, so that it is well suited to solve the problems on irregular domains.
Since the nodal basis property is not satisfied in meshfree discretizations, it is difficult to handle essential boundary conditions.
There have been attempts to enforce this property by modifying basis functions.
Instead, we employ the Lagrange multiplier approach to resolve this problem, but this will result in indefinite linear systems.

As a solver for this indefinite linear system, we propose a new Algebraic Multigrid (AMG) scheme based on aggregating elements and neighborhoods.
AMG based only on aggregation of elements was tried because of its simplicity, but it often shows slow convergence.
There have been many attempts to overcome this problem.
The new interpolation approach utilizes the information in neighborhood matrices as well as aggregation of elements.

Unlike classical AMG, this new approach can be applied to any symmetric, positive definite linear systems.
Moreover, with additional information for meshfree discretizations, we successfully modified this new AMG approach to solve symmetric indefinite linear systems arising from meshfree discretizations with essential boundary conditions.



\newpage	%% HEY

\begin{center}
% \rule{6in}{1pt}\\
{\large 34. \rule{0mm}{1.5em}Konstantin Lipnikov \\
{\bf Algebraic multilevel preconditioner with projectors \\
}}
\end{center}

\begin{center}
Konstantin Lipnikov \\
T-7, MS-B284, Los Alamos NM 87545 \\
Yuri Kuznetsov \\
Department of Mathematics, University of Texas, Houston TX 77204
\end{center}


The two-level algebraic preconditioner with projectors has been
proposed in [1] for symmetric positive definite stiffness matrices.
The underlying grid is partitioned into non-overlapping substructures
of arbitrary shapes (grid subdomains, superelements or simply
original grid cells).  A projector to the kernel of a subdomain
stiffness matrix is build for each subdomain.  A coarse grid matrix
is an approximation to the original stiffness matrix in the space
of images of the projectors.  If the original stiffness matrix is
a finite element approximation of an elliptic operator, the coarse
grid matrix describes connections between integral averages a
discrete function over subdomains.


In the talk we shall prove the convergence of the two-level scheme
for the case of elliptic operators.  We shall show that the
convergence rate is independent of jumps in coefficients between
subdomains.  A multilevel preconditioner based on the two-level
scheme will be studied numerically for the case of highly distorted
meshes and strongly varying coefficients.  In comparison with a
few other multilevel preconditioners, it demostrates robust behavior
over a much bigger class of problems [2].


[1] Yu.~Kuznetsov,
{\em Two-level preconditioners with projectors for unstructured grids},
Russian J.~Numer.~Anal.~Math.~Modelling {\bf 15} (2000), pp.247--256.

[2] Yu.~Kuznetsov, K.~Lipnikov,
{\em Parallel numerical methods for the diffusion and Maxwell
equations in heterogeneous media on strongly distorted meshes},
Technical Report, University of Houston, May, 2002.




\begin{center}
\rule{6in}{1pt}\\
{\large 35. \rule{0mm}{1.5em}Oren E. Livne \\
{\bf Coarsening by Compatible Relaxation }}
\end{center}

\begin{center}
Oren E.~Livne \\
Computer Science Department, Gates 2B \\
Stanford University, Stanford CA 94305-9025
\end{center}


Algebraic Multigrid (AMG) solvers of large sparse linear system of equations are based on multigrid principles but no do explicitly use the geometry of the grids.
The emphasis in AMG is on black on procedures for coarsening the set of equations, relying on its algebraic relations.
Although AMG is widely employed, e.g.
for solving second-order elliptic discretized PDEs with disordered coefficients on unstructured grids, its scope is rather limited to these cases.
However, AMG may be remedied, by systemtically understanding and improving its ingredients: relaxation, coarse levels and inter-level transfers.

We present a novel approach for selecting the coarse variables for AMG.
Classical AMG coarse set construction relies on the strength of local algebraic connections between variables, that are often misleading and inaccurate.
Alternatively, our coarse set quality measure is the rate of convergence of compatible relaxation (CR) [Brandt, ETNA, 2000].
It also leads to a construction algorithm of the coarse set.
We will present numerical examples of our algorithm for simple model problems, e.g., anistropic diffusion and the biharmonic operator, for which classical AMG often fails.

This work makes it possible to assess and construct a coarse set of variables for a given system, prior to its actual use by an AMG solver.
Undergoing research is focused on attaining CR rates (that play a similar role to the smoothing rate in geometric multigrid, predicting the ideal multigrid efficiency).
This relates to self-correcting AMG (a joint work with Prof.
A.
Brandt, Weizmann Institute) that iteratively derives its own interpolation operators.


\newpage	%% HEY

\begin{center}
\rule{6in}{1pt}\\
{\large 36. \rule{0mm}{1.5em}Irene Livshits \\
{\bf AMG Wave-Ray Solver for Helmholtz Eigenvalue \\
Problem: Preliminary Results }}
\end{center}

\begin{center}
Irene Livshits \\
Department of Mathematics, University of Central Arkansas \\
Conway AR 72035
\end{center}


Numerical solvers for indefinite Helmholtz equations have to deal with a range of erroneous components which have a slow convergence by standard relaxation procedures.
Brandt and Livshits suggested wave-ray approach that allows to overcome such slowness by introducing ray representations and ray treatment of problematic components.
This approach, however, does not have a straightforward extension for Helmholtz operators with variable coefficients, since it requires a knowledge of analytical solutions of homogeneous Helmholtz operator (principal components).


In this talk I will discuss the ways of modifying the wave-ray approach for solving eigenvalue problems for Helmholtz operators with variable coefficients.
The presented algorithm employs Galerkin based algebraic multigrid, and modified wave-ray approach, which employs the best current approximation to the solution instead of principal components.
The algorithm and the results of the numerical experiments will be discussed on the example of one-dimensional Helmholtz operator with periodic boundary conditions.
Joint work with A.~Brandt.




\begin{center}
\rule{6in}{1pt}\\
{\large 37. \rule{0mm}{1.5em}Scott MacLachlan \\
{\bf Adapting Algebraic Multigrid }}
\end{center}

\begin{center}
Scott MacLachlan \\
Department of Applied Mathematics \\
University of Colorado at Boulder \\
526 UCB, Boulder CO 80309-0526
\end{center}


Our ability to numerically simulate physical processes is severely constrained by our ability to solve the complex linear systems that are often at the core of the computation.
Multigrid methods offer an efficient solution technique for many such problems.
However, fixed multigrid processes are based on an overall assumption of smoothness that may not be appropriate for a given problem.
Our aim is to develop an adaptive multigrid scheme that replaces this predetermined sense of smoothness by one that is determined automatically.

This paper focuses on the principal component of such a scheme: adaptive interpolation.
Our method is based on computing a representative error component that is not quickly reduced by relaxation and fitting interpolation so that it is eliminated by the coarse-grid correction process.
Numerical results are given to support the efficiency of this approach.


This research has been performed in collaboration with Marian Brezina, Tom Manteuffel, Steve McCormick, and John Ruge at CU-Boulder, and Rob Falgout at CASC-LLNL.
It has been supported by an NSF SciDAC grant (TOPS).




\begin{center}
\rule{6in}{1pt}\\
{\large 38. \rule{0mm}{1.5em}William F. Mitchell \\
{\bf PHAML: A Parallel Adaptive Multilevel Program \\
for Elliptic PDEs }}
\end{center}

\begin{center}
William F.~Mitchell \\
Mathematical and Computational Sciences Division \\
NIST, Gaithersburg MD 20899 \\
\verb9http://math.nist.gov/~mitchell/9
\end{center}


A new program for the solution of elliptic partial differential equations,
PHAML, has recently been released.
PHAML stands for Parallel Hierarchical Adaptive MultiLevel.
It is written in Fortran 90, and compiles to a library of
routines to be called by the application program.
The primary routine solves an elliptic boundary value or eigenvalue problems using finite elements with bisection adaptive refinement of triangles,
a hierarchical multigrid solver,
and distributed memory message passing parallelism with MPI or PVM.
Examples illustrate how the program can be used to solve parabolic problems,
nonlinear problems and systems of equations.

Optional additional libraries that PHAML can make use of include OpenGL for graphics,
Zoltan for grid partitioning,
and PETSc for linear system solvers,
through which PHAML provides an environment for experimenting with different methods.
In this talk, we will describe the design and use of the PHAML software.
PHAML can be obtained at
\verb9http://math.nist.gov/phaml9.


\begin{center}
\rule{6in}{1pt}\\
{\large 39. \rule{0mm}{1.5em}J. David Moulton, Joel E. Dendy, M. Shashkov \\
{\bf A Comparison of Mimetic and Variational Preconditioners \\
for Mixed-Hybrid Discretizations of the Diffusion \\
Equation. }}
\end{center}

\begin{center}
J.~David Moulton, Joel E.~Dendy, M.~Shashkov \\
Mathematical Modeling and Analysis Group \\
Theoretical Division \\
Los Alamos National Laboratory, Los Alamos NM 87545 \\
Jim E.~Morel \\
Transport Methods Group \\
Computer and Computational Sciences Division \\
Los Alamos National Laboratory, Los Alamos NM 87545
\end{center}


In many application areas the diffusive component offers a significant challenge because it is characterized by a discontinuous diffusion coefficient with fine-scale anisotropic spatial structure; moreover, the underlying grid may be severely distorted.
Thus, increasingly Mixed and Mixed-Hybrid discretizations (e.g., Support Operator Methods, and Mixed-Hybrid finite element methods) are employed because they explicitly enforce important physical properties of the problem, such as mass balance.
However, these discretizations are based on the first order form, and hence, naturally lead to an indefinite linear system.
Thus, the primary hurdle in the widespread adoption of these methods has been the robust and efficient solution of this linear system.


In the mixed-hybrid case it is possible to eliminate the flux (i.e., the vector unknowns) locally to obtain a sparse symmetric positive definite system.
Unfortunately, the nonstandard sparsity structure of this reduced system has posed a significant challenge for the design of efficient multigrid preconditioners.
In this presentation we compare two different approaches to preconditioning these systems.

First, we consider a mimetic approach to developing alternative or approximate discretizations that generate reduced systems that have a sparsity structure amenable to existing robust multigrid solvers.
In contrast, we consider the direct application of variational coarsening, the robust but potentially costly technique that lies at the heart of most robust multigrid algorithms.

Both of these techniques have been used to generate a number of preconditioners for orthogonal grids, as well as distorted logically rectangular grids.
We will discuss the varying degrees of robustness associated with these preconditioners, as well as their potential extension to unstructured grids.




\begin{center}
\rule{6in}{1pt}\\
{\large 40. \rule{0mm}{1.5em}Maria Murillo, Xiao-Chuan Cai \\
{\bf A Fully Implicit Parallel Algorithm for Simulating \\
the Nonlinear Electrical Activation of the Heart \\
}}
\end{center}

Maria Murillo \\
Xiao-Chuan Cai \\
430 UCB, University of Colorado \\
Boulder CO 80309

In this research we developed and tested a fully implicit and highly parallel Newton-Krylov-Schwarz method for solving the bidomain equations representing the excitation process of the heart.
Newton-Krylov-Schwarz method has been used successfully for many nonlinear problems, but this is the first attempt to use this method for the bidomain system which consists of time dependent partial differential equations of mixed types.
Our experiments on parallel computers show that the method is scalable and robust with respect to many of the parameters in the bidomain system.


In the outer layer of the algorithm, we use a nonlinearly implicit backward Euler method to discretize the time derivative, and the resulting systems of large sparse nonlinear equations are solved using an inexact Newton method.
The Jacobian system required solving in each Newton iteration is solved with a preconditioned GMRES method.
The efficiency and robustness of the overall method depends heavily on the preconditioning step of the linear solver.
By comparing several preconditioners, we found the restricted additive Schwarz method offers the best performance.

Our parallel software is developed using the PETSc package of Argonne National Laboratory, and our numerical results were obtained on� the IBM-SP of the San Diego Supercomputer Center.




\begin{center}
\rule{6in}{1pt}\\
{\large 41. \rule{0mm}{1.5em}Daniel Oeltz \\
{\bf An Algebraic Multigrid Preconditioner for Topology \\
Optimization }}
\end{center}

\begin{center}
Daniel Oeltz \\
Rheinische Friedrich-Wilhelms-Universit\"{a}t Bonn \\
Institut f\"{u}r Angewandte Mathematik \\
Abteilung Wissenschaftliches Rechnen und Numerische Simulation \\
Wegelerstrasse 6, 53115 Bonn, Germany \\
Michael Griebel
\end{center}


Topology optimization is a relatively new field of structural mechanics which tries to find an optimal design of a structure for a given load.
One flexible and advantageous FE based method is the so called SIMP approach (Solid Isotropic Microstructures with Penalization for intermediate densities, cf.
[1], [2]).
In this approach the problem is reformulated to find an optimal density distribution of the material in the domain.

Therefore the optimization process leads to elasticity problems with large jumps in the material coefficients and linear systems which become nearly singular so that classical iteration schemes for solving the linear systems provide only very poor convergence.
Since the accuracy of the approximate solution plays an essential role for the optimized design (solving the linear system with low accuracy leads only to suboptimal designs) direct methods are often used during the optimization process which strictly limits the problem size.

To overcome this problem and accelerate the convergence of classical iteration schemes we propose an algebraic multigrid preconditioner.
Since algebraic multigrid methods can deal with large jumps in the coefficient functions and complex geometries, they seem to be well fit for the linear systems occuring in the SIMP approach.

However, due to the fact that we have linear systems stemming from the discretization of a system of PDEs and classical AMG methods are not well suited to solve such systems efficiently, we use the so-called point-block approach [3].
We will describe the modifications of this method to obtain an efficient preconditioner for the conjugate gradient method in our optimization process.
At the end we will present some numerical results which underline the efficiency of the method for topology optimization.

[1] Bendsoe, M.P.,
{\em Optimal Shape design as a material distribution problem},
Struct.~Optim {\bf 1} (1989), pp.193--202.

[2] Rozvany, G.I.N. and Zhou M.,
{\em Applications of COC method in layout optimization}.
In: Eschenauer H, Mattheck C and Olhoff N (eds.),
Proc.~Conf.~``Eng.~Opt.~in Design Processes'', Berlin,
Springer-Verlag, 1991, pp.59--70.

[3] Griebel, M., Oeltz, D. and Schweitzer, M.A.,
{\em An Algebraic Multigrid Method for Linear Elasticity},
SIAM J.~Sci.~Comp., to appear.


\newpage	%% HEY


\begin{center}
\rule{6in}{1pt}\\
{\large 42. \rule{0mm}{1.5em}Luke Olson \\
{\bf A  {\em Dual} Least-Squares Finite Element Method \\
for Linear Hyperbolic PDEs: A Numerical Study}}

Luke Olson \\
Department of Applied Mathematics, 526 UCB \\
University of Colorado at Boulder, Boulder CO 80309-0526 \\
Tom Manteuffel, Steve McCormick, Hans De Sterck
\end{center}


We develop a least-squares finite element method for linear Partial Differential Equations (PDEs) of hyperbolic type.
This formulation allows for discontinuities in the numerical approximation and yields a linear system which can be handled efficiently by Algebraic Multigrid solvers.

We pose the classical advection equation as a ``dual''-type problem and relate the formulation to previous attempts.
Convergence properties and solution quality for discontinuous solutions are investigated for standard, conforming finite element spaces on quasi-uniform tessellations of the domain and for the general case when the flow field is not aligned with the computational mesh.
Algebraic Multigrid results are presented for the linear system arising from the discretization and we study the success of this solver.



\begin{center}
\rule{6in}{1pt}\\
{\large 43. \rule{0mm}{1.5em}C. W. Oosterlee \\
{\bf An Efficient Multigrid Solver based on Distributive \\
Smoothing for Poroelasticity Equations }}
\end{center}

\begin{center}
C.W.  Oosterlee \\
Delft University of Technology, Mekelweg 4, 2628 CD Delft, the Netherlands \\
F.J.~Gaspar, F.J.~Lisbona, R.~Wienands
\end{center}


Poroelasticity has a wide range of applications in biology, filtration
and soil sciences.  It represents a model for problems where an
elastic porous solid is saturated by a viscous fluid.  The
poroelasticity equations were derived by Biot, studying the
consolidation of soils.  The equations have also been applied to
the study of soft tissue compression to model the deformation and
permeability of biological tissues.

We introduce an efficient multigrid method for the system of
poroelasticity equations.  In particular, we present a pointwise
smoothing method based on distributive iteration.  In distributive
smoothing the original system of equations is transformed by
post-conditioning in order to achieve favorable properties, such
as a decoupling of the equations and/or possibilities for pointwise
smoothing.

A specialty lies in the discretization approach employed.  We adopt
a so-called staggered grid for the poroelasticity equations, whereas
the usual way to discretize the equations is by means of finite
elements.  Standard finite elements (or finite differences), however,
do not lead to stable solutions without additional stabilization.
The staggered grid discretization, as proposed in [3], leads to a
natural stable discretization for the poroelasticity system.
Staggering is a well-known discretization technique in computational
fluid dynamics, in particular for incompressible flow.  The multigrid
method is developed with analysis possibilities of increasing
complexity on the basis of Fourier analysis.  After the analysis
of the determinant of the system of equations, the h-ellipticity
concept is discussed, which is fundamental for the existence of
point smoothers.  The smoother is developed based on insights in
distributive smoothers for Stokes and incompressible Navier-Stokes
equations [1,2,4,5].  It is evaluated and tuned with relaxation
parameters on the basis of Fourier smoothing analysis.  An
equation-wise decoupled smoother with as principal operators simple
Laplacian and biharmonic operators is obtained, although the blocks
in the system of poroelasticitye equations contain anisotropies.
Numerical experiments confirm the efficiency of the method proposed.

[1] A.~Brandt,
{\em Multigrid techniques: 1984 guide with applications to fluid dynamics},
GMD-Studie {\bf 85} (1984), Sankt Augustin, Germany.

[2] A.~Brandt and N.~Dinar,
{\em Multigrid solutions to elliptic flow problems}, in:
S.~Parter, ed., Numerical methods for partial differential equations
(1979), Academic Press, New York, pp.53--147.

[3] F.J.~Gaspar, F.J.~Lisbona and P.N.~Vabishchevich,
{\em A finite difference analysis of Biot's consolidation model},
Appl.~Num.~Math., to appear.

[4] U.~Trottenberg, C.W.~Oosterlee, and A.~Schueller,
{\em Multigrid} (2001), Academic Press, New York

[5] G.~Wittum,
{\em Multi-grid methods for Stokes and Navier-Stokes equations
with transforming smoothers: algorithms and numerical results}
(1989), Num.~Math. {\bf 54}, pp.543--563.




\begin{center}
\rule{6in}{1pt}\\
{\large 44. \rule{0mm}{1.5em}Andreas Papadopoulos \\
{\bf  Block Smoothed Aggregation Algebraic MultiGrid \\
for Reservoir Simulation Systems }}
\end{center}

\begin{center}
Andreas Papadopoulos \\
Oxford Univesrity Computing Labaratory, Wolfson Building, Parks Road, \\
Oxford, England OX1 3QD \\
Hamdi Tchelepi \\
ChevronTexaco Exploration \& Production Technology Co. \\
\verb9htch@chevrontexaco.com9
\end{center}


We study the applicability of Smoothed Aggregation Algebraic MultiGrid (SA-AMG) for oil reservoir simulation problems.
Previously, application of multigrid methods in reservoir simulation focused on solving the scalar near-elliptic pressure equation, which is obtained by the so-called IMPES (Implicit Pressure, Explicit Saturation) method.
Here, we focus on linear systems with multiple degrees of freedom per gridblock, or cell.
These systems are Jacobean matrices obtained from fully implicit discretization of the governing nonlinear species conservation equations.
We view the given discrete matrix problems as graphs, and we do not make any assumptions about the underlying structure of the grid.
A preprocessing step, which manipulates the discrete system to yield an equation that resembles the standard pressure equation, is performed.
The result is a system of equations of mixed character; a pressure equation that is near elliptic, while the remaining conservation equations are hyperbolic in the limit.

All of our computations, including this preprocessing step, take advantage of the natural block structure obtained by grouping the unknowns associated with a gridblock (i.e.
vertex in the graph) together.
Significant modifications to the components that make up the SA-AMG algorithm were necessary to obtain good convergence behavior.
We experimented with several aggregation schemes.
Aggregation of the cells, or gridblocks, based on a strong graph of the scalar pressure equation yielded the best results, both in terms of robustness and convergence rate.
That is, variables associated with a gridblock other than the pressure inherit the pressure-based choice of the aggregates.
We report results using different measures of strength to prune the full graph of the pressure equation.

Not surprisingly, we also found it crucial to pay close attention to the anisotropy.
This was the case for the aggregation stage, the smoothing of aggregation, and even the smoother used in the V-cycle.
We describe a worm-like aggregation algorithm designed to capture long-range strength of coupling.
This new aggregation scheme allows for aggressive coarsening, especially when the finest level (i.e.
the given problem) possesses severe anisotropy.
It is not uncommon for discretizations of three-dimensional reservoir simulation problems to exhibit extreme preferential directional coupling.
We obtained the best results when smoothing of interpolation was applied only to the pressure coefficients.

Attempts to improve the quality of the interpolation by applying multiple smoothing steps proved ineffective.
The results were sensitive to the particular threshold for filtering the original matrix, and we use a different parameter to control that threshold.
Finally, in conjunction with worm aggregation, we employ block smoothing in the V-cycle, where the blocks are the cells (and their associated degrees of freedom) that make up an aggregate.
We provide detailed performance results using several reservoir simulation examples that are of practical interest.




\begin{center}
\rule{6in}{1pt}\\
{\large 45. \rule{0mm}{1.5em}C. E. Powell, David Silvester \\
{\bf Black-Box Preconditioning for Mixed Formulation \\
of Second-Order Elliptic Problems }}
\end{center}

\begin{center}
C.E.~Powell, David Silvester \\
UMIST, Mathematics Department, \\
PO Box 88, Sackville Street, \\
Manchester, M60 1QD, UK
\end{center}


Raviart-Thomas mixed finite element approximation to scalar diffusion problems is well understood.
The method gives rise to a symmetric and indefinite linear system, that can be solved in a variety of ways.
For example, transformations to associated positive definite systems are common.
However, solving the original indefinite system using minimal residual schemes is not problematic.
Incorporating fast solvers for Laplacian or generalised diffusion operators into block diagonal preconditioners for the mixed system is a simple and highly effective technique.


The existence of freely available black-box algebraic multigrid codes makes this a feasible and unified preconditioning strategy for a wide class of saddle-point problems.
Further, it offers the possibility to treat problems with diverse coefficients in the same framework.
For variable and discontinuous coefficient problems, known preconditioning strategies can lose robustness.
The linear systems are generally ill-conditioned with repect to both the discretisation parameter and the PDE coefficients.


We discuss a robust, black-box approach to preconditioning the mixed diffusion problem, using freely available software.
The key tools are diagonal scaling for a weighted mass matrix and an algebraic multigrid V-cycle applied to a sparse approximation to a generalised diffusion operator.
Eigenvalue bounds are derived for the preconditioned system matrix.
Numerical results are presented to illustrate that the preconditioner is optimal with respect to the discretisation parameter and is robust with respect to the PDE coefficients.

[1] Powell, C.E., Silvester, D., Optimal preconditioning
for Raviart-Thomas mixed formulation of second-order elliptic PDEs,
submitted to {\em SIAM J. Matrix Anal.}, 2002.

[2] Powell, C.E., Silvester, D.,
Black-box preconditioning for mixed formulation of
self-adjoint elliptic PDEs,
{\em Manchester Centre for Computational Mathematics Report 415}, 2002.



\begin{center}
\rule{6in}{1pt}\\
{\large 46. \rule{0mm}{1.5em}Ulrich Ruede \\
{\bf Multigrid Accelleration of the Horn-Schuck Algorithm \\
for the Optical Flow Problem }}
\end{center}

\begin{center}
Ulrich Ruede \\
Institute of Computer Science X, Universitaet Erlangen-Nuernberg, Cauerstr.
6, D-91058 Erlangen, Germany \\
El Mostafa Kalmoun
\end{center}


It is currently recognized that optical flow computation has many
applications in image processing, pattern recognition, data
compression, and biomedical technology.  Differential approaches, which
estimate velocity vectors from the spatial and temporal intensity
derivatives are the most common visited techniques in the literature.
The basic assumption behind that is the intensity variations are weak
and only due to a movement in the image plan.

This constant brightness assumption leads to an ill-posed problem that
can only be solved by imposing additional assumptions.  A standard
technique, which is due to to Horn and Schunck, is to require the flow
field to be smooth by means of a standard regularization approach.
This results in a system of elliptic PDEs of reaction diffusion type.
The second order terms are induced by the regularization and become
straightforward Laplace operators.  The zero order terms are linear
(but variable) and are computed from the derivatives of the image
data.  Since the image data (and even more so its derivatives) are
usually nonsmooth, this poses some nontrivial problems.  The PDEs are
usually discretized by standard finite differences.

The Horn-Schuck algorithm is the standard way to discretize and solve
the PDEs.  In multigrid terminology, it is simply a coupled pointwise
relaxation.  As expected, it has acceptable convergence only if the
zero-order terms dominate but the performance is poor, when the
diffusion character dominates.  In this case, a multigrid strategy can
be expected to provide a significant accelleration by allowing a quick
propagation of information from non-zero flow field regions into
homogeneous or untextured image areas.  The Horn-Schunck algorithm can
still be used as a smoother for such a multigrid method.  The most
difficult aspect is to deal with the strongly discontinuous
coefficients in an efficient way.  This will be discussed in our
presentation.


\newpage	%% HEY


\begin{center}
\rule{6in}{1pt}\\
{\large 47. \rule{0mm}{1.5em}Ulrich Ruede \\
{\bf Accurate multigrid techniques for computing singular \\
solutions of elliptic problems. }}
\end{center}

\begin{center}
Ulrich Ruede \\
Institute of Computer Science X, Universitaet Erlangen-Nuernberg \\
Cauerstr. 6, D-91058 Erlangen, Germany \\
Harald Koestler, Marcus Mohr
\end{center}


Generalized functions occur in many practical applications as source
terms in partial differential applications.  Typical examples are point
sources and sinks in porous media flow that are described by Dirac
delta functions or point loads and dipoles as source terms inducing
electrostatic potentials.  We are particularly interested in
bioelectric field computations where the source terms are modeled by
dipoles and where the computational goal is to locate dipole sources as
accurately as possible from electroencephalographic measurements.


For analyzing the accuracy of such computations, standard techniques
cannot be used since they rely on global smoothness.  This is both true
for Sobolev space arguments for finite element based methods, and for
continuity and differentiability arguments in finite difference
analysis.  At the singularity, the solution tends to infinity and
therefore standard error norms will not even converge.


In this presentation we will demonstrate that these difficulties can be
overcome by using other metrics to measure accuracy and convergence of
the numerical solution.  Only minor modifications to the discretization
and solver are necessary to obtain the same asymptotic accuracy and
efficiency as for regular and smooth solutions.  In particular, no
adaptive refinement is necessary and it is also unnecessary to use
techniques which make use of the analytic knowledge of the
singularity.  Our method relies simply on a mesh-size dependent
representation of the singular sources constructed by appropriate
smoothing.

It can be proved that the pointwise accuracy is of the same
order as in the regular case.  The error coefficient depends on the
location and will deteriorate when approaching the singularity where
the error estimate breaks down.  Our approach is therefore useful for
accurately computing the global solution, except in a small
neighborhood of the singular points.  In the talk we will demonstrate
how these techniques can be integrated into a multigrid solver
exploiting additional techniques for improving the accuracy, such as
Richardson and tau-Extrapolation.  The talk is a follow-up to a paper
presented in Copper Mountain in 1987.




\begin{center}
\rule{6in}{1pt}\\
{\large 48. \rule{0mm}{1.5em}John W. Ruge \\
{\bf AMG for Higher-Order Discretizations of Second-Order \\
Elliptic Problems }}
\end{center}

\begin{center}
John W. Ruge \\
1005 Gillaspie Dr., Boulder CO 80305
\end{center}


Algebraic Multigrid (AMG) has been shown to be a very effective solver
for standard finite-difference and finite-element discretizations of a
wide range of second-order elliptic PDEs.  This talk covers some of the
issues related to the use of AMG on discretizations using higher-order
finite elements.  These include: problems encountered when no special
handling is used; variations of the AMG algorithm better suited for
such problems; the effect of the choices of basis functions; and
measures of accuracy versus work for various orders of approximation.
Numerical results are reported.




\begin{center}
\rule{6in}{1pt}\\
{\large 49. \rule{0mm}{1.5em}Dominik Smits, Stefan Vandewalle \\
{\bf Experiences with algebraic multigrid for a 2D \\
and 3D biological respiration-diffusion model \\
}}
\end{center}

\begin{center}
Dominik Smits, Stefan Vandewalle \\
Katholieke Universiteit Leuven, Department of Computerscience \\
Celestijnenlaan 200A, B-3001 Leuven, Belgium \\
Nico Scheerlinck, Bart Nicola \\
Katholieke Universiteit Leuven, Laboratory for PostHarvest Technology \\
De Croylaan 42, B-3001 Leuven, Belgium
\end{center}


At the Laboratory of PostHarvest Technology of the University of Leuven,
a respiration-diffusion model is being developed and studied for the oxygen consumption and carbon dioxyde production inside harvested fruit
(in particular for the Conference Pear).
The research aims at a better understanding of the respiratory activity
of fruit and the causes that affect the onset of certain fruit diseases (e.g.,
the diseases `brown and hollow' or `core breakdown').
The current mathematical model consist of a set of two coupled
non-linear reaction diffusion equations, defined on a two- or
three-dimensional domain, with a mixed type of boundary condition.


In this talk, we will present our experiences with using an algebraic
multigrid method for solving the set of equations obtained after a
finite element discretization of the mathematical model.
We have concentrated on the use of the recent version of the systems AMG code developed by Klaus Stuben,
at the Fraunhofer Institute for Algorithms and Scientific Computing,
Sankt Augustin, Germany.
We will consider its application for solving both the steady-state
problem and the time-evolution problem.
For the latter case, we will discuss the use of different time-discretisation methods of backward differentiation or implicit Runge-Kutta type.
The AMG-results will be compared with the results obtained with classical, single-level solvers.




\begin{center}
\rule{6in}{1pt}\\
{\large 50. \rule{0mm}{1.5em}Hans De Sterck \\
{\bf Least-Squares Methods for Hyperbolic Conservation \\
Laws }}
\end{center}

\begin{center}
Hans De Sterck \\
Department of Applied Mathematics, University of Colorado \\
526 UCB, Boulder CO 80309-0526 \\
Tom Manteuffel, Steve McCormick, Luke Olson
\end{center}


Least-squares finite element methods for the inviscid Burgers equation in space-time domains are presented.
Numerical results show that convergence of
a standard least-squares approach for
$\mbox{div}(u,u^2)=0$
with bilinear elements for $u$ and Newton linearization is problematic,
possibly due to difficulties with the linearizability of
the Burgers operator around discontinuous solutions.
Alternative $H(\mbox{div})$-conforming formulations are proposed
in terms of the flux variables (or their De Rahm-dual) using
face and edge elements.
The face element formulation does not exhibit exact numerical
conservation at the discrete level,
but it is shown that the formulation converges to a
conservative weak solution with the correct shock speed.
The dual edge element formulation is
strictly conservative at the discrete level.
The least-squares functional naturally provides a
sharp a posteriori error estimator that is used
for adaptive refinement in space-time.
Standard algebraic multigrid methods are applied
to the positive semi-definite matrices resulting
from the new least-squares formulations,
and multigrid efficiency as a function of problem size is investigated.
Extension of the new least-squares methods to general systems of hyperbolic conservation laws in multiple dimensions is discussed.


\begin{center}
\rule{6in}{1pt}\\
{\large 51. \rule{0mm}{1.5em}J. L. Thomas \\
{\bf Towards Textbook Multigrid Efficiency for Computational \\
Fluid Dynamics: Applications to Stagnation Flows
}}

J. L. Thomas \\
MS 128 NASA Langley Research Center, Hampton VA 23681 \\
B. Diskin, R. Mineck
\end{center}


The ingredients for attaining textbook multigrid efficiency in solution of CFD problems are discussed as they arise in application to stagnation flow problems.
The ingredients include principal linearization, factorizable schemes, local relaxation, etc.
Textbook multigrid efficiency is demonstrated for stagnation flows with a pressure-equation formulation of the incompressible fluid equations.
Both inviscid and viscous flows over a range of Reynolds number are considered.
Convergence of algebraic errors below discretization errors in one full multigrid cycle is attained using flow-dependent relaxation schemes.
The residual convergence rates for the systems are the same as for scalar elliptic equations on the same grid.




\begin{center}
\rule{6in}{1pt}\\
{\large 52. \rule{0mm}{1.5em}Danny Thorne \\
{\bf Cache Aware Multigrid on AMR Hierarchies }}
\end{center}

\begin{center}
Danny Thorne \\
325 McVey Hall -- CCS \\
Lexington KY 40506-0045
\end{center}


A cache optimized multilevel algorithm to solve variable coefficient elliptic boundary value problems on adaptively refined structured meshes is described here.
The algorithm is optimized to exploit the cache memory subsystem.
Numerical results are given demonstrating the efficiency of the cache optimization.




\begin{center}
\rule{6in}{1pt}\\
{\large 53. \rule{0mm}{1.5em}Ray Tuminaro \\
{\bf Algebraic Multigrid for Maxwell's Equations in \\
the Frequency Domain }}
\end{center}

\begin{center}
Ray Tuminaro \\
Computational Mathematics and Algorithms,
Sandia National Laboratories \\
P.O. Box 969, MS 9217, Livermore CA 94551 \\
Jonathan Hu \\
Computational Mathematics and Algorithms,
Sandia National Laboratories \\
Greg Newman \\
Sandia National Laboratories \\
Geophysical Technology,
Sandia National Laboratories \\
Pavel B.  Bochev \\
Computational Mathematics and Algorithms,
Sandia National Laboratories
\end{center}


We describe a parallel algebraic multigrid method for the solution of Maxwell's equations in the frequency domain.
The underlying formulation leverages off of an algebraic multigrid scheme for real valued Maxwell problems.
This real valued method uses distributed relaxation and a specialized grid transfer operator.
The key to this multilevel method is the proper representation of the ({\bf curl},{\bf curl}) null space on coarse meshes.
This is achieved by maintaining certain commuting properties of the inter-grid transfers.


To adapt the real value scheme to complex arithmetic, equivalent
real forms are considered.  The complex operator is written as a
$2\times2$ real block matrix system.  The real valued multigrid
algorithm can then be used to generate grid transfers which are
adapted to the equivalent real form of the problem.  In order to
complete the scheme a smoother must also be adapted to address the
equivalent real form.  We will show how application of the distributed
relaxation idea on the equivalent real form matrix leads to a nice
decoupling of the problem.  To complete the method, complex polynomial
smoothers are developed for use within the distributed relaxation
process.  While some care is required to develop the polynomials,
they work well in parallel and avoid difficulties associated with
parallel Gauss-Seidel.


Numerical experiments are presented for some 3D problems arising
in geophysical subsurface imaging.  The experiments illustrate the
efficiency of the approach on various parallel machines in terms
of both convergence and parallel speed-up.



\newpage	%% HEY

\begin{center}
\rule{6in}{1pt}\\
{\large 54. \rule{0mm}{1.5em}Markus Wabro \\
{\bf Algebraic Multigrid Methods for the Oseen Problem \\
}}
\end{center}

\begin{center}
Markus Wabro \\
J.K.  University Linz \\
Institute for Computational Mathematics \\
Altenbergerstr. 69, A-4040 Linz, Austria
\end{center}


We present and compare concepts for algebraic multigrid (AMG) solvers for the Oseen linearization of the Navier-Stokes equations for incompressible fluids.

The two main strategies in this area are the following.
The first one is the segregated approach, where the equations for velocity and pressure are iteratively decoupled, and AMG is used for the solution of the resulting scalar problems (examples in this direction are Uzawa or SIMPLE schemes, or preconditioners for Krylov-space methods, e.g.
as introduced by Silvester, Wathen et al., 2001).

The main topic of the talk will be the second strategy, the coupled approach, where an AMG method is developed for the whole saddle-point system.
We present the ingredients of this method (smoothers, coarse level construction) and pinpoint a major problem, the stability of the coarse-level systems.

Finally, we will show how the different methods perform in
``real life'' situations, i.e.,
when flows in complex 3D domains have to be simulated.


\begin{center}
\rule{6in}{1pt}\\
{\large 55. \rule{0mm}{1.5em}Justin Wan \\
{\bf An Energy Minimization Approach to Multigrid \\
for Convection Dominated Problems }}
\end{center}


\begin{center}
Justin Wan \\
School of Computer Science, University of Waterloo \\
200 University Avenue West, Waterloo, Ontario N2L 3G1, Canada \\
Randy Bank, University of California, San Diego \\
Zhenpeng Qu, University of Waterloo
\end{center}


In this talk, we present an energy minimization approach to constructing robust interpolation and restriction operators for multigrid for solving convection dominated problems.
Numerical results for PDEs in 1D and 2D, in particular, entering flow and recirculating flow problems will be presented.


Multigrid for solving elliptic partial differential equations (PDEs) with smooth coefficients has been proven, both numerically and theoretically, to be a successful and powerful techniques.
For PDEs whose coefficients are discontinuous or oscillatory, sophisticated interpolation techniques have been developed, for instance, algebraic multigrid, black box multigrid, etc.
Recently, Wan, Chan and Smith proposed a robust construction of interpolation operators by minimizing the energy norm of the coarse grid basis functions subject to the constraint that the coarse grid basis functions preserve the null space of the underlying PDE.
Fast multigrid convergence is resulted for several types of nonsmooth coefficient elliptic PDEs on Cartesian as well as general triangular meshes.


In this talk, we extend the energy minimization approach to non-elliptic PDEs; specifically, convection diffusion equations.
Since the discretization matrix is nonsymmetric in general, the energy norm induced by the symmetric positive definite matrix in the case of elliptic PDEs no longer applicable.
In other words, one cannot simply take the matrix and construct energy minimizing basis function using the induced matrix norm.
Our idea is to construct two sets of basis functions, one by minimizing the energy norm of the elliptic operator, and the other set by minimizing the
$L_2$ norm, subject to appropriate constraints.
In this way, the norms are well-defined so that norm minimization makes sense.
More importantly, we can capture the elliptic and hyperbolic parts of the underlying PDE by these two set of basis functions.
Then they are used in defining the interpolation and restriction operators for multigrid.
Numerical results for solving the entering and recirculating flow problems are presented to demonstrate the effectiveness of the proposed approach.




\begin{center}
\rule{6in}{1pt}\\
{\large 56. \rule{0mm}{1.5em}Yanqiu Wang \\
{\bf Overlapping Schwarz preconditioner for the mixed \\
formulation of plane elasticity }}


Yanqiu Wang \\
Department of Mathematics, Texas A\&M University, College Station TX 77843 \\
\end{center}


Recently a stable pair of finite element spaces for the mixed formulation of the plane elasticity system has been developed by Arnold and Winther.
Here we construct a two-level overlapping Schwarz preconditioner for the resulting discrete system.
Essentially, this reduces to finding an efficient preconditioner
for the form
$ (\cdot,\cdot)+(\div\cdot,\div\cdot) $
in the symmetric tensor space
$H\div$.
The main difficulty comes from the well known complexity of building preconditioners for the
{\bf div} operator.
We solve it by taking a decomposition similar to the Helmholz decomposition.
Both additive and multiplicative preconditioners are studied,
and the conditioner numbers are shown to be
uniform with respect to the mesh size.



\begin{center}
\rule{6in}{1pt}\\
{\large 57. \rule{0mm}{1.5em}Chad Westphal \\
{\bf First-Order System Least Squares (FOSLS) for \\
Geometrically-Nonlinear Elasticity }}
\end{center}

\begin{center}
Chad Westphal \\
526 UCB, Boulder CO 80309
\end{center}


In this talk we discuss developments in a first-order system least squares (FOSLS) approach for the numerical approximation of the solution of the equations of geometrically-nonlinear elasticity.
We follow a Newton-FOSLS algorithm where each linear step is solved as a two-stage, first-order system under a least squares finite element discretization.
With appropriate regularity we show $H^1$
equivalence of the quadratic part of the FOSLS functional norm in the case of pure displacement boundary conditions.

Results hold for deformations near the reference configuration, a set we show to be largely coincident with the set of deformations allowed by the physical model.
In this regime the discrete systems that result from using standard finite element subspaces of
$H^1$ can be solved with optimal complexity.
Numerical results are given for both pure displacement and mixed boundary conditions and confirm optimal multigrid performance and finite element approximation properties.




\begin{center}
\rule{6in}{1pt}\\
{\large 58. \rule{0mm}{1.5em}Jinchao Xu \\
{\bf Multigrid and grid adaptation }}
\end{center}

\begin{center}
Jinchao Xu \\
Department of Mathematics, Pennsylvania State University,
University Park PA 16802
\end{center}


This talk will be devoted to issues related
to multigrid method and grid adaptation.
Some new results will be reported,
including a new a posteriori error estimate
using idea from multigrid (joint work with Randy Bank).




\begin{center}
\rule{6in}{1pt}\\
{\large 59. \rule{0mm}{1.5em}Ulrike M. Yang \\
{\bf On the Use of Relaxation Parameters in Hybrid \\
Smoothers }}
\end{center}

\begin{center}
Ulrike Meier Yang \\
Center for Applied Scientific Computing \\
Lawrence Livermore National Laboratory \\
Box 808, L-560, Livermore CA 94551
\end{center}


With the advent of high performance computers with massively parallel
processors, it has become very important to develop scalable methods
such as AMG.  One of AMG's most important components is the smoother.
The most effective smoothers, such as Gauss-Seidel or even
multiplicative Schwarz smoothers are often sequential in nature,
whereas Jacobi or block Jacobi-like smoothers often fail, unless
an appropriate smoothing parameter is used and even then their
convergence is often slow.  Many efforts to parallelize Gauss-Seidel
have been made.  Possible variants include the use of multi-coloring
techniques or hybrid smoothers.  Multi coloring techniques are
difficult to implement and can be inefficient, if too many colors
are involved, which is often the case on the coarser levels on AMG.
Hybrid schemes use a sequential smoother on each processor, but
update in a Jacobi like fashion across processor boundaries, and
therefore often have similar disadvantages as Jacobi like smoothers.


In this talk, we investigate the use of relaxation parameters in
hybrid smoothers.  We analyze their influence on the smoothing
property, describe a procedure that automatically determines good
parameters on each level of AMG and present numerical results that
show significant improvement over AMG with unrelaxed hybrid smoothers.


*This work was performed under the auspices of the U.S.  Department
of Energy by University of California Lawrence Livermore National
Laboratory under contract number W-7405-Eng-48.



\begin{center}
\rule{6in}{1pt}\\
{\large 60. \rule{0mm}{1.5em}Irad Yavneh, Gregory Dardyk \\
{\bf A Multilevel Nonlinear Method }}
\end{center}

\begin{center}
Irad Yavneh \\
Department of Computer Science, Technion 32000, Haifa, Israel \\
Gregory Dardyk
\end{center}


Robust multigrid algorithms for linear boundary-value problems are
well-researched.  For nonlinear problems, two general approaches
are used widely: Global Linearization (GL) and Local Linearization
(LL).  In the GL approach, e.g., Newton's method, the discretized
problem is linearized and the resulting linear system is solved
approximately by a (robust) linear multigrid algorithm.  This is
repeated iteratively.  In the LL approach, the nonlinear fine-grid
operator is approximated nonlinearly on the coarser grids, and
explicit linearization is only performed locally, in the relaxation
process.

The best known of the LL methods is the so-called FAS
(Full Approximation Scheme).  For simple problems, the two approaches
often perform similarly, but a distinct behavior is exhibited in
more complicated situations, with the GL approach performing better
in some cases and the LL approach in others.  We propose a Multilevel
Nonlinear Method (MNM) which is designed to be at least as robust
as either one of the classical approaches and often more robust
than both.  A heuristic analysis, supported by numerical experiments,
suggests that MNM indeed yields superior convergence behavior for
difficult nonlinear problems.




\begin{center}
\rule{6in}{1pt}\\
{\large 61. \rule{0mm}{1.5em}Jie Zhao \\
{\bf Nonconforming V-cycle and F-cycle Multigrid methods \\
for the biharmonic problem using the Morley element}}
\end{center}

\begin{center}
Jie Zhao \\
Department of Mathematics, University of South Carolina \\
Columbia SC 29208
\end{center}


The asymptotic behavior of multigrid V-cycle and F-cycle algorithms for the biharmonic problem using the Morley element are presented in this talk.
By the use of an additive theory, we show that the contraction numbers of the algorithms can be uniformly improved as the number of smoothing steps increases, without assuming full elliptic regularity.


We describe the critical estimates required for the additive theory.
Experimental results are also presented for the algorithms on convex
and nonconvex domains.  The results are consistent with the
theoretical estimates.


\begin{center}
\rule{6in}{1pt}\\
{\large 62. \rule{0mm}{1.5em}Ludmil Zikatanov \\
{\bf On an Energy Minimizing Basis for Algebraic Multigrid \\
Methods }}
\end{center}

\begin{center}
Ludmil Zikatanov \\
Department of Mathematics, The Pennsylvania State University \\
University Park PA 16802 \\
Jinchao Xu
\end{center}


This paper is devoted to the study of an energy minimizing basis
first introduced by Chan, Smith and Wan in 2000 for algebraic
multigrid methods.  The basis will be first obtained in an explicit
and compact form in terms of certain local and global operators.
The basis functions are then proven to be locally homonic functions
on each coarse grid ``elememt''.  Using these new results, it is
illustrated that this basis can be numerically obtained in an
optimal fashion.  In addition to the intended application for
algebaric multigrid method, the energy minimizing basis may also
be applied for numerical homogenization.




\begin{center}
\rule{6in}{1pt}\\
{\large 63. \rule{0mm}{1.5em}Walter Zulehner \\
{\bf On kernel-preserving smoothers for saddle point \\
problems }}
\end{center}

\begin{center}
Walter Zulehner \\
Johannes Kepler University, A-4040 Linz, Austria \\
Joachim Schoeberl
\end{center}


Iterative methods which keep some problem--dependent subspace
invariant have been proven to be appropriate smoothers for multigrid
methods.  The Braess-Sarazin smoother for saddle point problems
belongs to this class, where the invariant subspace is the kernel
of the operator describing the constraints.  Performing one step
of this iterative method, however, requires the accurate solution
of some global linear system for the dual variables.  Another
iterative method with invariant subspace has been proposed by
Joachim Schoeberl for parameter dependent elliptic problems and
the limiting saddle point problem.

If this smoother is accompanied
with some specific (kernel-preserving) prolongation robust multigrid
convergence results were shown by Schoeberl.  The smoother as well
as the prolongation can be realized by solving a number of local
problems.  In this talk it will be shown that multigrid convergence
can also be achieved for standard prolongation and local
kernel--preserving smoothers for saddle point problems.



\end{document}
