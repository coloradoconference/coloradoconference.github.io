\documentclass{report}
\usepackage{amsmath,amssymb}
\setlength{\parindent}{0mm}
\setlength{\parskip}{1em}
\begin{document}
\begin{center}
\rule{6in}{1pt} \
{\large Konstantin Lipnikov \\
{\bf Error Minimization Based Rezone Strategy for ALE Methods}}

Los Alamos National Laboratory \\ T-7 \\ MS B284 \\ Los Alamos \\ NM 87545
\\
{\tt lipnikov@lanl.gov}\\
Mikhail Shashkov\end{center}

The philosophy of the Arbitrary Lagrangian-Eulerian (ALE)
methodology for solving multidimensional fluid flow problems
is to move the computational mesh, using the flow as a guide,
to improve the robustness, accuracy and efficiency of the simulation. The
main elements in the ALE simulation are an
explicit Lagrangian phase, a rezone phase in which a new mesh
is defined, and a remap phase in which a Lagrangian solution
is transferred to the new mesh. In the talk we shall address
different computational aspects of the rezone phase.

In most ALE codes, the main goal of the rezone phase is to
maintain geometrical quality of the mesh. The example of such
a strategy is the Reference Jacobian Matrix (RJM) method.
Recently we have developed a new Error-Minimization-Based (EMB)
rezone strategy which minimizes a global error and maintains
smoothness of the mesh. The global error on new time level
$t=t^{n+1}$ is a superposition of an interpolation error, an
error due to the time advancing method, and the space
discretization error. It is clear that if the mesh is not rezoned,
then the interpolation error is zero; however, the mesh at time
level $t=t^n$ may not be the best mesh to represent solution features at
time $t=t^{n+1}$. The goal of the EMB rezone strategy is to achieve
balance (if possible) between the errors which will minimize the global
error at time $t=t^{n+1}$. The numerical experiments demonstrate the
superiority of the EMB rezone strategy over the RJM rezone strategy.

The rezoned mesh is sought as a minimizer of a non-convex functional. The
essential part of the talk will be devoted to comparison of efficiency of
different non-linear optimization strategies for hyperbolic problems. In
particular, we shall consider a few global methods (e.g., a Polak-Ribiere
nonlinear conjugate gradient method with inexact line search, a truncated
Newton method) and a few local Gauss-Seidel type methods. We shall
present numerical results for 1D Burgers' equation and 1D system of gas
dynamics equations. The numerical experiments show that the rezone
methods perform well even if the rezoned mesh is a local
minimum of the functional. It allow us to use inexpensive optimization methods.

Generalization of the EMB rezone strategy for two and three dimensions is
a challenging task. It will be addressed quickly in the talk.


\end{document}
