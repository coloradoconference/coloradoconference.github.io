\documentclass{report}
\usepackage{amsmath,amssymb}
\setlength{\parindent}{0mm}
\setlength{\parskip}{1em}
\begin{document}
\begin{center}
\rule{6in}{1pt} \
{\large Brett Bader \\
{\bf A Comparison of Iterative Tensor Methods for Solving Large Systems of Nonlinear Equations}}

Sandia National Laboratories \\ P.O. Box 5800 \\ Albuquerque \\ NM  87185-1110
\\
{\tt bwbader@sandia.gov}\end{center}

We compare and contrast a variety of iterative tensor methods for solving
large-scale systems of nonlinear equations. Tensor methods are an
alternative to Newton-based methods and are based on using a limited
quadratic local model rather than a linear model. These higher order
models often provide information that is lacking in a (nearly) singular
Jacobian, thus making the solver more efficient when solving difficult
problems.

This talk has three areas of emphasis. First, we introduce several new
and recent iterative tensor methods for solving large-scale problems,
including tensor-Krylov methods and a new implementation that can use a
stand-alone linear solver. The methods have a range of characteristics
and other considerations for a practical implementation, which we
discuss. Second, we apply a curvilinear linesearch globalization
technique to the tensor methods that smoothly combines the Newton and
tensor directions. Our results show that the curvilinear linesearch is
more robust and efficient than other linesearch implementations. Finally,
we explore the performance of these large-scale tensor methods in
comparison to Newton-GMRES on several realistic problems, including some
Navier-Stokes fluid flow problems. All methods are implemented in an
object-oriented nonlinear software package called NOX that is being
developed at Sandia National Laboratories. Our results show that the
iterative tensor methods have computational advantages over Newton-GMRES,
especially when the Jacobian at the root is ill-conditioned or singular.


\end{document}
