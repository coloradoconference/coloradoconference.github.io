\documentclass{report}
\usepackage{amsmath,amssymb}
\setlength{\parindent}{0mm}
\setlength{\parskip}{1em}
\begin{document}
\begin{center}
\rule{6in}{1pt} \
{\large Amik St-Cyr \\
{\bf On Optimized Schwarz Preconditioning for High-Order Spectral Element Methods}}

National Center for Atmospheric Research \\  \\ 1850 Table Mesa Drive \\  \\ Boulder \\ CO 80305.
\\
{\tt amik@ucar.edu}\\
Martin. J. Gander\\
Stephen J. Thomas\end{center}

\begin{document}
\title{On Optimized Schwarz Preconditioning for High-Order
Spectral Element Methods}
\author{
A.~St-Cyr\footnote{{\bf amik@ucar.edu},
NCAR 1850 Table Mesa Drive Boulder, CO 80305, USA.}, \;\;
M.J. Gander\footnote{{\bf mgander@math.mcgill.ca},
McGill 805 Sherbrooke W. Montreal QC,Canada H3A2K6} \; and
S.~J.~Thomas\footnote{{\bf thomas@ucar.edu},
NCAR 1850 Table Mesa Drive Boulder, CO 80305, USA.}}
\maketitle
\begin{abstract}
Optimized Schwarz preconditioning is applied to a spectral element method
for the modified Helmholtz equation and pseudo-Laplacian arising in
incompressible flow solvers. The preconditioning is performed on an
element-by-element basis. The method enables one to use non-overlapping
elements, yielding an effective algorithm in terms of communication
between elements and implementation. Two approaches are tested. The first
consists of constructing a $P_1$ finite element problem on each
overlapping element. In the second, the preconditioner is applied
directly on a non-overlapping spectral element. Numerical results demonstrate
an improvement in the iteration count over the classical Schwarz algorithm.
\end{abstract}
\end{document}


\end{document}
