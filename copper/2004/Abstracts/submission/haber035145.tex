\documentclass{report}
\usepackage{amsmath,amssymb}
\setlength{\parindent}{0mm}
\setlength{\parskip}{1em}
\begin{document}
\begin{center}
\rule{6in}{1pt} \
{\large Eldad Haber \\
{\bf Preconditioners for volume preserving image registration}}

400 Dowman Dr. \\ Mathcs dept. \\ Atlanta GA \\ 30033
\\
{\tt haber@mathcs.emory.edu}\\
Jan Modersinski\end{center}

We consider the registration of two images, $R(x)$ and $T(x)$, that is,
our goal is to find a reasonable transformation $u$ such that
\begin{eqnarray}
\label{disteq}
\cD(\uu; T,R) = \hf \| T(x + u) - R(x) \|^2
\end{eqnarray}
is small.

The problem is ill-posed and therefore, in order to obtain a reasonable
transformation we need to integrate prior information about the problem.
Some of this information is in the form of the smoothness of the field; such
information can be integrated into the regularization of the problem.
Thus it is common to find the transformation $u$ by solving
the following optimization problem
\begin{eqnarray}
\label{opt1}
\hf \|T(u)-R\|^2 + \alpha S(u)\ =\ {\rm min}
\end{eqnarray}
where $S(u)$ is usually a quadratic regularization
operator and $\alpha$ is a regularization
parameter. For example in elastic registration
$$ S(u) = \int\ (\lambda ({\bf curl} u)^2
+ \mu ({\bf div} u)^2)\ dV $$
but many other differential operators can be chosen.

In this work we also assume that the transformation is Volume Preserving (VP).
This makes sense in cases when objects may deform but we want
to have the same overall volume.

From a mathematical point of view this implies that the transformation also satisfies
\begin{eqnarray}
\label{con1}
C(u) = {\rm det} (1+ \grad u) -1\, =\, {\rm div} u + N(u) \, =\, 0
\end{eqnarray}
where $N(u)$ is some nonlinear operator
with terms which depend on the
derivatives of $u$.
Thus the optimization problem \eqref{opt1} becomes an equality
constraint optimization problem of the form
\begin{subequations}
\begin{eqnarray}
\label{opt11}
{\rm min} && \hf \|T(u)-R\|^2 + \alpha S(u)\\
\label{st}
{\rm s.t} && {\rm div} u + N(u)= 0
\end{eqnarray}
\label{copt}
\end{subequation}


In this talk we will discuss the KKT systems which evolve from this
problem and their numerical solutions. We will show how to use
effective multigrid preconditioners for the solution of these systems.


\end{document}
