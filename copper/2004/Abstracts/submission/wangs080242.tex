\documentclass{report}
\usepackage{amsmath,amssymb}
\setlength{\parindent}{0mm}
\setlength{\parskip}{1em}
\begin{document}
\begin{center}
\rule{6in}{1pt} \
{\large Shun Wang \\
{\bf Approximate Inverse Preconditioners for Simulations with Adaptive Mesh Refinement}}

4333 Siebel Center \\ 201 N. Goodwin \\ Urbana,IL,61801-2302
\\
{\tt shunwang@uiuc.edu}\\
Eric de Sturler\end{center}

\begin{center}
Approximate Inverse Preconditioners for Simulations with Adaptive Mesh Refinement\\

\bigskip

Shun Wang and Eric de Sturler\\
shunwang@uiuc.edu, sturler@uiuc.edu\\
University of Illinois at Urbana-Champaign\\
\end{center}
\bigskip

The Interoperability Based Environment for Adaptive Meshes (IBEAM) is a
NASA funded object-oriented framework for astrophysical simulations on
high-performance, distributed memory, parallel computing platforms. In
this project, we aim to solve radiation-hydrodynamic models of Gamma-Ray
bursts. As such models require a high variation in the resolution of the
computational grid, we use the PARAMESH package to support adaptive mesh
refinement (AMR) on parallel machines. The PARAMESH package is developed
by the Computational Technologies
Team of NASA Goddard Space Flight Center.\\

PARAMESH was originally designed for explicit finite difference
methods. A drawback for our problems is that excessively small
time steps may be required for stability. To avoid this problem,
we are implementing implicit methods on the PARAMESH package.
These methods demand efficient solvers for large sparse linear
systems. PARAMESH represents the computational grid by a list of
many small grid blocks distributed for load balancing and
minimizing communication. It carries out grid refinements and
unrefinements at every time step. So, the system of equations
changes (possibly) every time step. This makes certain
preconditioners, like ILU, hard to realize in AMR. In addition,
every preconditioner that requires a significant amount of
preprocessing might become very expensive. Moreover, PARAMESH
redistributes the grid blocks among the processors, possibly at
every time step. The user does not have much control over this
distribution. This makes preconditioners, like domain
decomposition type preconditioners, less suitable for PARAMESH. A good
preconditioner for AMR discretizations should be easy to
compute and apply for a dynamically (re)distributed collection of grid
blocks, and should easily accommodate frequent changes in the grids.
Therefore, it should be easy to update when the mesh is changed. For all
these reasons, explicit approximate inverse preconditioners turn out to
be a good candidate for AMR based iterative solvers. In addition,
explicit approximate inverse preconditioners are easy to update to
accommodate small changes of the matrix due to adaptive time steps and
nonlinear PDEs.\\

As PARAMESH applies refinements and unrefinements to the grid the matrix
changes. The matrix may also change due to nonlinearities in the PDEs and
possibly changes in time step. Frequently regenerating the approximate
inverse from scratch would greatly increase the cost of computation.
Therefore, we consider a number of techniques to limit these costs,
mainly techniques to adapt an existing preconditioner for small changes
in the matrix. The first one is to generate the approximate inverse only
for newly refined blocks in every time step and keep the approximate
inverse on the corresponding coarse blocks to be used after possible
future unrefinements. The second one is to construct the approximate
inverse for newly refined blocks cheaply from the existing coarse grid
approximate inverse rather than generating it from scratch using, for
example, the standard Frobenius norm minimization. We will discuss
several other techniques
for cheap adaptation.\\

Another important issue is the effectiveness in reducing the number of
iterations. To be effective, the approximate inverse preconditioner needs
to incorporate global
information. We will use multigrid type techniques to incorporate global
information in the preconditioner in a more efficient way than using
Frobenius norm minimization.
Though a multigrid solver itself may not work well for
certain problems, see the presentation by Zhen Cheng (UIUC),
multigrid-like techniques are well-suited for the PARAMESH data
structure.\\

There is a trade-off between effectiveness and efficiency of
approximate inverse preconditioners. The more accurate the
approximate inverse is, the more effective the preconditioner will
be, but also the more it will cost to compute the approximate
inverse and apply it every iteration. There will be an optimal
point. We will analyze this trade-off with experimental
results.\\

At this point, we do not have the real model yet. So, our
experiments are based on convection-diffusion type model problems. Our
results indicate that approximate inverse preconditioners generally work
well compared with other preconditioners built on
PARAMESH, especially when the system is ill-conditioned. To
provide a good benchmark for convergence, we will also compare the
convergence of explicit approximate inverse preconditioners with
other standard
preconditioners, including some that are not easy to use in PARAMESH.\\


(The IBEAM project is sponsored under a Round III Grand Challenge
Cooperative Agreement with NASA's Computational Technologies
Project.)\\


\end{document}
