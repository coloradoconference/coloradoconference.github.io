\documentclass{report}
\usepackage{amsmath,amssymb}
\setlength{\parindent}{0mm}
\setlength{\parskip}{1em}
\begin{document}
\begin{center}
\rule{6in}{1pt} \
{\large T. K. Nilssen \\
{\bf On the Reuse of Standard Preconditioners for Higher Order Time Discretizations of Parabolic PDEs}}

Simula Research Laboratory \\ Martin Linges vei 17 \\ Fornebu \\ P.O. Box 134 \\ N-1325 Lysaker \\ Norway
\\
{\tt trygvekn@simula.no}\\
K.-A. Mardal\end{center}

\newtheorem{lemma}{Lemma}

In this abstract we will describe a preconditioner for some higher order
time discretizations of parabolic problems.
The preconditioner is optimal with respect to the spatial
discretization parameters, that typically are the characteristic mesh
size parameter $h$ and the polynomial degree $p$.
The preconditioner is also order optimal with respect to $\Delta t$. The
only assumption is that there exists a preconditioner for the low order
time discretization schemes such as Crank-Nicholson or implicit Euler.
Such preconditioners are standard, c.f. e.g., \cite{BD},
\cite{olshanskii00convergence} and \cite{Thomee}.

We study the model problem
\begin{eqnarray*}
\frac{\partial u}{\partial t} &=& \Delta u, \quad \mbox{ in } \Omega, t > 0 \\
u &=& 0, \quad \mbox{ on } \partial \Omega, t>0 \\
u &=& u_0, \quad \mbox{ in } \Omega, t=0.
\end{eqnarray*}

This equation is discretized in space and time to give the
following linear system to be solved for each time level
\begin{equation*}
Q_{kj} (\Delta t A) u^n = P_{kj} (\Delta t A) u^{n-1},
\end{equation*}
where $\Delta t $ is the time stepping parameter, the two polynomials
$Q_{kj}$ and $P_{kj}$ are the $(k,j)-$ Pad\'{e} approximation to the
exponential function and $A$ is a discrete Laplacian.
The polynomials are given by (c.f. \cite{Thomee}):
\begin{eqnarray*}
P_{kj}(\Delta t A )&=&
\sum_{i=0}^k \left( \begin{array}{c c}
k \\
i
\end{array} \right) \frac {(k+j-i)!}{(k+j)!} (\Delta t A)^i \\
Q_{kj}(\Delta t A )&=&P_{jk}(-\Delta t A ).
\end{eqnarray*}

As mentioned earlier, we will reuse preconditioners for
\[
R_c = (I - c \Delta t A).
\]
In fact, the proposed (exact) preconditioner is on the form,
\[
R^{-j}_c = (I - c \Delta t A)^{-j},
\]
where $c$ is determined such that the highest order term of $R^{-j}_c(\Delta t A)$
equals the highest order term of $Q_{kj} (\Delta t A)$.
This can be done by choosing
\begin{eqnarray*}
R_{kj} (\Delta t A) =\left( I- \sqrt[j]{\frac {j!}{(j+k)!}}\Delta t A \right)^j.
\end{eqnarray*}

Hence, $R_{kj}$ is a standard preconditioner for a low
order time discretization of a parabolic PDE, used
$j$ times.

In Table \ref{cond} we show an upper bound on the condition
number of the preconditioned system, using an exact preconditioner
for $R_{kj}$.
More details about the preconditioner and
the estimation of the condition number can be found in
the full paper that is also submitted.
Further we proove the following lemma.

\begin{lemma}\label{lem1}
The polynomials $R_{kj}(\Delta t A)$ and $Q_{kj}(\Delta t A)$ are spectrally
equivalent independent of $A$ and $\Delta t$, and the condition number of
the preconditioned system is bounded by
\begin{equation*}
\kappa \left( \left( R_{kj}(\Delta t A) \right)^{-1} Q_{kj}(\Delta t
A) \right) < 1.8 \cdot 1.09^j.
\end{equation*}
\end{lemma}

\begin{table}
\begin{center}
\begin{tabular}{|c|c|c|c|c|c|}
\hline
$j\backslash k$& $j$ & $j-1$ & $j-2$ \\ \hline
$2$ & 1.07 & 1.10 & 1.17 \\ \hline
$4$ & 1.26 & 1.31 & 1.40 \\ \hline
$6$ & 1.49 & 1.56 & 1.66 \\ \hline
$8$ & 1.76 & 1.85 & 1.97 \\ \hline
$10$ & 2.08 & 2.20 & 2.34 \\ \hline
\end{tabular}
\end{center}
\caption{Upper bound on the condition number for various values of $j$ and $k$.}
\label{cond}
\end{table}

\begin{thebibliography}{99}
\bibitem{BD} R. E. Bank and T. Dupont, An optimal order process for
solving finite element equations, Math. Comp., Vol. 36, pp. 35--51, 1981.
\bibitem{olshanskii00convergence} Maxim A. Olshanskii and Arnold
Reusken, On the Convergence of a Multigrid Method for Linear
Reaction-Diffusion Problems, Computing, Vol. 65(3), pp. 193--202, 2000.
\bibitem{Thomee} V. Thom\'{e}e, Galerkin Finite Element Methods for
Parabolic Problems, Springer-Verlag, 2nd edition, 1997.
\end{thebibliography}


\end{document}
