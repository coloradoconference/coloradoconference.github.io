\documentclass{report}
\usepackage{amsmath,amssymb}
\setlength{\parindent}{0mm}
\setlength{\parskip}{1em}
\begin{document}
\begin{center}
\rule{6in}{1pt} \
{\large Yogi A. Erlangga \\
{\bf On a complex Shifted-Laplace preconditioner for high wavenumber heterogeneous Helmholtz problems}}

Department of Applied Mathematical Analysis \\ Delft University of Technology \\ Mekelweg 4 \\ 2628 CD Delft \\ The Netherlands
\\
{\tt y.a.erlangga@math.tudelft.nl}\\
C. Vuik\\
C. W. Oosterlee\end{center}

In this presentation, we are concerned with the numerical solution of the
boundary value problem
\begin{eqnarray}
\mathcal{L} u \equiv \left( \partial_{xx} + \partial_{yy} + k^2(x,y) \right) u
&=& f \, \, \, \text{in} \, \, \, \Omega \in \mathbb{R}^2, \label{eq1}\\
\lim_{r \rightarrow \infty} \sqrt{r} \left(\frac{\partial u}{\partial n}
- i k u \right) &=& 0 \, \, \, \text{on} \, \, \, \Gamma = \partial \Omega. \label{eq2}
\end{eqnarray}
This Helmholtz problem arises in many applications, e.g. acoustics,
electromagnetics and geophysics. Here, we consider problems which mimic
geophysical applications with inhomogeneous material properties and high
wavenumbers $k$. Furthermore, we develop solution methods based on the
iterants in the Krylov subspace.

For high wavenumbers, the matrix $A$ obtained from any discretization of
(\ref{eq1}) and (\ref{eq2}) has the property of indefiniteness, with the
spectrum is distributed in the left and right complex half plane.
Furthermore, the condition number is extremely large. These two
properties are not favorable for Krylov subspace iterative methods.
Applying Krylov methods on the normal equations representation, $A^*A$,
does not provide a practical remedy since, even though the linear system
can be made definite, the condition number is even worse. Another remedy
is efficient preconditioning.

In {\it [Erlangga et. al., TU Delft-AMA Report 03-01]}, we propose a
class of preconditioners, called the Shifted-Laplace preconditioners, for
the Helmholtz problem. The preconditioners are constructed based on an
operator
\begin{eqnarray}
\mathcal{M} \equiv \partial_{xx} + \partial_{yy} - \alpha k^2(x,y), \label{eq3}
\end{eqnarray}
where $\alpha \in \mathbb{C}$. We have previously found that $\alpha =
i$, $(i^2 = -1)$ gives very satisfactorily convergence within this class
of preconditioners. We call the preconditioner (\ref{eq3}) with $\alpha =
i$ the Complex Shifted-Laplace preconditioner (CSL). We can show also
that (i) the spectrum of $M^{-1}A$ is bounded above by one, and (ii) the
lower bound of the spectrum of $M^{-1}A$ is of $\mathcal{O}(1/k^2)$. From
the latter insight, we may expect that the convergence rate for
increasing $k$ is mainly determined by the smallest eigenvalue. We have
used the preconditioner in combination with BiCGSTAB. A significant
reduction in the number of iterations is observed.

In this talk, we discuss another issue in relation with the
preconditioner solves. (Previously, we solved the preconditioner exactly
using direct methods. This process is very costly.) Since the
preconditioning matrix $M$ is complex, symmetric positive definite
(CSPD), several efficient iteration methods can be implemented to
approximate $M^{-1}$. We study the use of an incomplete LU decomposition
of $M$ and multigrid as the preconditioner solver. For ILU factorizations
for $M$, one level of fill-in is allowed. For constructing the LU
factors, an algorithm based on regular stencil is used. The algorithm is
not only fast in computing the LU factors (so it reduces initialization
cost) but also requires a small amount of storage.

For multigrid, we implement a geometric multigrid algorithm as discussed
in {\it [Oosterlee et. al., SIAM J. Sci. Comput. 19(1) (1998),
pp.87--110]}. It is originally developed for real-valued matrices for
structured grid applications. For our complex-valued applications, the
method needs not be modified.

We compare the number of iterations and the time to convergence for
various wavenumbers $k$. The conclusions are as follows:
\begin{itemize}
\item[(1)] ILU of $M$ results in better performance than ILU directly
applied to $A$. The improvement becomes more significant as $k$
increases.
\item[(2)] One V(1,1) multigrid iteration is sufficient to further
improve the computational performance. In comparison with the
unpreconditioned and ILU- preconditioned case, the computational time is
reduced almost by factor 10 and 3 for sufficiently large $k$,
respectively. Furthermore, the number of iterations are reduced by factor
of 50 and 8, respectively.
\item[(3)] Preconditioner solves using multigrid seem to be less
sensitive to the inhomogeneity of the media. In comparison with constant
wavenumbers, only less than 10 \% increase in the number of iterations is
observed with multigrid.
\end{itemize}


\end{document}
