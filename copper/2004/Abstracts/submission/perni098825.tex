\documentclass{report}
\usepackage{amsmath,amssymb}
\setlength{\parindent}{0mm}
\setlength{\parskip}{1em}
\begin{document}
\begin{center}
\rule{6in}{1pt} \
{\large Michael Pernice \\
{\bf Performance of a Newton-Krylov-FAC Method for Equilibrium Radiation Diffusion on Locally Refined Grids}}

Los Alamos National Laboratory \\ P O Box 1663 \\ MS B265 \\ Los Alamos \\ NM 87544
\\
{\tt pernice@lanl.gov}\end{center}

Radiation transport plays an important role in numerous fields of study,
including astrophysics, laser fusion, and combustion
applications such as modeling of coal-fired power generation systems and
wildfire spread. A diffusion approximation provides a reasonably accurate
description of penetration of radiation from a hot source to a cold
medium. This approximation often features a nonlinear conduction
coefficient that leads to formation of a sharply defined thermal front,
or Marshak wave, in which the solution can vary several orders of
magnitude over a very short distance.

Classic solution techniques use a linearized conduction coefficient. This
introduces a first order error in time, and requires small time steps to
manage the size of this error. Numerous recent studies have shown that
fully implicit time integration methods provide more accurate predictions
of the position of the thermal front, even when using large time steps.
Fully implicit approaches require the solution of a large-scale system of
nonlinear equations at each time step. Newton-Krylov methods, usually
preconditioned by a multigrid method, have been instrumental in
demonstrating that a fully implicit approach is practical.

The shape of the thermal front can be very complex as it interacts with
different materials having different conduction properties. Resolving
these localized features with a global fine mesh can be prohibitively
expensive. Adaptive mesh refinement (AMR) concentrates computational
effort by increasing spatial resolution only locally, and a properly
designed method is capable of greatly reducing the computational cost
needed to achieve a desired accuracy. AMR can readily be incorporated
into the Newton-Krylov solution framework, by properly accounting for the
presence of local refinement in the preconditioner. In particular, the
Fast Adaptive Composite grid (FAC) method of McCormick and Thomas is well
suited for this purpose.

We report on efforts to solve equilibrium radiation diffusion problems
using structured AMR and the Newton-Krylov-FAC method. While structured
AMR facilitates reuse of existing software written for logically
rectangular grids, discretization at locations near changes in resolution
must be treated carefully in order to avoid the creation of artificial
sources. We describe our FAC solver and report on its performance, both
as a standalone solver and as a preconditioner within Newton-Krylov
iterations.



\end{document}
