\documentclass{report}
\usepackage{amsmath,amssymb}
\setlength{\parindent}{0mm}
\setlength{\parskip}{1em}
\begin{document}
\begin{center}
\rule{6in}{1pt} \
{\large Roger P. Pawlowski \\
{\bf Desiging a Flexible Nonlinear Solver Library.}}

Sandia National Laboratories \\ PO Box 5800 \\ MS-0316 \\ Albuquerque \\ NM 87185-0316
\\
{\tt rppawlo@sandia.gov}\\
Tamara G. Kolda\\
Russell Hooper\\
	John N. Shadid\end{center}

Designing a nonlinear solver library with the flexiblity to supportmultiple applications with robust,
cutting-edge algorithms presents anumber of difficulties.
Not only must the library supply a largenumber of solution algorithms,
but it must also account for the manyapplication specific optimizations that production-level codesrequire.
This presentation will discuss the design of NOX,
anobject-oriented C++ library being developed to support applications atSandia National Laboratories.
We will discuss the design requirementsthat the nonlinear solver library must meet and how NOX handles thoseissues.
Specifics we plan to address include: \\1.
Desiging a flexible environment to allow for the efficientintroduction of new algorithms.
\\2.
Defining a flexible methodology for convergence and failurecriteria of the nonlinear algorithms (and allowing users to supplytheir own criteria).
\\3.
Abstracting the linear solvers and the linear algebra storageformat from the nonlinear algorithm.
This allows applications to usetheir own specialized algorithms for optimal performance.
\\4.
Allowing users to pass objects/arguments through the code efficiently.
\\5.
Library portability.
\\Examples will be presented from large-scale engineering applicationsdeveloped at Sandia National Laboratories including reacting flows(MPSalsa),
compressible flows (Premo),
and electrical circuit modeling(Xyce).
Comparisons of the nonlinear globalization algorithms willalso be briefly discussed.

\end{document}
