\documentclass{report}
\usepackage{amsmath,amssymb}
\setlength{\parindent}{0mm}
\setlength{\parskip}{1em}
\begin{document}
\begin{center}
\rule{6in}{1pt} \
{\large Robert Falgout \\
{\bf Coarse Grid Selection via Compatible Relaxation}}

Lawrence Livermore National Laboratory \\ P.O. Box 808 \\ L-561 \\ Livermore \\ CA  94551
\\
{\tt rfalgout@llnl.gov}\\
James Brannick\end{center}

The notion of {\em compatible relaxation} (CR) was introduced by
Brandt in \cite{ABrandt_2000} as a modified relaxation scheme that
keeps the coarse-level variables invariant. Brandt states that the
convergence rate of CR is a general measure for the quality of the set
of coarse variables. In \cite{RDFalgout_PSVassilevski_2003}, a
supporting theory for these ideas was introduced, and an outline for
an algebraic coarsening algorithm was also described.

In this talk, we will describe recent progress developing CR-based
coarsening algorithms for algebraic multigrid. In particular, we will
discuss some of the motivations (theoretical and heuristic) behind our
algorithm, point out its current strengths and weaknesses, and discuss
open questions and future directions.

This work was performed under the auspices of the U.S. Department of
Energy by University of California Lawrence Livermore National
Laboratory under contract No. W-7405-Eng-48.

\begin{thebibliography}{1}

\bibitem{ABrandt_2000}
{\sc A.~Brandt}, {\em General highly accurate algebraic coarsening},
Electronic Transactions on Numerical Analysis, 10 (2000), pp.~1--20.

\bibitem{RDFalgout_PSVassilevski_2003}
{\sc R.~D.~Falgout and P.~S.~Vassilevski}, {\em On Generalizing the
{AMG} Framework}, submitted. Also available as LLNL technical report
UCRL-JC-150807, 2003.

\end{thebibliography}


\end{document}
