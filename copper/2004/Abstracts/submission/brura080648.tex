\documentclass{report}
\usepackage{amsmath,amssymb}
\setlength{\parindent}{0mm}
\setlength{\parskip}{1em}
\begin{document}
\begin{center}
\rule{6in}{1pt} \
{\large Rafael Bru \\
{\bf Convergence of Additive Schwarz Iterations for Singular Systems}}

Instituto de Matematica Multidisciplinar \\ Dept. Matematica Aplicada \\ Univ. Politecnica de Valencia \\ 46022 Valencia \\ Spain
\\
{\tt rbru@mat.upv.es}\\
Francisco Pedroche\\
Daniel B Szyld\end{center}

A convergence analysis is presented for additive Schwarz iterations
when applied to consistent singular systems of equations $Ax=b$.
The theory applies to singular $M$-matrices with one-dimensional
null space, and is applicable in particular to systems representing
ergodic Markov chains.
The results are based on an algebraic formulation of Schwarz methods.
Let $R_i$ be the restriction operator so that $A_i = R_i A R_i^T$
is a symmetric permutation of a principal submatrix of $A$
(and thus nonsingular).
Given an initial vector $x^0$,
additive Schwarz iterations consist of the process
$x^{k+1}=T_{\theta}x^k+c$, $ k=0,1,\ldots ,$
where
$$% \begin{equation}
T_{\theta} = I - \theta \sum_{i=1}^{p} R_i^{T} A_i^{-1} R_i A,
$$% \end{equation}
$c = \theta \sum_{i=1}^{p} R_i^{T} A_i^{-1} R_i b$,
and $ 0 < \theta < 1$ is a damping parameter.
The key here is of course that one assumes that there is
overlap, i.e., that the restriction operators restrict to
subspaces with nontrivial intersection.
If there is no overlap, the method reduces to block Jacobi.

As $A$ is singular, one has $\rho(T_\theta)=1$.
It is shown that if $\theta < 1/p$ (or more generally
$\theta < 1/q$, with $q$ the measure of the overlap), then
$\gamma(T_\theta) =
max\{ | \lambda |, \lambda \in \sigma(T), \lambda \neq 1 \} < 1 $,
and that the index $ind(T)=1$.
Therefore
additive Schwarz iterations converge for any initial vector.
Furthermore, there exists a splitting $A=M-N$ such that
$M^{-1}N = T_\theta$.
Our results imply in particular that
zero is an isolated point of
the spectrum of the
preconditioned matrix $\sigma(M^{-1}A)$
and
the rest is inscribed in a circle
centered at one with radius $\gamma = \gamma(T_\theta) <1$.

This work complements the results of [Marek and Szyld,
{\em LAA}, in press], where multiplicative Schwarz iterations
are shown to converge for singular systems.

(joint work with Francisco Pedroche, Univ. Politecnica de Valencia,
Spain, and Daniel B Szyld, Temple University, Philadelphia)


\end{document}
