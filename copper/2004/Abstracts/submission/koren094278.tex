\documentclass{report}
\usepackage{amsmath,amssymb}
\setlength{\parindent}{0mm}
\setlength{\parskip}{1em}
\begin{document}
\begin{center}
\rule{6in}{1pt} \
{\large Yair Koren \\
{\bf Adaptive Multiscale Redistribution for Vector Quantization}}

P O Box 55405 \\ Haifa 34981 \\ Israel
\\
{\tt yair_k@cs.technion.ac.il}\\
Irad Yavneh\end{center}

Vector quantization is a classical problem which appears in many
fields. Unfortunately, the quantization problem is generally
non-convex, and therefore affords many local minima. The main
problem is finding an initial approximation, which is close to a
``good'' local minimum. Once such an approximation is found, the
Lloyd-Max method may be used to reach a local minimum near it. In
recent years, much improvement has been made with respect to
reducing computational costs of quantization algorithms, while the
task of finding better initial approximations received somewhat
less attention.

We present a novel multiscale iterative scheme for
the quantization problem. The scheme is based on redistributing
the representation levels among aggregates of decision regions at
changing scales. The rule governing the redistribution relies on
the so called point density function and the number of
representation levels in each aggregate. Our method focuses on
achieving better local minima than other contemporary methods such
as LBG. When quantizing signals with sparse and patchy histograms,
as may occur in color images for example, the improvement in
distortion relative to LBG may be arbitrarily large.


\end{document}
