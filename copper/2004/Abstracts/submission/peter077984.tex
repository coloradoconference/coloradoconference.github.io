\documentclass{report}
\usepackage{amsmath,amssymb}
\setlength{\parindent}{0mm}
\setlength{\parskip}{1em}
\begin{document}
\begin{center}
\rule{6in}{1pt} \
{\large Christer Peterson \\
{\bf Time-accurate solution of the Schr�dinger equation}}

Uppsala University \\ Department of Information Technology \\ Box 337 \\ SE-751 05 Uppsala \\ Sweden
\\
{\tt christer.peterson@it.uu.se}\\
Sverker Holmgren\end{center}

We consider iterative methods for solution of the time-dependent
Schr\"odinger equation. The aim is to perform accurate computations of
the dynamic behavior of small molecular systems, describing fundamental
chemical reactions occurring on a femtosecond time-scale. Accurate
predictions of such reactions allows for a deeper understanding of the
fundamentals of chemistry and complements modern experimental techniques.
For this type of computations, the Born-Oppenheimer approximation is
first applied for separating the problems for the nucleii and electrons
in the molecular system. We then solve the Schr\"odinger equation for the
wave functions representing the nucleii for $l$ interacting electronic
states. For example, for a two-state system, we solve
\[
i\frac{\partial}{\partial t}
\begin{pmatrix}
\psi_{1} \\
\psi_{2} \\
\end{pmatrix}
=
\begin{pmatrix}
\mathcal{K}_{1}+\mathcal{V}_{1} & \mathcal{C}_{12}^* \\
\mathcal{C}_{12} & \mathcal{K}_{2}+\mathcal{V}_{2} \\
\end{pmatrix}
\begin{pmatrix}
\psi_{1} \\
\psi_{2} \\
\end{pmatrix}.
\]
Here, the kinetic energy operators are given by
\[ \mathcal{K}_{1,2}=-\frac{1}{2m}\nabla^2, \] where $m$ is
corresponds to a mass. We assume that the time-independent diagonal
potential energy operators $\mathcal{V}_{1,2}$ and the
diagonal,time-dependent coupling operator $\mathcal{C}_{12}$ are given.
In practice, these operators are described by analytic models or, in the
case of $\mathcal{V}_{1,2}$, computed by solving the time-independent
Schr\"odinger equation for the electrons. In principle, the number of
spatial dimensions grows as $3^p$, where $p$ is the number of nucleii in
the molecular system. However, the number of dimensions is in practice
normally reduced by introducing approximations and specific
choices of coordinate systems. Still, to be able to consider new and
interesting problem settings in quantum chemistry, the solution of
high-dimensional PDE problems is required.


In applications, a standard time-integration method for the
Schr\"odinger equation is based on operator splitting
\cite{Strang}\cite{Feit_Flech_Steiger}. In the original original form of
this scheme, a pseudo-spectral spatial discretization is used while the
discretization in time is only second-order accurate. However, an
important feature of the operator splitting method is that it preserves
probability, which in many cases reduces the error from the from the
time marching. We examine several alternative methods for the
time-integration of the Schr\"odinger equation, which are all
probability preserving. Using the trapezoidal method in time implies that
some standard iterative solver has to be applied to the arising system of
equations. An interesting alternative to standard implicit
time-discretizations is to use the Lanczos method directly to approximate
the time evolution operator $\exp(-i\hbar Ht)$ \cite{Park_Light}. The
formal order of accuracy of such a scheme is determined by the number of
Lanczos steps, and the method preserves probability by construction. For
a non-symmetric Hamiltonian matrix, the Arnoldi method is used instead.
Another class of schemes which is considered is partitioned runge-kutta,
PRK, methods \cite{Gray_Manolopoulos}\cite{Blanes_Casas_Ros}. It is
possible to construct high-order PRK methods which have the desired
conservation property. For all these time-marching methods we use a
pseudo-spectral spatial discretization and compare the accuracy and
computational work to the standard operator splitting method. We also
present some experiments where we combine the time-integration schemes
with high-order finite difference discretizations in space. The goal here
is to employ an adaptive spatial discretization, concentrating the grid
points to the regions where the wave function is located.

\begin{thebibliography}{99}
\bibitem{Strang} G. Strang, \emph{On the construction and comparison
of difference schemes}, SIAM Journal of Numerical Analysis 5
(1968), pp. 506-517.
\bibitem{Feit_Flech_Steiger} M.D. Feit, J.A. Flech, Jr., and A.
Steiger, \emph{Solution of the Schr\"odinger Equation by a Spectral
Method}, Journal of Computational Physics 47 (1982), pp. 412-433.
\bibitem{Park_Light} T.J. Park and J.C. Light, \emph{Unitary quantum
time evolution by iterative Lanczos reduction}, Journal of
Chemical Physics 85 (1986), pp. 5870-5876.
\bibitem{Gray_Manolopoulos} S. Grey and D.E. Manolopoulos,
\emph{Symplectic integrators tailored to the time-dependent
Schr\"odinger equation}, Journal of Chemical Physics 104 (1966),
pp. 7099-7112.
\bibitem{Blanes_Casas_Ros} S. Blanes, F. Casas, and J. Ros, \emph{High
order optimized geometric integrators for linear differential
equations}, DAMTP technical report 2000/NA07, University of
Cambridge (2000).
\end{thebibliography}


\end{document}
