\documentclass{report}
\usepackage{amsmath,amssymb}
\setlength{\parindent}{0mm}
\setlength{\parskip}{1em}
\begin{document}
\begin{center}
\rule{6in}{1pt} \
{\large Andreas Stathopoulos \\
{\bf Iterative validation: a scheme for improving the reliability and performance of eigenvalue block iterative solvers}}

Department of Computer Science  \\           College of William and Mary \\ Williamsburg \\ VA 23187-8795
\\
{\tt andreas@cs.wm.edu}\\
James, R. McCombs\end{center}

In large, sparse eigenvalue calculations,
if the matrix can be factored the benefits are twofold;
First, using iterative methods in shift and invert mode very fast
convergence can be achieve toward eigenvalues close to a specified shift.
Second, it allows for calculating the matrix inertia
for predicting the exact number of eigenvalues within an interval
and thus provides assurance that no required eigenpairs have been missed.

In this work, we are interested in cases where the matrix is too dense
or too large to factor, and thus the above validation of results is
not possible.
In such cases, the required eigenvalues are usually
tightly clustered or multiple, causing Krylov methods to converge
slowly and often miss eigenpairs.
Several methods such as (Jacobi-)Davidson, LOBPCG and EIGIFP
exploit preconditioning to improve convergence and robustness of
eigenvalue iterative codes.
Still, however, no assurance is provided that no eigenvalues are missed.

Block Krylov methods work on an orthonormal set of vectors simultaneously
instead of one vector.
Provided that the initial set of vectors are not deficient in the direction
of required eigenvectors, block methods identify all clustered or
multiple eigenvalues within the size of the block.
Yet, eigenvalues can still be missed beyond the block size.
Moreover, knowledge about the problem is required to decide the
appropriate block size.
Computationally, block methods usually require more floating point
operations than single vector methods but they are more cache efficient.
For certain block sizes execution time is improved, but such
a choice is complicated as it affects the numerics of the problem.


Alternatively, a large number of (even multiple) eigenvalues can be
obtained through a stable form of deflation called locking.
When an eigenvalue $\lambda_1$ converges, all subsequent
computations are performed orthogonally to the corresponding
eigenvector $x_1$.
This guarantees that the method will converge to a different eigenpair,
but not necessarily the next required one.
In practice, locking does not miss eigenvalues and it even identifies
multiplicities if a tolerance close to machine precision is used.
Even though in theory single vector Krylov methods cannot obtain
multiplicities, floating point arithmetic introduces noise in the direction
of the invariant subspace that is gradually amplified.
In the presence of a very high multiplicity or a high number of multiplicities
block methods are more effective.
Similarly, with high tolerances, block methods are still
robust, while locking may miss or yield inaccurate eigenpairs.
Typically, single vector methods with locking are preferred because of
their efficiency, reverting to small block sizes when the problem is
known to have multiplicities.

To improve confidence on obtained results, eigenvalue practitioners
traditionally restart the iterative method with a new random initial guess,
locking against all previously computed eigenvectors.
If the new eigenvalue is in the required range, the computation
must be continued.
However, it is neither efficient nor robust to rely on a single
vector method to obtain highly clustered or multiple eigenvalues, especially
with lower accuracies.

We propose a new technique, which we call {\it iterative validation},
that acts as a wrapper calling another eigenvalue block iterative method
repeatedly until no missed eigenvalues can be identified.
The inner method can be any block iterative method, such as
block Lanczos or subspace iteration, that implements locking against
an externally provided set of vectors.
We describe iterative validation assuming the lower {\tt nev}
eigenpairs of a symmetric matrix are needed.

{\bf 1.}
The user calls the inner block method as (s)he would normally do,
specifying the smallest necessary block size (usually one) to obtain
efficiently most of the required eigenpairs.
After completion, if requested,
our validation technique starts from step 2.

{\bf 2.}
Using error bounds on the {\tt nev} Ritz values
$\lambda_1, \ldots , \lambda_{nev}$, we identify the largest
numerical multiplicity obtained thus far, say $m$.

{\bf 3.}
We set the block size equal to $m+1$, and call back the inner
block method with $m+1$ random initial guesses,
seeking the smallest eigenvalue of the matrix with
all previously computed eigenvectors locked.

{\bf 4.}
After convergence, we report all $m+1$ lowest eigenpairs in the block
$(\mu_1, z_1), \ldots , (\mu_{m+1}, z_{m+1})$.
We compute the error bound of $\mu_1$: $\epsilon_1$.

{\bf 5.}
If $\mu_1 - \epsilon_1 > \lambda_{nev}$, report the original $\lambda_i$
eigenvalues and stop.

{\bf 6.}
If $\mu_1 < \lambda_{nev}$, an eigenvalue was missed.
Insert $\mu_1$ among the eigenvalues $\lambda_i$,
and dispense with the eigenpair of $\lambda_{nev}$
\footnote{If space is available, it is better to keep the
eigenpair $\lambda_{nev}$ as it will help convergence in the
next validation step.}.
Consider $(z_2, \ldots , z_{m+1})$ as initial guesses. The rest
will be random vectors.
Go to step 2.

{\bf 7.}
if $\lambda_j < \mu_1 - \epsilon_1 < \lambda_{j+1}$, return to the
user with a note that the provided tolerance
was not sufficient to resolve an eigenvalue between $\lambda_{j+1}$
and $\mu_1$.

The above technique has several advantages.
First, it provides a relatively unobtrusive way to
validate results, dramatically improving robustness.
If the original code would miss eigenvalues,
the additional expense is more than justified.
Typically, the validation will be run once for each new class of
problems, and switched off afterwards.
Second, it provides a dynamic way to fine tune the block size without
wasting all previous effort.
Third, assuming a multiplicity or a cluster of $m$ eigenvalues,
a block size of $m$ would be slower for many
required eigenvalues that do not belong to the cluster or
the multiplicity.
Iterative validation with an original block size of 1 finds first
the easier part of the spectrum and it only uses the block size
$m$ at the clusters that need it.

We have implemented iterative validation as a Matlab wrapper for
\mbox{irbleigs} and \mbox{lobpcg} eigenvalue codes, and as a C
wrapper for our block Jacobi-Davidson.


\end{document}
