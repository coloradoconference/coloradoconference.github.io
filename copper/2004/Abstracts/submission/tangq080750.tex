\documentclass{report}
\usepackage{amsmath,amssymb}
\setlength{\parindent}{0mm}
\setlength{\parskip}{1em}
\begin{document}
\begin{center}
\rule{6in}{1pt} \
{\large Qing Tang \\
{\bf The Use of an Algebraic Multi-Grid Pre-Conditioner in a Newton-Krylov Based CFD Solver for Modeling Turbulent Reacting Flow}}

Reaction Engineering International \\ 77 West 200 South \\ Suite 210 \\ Salt Lake City \\ UT 84101
\\
{\tt tang@reaction-eng.com}\\
Martin Denison\\
Mike Maguire\\
	Mike Bockelie\end{center}

A Newton-Krylov based Computational Fluid Dynamic (CFD) modeling tool is
being developed for performing engineering calculations of turbulent
reacting flows in which finite rate chemistry effects are important. The
near term target applications are simulations of NOx emissions from
industrial combustion systems.

The finite rate chemistry effects are modeled using reduced chemical
kinetic mechanisms. Reduced mechanisms are noted for being able to
faithfully reproduce the reactions of a detailed kinetic mechanism while
tracking only a relatively small number of species. However, the
disadvantage to using reduced mechanisms is the (sometimes) severe
non-linearity and corresponding ?numerical stiffness? they induce in the
governing equations.

The new tool employs a matrix free Newton-Krylov iteration scheme which
is suitable for solving large scale systems containing severe
non-linearity. A GMRES method is used to solve for the inexact Newton
step. Experience has taught us that a pre-conditioner must be used in the
linear solves within the GMRES method due to the stiffness problem. In
our current research, an Algebraic Multi-Grid (AMG) solver has been
integrated into the pre-conditioning calls of the Newton-Krylov solver.
The AMG preconditioner uses a generalized method for agglomerating fine
grid cells to create the coarse grid problem. Preliminary results have
shown the improvement in the overall efficiency and robustness of the new
solver.

In this presentation, we will describe the formulation for our
Newton-Krylov solver with the AMG pre-conditioner. Performance of the new
solver will be highlighted through NOx simulations for a simple, ?hotbox?
furnace and a full scale coal fired electric utility boiler.

Acknowledgements:
This work was performed under support from the National Science
Foundation under SBIR Phase II grant no. DMI-0216590 (NSF Program Manager
Dr. Errol Arkilic). The authors would like to thank Dr. Michael Pernice
(Los Alamos National Laboratory) and Prof. J.-Y. Chen (U. of
California-Berkeley) for their input and guidance in this work.


\end{document}
