\documentclass{report}
\usepackage{amsmath,amssymb}
\setlength{\parindent}{0mm}
\setlength{\parskip}{1em}
\begin{document}
\begin{center}
\rule{6in}{1pt} \
{\large Todd Coffey \\
{\bf Optimization Based Initial Condition for Multiple-time Partial Differential Equation Methods Applied to Circuits}}

Sandia National Laboratories \\ PO BOX 5800 MS-1110 \\ Albuquerque \\ NM 87185-1110
\\
{\tt tscoffe@sandia.gov}\\
Martin Berggren\end{center}

Many highly oscillatory circuits have a wide separation of time scales between the underlying oscillation and the behavior of interest.
This is particularly true of communication circuits.
Multiple-time Partial Differential Equation (MPDE) methods offer substantial speed-up for these circuits by introducing a periodic artificial time variable that represents the highly oscillatory behavior.
This leaves just the slowly changing behavior of interest,
which can be integrated with much larger steps.
One problem of particular interest is the larger initial condition that must be specified for this periodic artificial time variable.
One possible solution is to formulate an optimization problem in the hopes of increasing the step sizes taken in the slow time direction.
This talk will discuss one possible unconstrained optimization problem for determining this initial condition.
Numerical results and comparisons to several other initial condition strategies will be presented in addition to MPDE background research and implementation issues.

\end{document}
