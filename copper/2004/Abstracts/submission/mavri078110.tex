\documentclass{report}
\usepackage{amsmath,amssymb}
\setlength{\parindent}{0mm}
\setlength{\parskip}{1em}
\begin{document}
\begin{center}
\rule{6in}{1pt} \
{\large Dimitri J. Mavriplis \\
{\bf Solution of High-Order Discontinuous Galerkin Methods using a Spectral Multigrid Approach}}

Department of Mechanical Engineering \\ Dept. 3295  \\ 1000 E. University Ave. \\ University of Wyoming \\ Laramie \\ WY 82071
\\
{\tt mavripl@uwyo.edu}\end{center}

The use of high-order Discontinuous Galerkin methods has become
more widespread in recent years.
These methods are attractive due to their compact support
and ease of implementation, particularly for high-order
approximations (higher than second order).
While much work has been performed in developing spatial discretizations
based on this approach, relatively few investigations have addressed the
issue of efficient
solvers for the resulting discretized equations, with most applications
relying on explicit time-stepping approaches.

In the interest of achieving fast steady-state and time-implicit
solvers for high-order Discontinuous Galerkin (DG) discretizations,
we propose the use of a multigrid method where the coarser levels
are obtained by reducing the order of the DG approximation
(reducing the spectral index "p") as opposed to the traditional multigrid
approach of reducing the grid spatial resolution (discretization index
"h").
An "element Jacobi" iterative method is used to drive the "p"-multigrid
algorithm. This technique consists of (directly) inverting the block
Jacobian associated with all degrees of freedom within each grid cell at
each iteration.

Numerical experiments on the linear two-dimensional wave equation
demonstrate order-independent convergence rates (for p=1,2,3,4)
using element Jacobi as a solver, while convergence degradation
with increasing spatial grid resolution (h-refinement) is observed.
When the element Jacobi solver is used as a smoother on each level
of a "p"-multigrid approach, order independent convergence rates
are retained, and rates which are nearly independent of the spatial
grid resolution are observed.
Future work is underway to improve the grid independence of this
approach, and to extend the method to more coplex systems
of equations, such as the Euler and Navier-Stokes equations.


\end{document}
