\documentclass{report}
\usepackage{amsmath,amssymb}
\setlength{\parindent}{0mm}
\setlength{\parskip}{1em}
\begin{document}
\begin{center}
\rule{6in}{1pt} \
{\large Alison Ramage \\
{\bf Some Characteristics of Multigrid Performance for the Two-Dimensional Convection-Diffusion Equation}}

Department of Mathematics \\ University of Strathclyde \\ Livingstone Tower \\ 26 Richmond Street \\ Glasgow G1 1XH \\ Scotland
\\
{\tt A.Ramage@strath.ac.uk}\\
Howard C. Elman\end{center}

The development of efficient numerical solution techniques for
convection-diffusion problems is an important area of current research in
the field of iterative methods. As well as being of interest in their own
right, convection-diffusion problems are closely linked to the
Navier-Stokes equations governing incompressible fluid flow which are
widely applicable in industrial settings. One possible approach which has
been successfully applied in practice is to use a multigrid method.
However, unlike with
linear self-adjoint elliptic boundary value problems, the development of
related convergence analysis for the convection-diffusion problem has to
date been limited. In addition, much of the published theory in the area
is very technical and can be hard for the non-expert to interpret. The
aim of this talk is to develop a 'simple to use' analysis of multigrid
convergence factors
for the two-dimensional convection-diffusion equation.

We will focus on the two-grid solution of the system of linear equations
arising from a bilinear finite element discretisation,
with streamline diffusion added when stabilisation is necessary. Most
usually, multigrid convergence is examined in terms of the behaviour of
the norms of matrices obtained by splitting the iteration matrix into two
parts, representing the approximation property and smoothing property
respectively. In this work, we follow this approach while incorporating a
matrix transformation which reduces the underlying discretisation
matrices to tridiagonal form. This simplifies the analysis, and also
enables cheaper numerical computation of norm bounds for large problems.
We will demonstrate this technique using a periodic variant of a model
problem and use the results obtained to illustrate which trends do and do
not agree with those seen when solving (more practically relevant)
Dirichlet problems.


\end{document}
