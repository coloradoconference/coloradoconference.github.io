\documentclass{report}
\usepackage{amsmath,amssymb}
\setlength{\parindent}{0mm}
\setlength{\parskip}{1em}
\begin{document}
\begin{center}
\rule{6in}{1pt} \
{\large Jos\'e Mar\'{\i}n \\
{\bf Block approximate inverse preconditioners using the Sherman-Morrison-Woodbury formula}}

Universitat Polit\'ecnica de Valencia \\ ETSIA \\ Cam\'{\i} de vera \\ 14 \\ 46022 Valencia \\ Spain \\ Supported by Spanish DGI Grant BFM2001-2641
\\
{\tt jmarinma@mat.upv.es}\\
Juana Cerd\'an\\
Jos\'e Mas\\
	Taher Faraj\end{center}

In this work we consider the solution of nonsymmetric linear systems of the form
\begin{equation*}
Ax=b \, ,
\end{equation*}
by preconditioned Krylov iterations where $A \in \mathbf{R}^{n \times n}$
is a sparse, nonsingular matrix. We introduce a block approximate inverse
preconditioner which is a generalization of the AISM preconditioner
presented
by Bru et al. [SIAM Journal on Scientific Computing}, 25(2):701--715,
2003]. The computation of the preconditioner involves the well known
Sherman-Morrison-Woodbury formula.

Consider the matrix $A$ partitioned in block form:
\begin{equation}\label{equ:Apartitioned}
A = \left[ \begin{array}{cccc}
A_{11} & A_{12} & \ldots & A_{1p}\\
A_{21} & A_{22} & \ldots & A_{2p}\\
\vdots & \vdots & \ddots & \vdots\\
A_{p1} & A_{p2} & \ldots & A_{pp} \end{array} \right]
\end{equation}
where $A_{ij} \in \mathbf{R}^{m_i \times m_j}$, $\sum_{k=1}^{p} m_k = n$.
We denote the block columns of $A$ by $A_i$, $i=1,\ldots,p$. That is,
\[ A_i = [A_{1i}, A_{2i},\ldots,A_{pi}]^T.\]
Let $X=I_n$ and $Y=(A - sI_n)^T$ be matrices partitioned accordingly
where $I_n$ is the identity matrix of size $n$ and $s$ is a positive scalar.
One has that,
\begin{equation}\label{equ:sumA}
A = sI_n + \sum_{k=1}^{p} X_k Y_k^T.
\end{equation}
By defining block column vectors $U_k,V_k$, $k=1,\ldots,p$ as,
\begin{equation}\label{equ:unew2}
U_{k}=X_{k}-\sum_{i=1}^{k-1}s^{-1}U_{i}T_{i}^{-1}V_{i}^{T}X_{k},
\end{equation}
\begin{equation}\label{equ:vnew2}
V_{k}=Y_{k}-\sum_{i=1}^{k-1}s^{-1}V_{i}T_{i}^{-T}U_{i}^{T}Y_{k},
\end{equation}
where
\begin{equation}\label{equ:pivotenew}
T_k = I_{m_k} + s^{-1} V_k^T X_k = I_{m_k} + s^{-1} V_{kk}^T,
\end{equation}
one obtains the following expression for the inverse of $A$,
\begin{equation}\label{equ:Afactorized}
A^{-1}= s^{-1} I_n - s^{-2} U T^{-1} V^T \, .
\end{equation}

The matrices $U, V$ have block columns $U_k, V_k$, $k=1,\ldots,p$, respectively, and
$T$ is a block diagonal matrix with diagonal blocks $T_k$, $k=1,\ldots,p$.

A sparse preconditioner is obtained by applying a dropping strategy
during the computation of $U_k$ and $V_k$. This strategy consists in
removing off-diagonal nonzero entries which are less than a given
threshold. Once the inexact factors $\bar{U}_k$ and $\bar{V}_k$ have been
computed, two different preconditioning strategies can be defined:
\begin{equation*}
M_1 := s^{-1} I_n - s^{-2} \bar{U} \bar{T}^{-1} \bar{V}^T,
\end{equation*}
and
\begin{equation*}\label{equ:AISM2}
M_2 := s^{-2} \bar{U} \bar{T}^{-1} \bar{V}^T .
\end{equation*}

It will be shown that the block preconditioner can be computed without breakdowns for
$M-$matrices and $H-$matrices.
The results of numerical experiments obtained for a representative set of
matrices will be presented. Compared with point AISM it will be shown
that the BiCGSTAB and GMRES methods preconditioned with block AISM
converge in less iterations. Indeed, for some problems where point AISM
fails to converge, as the UTM* matrices, the block version works
successfully. In addition, the effect of some reorderings of the
coefficient matrix on the performance of the preconditioner is also
considered. The results will show that AISM can benefit from them.


\end{document}
