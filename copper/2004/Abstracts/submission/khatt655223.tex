\documentclass{report}
\usepackage{amsmath,amssymb}
\setlength{\parindent}{0mm}
\setlength{\parskip}{1em}
\begin{document}
\begin{center}
\rule{6in}{1pt} \
{\large Sanjay Khattri \\
{\bf Which are Better Conditioned Meshes Adaptive, Uniform, Locally Refined or Locally Adjusted ?}}

Room 634 \\ Johannes Bruns gt 12 \\ N-5008 Bergen \\ Norway
\\
{\tt sanjay@mi.uib.no}\end{center}

Adaptive, locally refined and locally adjusted meshes are preferred over
uniform meshes for capturing singular or localised solutions. Roughly
speaking, for a given degrees of freedom a solution associated with
adaptive, locally refined and locally adjusted meshes is more accurate
than the solution given by uniform meshes. In this work, we answer the
question which meshes are better conditioned.
We found, for approximately same degrees of freedom (same size of
matrix), it is easier to solve a system of equations associated with an
adaptive mesh.


\end{document}
