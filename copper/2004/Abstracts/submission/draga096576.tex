\documentclass{report}
\usepackage{amsmath,amssymb}
\setlength{\parindent}{0mm}
\setlength{\parskip}{1em}
\begin{document}
\begin{center}
\rule{6in}{1pt} \
{\large Andrei Draganescu \\
{\bf A fast multigrid method for inverting linear parabolic problems}}

The University of Chicago \\ Department of Mathematics \\ 5734 S University Avenue \\ Chicago \\ IL 60637
\\
{\tt draga@math.uchicago.edu}\end{center}

This work started as a bid for a cost-efficient method for inverting
time-dependent partial differential equations. The original motivation
comes from the problem of comparing simulations with experimental
results. Given that only partial information is available from
experiments, we address the question of finding the best possible
match by controlling initial conditions and/or other parameters. This is
formulated as an optimization process where the cost functional
includes a measure of the distance between simulated and experimental
data and regularizing terms. Since in general we may be talking about
large problems, for which the forward simulation may take many hours
on a powerful machine, it is clear that the attached optimization
problem should take no longer than a factor of ten of that time, a
demand which is in conflict with the large size of the control space.

As our first model problem we have selected the following inverse
parabolic problem: suppose $(x, t) \mapsto U(x, t)$ satisfies a linear
parabolic initial value problem with $U(\cdot, 0) = v$, and we are
given a final state $f\in L^2$ (= the ``measurements'') at time $T$.
Because $f$ may be only a reconstruction from partial data
(e.g. point values, spatial averages), or simply the measurements
themselves contain errors, we assume that an initial condition
corresponding exactly to the measurements is not unique or doesn't
exist. If we have some knowledge about the initial condition leading
closely to $f$, i.e. we have a guess $v_0$ at the initial value, the
question is to find $v\in L^2$ that minimizes \mbox{$|\!|U(\cdot, T)
- f|\!|_{L^2}^2 + \epsilon |\!|v - v_0|\!|_{L^2}^2$}. The choice of
$\epsilon$ in the regularizer depends on the measure of uncertainty
in the ``measurements'' $f$ (an alternative regularizer is $\epsilon
|v - v_0|_{H^1}^2$). Being quadratic, the optimization problem
converges in one Newton step. The challenge is to invert the Hessian
$H_{\epsilon}$ of the cost functional efficiently. If we denote by
${\mathcal S}$ the operator $v\mapsto U(\cdot, T)$ and by ${\mathcal
S}^*$ its $L^2$-adjoint then $H_{\epsilon} = \epsilon I + {\mathcal
S}^*\:{\mathcal S}$. Here, unlike the case of standard multigrid for
the Poisson problem, residual computation is expensive:
Hessian-vector multiplication (both discrete and continuous) amounts
to two forward solves; moreover, the Hessian is a full matrix when
using the nodal basis for the finite element space. Direct spectral
analysis for the case of the finite difference discretization for the
heat equation on a uniform grid shows that the conjugate gradient
method will converge in at most $10$ iterations for reasonable
choices of all parameters involved, regardless of the resolution. This
is due to the exponential decay of the eigenvalues of $H_{\epsilon}$
down to~$\epsilon$.

The multigrid algorithm we propose -- a mixture between a ${\mathcal
V}$ and ${\mathcal W}$-cycle -- requires {\bf only one residual
computation} at fine ({\bf how} fine depends on $\epsilon$) resolution
and two residual computations at coarse resolution in order to
resolve the problem to optimal order. The multigrid iteration is based
on the design of a multilevel, recursively defined preconditioner for
the Hessian of the discrete cost functional. We introduce a
scale-free \emph{spectral distance} function on the set of linear
operators with positive-definite symmetric part; the spectral
distance between the inverse of an operator and its preconditioner
gives an estimate of the convergence rate in the simple iteration.
Our key result consists in showing that the spectral distance between
the inverse of the discrete Hessian and the constructed preconditioner
is $O\left((\epsilon\: T)^{-1} h^2\right)$, hence it has the same
magnitude as the error between the discrete and the continuous
solutions of the minimization problem. Work estimates for the full
multigrid algorithm show that in two dimensions the cost is at most
$2.5 \times$(the work for one residual computation) in addition to
the total cost of all exact solves performed at the coarsest level.

This work extends naturally in a few directions. The square-variation
regularizer produces a Hessian that is a sum between a smoothing
operator -- ${\mathcal S}^*\:{\mathcal S}$ -- and a roughening one:
the Laplacian. For this case we propose a slightly nonsymmetric
preconditioner that still approximates the inverse of the discrete
Hessian to optimal order in the spectral distance. A second direction
refers to the case of semilinear parabolic equations. Adjoint methods
allow for exact calculation of gradients and Hessian-vector
multiplication at a cost equivalent to two forward computations
each. The Hessian has almost the same smoothing properties and in the
linear parabolic case, and this is a basic argument in our
analysis. Finally, the nonstationary Stokes equations are parabolic,
hence we expect our methods to work well for the associated discrete
inverse problems, with possible extension to control problems for
Navier-Stokes.


\end{document}
