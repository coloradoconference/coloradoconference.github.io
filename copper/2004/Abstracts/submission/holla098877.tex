\documentclass{report}
\usepackage{amsmath,amssymb}
\setlength{\parindent}{0mm}
\setlength{\parskip}{1em}
\begin{document}
\begin{center}
\rule{6in}{1pt} \
{\large Ruth M Holland \\
{\bf Sparse Approximate Inverses and Target Matrices}}

Oxford University Computing Laboratory \\ Wolfson Building \\ Parks Road \\ Oxford \\ OX1 3QD
\\
{\tt ruth@comlab.ox.ac.uk}\\
Andy J Wathen\\
Gareth J Shaw\end{center}

Sparse approximate inverses are a popular choice of preconditioner
especially for use in a parallel environment, due to the high degree of
parallelizability in both their construction and application. This is
certainly true when the preconditioner, $P$, is calculated by minimizing
$\|I-PA\|_{F}$ via the solution of independent linear least squares
problems for the rows separately.

In this talk we will consider generalizations of such preconditioners,
which maintain this attractive feature: firstly through use of an
alternative norm, and secondly via the use of `target matrices'. The
former has been discussed previously though it appears not to have been
exploited. The second and main focus here will be on readily inverted
target matrices, $T$, which form part of a preconditioner when $\|T-PA\|$
is minimized. We will discuss two varieties of target matrices, and show
results when a target matrix is used in the solution of
advection-diffusion problems.


\end{document}
