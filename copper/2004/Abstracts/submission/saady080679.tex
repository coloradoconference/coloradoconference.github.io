\documentclass{report}
\usepackage{amsmath,amssymb}
\setlength{\parindent}{0mm}
\setlength{\parskip}{1em}
\begin{document}
\begin{center}
\rule{6in}{1pt} \
{\large Yousef Saad \\
{\bf PHIDAL: A Parallel ILU factorization based on a Hierarchical Interface Decomposition}}

University of Minnesota  \\ Dept. of Computer Science and Engineering \\ 200 Union st. SE \\ Minneapolis \\ MN 55455
\\
{\tt saad@cs.umn.edu}\\
Pascal Henon\end{center}


Ideas from domain decomposition have often been adapted and extended
to general sparse matrices to derive parallel solution algorithms for
sparse linear systems. The Parallel Hierarchical Interface
Decomposition ALgorithm presented in this talk is in this category.
The method is reminescent of the the 'wirebasket' techniques of domain
decomposition methods \cite{SMITH} and can also be viewed as a
variation and an extension of the pARMS algorithm \cite{pARMS} in
which the independent sets and levels are defined from a hierarchical
decomposition of the interface structure of the graph. From an
implementation viewpoint, PHIDAL is an ILU factorization based on a
nested dissection-type ordering, in which cross points in the
separators play a special role.

The algorithm is based on defining a 'hierarchical interface
structure'. The hierarchy consists of classes with the property that
Class $k$ nodes, with $k>0$, are separators for class $k-1$ nodes. In
each class, nodes are grouped in independent sets. Class $0$ nodes
are simply interior nodes of a domain in the graph partitioning of the
problem. These are naturally grouped in group-independent sets, in
which the blocks (groups) are the interior points of each domain.
Nodes that are adjacent to more subdomains will be part of the higher
level classes and are ordered last. The factorization uses dropping
strategies which attempt to preserve the independent set structure.

One the hierarchical interface decomposition is defined, the Gaussian
elimination process proceeds by levels: nodes of the first level are
eliminated first, followed by those of the second level etc. All
nodes of the first level can be eliminated independently - since there
is no fill-in between nodes $i$ and $j$ of two different connectors of
level 1. On the other hand fill-ins may appear between connectors at
higher levels.

Two options are considered for handling fill-ins. The first is not to
allow any fill-in between two uncoupled connectors. Elimination in
this case always proceeds in parallel. The second option, which is
less restrictive, is to allow fill-ins only between nodes $i,j$ that
are in the same processor. In this case, elimination should be done
in a certain order and parallel execution can be maintained by
exploiting indedepent sets.


\begin{thebibliography}{1}

\bibitem{Bank-Wagner-MLILU}
R.~E. Bank and C.~Wagner.
\newblock Multilevel {ILU} decomposition.
\newblock {\em Numerische Mathematik}, 82(4):543--576, 1999.

\bibitem{MRILU}
E.F.F. Botta and F.W. Wubs.
\newblock Matrix {Renumbering} {ILU:} an effective algebraic multilevel {ILU}.
\newblock {\em {SIAM} Journal on Matrix Analysis and Applications},
20:1007--1026, 1999.

\bibitem{pARMS}
Z.~Li, Y.~Saad, and M.~Sosonkina.
\newblock {pARMS}: a parallel version of the algebraic recursive multilevel
solver.
\newblock {\em Numerical Linear Algebra with Applications}, 10:485--509, 2003.

\bibitem{SMITH}
B.~Smith, P.~Bj{\o}rstad, and W.~Gropp.
\newblock {\em Domain decomposition: Parallel multilevel methods for elliptic
partial differential equations}.
\newblock Cambridge University Press, New-York, NY, 1996.

\end{thebibliography}


\end{document}
