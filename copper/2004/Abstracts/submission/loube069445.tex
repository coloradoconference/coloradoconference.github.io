\documentclass{report}
\usepackage{amsmath,amssymb}
\setlength{\parindent}{0mm}
\setlength{\parskip}{1em}
\begin{document}
\begin{center}
\rule{6in}{1pt} \
{\large Raphael Loubere \\
{\bf ALE INC.: a 2d Arbitrary-Lagrangian-Eulerian code on polygonal staggered grids for compressible hydrodynamic problems}}

Los Alamos National Laboratory \\ T7 Group  \\ MS B284 \\ Los Alamos \\ NM \\ 87545
\\
{\tt loubere@lanl.gov}\\
Mikhail Shashkov\end{center}

Classical ALE code are constructed with a Lagrangian scheme (to increment
the fluid variables and the mesh position),
a Rezoner (to improve the quality of the mesh) and a Remapper (to
interpolate between the Lagrangian and the rezoned grids).

ALE INC. deals with polygonal grid in 2d in order to treat complex
geometries. Moreover staggered grids are used to define
fluid variables (nodal velocity, cell-centered internal energy, subcell
densities) and to remap/interpolate with a high-order
accuracy.

Such a code is conservative (mass, momenta and total energy), linear and
bound preserving, positivity preserving,
and material interfaces are preserved during the rezone/remap stages.

Validation test cases and more complex fluid flow simulations
will be presented in 2d for polygonal grids.


\end{document}
