\documentclass{report}
\usepackage{amsmath,amssymb}
\setlength{\parindent}{0mm}
\setlength{\parskip}{1em}
\begin{document}
\begin{center}
\rule{6in}{1pt} \
{\large Misha Kilmer \\
{\bf Fast Forward Solvers in 3D Diffuse Optical Tomographic Imaging}}

Mathematics Dept. \\ Bromfield-Pearson Bldg. \\ Tufts University \\ Medford \\ MA 02155
\\
{\tt misha.kilmer@tufts.edu}\\
Eric de Sturler\end{center}

\begin{document}
Diffuse optical tomographic (DOT) medical imaging makes
use of modulated, near-infrared light transmitted into
tissue from photodiodes placed on the body.
Optical detectors then measure the photon fluence
resulting from the scattering and absorption of
photons within the region of interest. The goal is to reconstruct three-dimensional
images of the scattering and absorption within the tissue.
In the case of breast tissue imaging, for example,
anomalous regions of absorption and scattering may indicate the presence of a tumor.

We utilize a frequency-domain diffusion equation model
for data generation. We parameterize the diffusion (related to the
scattering) and absorption
in terms of a small number of unknown parameters, represented by the
entries in the vector $p$.
Under the diffusion model, the synthetic data, $h(p)$,
is a non-linear function of
absorption and scattering. If $y$ represents our measured data
then the problem we wish to solve is the weighted, non-linear least squares problem
\[ \min_p \| R^{-\frac{1}{2}}( y - h(p)) \|_2 ,\]
where $R$ is related to the noise in the measured data.

We use a damped Gauss-Newton method to solve the optimization problem.
The major difficulty
is that computing $h(p)$ and the entries in the Jacobian
requires the solution of several large-scale linear systems. The size
of each system depends on the number of voxels in the 3D image.
If $N_s$ denotes the number of sources, $N_d$ denotes the number of detectors and
$N_w$ is the number of frequencies, the total number of linear systems
that must be solved is
$O((N_s + N_d) N_w)$ per Gauss-Newton step.
Thus the linear solves are a huge computational bottleneck for the imaging problem.

In this talk, we analyze characteristics of our matrices and discuss
techniques for reducing the computational
complexity of the forward solves. We consider preconditioned and
unpreconditioned Krylov-subspace methods for
solving the systems at every outer (i.e. Gauss-Newton) iteration.
In particular, we focus on exploiting relationships among
the systems that help minimize the computational effort at a
fixed iteration as well as over successive outer iterations.

\end{document}


\end{document}
