\documentclass{report}
\usepackage{amsmath,amssymb}
\setlength{\parindent}{0mm}
\setlength{\parskip}{1em}
\begin{document}
\begin{center}
\rule{6in}{1pt} \
{\large Ron Morgan \\
{\bf Deflated GMRES and BiCGStab for Multiply Shifted Systems in QCD}}

Department of Mathematics \\ Baylor University \\ Waco \\ TX  76798-7328
\\
{\tt Ronald\_Morgan@baylor.edu}\\
Dean Darnell\\
Walter Wilcox\end{center}

Large systems of linear equations with complex coefficients must be
solved in QCD (quantum chromodynamics) physics. In some cases these
systems not only have multiple right-hand sides, but also several shifts
$A-\sigma_i I$ for each right-hand side. We will discuss version of GMRES
and BiCGStab for solving such problems.

For systems with multiple right-hand sides, it is important to share
information. Block methods are one possibility. We will look first at
another approach of computing eigenvectors during the solution of the
first right-hand side with deflated GMRES. These eigenvectors are used to
aid GMRES or BiCGStab in the solution of the other right-hand sides. This
approach can also be combined with block methods.

For multiply shifted systems, it is important to solve all of them in
about the same cost as for only one system. We give a version of GMRES-DR
(a deflated GMRES method) that works for multiple shifts with the first
right-hand side. For the other right-hand sides, it is difficult to
deflate eigenvalues and still solve multiple shifts unless the
eigenvectors have been determined exactly. However, we show a way to get
mostly around this difficulty using an Arnoldi recurrence for the
approximate eigenvectors generated by GMRES-DR.


\end{document}
