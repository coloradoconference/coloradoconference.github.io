\documentclass{report}
\usepackage{amsmath,amssymb}
\setlength{\parindent}{0mm}
\setlength{\parskip}{1em}
\begin{document}
\begin{center}
\rule{6in}{1pt} \
{\large Emeric MARTIN \\
{\bf Using spectral low rank preconditioners for large electromagnetic calculations.}}

CERFACS \\ 42 Avenue Gaspard Coriolis \\ \\ 31057 Toulouse Cedex \\ \\ France.
\\
{\tt emartin@cerfacs.fr}\\
Iain DUFF\\
Luc GIRAUD\\
	Julien LANGOU\end{center}

We focus on the solution of a sequence of linear systems arising in
electromagnetic radar cross section, and having the same
coefficient matrix but different right-hand sides.
The problem consists in solving $M_{1}AX=M_{1}B$, where $M_{1}$ is a left
preconditioner, $A$ a large dense complex symmetric matrix that arises
from boundary element method, $X$ the block of unknowns vectors, and $B$
the block of right-hand sides.\\
\\
Our study starts from the observation that when the matrix $M_{1}A$ has
some eigenvalues near zero, the convergence
of the Krylov methods is often slow. The following proposition from [1]
shows that we can construct an update ${\tilde{M}}_c$ from spectral
information of $M_1A$ to correct $M_1$ such as the new preconditioned
system $M_{2}Au = M_{2}b$ no longer has eigenvalues in a certain
neighbourhood of zero.
Assume that $M_1A$ is diagonalizable:
\begin{center}
$M_1A = V\Lambda V^{-1},$
\end{center}
with $\Lambda$ the diagonal matrix formed by the eigenvalues
${\{\lambda_i\}}_{i\in\{1,n\}}$ ordered by increasing magnitude, and
$V$ the associated right eigenvectors. We consider the $k$ smallest
eigenvalues and $V_k$ the associated right eigenvectors.\\
{\bf Proposition 1.~~~}
{\it Let $W$ be such that
$\tilde{A}_c = W^H A V_k $ has full rank,
$ {\tilde{M}}_c = V_k \tilde{A}_c^{-1} W^{H} $ and
$ M_2 = M_1 + {\tilde{M}}_c.$
Then $M_{2}A$ is similar to a matrix whose eigenvalues are
$$
\left \{
\begin{array}{l l l}
\eta_i = \lambda_i & \mbox{\rm if } & i > k , \\
\eta_i = 1 + \lambda_i & \mbox{\rm if } & i \leq k . \\
\end{array}
\right .
$$
}
The matrix ${\tilde{M}}_c$ is defined as the Spectral Low Rank Update
(SLRU) for the left preconditioner $M_1$.\\
\\
To illustrate the efficiency of this approach we consider a set of large
and challenging real life industrial problems.
We perform experiments with a parallel fast multipole code [2] to compute
the matrix-vector products involving A.
For $M_1$ we choose the preconditioner developed in [3], suitable for
implementation in a multipole framework on parallel distributed
platforms.
It is based on a sparse approximate inverse using a Frobenius norm
minimization with an a priori sparsity pattern selection strategy.
The spectral information is computed in a preprocessing phase by an
external eigensolver: ARPACK [4].\\
\\
In this talk, we present the gain in terms of times and matrix-vector
products, for the complete monostatic calculations [6].
We also illustrate the effects on the convergence rate of GMRES [5] of
parameters such as the dimension of the update, the accuracy
of the spectral information, the quality of the original preconditioner
or the size of the restart.
We conclude with some comments on our on-going work where we combine the
SLRU preconditioner and the Seed-GMRES or the GMRES-DR [7] solver.\\
\\
\\
This work has been developed in collaboration with G. All{\'e}on from
EADS-CCR and G. Sylvand from CERMICS-INRIA.\\
\\
\\
\begin{description}
\item{[1].}
B. Carpentieri and I. S. Duff and L. Giraud. A class of spectral
two-level preconditioners.
{\it SIAM J. Scientific Computing,} 25(2):749-765, 2003.
\item{[2].}
G. Sylvand. {\it La M\'ethode Multip\^ole Rapide en Electromagn\'etisme :
Performances, Parall\'elisation, Applications.}
Th\`{e}se, 2002, Ecole {N}ationale des {P}onts et {C}hauss\'ees.
\item{[3].}
G. All{\'e}on and M. Benzi and L. Giraud. Sparse Approximate Inverse
Preconditioning for Dense Linear Systems Arising in Computational
Electromagnetics.
\textit{Numerical Algorithms}, Vol. 16, 1997, pp 1-15.
\item{[4].}
R. B. Lehoucq and D. C. Sorensen and C. Yang. ARPACK User's guide: ARPACK
Users' Guide: Solution of Large-Scale Eigenvalue Problems with Implicitly
Restarted Arnoldi Methods.
\textit{SIAM}, 1997.\\
http://www.caam.rice.edu/software/ARPACK/
\item{[5].}
Y. Saad and M. H. Schultz. GMRES: A Generalized minimal residual
algorithm for solving nonsymmetric linear systems.
\textit{SIAM J. Scientific and Statistical Computing}, 1986, Vol. 7, pp 856-869.
\item{[6].}
I. S. Duff and L. Giraud and J. Langou and E. Martin. Using spectral low
rank preconditioners for large electromagnetic calculations.
Technical Report TR/PA/03/95, 2003, CERFACS.
\item{[7].}
R. B. Morgan. GMRES with Deflated Restarting.
\textit{SIAM J. Scientific Computing}, 2002, Vol. 24(1), pp 20-37.
\end{description}


\end{document}
