\documentclass{report}
\usepackage{amsmath,amssymb}
\setlength{\parindent}{0mm}
\setlength{\parskip}{1em}
\begin{document}
\begin{center}
\rule{6in}{1pt} \
{\large Rakhim Aitbayev \\
{\bf Multilevel Preconditioners for Nonselfadjoint or Indefinite Orthogonal Spline Collocation Problems}}

Department of Mathematics \\ New Mexico Tech \\ Socorro \\ NM 87801
\\
{\tt aitbayev@nmt.edu}\\
Bernard Bialecki\end{center}

\newtheorem*{theorem}{Theorem}\newcommand{\GP}{\mbox{${\cal G}$}}\newcommand{\SH}{\mbox{$\scriptscriptstyle H^{2}(\Omega)$}}\newcommand{\SL}{\mbox{$\scriptscriptstyle L^{2}(\Omega)$}}We develop and study symmetric multilevel preconditioners for the computation ofthe orthogonal spline collocation (OSC) solution of a Dirichlet boundary value problem(BVP) with a nonselfadjoint or an indefinite operator.The OSC solution is sought in the space of piecewise Hermite bicubic spline functionsdefined on a uniform partition.We consider an additive and a multiplicative multilevel preconditionersthat are used with the preconditioned conjugate gradient (PCG) method.Our results and algorithms are closely related to those in\cite{Aitbayev_Bialecki_2003},
\cite{Bialecki_1998},
\cite{Bialecki_Dryja_1997},and \cite{SBG_1996}.Let $\Omega$ be a unit square $(0,1)\times(0,1)$ with the boundary $\partial\Omega$,
andlet $x=(x_1,x_2)$.
We consider a BVP\begin{equation}\label{eq:bvp}Lu\equiv\sum_{i,j=1}^{2} a_{ij}(x)u_{x_ix_j}+\sum_{i=1}^{2} b_i(x) u_{x_i}+c(x)u = f(x),\;\;x\in\Omega,\quad u=0 \;\;\mbox{on}\;\;\partial\Omega.\end{equation}Operator $L$ could be non-selfadjoint or indefinite in $L^2$ inner product.We assume that the principal part of $L$ satisfies the uniform ellipticity conditionand that BVP (\ref{eq:bvp}) has a unique solution in $H^2(\Omega)$.Let $\pi_0$ be a uniform coarsest rectangular partition of $\Omega$.
We obtain a set of partitions$\{\pi_k\}_{k=0}^{K}$ by standard coarsening,
andlet $V_0\subset V_1\subset\ldots\subset V_K\equiv V_h$ be the set of corresponding nested spaces ofpiecewise Hermite bicubics that vanish on $\partial\Omega$.Let $\sum$ denote the 2-D composite Gauss quadrature corresponding to partition $\pi_h$ with 4 nodesin each element.
Let $\GP_h$ denote the corresponding set of Gauss points.The OSC discretization of BVP (\ref{eq:bvp}) is defined by\begin{equation}\label{eq:osc}u_h\in V_h,\quad Lu_h(\xi)=f(\xi),\quad\xi\in\GP_h,\end{equation}and it can be written as the operator equation $L_hu_h=f_h$ in the Hilbert space $V_h$ with theinner product $(v,w)_h=\sum vw$.We define and study multilevel additive$B_{\textup{\scriptsize a}}$ and multiplicative $B_{\textup{\scriptsize m}}$ preconditionersfor solving the normal equation $L_h^*L_hu_h=L_h^*f_h$,
where $L_h^*$ is the adjoint to $L_h$.The implementation of $B_{\textup{\scriptsize a}}$ and $B_{\textup{\scriptsize m}}$is based on relationships between basis functions for two consecutive partitionsand the implementation of $B_{\textup{\scriptsize m}}$ is similar to that forV(1,1)-cycle with the Gauss-Seidel smoothing.A problem on the coarsest partition is assumed sufficiently small,
and it is solved exactly.The computational cost of the preconditioning algorithms is $\mbox{O}(N_K)$.The following is our main result.\begin{theorem}There are positive independent of $h$ and $K$ constants$\alpha_{\textup{\scriptsize a}}$,$\beta_{\textup{\scriptsize a}}$,$\alpha_{\textup{\scriptsize m}}$,and$\beta_{\textup{\scriptsize m}}$,such that\begin{eqnarray}\label{eq:spectral_equiv}&&\alpha_{\textup{\scriptsize a}}\,
(B_{\textup{\scriptsize a}}v,v)_h\leq (L_h^*L_hv,v)_h \leq\beta_{\textup{\scriptsize a}}\,
(B_{\textup{\scriptsize a}}v,v)_h,\quad v\in V_h,\\\nonumber&&\alpha_{\textup{\scriptsize m}}\,
(B_{\textup{\scriptsize m}}v,v)_h\leq (L_h^*L_hv,v)_h \leq\beta_{\textup{\scriptsize m}}\,
(B_{\textup{\scriptsize m}}v,v)_h,\quad v\in V_h.\end{eqnarray}\end{theorem}In the following table,
we present results of our numerical computations;
that is,the ratios of spectral constants in (\ref{eq:spectral_equiv}),the convergence factor $\bar{\rho}$,
which is the geometric mean of consecutiveresidual ratios,
and the CPU time.\begin{center}\begin{tabular}{c|ccr|ccr|ccr}
& \multicolumn{3}{c|}{\textit{Additive}}
& \multicolumn{3}{c|}{\textit{Multiplicative}}&\multicolumn{3}{c}{\textit{General}}\\\cline{2-10}$J$
& $\beta_{\textrm{\scriptsize a}}/\alpha_{\textrm{\scriptsize a}}$
& $\bar{\rho}$
& $t(s)$
& $\beta_{\textrm{\scriptsize m}}/\alpha_{\textrm{\scriptsize m}}$
& $\bar{\rho}$
& $t(s)$
& $\beta_{\textrm{\scriptsize m}}/\alpha_{\textrm{\scriptsize m}}$
& $\bar{\rho}$
& $t(s)$\\\hline3
& 3.883
& 0.072
& 0.19
& 1.367
& 0.005
& 0.18
& 925.2
& 0.094
& 0.33\\4
& 4.490
& 0.101
& 0.59
& 1.435
& 0.007
& 0.90
& 515.2
& 0.096
& 1.80\\5
& 5.016
& 0.125
& 2.13
& 1.476
& 0.008
& 3.87
& 457.5
& 0.121
& 8.44\\6
& 5.488
& 0.142
& 9.13
& 1.500
& 0.009
& 16.45
& 402.4
& 0.166
& 42.67\\7
& 5.845
& 0.156
& 49.93
& 1.515
& 0.009
& 73.43
& 381.4
& 0.202
& 199.40\\8
& 6.162
& 0.168
& 278.10
& 1.524
& 0.009
& 334.60
& 377.3
& 0.224
& 995.60\end{tabular}\end{center}Under \textit{Additive} and \textit{Multiplicative},
we list results for Poisson's equation,and under \textit{General} -- results fora PDE with a general nonselfadjoint and indefinite operator $L$ with the coefficients\begin{eqnarray*}&&a_{11}(x)=e^{x_1x_2},\quad a_{12}(x)=0.5/(1+x_{1}+x_{2}),\quad a_{22}(x)=e^{-x_1x_2},\\&&b_{1}(x)=x_2e^{x_1x_2}+10\cos[\pi(x_1+x_2)],\quadb_{2}(x)=-x_1 e^{-x_1x_2}+50\sin (2\pi x_1x_2),\\&&c(x)=50[1+1/(1+x_1+x_2)].\end{eqnarray*}The problem with the general equation is solved using the multilevel multiplicative preconditioner.We set $\pi_0=\Omega$ and reduce the relative residual to less than $10^{-12}$.The numerical results demonstrate the efficiency of our preconditioning algorithms.\begin{thebibliography}{1}\bibitem{Aitbayev_Bialecki_2003}{\sc R.~Aitbayev and B.~Bialecki},
{\em A preconditioned conjugate gradientmethod for nonselfadjoint or indefinite orthogonal spline collocationproblems},
SIAM J.
Numer.
Anal.,
41 (2003),
pp.~589--604.\bibitem{Bialecki_1998}{\sc B.~Bialecki},
{\em Convergence analysis of orthogonal spline collocationfor elliptic boundary value problems},
SIAM J.
Numer.
Anal.,
35 (1998),pp.~617--631.\bibitem{Bialecki_Dryja_1997}{\sc B.~Bialecki and M.~Dryja},
{\em Multilevel additive and multiplicativemethods for orthogonal spline collocation problems},
Numer.
Math.,
77 (1997),pp.~35--58.\bibitem{SBG_1996}{\sc B.~F.
Smith,
P.~E.
Bj{\o}rstad,
and W.~D.
Gropp},
{\em Domain Decomposition:Parallel Multilevel Methods for Elliptic Partial Differential Equations},Cambridge University Press,
New York,
1996.\end{thebibliography}

\end{document}
