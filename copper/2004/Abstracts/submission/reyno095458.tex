\documentclass{report}
\usepackage{amsmath,amssymb}
\setlength{\parindent}{0mm}
\setlength{\parskip}{1em}
\begin{document}
\begin{center}
\rule{6in}{1pt} \
{\large Daniel R. Reynolds \\
{\bf Numerical Modeling of Nonlinear Thermodynamics in SMA Wires}}

Lawrence Livermore National Lab \\ P O Box 808 \\ L-551 \\ Livermore \\ CA 94551
\\
{\tt reynoldd@llnl.gov}\\
Petr Kloucek\end{center}

In this talk I present a mathematical model designed to describe the
thermodynamic behavior of shape memory alloy wires, as well as a
computational technique to solve the resulting system of partial
differential equations.

The material physics for such shape-changing materials may be modeled by
a system of conservation equations based on a nonconvex free energy
potential. To deal with this model, we first discretize the resulting
system of equations implicitly in time, using a piecewise linear
space-time Galerkin method to satisfy discrete conservation principles.
This discretization results in a nonlinear, nonconvex root-finding
problem for the time-evolved finite element expansion coefficients. To
solve this system, we introduce a computational technique that combines a
standard Newton-Krylov nonlinear solver with a viscosity-based
continuation method.

We find that the proposed solution method allows the model to better
handle dynamic applications where the temporally local behavior of
solutions is desired, as compared with previous solution techniques that
required the addition of large viscous effects to stabilize the nonlinear
system. Moreover, computational experiments document that this
combination of modeling and solution techniques appropriately predicts
the thermally- and stress-induced material phase transformations, as well
as the hysteretic behavior and production of latent heat associated with
such materials.


\end{document}
