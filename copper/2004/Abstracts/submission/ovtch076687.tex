\documentclass{report}
\usepackage{amsmath,amssymb}
\setlength{\parindent}{0mm}
\setlength{\parskip}{1em}
\begin{document}
\begin{center}
\rule{6in}{1pt} \
{\large Evgueni Ovtchinnikov \\
{\bf Deflation Techniques for Preconditioned Gradient Subspace Iterations}}

Mathematics Department \\ \\ University of Colorado at Denver \\ \\ 1250 14th Street \\ \\ Denver \\ CO 80202 (until May 2004) \\ Harrow School of Computer Science \\ \\ University of Westminster \\ \\ London HA1 3TP \\ UK (perm.)
\\
{\tt eovtchin@math.cudenver.edu}\end{center}

Simultaneous computation of several eigenpairs by subspace iteration has
two important advantages over computing them succesively by single vector
iterations with deflation: (i) generally faster convergence to extreme
eigenpairs, and (ii) 'cluster robustness', that is, the ability to
efficiently compute clustered eigenvalues. However, when the number of
computed eigenpairs is large, so is the computational cost per iteration,
and in order to reduce this cost one has to resort to this or that
deflation technique that would allow to effectively remove the eigenpairs
computed to a desired accuracy from the computation. This paper discusses
various deflation techniques that can be used in the framework of the
so-called preconditioned gradient subspace iterations, which combine the
preconditioned steepest descent and its conjugate gradient accelerations
with the Rayleigh-Ritz method.


\end{document}
