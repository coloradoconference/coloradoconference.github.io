\documentclass{report}
\usepackage{amsmath,amssymb}
\setlength{\parindent}{0mm}
\setlength{\parskip}{1em}
\begin{document}
\begin{center}
\rule{6in}{1pt} \
{\large Bruno Carpentieri \\
{\bf Screening spectral information in preconditioning design}}

CERFACS. \\ 42 \\ Avenue Gaspard Coriolis.  \\ 31057 Toulouse Cedex 01.  \\ France.
\\
{\tt carpenti@cerfacs.fr}\\
Luc Giraud\\
Serge Gratton\end{center}


It is well known that the convergence of the Conjugate Gradient algorithm for solving
symmetric positive definite systems $Ax=b$ depends on
the eigenvalue distribution of the iteration matrix.
For a wide class of problems,
the presence of isolated eigenvalues at the extreme of the spectrum slows
down the convergence
significantly. In this circumstance, the rate of convergence of CG can be
greatly improved
if components of the
residual in the directions of eigenvectors associated to the isolated eigenvalues
are deflated by means of a suitable preconditioner.
Several adaptive preconditioning techniques have been proposed in the past few years
that attempt to tackle this problem efficiently~[3,~4]. The preconditioner $M$
has generally the form of a low-rank matrix update, and is constructed
using spectral information
extracted from the coefficient matrix;
the spectral information can be computed in advance, prior to the
iterative solution, or gathered over the iterations from the Lanczos process.


In this talk, we perform a preliminary study on the effect of the accuracy
of the eigencomputation on the quality of the spectral preconditioning;
in our study, we consider in particular the spectral preconditioner
proposed in~[1,~2].
The analysis is based on the perturbation theory, and leads us to
proposing a criterion for screening
the spectral information in the design of the preconditioner.
In particular, we extend this analysis to the case
of solving a sequence of linear systems involved in a non-linear process, like those
arising in Gauss-Newton process.
We show results of numerical experiments from data assimilation
problems in oceanography to illustrate the effectiveness of our approach
in a real-life application.


\begin{enumerate}
\item[{[}1{]}] B.~Carpentieri, I.~S.~Duff and L.~Giraud.
A class of spectral two-level preconditioners.
{\em SIAM Journal of Scientific Computing},
25(2):749-765,
2003.

\item[{[}2{]}] M.~Fisher. Minimization Algorithms for Variational Data Assimilation.
In {\em Proc. ECMWF Seminar "Recent Developments in Numerical Methods for
Atmospheric Modelling",
7-11 Sept. 1998},
pages~364-385,
1998.

\item[{[}3{]}] J.~Frank and C.~Vuik. On the construction of
deflation-based preconditioners.
{\em SIAM Journal of Scientific Computing},
23(2):442-462,
2001.

\item[{[}4{]}] Y.~Saad, M.~Yeung, J.~Erhel and F.~Guyomarc'h.
A deflated version of the conjugate gradient algorithm.
{\em SIAM Journal of Scientific Computing},
21(5):1909-1926,
2000.

\end{enumerate}


\end{document}
