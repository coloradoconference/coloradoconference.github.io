\documentclass{report}
\usepackage{amsmath,amssymb}
\setlength{\parindent}{0mm}
\setlength{\parskip}{1em}
\begin{document}
\begin{center}
\rule{6in}{1pt} \
{\large Jun Zhang \\
{\bf A Class of Truly Parallel Multilevel ILU Preconditioning Techniques}}

Department of Computer Science \\ \\ University of Kentucky \\ \\ 773 Anderson Hall \\ \\ Lexington \\ KY 40506-0046
\\
{\tt jzhang@cs.uky.edu}\\
Chi Shen\end{center}

We present a class of parallel preconditioning techniques built on
a multilevel block incomplete LU (ILU) factorization strategy to
solve large sparse linear systems on distributed memory parallel
computers. The preconditioners are constructed by using the concept
of block independent sets, block incomplete LU factorization,
and Schur complement matrix. The most important step in building
such a preconditioner is the construction of the
block independent set. Two algorithms for constructing
block independent sets of a sparse matrix in a distributed
memory environment are proposed.

The new feature of this parallel multilevel preconditioner
is the fully parallel construction of the block independent
set in a distributed environment. Previous multilevel
preconditioners are either using a pseudo-multilevel
strategy or limited to two levels, due to the difficulty
in constructing the block independent set in a distributed
environment.

We compare a few implementations of the parallel multilevel ILU
preconditioners with different block independent set construction
strategies and different coarse level solution strategies.
For stability purpose, we utilize a diagonal thresholding strategy
both for the block independent set construction
and for the local block Schur complement ILU factorization.

We will comment on the advantages and disadvantages of
different strategies in constructing the block independent
set, and possible future research directions along this
line.

Numerical experiments indicate that our domain based
fully parallel multilevel block ILU preconditioners
are robust and efficient.

This research was supported in part by the U.S.
National Science Foundation under grants CCR-9902022,
CCR-9988165, CCR-0092532, and ACI-0202934, by
the U.S. Department of Energy Office of Science under
grant DE-FG02-02ER45961, by the
Japanese Research Organization for Information Science
\& Technology, and by the University of Kentucky
Research Committee.


\end{document}
