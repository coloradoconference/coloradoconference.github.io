\documentclass{report}
\usepackage{amsmath,amssymb}
\setlength{\parindent}{0mm}
\setlength{\parskip}{1em}
\begin{document}
\begin{center}
\rule{6in}{1pt} \
{\large Jens G. Schmidt \\
{\bf A Simulation Tool for Maxillo-Facial Surgery based on Finite Element Methods and Algebraic Multigrid}}

Jens Georg Schmidt \\ C&C Research Labs \\ NEC Europe Ltd. \\ Rathausallee 10 \\ 53757 St.Augustin \\ Germany
\\
{\tt schmidt@ccrl-nece.de}\\
Guntram Berti\\
Jochen Fingberg\end{center}


In this paper we describe a chain of computational tools (Maxillo-Facial
toolbox) for the simulation and planning of Maxillo-Facial surgery, known
as distraction osteogenesis. This surgery is performed to correct the
configuration of the bones of the midface of a patient. Therefore some of
the bones are cut and are given new positions. The repositioning of the
bones can be completed during the surgery or, if the displacements are
large, after the surgery by a pulling procedure that can last several
weeks.

While the final position of most of the bones is determined by medical
needs, the final shape of the facial tissue is not known a-priori. Often
additional plastic surgeries are necessary to correct the outcome of the
initial distraction. Our tool chain simulates the displacements of bones
and soft tissues and therefore provides the possibility to plan the
surgery in a way that the final shape of the facial tissues satisfies
patient and surgeon. The tool chain consists of the following parts:

The first tool of our simulation environment is a segmentation step that
takes a CT image of the patient's head as input and labels the volume
data into different anatomical structures. Then we generate a surface
mesh of the bone volume and use it as input for our virtual bone cutting
tool.

With the cutting tool the surgeon defines abitrary 3D cutting paths to
cut out parts of the skull and define their new positions.

The next tool in our chain creates a volume mesh based on the image data
of the head and incorporates the cuts and prescribed displacements. The
output of this tool is a Finite Element mesh of the head and a set of
boundary conditions corresponding to the prescribed cuts and
distractions.

In our main tool this mesh is used to simulate the distracting process by
our parallel Finite Element code which is combined with a fast multigrid
solver and can use a variety of material laws (linear elastic,
hyperelastic, viscoelastic) and different Finite Element formulations
(linear, quadratic, mixed elements of tetrahedral, hexahedral and
pyramidal shape). By using different discretizations our tool can provide
simulations of varying accuracy, starting from linear, static and very
fast simulations, that are used to provide a first impression for the
surgeon, up to very detailed ones that take into account non-linearities
and time dependencies of the underlying problem and need several hours of
computing time. The ability to vary the accuracy of the physical model
and its discretization is highly required by the physicians and one of
the major advantages of our tool. In a last step the results are
downloaded and post-processed.

In this paper we focus on the details of the parallel Finite Element
code. We show how our improvements in both areas from static linear
elastic models to time dependent viscoelastic ones lead to more reliable
results for the surgeon. We will also present our experience in combining
different discretization methods, such as pure displacement elements or
mixed elements, with different linear and non-linear solvers. We will
show that in order to build an efficient and robust simulation tool one
has to choose the combination of models and solvers very carefully.

\bigskip

{\bf Keywords:} Biomechanic, Maxillo-Facial Surgery, Parallel Finite
Element Method, Nonlinear Elasticity, Algebraic Multigrid


\end{document}
