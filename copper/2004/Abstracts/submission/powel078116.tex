\documentclass{report}
\usepackage{amsmath,amssymb}
\setlength{\parindent}{0mm}
\setlength{\parskip}{1em}
\begin{document}
\begin{center}
\rule{6in}{1pt} \
{\large Catherine, E. Powell \\
{\bf Efficient preconditioning for saddle-point systems arising in mixed finite element approximation of Darcy flow}}

UMIST \\ Maths Dept. \\ P.O.Box 88 \\ \\ Sackville Street \\ Manchester \\ \\ M60 1QD \\ UK.
\\
{\tt cp@fire.ma.umist.ac.uk}\end{center}

The main focus of this work is the design of optimal and computationally
efficient preconditioning strategies for a particular saddle-point system
that arises in the mathematical modelling of flow in porous media.
Specifically, we are solving a mixed formulation of a standard scalar
diffusion problem, derived from Darcy's law.Discretisation via the
lowest-order Raviart-Thomas finite element method yields a saddle-point
system matrix,
$$ C = \left(\begin{array}{cc} A & B^{T} \\ B & 0
\end{array}\right),$$
\noindent that is highly ill-conditioned with respect to the
discretisation parameter \textit{and} to the coefficients embedded in the
permeability tensor $\mathcal{A}.$

The system can be solved in a multitude of different ways. However,
permeability coefficients frequently exhibit anisotropies and
discontinuities and several previously suggested preconditioning schemes
are known to lose robustness in such cases. Further, many authors comment
on the difficulties of solving indefinite systems, and have developed
methodologies to convert the system in question to a positive definite
one. We demonstrate that solving the original unmodified saddle point
system using minimal residual schemes is not problematic provided
sufficient attention is paid to the coefficient term.

It is well known that the underlying variational problem is well-posed in
two pairs of function spaces, $H(div)\times L^{2}$ and $L^{2} \times
H^{1}$, leading to the possibility of two distinct types of
block-diagonal preconditioners. We evaluate the efficiency of both types
of approach when anisotropic and discontinuous diffusion coefficients are
present. The
first of these incorporates a known `specialised' multigrid approximation
to a weighted $H(div)$ operator. For the
second, we propose a simpler black-box method, the key tools
for which are diagonal scaling for a weighted mass matrix and an
algebraic multigrid V-cycle applied to a sparse approximation to a
generalised diffusion operator.

Eigenvalue bounds and numerical results are presented to illustrate not
only the optimality of both preconditioners with respect to the
discretisation parameter but also the superior robustness of the
black-box scheme with respect to the PDE coefficients. Today, the
existence of freely available algebraic multigrid codes makes it a
feasible preconditioning strategy for tackling saddle-point problems
arising in mixed finite element formulations of a wide range of other
second-order elliptic problems.

\begin{thebibliography}{10}
\bibitem{one} Powell, C.E., Silvester, D., Optimal preconditioning for
Raviart-Thomas mixed formulation of second-order elliptic PDEs.
\textit{To appear in SIAM J. Matrix Anal., 2004.}
\bibitem{two}
Powell, C.E., Silvester, D., Black-box preconditioning for
mixed formulation of self-adjoint elliptic PDEs. \textit{Lecture Notes in
Computer Science, 35, `Challenges in Scientific Computing,' 2003.}
\bibitem{three}
Powell C.E., Optimal preconditioning for mixed finite element
formulation of second-order elliptic problems, \textit{Ph.D.
thesis,UMIST, (England), 2003.}
\end{thebibliography}


\end{document}
