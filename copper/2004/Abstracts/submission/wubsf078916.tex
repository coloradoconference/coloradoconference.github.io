\documentclass{report}
\usepackage{amsmath,amssymb}
\setlength{\parindent}{0mm}
\setlength{\parskip}{1em}
\begin{document}
\begin{center}
\rule{6in}{1pt} \
{\large Fred W. Wubs \\
{\bf Preconditioners for Saddle Point Problems arising in Ocean Flow Computations}}

University of Groningen \\  \\ Research Institute for Mathematics and Computing Science \\  \\ P.O. Box 800 \\  \\ 9700 AV Groningen \\ \\ The Netherlands.
\\
{\tt wubs@math.rug.nl}\\
Arie C. de Niet\end{center}

Commonly used finite element and finite difference discretizations of the
(Navier-)Stokes equation lead to the saddle point problem
\[
K \left(\begin{array}{c}
u \\ p \end{array}\right)\equiv
\left(\begin{array}{cc}
A & B\\ B^T & 0 \end{array}\right)
\left(\begin{array}{c}
u \\ p \end{array}\right)=
\left(\begin{array}{c}
f \\ g \end{array}\right),
\]
where the unknown vector $(u,p)^T$ contains velocities and
pressures respectively. The matrix $A$ is positive definite (not
necessarily symmetric). This matrix contains also the essence of
matrices arising in ocean flow computations using THCM~[4]. In
those matrices we have to deal with non-symmetry of $A$ due to the
Coriolis force, which plays a very important role in the
equations.

In literature one can find several good preconditioners for the matrix
$K$. In general they exploit somehow the structure of the matrix. We
present two alternative preconditioners:
\[
\hat K_a = \left(\begin{array}{cc} A & B\\ B^T & -I/\omega \end{array}\right)
\;\;\mbox{ and } \;\;
\hat K_g = \left(\begin{array}{cc} A +\omega BB^T & 0\\ 0 & -I/\omega \end{array}\right),
\]
the {\it artificial compressibility} preconditioner and the {\it
grad-div} preconditioner respectively. The preconditioners are related to each other: to
solve an equation with $\hat K_a$ or $\hat K_g$ we have to solve an equation
with the matrix $A+\omega B B^T$.

In the table below we compare the number of iterations of our
preconditioners with that of {\sc simpler} [3] and the
preconditioner of Elman et al.~[2] on the 2D Stokes problem. In
the iterations we used an exact factorization of $A+\omega BB^T$
(in case of $\hat K_a$ and $\hat K_g$) and $A$ ({\sc simpler} and
Elman). Clearly the preconditioners perform better for $\omega$
large.
\begin{center}
\begin{tabular}{l|rrr|rrr|r|r|}
&\multicolumn{3}{|c|}{\sc grad-div} &\multicolumn{3}{|c|}{\sc art.
compr.} &\sc Elman & \sc simpler\\
\hline
$\omega$ & 1 & 16 & 256 & 1 & 16 & 256 & - & - \\
\sc iter& 4.5 & 3.0 & 3.0& 4.0 & 2.0 & 1.0 & 5.5& 16.5 \\
\hline
\end{tabular}
\end{center}

Of course, solving $A+\omega BB^T$ by a direct method is to
costly and therefore we employ {\sc mrilu}~[1]. We observed that
for large $\omega$, the system became harder to solve since then
the matrix is far from diagonal dominant. Experiments showed that
good preconditioners are constructed for $\omega$ ranging from 0
to 2.

For $\omega=1$, we compared the performance of {\sc mrilu} to the
performance of an exact factorization in {\sc matlab} for a range
of problem sizes. For the exact factorization we used an
approximate minimum degree ordering to reduce the fill. For a
repeated doubling of the number of variables in each direction it
appeared that the construction time in the exact factorization
grows with almost a factor $8$, while the construction time in
{\sc mrilu} grows with a factor close to $4$; the latter is also
the increase in problem size. For a problem of size about 1
million the fill in the {\sc MRILU} factorization is more than 4
times less than that in the exact factorization, leading to a more
than proportional reduction of a factor 7 for the construction
time. In order to gain 6 digits of accuracy 10 to 30 iterations
where needed using only a modest amount of time.

In order to study whether the above is also applicable in the
presence of a Coriolis force we took a matrix from THCM and
stripped it from parts related to the vertical velocity, the salt
transport and the energy transport. It turned out that with an
exact factorization of $A+\omega BB^T$ the behavior is similar to
that shown in the table above. Using {\sc MRILU} to solve the
system with $A+\omega BB^T$ showed a behavior similar to that
observed for the Stokes equation.

We concluded that for small values of $\omega$, MRILU combined
with the grad-div or artificial compressibility preconditioner
leads an effective method for the considered saddle point
problems.

In the talk we will present results of an eigenvalue analysis for
both preconditioners, which shows the effect of the parameter
$\omega$. Moreover we intend to show more results on the
performance of this approach, especially for the full ocean flow
equations.

\subsubsection*{References}
{\small

\noindent
[1] E. Botta and F. Wubs.
\newblock Matrix renumbering {I}{L}{U}: an effective algebraic multilevel
{I}{L}{U} preconditioner for sparse matrices.
\newblock {\em SIAM J. Matrix Anal. Appl.}, 20(4):1007--1026,
1999.

\noindent
[2] H. Elman, D. Silvester, and A. Wathen.
\newblock Performance and analysis of saddle point preconditioners for the
discrete steady-state {N}avier-{S}tokes equations.
\newblock {\em Numer. Math.}, 90(4):665--688, 2002.

\noindent
[3] C.~Vuik and A.~Saghir.
\newblock The {K}rylov accelerated {SIMPLE(R)} method for incompressible flow.
\newblock Report 02-01, Delft Univ. of Techn., Dep. Appl. Math. Anal., Delft, 2002.

\noindent
[4] W.~Weijer, H.~Dijkstra, H.~Oksuzoglu, F.~Wubs, and
A.~de~N~iet.
\newblock A fully implicit model of the global ocean circulation.
\newblock {\em J. Comput. Phys.}, 192:452--470, 2003.

}


\end{document}
