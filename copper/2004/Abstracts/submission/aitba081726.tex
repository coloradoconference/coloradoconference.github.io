\documentclass{report}
\usepackage{amsmath,amssymb}
\setlength{\parindent}{0mm}
\setlength{\parskip}{1em}
\begin{document}
\begin{center}
\rule{6in}{1pt} \
{\large Rakhim Aitbayev \\
{\bf Multilevel Preconditioners for Nonselfadjoint or Indefinite Orthogonal Spline Collocation Problems}}

Department of Mathematics  \\ New Mexico Tech  \\ Socorro \\ NM 87801
\\
{\tt aitbayev@nmt.edu}\\
Bernard Bialecki\end{center}

\newcommand{\GP}{\mbox{${\cal G}$}}
\newcommand{\SH}{\mbox{$\scriptscriptstyle H^{2}(\Omega)$}}
\newcommand{\SL}{\mbox{$\scriptscriptstyle L^{2}(\Omega)$}}

We develop and study symmetric multilevel preconditioners for
the computation of the orthogonal spline collocation (OSC) solution of a
Dirichlet boundary value problem (BVP) with a nonselfadjoint or an
indefinite operator. The OSC solution is sought in the space of
piecewise Hermite bicubic spline functions defined on a uniform
partition. We consider an additive and a multiplicative multilevel
preconditioners that are used with the preconditioned conjugate
gradient (PCG) method. Results and algorithms presented in this paper
are closely related to those in [1], [2], [3], and [4].

Let $\Omega$ be a unit square $(0,1)\times(0,1)$ with the boundary
$\partial\Omega$, and let $x=(x_1,x_2)$. We consider a BVP
\begin{equation}
\label{eq:bvp}
Lu\equiv\sum_{i,j=1}^{2} a_{ij}(x)u_{x_ix_j}+\sum_{i=1}^{2} b_i(x)
u_{x_i} +c(x)u = f(x),
\;\;x\in\Omega,\quad u=0 \;\;\mbox{on}\;\;\partial\Omega.
\end{equation}
Operator $L$ could be non-selfadjoint or indefinite in $L^2$ inner
product. We assume that the principal part of $L$ satisfies the
uniform ellipticity condition and that BVP (\ref{eq:bvp}) has a
unique solution in $H^2(\Omega)$.

Let $\pi_0$ be a uniform coarsest rectangular partition of $\Omega$.
We obtain a set of partitions $\{\pi_k\}_{k=0}^{K}$ by standard
coarsening, and let $V_0\subset V_1\subset\ldots\subset V_K\equiv V_h$
be the set of corresponding nested spaces of piecewise Hermite
bicubics that vanish on $\partial\Omega$. Let $\sum$ denote the
two-dimensional composite Gauss quadrature corresponding to partition
$\pi_h$ with 4 nodes in each element. Let $\GP_h$ denote the
corresponding set of Gauss points. The OSC discretization of BVP
(\ref{eq:bvp}) is defined by
\begin{equation}
\label{eq:osc}
u_h\in V_h,\quad Lu_h(\xi)=f(\xi),\quad\xi\in\GP_h,
\end{equation}
and it can be written as the operator equation $L_hu_h=f_h$ in the
Hilbert space $V_h$ with the inner product $(v,w)_h=\sum vw$.


Let $\{\psi_{k,j}\}_{j=1}^{N_k}$ be the standard finite element basis
of $V_k$ consisting of products of one-dimensional value and slope
basis functions. Using space decomposition
\[
V_h=V_0+\sum_{k=1}^{J}\sum_{j=1}^{N_k}V_{k,j}, \quad V_{k,j}
= \mbox{span}(\psi_{k,j}),
\]
we define and study multilevel additive $B_{\textup{\scriptsize a}}$
and multiplicative $B_{\textup{\scriptsize m}}$ preconditioners for
solving the normal equation $L_h^*L_hu_h=L_h^*f_h$, where $L_h^*$
is
the adjoint to $L_h$. The implementation of $B_{\textup{\scriptsize
a}}$ and $B_{\textup{\scriptsize m}}$ is based on relationships
between basis functions for two consecutive partitions and the
implementation of $B_{\textup{\scriptsize m}}$ is similar to that
for
V(1,1)-cycle with the Gauss-Seidel smoothing. A problem on the
coarsest partition is assumed sufficiently small, and it is solved
exactly. The computational cost of the preconditioning algorithms
is
$\mbox{O}(N_K)$. The following is our main result.

\textbf{Theorem.}
{\em There are positive independent of $h$ and $K$ constants
$\alpha_{\textup{\scriptsize a}}$, $\beta_{\textup{\scriptsize a}}$,
$\alpha_{\textup{\scriptsize m}}$, and $\beta_{\textup{\scriptsize m}}$,
such that}
\begin{eqnarray}
\label{eq:spectral_equiv}
&&\alpha_{\textup{\scriptsize a}}\, (B_{\textup{\scriptsize a}}v,v)_h
\leq (L_h^*L_hv,v)_h \leq
\beta_{\textup{\scriptsize a}}\,
(B_{\textup{\scriptsize a}}v,v)_h,\quad v\in V_h,
\\\nonumber
&&\alpha_{\textup{\scriptsize m}}\,
(B_{\textup{\scriptsize m}}v,v)_h
\leq (L_h^*L_hv,v)_h \leq \beta_{\textup{\scriptsize m}}\,
(B_{\textup{\scriptsize m}}v,v)_h,\quad v\in V_h.
\end{eqnarray}



To obtain these results, we prove the key assumptions in the general
theory of Schwarz methods presented in [4], and use the inequalities
\[
C^{-1}\|v\|_{\SH}^2\leq a_h(v,v)\leq C\|\Delta v\|_{\SL}^2,\quad v\in V_h,
\]
obtained in [2]. We present numerical results that demonstrate
the efficiency of our preconditioning algorithms.

\bigskip
\noindent
\textbf{References}

\medskip
\noindent
[1] {\sc R.~Aitbayev and B.~Bialecki}, {\em A preconditioned conjugate
gradient method for nonselfadjoint or indefinite orthogonal spline
collocation problems}, SIAM J. Numer. Anal., 41 (2003), pp.~589--604.

\medskip
\noindent
[2]
{\sc B.~Bialecki}, {\em Convergence analysis of orthogonal
spline
collocation for elliptic boundary value problems}, SIAM J. Numer.
Anal., 35 (1998),
pp.~617--631.

\medskip
\noindent
[3]
{\sc B.~Bialecki and M.~Dryja}, {\em Multilevel additive and
multiplicative methods for orthogonal spline collocation problems},
Numer. Math., 77 (1997), pp.~35--58.

\medskip
\noindent
[4]
{\sc B.~F. Smith, P.~E. Bj{\o}rstad, and W.~D. Gropp},
{\em Domain Decomposition: Parallel Multilevel Methods for Elliptic Partial
Differential Equations}, Cambridge University Press, New York, 1996.



\end{document}
