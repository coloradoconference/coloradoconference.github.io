\documentclass{report}
\usepackage{amsmath,amssymb}
\setlength{\parindent}{0mm}
\setlength{\parskip}{1em}
\begin{document}
\begin{center}
\rule{6in}{1pt} \
{\large John D. Wilson \\
{\bf Multigrid, Mixed Finite Element Methods, And Saddle-Point Problems}}

Denver Federal Center \\ Box 25046 \\ MS 413 \\ Denver \\ CO 80225
\\
{\tt johndw@usgs.gov}\\
Richard L. Naff\end{center}

Lowest-order Raviart-Thomas mixed finite element approximations to
second-order elliptic PDEs result in a linear system which is symmetric
and indefinite (saddle-point problem). The mixed system of equations can
be transformed into coupled symmetric positive definite matrix equations,
or a Schur complement problem, using block Gauss elimination. Nested
iteration and preconditioned conjugate gradient algorithms are the
simplest methods for solving the Schur complement problem. The mixed
finite element method is closely related to the cell-centered finite
difference scheme for solving second-order elliptic problems with
variable coefficients.
Simple but effective cell-centered multigrid methods have
been developed by others in the past. We exploit these methods
for solving subsurface flow problems on three-dimensional
(possibly distorted) hexahedral elements with discontinuous
hydraulic-conductivity coefficients (possibly anisotropic).
There are some questions that remain open concerning
how to solve problems with large amounts of anisotropy on distorted grids.


\end{document}
