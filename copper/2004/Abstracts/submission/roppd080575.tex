\documentclass{report}
\usepackage{amsmath,amssymb}
\setlength{\parindent}{0mm}
\setlength{\parskip}{1em}
\begin{document}
\begin{center}
\rule{6in}{1pt} \
{\large David L. Ropp \\
{\bf On the Instability of Operator Splitting Methods: Diffusion/Reaction Systems*}}

P.O. Box 5800 \\ MS 1110 \\ Sandia National Labs \\ Albuquerque \\ NM 87185
\\
{\tt dlropp@sandia.gov}\\
John N. Shadid\end{center}

Recently numerical experiments have demonstrated the instability of some common
second-order operator splitting methods when applied to the Brusselator system.
In this talk we demonstrate that this instability can be related to the
stability properties of the method used to integrate the diffusion operator. We
then consider the stability of operator splitting methods applied to a
model diffusion/reaction system that exhibits decay (a negative definite
operator) but has a
reaction operator which allows growth (a positive semi-definite or indefinite?).
Using this equation we consider the A-stability of the
operator split schemes and derive conditions on the
diffusion integrator to achieve A-stability. We demonstrate consequences of this theorem
for this model problem and demonstrate the result in the case of the Brusselator
system as well.


*This work was partially supported by the ASCI program and the DOE Office
of Science MICS program at Sandia National Laboratory under contract
DE-AC04-94AL85000.


\end{document}
