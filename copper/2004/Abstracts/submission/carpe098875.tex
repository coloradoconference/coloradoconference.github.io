\documentclass{report}
\usepackage{amsmath,amssymb}
\setlength{\parindent}{0mm}
\setlength{\parskip}{1em}
\begin{document}
\begin{center}
\rule{6in}{1pt} \
{\large Bruno Carpentieri \\
{\bf Two grid spectral preconditioning for general sparse linear systems}}

CERFACS \\ 42 \\ Avenue Gaspard Coriolis \\ 31057 Toulouse Cedex 01 \\ France
\\
{\tt carpenti@cerfacs.fr}\\
Luc Giraud\\
Serge Gratton\end{center}


Multigrid methods are among the fastest techniques to solve a linear
system $Ax = b$ arising from the
discretization of a partial differential equation.
The core of the multigrid algorithms is a two-grid procedure that is applied recursively.
The basic idea of the two-grid solver is :
\begin{enumerate}
\item given $x_0$, perform a few ($\mu_1$) steps of a basic stationary method of
the form $x^{(k+1)}~=~(I~-~MA)x^{(k)}~+~g $ to compute $x^{\mu_1}$.
This step is referred to as the pre-smoothing.
\item project the residual $r = b- Ax^{\mu_1}$ on a coarse space using a
restriction operator $R$ and solve
the linear system $RAP e = R r$, where $P$ is the prolongation operator.
\item prolongate the error in the fine space and update $x = x^{\mu_1} + P e$.
\item perform few ($\mu_2$) steps of a basic stationary method of
the form \\ $x^{(k+1)}~=~(I~-~MA)x^{(k)}~+~g $ to compute $x^{\mu_2}$.
This step is referred to as the post-smoothing.
\item If $x^\mu_2$ is accurate enough stop, else $x_0 = x^{\mu_2}$, go to Step 1.
\end{enumerate}
The smoother iterations aim at reducing the high frequencies of
the error (i.e. the components of the error in the space spanned by the
vectors associated
with the largest eigenvectors of $A$).
The restriction operator and consequently the coarse space is chosen so
that this space contains the
low frequency of the error (i.e. the components associated with the
smallest eigenvalues).
In classical multigrid, the coarse space is not defined explicitly
through the knowledge of the
eigencomponents but by the selection of a space that is expected to capture them.
The scheme presented above is a multiplicative algorithm~[1] but additive
variants~[2] also exist.

In this work, we exploit the idea of the two-grid method to design
additive and multiplicative
preconditioners for general linear systems.
We explicitly define the coarse space by computing the eigenvectors $V$
associated with the
smallest eigenvalues of $MA$ (that is, the components of the error that are not
damped efficiently by the smoother).
In that context, the prolongation operator is $P = V$.
We show that our preconditioners shift the smallest eigenvalues of $MA$
to one and tend to
cluster those that were already in the neighbourhood of one closer to one.
We illustrate the performance of our method through numerical experiments
on a set of general linear systems, both symmetric and positive definite and unsymmetric.
Finally, we consider a case study of a non-overlapping domain decomposition method
of semiconductor device modelling for the solution of the drift-diffusion equations.



\begin{enumerate}
\item[{[}1{]}] W. Hackbusch.
Multigrid methods and applications.
Springer-Verlag,
1985.

\item[{[}2{]}] R.~S.~Tuminaro. A highly parallel multigrid-like method
for the solution of the
Euler equations.
{\em SIAM J. Scientific and Statistical Computing}
vol.~13, pages~88-100,
1992.

\end{enumerate}


\end{document}
