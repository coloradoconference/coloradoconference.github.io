\documentclass{report}
\usepackage{amsmath,amssymb}
\setlength{\parindent}{0mm}
\setlength{\parskip}{1em}
\begin{document}
\begin{center}
\rule{6in}{1pt} \
{\large Jim E. Jones \\
{\bf A Multigrid Solver for Maxwell's Equations Based on Zero Circulation Functions}}

Center for Applied Scientific Computing \\ Lawrence Livermore National Laboratory \\ Box 808 \\ L-561 \\ Livermore \\ CA 94551
\\
{\tt jjones@llnl.gov}\\
David Alber\\
Barry Lee\end{center}

In this talk, we consider the curl-curl formulation of the
Maxwell's equations,
$\nabla\times \mu \nabla\times E + \sigma E = f,$
discretized using Nedelec edge elements.
Previous researchers' work (Reitzinger and Schoeberl [2002];
Bochev, Garasi, Hu, Robinson and Tuminaro [2003]) on developing
fast algebraic multigrid
solvers for this equation have used aggregation techniques
and carefully designed interpolation to preserve the null-space
of the curl operator.

In this talk we describe a new multigrid solver based on
element agglomeration (Jones and Vassilevski [2001]).
A key component of the method is the notion of a cycle:
a closed path of edges in the grid. Coarse grid cycles can be constructed
which are a union of fine grid cycles. The multigrid
solver then uses an interpolation
operator with the property that functions with zero circulation
on a coarse cycle are interpolated to have zero circulation on
the corresponding fine cycles. This property guarantees that the
null-space of the curl operator is preserved (for simply
connected domains).

The property of preserving zero circulation functions
does not uniquely determine the interpolation operator.
Further, this property alone is not enough to yield efficient
multigrid solvers. We discuss various methods for defining a unique
interpolation operator and present numerical results exploring
their effectiveness.


\end{document}
