\documentclass{report}
\usepackage{amsmath,amssymb}
\setlength{\parindent}{0mm}
\setlength{\parskip}{1em}
\begin{document}
\begin{center}
\rule{6in}{1pt} \
{\large Daniel B Szyld \\
{\bf The effect of non-optimal bases on the convergence of Krylov subspace methods}}

Valeria Simoncini \\
Department of Mathematics, Universit\`a di Bologna \\
and IMATI, CNR, Pavia, Italy

Daniel B Szyld \\
Department of Mathematics (038-16), Temple University \\
1805 N Broad Street \\ Philadelphia PA 19122-6094 \\
{\tt szyld@math.temple.edu}

\end{center}

There are many examples where non-orthogonality of a basis
for Krylov subspace methods arises naturally from the
application. Similarly, in many occasions it is desirable
not to orthogonalize with respect to all previous vectors,
thus obtaining truncated bases.
One such example is when $A$ is symmetric positive definite,
and we use a preconditioner $P=LU$. The preconditioned matrix
$L^{-1} A U^{-1}$ is nonsymmetric. One option is to use
an ``optimal" method for the preconditioned problem, say
GMRES or FOM. Another option is to use symmetric Lanczos,
i.e., orthogonalizing only with respect to the last two vectors
of the basis, thus obtaining a "non-optimal" basis.
This is less expensive, but the convergence is ``delayed."
We explore the question on what is the effect of having this
non-optimal basis. In particular, we compare the
residuals in the two cases. We conclude that the orthogonality
of the basis is not important.
We provide a bound on the ``delay" which depends on the
linear independence of the basis vectors.
Numerical examples illustrate our theoretical results.



\end{document}
