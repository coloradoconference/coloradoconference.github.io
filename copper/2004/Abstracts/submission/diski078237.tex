\documentclass{report}
\usepackage{amsmath,amssymb}
\setlength{\parindent}{0mm}
\setlength{\parskip}{1em}
\begin{document}
\begin{center}
\rule{6in}{1pt} \
{\large Boris Diskin \\
{\bf Quantitative Analysis Methods for Multigrid Solvers}}

National Institute of Aerospace \\ \\ 144 Research Drive \\ \\ Hampton \\ VA 23666
\\
{\tt bdiskin@nianet.org}\\
James L. Thomas\\
Raymond E. Mineck\end{center}


Limitations of the classical local-mode Fourier (LMF) analysis in application
to variable-coefficient problems have been discussed. Some restrictions
are required in both the
physical and frequency spaces to ensure accurate
predictions of convergence in these applications. With these
restrictions, certain solution regions,
are inaccessible to the LMF analysis.


Alternative, very general, quantitative analysis methods for
multigrid solutions of partial differential equation are
introduced.
The methods are applied to available, non-perfect multigrid solvers
that deal with practical problems;
parts of the solvers responsible for the less-than-optimal performance
are isolated, identified and improved.
The analysis methods considered in this talk focus
on the main complimentary parts
of a multigrid cycle: relaxation and coarse-grid correction.
Ideal relaxation (IR) and ideal coarse-grid (ICG) iterations
are introduced. In these iterations,
one part of the cycle (coarse-grid correction for IR iterations
and relaxation for ICG iterations) is actual
and its complimentary part is
replaced with an ideal imitation.
The IR and ICG iterations are very general and can be directly applied
in the most complicated situations including highly variable
(or nonlinear) coefficients,
complex geometries, and unstructured grids.
The results of this analysis
are not single-number estimates;
they are rather convergence patterns of the iterations
that may either confirm or refute expectations indicating
what part of the actual solver should be improved.
The generality of the analysis
makes it very valuable tool for analyzing
complicated large-scale computational problems, where no other
analysis methods are currently available.

Applications of these new analysis methods are demonstrated for model
problems arising in computational fluid dynamics.


\end{document}
