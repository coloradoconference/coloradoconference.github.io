\documentclass{report}
\usepackage{amsmath,amssymb}
\setlength{\parindent}{0mm}
\setlength{\parskip}{1em}
\begin{document}
\begin{center}
\rule{6in}{1pt} \
{\large Jeffrey J. Heys \\
{\bf Modeling Fluid-Elastic Interaction in 3-D with First-Order System Least Squares (FOSLS)}}

Department of Applied Mathematics \\ University of Colorado at Boulder \\ Campus Box 526 \\ Boulder \\ CO  80309-0526
\\
{\tt heys@colorado.edu}\\
Thomas A. Manteuffel\\
Stephen F. McCormick\end{center}

The mechanical interaction between a fluid and solid can me
mathematically modeled using a number of different approaches depending
on the physical characteristics of the problem being solved. We are
interested in systems consisting of a Newtonian fluid, modeled using the
Navier-Stokes equations, and a linear elastic material with properties
similar to a soft tissue. These coupled fluid-elastic problems are
inherently nonlinear because the shape of the fluid domain is not known a
priori, and the computational grid must be moved or mapped. We typically
use elliptic grid generation (EGG) to map the physical domain to a fixed
computational domain. A FOSLS formulation of the Navier-Stokes, EGG, and
linear elasticity equations provides a number of benefits to solving
coupled systems problems, including: optimal finite element approximation
in a desirable norm (H^1), optimal multilevel solver performance, optimal
scalability, and a sharp a posteriori error measure. The optimality and
performance of the formulation has been demonstrated extensively in 2-D
for a variety of problems, including the fully coupled fluid-elastic
system. However, as expected, the extension to 3-D brings new challenges
for both the whole and the individual parts of the coupled system. Some
of the issues associated with the extension to 3-D have been partially or
fully addressed, such as: growing complexity in the multilevel solver,
iteration schemes between the components of the fully coupled system,
extension to parallel computers, and proper scaling of the equations.
Other questions we are only beginning to answer, including the handling
of singularities and p-refinement.


\end{document}
