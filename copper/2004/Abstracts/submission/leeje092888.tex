\documentclass{report}
\usepackage{amsmath,amssymb}
\setlength{\parindent}{0mm}
\setlength{\parskip}{1em}
\begin{document}
\begin{center}
\rule{6in}{1pt} \
{\large Jeonghwa Lee \\
{\bf Two-level Preconditioning for Electromagnetic Scattering from Composite Conducting and Dielectric Objects}}

Department of Computer Science \\ University of Kentucky \\ 773 Anderson Hall \\ Lexington \\ KY 40506-0046 \\ USA
\\
{\tt leejh@engr.uky.edu}\\
Jun Zhang\\
Cai-Cheng Lu\end{center}

A coupled surface-volume integral equation approach is used for the
calculation of electromagnetic scattering with composite
conducting and dielectric materials. This approach has
high accuracy in solutions and can be accelerated by the multilevel
fast multipole algorithm (MLFMA) to reduce the CPU time and memory.
By using the method of moments (MoM), the coupled surface-volume
integral equations are converted into discrete matrix equations.
The near part of the resulting matrix consists
of four block components. Due to this specific data pattern, we
can construct two-level (TL) preconditioners to treat the
conducting and dielectric parts more efficiently.
The main purpose of this study is to show that
the TL preconditioning is effective with the MLFMA and
can reduce the number of
Krylov iterations substantially by using subdivided domain of
the near part matrix for each level.
Our experimental results indicate that the
TL preconditioning maintains the computational complexity of
the MLFMA, but converges a lot faster, thus effectively reduces
the overall simulation time for the coated material objects.


\end{document}
