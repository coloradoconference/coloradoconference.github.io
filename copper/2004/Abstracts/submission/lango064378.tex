\documentclass{report}
\usepackage{amsmath,amssymb}
\setlength{\parindent}{0mm}
\setlength{\parskip}{1em}
\begin{document}
\begin{center}
\rule{6in}{1pt} \
{\large Julien LANGOU \\
{\bf For a few iterations less}}

The University of Tennessee \\ Department of Computer Science \\ 1122 Volunteer Blvd. \\ Knoxville \\ Tennessee 37996-3450 USA
\\
{\tt langou@cs.utk.edu}\\
Gerard Sleijpen\end{center}

The purpose of this talk is to give a method that solves a large linear
systems with multiple right-hand
sides efficiently. We focus on Block Generalized Minimum Residual-like methods.

Given an $m$-by-$m$ nonsingular matrix $A$ and an initial set $r_0$ of $p_0$ residuals,
at step $n$ of the algorithm,
the (Block) Krylov subspace $\mathcal{K}_n(A,r_0) = [
r_0,Ar_0,A^2r_0,...,A^nr_0 ]$ is constructed and
an approximate (Block) solution $x_n$ is given to realize
$$\min_{x\in\mathcal{K}_n(A,r_0)} = \| r_0 - A x \|_{frobenius}. $$
We note $r_n$ the residual associated to $x_n$ and observe that
$\mathcal{K}_n(A,r_0) = [ r_0,Ar_0,Ar_1,\ldots,Ar_{n-1} ]$.

During the iterations, residuals tend to be colinear. In exact
arithmetic, if the residuals
become rank deficient ( $ r_n = u_n s_n $ with $u_n$ an $m$-by-$p_n$ nonsingular matrix
and $s_n$ a $p_n$-by-$p_0$ matrix with $ p_n < p_0 $ ), then it amounts
that the Krylov expansion at step
$(n+1)$ can be made via $u_n$ instead of $r_n$ (or $A^{n-1}r_0$).
This reduction of the size of the block used in the expansion process
is called \textit{deflation} and enables to save useless matrix-vector products
(in Block CG and Block QMR, it also have the interest of stabilizing the method).
A short bibliography of may be : Block CG [1], Block GMRES [2], Block QMR [4].

In this talk, the question of deflation is addressed in several points :
\begin{enumerate}
\item
In exact arithmetic, Block GMRES enables deflations; and that
even if the Krylov basis is not directly based upon the residuals.
If at step $n$, the residuals
become linearly dependent, the associated Krylov block also; and reciprocally.
\item
We report numerical results where, while the block residuals become strongly deficient,
the expansion vectors in Block GMRES are still well conditioned.
Therefore no obvious deflation appears.
This violates first item.
\item
If aggressive deflation is performed in the Krylov basis upon the loss of
independency of the residuals (several strategies are tried),
the convergence is noticeably delayed.
\item
A simple way to have a stable algorithm with deflation is to use Block GCR
since the expansion is based directly on the residuals.
\item
We give a criterion to deflate the block-residual in Block-GCR while
maintaining stability (numerical experiments are provided)
\item
Finally, an algorithm based on an ``expansion on the principal direction
of the residuals'' is given and tested.
Numerical experiments let think that this algorithm is the strategy that enables
to converge in the less matrix-vector products for Block GCR. This
algorithm is a variant to the block GCR given by Paul Soudais~[3].
\end{enumerate}

Two last points. First, the main drawback of GCR opposed to GMRES is that
GCR needs twice as memory as GMRES. However,
we intend to use this algorithm in an inner--outer scheme. In that case
FlexibleGMRES(outer)/GMRES(inner)
versus GCR(outer)/GMRES(inner) have the same storage requirements.
Second, item 1 and 2 have also been remarked by Micka\"el Robb\'e and
Miloud Sadkane~[5], some French colleagues,
they also propose an alternative to BlockGMRES
that performs deflation while maintaining the convergence rate. The
resulting algorithm is more close
to Block GCR than BlockGMRES. Some comparisons between our work and their
work is under way.

\begin{itemize}
\item[1.]
Dianne P. O'Leary.
The Block Conjugate Gradient Algorithm and Related Methods.
\textit{Linear Algebra and its Applications}, 29:293--32, 1980.
\item[2.]
Brigitte Vital.
\textit{\'Etude de quelques m{\'e}thodes de r{\'e}solution de
probl{\`e}mes lin{\'e}aires de grande taille sur multiprocesseur}.
{P}h.{D}. dissertation, Universit{\'e} de Rennes, November 1990.
\item[3.]
Paul Soudais.
Iterative solution of a 3--D scattering problem from arbitrary shaped
multidielectric and multiconducting bodies.
\textit{IEEE Antennas and Propagation Magazine}, 42(7):954--959, 1994.
\item[4.]
Roland W. Freund and Manish Malhotra.
A Block {QMR} Algorithm for Non-{Hermitian} Linear Systems With Multiple
Right-Hand Sides".
\textit{Linear Algebra and its Applications}, 254(1--3):119--157, 1997.
\item[5.]
Micka\"el Robb\'e and Miloud Sadkane.
Theoretical numerical analysis of breakdown in block versions of FOM and GMRES methods.
Personnal communication, 2003.
\end{itemize}


\end{document}
