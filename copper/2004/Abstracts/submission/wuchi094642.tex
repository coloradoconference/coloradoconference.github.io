\documentclass{report}
\usepackage{amsmath,amssymb}
\setlength{\parindent}{0mm}
\setlength{\parskip}{1em}
\begin{document}
\begin{center}
\rule{6in}{1pt} \
{\large Chin-Tien Wu \\
{\bf A Comparison of Geometric and Algebraic Multigrid for the Discrete Convection-Diffusion Equation}}

Department of Mathematics \\ University of Maryland \\ College Park MD 20742
\\
{\tt ctw@math.umd.edu}\\
Howard Elman\end{center}

The purpose of this paper is to evaluate
different solution strategies for the linear systems obtained from
discretization of the convection-diffusion equation \vspace{3mm}
\[
\left\{ \begin{array}{ll}
&-\epsilon \triangle u + b\cdot \nabla u = f,\\
&u=g \hspace{5mm} \mbox{on $\partial \Omega$},
\end{array} \right.\vspace{3mm}
\]
where $b$ and $f$ are sufficiently smooth and the
domain $\Omega$ is convex with Lipschitz boundary $\partial
\Omega$. We are interested in convection-dominated case, i.e. $|b| \gg
\epsilon$. In this setting, the solution typically has steep gradients in
some parts of the domain $\Omega$. These may take the form of boundary
layers caused by Dirichlet conditions on the outflow boundaries or
internal layers caused by discontinuities in the inflow boundaries.

It is well known that the standard Galerkin finite element
discretization on uniform grids produce inaccurate oscillatory
solutions to these problems. On the other hand, with carefully
chosen stabilization parameter, the streamline diffusion finite
element discretization (SDFEM) is able to eliminate most
oscillations and produce accurate solutions in the regions where
no layers are present. To increase accuracy of the solutions in
the regions where layers are present, an adaptive mesh refinement process
can be used. The adaptive mesh refinement process in this paper consists
of the following steps.\vspace{3mm}
\begin{enumerate}
\item The posteriori error estimator proposed by Kay and
Silvester is computed.
\item The maximum marking strategy is employed to select elements in
which the values of the error estimator are large.
\item The selected elements are refined by regular refinement.
\end{enumerate}
\vspace{3mm} This adaptive mesh refinement process can be applied
recursively until a certain tolerance of the error between the finite
element solution and the true solution is met.

In this paper, we are concerned with the costs of solving the
discrete system obtained when the effective discretization
strategy SDFEM together with an adaptive mesh refinement process
are used. We would like to explore the effectiveness of the
geometric multigrid (GMG) and algebraic multigrid (AMG) methods
for solving this linear system. Our numerical studies suggest that the
Krylov-subspace accelerated AMG is a robust solver for the
convection-diffusion problems.


\end{document}
