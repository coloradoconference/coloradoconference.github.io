\documentclass{report}
\usepackage{amsmath,amssymb}
\setlength{\parindent}{0mm}
\setlength{\parskip}{1em}
\begin{document}
\begin{center}
\rule{6in}{1pt} \
{\large Eric R. Keiter \\
{\bf Continuation Algorithms for Circuit Simulation}}

Computational Sciences Department \\ Sandia National Laboratories \\ P O Box 5800 \\ Mail Stop 0316 \\ Albuquerque \\ NM 87185-0316
\\
{\tt erkeite@sandia.gov}\\
Roger P.  Pawlowski\\
 Tom V.  Russo\\
	Robert J. Hoekstra, Scott A. Hutchinson, Jaijeet Roychowdhury\end{center}

Large-scale circuit simulation presents many challenges to theunderlying solver algorithms.
Difficulties usually center aroundobtaining a steady-state solution for the DC operating point,
buthomotopy/continuation methods can be attractive for such problemsbecause they offer theoretical guarantees of global convergence.Unfortunately,
often the most obvious choices for continuation parameter are ineffective for large problems.
This talk will focus on current efforts to develop continuation methods specific to the DC op calculation of large integrated circuits.
We will present a comparison of several different approaches,
applied to a number of representative circuits.

\end{document}
