\documentclass{report}
\usepackage{amsmath,amssymb}
\setlength{\parindent}{0mm}
\setlength{\parskip}{1em}
\begin{document}
\begin{center}
\rule{6in}{1pt} \
{\large Vivek Sarin \\
{\bf Preconditioning Techniques for Dense Linear Systems}}

Department of Computer Science \\ Texas A\&M University \\ College Station \\ TX 77843-3112
\\
{\tt sarin@cs.tamu.edu}\end{center}

Dense linear systems arise in a number of applications where one needs to
solve a set of integral equations. Since the dense coefficient matrix is
never computed and stored explicitly due to prohibitive cost, one must
use iterative methods to solve these systems. Matrix-vector products with
the coefficient matrix are computed via approximate hierarchical
techniques such as the Fast Multipole Method (FMM) which reduce the
complexity from $O(n^2)$ to $O(n)$ for a matrix of size $n$.
Unavailability of the coefficient matrix makes it difficult to construct
preconditioners for the system.


In this talk, we present techniques to transform the dense linear systems
into sparse systems which can then be preconditioned by incomplete
factorization techniques. The dense approximate matrix generated by the
hierarchical methods is represented as a product of sparse matrices. This
fact is exploited to convert the dense linear system into a sparse
system. The sparse coefficient matrix of the transformed system is
computed explicitly. Preconditioners for the sparse system are computed
via incomplete factorizations.


This approach has been successfully applied to capacitance extraction of
VLSI circuits in which the charge on a set of conductors is determined by
solving a dense linear system that relates known potential on each
conductor with the unknown charge distribution. The number of
discretization panels on the surface of the conductors determine the size
of the system. Non-symmetric systems are obtained when multiple
dielectrics are present. We will present numerical experiments to
demonstrate the effectiveness of the approach.


\end{document}
