\documentclass{report}
\usepackage{amsmath,amssymb}
\setlength{\parindent}{0mm}
\setlength{\parskip}{1em}
\begin{document}
\begin{center}
\rule{6in}{1pt} \
{\large Yogi A. Erlangga \\
{\bf On a complex Shifted-Laplace preconditioner for high wavenumber heterogeneous Helmholtz problems}}

Department of Applied Mathematical Analysis \\ Delft University of Technology \\ Mekelweg 4 \\ 2628 CD Delft \\ The Netherlands
\\
{\tt y.a.erlangga@math.tudelft.nl}\\
C. Vuik\\
C. W. Oosterlee\end{center}

We are concerned with numerical solutions of the boundary value problem
\begin{eqnarray}
\mathcal{L} u \equiv \left( \partial_{xx} + \partial_{yy} + k^2(x,y) \right) u
&=& f \, \, \, \text{in} \, \, \, \Omega \in \mathbb{R}^2, \label{eq1}\\
\lim_{r \rightarrow \infty} \sqrt{r} \left(\frac{\partial u}{\partial n}
- i k u \right) &=& 0 \, \, \, \text{on} \, \, \, \Gamma = \partial \Omega. \label{eq2}
\end{eqnarray}
This boundary value problem, so-called the Helmholtz problem, arises in
many applications, e.g. acoustics, electromagnetics and geophysics. Here,
we consider problems which mimic geophysical applications, in which
inhomogeneous property and high wavenumber $k$ are used. Furthermore, we
are interested in the solution methods based on the construction of the
iterants in the Krylov subspace.

For high wavenumbers, the matrix $A$ obtained from any discretization of
(\ref{eq1}) and (\ref{eq2}) has a property of indefiniteness, where the
spectrum is highly distributed in the left and right complex plane.
Furthermore, the condition number is also very large. These two
properties are not favorable for Krylov subspace iterative methods. So
far, there is no iterative method specially developed for indefinite
systems. Applying Krylov methods on its normal equations representation,
$A^*A$, does not provide a practical remedy since, even though the linear
system can be made definite, the condition number is even worse. Another
remedy is by suggesting a preconditioner.

In {\it [Erlangga et. al., TU Delft-AMA Report 03-01]}, we propose a
class of preconditioners, called Shifted-Laplace preconditioner, acting
on the Helmholtz problem. We also analyze some properties of these
preconditioners. The preconditioners are constructed based on an operator
\begin{eqnarray}
\mathcal{M} \equiv \partial_{xx} + \partial_{yy} - \alpha k^2(x,y), \label{eq3}
\end{eqnarray}
where $\alpha \in \mathbb{C}$. We have previously found that $\alpha =
i$, $i^2 = -1$ gives the best convergence for this class of
preconditioners. We call the preconditioner (\ref{eq3}) with $\alpha = i$
as the Complex Shifted-Laplace (CSL). We have shown also that (i) the
spectrum of $M^{-1}A$ is bounded above by one, and (ii) the lower bound
of the spectrum of $M^{-1}A$ is $\mathcal{O}(1/k^2)$. From the latter
conclusion, we may expect that the convergence rate for increasing $k$ is
only determined by the smallest eigenvalue. We have used the
preconditioner in BiCGSTAB. A significant reduction in the number of
iteration is observed.

In this paper, we discuss another issue in relation with preconditioner
solves of any Krylov subspace method. Previously, we solved the
preconditioner exactly using direct methods. This process is very costly.
Since the preconditioning matrix $M$ is complex, symmetric positive
definite (CSPD), several efficient methods can be implemented. We study
the use of incomplete LU decomposition of $M$ and multigrid as the
preconditioner solver.

For ILU factorizations, level-of-fill ILU acting on $M$ (or ILU(M)) is
used. For constructing the LU factors, an algorithm based on regular
stencil is used. The algorithm is not only fast in computing the LU
factors (so reduce initialization cost) but also requires less storage.
Only some diagonals should be stored, the remaining entries are
recomputed as needed. As for multigrid, we implement an algorithm as
discussed in {\it [Oosterlee et. al., SIAM J. Sci. Comput. 19(1) (1998),
pp.87--110]}, called MG1. MG1 is originally developed for real-valued
matrix. For our applications no modification is made to MG1.

We compare the number of iteration and time to convergence from the two
approaches with ILU preconditioner based on $A$ (or ILU(A)) for various
wavenumber $k$. The conclusions are as follows:
\begin{itemize}
\item[(1)] ILU(M) results in better performance than ILU(A). The
improvement becomes more significant as $k$ increases.
\item[(2)] One V(1,1) multigrid iteration is adequate to further improve
the computational performance. In comparison with the unpreconditioned
and ILU(M)-preconditioned case, for sufficiently large $k$, the
computational time is reduced almost by factor 10 and 3, respectively.
Furthermore, the number of iteration can be reduced by factor of 50 and
8, respectively.
\item[(3)] Preconditioner solves using multigrid seem to be less
sensitive to the inhomogeneity of the media. In comparison with constant
media, only less than 10 \% increase of the number iteration is observed
for multigrid. For ILU(M) and ILU(A), the number of iteration increases
by almost 100 \%.
\end{itemize}

In the future, the research will be geared towards an effective multigrid
algorithm for CSPD linear systems, such that the similar reduction factor
in the computation time and number of iteration can be achieved.


\end{document}
