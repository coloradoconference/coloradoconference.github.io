\documentclass{report}
\usepackage{amsmath,amssymb}
\setlength{\parindent}{0mm}
\setlength{\parskip}{1em}
\begin{document}
\begin{center}
\rule{6in}{1pt} \
{\large Ioannis G. Kevrekidis \\
{\bf Equation-Free Computation for Complex/Multiscale Systems Modeling}}

Department of Chemical Engineering and PACM \\ E-quad \\ Olden Street \\ Princeton University \\ Princeton \\ NJ 08544
\\
{\tt yannis@princeton.edu}\end{center}

In current modeling , the best available descriptions of a system often
come at a fine level
(atomistic, stochastic, microscopic, individual-based) while the
questions asked and the
tasks required by the modeler (prediction, parametric analysis,
optimization and control) are at a
much coarser, averaged, macroscopic level. Traditional modeling
approaches start
by first deriving macroscopic evolution equations from the microscopic
models, and
then bringing our arsenal of mathematical and algorithmic tools to
bear on these macroscopic descriptions.

Over the last few years, and with several collaborators, we have
developed and validated a
mathematically inspired, computational enabling technology that allows
the modeler to
perform macroscopic tasks acting on the microscopic models directly.
We call this the ``equation-free� approach, since it circumvents the
step of obtaining
accurate macroscopic descriptions and links with matrix-free
iterative linear algebra.

I will argue that the backbone of this approach is the design of
(computational) experiments.
In traditional numerical analysis, the main code �pings� a subroutine
containing the model,
and uses the returned information (time derivatives, function
evaluations, functional
derivatives) to perform computer-assisted analysis. In our approach the
same main code �pings� a subroutine that sets up a short ensemble of
appropriately initialized
computational experiments from which the same quantities are estimated
(rather than evaluated).
Traditional continuum numerical algorithms can thus be viewed as
protocols for experimental design
(where �experiment� means a computational experiment set up and
performed with a model at a
different level of description).

Ultimately, what makes it all possible is the ability to initialize
computational experiments at will.
Short bursts of appropriately initialized computational experimentation
�through matrix-free
numerical analysis and systems theory tools like variance reduction and
estimation- bridges
microscopic simulation with macroscopic modeling.
Remarkably, if enough control authority exists to initialize
laboratory experiments "at will", this computational enabling technology
can become a set of experimental protocols for the equation-free
exploration of complex system dynamics.


\end{document}
