\documentclass{report}
\usepackage{amsmath,amssymb}
\setlength{\parindent}{0mm}
\setlength{\parskip}{1em}
\begin{document}
\begin{center}
\rule{6in}{1pt} \
{\large Matthew Lasater \\
{\bf Using Continuation Methods for Modeling Nanoscale Semiconductor Devices}}

Department of Mathematics \\ Box 8205 \\ Center for Research in Scientific Computing \\ North Carolina State University \\ Raleigh \\ NC  27695-8205
\\
{\tt mslasate@unity.ncsu.edu}\\
C. T. Kelley\\
Andrew Salinger\\
	Dwight Woolard\end{center}

Resonant tunneling structures are quantum sized semiconductor devices,
which both theory and numerical simulation predict can sustain terahertz
current oscillations. The current in these devices are modeled by the
Wigner-Poisson Equations: a nonlinear PDE which describes the
time-evolution of the electrons and Poisson's equation to incorporate the
potential effects of the electrons. To study the steady-state solutions
of the PDE, we connected our simulator to LOCA (Library of Continuation
Algorithms), a software library developed at Sandia National
Laboratories. These algorithms track out steady-state solutions branches
as a function of a parameter using Newton's method and can also determine
the stability of these solutions.
In this talk, the equations, simulator, and the algorithms of LOCA will
be discussed. A scalable preconditioner was developed for the
parallelized simulator to accelerate the convergence of the linear solves
in Newton's method. Numerical results, including scalability and
efficiency statistics of the simulator,
will be presented.


\end{document}
