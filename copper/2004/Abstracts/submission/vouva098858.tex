\documentclass{report}
\usepackage{amsmath,amssymb}
\setlength{\parindent}{0mm}
\setlength{\parskip}{1em}
\begin{document}
\begin{center}
\rule{6in}{1pt} \
{\large Marinos, N. Vouvakis \\
{\bf A Fast DP-FETI like Domain Decomposition Algorithm for the solution of Large Electromagnetic Problems}}

Marinos N Vouvakis \\ The Ohio State University \\ ElectroScience Lab \\ 1320 Kinnear Rd  \\ Room 121 \\ Columbus \\ OH 43212 \\ Tel: (614) 292-7981 \\ Fax: (614) 292-7297
\\
{\tt vouvakis.1@osu.edu}\\
Jin-Fa Lee\end{center}

This paper introduces an efficient domain decomposition
algorithm of a class of time-harmonic Maxwell problems in $\bf{R}^3$. The
present domain decomposition method is a non-overlapping one; it
utilizes a set of Lagrange multipliers on the inter-domain interfaces.
In addition, the method allows for non-conforming/non-matching
triangulations across the interfaces. The proposed algorithm resembles
the well known Dual-Primal Finite Element Tearing and Interconnecting
(DP-FETI); since both methods eliminate the primal variables, and
solve the dense Lagrange multiplier block for the dual unknowns. To
achieve convergence of the outer iteration loop, the Robin
transmission condition is used to communicate information across
interfaces. Using the FETI like algorithm, the present method solves,
in the preprocessing step, for the Robin primal sub-domain problems
multiple times, by exciting one dual unknown each time. This step
generates an iteration matrix that is used to update the dual unknowns
in the outer-loop iteration. The present method becomes extremely
efficient for problems with geometric repetitions, such as antenna
arrays, photonic and electromagnetic band gap structures (PBG/EBGs),
frequency selective surfaces (FSSs) etc.

The analysis of electrically large electromagnetic (EM) problems with
spatial repetitions are of vital importance in practical industrial
and engineering applications. Of particular interest is the wave
radiation from large finite arrays, since such radiators are the
cornerstone of every modern RADAR, satellite communication system or
mobile phone base station. Yet another important class of problems
that excibits spatial repetitions optics with the photonic
crystals and photonic band gap structures. Using ``traditional'' PDE
or even fast IE methods, most of the above-mentioned problems can be
quite challenging or even impossible to solve, without using mainframe
and parallel computer architectures. In this paper, we propose a
domain decomposition methodology based on the Finite Element (FE)
approximation that enables the solution of such large scale problems
in single a PC.

In this paper a dual primal domain decomposition formulation is
developed for the three dimensional Maxwell problem. The method is
based on three core ingredients: first an optimized Robin type
transmission condition, second a mortar type formulation, and thirdly
an efficient FETI solution algorithm. The optimized Robin type
transmission condition employed herein are a 3-D vector wave extension
of the optimal conditions proposed in the literature. The
non-conforming mortar formulation is implemented to relax the mesh
generation, and avoid periodic meshes constrains in each domain. In this
work, the primal unknowns are eliminated in
the prepossessing step, without the use of the Schur
complement. Unlike FETI, the dual
problem is solved with a computational cheap Gauss-Seidel solver. The
Gauss-Seidel iteration matrix is constructed in the preprocessing
step, by eliminating of the primal variables.

Numerical results on complex large scale real life industrial
problems are used to illustrate the efficiency of the method.


\end{document}
