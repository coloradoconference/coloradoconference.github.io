\documentclass{report}
\usepackage{amsmath,amssymb}
\setlength{\parindent}{0mm}
\setlength{\parskip}{1em}
\begin{document}
\begin{center}
\rule{6in}{1pt} \
{\large Rakhim Aitbayev \\
{\bf Multilevel Preconditioners for Nonselfadjoint or Indefinite Orthogonal Spline Collocation Problems }}

Department of Mathematics \\ New Mexico Tech \\ Socorro \\ NM 87801
\\
{\tt aitbayev@nmt.edu}\\
 Bernard  Bialecki \end{center}

\newcommand{\GP}{\mbox{${\cal G}$}}\newcommand{\SH}{\mbox{$\scriptscriptstyle H^{2}(\Omega)$}}\newcommand{\SL}{\mbox{$\scriptscriptstyle L^{2}(\Omega)$}}We develop and study symmetric multilevel preconditioners forthe computation of the orthogonal spline collocation (OSC) solution of aDirichlet boundary value problem (BVP) with a nonselfadjoint or anindefinite operator.
The OSC solution is sought in the space ofpiecewise Hermite bicubic spline functions defined on a uniformpartition.
We consider an additive and a multiplicative multilevelpreconditioners that are used with the preconditioned conjugategradient (PCG) method.
Results and algorithms presented in this paperare closely related to those in [1],
[2],
[3],
and [4].Let $\Omega$ be a unit square $(0,1)\times(0,1)$ with the boundary$\partial\Omega$,
and let $x=(x_1,x_2)$.
We consider a BVP\begin{equation}\label{eq:bvp}Lu\equiv\sum_{i,j=1}^{2} a_{ij}(x)u_{x_ix_j}+\sum_{i=1}^{2} b_i(x)u_{x_i} +c(x)u = f(x),\;\;x\in\Omega,\quad u=0 \;\;\mbox{on}\;\;\partial\Omega.\end{equation}Operator $L$ could be non-selfadjoint or indefinite in $L^2$ innerproduct.
We assume that the principal part of $L$ satisfies theuniform ellipticity condition and that BVP (\ref{eq:bvp}) has aunique solution in $H^2(\Omega)$.Let $\pi_0$ be a uniform coarsest rectangular partition of $\Omega$.We obtain a set of partitions $\{\pi_k\}_{k=0}^{K}$ by standardcoarsening,
and let $V_0\subset V_1\subset\ldots\subset V_K\equiv V_h$be the set of corresponding nested spaces of piecewise Hermitebicubics that vanish on $\partial\Omega$.
Let $\sum$ denote thetwo-dimensional composite Gauss quadrature corresponding to partition$\pi_h$ with 4 nodes in each element.
Let $\GP_h$ denote thecorresponding set of Gauss points.
The OSC discretization of BVP(\ref{eq:bvp}) is defined by\begin{equation}\label{eq:osc}u_h\in V_h,\quad Lu_h(\xi)=f(\xi),\quad\xi\in\GP_h,\end{equation}and it can be written as the operator equation $L_hu_h=f_h$ in theHilbert space $V_h$ with the inner product $(v,w)_h=\sum vw$.Let $\{\psi_{k,j}\}_{j=1}^{N_k}$ be the standard finite element basisof $V_k$ consisting of products of one-dimensional value and slopebasis functions.
Using space decomposition\[V_h=V_0+\sum_{k=1}^{J}\sum_{j=1}^{N_k}V_{k,j},
\quad V_{k,j}= \mbox{span}(\psi_{k,j}),\]we define and study multilevel additive $B_{\textup{\scriptsize a}}$and multiplicative $B_{\textup{\scriptsize m}}$ preconditioners forsolving the normal equation $L_h^*L_hu_h=L_h^*f_h$,
where $L_h^*$isthe adjoint to $L_h$.
The implementation of $B_{\textup{\scriptsizea}}$ and $B_{\textup{\scriptsize m}}$ is based on relationshipsbetween basis functions for two consecutive partitions and theimplementation of $B_{\textup{\scriptsize m}}$ is similar to thatforV(1,1)-cycle with the Gauss-Seidel smoothing.
A problem on thecoarsest partition is assumed sufficiently small,
and it is solvedexactly.
The computational cost of the preconditioning algorithmsis$\mbox{O}(N_K)$.
The following is our main result.\textbf{Theorem.}{\em There are positive independent of $h$ and $K$ constants$\alpha_{\textup{\scriptsize a}}$,
$\beta_{\textup{\scriptsize a}}$,$\alpha_{\textup{\scriptsize m}}$,
and $\beta_{\textup{\scriptsize m}}$,such that}\begin{eqnarray}\label{eq:spectral_equiv}&&\alpha_{\textup{\scriptsize a}}\,
(B_{\textup{\scriptsize a}}v,v)_h\leq (L_h^*L_hv,v)_h \leq\beta_{\textup{\scriptsize a}}\,(B_{\textup{\scriptsize a}}v,v)_h,\quad v\in V_h,\\\nonumber&&\alpha_{\textup{\scriptsize m}}\,(B_{\textup{\scriptsize m}}v,v)_h\leq (L_h^*L_hv,v)_h \leq \beta_{\textup{\scriptsize m}}\,(B_{\textup{\scriptsize m}}v,v)_h,\quad v\in V_h.\end{eqnarray}To obtain these results,
we prove the key assumptions in the generaltheory of Schwarz methods presented in [4],
and use the inequalities\[C^{-1}\|v\|_{\SH}^2\leq a_h(v,v)\leq C\|\Delta v\|_{\SL}^2,\quad v\in V_h,\]obtained in [2].
We present numerical results that demonstratethe efficiency of our preconditioning algorithms.\bigskip\noindent\textbf{References}\medskip\noindent[1] {\sc R.~Aitbayev and B.~Bialecki},
{\em A preconditioned conjugategradient method for nonselfadjoint or indefinite orthogonal splinecollocation problems},
SIAM J.
Numer.
Anal.,
41 (2003),
pp.~589--604.\medskip\noindent[2]{\sc B.~Bialecki},
{\em Convergence analysis of orthogonalsplinecollocation for elliptic boundary value problems},
SIAM J.
Numer.Anal.,
35 (1998),pp.~617--631.\medskip\noindent[3]{\sc B.~Bialecki and M.~Dryja},
{\em Multilevel additive andmultiplicative methods for orthogonal spline collocation problems},Numer.
Math.,
77 (1997),
pp.~35--58.\medskip\noindent[4]{\sc B.~F.
Smith,
P.~E.
Bj{\o}rstad,
and W.~D.
Gropp},{\em Domain Decomposition: Parallel Multilevel Methods for Elliptic PartialDifferential Equations},
Cambridge University Press,
New York,
1996.

\end{document}
