\documentclass{report}
\usepackage{amsmath,amssymb}
\setlength{\parindent}{0mm}
\setlength{\parskip}{1em}
\begin{document}
\begin{center}
\rule{6in}{1pt} \
{\large Edmond Chow \\
{\bf An aggregation multilevel method based on smoothed error vectors}}

Lawrence Livermore National Laboratory \\ L-560 \\ Box 808 \\ Livermore \\ CA 94551
\\
{\tt echow@llnl.gov}\end{center}

We will present a new method for constructing the prolongation
operator in aggregation multilevel methods. Suppose a fine grid
has been partitioned by nodes into aggregates. For scalar
elliptic problems, the error is often represented (roughly) by a
piecewise constant function, i.e., the error is represented
by a constant function over each aggregate. This constant
function is called a basis function for the aggregate.

The performance of methods like the above can be improved
for various types of problems by increasing the number of
basis functions. This is particularly important when the dimension of the
near null-space of the operator is greater than unity, for example
in elasticity problems. For elasticity, the rigid body modes
are used as basis functions. For other problems, it is not clear
how to choose the basis functions, however, they are generally
chosen to be smooth functions, e.g., functions of the coordinates
of the grid nodes. Related current research has proposed using
low-energy eigenvectors of the local stiffness matrices associated
with each aggregate.

The prolongation operator $P$ should be able to represent, as well as possible,
slow-to-converge error. We thus propose the following method.
First, generate $m$ samples of algebraically smooth
error by applying the smoother to $Ax=0$ with a random initial guess.
For a given aggregate,
let $V$ denote the matrix of $k$ basis vectors being sought, and let $S$ denote
the matrix of sample vectors over that aggregate. We seek
\[
\min_{V,X} \| V X - S \| .
\]
The minimum is achieved when $VX$ is the rank-$k$ matrix nearest to $S$.
The matrix $V$ is thus the first $k$ left singular vectors
from the singular value decomposition of $S$. The singular values can
be used to select $k$. $V$ is used to construct $P$. Note that a
different $k$ may be used for each aggregate.

This technique produces basis vectors that are matrix-dependent.
In particular, anisotropies and physical jumps in the smoothed error
are reflected in the basis vectors. The method can easily be extended
to multiple levels. Our experiments show that for an arbitrarily
scaled linear elasticity problem, the method can perform as well
as if the scaled rigid body modes were known. The method can also
be used adaptively, by using V-cycles to generate the smooth
error vectors $S$.

This work was performed under the auspices of the U.S. Department of
Energy by University of California Lawrence Livermore National
Laboratory under Contract W-7405-Eng-48.


\end{document}
