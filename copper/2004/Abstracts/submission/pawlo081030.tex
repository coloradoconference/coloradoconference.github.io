\documentclass{report}
\usepackage{amsmath,amssymb}
\setlength{\parindent}{0mm}
\setlength{\parskip}{1em}
\begin{document}
\begin{center}
\rule{6in}{1pt} \
{\large Roger P. Pawlowski \\
{\bf Desiging a Flexible Nonlinear Solver Library.}}

Sandia National Laboratories \\ PO Box 5800 \\ MS-0316 \\ Albuquerque \\ NM 87185-0316
\\
{\tt rppawlo@sandia.gov}\\
Tamara G. Kolda\\
Russell Hooper\\
	John N. Shadid\end{center}

Designing a nonlinear solver library with the flexiblity to support multiple applications with robust,
cutting-edge algorithms presents a number of difficulties.
Not only must the library supply a large number of solution algorithms,
but it must also account for the many application specific optimizations that production-level codes require.
This presentation will discuss the design of NOX,
an object-oriented C++ library being developed to support applications at Sandia National Laboratories.
We will discuss the design requirements that the nonlinear solver library must meet and how NOX handles those issues.
Specifics we plan to address include: \\1.
Desiging a flexible environment to allow for the efficient introduction of new algorithms.
\\2.
Defining a flexible methodology for convergence and failure criteria of the nonlinear algorithms (and allowing users to supply their own criteria).
\\3.
Abstracting the linear solvers and the linear algebra storage format from the nonlinear algorithm.
This allows applications to use their own specialized algorithms for optimal performance.
\\4.
Allowing users to pass objects/arguments through the code efficiently.
\\5.
Library portability.
\\Examples will be presented from large-scale engineering applications developed at Sandia National Laboratories including reacting flows (MPSalsa),
compressible flows (Premo),
and electrical circuit modeling (Xyce).
Comparisons of the nonlinear globalization algorithms will also be briefly discussed.

\end{document}
