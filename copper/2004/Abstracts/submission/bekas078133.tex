\documentclass{report}
\usepackage{amsmath,amssymb}
\setlength{\parindent}{0mm}
\setlength{\parskip}{1em}
\begin{document}
\begin{center}
\rule{6in}{1pt} \
{\large Constantine Bekas \\
{\bf AMLS and Spectral Schur Complements}}

EE/CS Bld. \\ 200 Union St. SE \\ Computer Science and Engineering Dept. \\ University of Minnesota \\ 55455 \\ Minneapolis \\ MN
\\
{\tt bekas@cs.umn.edu}\\
Yousef Saad\end{center}

In the last few decades, Krylov projection methods such as the Lanczos
algorithm and its variants, have dominated the scene of algorithms for
eigenvalue problems. Recently, an alternative approach has emerged in
structural engineering as a competitor to the standard shift-and-invert
Lanczos approach. The algorithm, called Automated Multilevel
Substructuring method ({\tt AMLS}) is rooted in a domain
decomposition framework. It has been reported as being capable of
computing thousands of the smallest normal modes of dynamic structures on
commodity workstations and of being orders of magnitude faster standard
methods.

A theoretical framework for {\tt AMLS} was recently presented by Benighof
and Lehoucq in from the point of view of domain decomposition, using
adequate functional spaces and operators on them. The goal of this talk
is to present a complementary viewpoint, which is entirely algebraic.
{\tt AMLS} is essentially a
Schur complement method. Schur complement techniques are well understood
for solving linear systems and play a major role in Domain Decomposition
techniques. Relatively speaking, the formulation of this method for
eigenvalue problems has been essentially neglected so far. One could of
course extend the approach used for linear systems
in order to compute eigenvalues, by formulating a Schur complement
problem for each different eigen-pair, (e.g., by solving the eigenvalue
problem as a sequence of linear systems through shift-and-invert). This
viewpoint was considered quite early on by Abramov and Chichov who
presented what may termed a spectral Schur complement method. It can
easily be verified that a scalar $\lambda$ is an eigenvalue of a matrix
$A$ partitioned as
\[ A = \begin{pmatrix} B & F \cr E & C \end{pmatrix}, \]
if and only if it is an eigenvalue of $S(\lambda) = C - E (B - \lambda I)
{\rm inv} F $ (this is clearly restricted to those $\lambda $'s that are
not in the spectrum of $B$). This nonlinear eigenvalue problem may be
solved by a Newton-type approach. Alternatively, one can also devise
special iterative schemes based on the above observation. An approach of
this type is clearly limited by the fact that a Schur complement (or
several consecutive ones in an iterative process) is required for each
different eigenvalue. It can, however, work well for computing one, or a
few, eigenvalues or in some other special situations. For example, this
nonlinear viewpoint led to the development of effective
shifts of origin for the QR algorithm for tridiagonal matrices.

The fundamental premise of {\tt AMLS}, and its attraction, is that it is
capable of extracting very good approximations to a large number of the
smallest eigenvalues {\it with only one Schur complement}. To achieve
this, {\tt AMLS} relies on clever projection techniques. It builds good
bases from one Schur complement, and expands them in an effective way to
bigger and bigger domains.

In this talk we adopt a purely algebraic viewpoint and demonstrate that
{\tt AMLS} can be viewed as a method which exploits a first order
approximation to a nonlinear eigenvalue problem in order to extract a
good subspace for a Rayleigh-Ritz projection process. This technique
leads to approximations from a single Schur complement derived from a
domain decomposition of the physical problem. Exploiting this
observation, we have devised several possible enhancements in two main
directions. The first introduces Krylov subspaces to the technique, and
the second considers a more accurate (second order instead of first
order) scheme, which is based on a quadratic eigenvalue problem. Finally,
combinations of the above two strategies have been considered with a goal
of enhancing robustness.
Currently, {\tt AMLS} is a one-shot algorithm in the sense that certain
approximate eigenvectors are build from the last level up to the highest
level and no further refinements are made. The current framework does
iteratively refine these approximations. We will
discuss this issue and will explore the feasibility of an iterative
scheme based on {\tt AMLS}.


\end{document}
