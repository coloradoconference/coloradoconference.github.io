\documentclass{report}
\usepackage{amsmath,amssymb}
\setlength{\parindent}{0mm}
\setlength{\parskip}{1em}
\begin{document}
\begin{center}
\rule{6in}{1pt} \
{\large Allison H. Baker \\
{\bf On improving linear solver performance: a block variant of GMRES}}

Center for Applied Scientific Computing \\ Lawrence Livermore National Laboratory  \\ Box 808 \\ L-551  \\ Livermore \\ CA 94551
\\
{\tt abaker@llnl.gov}\\
J.M. Dennis\\
E.R. Jessup\end{center}

Two approaches to improving the performance, i.e. time to solution, of
an iterative linear solver algorithm are particularly viable. First,
algorithmic changes that improve convergence properties result in
faster convergence due to fewer overall floating-point operations.
Second, modifications to an algorithm that reduce the movement of data
through memory greatly impact performance because of the growing gap
between CPU floating-point performance and memory access time.
Ideally, a balance is achieved between improving the efficiency of an
iterative linear solver from a memory-usage standpoint and maintaining
favorable numerical properties. In this talk, we discuss the
restarted generalized minimum residual (GMRES) method in the context
of both approaches to improving performance. In particular, we
present an alternative to the standard restarted GMRES algorithm for
solving a single right-hand side linear system $Ax=b$ based on solving
the block linear system $AX=B$. Additional starting vectors and
right-hand sides are chosen to accelerate convergence.


\end{document}
