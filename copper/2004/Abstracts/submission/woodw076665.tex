\documentclass{report}
\usepackage{amsmath,amssymb}
\setlength{\parindent}{0mm}
\setlength{\parskip}{1em}
\begin{document}
\begin{center}
\rule{6in}{1pt} \
{\large Carol Woodward \\
{\bf Newton-Krylov Methods for Expensive Nonlinear Function Evaluations}}

Center for Applied Scientific Computing \\ P.O. Box 808 \\ L-561 \\ Livermore \\ CA  94551
\\
{\tt cswoodward@llnl.gov}\\
Peter Brown\\
Homer Walker\\
	Rebecca Wasyk\end{center}

Newton-Krylov methods have proven useful for solving large scale
nonlinear systems. An advantage of these iterative methods is that they
do not require storage of the system Jacobian, but only require knowledge
of how the Jacobian acts on a vector. A difference quotient evaluated at
each linear iteration is often used to approximate this action without
slowing the convergence rate of the method. For systems with expensive
nonlinear function evaluations, however, the requirement of a function
evaluation for each linear iteration can result in a very costly
computation. We present results exploring convergence rates and time
savings associated with using different approximations to the system
function in the difference quotient.

This work was performed under the auspices of the U.S. Department of
Energy by University of California Lawrence Livermore National Laboratory
under contract No. W-7405-Eng-48.


\end{document}
