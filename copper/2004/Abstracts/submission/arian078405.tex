\documentclass{report}
\usepackage{amsmath,amssymb}
\setlength{\parindent}{0mm}
\setlength{\parskip}{1em}
\begin{document}
\begin{center}
\rule{6in}{1pt} \
{\large Eyal Arian \\
{\bf Preconditioining Techniques in TRANAIR: an Industrial Application}}

Mathematics & Engineering Analysis \\ The Boeing Company \\ MC 7L-21         \\ PO Box 3707 \\ Seattle \\ WA 98124-2207
\\
{\tt eyal.arian@boeing.com}\\
David P. Young\\
Robin G. Melvin\\
	Hilmes Craig, Johnson Forrester T.\end{center}

TRANAIR is a general geometry computational fluid dynamics
(CFD) tools used by Boeing Commercial
Airplanes for aerodynamic design, optimization,
and analysis and is run thousands of times a year.
It is applied to large scale problems containing millions of
variables and is Boeing Commercial
main CFD ``work horse''. It can be used to analyze or design aerodynamic
configurations using the
nonlinear full potential equation with a coupled integral boundary layer.
In the far field the
full potential equation reduces to the linear Prandtl-Glauert equation.
Unsteady TRANAIR computes the effects of a linear harmonic perturbation
about the steady velocity field,
representing a flow solution about an aircraft configuration.
In the far field the latter reduces to
the convective Helmholtz equation.

For exterior flow problems the far field boundary condition is
a non-trivial issue that needs careful care.
To achieve high accuracy a charge formulation is implemented in TRANAIR.
Assuming the flow is linear at the far field, the charge formulation
amounts to having the far field
boundary at infinity. The variables are split into two major sets:
global grid $(X)$ and
refined grid $(Y)$ variables. The global grid variables are transformed
to charge variables by $X=T^{-1} Q$,
where $T$ is a linear operator that is fast to invert, and $Q$ is
different than zero only where the non
linear operator, $L$, is not equal to $T$. This approach is
reminiscent of a two level multigrid method
where $Q$ and $T$ are the ``coarse grid'' variable and operator respectively.
An inexact Newton solver in combination with GMRES
(Generalized Minimum Residual Method) is applied to
this problem taking advantage of the fact that it is very inexpensive
to multiply a
vector of $Q$ by $T^-1$ and by $L$.
The calculation of the preconditioned residual, $R$,
is computed by $R=N^{-1}\times L\times T^{-1}\times Q,Y]$. The left
preconditioning matrix, $N$,
is approximating $L$ excluding the far field. It is large and sparse and
therefore it is feasible to apply a
direct sparse incomplete factorization of $N$. A drop tolerance can be
introduced into the sparse
elimination process allowing small elements in the decomposition to be
dropped as they are generated.
A grid based nested dissection ordering is generated, which reduces
fill during elimination.

Most preconditioning techniques assume a given matrix that
represents the linear system at hand.
One difficulty with the far field charge formulation approach is that the
full Jacobian matrix of the
linearized system is not accessible, limiting the scope of
preconditioning opportunities (such as right preconditioning).
Recently, a new approach has been developed in which the charges
are eliminated and instead a discrete integral equation is
applied at the far field.
That equation was obtained by calculating the discrete
Green's identity in the domain between
the computational box and infinity using the discrete Greens function.
The resulting Jacobian
matrix is mostly sparse but significantly denser at the far field.
Given a uniform $N^3$ grid in
$3$ space dimensions, $O(N^2)$ of the Jacobian rows will have
$O(N^2)$ elements while the remaining rows
will have $O(1)$ elements.


In the talk recent preconditioning results using TRANAIR will be presented,
focusing on the role of the far field condition in exterior flow.
Among the issues that are addressed are right and left preconditioning,
and reordering schemes to account for the
denser parts of the Jacobian matrix.


\end{document}
