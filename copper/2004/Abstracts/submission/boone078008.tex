\documentclass{report}
\usepackage{amsmath,amssymb}
\setlength{\parindent}{0mm}
\setlength{\parskip}{1em}
\begin{document}
\begin{center}
\rule{6in}{1pt} \
{\large Tim Boonen \\
{\bf A multigrid method for the 3D magnetostatic Maxwell equations with solenoidal smoothing}}

Celestijnenlaan 200 A \\ B-3001 Leuven \\ BELGIUM
\\
{\tt tim.boonen@cs.kuleuven.ac.be}\\
Stefan Vandewalle\end{center}

We consider magnetostatic problems formulated as the curlcurl-equation
and discretized using first order edge elements. The corresponding system
matrix has a very large kernel.
Components in this kernel disturb the effectiveness of multigrid when
classical pointwise smoothers are used.
\\
The hybrid smoother of R. Hiptmair offers a solution to this problem.
This smoother performs additional Gauss-Seidel smoothing in the
nodal-based gradient space, hereby damping the kernel-components.
Gradient space is accessible as the solution
space of the Galerkin product matrix $G^TAG$, with $G$ being the discrete
primal gradient matrix with respect to the topology at hand. If the
multigrid hierarchy is constructed such that the kernels are nested, then
application of the hybrid smoother on all levels gives good results. R.
Hiptmair used this hybrid smoother in a geometric multigrid algorithm
[1], S. Reitzinger and J. Sch\"oberl integrated it in an algebraic
multigrid algorithm [2].
\\
In this talk, a smoother will be presented that only operates on the
orthogonal complement of the kernel, leaving the kernel itself untouched.
This is achieved by Gauss-Seidel smoothing on facet-based solenoidal
space. Solenoidal space is accessible as the solution space of the
Galerkin product matrix $\tilde{R}^T A \tilde{R}$, with $\tilde{R}$ being
the dual curl matrix with respect to the topology of the problem at hand.
This smoother can be applied succesfully on all levels of a multigrid
hierarchy, if the orthogonal complements of the kernels are nested.
\vspace*{5mm}
\\
REFERENCES:
\begin{enumerate}
\item R. Hiptmair, \emph{Multigrid method for Maxwell's equations}, SIAM
J.Numer.Anal. 36(1999), no.1,
204-255.
\item S. Reitzinger, J. Sch\"oberl, \emph{An algebraic multigrid method
for finite element discretizations
with edge elements}, Numer.Lin.Alg., 9(2002), 223-238.
\end{enumerate}


\end{document}
