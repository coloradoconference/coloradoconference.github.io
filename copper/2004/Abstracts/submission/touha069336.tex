\documentclass{report}
\usepackage{amsmath,amssymb}
\setlength{\parindent}{0mm}
\setlength{\parskip}{1em}
\begin{document}
\begin{center}
\rule{6in}{1pt} \
{\large Ahmed TOUHAMI \\
{\bf A comparative study of iterative solvers exploiting spectral information for SPD systems}}

ENSEEIHT-IRIT \\ 2 rue C. Camichel \\ 31071 Toulouse Cedex \\ France
\\
{\tt Ahmed.Touhami@enseeiht.fr}\\
Luc GIRAUD\\
Daniel RUIZ\end{center}


When solving the Symmetric Positive Definite (SPD) linear system
$${\bf A x}= {\bf b} $$
with the Conjugate Gradient (CG) method, the smallest eigenvalues of the
matrix $A$ often slow down the convergence.
This is highlighted by the bound on the rate of convergence of CG given by
$$ ||{\bf e}^{(k)}||_{\bf A} \le {\bf 2}\,||{\bf e}^{(0)}||_{\bf A}
\displaystyle\Bigg(\frac{\sqrt \kappa -1}{\sqrt \kappa + 1}\Bigg)^k,$$
where ${\bf e}^{(n)}= {\bf x}^* - {\bf x}^{(n)}$ denotes the forward
error associated with the iterate at step $k$, $\displaystyle \kappa =
\kappa({\bf A}) = \lambda_{max} / \lambda_{min}$ denotes the condition
number of ${\bf A}$ and $||{\bf x}||_{{\bf A}} = ({\bf x}^T{\bf A}{\bf
x})^{1/2}$ denotes the ${\bf A-}$norm of ${\bf x}$. \\
From this bound, it can be seen that increasing the size of the smallest
eigenvalues will improve the convergence rate of CG.
Consequently if the smallest eigenvalues of ${\bf A}$ could be somehow
``removed'' the convergence will be improved.
This observation is still of course true when a preconditioner is used and some
extra techniques might be investigated to improve the convergence rate of
CG on the preconditioned system.
Several techniques have been proposed in the literature that either consists of
updating the preconditioner or enforcing CG to work in the orthogonal of
invariant subspace associated with small eigenvalues.
The aim of this work it to compare several of these techniques in terms
of numerical efficiency and computational complexity.
Among these approaches we consider first the two-phase algorithm recently
proposed in~\cite{arru:02}.
In a first stage this algorithm computes a partial spectral decomposition
simply using matrix-vector products.
More precisely it combines Chebyshev iterations with a block Lanczos
procedure to accurately compute an orthogonal basis of the invariant
subspace associated with the smallest eigenvalues of $\Am$.
Then, the solution on this small subspace is computed using a direct
solver while the solution in the orthogonal complement is obtained with
Chebyshev iterations that benefits from the reduced condition number.\\
For sake of comparison this eigen-information is used in combination with
other techniques.
In particular we consider the deflated version of conjugate gradient
described in~\cite{syeg:00}.
This algorithm exploits the link between the Lanczos algorithm and the
standard conjugate gradient algorithm, and is mathematically equivalent
to Nicolaide's algorithm which is derived from a ``deflated'' Lanczos
procedure~\cite{nico:87}.\\
As representative of techniques exploiting the spectral information to
update the preconditioner we consider the approaches proposed
in~\cite{cadg:03a}~and~\cite{navu:03} that attempts to shift the smallest
eigenvalues close to one where most of the eigenvalues of the
preconditioned matrix should be located.
In this talk, we will describe theses various variants as well as the
observed numerical behaviour on a set of model problems from Matrix
Market or arising from the discretization via finite element technique of
some heterogeneous diffusion PDEs.
We will discuss their numerical efficiency, computational complexity and
sensitivity to the accuracy of the eigencalculation.
Perspectives for future work will also be debated.
\begin{thebibliography}{99}
\bibitem{arru:02}
{M.~Arioli and D.~Ruiz}.
{A {C}hebyshev-based two-stage iterative method as an alternative to the
direct solution of linear systems}.
{Technical Report RAL-TR-2002-021, Rutherford Appleton Laboratory, Atlas
Center, Didcot, Oxfordshire, OX11 0QX, England, 2002}.

\bibitem{cadg:03a}
{B.~Carpentieri,~I.~S.~Duff and L.~Giraud}.
{A class of spectral two-level preconditioners}.
{\em SIAM {J}ournal on {S}cientific {C}omputing}, 25:{749--765}, 2003.

\bibitem{navu:03}
{R. Nabben and C. Vuik}.
{A comparison of deflation and coarse grid correction applied to porous media flow}.
{Delft University of Technology, Department of Applied Mathematical
Analysis}, {Report 03-10}, {Delft}, 2003.

\bibitem{nico:87}
{R.~Nicolaides}.
{Deflation of conjugate gradients with applications to boundary value problems}.
{\em SIAM Journal on Numerical Analysis}, 24:{355-365}1987.

\bibitem{syeg:00}
{Y.~Saad,~ M.~Yeung,~J.~Erhel~and~F.~Guyomarc'h}.
{A deflated version of the conjugate gradient algorithm}.
{\em SIAM Journal on Scientific Computing}, 21:{1909-1926}, 2000.
\end{thebibliography}


\end{document}
