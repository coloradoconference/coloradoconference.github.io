\documentclass{report}
\usepackage{amsmath,amssymb}
\setlength{\parindent}{0mm}
\setlength{\parskip}{1em}
\begin{document}
\begin{center}
\rule{6in}{1pt} \
{\large Bert W. Seynaeve \\
{\bf Stochastic iterative methods for PDEs with random parameters}}

Departement Computerwetenschappen \\ Celestijnenlaan 200 A \\ 3001 Heverlee \\ Belgium
\\
{\tt bert.seynaeve@cs.kuleuven.ac.be}\\
Stefan Vandewalle\\
Bart Nicolai\end{center}

We consider the numerical solution of elliptic or parabolic
partial differential equations with stochastic coefficients.
Such equations appear in reliability problems. Various
approaches exist for dealing with the uncertainty propagation
question: Monte Carlo methods, perturbation techniques, variance
propagation, etc. Here, we deal with the stochastic finite
element method (SFEM) [1]. This method transforms a system of
PDEs with stochastic parameters into a stochastic linear system
$Ku = f$, by means of a finite element Galerkin discretization.
In the classical SFEM-approach, the finite element solution
vector $u$ is approximated by a linear combination of
deterministic vectors; the coefficients are orthogonal
polynomials in the random variables that occur in $K$ and $f$.
Unlike commonly used methods such as the perturbation method,
SFEM gives a result that contains all stochastic characteristics
of the solution. It also improves Monte Carlo methods
significantly because sampling can be done after solving the
system of PDEs.

Here, we will show that applying a linear iterative solver
(e.g. Gauss-Seidel) to the discretized stochastic system also
allows rational instead of polynomial approximations. The
computational efficiency can then be improved by using the
chosen iterative method as a smoother within a multigrid
context, adapting classical ingredients of multigrid methods to
the case of rational expressions in random variables. The
coefficients in the resulting expressions can differ slightly,
however, depending on the amount of information, concerning the
random variables, that is taken into account. We will also show
how the method can be used efficiently in the case of large
stencil discretizations.


References:

[1] R.G. Ghanem and P.D. Spanos. Stochastic finite elements:
a spectral approach. Springer-Verlag, New York, 1991.


\end{document}
