\documentclass{report}
\usepackage{amsmath,amssymb}
\setlength{\parindent}{0mm}
\setlength{\parskip}{1em}
\begin{document}
\begin{center}
\rule{6in}{1pt} \
{\large Vincent A. Mousseau \\
{\bf A Comparison Between an Implicitly Balanced Solution and a Linearized and Operator Split Solution of the Thermal Hydraulic Equations.}}

T-3 \\ Fluid Dynamics \\ MS B216 \\ Los Alamos National Laboratory \\ Los Alamos \\ NM  87545
\\
{\tt vmss@lanl.gov}\\
Dana A. Knoll\end{center}

In this presentation we study time integration and solver requirements
for a simplified model of the cooling system of a nuclear power reactor.
The model includes a one-dimensional, two-phase flow model (six
equations) coupled to a two-dimensional nonlinear heat conduction model
(one equation). This physical model includes fast time scales that
account for the mass, momentum, and energy exchange between the two
phases (water and steam), and for the momentum and energy exchange
between both phases and the wall. The two-dimensional nonlinear heat
conduction in the wall accounts for the slow time scale in the problem
which is related to the rate that the wall accepts or rejects energy from
the water and steam.

Results will be presented that will show that the traditional linearized
and operator split solution method can be employed as an effective and
efficient preconditioner to an implicitly balanced solution method
(Jacobian-Free Newton-Krylov). Results will also compare the accuracy of
these two approaches when the rates of the fast time scales and the slow
time scales are varied.

The traditional linearized and split solution methods employ time step
control algorithms based on the stability of the algorithm. Since the
implicitly balanced approach is unconditionally stable, a new time step
control algorithm will be presented for this approach that is designed to
address the accuracy of the solution.


\end{document}
