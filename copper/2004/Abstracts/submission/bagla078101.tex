\documentclass{report}
\usepackage{amsmath,amssymb}
\setlength{\parindent}{0mm}
\setlength{\parskip}{1em}
\begin{document}
\begin{center}
\rule{6in}{1pt} \
{\large James Baglama \\
{\bf Solving Eigenvalue and SVD Problems with Augmented Krylov Subspaces}}

Department of Mathematics  \\ University of Rhode Island \\ Kingston \\ Rhode Island 02881
\\
{\tt jbaglama@math.uri.edu}\end{center}

Augmenting the Krylov subspace with Ritz vectors has proven
to be an efficient and equivalent implementation of Sorensen's
(1992)Implicitly Restarted Arnoldi (IRA) method for solving
eigenvalue problems. We have applied the method of augmenting
to develop a block Householder eigenvalue algorithm and a SVD
algorithm. For the eigenvalue problem, we have developed a
block Householder implicitly restarted Arnoldi method. This method
maintains strong orthogonality and utilizes level 3 BLAS
matrix-matrix products. Restarting is implemented by augmentation of
Krylov subspaces. The SVD algorithm computes a sequence of partial
Lanczos bidiagonalizations of A with judiciously chosen initial vectors.
Restarting is also implemented by augmentation of Krylov subspaces. Both
methods were derived by adapting Wu and Simon's (2001) approach for
solving symmetric eigenvalue problems. MATLAB codes are available for
both methods.


\end{document}
