\documentclass{report}
\usepackage{amsmath,amssymb}
\setlength{\parindent}{0mm}
\setlength{\parskip}{1em}
\begin{document}
\begin{center}
\rule{6in}{1pt} \
{\large Martin J. Gander \\
{\bf Domain Decomposition Methods with Convergence Rates Faster than Multigrid}}

Dept. of Mathematics and Statistics \\ McGill University \\ 805 Sherbrooke Street West \\ Montreal \\ QC \\ H3A 2K6
\\
{\tt mgander@math.mcgill.ca}\end{center}

Most large scale simulations, from aircraft carriers to Jumbo-Jets, are
only possible with parallel computers. Often codes are available for
partial problems, like the wing or the engine of the plane. A natural
paradigm to combine such codes to simulate the entire aircraft is to use
domain decomposition techniques. But the performance of these techniques
depends very much on the strength of the physical coupling between the
pieces of the model.

The Schwarz methods are a class of domain decomposition methods. They are
based on a theoretical tool to prove existence and uniqueness, invented
by Schwarz in 1869, have been investigated in detail over the last two
decades and are now well understood, see for example the survey articles
by Chan and Mathew, Xu, or the book by Smith, Bj{\o}rstad, and Gropp, and
the book by Quarteroni and Valli. Optimal convergence results exist for
the Schwarz methods in the sense that the condition number of the
preconditioned system is independent of (or only weakly dependent on) the
mesh parameter and the number of subdomains. Thus asymptotically Schwarz
methods have optimal scalability.

These optimality results contain however constants which remain unknown
in the analysis. Thus they do not imply that the current Schwarz methods
have optimal performance. They do not guarantee either that domain
decomposition methods are competitive to other parallel methods. Thus the
word "optimal" can be misleading. Indeed a comparison by Gaertel and
Kessel in 1992 of an "optimal" domain decomposition method with a simple
multi-grid algorithm implemented in parallel showed that, although the
domain decomposition algorithm scales optimally, the parallelized
multi-grid algorithm is more than an order of magnitude faster on 9 and
16 processors for an elliptic model problem. So why are the Schwarz
methods so slow ? Our analysis reveals that the main reason are the
transmission conditions employed at the artificial interfaces between the
subdomains. The classical Schwarz algorithm uses Dirichlet transmission
conditions to exchange information between subdomains, which is fatal for
the performance. But fast convergence was not of interest to Schwarz, who
used his tool only to prove existence and uniqueness of solutions, and
not to actually compute them.

Instead of Dirichlet transmission conditions one should use transmission
conditions which decouple the subdomain problems as much as possible. And
precisely this is possible: there are well known boundary conditions to
truncate infinite domains when the corresponding problem needs to be
solved on a finite computer, namely the transparent or absorbing boundary
conditions. They decouple as well as possible the exterior problem (which
is then not even solved) from the interior, computational one.
Fundamental contributions for hyperbolic problems can be found in an
early paper by Enquist and Majda from 1977, and for the case of parabolic
problems in the work by Halpern in 1986. Optimized Schwarz methods use,
instead of the classical transmission conditions of Dirichlet type,
absorbing or approximately absorbing transmission conditions. The basic
algorithm stays the same, only the information which is exchanged between
subdomains is replaced by physically more valuable information, which
decouples the subdomain problems much more effectively. The impact of
this small change on the performance is dramatic. If one uses absorbing
boundary conditions to exchange information, optimal convergence results
limited only by the physics of the problem can be achieved: convergence
is reached in a finite number of steps related to the number of
subdomains used, a result first derived for elliptic problems and a
special decomposition by Nataf et al. in 1995. But simple local
approximations to the absorbing boundary conditions suffice already to
speed up the algorithms by orders of magnitudes.

Optimized Schwarz methods are designed to weaken the coupling between
subdomain problems, even if the physical coupling is strong, and use
transmission conditions which take the physics of the underlying problem
into account. I will present in this talk an analysis and numerical
experiments for a symmetric positive definite model problem to illustrate
the dramatic change in the convergence rate of optimized Schwarz methods
compared to classical ones. The optimized methods converge often with an
order of magnitude less iterations than classical Schwarz methods at the
same cost per iteration, and attain contraction rates which are
comparable to those of multigrid methods.


\end{document}
