\documentclass{report}
\usepackage{amsmath,amssymb}
\setlength{\parindent}{0mm}
\setlength{\parskip}{1em}
\begin{document}
\begin{center}
\rule{6in}{1pt} \
{\large Kathleen R. Fowler \\
{\bf Optimal Groundwater Remediation Design}}

Clarkson University \\ Department of Mathematics and Computer Science \\ Box 5815 \\ Potsdam \\ NY 13676
\\
{\tt kfowler@clarkson.edu}\\
C.T. Kelley\\
C ass T. Miller\end{center}

We describe how optimization algorithms can aid in the decision making
process of cleaning up a contaminated groundwater site. These problems
are challenging in that optimization is simulation based and often
gradient information is not available. Objectives functions and
constraints can be nonsmooth, nonconvex, and discontinuous and we are
investigating the use of sampling methods on this class of problems. We
describe a contaminant plume capture problem proposed in the literature
for benchmarking purposes for the optimization community and present
numerical results obtained with a sampling method called implicit
filtering. Implicit filtering follows a projected quasi-Newton method
using finite difference gradients and reduces the difference increment as
the optimization progresses.


\end{document}
