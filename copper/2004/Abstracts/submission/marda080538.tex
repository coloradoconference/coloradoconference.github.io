\documentclass{report}
\usepackage{amsmath,amssymb}
\setlength{\parindent}{0mm}
\setlength{\parskip}{1em}
\begin{document}
\begin{center}
\rule{6in}{1pt} \
{\large Kent-Andre Mardal \\
{\bf Block Preconditioners for PDEs Arising from the Bidomain Model of the Electrical Activity in the Heart}}

Kent-Andre Mardal \\ P.O. Box 134 \\ NO-1325 Lysaker \\ Norway
\\
{\tt kent-and@simula.no}\\
Aslak Tveito\end{center}


\renewcommand{\AA}{\mathcal{A}}
\newcommand{\BB}{\mathcal{B}}
\newcommand{\xx}{\mathcal{x}}
\newcommand{\bb}{\mathcal{b}}






\paragraph{Introduction.}

In this abstract we will study three different (block) preconditioners for
two alternative formulations of the Bidomain model.
The Bidomain model describes the electrical activity in
the heart and is of importance to understand cardiac
diseases, which are a major problem in particular in
the western world. Here, we will study preconditioners
for the following linearized version of the Bidomain model,
\begin{eqnarray}
\label{first:1}
v_t &=& \nabla\cdot(M_i\nabla v)+\nabla\cdot(M_i\nabla u),\\
\label{first:2}
0 &=& \nabla\cdot(M_i\nabla v)+\nabla\cdot((M_i+M_e)\nabla u),
\end{eqnarray}
where $u$ and $v$ are the extracellular potential and the transmembrane
potential, respectively,
and $M_i$ and $M_e$ are conductivity tensors in the heart.
An alternative formulation can be obtained by expressing the
equations with the intracellular potential, $w = u +v $ instead of $v$.
These equations are,
\begin{eqnarray}
\label{second:1}
w_t &=& \nabla\cdot(M_i\nabla w)+ u_t ,\\
\label{second:2}
u_t &=& w_t+\nabla\cdot(M_e\nabla u).
\end{eqnarray}
The problems \eqref{first:1}-\eqref{first:2}
and \eqref{second:1}-\eqref{second:2} must
be equipped with suitable boundary and initial
conditions.
More details on these equations can
be found in, e.g. \cite{Pennacchio} and \cite{block-mg}.



The two alternative formulations have the
same solution, but give rise to two different
linear systems to be solved. The first one
reads,
\begin{equation}
\begin{pmatrix}
I+\frac{\Delta t}{2}A_i & \frac{\Delta t}{2}A_i\\
\frac{\Delta t}{2}A_i &\frac{\Delta t}{2}A_{i+e}
\end{pmatrix}
\begin{pmatrix}
v^n \\ u^n
\end{pmatrix}
=
\begin{pmatrix}
b^{n-1} \\ c^{n-1}
\end{pmatrix} .
\label{first:matrix}
\end{equation}
And the second linear system is,
\begin{equation}
\begin{pmatrix}
I+\frac{\Delta t}{2}A_i & -I \\
-I & I + \frac{\Delta t}{2}A_{e}
\end{pmatrix}
\begin{pmatrix}
w^n \\ u^n
\end{pmatrix}
=
\begin{pmatrix}
b^{n-1} \\ c^{n-1}
\end{pmatrix} .
\label{second:matrix}
\end{equation}
We have used a Crank-Nicholson scheme in time and
a finite element discretization in space.
Hence, $I$ is the mass matrix, while the various versions of $A_\alpha$,
where $\alpha=i,e,i+e$,
are "similar" to a discrete Laplacian, because the $M$ tensors
are positive definite and bounded. Notice further that
preconditioners for $I$, $A_\alpha$ and
$I + \frac{\Delta t}{2}A_\alpha$ are off-the-shelves
preconditioners that are order optimal
(with respect to both $h$ and $\Delta t$). These preconditioner can be
made by e.g., multigrid
or domain decomposition.



We will investigate the efficiency of the three following (block) preconditioners for
both formulations of the Bidomain model,
\begin{eqnarray*}
\BB_J^{-1} &=&
\begin{pmatrix}
A & 0\\
0 & D
\end{pmatrix}, \quad \mbox{block Jacobi}, \\
\BB_{GS}^{-1} &=&
\begin{pmatrix}
A & 0\\
C & D
\end{pmatrix}, \quad \mbox{block Gauss-Seidel (GS)}, \\
\BB_{SGS}^{-1} &=&
\begin{pmatrix}
A & 0\\
C & D
\end{pmatrix}
\begin{pmatrix}
A^{-1} & 0\\
0 & D^{-1}
\end{pmatrix}
\begin{pmatrix}
A & B\\
0 & D
\end{pmatrix}, \ \mbox{symmetric block GS}.
\end{eqnarray*}

To check the efficiency of the preconditioner we measure the condition
number of the preconditioned matrix, $\BB \AA$ .
The condition number is computed via singular value decomposition.


\paragraph{Numerical Experiments.}
In all the experiments we have partioned the unit square into
bilinear elements with size $h$ and used the standard
Galerkin finite element method to generate the matrices.

In Table \ref{first:exact} and \ref{second:exact} the three different
exact preconditioners are checked for the two alternative formulations.
It seems that the preconditioners for the second formulation \eqref{second:matrix}
are the best.
However, the condition numbers seem to increase as $\Delta t$ decreases. In contrast,
the condition numbers for the first formulation \eqref{first:matrix} seem
to be bounded independent of both $h$ and $\Delta t$.
In Table \ref{first:mg} and \ref{second:mg} we investigate this behavior further,
but we use a multigrid preconditioner instead of the exact preconditioner,
due to the large number of
unknowns. The condition number is estimated as a bi-product of the
Conjugate-Gradient method as described in [where-ever-this-was].
The Conjugate-Gradient method is stopped when the relative residual
is less than $10^{-18}$.
The condition number for the preconditioned matrix using the second formulation
is clearly dependent on $\Delta t$. Therefore, it seems that the first formulation
should be applied if $\Delta t$ is small, while the second formulation is
better when $\Delta t$
is large.



\begin{table}[h]
\begin{center}
\begin{tabular}{|c||c|c|c|c||c|c|c|c||c|c|c|c||} \hline
Prec & \multicolumn{4}{c|}{$\BB_J$}& \multicolumn{4}{c|}{$\BB_{GS} A $} &
\multicolumn{4}{c|}{$\BB_{SGS} A $}\\ \hline
$h\backslash \Delta t $ &
$2^{-2}$ & $2^{-3}$ & $2^{-4}$ & $2^{-5}$ &
$2^{-2}$ & $2^{-3}$ & $2^{-4}$ & $2^{-4}$ &
$2^{-2}$ & $2^{-3}$ & $2^{-4}$ & $2^{-5}$ \\ \hline \hline
$2^{-1}$ & 5.0 & 4.3 & 3.4 & 2.7 & 2.3 & 2.2 & 2.0 & 1.9 & 1.9 & 1.7 & 1.5 & 1.3 \\
$2^{-2}$ & 6.2 & 6.0 & 5.7 & 5.3 & 2.6 & 2.6 & 2.6 & 2.5 & 2.1 & 2.1 & 2.0 & 1.9 \\
$2^{-3}$ & 6.3 & 6.3 & 6.2 & 6.1 & 2.6 & 2.7 & 2.7 & 2.7 & 2.2 & 2.1 & 2.1 & 2.1 \\
$2^{-4}$ & 6.3 & 6.3 & 6.3 & 6.3 & 2.6 & 2.7 & 2.7 & 2.7 & 2.2 & 2.2 & 2.2 & 2.2 \\
\hline
\end{tabular}
\end{center}
\caption{Condition numbers for the exact preconditioners, using the
first formulation.}
\label{first:exact}
\end{table}

\begin{table}[h]
\begin{center}
\begin{tabular}{|c||c|c|c|c||c|c|c|c||c|c|c|c||} \hline
Prec & \multicolumn{4}{c|}{$\BB_J$}& \multicolumn{4}{c|}{$\BB_{GS} A $} &
\multicolumn{4}{c|}{$\BB_{SGS} A $}\\ \hline
$h\backslash \Delta t $ &
$2^{-2}$ & $2^{-3}$ & $2^{-4}$ & $2^{-5}$ &
$2^{-2}$ & $2^{-3}$ & $2^{-4}$ & $2^{-4}$ &
$2^{-2}$ & $2^{-3}$ & $2^{-4}$ & $2^{-5}$ \\ \hline \hline
$2^{-1}$ & 1.3 & 1.7 & 2.3 & 3.7 & 1.2 & 1.3 & 1.6 & 2.2 & 1.0 & 1.1 & 1.2 & 1.6 \\
$2^{-2}$ & 1.4 & 1.8 & 2.5 & 4.1 & 1.2 & 1.3 & 1.7 & 2.4 & 1.0 & 1.1 & 1.3 & 1.7 \\
$2^{-3}$ & 1.4 & 1.8 & 2.6 & 4.2 & 1.2 & 1.4 & 1.7 & 2.4 & 1.0 & 1.1 & 1.3 & 1.7 \\
$2^{-4}$ & 1.4 & 1.8 & 2.6 & 4.2 & 1.2 & 1.4 & 1.7 & 2.4 & 1.0 & 1.1 & 1.3 & 1.8 \\
\hline
\end{tabular}
\end{center}
\caption{Condition numbers for exact preconditioners, using the second
formulation.}
\label{second:exact}
\end{table}




\begin{table}[h]
\begin{center}
\begin{tabular}{|c||c|c|c|c|c|c|c|c|c|c|} \hline
$h\backslash \Delta t $ & $2^{-2}$ & $2^{-3}$ & $2^{-4}$ & $2^{-5}$ &
$2^{-6}$ & $2^{-7}$ \\ \hline
$2^{-2}$ & 5.7 & 5.6 & 5.4 & 5.0 & 4.4 & 3.7 \\
$2^{-3}$ & 5.8 & 5.8 & 5.7 & 5.6 & 5.5 & 5.2 \\
$2^{-4}$ & 5.9 & 5.9 & 5.8 & 5.8 & 5.7 & 5.6 \\
$2^{-5}$ & 6.1 & 6.0 & 5.9 & 5.9 & 5.8 & 5.8 \\
$2^{-6}$ & 6.2 & 6.1 & 6.1 & 6.0 & 5.9 & 5.9 \\
\hline
\end{tabular}
\end{center}
\caption{Condition numbers for Jacobi preconditioner with
multigrid, first version. }
\label{first:mg}
\end{table}




\begin{table}[h]
\begin{center}
\begin{tabular}{|c||c|c|c|c|c|c|c|c|c|c|} \hline
$h\backslash \Delta t $ & $2^{-2}$ & $2^{-3}$ & $2^{-4}$ & $2^{-5}$ &
$2^{-6}$ & $2^{-7}$ \\ \hline
$2^{-2}$ & 1.4 & 1.8 & 2.5 & 4.1 & 7.2 & 13.3 \\
$2^{-3}$ & 1.4 & 1.8 & 2.6 & 4.2 & 7.4 & 13.8 \\
$2^{-4}$ & 1.4 & 1.8 & 2.6 & 4.2 & 7.5 & 13.9 \\
$2^{-5}$ & 1.4 & 1.8 & 2.6 & 4.2 & 7.5 & 14.0 \\
$2^{-6}$ & 1.4 & 1.8 & 2.6 & 4.2 & 7.5 & 14.0 \\
\hline
\end{tabular}
\end{center}
\caption{Condition numbers for Jacobi preconditioner with
multigrid, second version. }
\label{second:mg}
\end{table}


This study is an extension of the work in \cite{Pennacchio} and \cite{block-mg}.
The above block algorithms are described in e.g., \cite{Hackbusch}.


\begin{thebibliography}{1}

\bibitem{Hackbusch}
W.~Hackbusch.
\newblock {\em Iterative Solution of Large Sparse Systems of Equations}.
\newblock Springer-Verlag, 1994.

\bibitem{Pennacchio}
M.~Pennacchio and V.~Simoncini.
\newblock Efficient algebraic solution of reaction-diffusion systems for the
cardiac excitation process.
\newblock {\em Journal of Computational and Applied Mathematics}, 145:49--70,
2002.

\bibitem{block-mg}
J.~Sundnes, G.T. Lines, K.-A. Mardal, and A.~Tveito.
\newblock Multigrid block preconditioning for a coupled system of partial
differential equations modeling the electrical activity of the heart.
\newblock {\em Computer Methods in Biomechanics and Biomedical Engineering},
5:397--409, 2002.

\end{thebibliography}


\end{document}
