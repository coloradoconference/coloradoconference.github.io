\documentclass{report}
\usepackage{amsmath,amssymb}
\setlength{\parindent}{0mm}
\setlength{\parskip}{1em}
\begin{document}
\begin{center}
\rule{6in}{1pt} \
{\large Scott MacLachlan \\
{\bf Preconditioning with Adaptive AMG}}

Department of Applied Mathematics \\ \\ University of Colorado at Boulder \\ \\ UCB 526 \\ \\ Boulder \\ CO \\ \\ 80309-0526
\\
{\tt maclachl@colorado.edu}\\
Tom Manteuffel\\
Steve McCormick\end{center}

Our ability to numerically simulate physical processes is severely
constrained by our ability to solve the complex linear systems that are
often at the core of the computation. Multigrid methods offer an
efficient solution technique for many such problems. However, fixed
multigrid processes are based on an overall assumption of smoothness that
may not be appropriate for a given problem. Adaptive multigrid methods,
such as the adaptive AMG and SA algorithms have proven effective in
achieving efficient multigrid solutions for problems where the needed
algebraically smooth components are not available a priori.

In this talk, we present progress to date on using the adaptive AMG
method as a preconditioner for the conjugate gradient algorithm. In the
setting of the Lanczos process, approximations of the lowest energy modes
of the preconditioned system are easily computed. These same modes are
the slowest converging ones of the multigrid process. We examine the use
of these modes in the adaptive AMG process as representatives of the
algebraically smooth modes needed for optimal multigrid performance.


\end{document}
