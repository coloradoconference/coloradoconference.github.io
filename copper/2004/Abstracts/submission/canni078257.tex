\documentclass{report}
\usepackage{amsmath,amssymb}
\setlength{\parindent}{0mm}
\setlength{\parskip}{1em}
\begin{document}
\begin{center}
\rule{6in}{1pt} \
{\large Andrew Canning \\
{\bf Iterative Eigensolvers for Electronic Structure Calculations in Materials Science and Nanoscience}}

LBNL MS50F-1611 \\ One Cyclotron Road  \\ Berkeley  \\ CA94720
\\
{\tt acanning@lbl.gov}\end{center}

Density functional based electronic structure calculations
have become the most heavily used approach in materials science to
calculate materials properties with the accuracy of a full quantum
mechanical treatment of the electrons. This results in an approximate
single particle form of the Schrodinger equation which is an
eigenfunction problem.
In this talk I will present a brief introduction to density functional
theory and electronic structure calculations followed by a discussion
of the iterative eigensolvers we have developed for these problems.
I will discuss and compare the performance of a few different solvers,
such as conjugate gradient based methods,
on large parallel computers for a variety of physical problems.
I will also discuss some of the eigenvalue problems associated
with newer electronic structure methods for studying large nanosystems.
(This work was supported by the Director, Office of
Advanced Scientific Computing Research, Division of Mathematical,
Information and Computational Sciences of the U.S. Department
of Energy and the Laboratory Directed Research and Development
Program of Lawrence Berkeley National Laboratory
under contract number DE-AC03-76SF00098)


\end{document}
