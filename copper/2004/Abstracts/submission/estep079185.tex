\documentclass{report}
\usepackage{amsmath,amssymb}
\setlength{\parindent}{0mm}
\setlength{\parskip}{1em}
\begin{document}
\begin{center}
\rule{6in}{1pt} \
{\large Donald Estep \\
{\bf DETECTING AND COUNTERING INSTABILITIES ARISING FROM OPERATOR SPLITTING IN THE NUMERICAL SOLUTION OF REACTION-DIFFUSION EQUATIONS}}

Department of Mathematics \\ Colorado State University \\ Fort Collins \\ CO  80523
\\
{\tt estep@math.colostate.edu}\\
David Ropp\\
John Shadid\\
	Travis King\end{center}

One of the most effective approaches to the numerical solution of
reaction-diffusion equations are the operator splitting or split-step
methods. To obtain a numerical solution of the full problem over a time
step, the diffusion and reaction components are integrated independently
in an crudely iterative fashion, with the results from the solution of
one component being supplied as data for the solution of the other
component. By combining the solutions of the two components properly,
accuracy can be achieved at a relatively low computational cost. In
particular, splitting the diffusion and reaction components allows a more
efficient treatment of the differences in scale of behavior and in
stability properties that are commonly found between the two components.
The prototypical example is a problem with a slow, stabilizing diffusion
component, which is best integrated with a implicit, stable scheme, and a
fast, de-stabilizing reaction component, which is best integrated with an
efficient, adaptive solver using
many time steps.

Several popular split-step integrators for reaction-diffusion
equations exhibit an interesting, and often dangerous, form of
instability when used on problems in which the reaction tends to cause
instability in the solution. Such conditions are
commonplace, with examples ranging from the Brusselator model in chemical
dynamics to chaotic problems to equations exhibiting blow-up behavior.
This instability, which can result in nonphysical chaotic behavior and
even blowup, does not arise
because of inaccuracy or instability in the integration schemes. Rather,
it is a direct consequence of the operator splitting itself. Roughly
speaking, in the full problem, the instability of the reaction is
balanced at every instance by the stabilizing effects of the diffusion.
However, in a split-step method, this balance occurs only at discrete
time nodes, and this can give rise to instability.

In many instances, this instability can be difficult to detect by
standard error estimators. In this paper, we develop a relatively
inexpensive method for detecting this kind of instability using a
posteriori error estimates based on generalized Green's functions. We
provide a quantitative measure of the instability in the reaction
component that can then be compared to the stabilizing effects of the
diffusion. We also discuss mechanisms for reducing the effects of this
instability through step size adjustment in a variety of situations. This
leads to new step selection mechanisms
based on stability, rather than accuracy. Finally, we explain how these
new step size selection mechanisms can be combined with a standard error
control mechanism based on accuracy.


\end{document}
