\documentclass{report}
\usepackage{amsmath,amssymb}
\setlength{\parindent}{0mm}
\setlength{\parskip}{1em}
\begin{document}
\begin{center}
\rule{6in}{1pt} \
{\large Domenico Lahaye \\
{\bf A Velocity Model Inference from the Inversion of 3D Zero-Offest Seismic Traces in the Space-Frequency Domain}}

PO Box 94079  \\ 1090 GB Amsterdam  \\ The Netherlands
\\
{\tt d.lahaye@cwi.nl}\end{center}

The goal of this work is to infer a reliable velocity model from zero-offset
seismic traces by minimizing the difference between input and simulated data
in the space-frequency domain. Typical applications of this inverse problem
include subsoil imaging for hydrocarbon prospecting, near surface
geophysics, environmental monitoring, archaeological investigation and
risk assessment
analysis.

In geophysical surveys, data are
collected from multi-shot acquisitions covering wide areas. For each shot
gather, the seismic traces are recorded by an array of receivers; the total
amount of stored information can reach the order of terabytes. Seismic data are
then averaged in such a way to simulate zero-offset traces resulting from a
virtual experiment where sources and receivers are coincident. The resulting
compressed datasets are at most of the order of gigabytes. Transforming these
data in the time frequency domain yields a further compression, making full 3D
inversions feasible.

In the standard industrial approach, a rough velocity estimate results from a
simplified processing on the complete set of shot gathers, coupled to a long
interpretation phase. The proposed automatic procedure is innovative in the
sense that it requires finding the medium velocities that minimizes the L2
norm of the misfit between {\em zero-offset} and simulated data. The resulting
formulation gives rise to non-analytic and non-linear problem with about
10$^8$ unknowns. The direct problem consists in the simulation of zero-offset
data by propagating {\em upward} the acoustic wave-field using the one-way
wave equation. The reflectivity of the medium, computed along zero-offset ray
trajectories, is used as a source term for the upward propagation. The
estimate of the reflectivity is possible by a simple edge detection
filtering of the velocity field.

To control the velocity model updating,
we have adopted a Lagrange minimization approach in which the enlarged
objective function includes an additional field multiplying the one-way
equation. From the first variation of the objective function, one obtains the
equation propagating the Lagrange multiplier. As a matter of fact, this field
is downward propagated by the adjoint of the one-way operator including a
source term equal to the residual between simulated and zero-offset
data. Because of the nature of the direct operator, it is natural to
approximate the adjoint operator by its conjugate.

Finally, this formulation provides the
gradient of the objective function with respect to the velocity field as the
integral over all frequencies of the wavefield times the velocity derivative of
the direct operator applied to the Lagrange field. The gradient evaluation loop
is concurrent over the frequencies and the only communication phase is just the
sum over all frequencies of each partial contribution.

A {\em projected conjugate gradient} algorithm for the velocity updating solves
the minimization problem simultaneously for all frequencies. At each step, once
computed the conjugate direction, a Brent's method bracketing performs
the minimum line search. Numerical results on synthetic and almost real
datasets will be provided to demonstrate the accuracy and the robustness of
the overall procedure.


\end{document}
