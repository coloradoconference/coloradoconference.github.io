\documentclass{report}
\usepackage{amsmath,amssymb}
\setlength{\parindent}{0mm}
\setlength{\parskip}{1em}
\begin{document}
\begin{center}
\rule{6in}{1pt} \
{\large Hemant Mahawar \\
{\bf Parallel Preconditioned Iterative Solvers\\ for Inductance Extraction}}

301 Harvey R. Bright Building \\ \\ Department of Computer Science \\ \\ Texas A\&M University \\ \\ College Station \\ TX 77843-3112\\
\\
{\tt mahawarh@cs.tamu.edu}\\
Vivek Sarin\end{center}

The design phase of VLSI circuits needs to take into account the signal
delays due to parasitic resistance (R), capacitance (C), and inductance
(L). With increasing clock frequency and integration density, signal
delays in next-generation circuits with millions of interconnect segments
will be dominated by parasitic inductance. In this paper, we develop fast
and accurate inductance extraction techniques that are critical to the
design of future VLSI circuits. We present a parallel preconditioned
iterative algorithm for solving the linear system arising in inductance
extraction. The reduced system, obtained in divergence free subspace
using solenoidal basis method, is solved efficiently by preconditioned
GMRES method. Parallel implementations of the algorithm have also shown
good scalability and speedup on multiprocessors.

For a set of interconnect segments, the inductance extraction problem
consists of finding an impedance matrix that represents pairwise mutual
impedance among the segments at a given frequency. Each interconnect
segment is discretized using a uniform mesh whose edges represent current
carrying filaments. The equation for the potential drop across each
filament and Kirchoff's conservation law on node currents give rise to
the following linear system [2]:
\begin{equation}
\label{eq:basic}
\left[\begin{array}
{c c} {\bf Z} & -{\bf B}\\
{\bf B}^{T} & 0 \end{array} \right ]
\left[\begin{array}
{c}{\bf I}_f \\ {\bf V}_n
\end{array}
\right]
=
\left[\begin{array}
{c}{\bf F} \\
{\bf 0}
\end{array}
\right] ,
\end{equation}
where ${\bf Z} = {\bf R} + j\omega{\bf L}$ is $n \times n$ impedance
matrix, ${\bf R}$ is diagonal matrix of filament resistance, ${\bf L}$ is
a dense, symmetric, positive definite inductance matrix denoting the
inductive coupling between current carrying filaments. ${\bf B}^T$ is the
$m \times n$ branch index matrix of filaments and nodes containing only
two nonzero entries, -1 and 1, per row. ${\bf I}_f$ is the unknown
filament current vector, and ${\bf V}_n$ is the unknown node potential
vector. Here, $j$ denotes the complex root of unity and $\omega$
represents the operational frequency. The coefficient matrix in
(\ref{eq:basic}) consists of a sparse submatrix and a dense submatrix.
Entries in ${\bf L}$ are given by:

$$
{\bf L}_{kl}
=\frac{\mu}{4\pi}\frac{1}{a_{k}a_{l}}\int_{{r_k}\epsilon{V_k}}
\int_{{r_l}\epsilon{V_l}}\frac{{\bf u}_{k}{\bf .}{\bf u}_{l}}{||{\bf
r}_{k}
- {\bf r}_{l} ||} dV_{k}dV_{l},
$$

The current ${\bf I}_f$ may be expressed as the matrix vector product
${\bf I}_f = {\bf Px}$ for arbitrary ${\bf x}$, where ${\bf P}$ is a
basis for null space of ${\bf B^T}$. This leads to a reduced system of
the form:
\begin{equation}
\label{eq:reduced_system}
{\bf P^TZPx} = {\bf P^TF}
\end{equation}
which is solved using an iterative method such as GMRES. A sparse basis
for the null space of ${\bf B^T}$ can be constructed using local
solenoidal functions that represent a unit current
circulating along the filaments of a single mesh cell. Each such local
flow contributes a column to the matrix ${\bf P}$. (see, e.g., [4] for
more details.)

The most time consuming step in the iterative solver is matrix vector
product with {\bf L} which appears in the reduced system
(\ref{eq:reduced_system}). Use of fast hierarchical methods like Fast
Multipole Method [1] makes the algorithm matrix-free and reduces the time
complexity from O$(n^2)$ to O$(n)$. In absence of the system matrix,
however preconditioning the iterative method becomes a challenging task.

We present a nearly optimal preconditioner [4] for the reduced system
that demonstrates convergence independent of the problem parameters,
namely the mesh size $h$ and the frequency of operation $\omega$.
Preconditioning step is analogous to computing the mutual impedance of
filaments placed at the center of each mesh cell. The preconditioner has
a structure similar to ${\bf L}$, and is applied to a vector using fast
hierarchical methods. Following table shows the number of right
preconditioned GMRES iterations to compute the self-impedance of the
ground plane conductor problem.

\begin{center}
\begin{tabular}{|c|c|c|c|c|} \hline
{Mesh} & {Filament} & \multicolumn {3}{|c|}{Frequency}\\
\cline{3-5}
{Size} & {Length} & 1 GHz & 10 GHz & 100 GHz \\ \hline
$65 \times 65$ & $2^{-6}$cm & 6 & 6 & 5 \\ \hline
$129 \times 129$ & $2^{-7}$cm & 7 & 6 & 6 \\ \hline
$257 \times 257$ & $2^{-8}$cm & 9 & 7 & 7 \\ \hline
$513 \times 513$ & $2^{-9}$cm & 12 & 8 & 7 \\ \hline
\end{tabular}
\end{center}

On a set of benchmark problems, a serial implementation of this technique
is up to 5 times faster than {\it FastHenry}, a popular inductance
extraction software, with only one-fifth of memory requirements. Further
enhancements to the implementation are achieved by exploiting the
parallelism in the algorithm [3]. Following table shows the parallel
efficiency of the code on the IBM p690 multiprocessor. Ongoing work
involves development of an MPI-based parallel object oriented code for
the algorithm for distributed memory multiprocessors. Performance
evaluation of a hybrid MPI-OpenMP implementation is also underway.

\begin{center}
\begin{tabular}{|c|c|c|c|c|c|c|} \hline
No. of & \multicolumn {6}{|c|}{Mesh Size}\\ \cline{2-7}
Processors & \multicolumn {2}{|c|}{128 $\times$ 128} & \multicolumn
{2}{|c|}{256 $\times$ 256} & \multicolumn {2}{|c|}{512 $\times$ 512}\\
\cline{2-7}
& Time & Efficiency (\%) & Time & Efficiency (\%) & Time & Efficiency (\%) \\ \hline
1 & 769.87 & 100.0 & 3727.04 & 100.0 & 17163.5 & 100.0 \\ \hline
2 & 385.00 & 100.0 & 1864.51 & 100.0 & 8610.04 & 99.7 \\ \hline
4 & 192.74 & 99.8 & 937.95 & 99.3 & 4315.18 & 99.4 \\ \hline
8 & 102.72 & 93.8 & 482.45 & 96.6 & 2203.06 & 97.4 \\ \hline
16 & 54.84 & 87.7 & 249.91 & 93.3 & 1192.77 & 89.9 \\ \hline
\end{tabular}
\end{center}

\section*{Reference}
[1] L.~Greengard.
\newblock {\em The Rapid Evaluation of Potential Fields in Particle Systems}.
\newblock The MIT Press, Cambridge, Massachusetts, 1988. \\

\parindent=0pt
[2] M.~Kamon, M.~J. Tsuk, and J.~White.
\newblock {FASTHENRY}: A multipole-accelerated 3{D} inductance extraction program.
\newblock {\em IEEE Transaction on Microwave Theory and Techniques}, Vol.
42(9):1750--1758, September, 1994. \\

[3] H.~Mahawar and V.~Sarin.
\newblock Parallel iterative methods for dense linear systems in inductance extraction.
\newblock {\em Parallel Computing}, Vol. 29:1219--1235, September, 2003. \\

[4] H.~Mahawar, V.~Sarin, and W.~Shi.
\newblock A solenoidal basis method for efficient inductance extraction.
\newblock In {\em Proceedings of the $39^{th}$ Conference on Design
Automation}, pages 751--756, New Orleans, Louisiana, June 2002.


\end{document}
