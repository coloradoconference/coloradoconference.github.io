\documentclass{report}
\usepackage{amsmath,amssymb}
\setlength{\parindent}{0mm}
\setlength{\parskip}{1em}
\begin{document}
\begin{center}
\rule{6in}{1pt} \
{\large Dennis C. Smolarski \\
{\bf Implementing Chebyshev Iteration on Parallel Architectures: Some Preliminary Results}}

Department of Mathematics and Computer Science \\ Santa Clara University \\ 500 El Camino Real \\ Santa Clara \\ CA  95053-0290
\\
{\tt dsmolarski@math.scu.edu}\\
Paul E. Saylor\\
F. Douglas Swesty\\
	Szypowski, Ryan S.\end{center}

One of the known bottlenecks to parallel scalability of Krylov
subspace algorithms is the need for inner products. For this
reason, Chebyshev iteration has often been mentioned, but
rarely studied, as an optimal Krylov subspace algorithm for
parallel architectures since it does not require the inner
products if the iterative parameters are known.

In this paper, we consider {\sc Chebycode,} a hybrid Chebyshev
algorithm developed by Howard Elman, Steve Ashby, and Tom
Manteuffel, that estimates iterative parameters by means of
variants of the power method. A parallel F90+MPI implementation of
this algorithm was recently developed by Ryan Szypowski and we
have been employing this algorithm in conjunction with sparse,
parallel approximate inverse preconditioners in radiation
transport simulations. We present some preliminary results on the
scalability and effectiveness of this method in comparison with
other Krylov subspace algorithms.


\end{document}
