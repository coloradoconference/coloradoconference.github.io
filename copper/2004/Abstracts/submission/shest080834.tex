\documentclass{report}
% \usepackage{amsmath,amssymb}
\setlength{\parindent}{0mm}
\setlength{\parskip}{1em}
\begin{document}
\begin{center}
\rule{6in}{1pt} \
{\large Aleksei I. Shestakov \\
{\bf Solution of the nonlinear multifrequency radiation diffusion equations using pseudo transient
continuation\footnote{This work
% continuation\thanks{This work
was performed under the auspices of the U.S.\ Dept.\ of Energy by the Univ.\ of Calif.\ Lawrence Livermore National Laboratory under contract No.\ W-7405-Eng-48.}}}

Lawrence Livermore National Laboratory \\ POB 808, L-38 \\ Livermore  CA  94550
\\
{\tt shestakov@llnl.gov}\end{center}

\newcommand{\be}{\begin{equation}}
\newcommand{\ee}{\end{equation}}
\newcommand{\bey}{\begin{eqnarray}}
\newcommand{\eey}{\end{eqnarray}}

\newcommand{\pref}[1]{(\ref{#1})}

\newcommand{\cv}{c_{\rm v}}
\newcommand{\dt}{\Delta t}
\newcommand{\gdt}{\Delta \tau}


Computer codes simulating high energy density physics
consist of modules for distinct physical processes, e.g.,
compressible hydrodynamics and radiation transport. For the
latter, one model assumes tight coupling between radiation
and matter. The dependent variables are the
spectral radiation energy density $u( x, \nu, t)$ and the matter
temperature $T( x, t)$, where $x$, $\nu$, and $t$ denote position,
frequency, and time, respectively.
The system is of parabolic form,
\bey
\partial_t u & = & \nabla \cdot (D \nabla u) +
c \, \rho \, \kappa \, ( B - u ) \; , \label{ueq} \\
\rho \, \cv \, \partial_t T & = & - c \, \rho \int_0^{\infty} d \nu
\, \kappa \, ( B - u ) \; . \label{eeq}
\eey

In \pref{ueq}-\pref{eeq}, $c$ is the speed of light, $\rho$ the
mass density, $\kappa$ the opacity, and $\cv$ the specific heat.
The Planck function
$B \propto y^3 / (e^y - 1)$, where $y \propto \nu / T$.
In our context, $\rho( x, t)$ is a known function. The opacity
is a complicated function of $\rho$, $T$, and $\nu$.
For ``free-free transitions'', $\kappa \propto \nu^{-3}$.
The coefficient $D$ depends on the mean free path
$\ell \doteq 1 / \rho \, \kappa$ and, to mitigate unphysical
propagation speeds, a flux limiter is introduced. One common
description is $D = c / [ f(u) + 3/ \ell]$, where $f= |\nabla u| / u$.

Equations \pref{ueq}-\pref{eeq} are solved by discretizing the
spectrum $0 \le \nu \le \infty$ into $G$ {\em groups\/}
defined by $\{ \nu_j \}_{j = 0}^G$.
Integration over each interval $( \nu_{j-1}, \nu_j)$,
yields the multigroup equations in which the integral
over $\nu$ is replaced by a sum of $G$ terms.
The system is difficult to solve because of
its nonlinearity and wide ranges of time and spatial scales.
The ranges are evidenced by the coupling
$c \, \rho \, \kappa \; (= c / \ell)$
and diffusion $D \sim c \ell$ terms.
High frequency radiation is characterized by
$\ell \gg 1$, i.e.,
slow coupling and fast transport. The
opposite holds for low
frequencies. In simulations,
the coefficients can vary over 10 to 30 orders of magnitude.
The difficulty is compounded by nonlinearity and material
composition since the coefficients depend on $\rho$ and $T$.

In this talk we describe a scheme to solve \pref{ueq}-\pref{eeq}
for multiphysics codes containing a separate hydrodynamic module.
Since such codes typically run at
the Courant-limited {\em sound\/} speed, for our
applications, the time step $\dt$ is arbitrarily large.
Hence, we use backward Euler temporal
differencing. After multiplying through by $\dt$,
we obtain,
\bey
0 & = & u_j^0 - u_j + \nabla \cdot (D_j' \nabla u_j) +
k_j \, ( B_j - u_j ) \; ,
\;\; j = 1, \ldots \, , G \; , \label{ujeq2} \\
0 & = &
\rho \, \cv \, ( T^0 - T) - \sum_{j-1}^G k_j
\, ( B_j - u_j ) \; , \label{Teq2}
\eey
where $D_j' = \dt \, D_j$, $k_j = c \, \dt \, \rho \, \kappa_j$,
and the superscript $0$ denotes the solution at the prior time level.
The index $j$ replaces the frequency dependence.
Thus, $u_j = \int d \nu \, u$ and similarly for $B$,
where the integration is over $( \nu_{j-1}, \, \nu_j)$.
The coefficients $D_j$ and $\kappa_j$ denote averages
over the interval.

Viewing \pref{ujeq2}-\pref{Teq2} as a nonlinear elliptic system, we
introduce pseudo
transient continuation $(\Psi$tc). On the LS
of \pref{ujeq2}, we place $(u_j - u_j^*)/\gdt$,
where $\gdt$ is the $\Psi$tc parameter and
$u_j^*$ is the solution at the previous {\em pseudo\/}
time. Similarly, the LS of \pref{Teq2} becomes
$\rho \, \cv \, ( T - T^*)/\gdt$. The desired solution
is the pseudo time steady-state.

For each $\Psi$tc step, we linearize $B_j$
about the previous iterate,
$B_j = B_j^* + (\partial B_j / \partial T)|_{T = T^*} (T - T^*)$.
Remaining coefficients, e.g., $k_j$, are evaluated at $T = T^*$.
We avoid a full Newton linearization in order to maintain
robustness. (Coefficients such as $\kappa_j$
are only known approximately and are given in tabular
form.) For the first $\Psi$tc step, $u_j^* = u_j^0$ and
$T^* = T^0$.

The energies $u_j$ are directly coupled to $T$
through the coefficients $k_j$.
The equation for $T$ does not contain any spatial derivatives.
After linearizing, we solve for $T$ analytically and
substitute the result into the $u_j$ equations,
(Schur complement). This yields $G$
equations in which each $u_j$ is explicitly
coupled to the rest. The linear system is of
order $N \, G$, where $N$ is the number of spatial
points and is of the form
\begin{equation}
% \be
(\Lambda - M_1 - M_2) \, u = b \; . \label{linsys}
\end{equation}
% \ee
In \pref{linsys},
$\Lambda$ is diagonal, $M_1$ contains the offdiagonal terms
stemming from diffusion and $M_2$, from
intergroup coupling. The parameter $1/\gdt$
appears in both $\Lambda$ and $b$; in both places, it contributes
to robustness.

We derive conditions on $1/\gdt$ that yield
diagonal dominance and non-negative RS, $b \ge 0$.
The conditions determine the initial value of $1/\gdt$.
Our strategy ensures that each $\Psi$tc iterate,
yields a physically reasonable result.
In ``real'' problems, the requirement is crucial since the
solution of \pref{linsys} is used to obtain $T$, which
in turn determines updates of $k_j$, $B_j$, etc.
A conventional Newton iteration may generate an unphysical
value, e.g., $T = -1$, causing the code to halt.

The scheme has been implemented in a radiation-hydrodynamic
code. Results will be presented comparing
the $\Psi$tc scheme with a more conventional one.


\end{document}
