\documentclass{report}
\usepackage{amsmath,amssymb}
\setlength{\parindent}{0mm}
\setlength{\parskip}{1em}
\begin{document}
\begin{center}
\rule{6in}{1pt} \
{\large John, N Shadid \\
{\bf The Split Personality of Operator Splitting Methods: Diffusion/Reaction Systems*}}

Computational Science Department \\ PO Box 5800 MS 1111 \\ Sandia National Laboratories \\ Albuquerque \\ NM 87185
\\
{\tt jnshadi@sandia.gov}\\
David, L Ropp\end{center}

In this talk we present numerical experiments of time integration methods
applied to coupled time dependent nonlinear systems of reaction-diffusion
equations. Our main interest is in evaluating the relative accuracy and
asymptotic order of accuracy of the methods on problems which exhibit an
approximate balance between the competing component time scales. Nearly
balanced systems can produce a significant coupling of the physical
mechanisms and introduce a slow dynamical time scale of interest. These
problems provide a challenging test for this evaluation and tend to
reveal subtle differences between the various methods. The methods we
consider include first- and second-order fully implicit, and
operator-splitting techniques. The test problems include a prototype
propagating nonlinear reaction-diffusion wave, a non-equilibrium
radiation-diffusion system, and a Brusselator chemical dynamics system.
In this evaluation we demonstrate a ``split-personality'' for the
operator-splitting methods that we consider, in that, while they often
have very good accuracy, they are not always robust.


*This work was partially supported by the ASCI program and the DOE Office
of Science MICS program at Sandia National Laboratory under contract
DE-AC04-94AL85000.


\end{document}
