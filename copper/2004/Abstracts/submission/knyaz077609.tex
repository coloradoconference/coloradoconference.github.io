\documentclass{report}
\usepackage{amsmath,amssymb}
\setlength{\parindent}{0mm}
\setlength{\parskip}{1em}
\begin{document}
\begin{center}
\rule{6in}{1pt} \
{\large Andrew Knyazev \\
{\bf Hard and soft locking in iterative methods for symmetric eigenvalue problems}}

Department of Mathematics \\ University of Colorado at Denver P.O. Box 173364 \\ Campus Box 170 \\ Denver \\ CO 80217-3364.
\\
{\tt andrew.knyazev@cudenver.edu}\end{center}

Let us consider a problem of computing a number of eigenvectors of a
symmetric matrix by simultaneous iterations. When computing several
eigenvectors simultaneously it is often observed that some eigenvectors
converge faster than the others. To avoid unnecessary computational work,
it is common to �lock� the eigenvectors that have already converged
within a required tolerance while continue iterating other eigenvectors.
A typical locking technique is a deflation by restriction, where the
locked eigenvectors are no longer changed, at the same time as the
eigenvectors that are still iterating are kept orthogonal to the locked
ones. We call this technique �hard locking.�

A different technique, called �soft locking� has been suggested by the
author earlier, e.g., [1], but it apparently went largely unnoticed. The
soft locking is for simultaneous iterations combined with the
Rayleigh�Ritz method. The core idea of the hard and soft locking is the
same: the locked vectors do not participate in the main iterative step.
The soft locked vectors, however, fully participate in the Rayleigh�Ritz
method and thus are allowed to be changed by the Rayleigh�Ritz method.
The orthogonality of the soft locked vectors to iterative vectors follows
from the well-known fact that the Ritz vectors can be chosen to be
orthogonal automatically.

The soft locking is computationally more expensive compared to the hard
locking. The advantage of the soft locking is that it provides more
accurate results. First and foremost, adding more vectors to the hard
locking procedure decreases the accuracy of the orthogonal complement to
their span, which may prevent the iterative vectors from reaching the
required tolerance. This effect seems less problematic for soft locking.
Second, the soft locked vectors tend to become more accurate from
participating in the Rayleigh�Ritz method even though they are removed
from the main iterative step.

In the present talk, we give a detailed description of soft locking and
present preliminary theoretical and numerical results demonstrating the
effectiveness of the soft locking compared to the traditional hard
locking and discuss peculiarities of possible implementations of the hard
and soft locking in preconditioned block eigenvalue solvers such as the
Locally Optimal Block Preconditioned Conjugate Gradient Method [1-2].

[1] A. V. Knyazev, Preconditioned eigensolvers. In Templates for the
Solution of Algebraic Eigenvalue Problems: A Practical Guide. Editors:
Zhaojun Bai, James Demmel, Jack Dongarra, Axel Ruhe, and Henk Van der
Vorst, SIAM, pp. 337-368, 2000.
http://www.cs.utk.edu/$\,\tilde{}\,$dongarra/etemplates/node410.html

[2] A. V. Knyazev, Toward the Optimal Preconditioned Eigensolver: Locally
Optimal Block Preconditioned Conjugate Gradient Method. SIAM Journal on
Scientific Computing 23 (2001), no. 2, pp. 517-541.


\end{document}
