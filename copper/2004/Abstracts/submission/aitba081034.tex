\documentclass{report}
\usepackage{amsmath,amssymb}
\setlength{\parindent}{0mm}
\setlength{\parskip}{1em}
\begin{document}
\begin{center}
\rule{6in}{1pt} \
{\large Rakhim Aitbayev \\
{\bf Multilevel Preconditioners for Nonselfadjoint or Indefinite Orthogonal Spline Collocation Problems}}

Department of Mathematics \\ New Mexico Tech \\ Socorro \\ NM 87801
\\
{\tt aitbayev@nmt.edu}\\
Bernard Bialecki\end{center}

\newtheorem{theorem}{Theorem}\newcommand{\GP}{\mbox{${\cal G}$}}\newcommand{\SH}{\mbox{$\scriptscriptstyle H^{2}(\Omega)$}}\newcommand{\SL}{\mbox{$\scriptscriptstyle L^{2}(\Omega)$}}We develop and study symmetric multilevel preconditionersfor the computation of the orthogonal spline collocation (OSC)solution of a Dirichlet boundary value problem (BVP) with anonselfadjoint or an indefinite operator.
The OSC solution issought in the space of piecewise Hermite bicubic splinefunctions defined on a uniform partition.
We consider anadditive and a multiplicative multilevel preconditionersthat are used with the preconditioned conjugate gradient (PCG)method.Let $\Omega$ be a unit square $(0,1)\times(0,1)$ with theboundary $\partial\Omega$,
and let $x=(x_1,x_2)$.
We consider aBVP\begin{equation}\label{eq:bvp}Lu\equiv\sum_{i,j=1}^{2} a_{ij}(x)u_{x_ix_j}+\sum_{i=1}^{2}b_i(x) u_{x_i}+c(x)u = f(x),\;\;x\in\Omega,\quad u=0 \;\;\mbox{on}\;\;\partial\Omega.\end{equation}Operator $L$ could be non-selfadjoint or indefinite in $L^2$inner product.
We assume that the principal part of $L$ satisfies the uniform ellipticity condition and that BVP (\ref{eq:bvp}) hasa unique solution in $H^2(\Omega)$.Let $\pi_0$ be a uniform coarsest rectangular partition of$\Omega$.
We obtain a set of partitions $\{\pi_k\}_{k=0}^{K}$ bystandard coarsening,
and let$V_0\subset V_1\subset\ldots\subset V_K\equiv V_h$ be the set ofcorresponding nested spaces of piecewise Hermite bicubics thatvanish on $\partial\Omega$.
Let $\sum$ denote the two-dimensionalcomposite Gauss quadrature corresponding to partition $\pi_h$with 4 nodes in each element.
Let $\GP_h$ denote thecorresponding set of Gauss points.
The OSC discretization of BVP(\ref{eq:bvp}) is defined by\begin{equation}\label{eq:osc}u_h\in V_h,\quad Lu_h(\xi)=f(\xi),\quad\xi\in\GP_h,\end{equation}and it can be written as the operator equation $L_hu_h=f_h$ inthe Hilbert space $V_h$ with the inner product $(v,w)_h=\sum vw$.Let $\{\psi_{k,j}\}_{j=1}^{N_k}$ be the standard finite elementbasis of $V_k$ consisting of products of one-dimensional valueand slope basis functions.
Using space decomposition\[V_h=V_0+\sum_{k=1}^{J}\sum_{j=1}^{N_k}V_{k,j},
\quad V_{k,j}= \mbox{span}(\psi_{k,j}),\]we define and study multilevel additive$B_{\textup{\scriptsize a}}$ and multiplicative$B_{\textup{\scriptsize m}}$ preconditioners for solving thenormal equation $L_h^*L_hu_h=L_h^*f_h$,
where$L_h^*$ is the adjoint to $L_h$.
The implementation of$B_{\textup{\scriptsize a}}$and $B_{\textup{\scriptsize m}}$ is based on relationshipsbetween basis functions for two consecutive partitions and theimplementation of $B_{\textup{\scriptsize m}}$is similar to that for V(1,1)-cycle with the Gauss-Seidelsmoothing.
A problem on the coarsest partition is assumedsufficiently small,
and it is solved exactly.
The computationalcost of the preconditioning algorithms is $\mbox{O}(N_K)$.The following is our main result.\begin{theorem}There are positive independent of $h$ and $K$ constants$\alpha_{\textup{\scriptsize a}}$,$\beta_{\textup{\scriptsize a}}$,$\alpha_{\textup{\scriptsize m}}$,and$\beta_{\textup{\scriptsize m}}$,such that\begin{eqnarray}\label{eq:spectral_equiv}&&\alpha_{\textup{\scriptsize a}}\,(B_{\textup{\scriptsize a}}v,v)_h\leq (L_h^*L_hv,v)_h \leq\beta_{\textup{\scriptsize a}}\,(B_{\textup{\scriptsize a}}v,v)_h,\quad v\in V_h,\\\nonumber&&\alpha_{\textup{\scriptsize m}}\,(B_{\textup{\scriptsize m}}v,v)_h\leq (L_h^*L_hv,v)_h \leq\beta_{\textup{\scriptsize m}}\,(B_{\textup{\scriptsize m}}v,v)_h,\quad v\in V_h.\end{eqnarray}\end{theorem}We present numerical results that demonstrate the efficiency ofour preconditioning algorithms.

\end{document}
