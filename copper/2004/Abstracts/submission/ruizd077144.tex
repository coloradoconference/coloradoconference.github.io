\documentclass{report}
\usepackage{amsmath,amssymb}
\setlength{\parindent}{0mm}
\setlength{\parskip}{1em}
\begin{document}
\begin{center}
\rule{6in}{1pt} \
{\large Daniel Ruiz \\
{\bf Adaptive preconditioners for Newton-Krylov methods}}

INPT-ENSEEIHT \\ 2 \\ rue Camichel \\ 31071 Toulouse CEDEX \\ France
\\
{\tt Daniel.Ruiz@enseeiht.fr}\\
Daniel Loghin\\
Ahmed Touhami\end{center}

The use of preconditioned Newton-Krylov methods is in many applications
mandatory for computing efficiently the solution of large nonlinear
systems of equations. However, the available preconditioners are often
sub-optimal, due to the changing nature of the linearized operator. This
the case, for instance, for quasi-Newton methods where the Jacobian (and
its preconditioner) are kept fixed at each non-linear iteration, with the
rate of convergence usually degraded from quadratic to linear. Updated
Jacobians, on the other hand require updated preconditioners, which may
not be readily available. In this work we introduce an adaptive
preconditioning technique based on the Krylov subspace information
generated at previous steps in the nonlinear iteration. In particular, we
use to advantage a deflation technique suggested in [1] for restarted
GMRES to enhance existing preconditioners with information about (almost)
invariant subspaces constructed by GMRES at previous stages in the
nonlinear iteration. We provide guidelines on the choice of
invariant-subspace basis used in the construction of our preconditioner
and demonstrate the improved performance on various test problems. As a
useful general application we consider the case of augmented systems
preconditioned by block triangular matrices based on the structure of the
system matrix. We show that a sufficiently good solution involving the
primal space operator allows for an efficient application of our adaptive
technique restricted to the space of dual variables.


[1] J. Baglama, D. Calvetti, G. H. Golub, and L. Reichel.
Adaptively preconditioned GMRES algorithms.
SIAM J. Sci. Comput., 20(1):243--269 (electronic), 1998.


\end{document}
