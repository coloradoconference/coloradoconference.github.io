\documentclass{report}
\usepackage{amsmath,amssymb}
\setlength{\parindent}{0mm}
\setlength{\parskip}{1em}
\begin{document}
\begin{center}
\rule{6in}{1pt} \
{\large Per Sundqvist \\
{\bf Boundary Summation Equations}}

Dept. of Information Technology \\ Scientific Computing \\ Uppsala University \\ Box 337 \\ SE-751 05 Uppsala \\ SWEDEN
\\
{\tt per.sundqvist@it.uu.se}\end{center}

The boundary element method (BEM) relies on the fact that a linear
partial differential equation (PDE) with constant coefficients can be
reduced to a boundary integral equation (BIE). In this presentation,
it is shown that a similar reduction is possible for linear difference
equations with constant coefficients. The reduced equations involve a
summation operator and are called boundary summation equations (BSE).

There are two key properties that BSE and discretized BIE have in
common; the number of unknowns is low (compared to methods where the
interior domain is discretized) and the resulting linear system is
full. The latter property implies that special techniques are required
in an iterative solver, especially in higher dimensions. One example
for BIE is the multipole method, which can be used to apply an
integral operator. For the BSE, it is possible to apply the summation
operator efficiently by FFT, provided that the domain is simple
enough.

The concepts of BSE are illustrated by a two dimensional example. Let
\[
Pu=f,
\]
be the linear system that represents the usual five point Laplacian
with Dirichlet boundary conditions on a rectangular domain
$\bar{\Omega}=(0,\ldots,n_1)\times(0,\ldots,n_2)$. Let also $E$ be a
discrete fundamental solution, i.e.~a function that satisfies
\[
\hat{P}E_i = \left\{ \begin{array}{ll}
1, &i=0\\
0, &i\ne 0 \end{array} \right. ,
\]
where $\hat{P}$ represents the discrete Laplacian. Define the
summation operator $K$ according to
\[
Kv_i=\sum_{j\in\bar{\Omega}}E_{i-j}v_j.
\]
Now, $P$ can be reordered such that
\begin{equation}
\label{eq:Puf}
\left( \begin{array}{cc} P_{\Gamma}&P_{\Gamma\Omega}\\
P_{\Omega\Gamma}&P_{\Omega} \end{array} \right)
\left( \begin{array}{c} u_{\Gamma}\\u_{\Omega} \end{array} \right)
=\left( \begin{array}{c} f_{\Gamma}\\f_{\Omega} \end{array} \right),
\end{equation}
where subscript $\Gamma$ denoted boundary points and $\Omega$ denotes
interior points. Mixed subscripts denote the couplings. If the same
ordering is applied when $K$ is represented by a matrix, one can show
that
\begin{equation}
\label{eq:KP}
\left( \begin{array}{cc} P_{\Gamma}&P_{\Gamma\Omega}\\
P_{\Omega\Gamma} & P_{\Omega} \end{array} \right)
\left( \begin{array}{cc} K_{\Gamma}&K_{\Gamma\Omega}\\
K_{\Omega\Gamma}&K_{\Omega} \end{array} \right)
=\left( \begin{array}{cc} A&B\\0&I \end{array} \right),
\end{equation}
where
\[
A=P_{\Gamma}K_{\Gamma} + P_{\Gamma\Omega}K_{\Omega\Gamma},
\]
and
\[
B=P_{\Gamma}K_{\Gamma\Omega}+P_{\Gamma\Omega}K_{\Omega}.
\]
A new vector $v$ of unknowns is introduced, such that
\[
\left(\begin{array}{c} u_{\Gamma}\\u_{\Omega} \end{array}\right)
=\left( \begin{array}{cc} K_{\Gamma}&K_{\Gamma\Omega}\\
K_{\Omega\Gamma}&K_{\Omega} \end{array} \right)
\left(\begin{array}{c} v_{\Gamma}\\v_{\Omega} \end{array} \right).
\]
By inserting the new vector into (\ref{eq:Puf}) and using
(\ref{eq:KP}), it is easy to see that $v_{\Omega}=f_{\Omega}$ and
then, by elimination, that
\begin{equation}
\label{eq:AvBf}
Av_{\Gamma}=f_{\Gamma}-Bf_{\Omega}.
\end{equation}

The original problem, with $(n_1+1)(n_2+1)$ unknowns, is hence reduced
to a system of equations with $2(n_1+n_2)$ unknowns. The additional
cost for the reduction process lies in the construction of a
fundamental solution. An algorithm requiring $\mathcal{O}(n_1n_2\log
n_1n_2)$ arithmetic operations is described in \cite{BrSu03}.

In this case, where the domain is rectangular, it is possible to apply
the matrix $A$ by an embedding technique, related to the way in which
Toeplitz matrices can be applied by embedding them in circulants. The
complexity for this operation is also $\mathcal{O}(n_1n_2\log
n_1n_2)$. The same holds for computing the original vector of
unknowns from the solution of the reduced system. The over all cost
for solving the original problem is hence $\mathcal{O}(k n_1n_2\log
n_1n_2)$, where $k$ is the number of iterations needed for the reduced
system to converge to a desired level. One detail that is worth
pointing out is that there is no additional cost in determining the
residual of the original system when performing iterations on the
reduced system.

The reduction process generalizes to several dimensions and is not
restricted to any special kind of difference operator, as long as it
is linear and has constant coefficients. The fast application
technique requires that the domain is a hyper cuboid. Fast application
techniques for more complicated domains is a subject for future
research. Numerical experiments for the solution of~(\ref{eq:AvBf})
will be presented in the talk. Results for various difference
equations and various boundary conditions will be shown and
preliminary results indicate grid independent GMRES iteration counts
for discretizations of first order PDEs.

\begin{thebibliography}{9}
\bibitem{BrSu03} H. Brand\'en and P. Sundqvist \emph{An Algorithm for
Computing Fundamental Solutions of Difference Operators} Tech.
Rep. 2003-006 Dept. of Information Technology, Uppsala Univ.,
Uppsala, Sweden, 2003
\end{thebibliography}


\end{document}
