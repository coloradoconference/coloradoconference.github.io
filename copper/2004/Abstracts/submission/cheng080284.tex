\documentclass{report}
\usepackage{amsmath,amssymb}
\setlength{\parindent}{0mm}
\setlength{\parskip}{1em}
\begin{document}
\begin{center}
\rule{6in}{1pt} \
{\large Zhen Cheng \\
{\bf Multigrid-Type Solvers for Simulations with Adaptive Mesh Refinement}}

4333 Siebel Center \\ 201 N. Goodwin \\ Urbana,IL,61801-2302
\\
{\tt zcheng@uiuc.edu}\\
Eric de Sturler\end{center}

The Interoperability Based Environment for Adaptive Meshes (IBEAM) is
a NASA funded object-oriented framework for astrophysical
simulations on high-performance, distributed memory, parallel
computing platforms. In this project, we aim to solve
radiation-hydrodynamic models of Gamma-Ray bursts. As such
models require a high variation in the resolution of the computational grid,
we use the PARAMESH package to support adaptive mesh refinement (AMR) on
parallel machines. The PARAMESH package is developed by the Computational Technologies
Team of NASA Goddard Space Flight Center.\\

PARAMESH was originally designed for explicit finite difference
methods. A drawback for our problems is that excessively small
time steps may be required for stability. To avoid this problem,
we are implementing implicit methods on the PARAMESH package.
These methods demand efficient solvers for large sparse linear
systems. However, in principle PARAMESH refines and unrefines the
grids at every time step. Moreover, PARAMESH stores the grids as a
large collection of relatively small grid blocks (with pointers
for neighbors, children and parent blocks), which are
redistributed frequently over the processors for load balancing.
The user has no control over the distribution of grid blocks over
the processors. Finally, the matrix is typically not computed
explicitly. These features make many preconditioners typically
used with Krylov subspace methods difficult to implement and/or
expensive to use, such as domain decomposition type
preconditioners and ILU type
preconditioners.\\

Multigrid is in principle well-suited for such an environment. However,
typical problems may involve strong convection, jumps in coefficients, and
strong anisotropy. Such features typically require robust versions of the
multigrid algorithm. However, these features,too, may be difficult to implement
and/or expensive.\\

Therefore, we study a number of combinations of
preconditioned Krylov subspace methods and multigrid methods. An important feature
of our problems (and many other dynamic AMR discretizations) is that most refinements
and unrefinements occur above a certain level (which may vary over the computational
domain). Hence, we can often identify a (relatively high) level where changes in the
grid are relatively rare. More expensive computations, such as computing a very good
preconditioner, at this level can be amortized over many time steps.\\

The multigrid method consists of two main components: smoothing,
to reduce high-frequency error, and coarse
grid correction, to reduce low-frequency error.

For hard problems such as diffusion-convection problems, problems with
jumps in the coefficients, and problems with strong anisotropy,
robust smoothers are needed. In the PARAMESH environment, line, and plane
smoothers are hard to implement, relatively expensive, and (parallel)
direct solvers
for such subproblems may need to be recomputed at every time step.
Therefore, we experiment with other block smoothers.
For the coarse grid correction, the full approximate scheme (FAS) is used
instead of the usual residual-error correction. This is required
on AMR type meshes, because some parts of the domain are not
covered by finer grids, so that we have to solve for the solution
itself. In order to improve robustness in the coarse grid solves
we experiment with a relatively high level 'direct' solver using a
preconditioned Krylov method at levels where changes in the grid(s) are
relatively infrequent. As preconditioner we consider explicit
sparse approximate inverse preconditioners. These preconditioners are fairly
insensitive to the redistribution of blocks over the processors, and can
be updated for changes in the grid. Techniques for doing such updates will be
presented by Shun Wang (UIUC) in a separate presentation. Sparse approximate
inverses can also be used effectively in block smoothers.\\

Based on the designs discussed above, we have implemented several variations of
multigrid/Krylov subspace solvers on PARAMESH, including multigrid with a
direct solver on
the coarsest grid level, multigrid with a Krylov subspace iterative
solver on a specified grid level, and multigrid with a
preconditioned Krylov iterative solver on a specified grid level,
all with several types of block smoothers.
Some of these variations are built on top of Krylov subspace methods.
More specifically, rather than carrying out a V-cycle (or other scheme) to the coarsest
grid level and then using a direct solver, we carry out the V-cycle (or
other scheme) to some
relatively finer grid level, and then use a Krylov subspace solver as a
replacement for a direct solver on that level. We use an iterative
solver instead of a direct solver here since the matrix is not
stored explicitly, which makes a direct solver difficult without
assembling the large sparse matrix. Furthermore, it would be hard to deal with the
redistribution of grid blocks and the grid adaptations in a direct solver.
Moreover, preconditioned iterative
methods will typically achieve better parallel efficiency.
Sparse Approximate inverse preconditioners are efficient for the same
reasons as we mentioned above. Notice that these methods can also be
considered as Krylov subspace methods at the more static grid levels enhanced
with multigrid techniques to deal with the more dynamic grid levels.\\

Detailed experiment results will be presented and
discussed in the talk.\\

The IBEAM project is sponsored under a Round III Grand Challenge
Cooperative Agreement with NASA's Computational Technolgies
Project.


\end{document}
