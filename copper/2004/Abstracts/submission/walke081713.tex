\documentclass{report}
\usepackage{amsmath,amssymb}
\setlength{\parindent}{0mm}
\setlength{\parskip}{1em}
\begin{document}
\begin{center}
\rule{6in}{1pt} \
{\large Homer F. Walker \\
{\bf Inexact Newton dogleg methods}}

Mathematical Sciences Department \\ Worcester Polytechnic Institute \\ Worcester \\ MA 01609-2280
\\
{\tt walker@wpi.edu}\\
Roger P.  Pawlowski\\
John N.  Shadid\\
	Simonis, Joseph P. \end{center}

The dogleg method is a classical trust-regiontechnique for globalizing Newton's method.
While it is widelyused in optimization,
including large-scale optimization viatruncated-Newton approaches,
its implementation in generalNewton-Krylov methods or other Newton-iterative methods canbe problematic.
In this talk,
we first outline a doglegmethod suitable for the general inexact Newton context andprovide a global convergence analysis for it.
We then discusscertain issues that may arise with standard doglegimplementational strategies and propose modified strategiesthat avoid them.
We conclude with a report on numericalexperiments involving benchmark CFD problems.

\end{document}
