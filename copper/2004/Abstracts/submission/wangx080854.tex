\documentclass{report}
\usepackage{amsmath,amssymb}
\setlength{\parindent}{0mm}
\setlength{\parskip}{1em}
\begin{document}
\begin{center}
\rule{6in}{1pt} \
{\large Xue Wang \\
{\bf Preconditioned Solenoidal Basis Method for Saddle-Point Problems}}

Department of Computer Science \\ Texas A\&M University \\ College Station \\ TX 77843-3112
\\
{\tt wangx@cs.tamu.edu}\\
Vivek Sarin\end{center}

Saddle-point systems arise in a number of applications where conservation
laws are enforced through discrete linear constraints. For incompressible
fluids, a divergence-free constraint forces the fluid velocity to lie in
the null space of a constraint matrix. This talk describes the solenoidal
basis method for solving saddle-point problems. The solenoidal basis
method is a projection technique that uses a discrete divergence-free
basis to satisfy the linear constraints. A reduced system in the
projected space is solved by an iterative method. The main challenges in
this approach include construction of the discrete solenoidal basis and
preconditioning the reduced system. In this talk, we present algebraic
schemes to construct sparse and dense bases for the null space of the
constraint matrix. These bases can be used to represent the discrete
divergence-free subspace. We also outline preconditioning techniques for
the reduced system and discuss the performance of the preconditioned
solenoidal method.


\end{document}
