\documentclass{report}
\usepackage{amsmath,amssymb}
\setlength{\parindent}{0mm}
\setlength{\parskip}{1em}
\begin{document}
\begin{center}
\rule{6in}{1pt} \
{\large Suely Oliveira \\
{\bf Algebraic Multigrid (AMG) for saddle point systems from meshfree discretizations}}

The University of Iowa \\ Department of Computer Science \\ 14 MLH \\ Iowa City \\ IA 52242
\\
{\tt oliveira@cs.uiowa.edu}\end{center}

Meshfree discretizations construct approximate solutions to partial
differential equation based on particles, not on meshes, so that it
is well suited to solve the problems on irregular domains. Since the
nodal basis property is not satisfied in meshfree discretizations,
it is difficult to handle essential boundary conditions. In this paper,
we employ the Lagrange multiplier approach to solve this problem,
but this will result in an indefinite linear system of a saddle point
type. We adapt a variation of the smoothed aggregation AMG method
of Van\v{e}k, Mandel \& Brezina to this saddle point system. We give
numerical results showing that this method is practical and competitive
with other methods with convergence rates that are $\sim c/\log N$.

Meshfree discretizations are Galerkin approximations to the weak form
of partial differential equations, where each unknown corresponds to a
``particle'' --- a smooth function with compact support. In most meshfree
methods these particle functions are not arbitrary, but are constructed
to satisfy certain properties in order to achieve good approximation
properties. In particular, we work with Reproducing Kernel Particle
Methods (RKPM's). In RKPM's, an initial collection of smooth functions
with compact support (kernel functions $\Phi_i$ with associated nodes $x_i$)
are processed to construct new smooth functions with compact support
(basis functions $\Psi_i$) which satisfy the discrete reproducing condition
\[ \sum_i x_i^\alpha \, \Psi_i(x) = x^\alpha \]
where $\alpha$ is any multiindex $|\alpha|\leq p$.

Meshfree discretizations are Galerkin approximations to the weak form
of partial differential equations, where each unknown corresponds to a
``particle'' --- a smooth function with compact support. In most meshfree
methods these particle functions are not arbitrary, but are constructed
to satisfy certain properties in order to achieve good approximation
properties. In particular, we work with Reproducing Kernel Particle
Methods (RKPM's). In RKPM's, an initial collection of smooth functions
with compact support (kernel functions $\Phi_i$ with associated nodes $x_i$)
are processed to construct new smooth functions with compact support
(basis functions $\Psi_i$) which satisfy the discrete reproducing condition
\[ \sum_i x_i^\alpha \, \Psi_i(x) = x^\alpha \]
where $\alpha$ is any multiindex $|\alpha|\leq p$.

Since there is no particular relationship between the nodes and the kernel
functions, the nodal basis property fails for the RKPM and other standard
meshfree methods. The nodal basis property can be re-established by
modifying the basis functions using a variety of techniques (singular kernel
functions, hybrid finite element/meshfree methods, etc.). Instead we
consider a Lagrange multiplier method. Lagrange multiplier methods are
often vulnerable to numerical instabilities if the discrete Babuska--Brezzi
conditions fail, as they typically do for meshfree methods. To overcome
this problem, we construct a separate family of basis functions for the
boundary for approximating the Lagrange multiplier function.

The starting point for our preconditioner is a smoothed aggregation
method of Van\v{e}k, Mandel \& Brezina. However, to construct our
smoothed interpolation operator we use a different way of deriving
the interpolation operator to that of Van\v{e}k \emph{et al}.

We have a large-scale saddle point system to solve. Since we need to
maintain the discrete Babuska--Brezzi condition for the grid coarsenings,
we also coarsen the boundary discretization. Smoothed aggregation
techniques similar to those for the interior of the domain, are used
to construct the boundary interpolation and restriction operators.
The coarse grid operators are constructed using a Galerkin approach ---
pre-multiplying the fine-grid saddle-point operator with the block
diagonal matrix consisting of the interpolation operators for
the interior and the boundary, and post-multiplying by the transpose of the
above block-diagonal matrix. We use a saddle-point JOR-type method
for the pre- and post-smoother. The resulting multigrid V-cycle operator
is then used to precondition GMRES.

The numerical results we obtain are certainly encouraging, giving a number
of iterations which grows like $c\,\log N\log(1/\epsilon)$ where $N$ is the
number of unknowns and $\epsilon$ is the error tolerance.
Unfortunately, the number of flops per V-cycle is not constant because
of reductions in the sparsity of the course-grid operators, and the smoother
used. In spite of these problems, the method out-performed the methods
we compared it with (including straight aggregation, using the
smoother as a preconditioner, and no preconditioner at all),
especially for larger problems. Good performance can be seen
both in the number of iterations and the total time taken.
This is joint work with Koung Hee Leem and David Stewart.


\end{document}
