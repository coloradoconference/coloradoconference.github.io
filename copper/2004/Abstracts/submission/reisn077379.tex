\documentclass{report}
\usepackage{amsmath,amssymb}
\setlength{\parindent}{0mm}
\setlength{\parskip}{1em}
\begin{document}
\begin{center}
\rule{6in}{1pt} \
{\large J. Reisner \\
{\bf A fully-implicit hurricane model with physics-based preconditioning}}

Los Alamos National Laboratory \\ Mail Stop D401/EES-2 \\ Los Alamos \\ NM 87545
\\
{\tt reisner@lanl.gov}\\
V. Mousseau\\
A. Wyszogrodzki\\
	Knoll, D.\end{center}

A numerical framework for simulating hurricanes based
upon solving a nonlinear equation set in a consistent
manner without time splitting is described in this
paper. The physical model is the Navier-Stokes
equations plus a highly simplified and differentiable
microphysics parameterization package. Because the
method is fully implicit, the approach is able to
employ time steps that result in
Courant-Friedrichs-Lewy (CFL) numbers greater than one
for advection, gravity, and sound waves; however, the
dynamical time scale of the problem must still be
respected for accuracy. The physical model is solved
via the Jacobian-Free Newton-Krylov (JFNK) method. The
JFNK approach typically requires the approximate
solution of a large linear system several times per
time step. To increase the efficiency of the linear
system solves, a physics-based preconditioner has been
employed. To quantify the accuracy and efficiency of
the new approach against traditional time-split
approaches, the fully-implicit solver was first
compared against the semi-implicit approach for the
simulation of a precipitating moist bubble. The
moist-bubble simulations not only demonstrated the
ability of the fully-implicit approach to achieve
second-order accuracy in time, but also the ability of
the fully-implicit approach to achieve a given level of
accuracy in a more efficient manner than traditional
approaches. This behavior is further illustrated in
first-of-a-kind three-dimensional fully-implicit
hurricane simulations that reveal the semi-implicit
algorithm needs to take a time step at least 60 times
smaller than the fully-implicit algorithm to produce a
comparable change in the intensity of a hurricane.


\end{document}
