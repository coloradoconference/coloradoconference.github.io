\documentclass{report}
\usepackage{amsmath,amssymb}
\setlength{\parindent}{0mm}
\setlength{\parskip}{1em}
\begin{document}
\begin{center}
\rule{6in}{1pt} \
{\large Gregory, A Newman \\
{\bf An AMG solver for the 3D time-harmonic Maxwell equations}}

Earth Science Division \\ Lawrence Berkeley National Laboratory \\ One Cyclotron \\ MS 90-1116 \\ Berkeley Ca \\ 94720
\\
{\tt ganewman@lbl.gov}\\
Jonathon Hu\\
Ray Tuminaro\end{center}

We describe a parallel algebraic multigrid (AMG) method for the solution
of the 3D Maxwell�s equations in the frequency domain in the quasi-static
limit. This AMG method is intended as a preconditioner to a Krylov
iterative method such as quasi-minimum residual, where the application of
interest is the repeated solution of the forward modeling problem arising
in geophysical subsurface imaging applications. The underlying
formulation of the new frequency-domain AMG method leverages off of an
AMG scheme for real values Maxwell problems arising from the
time-dependent eddy current equations. The central components of the real
valued AMG method are distributed relaxation for the smoother and a
specialized grid transfer operator for the coarse grid correction. The
key to this AMG method is the proper representation of the (curl, curl)
null space on coarse meshes. This is achieved by maintaining certain
commuting properties of the inter-grid transfers. To adapt the real
valued AMG scheme to complex arithmetic, the complex operator is first
written as a 2x2 real block matrix system, called an equivalent real form
(ERF). The inter-grid transfers for the ERF are then generated via the
real valued AMG algorithm. The smoother for the ERF is also adapted from
the real value AMG method. This distributed relaxation on the ERF matrix
in turn leads to a nice decoupling of the problem. To complete the
method, a variety of smoothers, including complex polynomial, jacobi and
one-step krylov smoothers are developed for use within the distributed
relaxation process. While some care is required to develop these
smoothers, they work well in parallel and avoid difficulties associated
with parallel Gauss-Seidel. Numerical experiments are presented for some
3D problems arising in geophysical subsurface imaging applications. The
experiments illustrate the efficiency of the approach on various parallel
machines in terms of both convergence and parallel speed-up.


\end{document}
