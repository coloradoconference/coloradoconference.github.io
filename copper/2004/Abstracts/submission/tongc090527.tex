\documentclass{report}
\usepackage{amsmath,amssymb}
\setlength{\parindent}{0mm}
\setlength{\parskip}{1em}
\begin{document}
\begin{center}
\rule{6in}{1pt} \
{\large Charles Tong \\
{\bf Parallel Performance of Algebraic Multigrid Methods for Structural Dynamics }}

Lawrence Livermore National Lab \\ MS 560 \\ P O 808 \\ 7000 East Avenue \\ Livermore \\ CA 94551-0808
\\
{\tt chtong@llnl.gov}\\
Rich Becker\end{center}

Multigrid preconditioning have been demonstrated to be efficient solution
techniques for many scientific problems. The idea of multigrid is to
capture errors at different scales by using grids of different fineness.
By traversing and relaxing between the fine and the coarse grids, optimal
convergence rates are often observed. In adidition, the computation on
each grid can be performed in parallel, making it relatively suitable for
parallel implementation (the degree of parallelism, however, decreases
rapidly with coarser grids). The difficulties in using geometric multigrid
for unstructured grid problems have prompted the development of algebraic
multigrid methods.

In this talk we present a study of parallel performance for a number of
algebraic multigrid (AMG) methods on a few selected structural mechanics
problems. The first one is the classical algebraic multigrid by Ruge and
Stuben. This classical AMG has been implemented in the HYPRE solver
library with many modifications for parallel computations. A second AMG
method is the smoothed aggregation method proposed by Vanek, Brezina, and
others. We also introduce a substructure-based aggregation AMG method that
uses local eigenvector information for construction prolongation operators
for smoothed aggregation multigrid. We will discuss scalability and robustness
issues with these algebraic multigrids.



This work was performed under the auspices of the U.S. Department of
Energy by the University of California, Lawrence Livermore National
Laboratory under Contract No. W-7405-Eng-48.


\end{document}
