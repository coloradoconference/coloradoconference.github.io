\documentclass{report}
\usepackage{amsmath,amssymb}
\setlength{\parindent}{0mm}
\setlength{\parskip}{1em}
\begin{document}
\begin{center}
\rule{6in}{1pt} \
{\large J David Moulton \\
{\bf Performance Tuning of Parallel Structured Multigrid}}

MS B284 \\ Los Alamos National Laboratory \\ Los Alamos \\ NM 87545
\\
{\tt moulton@lanl.gov}\\
T. Austin\\
B. K. Bergen\\
	Berndt, M. 
	Dendy, J. E.\end{center}

The algorithmic scaling of robust variational multigrid methods has
generated significant interest in the development of efficient and
scalable parallel implementations, particularly on distributed memory
architectures. However, the hierarchical nature of these systems presents
significant challenges for both efficiency and scalability. In
particular, we are interested in both high-end proprietary clusters, such
as the Q supercomputer at LANL, and in commodity based Beowulf clusters.
In both cases the structure of the hierarchy is very similar (i.e, node
interconnect, local memory, local cache design, CPU pipelining) although
the performance of a particular component may vary significantly. Thus,
in both cases, the objective is to employ various techniques to minimize
the impact of this hierarchy on the overall performance of the code.

To this end we have implemented an MPI-based parallel version of the
Dendy's Black Box Multigrid Code (BoxMG) for structured grids in both two
and three dimensions. This code utilizes a customized version of the
Message Passing for Structured Grid Toolkit (MSG) from netlib, and
computational kernels from the serial BoxMG code. In this presentation we
will discuss profiling the serial components (e.g., with PAPI and DCPI),
focusing on the influence of data and loop structures that increase
cache-awareness. Also, we will discuss profiling the message passing
components of the code (e.g., with Vampir), focusing on implementation
issues that should improve parallel performance.


\end{document}
