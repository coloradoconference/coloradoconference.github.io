\documentclass{report}
\usepackage{amsmath,amssymb}
\setlength{\parindent}{0mm}
\setlength{\parskip}{1em}
\begin{document}
\begin{center}
\rule{6in}{1pt} \
{\large Doug Swesty \\
{\bf A comparison of iterative techniques for the solution of the time dependent Boltzmann equation for radiation transport.}}

Dept. of Physics & Astronomy \\ State University of NY at Stony Brook \\ Stony Brook \\ NY 11794-3800
\\
{\tt dswesty@mail.astro.sunysb.edu}\end{center}

The discrete ordinates Boltzmann equation is often employed to
describe radiation transport and radiation hydrodynamics in a variety of
physical contexts. Usually the solution of this equation relies on the
use of implicit finite-difference, finite-volume, or finite-element
methods. The use of implicit methods gives rise to large, sparse linear
systems which mush be solved at every timestep of a simulation. One
widely used method of iteratively solving this linear system involves
splitting the linear system through a method known as {\it
source-iteration}. This particular iterative method has been favored by
the nuclear
engineering community. However, this method suffers from a number of
difficulties including convergence problems in the diffusive limit and
algorithmic impediments to parallel implementation. An alternative method
for iteratively solving the linear system is to attack the full, i.e.
unsplit, linear system with Krylov subspace techniques combined with
parallel preconditioners. In this talk we present results from a
comparison of the source-iteration and
full-linear-system approaches on a series of radiation transport test
problems. Our study addresses the issues of both numerical and parallel
efficiency of these two different iterative methods.


\end{document}
