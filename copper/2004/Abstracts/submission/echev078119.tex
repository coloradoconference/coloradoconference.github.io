\documentclass{report}
\usepackage{amsmath,amssymb}
\setlength{\parindent}{0mm}
\setlength{\parskip}{1em}
\begin{document}
\begin{center}
\rule{6in}{1pt} \
{\large D. Echeverria Ciaurri \\
{\bf Optimization in Electromagnetics with the Space-Mapping Technique}}

P.O. Box 94079 \\ 1090 GB Amsterdam \\ The Netherlands
\\
{\tt D.Echeverria@cwi.nl}\\
P.W. Hemker\end{center}

The Space-Mapping (SM) [1] technique was introduced as an efficient design optimization
tool for microwave circuits. The underlying idea however is quite general
and applicable to a broader class of minimization problems.

Optimization procedures in practice are based on high accuracy models that typically
have an excessive computational cost. Within SM terminology, these models are called
{\it fine models} and we will denote them as ${\mathbf f}({\mathbf x})$,
${\mathbf x} \in X$ being the design parameters. A finite element solution of Maxwell's
equations is an example of a model of this type. SM needs a second and computationally
cheaper model. This is called the {\it coarse model} and acts as a surrogate
for the fine one. We will denote it as ${\mathbf c}({\mathbf z})$, ${\mathbf z} \in Z$
being the design parameters. An equivalent electromagnetic circuit can for instance be
used as a coarse model.

The SM method relies on the coarse model to speed up the fine model optimization. The key
element in this technique is the {\it SM function} ${\mathbf p}:X \rightarrow Z$, also
known as {\it parameter extraction}. Its purpose is to obtain a relation
between the design
parameters of the fine and coarse models. With this relation both models become
as similar as possible by aligning their two responses:
${\mathbf f}({\mathbf x}) \approx {\mathbf c}({\mathbf p}({\mathbf x}))$.
The SM algorithms come in two usually equivalent brands. On one hand, the
{\it original SM}
minimizes the coarse model cost function and translates afterwards the result to the
fine one via the SM function. On the other hand, the {\it dual SM}
optimizes the mapped coarse
model ${\mathbf c}({\mathbf p}({\mathbf x}))$. None of these two algorithms is consistent
in the sense that the fine model optimal solution is always achieved. In
order to do that,
some other classical minimization methods must be combined with SM [2]. The SM technique
can be then either used as a solver or as a preconditioner, depending on
the desired accuracy.

In the first part of the talk, SM theory will be analyzed from the well understood
Defect Correction (DC) [3] framework. DC uses the same multiscale principles that SM
in order to solve an operator equation. This alternative perspective of SM explains
its reported success in optimization tasks [1,4]. Moreover, DC suggests the multilevel
SM generalization in a manner analogous to the multigrid philosophy. If a hierarchy
of models with respect to computational cost and precision is available, the multilevel
SM approach can be applied like full multigrid.

In the second part of the talk we will show results of SM applied to
electromagnetic shape
optimization. The first design problem is a C-shaped magnetic circuit in which a given
flux density is desired in a certain region in space. In the second
problem we optimize the
force response in a linear actuator. For this actuator a hierarchy of
models is available,
allowing then the use of the multilevel SM approach. For both problems the fine model
is a vector potential formulation discretized by finite elements. The results give
evidence of the promising nature of the SM technique for optimization in
electromagnetics.

\section*{References}
\begin{description}

\item[1] J.W. Bandler, R.M. Biernacki, S.H. Chen, P.A. Grobelny and R.H. Hemmers,
"{\it Space mapping technique for electromagnetic optimization}", IEEE Trans. Microwave
Theory Tech, vol. 42, pp. 2536-2544, 1994.

\item[2] K. Madsen and J. S\o ndergaard, "{\it Convergence of hybrid
space mapping algorithms}",
submitted, Optimization and Engineering, 2003.

\item[3] K. B\"{o}hmer, P.W. Hemker and J. Stetter, "{\it The defect
correction approach}", in Defect
Correction Methods: Theory and Applications, K. B\"{o}hmer and H.J.
Stetter eds., Computing Suppl. 5,
Springer-Verlag, Berlin, Heidelberg, New York, Tokyo, pp. 1-32, 1984.

\item[4] M. Bakr, J. Bandler, K. Madsen and J. S\o ndergaard, "{\it
Review of the space mapping
approach to engineering optimization and modeling}", Optimization and
Engineering, vol. 1,
pp. 241-276, 2000.

\end{description}


\end{document}
