\documentclass{report}
\usepackage{amsmath,amssymb}
\setlength{\parindent}{0mm}
\setlength{\parskip}{1em}
\begin{document}
\begin{center}
\rule{6in}{1pt} \
{\large Ning Kang \\
{\bf Parallel Preconditioning in the Analysis of Anisotropic Diffusion Simulation with the Human Brain Diffusion Tensor MRI Data}}

Department of Computer Science \\ University of Kentucky \\ 773 Anderson Hall \\ Lexington \\ KY 40506-0046
\\
{\tt nkang2@cs.uky.edu}\\
Jun Zhang\\
Eric S. Carlson\end{center}

We conduct simulations for the 3D unsteady state anisotropic diffusion process in
the human brain by discretizing the governing diffusion
equation on Cartesian grid and adopting a high performance
differential-algebraic
equation (DAE) solver, the parallel version of
implicit differential-algebraic (IDA) solver,
to tackle the resulting large scale system of DAEs.
Parallel preconditioning techniques
including sparse approximate inverse and banded-block-diagonal preconditioners
are used with the GMRES method to accelerate the
convergence rate of the iterative
solution. We then investigate and compare the efficiency and
effectiveness
of the two parallel preconditioners. The computational results of the
diffusion simulations on a parallel supercomputer show that the sparse
approximate inverse preconditioning strategy,
which is robust and efficient with good scalability,
gives a much better overall
performance than the banded-block-diagonal preconditioner.


\end{document}
