\documentclass{report}
\usepackage{amsmath,amssymb}
\setlength{\parindent}{0mm}
\setlength{\parskip}{1em}
\begin{document}
\begin{center}
\rule{6in}{1pt} \
{\large Eunjung,EJ Lee \\
{\bf LL* method for Eddy current problem in 3D with edges.}}

Department of Applied Mathematics \\ 526 UCB \\ University of Colorado at Boulder \\ Boulder \\ CO  80309-0526
\\
{\tt eunjung@colorado.edu}\\
Thomas Manteuffel\end{center}

\documentclass{article}

\usepackage{amsmath}

\newcommand{\curl}{\nabla\times}
\newcommand{\mydiv}{\nabla\cdot}
\newcommand{\bH}{\boldsymbol{H}}
\newcommand{\bE}{\boldsymbol{E}}
\newcommand{\bU}{\boldsymbol{U}}


\begin{document}

Maxwell's equations are a set of fundamental equations governing
all macroscopic electromagnetic phenomena. The equations can be
written in both differential and integral form, but we present
them only in simplified differential form, so called Eddy current
problem. Here we consider the following two basic laws of
electricity and magnetism:
\begin{eqnarray*}
\text{Faraday's Law} && \frac{\partial \mu \bH}{\partial t} + \curl \bE = 0\\
\text{Amp$\grave{\text{e}}$re's Law} && \curl \bH - \sigma \bE = 0.
\end{eqnarray*}
Here we consider the $L L^*$ method to solve least squares of the
first order system for partial differential equations. The main
goal of the $L L^*$ approach was to use a first order system least
squares method for partial differential equations, which do not
fulfill the regularity requirements of the standard first order
system least squares method. The main idea of the first order
system least squares method with the first order equations
$L\bU=\boldsymbol{F}$ is minimizing the functional
$||L\bU-\boldsymbol{F}||_0$, whose bilinear part is equivalent to
the product $H^1$ norm. The standard FOSLS method approximates to
the unknown $\bU$ in the given $H^1$ finite element spaces. But
this $H^1$-equivalence is provided only under sufficient
smoothness assumptions on the original problem like the domain,
coefficients, and data. $LL^*$ method was introduced to overcome
the problem of standard FOSLS coming from lack of sufficient
smoothness by solving the dual problem $L^*\boldsymbol{W} =\bU$
with the dual variable $\boldsymbol{W}$ and the adjoint operator
$L^*$. So the original problem is recast as one of minimizing the
functional $||L^*\boldsymbol{W}-\bU||_0$ which has the same
minimizer of the functional
$||L^*\boldsymbol{W}||^2-2\left<\boldsymbol{W},
\boldsymbol{F}\right>$. \\
FOSL$\text{L}^*$ has been successfully performed to achieve an
accurate approximation using $H^1$-conforming finite element
spaces for the PDE having discontinuous coefficients or irregular
boundary points in the bounded plane. But that known method is
hard to apply for 3D singular problems specially for the case
having irregular boundary conditions. So we focus on the problem
in the domain having irregular boundaries, i.e., reentrant
corners. If domain is bounded and either boundary is
$\mathcal{C}^{1,1}$ or domain is a convex polyhedron that is given
some boundary conditions, then $H(\mydiv)\cap H(\curl)$ is
continuously imbedded into $H^1(\Omega)^3$. Since boundary has
reentrant corners, the original solution $\bU$ is not anymore in
$H^1$ and also the dual solution $\boldsymbol{W}$ is not in $H^1$
which cannot be approximated by $H^1$ finite element spaces. That
is, in the case that the domain has edges, the standard finite
element method loses its global accuracy because of the
singularities on the boundary. We modify the dual operator $L^*$
and look for a sequence in $H^1$ converging to the dual solution
$\boldsymbol{W}$. Here we don't care the injectivity of $L^*$
because we want to find a solution $\boldsymbol{W}$ for
$L^*\boldsymbol{W}=\bU$.

\end{document}


\end{document}
