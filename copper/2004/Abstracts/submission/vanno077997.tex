\documentclass{report}
\usepackage{amsmath,amssymb}
\setlength{\parindent}{0mm}
\setlength{\parskip}{1em}
\begin{document}
\begin{center}
\rule{6in}{1pt} \
{\large T. L. Van Noorden \\
{\bf Optimal Time-Dependent Perturbations: A Constraint Optimization Problem}}

Utrecht University \\ Department of Mathematics \\ \\ Budapestlaan 6 \\ 3584 CD \\ Utrecht \\ The Netherlands
\\
{\tt noorden@math.uu.nl}\\
J. Barkmeijer\end{center}

We discuss the efficient numerical solution of the following
constraint optimization problem.
Let $N$ be a (high dimensional) positive semi-definite operator, which can
be thought of in block form
\begin{eqnarray}
N=\left ( \begin{array}{cccc} N_{11} & N_{12} & \hdots & N_{1p} \\
N_{21} & N_{22} & \hdots & N_{2p} \\
\vdots & \vdots & \ddots & \vdots \\
N_{p1} & N_{p2} & \hdots & N_{pp} \end{array} \right ),
\end{eqnarray}
where $N_{ij}\in \mathbb{R}^{n\times n}$. We would like to find the vector
$x=(x_1,x_2,\hdots,x_p)$ that maximizes $x^TNx$ under the constraint
\begin{eqnarray}
||x_1||\leq 1,\,\,\,||x_2||\leq 1,\hdots,||x_p||\leq 1.
\end{eqnarray}
Usually we do not have the operator $N$ in matrix form. We can only
compute the action of the operator on a vector.
The motivation for studying this optimization problem is the
sensitivity analysis of large scale
systems of linear(ized) differential equations with respect
to time-dependent perturbations.

We show that the optimum is attained on the boundary of the set over which
is optimized, so that we may replace the constraints in (2) with
\begin{eqnarray}
||x_1||=||x_2||=\hdots=||x_p||=1.
\end{eqnarray}
These equality constraints allow us to derive an Euler-Lagrange system of
equations that has many connections with a symmetric eigenvalue problem:
\begin{eqnarray*}
\left ( \begin{array}{cccc} N_{11} & N_{12} & \hdots & N_{1p} \\
N_{21} & N_{22} & \hdots & N_{2p} \\
\vdots & \vdots & \ddots & \vdots \\
N_{p1} & N_{p2} & \hdots & N_{pp} \end{array} \right )
\left ( \begin{array}{c} x_1 \\ x_2 \\ \vdots \\ x_p \end{array} \right )&=&
\left ( \begin{array}{c} \lambda_1 x_1 \\ \lambda_2 x_2 \\ \vdots \\
\lambda_p x_p \end{array} \right ) \\
x_1^Tx_1&=&1 \\
&\vdots& \\
x_p^Tx_p&=&1
\end{eqnarray*}
We give a geometric characterization of the solutions to these
Euler-Lagrange equations and conjecture about the number
of solutions.

Based on the connections with the symmetric eigenvalue problem,
we study numerical methods that are variants of their eigenvalue
counterparts. We show that there exists an analogue of the power method
for our optimization problem. This power method consists of the iteration
of the map $G$ defined by
\begin{eqnarray}
G\left ( \begin{array}{c} x_1 \\ x_2 \\ \vdots \\ x_p \end{array} \right )
=\left ( \begin{array}{c} \frac{N_{11}x_1+N_{12}x_2+ \hdots +N_{1p}x_p}
{||N_{11}x_1+N_{12}x_2+ \hdots +N_{1p}x_p||} \\
\frac{N_{21}x_1+N_{22}x_2+ \hdots +N_{2p}x_p}
{||N_{21}x_1+N_{22}x_2+ \hdots +N_{2p}x_p||} \\
\vdots \\
\frac{N_{p1}x_1+N_{p2}x_2+ \hdots +N_{pp}x_p}
{||N_{p1}x_1+N_{p2}x_2+ \hdots +N_{pp}x_p||}
\end{array} \right ).
\end{eqnarray}
We prove convergence of the iterations of $G$ to local maxima.

Since the power method usually converges slowly, we propose a numerical
procedure to accelerate the convergence of the power method, which is
in fact an adopted version of the Arnoldi or Lanczos method for eigenvalue
problems \cite{golub}.

Using a number of test problems, we show that the proposed procedure is
effective, particularly for large scale problems.
Among the test problems is the Marshall and Molteni quasi-geostrophic model
\cite{molteni} for atmospheric dynamics.

Other numerical methods that could be used to numerically solve the
constraint optimization problem, are the optimization methods described
and developed by Smith \cite{smith}.
These methods are optimization methods that work on Riemannian manifolds.
The constraints in our optimization problem are such that we optimize over
a direct sum of $p$ $n$-dimensional spheres. This is a $p(n-1)$-dimensional
hyper-surface in $\mathbb{R}^{np}$.
The methods mentioned in \cite{smith} do not construct a search
space. This is advantageous for the computer memory usage.
However, we would like not only to find the global optimum, but also
a subspace comprising the most amplifying perturbations.
This is a feature of the variants of Krylov subspace methods.

\bibitem{golub}
G. H. Golub and C. F. Van Loan, {\it Matrix Computations}, John Hopkins
Studies in the Mathematical Sciences, John Hopkins University Press,
Baltimore, MD, third ed., 1996.

\bibitem{molteni}
J. Marshall and F. Molteni, {\it Towards a dynamical understanding of
planetary-scale flow regimes}, J. Atmos. Sci, {\bf 50}, (1993) 1792--1818.

\bibitem{smith}
S. T. Smith, {\it Optimization techniques on {R}iemannian manifolds}, in
Hamiltonian and gradient flows, algorithms and control, vol. 3 of Fields
Inst. Commun., Amer. Math. Soc., 1994, pp. 113--136.


\end{document}
