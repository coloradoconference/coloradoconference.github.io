\documentclass{report}
\usepackage{amsmath,amssymb}
\setlength{\parindent}{0mm}
\setlength{\parskip}{1em}
\begin{document}
\begin{center}
\rule{6in}{1pt} \
{\large Duane Rosenberg \\
{\bf The GASpAR code for geophysical turbulent flows}}

Geophysical Turbulence Program \\ NCAR \\ PO Box 3000 \\ Boulder CO 80307-3000
\\
{\tt duaner@ucar.edu}\\
Aim\'e Fournier\\
Annick Pouquet\end{center}

We present a status report on the Geophysical Astrophysical Spectral element
Adaptive Refinement (GASpAR) code developed within the Geophysical
Turbulence Program at NCAR.

Turbulent flows are ubiquitous, and as manifestations of one of the
last outstanding unsolved problems of classical physics, they form
today the focus of numerous investigations; they
are linked to many issues in the geosciences, {\it e.g.,} in meteorology,
oceanography, climatology, ecology, solar--terrestrial interactions
and fusion, as well as dynamo effects.
Nonlinearities prevail in turbulent flows when the
Reynolds number $\Rn$
is large; the number of degrees of freedom increases as
$\Rn^{9/4}$ for $\Rn\gg1$ in the Kolmogorov (1941) framework,
and for geophysical flows, often $\Rn> 10^8$.
The ability to probe large $\Rn$, and to examine in detail the
large-scale behavior of turbulent flows, depends
critically on the ability to resolve adequately such a large number of
spatial and temporal scales, or else to model them, or a combination of both.

One intriguing observation concerning turbulent flows
resides in the
departure from normality in the probability distribution
functions.
The origin of these fat wings is not understood;
one does not know yet what structures are key to our
understanding the statistical properties of turbulent flows ({\it e.g.,}
vortex sheets, spirals or filaments,
shocks or fronts, blobs, plumes or tetrads, knots, helices, tubes or
arches).
The link between structures and non-Gaussian statistics is the basis
for the notion of intermittency which plays a role {\it e.g.,} in
reactive flows, in convective plumes, in combustion or in the heating of
the solar corona.

Theory demands that computations of turbulent flows
reflect a clear scale separation between the energy-containing,
inertial (self-similar) and dissipative ranges.
Convergence
studies on compressible flows show that to achieve the desired
scale separation between the
energy-containing modes and the dissipation regime, it is necessary to
compute on regular grids of at least $2048^3$ cells (Sytine {\it et
al.}, {\it J.\ Comp.\ Phys.\ }{\bf 158}, 225, 2000).

Today such computations can barely be accomplished.
A pseudo-spectral Navier-Stokes code on a grid of $4096^3$ regularly spaced
points has been run on the Earth Simulator (Isihara {\it et al.},
{\it J.\ Phys.\ Soc.\ Japan} {\bf 72}, 983, 2003)
but the Taylor Reynolds number is still of the order of $500$, still very far
from what is asked for in geophysics.
Adaptivity seems to be needed, provided the
significant structures of the flow are indeed sparse so that their
dynamics be followed accurately although such structures are embedded
in random noise.

We have built an object-oriented code, GASpAR, that is flexible enough to
be applicable to a wide class of turbulent-flow and other multi-scale problems.
The computational core is based on spectral element operators, which are
represented as objects.
The formalism accommodates both conforming and
nonconforming elements, and their associated data structures for
handling inter-element communications in a parallel environment.
Dynamic
adaptive mesh (nonconforming {\it h}-type) refinement is provided for, and its
suitability for turbulent flow models will be examined.
The first application of the code will concern the Hwa-Kardar
({\it Phys.\ Rev.\ A} {\bf 45}, 7002, 1992) equations,
which arise when writing the dynamical version of self-organized criticality.
These equations can serve as a model of solar flares when viewed as overlapping
avalanches (Liu {\it et al.}, {\it Phys.\ Rev.\ E} {\bf 66}, 056111, 2002).
\vskip0.2truein

{\it Acknowledgments:}
We are thankful to Huiyu Feng, Paul Fisher and Catherine Mavriplis
for numerous helpful discussions, and to the CSS group at NCAR for
early help in using spectral element methods.


\end{document}
