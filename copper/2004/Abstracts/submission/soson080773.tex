\documentclass{report}
\usepackage{amsmath,amssymb}
\setlength{\parindent}{0mm}
\setlength{\parskip}{1em}
\begin{document}
\begin{center}
\rule{6in}{1pt} \
{\large Masha Sosonkina \\
{\bf Tuning a graph partitioner to improve parallel preconditioners}}

Ames Laboratory \\ Iowa State University \\ 236C Wilhelm Hall \\ Ames IA 50011
\\
{\tt masha@scl.ameslab.gov}\\
Yousef Saad\end{center}

Various graph partitioning algorithms are employed to partition a
sparse matrix for parallel processing, which exploit a number of graph
representations of a matrix. A standard approach is to represent rows
(or unknowns) as graph vertices and the nonzero entries as the edges
of the graph. The partitioning objective here has been to minimize the
edge cut while retaining a balance between the sizes of the
partitions. However, it is known \cite{hendrickson00graph} that such
graph partitionings are not necessarily optimal for certain simple
types of parallel computations, such as sparse matrix vector
products, and for problems with irregular structure.
Alternative approaches have been suggested to overcome this
shortcoming.
One of them is the hypergraph model, which generalizes the notion of a
graph by permitting edges defined by more than two vertices.
It has been shown that, for certain definitions of vertices and
hyperedges, the hypergraph model minimizes correctly
communication overhead for sparse matrix operations and
represents adequately non-symmetric matrices. Typically the
hypergraph approach, as well as other graph partitionings, disregards
the numerical values of matrix entries.
Numerical properties, however, are an important factor to
consider when partitioning a difficult linear system for an iterative
solution.
When matrix values are taken into account, the resulting subproblems may
be better conditioned if, for example, all strongly coupled coefficients
reside in the same subproblem \cite{saad-sosonkina-acpc99}. In this
talk, we show a way to define hyperedges and their weights based
on a preprocessing analysis of linear system. Specifically, first, we
compute a relative degree of diagonal dominance for each matrix
row. Then we construct a hyperedge to include a certain number of
adjacent vertices depending on this relative degree. The larger this
degree the smaller the number of vertices that constitute a
hyperedge. We also show how criteria of load
balancing and of minimization of communication overhead affect our
hyperedge definitions.

Numerical experiments with linear systems arising in real world
applications will be reported. We compare several graph partitioning
techniques, including hypergraph models with typical hyperedge
definitions. Along with measuring communication overhead and load
balance, we monitor the quality of the resulting distributed
preconditioner, keeping in mind that the ultimate goal is to produce
faster convergence in terms of number of iterations and execution
time.

\begin{thebibliography}{1}

\bibitem{hendrickson00graph}
Bruce Hendrickson and Tamara~G. Kolda.
\newblock Graph partitioning models for parallel computing.
\newblock {\em Parallel Computing}, 26(12):1519--1534, 2000.

\bibitem{saad-sosonkina-acpc99}
Y.~Saad and M.~Sosonkina.
\newblock Non-standard parallel solution strategies for distributed sparse
linear systems.
\newblock In A.~Uhl P.~Zinterhof, M.~Vajtersic, editor, {\em Parallel
Computation: Proc. of ACPC'99}, Lecture Notes in Computer Science, Berlin,
1999. Springer-Verlag.

\end{thebibliography}


\end{document}
