\documentclass{report}
\usepackage{amsmath,amssymb}
\setlength{\parindent}{0mm}
\setlength{\parskip}{1em}
\begin{document}
\begin{center}
\rule{6in}{1pt} \
{\large Erik B{\"a}ngtsson \\
{\bf Numerical simulation of glacial rebound using preconditioned iterative solution methods}}

Department of Information Technology \\ Box 337 \\ SE-751 05 Uppsala \\ Sweden
\\
{\tt erikba@it.uu.se}\\
Maya Neytcheva\end{center}

We consider the problem to compute the stress ($\sigma$) and
displacement ($\bf{u}$) fields in a (visco)elastic inhomogeneous layered
media,in response to a surface load. The underlying physical phenomenon,
which is modeled, is glacial advance and recession, and the post-glacial
rebound caused by the latter, which reflects the viscoelastic properties
of the mantle.

Recently, this problem has attracted much attention
and simulations are performed using available commercial
finite element packages, the consequences of this being that

\begin{tabular}{cp{5.62in}}
(a)&only direct solution methods are used which entail in general high
demands on the computer resources and\\
(b)& almost all well-tested finite element packages are
engineering-oriented and are designed to solve the stiffness equation
$\nabla\underline{\sigma} + {\bf f} = 0$,
with ${\bf f}$ being the acting forces, which turns out to be overly
simplified for geophysical applications and does not include some very
important phenomena, such as the so-called advection of pre-stress, for
instance.
\end{tabular}

The material incremental momentum equation for quasi-static infinitesimal
perturbations of a stratified, compressible fluid Earth, initially in
hydrostatic equilibrium, subject to gravitational forces but neglecting
internal forces (cf. \cite{KlemannWuWolf}) is
$$
\underset{(A)}{\underbrace{\nabla\cdot\sigma}} +
\underset{(B)}{\underbrace{\nabla(\bf{u}\cdot\nabla p^{(0)})}} +
\underset{(C)}{\underbrace{\rho^{(\Delta)}\bf{g}^{(0)}}} +
\underset{(D)}{\underbrace{\rho^{(0)}\nabla\bf{g}^{(\Delta)}}} = \bf{0}.
$$
Here, term $(A)$ describes the force from spatial gradients in stress. If
a large elastic solid is put in a gravitational field, it becomes
gravitationally pre-stressed with pressure $p^{(0)}$. This pressure can
be regarded as an initial condition imposed on the problem and does not
cause deformations. Term $(B)$ represents the advection of this
pre-stress and describes how it is carried by the moving
material. Terms $(C)$ and $(D)$ describe perturbations of the
gravitational force and gravitational acceleration due to changes of
density.

In the present study, an incompressible non-selfgravitating (flat) Earth
model is used, which implies constant gravity field and constant density,
so that these two terms vanish. Term (B)
is further simplified assuming that the advection term describes the
advection in the direction of the gravity field only.

Incorporating the above simplifications with respect to terms
$(B)$, $(C)$ and $(D)$, we obtain the following form of the governing
equilibrium equation
\begin{equation}
\begin{array}{l}
\nabla\cdot \sigma +
\rho^{(0)}g^{(0)} \nabla(u_d) = \mathbf{0} \quad
\mathbf{x}\in\Omega\subset\mathbb{R}^d, d=2,3
\end{array}
\label{eq_rebound2}
\end{equation}
with suitable boundary conditions.

In its full complexity, the model includes viscoelastic constitutive
relations. In this work we discuss a purely elastic material behavior
only, as is analyzed in \cite{KlemannWuWolf}, for instance.

Problem (\ref{eq_rebound2}) is discretized using stable mixed finite
element pairs or a suitable stabilized formulation,
which lead to a system of linear equations with a nonsymmetric
two-by-two block matrix of a saddle point form.

Results from numerical experiments solving the so-arising algebraic
system with preconditioned iterative solution methods are presented.
Several preconditioning strategies are tested, based on the techniques
and experience described in \cite{Klawonn}, \cite{elman},
\cite{AxelssonNeytcheva}, \cite{AxelssonNeytcheva1} and other authors.
The performance of the tested preconditioned iterative solution methods
is compared with that of a commercial FEM package solver.

\begin{thebibliography}{ABC}
\bibitem{AxelssonNeytcheva}
Axelsson O, Neytcheva M.
Preconditioning methods for linear systems arising in constrained optimization problems,
{\em Numerical Linear Algebra with Applications}, 10 (2003), 3-31.

\bibitem{AxelssonNeytcheva1}
Axelsson O, Neytcheva M.
Preconditioning methods for for constrained optimization problems with
applications for the linear elasticity equations
Report 0302, January 2003, Department of Mathematics,
University of Nijmegen, The Netherlands.

\bibitem{elman}
Elman H, Silvester D, Wathen A. Performance and analysis of saddle point
preconditioners for the discrete steady-state Navier-Stokes equations.
\textit{Numerische Mathematik}, {\bf 90} (2002) 665-688.

\bibitem{Klawonn} Klawonn A. Block-triangular preconditioners for saddle
point problems with a penalty term. {\em SIAM Journal on Scientific
Computing} 1998;
19:172-184.

\bibitem{KlemannWuWolf}
Klemann V, Wu P, Wolf D. Compressible viscoelasticity: stability of
solutions for homogeneous plane-Earth models,
\textit{Geophysical Journal}, 2003; \textbf{153}:569-585.

\end{thebibliography}


\end{document}
