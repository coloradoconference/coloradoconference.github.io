\documentclass{report}
\usepackage{amsmath,amssymb}
\setlength{\parindent}{0mm}
\setlength{\parskip}{1em}
\begin{document}
\begin{center}
\rule{6in}{1pt} \
{\large Amik St-Cyr \\
{\bf On Optimized Schwarz Preconditioning for High-Order Spectral Element Methods}}

National Center for Atmospheric Research \\ 1850 Table Mesa Drive \\ \\ Boulder \\ CO 80305.
\\
{\tt amik@ucar.edu}\\
Martin. J. Gander\\
Stephen J. Thomas\end{center}

\begin{document}
\title{On Optimized Schwarz Preconditioning for High-Order
Spectral Element Methods}
\author{}
\maketitle
\begin{abstract}
Optimized Schwarz preconditioning is applied to a spectral element method
for the modified Helmholtz equation and pseudo-Laplacian arising in
incompressible flow solvers. The preconditioning is performed on an
element-by-element basis. The method enables one to use non-overlapping
elements, yielding an effective algorithm in terms of communication
between elements and implementation. Two approaches are tested. The first
consists of constructing a $P_1$ finite element problem on each
overlapping element. In the second, the preconditioner is applied
directly on a non-overlapping spectral element. Numerical results demonstrate
an improvement in the iteration count over the classical Schwarz algorithm.

\nonindent {\bf Introduction}
The classical Schwarz algorithm uses Dirichlet transmission conditions
between subdomains. By introducing a more general Robin boundary
condition, it is possible to optimize the convergence characteristics
of the original algorithm (Charton et al. 1991; Chevalier et al. 1998;
Gander et al 2002; Gander 2003). In this work, a study of
the model equations $u - \Delta u = f$ and pseudo-Laplacian arising
in incompressible flow solvers is performed. As suggested by the work
of Fischer et al. (2000), the preconditioning is either implemented via a $P_1$
finite element formulation of the original problem build on the spectral
element grid, or directly by solving a smaller spectral element problem
without overlap on each spectral element.
Although traditional Schwarz preconditioning combined with a coarse
grid solver is quite efficient, the need for even more powerful
preconditioning techniques stems from atmospheric modeling. Recently
(see Thomas and Loft 2002; St-Cyr and Thomas 2004), a semi-implicit SEM was combined with
OIFS time stepping (Maday et al. 1990), enabling time steps on the order of
20 times the advective CFL condition (Xiu and Karniadakis 2001). This directly reflects
as a significant increase in the number of conjugate gradient iterations
required to perform the semi-implicit step.

\tiny
\noindent {\sc P. Charton, F. Nataf and F. Rogier (1991)},
{\em M\'ethode de d\'ecomposition de domaines pour l'\'equation d'advection-diffusion},
C. R. Acad. Sci., Vol. 313, No. 9, pp. 623-626.

\noindent {\sc P. Chevalier and F. Nataf (1998)},
{\em Symmetrized method with optimized second-order conditions for the
Helmholtz equation},
In Domain decomposition methods, 10 (Boulder, CO, 1997), pp. 400-407, Amer. Math. Soc.,
Providence, RI.

\noindent {\sc P.F. Fischer, N.I. Miller and H.M. Tufo (2000)},
{\em An overlapping Schwarz method for spectral element simulation of
three-dimensional
incompressible flows},
in Parallel Solution of Partial Differential Equations, P. Bjorstad and M.
Luskin, eds.,
Springer-Verlag, pp.159-180.

\noindent {\sc M. J. Gander}, {\em Optimized Schwarz Methods (2003)},
Research Report, No. 2003-01, Dept. of Mathematics and Statistics,
McGill University, 33 pages, submitted.

\noindent {\sc M.J. Gander, F. Magoules and F. Nataf (2002)},
{\em Optimized Schwarz Methods without Overlap for the Helmholtz
Equation},
SIAM Journal on Scientific Computing, Vol. 24, No 1, pp. 38-60.

\noindent {\sc J.W. Lottes and P.F. Fischer (2003)},
{\em Hybrid Multigrid/Schwarz Algorithms for the Spectral Element Method},
submitted.

\noindent {\sc Y.~Maday, A.~T.~Patera, and E.~M.~R{\o}nquist (1990)},
{\em An operator-integration-factor splitting method for time-dependent
problems: application to incompressible fluid flow},
J.~Sci.~Comput., {\bf 5}(4), pp.~263--292.

\noindent {\sc A. St-Cyr and S.J. Thomas (2004)},
{\em Non-linear operator integration factor splitting for the shallow
water equations},
In preparation for J. Sci. Comp.

\noindent {\sc S. J. Thomas, J. M. Dennis, H. M. Tufo, and P. F. Fischer (2003)},
{\em A Schwarz Preconditioner for the Cubed-Sphere},
SIAM J. Sci. Comp., Vol. 25, No. 2, pp. 442-453.

\noindent {\sc S.J. Thomas and R.D. Loft (2002)},
{\em Semi-implicit spectral element atmospheric model}.
Journal of Scientific Computing, vol 17, 339-350.

\noindent {\sc D. Xiu and G.E. Karniadakis (2001)},
{\em A semi-Lagrangian high-order method for Navier-Stokes equations},
J.C.P., No. 172, pp. 658-684.
\end{abstract}
\end{document}


\end{document}
