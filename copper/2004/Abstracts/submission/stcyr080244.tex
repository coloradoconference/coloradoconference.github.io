\documentclass{report}
\usepackage{amsmath,amssymb}
\setlength{\parindent}{0mm}
\setlength{\parskip}{1em}
\begin{document}
\begin{center}
\rule{6in}{1pt} \
{\large Amik St-Cyr \\
{\bf On Optimized Schwarz Preconditioning for High-Order Spectral Element Methods}}

National Center for Atmospheric Research \\  \\ 1850 Table Mesa Drive \\  \\ Boulder \\ CO 80305.
\\
{\tt amik@ucar.edu}\\
Martin. J. Gander\\
Stephen J. Thomas\end{center}

\begin{document}
\title{On Optimized Schwarz Preconditioning for High-Order
Spectral Element Methods}
\author{
A.~St-Cyr\footnote{{\bf amik@ucar.edu},
NCAR 1850 Table Mesa Drive Boulder, CO 80305, USA.}, \;\;
M.J. Gander\footnote{{\bf mgander@math.mcgill.ca},
McGill 805 Sherbrooke W. Montreal QC,Canada H3A2K6} \; and
S.~J.~Thomas\footnote{{\bf thomas@ucar.edu},
NCAR 1850 Table Mesa Drive Boulder, CO 80305, USA.}}
\maketitle
\begin{abstract}
Optimized Schwarz preconditioning is applied to a spectral element method
for the modified Helmholtz equation and pseudo-Laplacian arising in
incompressible flow solvers. The preconditioning is performed on an
element-by-element basis. The method enables one to use non-overlapping
elements, yielding an effective algorithm in terms of communication
between elements and implementation. Two approaches are tested. The first
consists of constructing a $P_1$ finite element problem on each
overlapping element. In the second, the preconditioner is applied
directly on a non-overlapping spectral element. Numerical results demonstrate
an improvement in the iteration count over the classical Schwarz algorithm.

\small
\noindent {\bf Introduction}

The classical Schwarz algorithm uses Dirichlet transmission conditions
between subdomains. By introducing a more general Robin boundary
condition, it is possible to optimize the convergence characteristics
of the original algorithm \cite{cnr91,cn98,gmn02,g04}. In this work, a study of
the model equations $u - \Delta u = f$ and pseudo-Laplacian arising
in incompressible flow solvers is performed. As suggested by the work
of \cite{fmt00}, the preconditioning is either implemented via a $P_1$
finite element formulation of the original problem build on the spectral
element grid, or directly by solving a smaller spectral element problem
without overlap on each spectral element.
Although traditional Schwarz preconditioning combined with a coarse
grid solver is quite efficient, the need for even more powerful
preconditioning techniques stems from atmospheric modeling. Recently
(see \cite{tl02,st04}), a semi-implicit SEM was combined with
OIFS time stepping, enabling time steps on the order of
20 times the advective CFL condition \cite{xk01}. This directly reflects
as a significant increase in the number of conjugate gradient iterations
required to perform the semi-implicit step.

\end{abstract}
\end{document}


\end{document}
