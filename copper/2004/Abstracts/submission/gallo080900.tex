\documentclass{report}
\usepackage{amsmath,amssymb}
\setlength{\parindent}{0mm}
\setlength{\parskip}{1em}
\begin{document}
\begin{center}
\rule{6in}{1pt} \
{\large Efstratios Gallopoulos \\
{\bf Computing pseudospectra of parameter-dependent matrices}}

Department of Computer Engineering & Informatics \\ University of Patras \\ 26500 Patras \\ Greece
\\
{\tt stratis@ceid.upatras.gr}\\
Costas Bekas\\
Kyriakos Petrakos\end{center}

The motivation for studying the matrix pseudospectrum is that it
frequently provides information regarding matrix modeled processes that
is not readily available from the spectrum. Only a few years ago,
computing pseudospectra was considered to be very expensive even for
moderately sized matrices. Recent research, however, has brought down the
cost considerably and makes their computation feasible even for
relatively large matrices. A general assumption in these efforts was that
the matrix under study does not change. In this presentation we relax
this constraint and consider the pseudospectrum of matrices that depend
on parameters, such as time. Typical applications in which such matrices
arise include the numerical solution of PDEs with time varying
coefficients and bifurcation problems in dynamical systems using
Newton-Krylov schemes. The objective is to design methods that follow the
evolution, with $t$, of the pseudospectrum $\Lambda_\epsilon(A(t))$ of
some matrix function $A(t)$. A recently designed method for the parallel
computation of pseudospectra, called {\sc psdm}, will serve as a primary
tool of these investigations.


(Work conducted in the context of and supported in part by a University
of Patras CARATHEODORY grant.)


\end{document}
