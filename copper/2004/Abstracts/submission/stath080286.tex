\documentclass{report}
\usepackage{amsmath,amssymb}
\setlength{\parindent}{0mm}
\setlength{\parskip}{1em}
\begin{document}
\begin{center}
\rule{6in}{1pt} \
{\large Andreas Stathopoulos \\
{\bf Iterative validation: a scheme for improving the reliability and performance of eigenvalue block iterative solvers}}

Department of Computer Science  \\           College of William and Mary \\ Williamsburg \\ VA 23187-8795
\\
{\tt andreas@cs.wm.edu}\\
James, R. McCombs\end{center}

\begin{document}
\maketitle

Computationally intensive applications depend increasingly on
Krylov and Krylov-like iterative methods to solve large, and often sparse,
eigenvalue problems.
Some of the most widely used methods include subspace iteration,
the Lanczos and the Arnoldi methods.
If the matrix can be factored the benefits are twofold;
First, using the above methods in shift and invert mode
(e.g., the shift-invert Arnoldi) very fast convergence can be achieved
toward eigenvalues close to a specified shift.
Second, calculating the matrix inertia through an additional factorization
allows an exact prediction of the number of eigenvalues within an interval
and thus assurance that no required eigenpairs have been missed.

In this work, we are interested in cases where the matrix is too dense
or too large to factor, and thus the above validation of results is
not possible.
In such cases, it is also common that the required eigenvalues are
tightly clustered or even multiple, causing Krylov methods to converge
slowly and often miss eigenpairs.
Several methods such as Davidson, Jacobi-Davidson, LOBPCG and EIGIFP
exploit preconditioning to improve convergence and robustness of
eigenvalue iterative codes.
Still, however, no assurance is provided that no eigenvalues are missed.

Block Krylov methods work on an orthonormal set of vectors simultaneously
instead of just one vector.
Provided that the initial set of vectors are not deficient in the direction
of required eigenvectors, block methods identify all clustered or
multiple eigenvalues within the size of the block.
Yet, there is no assurance that eigenvalues are not missed beyond the
block size.
Moreover, knowledge about the problem is required to decide on the
appropriate block size.
Computationally, block methods usually require more floating point
operations than single vector methods but they are more cache efficient.
For certain block sizes they improve overall execution time, but such
a choice is complicated as it is affected by the numerics of the problem.

Alternatively, a large number of (even multiple) eigenvalues can be
obtained through a stable form of deflation called locking.
Assume for example a symmetric matrix where the lowest {\tt nev}
eigenvalues are required.
An iterative method is run until the lowest eigenvalue $\lambda_1$ converges.
Then, the corresponding eigenvector $x_1$ is placed in a special locked array,
and all computations in the iterative subspace are performed orthogonally
to $x_1$.
This guarantees that the method will converge to a different eigenpair,
but not necessarily the next lowest one.
In practice, locking does not miss eigenvalues and it even identifies
multiplicities if a tolerance close to machine precision is used.
Even though in theory single vector Krylov methods cannot obtain
more than one eigenvector per multiplicity, floating point arithmetic
introduces noise in the direction of the invariant subspace
that is gradually amplified.
In the presence of a very high multiplicity or a high number of multiplicities
single vector methods with locking are usually not as effective as
block methods.
Similarly, with high tolerances, block methods still provide a very
robust choice, while locking may miss or yield inaccurate eigenpairs.
Typically, single vector methods with locking are preferred because of
their efficiency, reverting to small block sizes when the problem is
known to have multiplicities.

To improve confidence on results obtained from some calculation,
eigenvalue practitioners traditionally restart the iterative method
with a new random initial guess, locking against all previously computed
eigenvectors.
If the new eigenvalue is inside the required range, the computation has
to be continued to obtain the missed eigenvalues.
However, it is neither efficient nor robust to rely on a single
vector method to obtain highly clustered or multiple eigenvalues, especially
with lower accuracies.

We propose a new technique, which we call {\it iterative validation},
that acts as a wrapper calling another eigenvalue block iterative method
repeatedly until no missed eigenvalues can be identified.
The inner method can be any block iterative method, such as
block Lanczos or subspace iteration, that implements locking against
an externally provided set of vectors.
The iterative validation can be described easier assuming the lower {\tt nev}
eigenpairs of a symmetric matrix are needed.

\begin{enumerate}
\setlength{\itemsep}{-2pt}
\item
The first time through,
the user calls the inner block method as (s)he would normally do,
specifying the smallest necessary block size (usually one) to obtain
efficiently most of the required eigenpairs.
After completion of the inner method, if requested,
our validation technique starts from step 2.

\item
Using error bounds on the {\tt nev} Ritz values
$\lambda_1, \ldots , \lambda_{nev}$, we identify the largest
numerical multiplicity obtained thus far, say $m$.

\item
We set the block size equal to $m+1$, and call back the inner
block method with $m+1$ random initial guesses,
looking for the smallest eigenvalue of the matrix
that is deflated from all previously computed eigenvalues.
For numerical stability, locking is used for deflation.

\item
When the inner method converges, we report not only the first
but all $m+1$ lowest eigenpairs in the block
$(\mu_1, z_1), \ldots , (\mu_{m+1}, z_{m+1})$.
We compute the error bound of $\mu_1$: $\epsilon_1$.

\item
If $\mu_1 - \epsilon_1 > \lambda_{nev}$, report the original $\lambda_i$
as the required eigenvalues and stop.

\item
If $\mu_1 < \lambda_{nev}$, an eigenvalue was missed.
Insert $\mu_1$ in its proper place among the eigenvalues $\lambda_i$,
and dispense with the eigenpair of $\lambda_{nev}$
\footnote{If space is available, it is better to keep the
eigenpair $\lambda_{nev}$ as it will help convergence in the
next validation step.}.
Consider $(z_2, \ldots , z_{m+1})$ as initial guesses. The rest
will be filled with random vectors.
Go back to step 2.

\item
if $\lambda_j < \mu_1 - \epsilon_1 < \lambda_{j+1}$, return to the
user the above eigenvalues noting that the provided tolerance
was not sufficient to resolve an eigenvalue between $\lambda_{j+1}$
and $\mu_1$.

\end{enumerate}

The above technique has several advantages.
First and foremost, it provides a relatively unobtrusive way to
validate results, thus dramatically improving robustness.
If the original code would miss eigenvalues,
the additional user specified expense is more than justified.
Typically, the validation will be run once for each new class of
problems, and switched off afterwards.
Second, it provides a dynamic way to fine tune the block size without
wasting all previous effort.
Third, assuming a multiplicity or a cluster of $m$ eigenvalues,
a block size of $m$ would be slower for many other required
required eigenvalues that do not belong to the cluster or
the multiplicity.
Iterative validation with an original block size of 1 finds first
the easier part of the spectrum and it only uses the block size
$m$ at the clusters that need it.

We have implemented iterative validation as a MATLAB wrapper for
{\bf irbleigs} and {\bf LOBPCG} eigenvalue codes, and as a C
wrapper for our {\bf block Jacobi-Davidson} code.
In one extreme example demonstrating robustness, irbleigs with block
size of 33 was not able to obtain all 74 multiplicities of matrix BCSSTK16.
Using irbleigs with block size of 5, followed by iterative validation with
a block size of no more than 33 found all the multiplicities.

\end{document}


\end{document}
