\documentclass{report}
\usepackage{amsmath,amssymb}
\setlength{\parindent}{0mm}
\setlength{\parskip}{1em}
\begin{document}
\begin{center}
\rule{6in}{1pt} \
{\large Chad Westphal \\
{\bf First-Order System Least-Squares (FOSLS) for Problems with Boundary Singularities}}

Campus Box 526 \\ Boulder \\ CO 80309-0526
\\
{\tt chad.westphal@colorado.edu}\\
Thomas Manteuffel\\
Eunjung Lee\end{center}

Many elliptic boundary value problems have the fortunate property of a
guaranteed smooth solution as long as the data and domain are smooth.
However, many problems of interest are posed in nonsmooth domains and, as
a consequence, lose this property at the boundary. In this talk we
consider problems that have nonsmooth solutions at ``irregular boundary
points", that is, points that are corners of polygonal domains, locations
of changing boundary condition type, or both.

Least-squares discretizations in particular suffer from a global loss of
accuracy due to the reduced smoothness of the solution. We investigate a
weighted-norm least squares method that recovers optimal order accuracy
in the weighted functional norm and weighted $H^1$ norm, and retains
$L^2$ convergence even near the singularity. The method requires only
{\em a priori} knowledge of the power of the singularity, not the actual
singular solution. The theory of this general technique is studied in
terms of a simplified div-curl system and shown to be similarly effective
when applied to other problems.


\end{document}
