\documentclass{report}
\usepackage{amsmath,amssymb}
\setlength{\parindent}{0mm}
\setlength{\parskip}{1em}
\begin{document}
\begin{center}
\rule{6in}{1pt} \
{\large Eric Phipps \\
{\bf Bifurcation Methods for Large-Scale Problems}}

Sandia National Laboratories \\ Computational Sciences Department \\ PO Box 5800 MS-1111 \\ Albuquerque \\ NM  87185
\\
{\tt etphipp@sandia.gov}\\
Andy Salinger\\
David Day\end{center}

Computing families of solutions to nonlinear problems and their
bifurcations can provide invaluable insight to understanding the dynamics
of physical and biological systems. However, developing continuation and
bifurcation tracking algorithms that are scalable to a large-scale
context is quite challenging. In particular, limited memory, lack of
derivative information, inexact linear
solves from iterative linear solvers, and an inability to add rows and
columns to the system Jacobian matrix severly restrict the algorithmic
choices available.

In this talk, we will discuss how these difficulties are overcome in
LOCA: The Library of Continuation Algorithms, a software package for
large-scale continuation and bifurcation tracking developed at Sandia
National Laboratories, focusing on the turning point or saddle-node
bifurcation commonly observed in nonlinear systems. This algorithm has
the advantage of requring little additional information from the
application code than it
already most likely provides, and only requires linear solves of the
underlying system Jacobian matrix. However, these linear solves become
extremely ill-conditioned near the turning point bifurcation, limiting
the accuracy and robustness of the turning point tracking algorithm. We
will then describe a promising new algorithm recently developed at Sandia
based upon the idea of
applying the Moore-Penrose pseudo-inverse near the bifurcation point.
This algorithm avoids ill-conditioned matrix solves and has shown to be
significantly more robust and accurate than the original turning point
tracking algorithm. Its implementation requires a linear solver with a
"matrix free" mode that only requires matrix-vector and
preconditioner-vector products. Results from applying these techniques to
physical systems of practial interest will be presented.



\end{document}
