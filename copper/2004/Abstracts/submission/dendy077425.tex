\documentclass{report}
\usepackage{amsmath,amssymb}
\setlength{\parindent}{0mm}
\setlength{\parskip}{1em}
\begin{document}
\begin{center}
\rule{6in}{1pt} \
{\large Joel E. Dendy \\
{\bf Alternative RAP}}

MS-B284 \\ Los Alamos National Laboratory \\ Los Alamos \\ NM 87545
\\
{\tt jed@lanl.gov}\\
J. D. Moulton\end{center}

General geometric multigrid codes, like BOXMG, and algebraic
multigrid codes, like LAMG, employ Galerkin coarsening to form the
coarse grid operators. That is, if the fine grid operator is $A$,
then the coarse gird operator is $RAP$, where $P$ is the
prolongation, or interpolation, operator, from the coarse grid to
the fine grid, and where $R$ is the restriction operator from the
fine grid to the coarse grid. BOXMG, as well as other such
structured codes, employs a closed form expression for $RAP$. An
alternative is to form $RAP$ by placing $1$ at only one coarse grid
point, and $0$ at all other coarse gird points, then applying $P$,
$A$, and $R$ recursively; the result is $RAP$ at that coarse grid
point. This alternative is useful for some theoretical
investigations but is obviously too expensive for practical
calculations. This paper explores the idea of doing this calculation
locally using temporary $7\times7$ [in two dimensions] or
$7\times7\times7$ [in three dimensions] matrices. Initial
investigations have shown this method to be just as fast as the
original method for compiler-optimized serial computation. The real
advantage of the new method, however, is its simplicity in
comparison with the old method. We expect this simplicity to result
in more flexible and efficient computation on parallel architectures. We
will report on the realization of this expectation.

LAMG, and other algebraic multigrid codes, employ sparse matrix
multiplication techniques, to form the triple product $RAP$. We are
currently investigating whether a generalization of the above
procedure can be applied in this case as well.


\end{document}
