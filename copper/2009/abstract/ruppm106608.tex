\documentclass{report}
\usepackage{amsmath,amssymb}
\setlength{\parindent}{0mm}
\setlength{\parskip}{1em}
\begin{document}
\begin{center}
\rule{6in}{1pt} \
{\large Martin Rupp \\
{\bf Eigenmodes of a Guitar Top Plate -- Application for Filtering Algebraic Multigrid and Preconditioned Inverse Iteration}}

Simulation and Modelling \\ Goethe-Center for Scientific Computing \\ Goethe-University \\ Kettenhofweg 139 \\ 60325 Frankfurt am Main \\ Germany
\\
{\tt martin.rupp@gcsc.uni-frankfurt.de}\\
Arne N\"agel\\
{\em Goethe-University, Frankfurt/Main, Germany}\\
Gabriel Wittum\\
{\em Goethe-University, Frankfurt/Main, Germany}\end{center}

The top plate of a guitar is crucial for sound amplification. The
frequencies in the spectrum of the vibrating string which are close to
resonance frequencies of the top plate are amplified the most, and
determine (among other factors) the sound of the guitar. We employ a
finite element discretisation of the Lam� equation for the numerical
simulation of this problem.\\

To compute the resonance frequencies, we use a projection method for
generalized symmetric eigenvalue problems $Au = \lambda Mu$. The key step
is a minimization of the Rayleigh-Quotient $\lambda(u) := \frac{\langle
Au, u \rangle}{\langle Mu, u \rangle}$ over a suitable constructed
subspace $\mathcal L$ via the Rayleigh-Ritz-method. To construct
$\mathcal L$, we use the \emph{Preconditioned Inverse Iteration}
(PINVIT):
The approximate solution of the Inverse Iteration equation leads to the correction
\begin{equation} \label{eq:pinvit}
v^{(k+1)} = v^{(k)} - c^{(k)} = v^{(k)} - B (A - \lambda(v^{(k)}) M) v^{(k)}
\end{equation}
of the eigenvalue approximation $v^{(k)}$, which motivates the use of
$$ \mathcal L ^{(k+1)}= \mathrm{span} \left\langle{c^{(k)}, v^{(k)}}\right\rangle \;.$$
Additional vectors in $\mathcal L$ may increase convergence like in
\emph{LOPCG} \cite{Knyazev:TowardOptPrecondEigensolver} or, more general,
in PINVIT(s) \cite{Neymeyr:OnPrecondEigensolvers}. \\

The preconditioner $B$ used in (\ref{eq:pinvit}) should approximate the
inverse of the stiffness matrix $A$, and must thus be robust with respect
to anisotropies in the geometry and the material properties. We seek to
treat the arising linear system using algebraic multigrid. In particular,
we focus on the {\it Filtering Algebraic Multigrid} approach
\cite{wagner, naegel}. The key idea of this AMG variant is to construct
the interpolation operator $P$, such that the norm of the two-grid
operator is minimized in a certain sense. At the same time, constraints
are imposed to guarantee filter conditions for certain test vectors $t$:
\begin{eqnarray*}
\min_{P} \|(I-PR^{(inj)}) S \|, \quad \text{s.t.} \; (I-PR^{(inj)}) S t = 0
\end{eqnarray*}
We comment on the theory and outline a pointwise version of the method,
which is suitable to treat systems of equations. For linear elasticity
problems, local representations of the rigid body modes are used to
obtain robustness on the Neumann boundaries. Numerical results and
illustrative examples are provided and conclude the talk.

\begin{thebibliography}{99}
\small

\bibitem[1]{Knyazev:TowardOptPrecondEigensolver} Knyazev, A.V. :
\textit{Toward the Optimal Preconditioned Eigensolver: Locally Optimal
Block Preconditioned Conjugate Gradient Method}. SIAM J. Sci. Comput.,
Vol. 23, No. 2, pp. 517--541, 2001.

\bibitem[2]{Neymeyr:OnPrecondEigensolvers} K. Neymeyr: \textit{On
preconditioned eigensolvers and Invert-Lanczos processes}. Preprint 2007.

\bibitem[3]{wagner} Wagner C.: \textit{On the algebraic construction of
multilevel transfer operators}. Computing, Vol. 65, No. 1, pp. 73--95,
2000.

\bibitem[4]{naegel} N\"agel A., Falgout R.D., Wittum G.:
\textit{Filtering algebraic multigrid and adaptive strategies}. Computing
and Visualization in Science, Vol. 11, No. 3, pp. 159--167, 2008.
\end{thebibliography}


\end{document}
