%-orig: \documentclass[12pt]{amsart}
%-orig: \begin{document}
%
\renewcommand{\div}{\mathbf{div}}
\newcommand{\curl}{\mathbf{curl}}
\def\Nedelec{N\'ed\'elec}

%\title{Coarse Spaces by Constrained Energy Minimization}
%
%\author{Panayot S.  Vassilevski}
%\address{Center for Applied Scientific Computing,
%           Lawrence Livermore National Laboratory,
%             P.O. Box 808, L-560,
%             Livermore, CA 94551, U.S.A.}
%\email{panayot@llnl.gov}
%%\thanks{This work performed under the auspices of the U.S. Department of 
%Energy by Lawrence Livermore National Laboratory under Contract DE-AC52-07NA27344.} 


\documentclass{report}
\usepackage{amsmath,amssymb}
\setlength{\parindent}{0mm}
\setlength{\parskip}{1em}
\begin{document}
\begin{center}
\rule{6in}{1pt} \
{\large Panayot S.  Vassilevski \\
{\bf Coarse Spaces by Constrained Energy Minimization}}

Center for Applied Scientific Computing \\
Lawrence Livermore National Laboratory \\
P.O. Box 808, L-560 \\
Livermore, CA 94551, U.S.A.
\\
{\tt panayot@llnl.gov} %\\
\end{center}

%-mb: \begin{abstract}
We consider an unified approach of constructing operator-dependent discretization spaces on relatively coarse 
computationally feasible meshes. The approach utilizes natural energy functionals associated with the PDEs of interest. 
We construct local basis functions by minimizing the underlined functional subject to a set of constraints. 
The constraints are chosen so that the resulting spaces possess increasingly high order of approximation.  
We investigate the proposed approach from an upscaling discretization point of view which we 
illustrate with some preliminary numerical examples. 
%-mb: \end{abstract}
%-mb: \maketitle


\end{document}

