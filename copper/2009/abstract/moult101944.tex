\documentclass{report}
\usepackage{amsmath,amssymb}
\setlength{\parindent}{0mm}
\setlength{\parskip}{1em}
\begin{document}
\begin{center}
\rule{6in}{1pt} \
{\large David Moulton \\
{\bf Challenges for Multilevel Sampling in Statistical Inference}}

Applied Mathematics and Plasma Physics \\ MS B284 \\ Los Alamos National Laboratory \\ Los Alamos \\ NM 87545
\\
{\tt moulton@lanl.gov}\\
Dave Higdon\\
{\em Statistical Sciences, Los Alamos National Laboratory}\\
Jasper Vrugt\\
{\em Center for Nonlinear Studies, Los Alamos National Laboratory}\end{center}

Statistical inference through the Bayesian framework provides a
powerful probabilistic approach to characterizing the solution of
inverse problems. A canonical example is the recovery of the
electrical conductivity of an object from measurements of potential
and flux at the boundary, often called electrical impedance tomography
(EIT). The mathematical model that maps the state (e.g.,
conductivity) to the measured data (e.g., potential and flux) is
refered to as the forward model. In this statistical framework,
solving the inverse problem corresponds to quantifying statistics
about the posterior distribution of the state (e.g., conductivities)
conditioned on measurements. Although this formulation is very
flexible, the calculation of integrals with respect to the posterior
distribution is typically done through Markov Chain Monte Carlo (MCMC)
sampling, and hence it is a significant computational burden for
complex multiscale forward models. To alleviate this burden
applications often restrict their analysis to oversimplified forward
models, in turn limiting the utility of their analysis.

To avoid these undesirable simplifications and additional uncertainty,
considerable research has focused on schemes that improve the
efficiency of the MCMC work horse, single-site Metropolis. In fact, a
variety of approaches have been proposed to improve the efficiency of
this rudimentary MCMC sampling scheme for high-dimensional posteriors.
Multivariate updating approaches have been introduced that adjust
multiple parameters simultaneously to construct a new proposal, while
two-level delayed acceptance schemes rigorously allow the use of an
approximate forward map (or approximate solve) in an initial screening
step. In addition, high-level or templated-based priors (i.e.,
distributions of blobs or layers of conductivity) have been used to
effectively reduce the dimensionality of the problem. Each of these
techniques have demonstrated significant gains in efficiency for some
problems, but achieving a significant improvement in scalability over
a broad class of inverse problems seems to remain tied to reducing the
dimensionality.

This desire to reduce the dimensionality creates an allure to leverage
concepts from multilevel iterative solvers, which achieve their
optimal scaling through the recursive use of coarser grids, with MCMC
sampling. In this talk we highlight key issues in pursing this
connection, such as the interpretation of the hierarchy of
coarse-scale models, as well as the scale-dependent interpretation of
the model parameters and their distributions. Specifically, we
explore the use of multilevel solvers within a delayed acceptance
scheme, and discuss the potential speedup of this approach. Then to
combine the strengths of this approach with concepts from high-level
proposals we explore a framework that directly samples the discrete
hierarchy of models.


\end{document}
