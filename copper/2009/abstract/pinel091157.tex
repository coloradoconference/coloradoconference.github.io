\documentclass{report}
\usepackage{amsmath,amssymb}
\setlength{\parindent}{0mm}
\setlength{\parskip}{1em}
\begin{document}
\begin{center}
\rule{6in}{1pt} \
{\large Xavier PINEL \\
{\bf Theoretical and experimental analysis of a perturbed two-grid preconditioner for indefinite Helmholtz problems}}

CERFACS \\ 42 Avenue Coriolis \\ 31057 TOULOUSE cedex 1 \\ FRANCE
\\
{\tt xavier.pinel@cerfacs.fr}\\
Henri CALANDRA\\
{\em TOTAL, Centre Scientifique et Technique Jean Feger, Avenue Larribau, F-64018 Pau cedex, France, Henri.CALANDRA@total.com}\\
Serge GRATTON\\
{\em CNES, avenue Edouard Belin, F-31401 Toulouse cedex 4, France and 
CERFACS, 42 avenue Gaspard Coriolis, F-31057 Toulouse cedex 1, France, serge.gratton@cnes.fr and Serge.Gratton@cerfacs.fr}\\
Xavier VASSEUR\\
{\em CERFACS, 42 avenue Gaspard Coriolis, F-31057 Toulouse cedex 1, France, Xavier.Vasseur@cerfacs.fr}\end{center}

We study the three-dimensional Helmholtz equation written in the
frequency domain modeled by the following partial differential equation:
$$
-\Delta u - k^2 u = g
$$
with some absorbing boundary conditions, where $u$ is the pressure of the
wave, $k$ its wavenumber and $g$ is a Dirac function that represents the
wave source. This problem is discretized using second-order accurate
finite difference techniques leading to huge linear systems for large
wavenumbers. \\

Following Elman \cite{Elman:2001}, we use a geometric two-grid
preconditioner for a Krylov subspace method (namely flexible GMRES
\cite{Saad:1993}) as a solution method. Due to the large dimension of the
coarse grid problem we focus on the behavior of a two grid algorithm
where the coarse problem is not solved exactly. We use a Krylov subspace
method to solve this problem only approximately. This leads us to analyze
two questions:
\begin{itemize}
\item Which stopping criterion and convergence threshold should be used
for the coarse grid solver ?
\item What is the impact of an approximate coarse grid solution on the
preconditioning properties ?
\end{itemize}

In this talk, we will bring some elements to answer both questions. First
a local Fourier analysis - assuming Dirichlet boundary conditions and
small wavenumbers - will provide convergence rate estimates for this
perturbed two-grid algorithm used as a solver. Secondly we will
investigate the numerical behaviour of the algorithm when this two-grid
method is used as a preconditioner. For that purpose we study the
spectrum of an equivalent matrix constructed inside the flexible variant
of GMRES. This will help us understanding the effects of the approximate
coarse grid solution on the preconditioner for both absorbing boundary
conditions and large wavenumbers. Parallel numerical experiments will
conclude this talk showing the efficiency of this perturbed
preconditioner even for large wavenumbers.\\

\begin{thebibliography}{2}

\bibitem{Elman:2001}
Elman, H. R., Ernst, O. G. and O'Leary, D. P., A multigrid method enhanced by {K}rylov
subspace iteration for discrete {H}elmholtz equations. {\itshape SIAM J. Sci. Comput.,}
{\bfseries 23}, 1291--1315, 2001.

\bibitem{Saad:1993}
Saad, Y., A flexible inner-outer preconditioned GMRES algorithm.
{\itshape SIAM J. Sci. Comput.,}
{\bfseries 14}, 461--469, 1993.

\end{thebibliography}


\end{document}
