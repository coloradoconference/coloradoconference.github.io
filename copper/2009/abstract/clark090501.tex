\documentclass{report}
\usepackage{amsmath,amssymb}
\setlength{\parindent}{0mm}
\setlength{\parskip}{1em}
\begin{document}
\begin{center}
\rule{6in}{1pt} \
{\large Michael A Clark \\
{\bf Adaptive multigrid algorithm for the lattice QCD Wilson-Dirac matrix }}

Center for Computational Sciences \\ Boston University \\ 3 Cummington St \\ Boston \\ MA 02215
\\
{\tt mikec@bu.edu}\\
James Brannick\\
{\em Penn State}\\
Richard Brower\\
{\em Boston University}\\
Tom Manteuffel \\
{\em University of Colorado at Boulder}\\
Steve McCormick\\
{\em University of Colorado at Boulder}\\
James Osborn\\
{\em Argonne National Laboratory}\\
Claudio Rebbi\\
{\em Boston University}\end{center}

The Wilson-Dirac matrix is the discretized Dirac operator that describes
fermions in quantum field theory. The parameter of interest in this
matrix is the fermion mass parameter; the matrix becomes singular as this
parameter is taken to zero. Solving such systems of linear equations at
near zero fermion mass is the prevalent computational cost in lattice QCD
calculations.

Although the matrix has a regular geometric structure, the random nature
of the coefficients of the matrix results in the failure of naive
multigrid approaches. A geometric adaptive multigrid algorithm based on
adaptive smooth aggregation (\(\alpha\)SA) has been developed
specifically for this matrix. Special consideration must be paid to the
smoother used since the matrix is not Hermitian positive definite. The
resulting algorithm exhibits very little dependence on the condition
number of the matrix.


\end{document}
