\documentclass{report}
\usepackage{amsmath,amssymb}
\setlength{\parindent}{0mm}
\setlength{\parskip}{1em}
\begin{document}
\begin{center}
\rule{6in}{1pt} \
{\large Harald Koestler \\
{\bf A Multigrid Prolongation Method Based on Local Minimization of the Constant in the Strengthened Cauchy-Schwarz Inequality}}


University of Erlangen-Nuremberg \\ Lehrstuhl f�r Informatik 10 \\ Cauerstrasse 6 \\ D-91056 Erlangen \\ Germany
\\
{\tt harald.koestler@informatik.uni-erlangen.de}\\
Christoph Pflaum\\
{\em University of Erlangen-Nuremberg}\end{center}

We discuss ways to construct prolongation operators for multigrid methods
both for geometric and pure algebraic settings. Our theoretical
derivation of the prolongation operator is based on finite element
analysis, in detail we try to minimize the constant in the strengthened
Cauchy-Schwarz
inequality locally. Therefore, we first write the fine grid space as a
direct sum of the coarse grid space and a suitable complementary space.
The local coarse subspaces are chosen such that they can locally
represent the kernel of the
operator.
In order to solve the arising local minimization problems and to
determine the prolongation weights only a few local operations are required.
We show first numerical results in 2D and outline applications of our
method like linear elasticity.


\end{document}
