\documentclass{report}
\usepackage{amsmath,amssymb}
\setlength{\parindent}{0mm}
\setlength{\parskip}{1em}
\begin{document}
\begin{center}
\rule{6in}{1pt} \
{\large Jeffrey J. Heys \\
{\bf Algebraic Multigrid Solvers for Weighted Least-Squares Finite Elements with Application to Particle Imaging Velocimetry Analysis}}

Chemical and Biological Engineering \\ Montana State University \\ PO Box 173920 \\ Bozeman \\ MT 59717-3920
\\
{\tt jeff.heys@gmail.com}\\
Michele Milano\\
{\em Arizona State University}\\
Marek Belohlavek\\
{\em Mayo Clinic Scottsdale}\end{center}

The combination of ultrasound and microbubbles injected into the blood
makes it possible to noninvasively obtain 2-dimensional velocity data
inside the left ventricle of the heart. A long term goal is to translate
this velocity data into information about energy loss, pressure
gradients, and overall health of the heart, but this goal requires a full
3-dimensional velocity field. The question that we seek to answer is
whether the 2-dimensional velocity data can be incorporated into a full
3-dimensional computational fluid dynamics simulation in an appropriate
and computationally practical way. For addressing this problem, we
examine the potential of least-squares finite element methods ({LSFEM})
because of their flexibility in the enforcement of various boundary
conditions and their natural compatibility with multigrid solvers. By
weighting the boundary conditions in a manner that properly reflects the
accuracy with which the boundary values are known, we develop the
weighted LSFEM. The potential of weighted LSFEM is explored for two
different test problems: the first uses randomly generated Gaussian noise
to create artificial `experimental' data in a controlled manner, and the
second uses experimental particle imaging velocimetry data. In both test
problems, weighted LSFEM produces accurate results even for cases where
there is significant noise in the experimental data and provides
excellent computational scalability when used with a parallel algebraic
multigrid solver.


\end{document}
