\documentclass{report}
\usepackage{amsmath,amssymb}
\setlength{\parindent}{0mm}
\setlength{\parskip}{1em}
\begin{document}
\begin{center}
\rule{6in}{1pt} \
{\large Yvan Notay \\
{\bf Algebraic analysis of MG methods: the nonsymmetric case}}

Universite Libre de Bruxelles \\ Service de Metrologie Nucleaire (CP 165/84) \\ 50 av F D Roosevelt \\ 1050 Bruxelles \\ BELGIUM
\\
{\tt ynotay@ulb.ac.be}\end{center}

We develop an algebraic analysis of two-grid methods that does
not require any symmetry property. Several equivalent expressions are provided
that characterize {\em all} eigenvalues of the iteration matrix. In the
symmetric positive definite case, these expressions reproduce the
sharp two-grid convergence estimate obtained by Falgout, Vassilevski and
Zikatanov [Numer. Lin. Alg. Appl., 12 (2005), pp.~471--494], and also
previous algebraic bounds
which can be seen as corollaries of this estimate.

These results allow to measure the convergence by checking ``approximation
properties''. In this talk, proper extentions of the latter
to the nonsymmetric case are presented. Sometimes approximation properties
for the symmetric positive definite case are summarized in
loose terms; e.g.: {\em Interpolation must be
able to approximate an eigenvector with error bound proportional to the size
of the eigenvalue} [SIAM J. Sci. Comput., 22 (2000),
pp.~1570--1592]. It is shown that this can be applied
to nonsymmetric problems too,
understanding ``size'' as ``modulus''.

Eventually, an analysis is developed, for the nonsymmetric case, of
the theoretical
foundations of ``compatible relaxation'', according to which a
Fine/Coarse partitioning may be checked
and possibly improved.


\end{document}
