\documentclass{report}
\usepackage{amsmath,amssymb}
\setlength{\parindent}{0mm}
\setlength{\parskip}{1em}
\begin{document}
\begin{center}
\rule{6in}{1pt} \
{\large Kirk E. Jordan \\
{\bf Title: Petascale Barrier Surpassed But How Will We Solve Real Problems on Petascale Systems and/or Exascale Systems? }}

IBM T J Watson Research Center \\ http://www-3.ibm.com/software/info/university/people/kjordan.html
\\
{\tt kjordan@us.ibm.com}\end{center}

High performance computing (hpc) is a tool frequently used to understand
complex problems in numerous areas such as aerospace, biology, climate
modeling and energy. Scientists and engineers working on problems in
these and other areas demand ever increasing compute power for their
problems. In order to satisfy the demand for increase performance to
achieve breakthrough science and engineering, we turn to parallelism
through large systems with multi-core chips. For these systems to be
useful massive parallelism at the chip level is not sufficient. I will
describe some of the challenges that will need to be considered in
designing Petascale and eventually Exascale systems. However, the
hardware development is not as hard as designing algorithms that will
exploit these systems. Through the combination of hpc hardware coupled
with novel algorithmic approaches, such as multigrid methods, some
efforts toward breakthroughs in science and engineering are described.
While progress is being made, there remain many challenges for the
computational science community to apply ultra-scale, multi-core systems
to ``Big'' science problems with impact on society. In conclusion, some
discussion not only on the most obvious way to use ultra-scale,
multi-core hpc systems will be given but also some thoughts on how one
might use such systems to tackle previously intractable problems.


\end{document}
