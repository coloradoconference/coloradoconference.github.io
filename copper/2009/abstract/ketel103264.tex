\documentclass{report}
\usepackage{amsmath,amssymb}
\setlength{\parindent}{0mm}
\setlength{\parskip}{1em}
\begin{document}
\begin{center}
\rule{6in}{1pt} \
{\large Christian Ketelsen \\
{\bf Multigrid Solvers for Quantum Electrodynamics}}

Department of Applied Mathematics \\ 526 UCB \\ University of Colorado \\ Boulder \\ CO 80309-0526
\\
{\tt ketelsen@colorado.edu}\\
Tom Manteuffel \\
{\em University of Colorado at Boulder}\\
Steve McCormick\\
{\em University of Colorado at Boulder}\end{center}

The Schwinger Model of quantum electrodynamics (QED) describes the
interaction between electrons and photons. Large scale numerical
simulations of the theory require repeated solution of the
two-dimensional Dirac equation, a system of two first order partial
differential equations coupled to a background U(1) gauge field.
Traditional discretizations of this system are sparse and highly
structured, but contain random complex entries introduced by the
background field. For even mildly disordered gauge fields the near kernel
components of the system are highly oscillatory, rendering standard
multilevel methods ineffective.

We consider an alternate formulation of the governing equations that is
more amenable to multigrid solvers. The alternate model is obtained by a
similarity transformation of the continuum operator which essentially
eliminates the gauge field. The resulting formulation resembles uncoupled
diffusion-like problems with variable diffusion coefficients. The form of
the transformed system is ideal as adaptive multilevel methods have
proven effective in the past at solving such problems. Next, we
discretize the transformed system using least-squares finite elements.
Finally, adaptive smoothed aggregation multigrid is used to solve the
resulting linear system. We present numerical results and discuss
implications of the transformed formulation for the physical theory.


\end{document}
