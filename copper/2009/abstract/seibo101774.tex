\documentclass{report}
\usepackage{amsmath,amssymb}
\setlength{\parindent}{0mm}
\setlength{\parskip}{1em}
\begin{document}
\begin{center}
\rule{6in}{1pt} \
{\large Benjamin Seibold \\
{\bf AMG for Meshfree Finite Difference Methods}}

Department of Mathematics \\ 2-346 \\ Massachusetts Institute of Technology \\ 77 Massachusetts Avenue \\ Cambridge MA 02139 \\ USA
\\
{\tt seibold@math.mit.edu}\end{center}

Lagrangian particle methods, such as Smoothed Particle
Hydrodynamics, approximate the equations of fluid flow
on a cloud of points that move with the flow. Thus,
convective terms are treated exactly, which is a great
advantage compared to Eulerian approaches. The
fundamental challenge in particle methods are
unstructured node distributions. A common strategy is to
approximate differential operators by meshfree finite
difference stencils. These can be derived for instance
by the moving least squares method.

Incompressible flows can be simulated by a pressure
correction in each time step. The arising Poisson
equations can be approximated on the cloud of particles
by meshfree finite differences. The generation and
solution of the arising linear systems is often times
very costly, thus multigrid methods are of great
interest. A natural approach is Algebraic Multigrid,
since geometric strategies to refine a fully
unstructured point cloud are difficult to formulate.

The finite difference matrices arising from classical
meshfree least squares approaches have three properties
that are challenging with respect to AMG:
\begin{itemize}
\item
Finite difference methods on unstructured data sets
approximate the Laplace operator by non-symmetric
matrices.
\item
The arising matrices are about six times as dense as
finite difference matrices on rectangular grids.
\item
An M-matrix structure is not guaranteed, and is---unless
the point geometry is sufficiently nice---in general
violated.
\end{itemize}
While here the non-symmetry is in some sense weaker than
coming from the approximation of a convection operator,
it can not be overcome.
Better sparsity and the M-matrix structure, however,
\emph{can} be achieved, by a new finite difference
approach, that is based on linear sign-constrained
optimization.

In this talk, I will outline meshfree finite difference
methods, present the new finite difference approach, show
results when applied in the context of AMG methods, and
pose open questions. Numerical tests were done using
SAMG (Fraunhofer SCAI, K.~St{\"u}ben et al.),
as well as AMLI (C.~Mense and R.~Nabben).


\end{document}
