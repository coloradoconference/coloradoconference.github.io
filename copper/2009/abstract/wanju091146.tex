\documentclass{report}
\usepackage{amsmath,amssymb}
\setlength{\parindent}{0mm}
\setlength{\parskip}{1em}
\begin{document}
\begin{center}
\rule{6in}{1pt} \
{\large Justin Wan \\
{\bf A Multigrid Method for Solving Partial Integro-Differential Equations Arising from Option Pricing Under the Levy Process}}

School of Computer Science \\ University of Waterloo \\ 200 University Ave West \\ Waterloo \\ ON N2L 3G1 \\ Canada
\\
{\tt jwlwan@uwaterloo.ca}\\
Iris R. Wang\\
{\em University of Waterloo}\end{center}

Recently, Levy process models for pricing options have become
popular in the financial literature. Based on the Black-Scholes
model, the option value of a contingent claim, $V(S,\tau)$,
under the Levy process satisfies the following partial
integro-differential equation (PIDE):
\begin{eqnarray*}
V_\tau =
\frac{\sigma^2}{2}S^2V_{SS} +
(r-q)SV_S - rV + \int_{-\infty}^{\infty} \nu(y)[V(Se^y,\tau) - V(S,\tau) -
S(e^y - 1) V_S]dy,
\end{eqnarray*}
where $S$ is the underlying stock price, $\tau$ the time to
expiry, $r$ the risk-free interest rate, $q$ the dividend yield,
and $\sigma$ the volatility associated with the continuous
component of Levy process. The Levy measure $\nu(y)$ is
typically singular at $y=0$.

Due to the singularity of $\nu(y)$, standard discretization
methods may not achieve the usual accuracy. Moreover, the
computation of the integral term can be very expensive.
Recently, an implicit discretization method has been proposed
that can obtain second order convergence for finite variation
case and better than first order accuracy for infinite variation
processes. In each time step, a linear system needs to be
solved, which forms the bottle neck of the entire computation.
A fixed point and a preconditioned BiCGSTAB iterations have been
proposed. While they converge quite rapidly for mildly singular
cases, the number of iterations grows significantly in the case
of infinite variation when the mesh size decreases.

In this talk, we present a multigrid method for solving the
PIDE. The multigrid method uses a fixed point iteration for
smoothing. We prove by Fourier analysis that the smoother damps
away the high frequencies. Linear interpolation and full
weighting are used for intergrid transfer. Since it is too
expensive to form the integral term explicitly, direct
discretization is used for constructing the coarse grid matrices.
However, it turns out that it is not as effective as the
Galerkin course grid operators. We discuss how a combination
of the two is used to form the coarse grid matrices. We
demonstrate numerically the effectiveness of the multigrid
method and show that the convergence rate is independent of the
mesh size.


\end{document}
