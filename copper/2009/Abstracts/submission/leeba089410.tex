\documentclass{report}
\usepackage{amsmath,amssymb}
\setlength{\parindent}{0mm}
\setlength{\parskip}{1em}
\begin{document}
\begin{center}
\rule{6in}{1pt} \
{\large Barry Lee \\
{\bf Improved Multiple-Coarsening/Semi-Coarsening Methods for Sn Discretizations of the Boltzmann Equation}}

Center for Applied Scientific Computing \\ L-561 \\ Lawrence Livermore National Lab \\ Livermore \\ CA 94551
\\
{\tt lee123@llnl.gov}\end{center}

In a recent series of articles, the author presented a
multiple-coarsening multigrid method for
solving $S_n$ discretizations of the Boltzmann transport equation. This
algorithm is applied to an
integral equation for the scalar flux or moments. Although this algorithm is very
efficient over parameter regimes that describe realistic neutron/photon
transport applications, improved methods that
can reduce the computational cost are presented in this talk. These
improved methods are derived
through a careful examination of the frequencies, particularly the
near-nullspace, of the integral equation.
In the earlier articles, the near-nullspace components were shown to be
smooth in angle in the
sense that the angular fluxes generated by these components are smooth in angle.
In this talk, we present a spatial description of these near-nullspace components.
Using the angular description of the earlier papers together
with the spatial description reveals the intrinsic space-angle dependence
of the integral equation's frequencies.
This space-angle dependence then is used to determine the appropriate
space-angle grids to represent and
efficiently attenuate the near-nullspace error components on.
It will be shown that these components can have multiple spatial scales.
By using only the appropriate space-angle grids
that can represent these spatial scales in the original multiple-coarsening
algorithm, an improved algorithm is obtained. Moreover, particularly for
anisotropic scattering, recognizing the strong angle
dependence of the angular fluxes generated by the high frequencies of the
integral equation, an improved multiple-coarsening
scheme is derived. Restricting this scheme to the
appropriate space-angle grids produces a very efficient method.


\end{document}
