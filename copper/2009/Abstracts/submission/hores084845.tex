\documentclass{report}
\usepackage{amsmath,amssymb}
\setlength{\parindent}{0mm}
\setlength{\parskip}{1em}
\begin{document}
\begin{center}
\rule{6in}{1pt} \
{\large Raya Horesh \\
{\bf A Multi-grid framework for volume preserving constrained image registration on OcTree structure}}

Department of Mathematics and Computer Science \\ Emory University \\ 400 Dowman Dr \\ Atlanta \\ GA 30322 \\ USA
\\
{\tt rshindm@emory.edu}\\
Eldad Haber\\
Jan Modersitzki\end{center}

Image registration is one of the fundamental missions of image
processing. It can be simply considered as a process of aligning/matching
two or more images having similar contents in some sense. For example,
the images could have been captured at different times, from different
viewpoints and/or using different types of sensors. Since the problem of
image registration is ill-posed, one may wish to add additional
information. In this study, the deformation is controlled in terms of the
determinant of the Jacobian of the transformation. This approach
guarantees regularity of the grid and prevent folding effects.

In order to keep the computational time reasonable, it is desirable to
reduce the amount of data which need to be processed. This can naively be
obtained by down-sampling. However, this approach may results in the loss
of possibly important image features. OcTree provides a straightforward
approach for such a context driven data sparsification. The OcTree data
presentation uses few voxels for representing regions with small
variability and many voxels for regions of high variability.

Discritization of the volume preserving constrained registration problem
yields a KKT system. This large-scale system is indefinite and
ill-conditioned; therefore, we propose a multi-grid framework for
effective preconditioning of this system. We present
mesh-size-independent results that demonstrate the optimality of the
proposed method.


\end{document}
