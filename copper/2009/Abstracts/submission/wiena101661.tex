\documentclass{report}
\usepackage{amsmath,amssymb}
\setlength{\parindent}{0mm}
\setlength{\parskip}{1em}
\begin{document}
\begin{center}
\rule{6in}{1pt} \
{\large Roman Wienands \\
{\bf On the Construction of Prolongation Operators for Multigrid}}

Mathematical Institute \\ University of Cologne \\ Weyertal 86-90 \\ 50931 Cologne \\ Germany
\\
{\tt wienands@math.uni-koeln.de}\\
Harald Koestler\\
{\em Department of Computer Science 10, University of Erlangen-Nuremberg, Germany}\\
Irad Yavneh\\
{\em Department of Computer Science, Technion-Israel Institute of Technology}\end{center}

\def\bfP{{\bf P}}
\def\bfS{{\bf S}}
\def\bfv{{\bf v}}
\def\bfb{{\bf b}}
The quality of multigrid methods crucially depends on an efficient
interplay between the smoothing operator $\bfS$ and prolongation $\bfP$.
More precisely, the range of $\bfS$ should be sufficiently well
represented in the range of $\bfP$. In the limit case of an optimal (but
usually impractical) choice of $\bfP$ and $\bfS$ this means that
\begin{equation}
\bfS \bfv = \bfP \bfv_{C} \qquad \text{with}
\qquad \bfv = \left( \begin{matrix} \bfv_{F} \\ \bfv_{C}
\end{matrix} \right) \label{eq:basiseq}
\end{equation}
must hold for an arbitrary vector $\bfv$ where the variables and
equations have been permuted such that F-variables come first.
(Those variables which are to be contained in the coarse grid are
commonly referred to as C-variables whereas the complementary set is
related to the F-variables.)

In this talk we present a formalism for the construction of proper
prolongation operators in order to accomplish the transfer from coarse to
fine grids within a multigrid algorithm. The main idea is to force
relation
(\ref{eq:basiseq}) only locally (i.e. at each coarse variable $i$) for a
certain subset of (algebraically) smooth basis vectors $\bfb^{k(i)}$
$(k(i) = 1,\dots,n(i))$. Following this
idea, classical matrix-dependent prolongations or prolongation
operators based on smoothed aggregation can be recovered. Moreover, this
rather general framework gives rise to many other variants whose
complexity and accuracy can be controlled by a careful selection of basis
vectors and smoothing operators, which are applied to construct $\bfP$.
For example, geometric or other problem-dependent information can be
easily exploited to tailor the prolongation to the particular application
at hand.

At the end of the talk we will present numerical results for several two-
and three-dimensional diffusion problems including jumping coefficients.
In particular a real-world three-dimensional image segmentation problem
from a medical application is discussed.


\end{document}
