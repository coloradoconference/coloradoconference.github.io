\documentclass{report}
\usepackage{amsmath,amssymb}
\setlength{\parindent}{0mm}
\setlength{\parskip}{1em}
\begin{document}
\begin{center}
\rule{6in}{1pt} \
{\large Travis Austin \\
{\bf Automatic Construction of Sparse Preconditioners for High-Order Finite Element Problems}}

Tech-X Corporation \\ 5621 Arapahoe Ave \\ Suite A \\ Boulder \\ CO 80303
\\
{\tt austin@txcorp.com}\\
Marian Brezina\\
{\em University of Colorado at Boulder}\\
Thomas Manteuffel\\
{\em University of Colorado at Boulder}\\
John Ruge\\
{\em University of Colorado at Boulder}\end{center}

In recent years, the advantage to using high-order finite element methods
and spectral element methods for the discretization of partial
differential equations has become fully realized. The accuracy that can
be achieved with high-order methods relative to the work required is more
attractive than with first-order finite element methods. Furthermore, in
climate modeling, researchers want a high-degree of precision at each
time step in order to minimize the accumulation of errors due to long
time integrations, and in MHD modeling, strong magnetic field
anisotropies are not accurately resolved without high-order finite
element methods. As a result of this increased interest in high-order
finite element methods, it is worthwhile to reconsider the development of
optimized multigrid solution methods for systems derived from high-order
finite elements and spectral elements.

High-order finite elements yield much denser systems of equations
requiring greater memory consumption for both the matrix and for the
corresponding solver infrastructure, like AMG. For large scale MHD
calculations, it has been observed that memory is a limit due to the
memory consumption of the high-order matrices. Thus, there is a need for
a sparser approximation of the high-order finite element systems that
still yields reasonable convergence. A well-known method for generated a
sparser preconditioner is to use low-order finite elements to generate a
sparse approximation. In this talk, we show that these sparse
preconditioners require less memory and can produce better convergence
behavior for 3D problems when inverted with an AMG method. We also
introduce our concept for automatically constructing these
preconditioners and address the computational cost and memory consumption
of such an approach.


\end{document}
