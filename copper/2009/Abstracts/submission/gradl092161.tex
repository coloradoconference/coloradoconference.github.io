\documentclass{report}
\usepackage{amsmath,amssymb}
\setlength{\parindent}{0mm}
\setlength{\parskip}{1em}
\begin{document}
\begin{center}
\rule{6in}{1pt} \
{\large Tobias Gradl \\
{\bf High Performance Adaptive Mesh Refinement}}

Lehrstuhl fuer Informatik 10 \\ Cauerstr 6 \\ 91058 Erlangen \\ Germany
\\
{\tt tobias.gradl@cs.fau.de}\\
Ulrich R�de\\
{\em Department of Computer Science 10, University of Erlangen-Nuremberg}\end{center}

Getting multigrid software to solve for as many unknowns per second as
possible on current high performance computers means designing its data
access patterns to be as regular as possible. \emph{Regularly structured
data} allows for optimally exploiting the processor capabilities (e.\,g.\
vector processing) and the interconnection network (e.\,g.\ by grouping
communication operations). Just as important as solving for many unknowns
in short time is reducing the overall number of unknowns. \emph{Adaptive
mesh refinement (AMR)} can significantly reduce the problem size in many
cases. However, this technique is known to impede the use of regular data
structures, and, thus, to drastically restrict the possible performance
of the solver. In our talk we show how AMR can be implemented without
destroying the regularity of the data structures, maintaining a fast
execution. The framework for the AMR implementation is provided by
\emph{Hierarchical Hybrid Grids (HHG)}, a multigrid finite element solver
designed for the use on massively parallel computers with several
thousand processors. The software uses semi-structured meshes in order to
achieve high performance on such computers. In the talk both red-green
refinement and refinement with hanging nodes are covered, because both
methods have individual advantages, and the decision for one of the
methods---or for a combination of both---has to be made depending on the
problem characteristics. We develop the mathematical foundation for using
AMR with the \emph{correction scheme}, and we show its feasibility in
practical examples.


\end{document}
