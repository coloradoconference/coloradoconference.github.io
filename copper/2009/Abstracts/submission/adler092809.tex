\documentclass{report}
\usepackage{amsmath,amssymb}
\setlength{\parindent}{0mm}
\setlength{\parskip}{1em}
\begin{document}
\begin{center}
\rule{6in}{1pt} \
{\large James Adler \\
{\bf Nested Iteration First-Order Least Squares on Incompressible Resistive Magnetohydrodynamics}}

2075 Goss St #11 \\ Boulder \\ CO 80302
\\
{\tt james.adler@colorado.edu}\end{center}

Magnetohydrodynamics (MHD) is a single-fluid theory that describes Plasma
Physics. MHD treats the plasma as one fluid
of charged particles. Hence, the equations that describe the plasma form
a nonlinear system that couples Navier-Stokes
with Maxwell's equations. To solve this system, a
nested-iteration-Newton-FOSLS-AMG approach is taken. The goal is to
determine the most efficient algorithm in this context. One would like to
do as much work as
possible on the coarse grid including most of the linearization. Ideally,
it would be good to show that at most one
Newton step and a few V-cycles are all that is needed on the finest grid.
This talk will develop theory that supports
this argument, as well as show experiments to confirm that the algorithm
can be efficient for MHD problems. Currently, a reduced 2D time-dependent
formulation is studied. These equations can simulate
a "large aspect-ratio" tokamak, with non-circular cross-sections. Here,
the problem was reformulated in a way that is
suitable for FOSLS and FOSPACK. This talk will discuss two test problems
from this formulation: the Tearing Mode instability and the Island
Coalescence Instability.


\end{document}
