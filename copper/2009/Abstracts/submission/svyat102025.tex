\documentclass{report}
\usepackage{amsmath,amssymb}
\setlength{\parindent}{0mm}
\setlength{\parskip}{1em}
\begin{document}
\begin{center}
\rule{6in}{1pt} \
{\large Daniil Svyatskiy \\
{\bf An Adaptive Hierarchy of Discretizations for Two-Phase Flows in Porous Media}}

Applied Mathematics and Plasma Physics \\ Theoretical Division \\ Mail Stop B284 \\ Los Alamos National Laboratory \\ Los Alamos \\ NM 87545 \\ U S A
\\
{\tt dasvyat@lanl.gov}\\
Konstantin Lipnikov\\
{\em Los Alamos National Laboratory}\\
David Moulton\\
{\em Los Alamos National Laboratory}\end{center}

We present a multiscale method which builds recursively a
problem-dependent multilevel hierarchy of models for flow in
heterogeneous porous media. Each model preserves important physical
properties, such as local mass conservation. In contrast with classical
two-level methods that achieve a total coarsening factor of approximately
10, in each coordinate direction, the multilevel hierarchal approach
facilitates large total coarsening factors of 100 or more.
Maintenance of the hierarchy of models incurs only a modest
computational overhead due to the efficiency of recursive
coarsening and adaptive update strategies. The method supports full
diffusion tensors on unstructured polyhedral meshes and accommodates
general coarsening strategies. Due to its algebraic nature, the method
can be adapted to different types of discretization methods, such as the
Mixed Finite
Element, Finite Volume and Mimetic Finite Differences methods.


To efficiently maintain fine-scale accuracy in the multiscale
solution, the method incorporates two adaptive
strategies. First, the hierarchy of models is updated locally when
the relative permeability, which depends on the water saturation,
changes significantly. Second, an efficient error indicator which
controls the temporal updates of the flux coarsening parameters. This new
strategy concentrates updates around critical times when the invading
fluid (water) first enters key features of the reservoir.

To discretize this system in time we use the IMPES time discretization
scheme (IMplicit Pressure and Explicit Saturation). The saturation is
updated using the Darcy velocities provided by the pressure solver. The
permeability fields with long correlation lengths are challenging
problems for multiscale methods. Numerical simulations for these type of
fields show that even with large coarsening factors, such as 256 in the
vicinty of wells, in each coordinate direction the multiscale solution
remains within 5\% of the finescale solution.


\end{document}
