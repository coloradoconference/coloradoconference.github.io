\documentclass{report}
\usepackage{amsmath,amssymb}
\setlength{\parindent}{0mm}
\setlength{\parskip}{1em}
\begin{document}
\begin{center}
\rule{6in}{1pt} \
{\large Pablo Navarrete \\
{\bf Signal Processing Approach to avoid Smoothing Iterations in Multi-grid Methods}}

Department of Electrical Engineering \\ Universidad de Chile \\ Av Tupper 2007 \\ Santiago \\ RM 8370451 \\ Chile
\\
{\tt pnavarrete@ing.uchile.cl}\end{center}

Modifications of the conventional muti-grid algorithm are studied to
avoid the use of smoothing iterations. In the full multi�grid algorithm,
classical smoothing iterations (e.g. Gauss�Seidel) reduce high�frequency
components of the error and a coarse�grid approach reduces the
low�frequency components of the error. The problem here is that two
methods with different structures are being combined, which introduces
additional complexity in the convergence analysis of multi�grid methods.
Then, the idea is to avoid the use of smoothing iterations by using
different inter�grid configurations and the concept of quadrature mirror
filters, which are well known in the area of signal processing and
particularly in wavelet analysis. This framework can be introduced by
using the structure of the extended convergence analysis introduced in
[1] from which the classical Local Fourier Analysis (LFA) is a particular
case. This can provide an integrated configuration to efficiently reduce
low� and high� frequency components of the error as well as aliasing
effects between the low- and high- frequency components of the error. The
possibility of a direct solver is studied and the conditions under which
it can be implemented.

[1]P. Navarrete, and E.J. Coyle, ``A semi-algebraic approach that enables
the design of inter-grid operators to optimize multigrid convergence,''
Numerical Linear Algebra with Applications, Vol. 15, No. 2-3, pp.
219-247, March 2008.


\end{document}
