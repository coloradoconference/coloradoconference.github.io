\documentclass{report}
\usepackage{amsmath,amssymb}
\setlength{\parindent}{0mm}
\setlength{\parskip}{1em}
\begin{document}
\begin{center}
\rule{6in}{1pt} \
{\large Scott MacLachlan \\
{\bf An angular multigrid method for modeling charged-particle transport in Flatland }}

Department of Mathematics \\ Tufts University \\ Bromfield-Pearson Hall \\ 503 Boston Ave \\ Medford \\ MA 02155
\\
{\tt scott.maclachlan@tufts.edu}\\
Christoph B\"orgers\\
{\em Department of Mathematics, Tufts University}\end{center}

Beams of microscopic particles penetrating scattering background matter
play an important role in several applications. In this work, we consider
parameter choices that are motivated by the problem of electron-beam
cancer therapy planning. Mathematically, a steady particle beam
penetrating matter, or a configuration of several such beams, is modeled
by a boundary value problem for a Boltzmann equation. Grid-based
discretization of such a problem leads to a system of algebraic
equations, which is typically very large because of the large number of
independent variables in the Boltzmann equation (six if no
dimension-reducing assumptions other than time independence are made). If
grid-based methods are to be practical for these problems, it is
necessary to develop fast solvers for the discretized problems.

For beams of mono-energetic particles interacting with a passive
background, but not with each other, in two space dimensions, an angular
domain decomposition was proposed by B\"orgers in 1997. In this talk, we
discuss an angular multigrid algorithm for the same model problem, based
on a careful choice of relaxation and coarse-grid correction processes.
Our numerical experiments show rapid, grid-independent convergence for
the forward-peaked scattering typical of electron beams. Unlike angular
domain decomposition, the angular multigrid method works well even when
the angular diffusion coefficient is fairly large.


\end{document}
