\documentclass{report}
\usepackage{amsmath,amssymb}
\setlength{\parindent}{0mm}
\setlength{\parskip}{1em}
\begin{document}
\begin{center}
\rule{6in}{1pt} \
{\large Sergii Veremieiev \\
{\bf Multigrid strategies for the efficient and accurate solutions of free-surface film flow on man-made and naturally occurring functional substrates}}

School of Mechanical Engineering \\ University of Leeds \\ Leeds \\ West Yorkshire \\ United Kingdom \\ LS2 9JT
\\
{\tt s.veremieiev@leeds.ac.uk}\\
Phillip H. Gaskell\\
{\em School of Mechanical Engineering,University of Leeds,Leeds, West Yorkshire,United Kingdom.LS2 9JT}\\
Yeaw Chu Lee\\
{\em School of Mechanical Engineering,University of Leeds,Leeds, West Yorkshire,United Kingdom.LS2 9JT}\\
Harvey M. Thompson\\
{\em School of Mechanical Engineering,University of Leeds,Leeds, West Yorkshire,United Kingdom.LS2 9JT}\end{center}

The deposition and flow of continuous thin liquid films over man-made or
naturally occurring functional substrates containing regions of
micro-scale topography (which may be fully submerged or extend through
the free-surface of the film itself) plays an important role in numerous
engineering and biologically related fields. For example, in the context
of engineering processes thin film flows form a key role in
photo-lithography [1] and precision coating processes [2], while in
biological systems they occur in areas as diverse as tissue engineering
[3] and plant disease control [4].

Current analytic and experimental methods are incapable, and likely to
remain so for the foreseeable future, of either meeting the considerable
challenges posed in unravelling the underlying physics of so wide a range
of real problems or of providing the necessary insight for improving
man-made functional substrates and the design and creation of novel ones.
Equally, it would be naive to pretend that there currently exists an
off-the-shelf computational answer involving a combined multi-scale
modelling approach comprised of a strategic mix of molecular dynamics,
meso-scale and continuum methods. The reality is that at present, the
modelling of three-dimensional free-surface film flows over substrates
containing complex topography is still at an early stage of development,
for which exploitation of the long-wave approximation is the focus. The
key simplifying feature of the latter is that it enables the reduction of
the governing time-dependent Navier-Stokes equations to a more tractable
coupled system of partial differential equations; nevertheless the
resulting reduced equation set is still required to be solved both
efficiently and accurately.

In the context of the above, two such formulations are explored and
solved for: (i) a simple lubrication (LUB) approach [5], where the
dependent variables are the film height and pressure; (ii) a novel depth
averaged form (DAF), which involves the determination of three variables
� two velocity components and the film height. Approach (ii), unlike (i),
enables thin film flows with inertia to be predicted and its effect
quantified. Solution of the nonlinear problems of interest is extremely
challenging since large computational domains and fine mesh structures
are required: (a) to ensure grid independent solutions; (b) to capture
persistent free surface disturbances caused by both localised (single)
and distributed (multiply-connected) topography; (c) as the length-scale
of the topographical features involved become smaller and the resolution
required to capture the resultant flow accurately becomes increasingly
important.

The research reported describes the development and application of an
efficient, multigrid algorithm to solve the two different resulting
equation sets, which implements the full approximation storage (FAS) and
full multigrid (FMG) schemes together with a standard fixed number of
pre- and post-smoothing V-cycles and appropriate inter-grid transfer and
smoothing operators paired with Newton-Raphson iteration; appropriate
account is also taken of the co-located (LUB) or staggered (DAF) meshing
strategies employed. The associated time-descretisation includes the use
of an explicit predictor [6] and a semi-implicit $\beta$-method [7]
solution stages where automatic adaptive time-stepping is controlled in
terms of the local truncation error on the finest grid level.

The utility of the multigrid methodology is explored in two different ways:
\begin{enumerate}
\item As a scalable parallel, portable object-oriented algorithm
implemented on the following HPC architectures: HECToR, HPCx and
BlueGene/P.
\item As a serial algorithm but with the additional feature of error
controlled automatic mesh adaption [8].
\end{enumerate}

In terms of achieving efficient and accurate solutions, the former is
shown to deliver the expected benefits that parallelisation and the use
of multiple processor computing platforms bring and to be particularly
well suited to predicting thin film flow on surfaces containing densely
packed and complex topographical features; the latter is found to be
preferentially suited to the case of film flow over localised or sparsely
distributed topographical features were mesh adaption leads to
considerable savings in CPU times without loss of accuracy.

The examples chosen to illustrate the range of applicability of the above
solution strategies are centred on the gravity-driven flow of thin films
over substrates inclined at an angle $\theta$ to the horizontal and
include: (i) comparison of the accuracy of predicted thin film profiles
against experimental and complementary finite element solutions of the
full Navier-Stokes problem, for varying flow parameters and both LUB and
DAF approaches, in the case of two-dimensional span-wise topographies
such as a step-up, a step-down, trenches and peaks; (ii) flow past
three-dimensional localised surface patterns illustrating the influence
of inertia and its effect with varying topography shape and size; (iii)
the analysis of flow over an engineered man-made multiple-connected
surface pattern where the practical goal is that of minimising
free-surface disturbances caused by the topography; (iv) thin film flow
over leaf sections exhibiting heterogeneous surface patterning.

\flushleft{
[1] W.K. Ho, A. Tay, L.L. Lee, C.D. Schaper, �On control of resist film
uniformity in the microlithography process�, Control. Eng. Prac., 12(7),
881-892, 2004.

[2] A. Clarke, �Coating on a rough surface�, A.I.Ch.E. Journal, 48, 21492156, 2002.

[3] D.P. Gaver, J.B. Grotberg, �The Dynamics of a Localized Surfactant on
a Thin-Film�, J. Fluid Mech., 213, 127-148, 1990.

[4] D. R. Walters, �Disguising the leaf surface: the use of leaf coatings
for plant disease control�, Euro. J. Plant Pathology, 114, 255-260, 2006.

[5] P.H. Gaskell, P.K. Jimack, M. Sellier, H.M. Thompson, M.C.T. Wilson,
�Gravity-driven flow of continuous thin liquid films on non-porous
substrates with topography�, J. Fluid Mech., 509, 253-280, 2004.

[6] P.H. Gaskell, P.K. Jimack, M. Sellier, H.M. Thompson, �Efficient and
accurate time adaptive multigrid simulations of droplet spreading�, Int.
J. Num. Meth. Fluids, 45, 1161-1186, 2004.

[7] T.J. Chung, Computational Fluid Dynamics. Cambridge: Cambridge
University Press, 2002.

[8] Y.C. Lee, H.M. Thompson, P.H. Gaskell, �An efficient adaptive
multigrid algorithm for predicting thin film flow on surfaces containing
localized topographic features�, Computers \& Fluids, 38, 838-855, 2007.
}


\end{document}
