\documentclass{report}
\usepackage{amsmath,amssymb}
\setlength{\parindent}{0mm}
\setlength{\parskip}{1em}
\begin{document}
\begin{center}
\rule{6in}{1pt} \
{\large Chen Greif \\
{\bf Cyclic Reduction and Multigrid}}

2366 Main Mall \\ Vancouver B C V6T 1Z4 \\ Canada
\\
{\tt greif@cs.ubc.ca}\\
Irad Yavneh\\
{\em Technion - Israel Institute of Technology}\end{center}

Cyclic reduction is a process by which red/black ordering is initially
applied, and then the unknowns that correspond to one of the unknowns are
eliminated. The resulting operator is a Schur complement matrix that is
easy to construct if a 5-pt (7-pt) computational molecule was originally
used in 2D (3D) on the fine mesh. In this talk we show analytically and
experimentally that in fact the cyclically reduced operator has very good
h-ellipticity properties. We discuss various ways of using the operator
as an effective smoother for convection-diffusion equations.


\end{document}
