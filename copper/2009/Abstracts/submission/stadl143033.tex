\documentclass{report}
\usepackage{amsmath,amssymb}
\setlength{\parindent}{0mm}
\setlength{\parskip}{1em}
\begin{document}
\begin{center}
\rule{6in}{1pt} \
{\large Georg Stadler \\
{\bf Experience with parallel AMG for large-scale adaptive mantle convection}}

Institute for Computational Engineering and Sciences (ICES) \\ The University of Texas at Austin \\ 1 University Station \\ C0200 \\ Austin \\ TX 78712
\\
{\tt georgst@ices.utexas.edu}\\
Carsten Burstedde\\
{\em Institute for Computational Engineering and Sciences (ICES)
The University of Texas at Austin, Austin, TX}\\
Omar Ghattas\\
{\em Institute for Computational Engineering and Sciences (ICES) and Depts.\ of Geological Sciences and Mechanical Engineering
The University of Texas at Austin, Austin, TX}\\
Michael Gurnis\\
{\em Seismological Laboratory, California Institute of Technology, Pasadena, CA}\\
Eh Tan\\
{\em Computational Infrastructure for Geodynamics, California Institute of Technology, Pasadena, CA}\\
Tiankai Tu\\
{\em Institute for Computational Engineering and Sciences (ICES)
The University of Texas at Austin, Austin, TX}\\
Lucas W. Wilcox\\
{\em Institute for Computational Engineering and Sciences (ICES)
The University of Texas at Austin, Austin, TX}\\
	Zhong, Shijie\\
{\em Department of Physics, University of Colorado, Boulder, CO}\end{center}

Mantle convection is the principal control on the thermal and
geological evolution of the Earth. Its modeling involves solution of the
mass, momentum, and energy equations for a viscous, creeping,
incompressible non-Newtonian fluid at high Rayleigh and Peclet numbers.
We are developing the parallel adaptive mantle convection code
\texttt{Rhea}, which builds on our \textbf{A}daptive \textbf{L}arge-scale
\textbf{P}arallel \textbf{S}imulations (ALPS) framework.

The numerical simulation of mantle convection requires the solution of a
stationary, highly variable-viscosity Stokes system at each time step.
The Stokes equation is discretized using trilinear finite elements for
both velocity and pressure, and polynomial pressure projection is used to
stabilize this equal-order approximation. For the solution of the
discrete Stokes saddle point system we use the preconditioned minimal
residual (MINRES) method. The preconditioner is based on an approximate
Schur complement for the pressure component and involves a solve with a
positive definite operator for the velocity filed. This block solve is
approximated with one V-cycle
of algebraic multigrid. Results obtained with the parallel AMG
libraries {\em BoomerAMG} from {\em hypre} and with {\em ML} from
{\em Trilinos} will be shown. We discuss the parallel scalability of the
solver on up to 16K cores, as well as its dependence on viscosity
variations and boundary conditions.


\end{document}
