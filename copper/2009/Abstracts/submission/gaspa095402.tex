\documentclass{report}
\usepackage{amsmath,amssymb}
\setlength{\parindent}{0mm}
\setlength{\parskip}{1em}
\begin{document}
\begin{center}
\rule{6in}{1pt} \
{\large Francisco Gaspar \\
{\bf Design of geometric multigrid methods on semi-structured grids}}

Department of Applied Mathematics \\ University of Zaragoza \\ c/Maria de Luna 3 \\ 50018 Zaragoza \\ Spain
\\
{\tt fjgaspar@unizar.es}\\
Jose Luis Gracia\\
{\em University of Zaragoza}\\
Francisco Lisbona\\
{\em University of Zaragoza}\\
Carmen Rodrigo\\
{\em University of Zaragoza}\end{center}

We are interested in the design of efficient geometric multigrid methods on hierarchical
triangular grids for problems in two dimensions. Assuming that the
coarsest grid is rough enough in order to fit the geometry of the domain, a hierarchy
of globally unstructured grids is generated. This kind of meshes are suitable
for use with geometric multigrid. To discretize problems with constants coefficients
on these type of meshes, explicit assembly of the global stiffness matrix for
the finite element method is not necessary and this can be implemented using
stencils. As the stencil for each coarsest triangle is the same for all unknowns
that are interior to it, one stencil suffices to represent the discrete operator reducing
drastically the memory required. Fourier analysis is a well-known useful
tool in multigrid for the prediction of two-grid convergence rates. With the
help of the Fourier Analysis on triangular grids, we design efficient geometric
multigrid methods for different problems on hierarchical triangular grids.


\end{document}
