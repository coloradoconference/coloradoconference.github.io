\documentclass{report}
\usepackage{amsmath,amssymb}
\setlength{\parindent}{0mm}
\setlength{\parskip}{1em}
\begin{document}
\begin{center}
\rule{6in}{1pt} \
{\large Allison H. Baker \\
{\bf Algebraic Multigrid for Elasticity}}

Lawrence Livermore National Laboratory \\ Box 808 \\ L-560 \\ Livermore \\ CA 94551-0808
\\
{\tt abaker@llnl.gov}\\
Tzanio V. Kolev\\
{\em Lawrence Livermore National Laboratory}\\
Ulrike Meier Yang\\
{\em Lawrence Livermore National Laboratory}\end{center}

We are interested in the efficient solution of large systems of PDEs
arising from elasticity applications. When solving linear systems
derived from systems of PDEs with AMG, two accepted approaches are
treating variables corresponding to the same unknown separately (the
"unknown" approach) and treating variables corresponding to the same
physical node together (the "nodal" approach). While the unknown
approach is typically chosen for elasticity problems, convergence is
often far from ideal.

In this talk, we investigate improving the interpolation of the rigid
body modes. In particular, we propose extending the AMG interpolation
operator to exactly interpolate the rotational rigid body modes by
adding additional degrees of freedom (dofs) at each node. Our approach
is an unknown-based approach that builds upon any existing AMG
interpolation strategy and requires nodal coarsening. The approach
fits easily into the AMG framework and does not require any matrix
inversions. We demonstrate the effectiveness on several 2D and 3D
elasticity problems.


\end{document}
