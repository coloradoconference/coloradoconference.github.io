\documentclass{report}
\usepackage{amsmath,amssymb}
\setlength{\parindent}{0mm}
\setlength{\parskip}{1em}
\begin{document}
\begin{center}
\rule{6in}{1pt} \
{\large Pascal POULLET \\
{\bf Multilevel methods for solving the Helmholtz equation in unbounded domains}}

Laboratoire GRIMAAG \\ Universite des Antilles et de la Guyane \\ B P 250 \\ Pointe-a-Pitre F-97157 Cedex \\ Guadeloupe F W I
\\
{\tt Pascal.Poullet@univ-ag.fr}\\
Amir BOAG\\
{\em School of Electrical Engineering, Tel Aviv University, Israel}\end{center}

The incremental unknowns (IU) method has been employed for nearly
two decades as a tool of computational fluid dynamics while providing
features and benefits similar to those of the classical multigrid method.
In particular, several research groups proved that Poisson's equation
with Dirichlet boundary conditions can be solved roughly as efficiently
using the IU approach as by the classical multigrid \cite{CT91, CT93}.
A recent study \cite{PB07} conducted by the authors proved that,
for solving the Helmholtz equation, one can develop a multilevel
computational scheme based on the IUs, which is efficient in high and low
frequency regimes.
We have shown on a sample 2D acoustic scattering problem that for the
best performance the number of IU levels used in the preconditioner
should be designed for the coarsest grid to have roughly two points per
linear wavelength. A recently proposed novel class of methods relying on
incomplete factorization of shifted Laplacian operator provides
attractive results, but
requires significant storage \cite{E08}. Memory considerations are
especially important when we need to solve 3D problems. To that end, the
IU approach appears to be especially advantageous thanks to its low
memory requirements. Even if not as efficient as for 2D problems, studies
of IU-based techniques for 3D problems proved a promising behavior of the
method for solve elliptic problems \cite{MCT95, SW07}.

A new multilevel preconditioner for the Helmholtz equation in 2D
using two types of incremental unknowns has been developed \cite{PB08}.
The transition between the two occurs when the mesh size reaches a
predetermined fraction of the wavelength - roughly one quarter
wavelength. Conventional IUs based on bilinear interpolation are employed
for the fine meshes like in the previous paper \cite{PB07} by the
authors. For coarse mesh sizes novel IUs are designed using a
Helmholtz/wave equation-based interpolation. The interpolation
coefficients for the coarse meshes with dimension higher than one are
derived numerically for stencils resembling integral representations for
interior points. In two dimensions the IUs are located on crosses
surrounded by square contours. The proposed approach has been proved
effective in reducing the condition numbers and accelerating
convergence for coarse grids with mesh sizes exceeding the wavelength.
Our research has been aimed at devising novel computational schemes that
facilitate analysis of large scattering problems. The effort so far has
been directed at the development of two-dimensional multilevel
preconditioner for high and low frequency cases. The above fast approach
has been incorporated into an existing iterative solver. On one hand, one
can consider our technique as being related to the multigrid method, in
the sense of using stencils constructed based on the ideas stemming from
a boundary element method. On the other hand, the idea is not too far
from a domain decomposition technique.

Recently, an extension of this method for solving the Helmholtz equation
in 3D is considered. While, the first type of IU consists of the
adaptation to an unbounded domain problem of the formulas which have been
introduced in \cite{MCT95}. The second type is designed using a
Helmholtz/wave equation-based interpolation similar to that introduced in
\cite{PB08}.
In three dimensions, the IUs are located at the intersections between
pairs of planes (three pairs) parallel to the axes, and enclosed in a
surface of the surrounding cube.
An optimal strategy to combine the two types of IUs is expected to emerge
depending on the wavenumber and the size of the mesh.

\begin{references}{99}

\bibitem{CT91} M. Chen and R. Temam,
{\em Incremental unknowns for solving partial differential equations},
Numer. Math. {\bf 59 (3)}, (1991) 255-271.

\bibitem{CT93} M. Chen and R. Temam,
{\em Incremental unknowns in finite differences: condition number of the matrix},
SIAM J. Matrix Anal. Appl. {\bf 14 (2)}, (1993) 432-455.

\bibitem{MCT95} M. Chen, A. Miranville and R. Temam,
{\em Incremental unknowns in finite differences in three space dimensions},
Mat. Appl. Comput. {\bf 14 (3)}, (1995) 219-252.

\bibitem{E08} Y.A. Erlangga,
{\em Advances in iterative methods and preconditioners for the
Helmholtz equation},
Arch. Comput. Meth. Eng. {\bf 15}, (2008) 37-66.

\bibitem{PB07} P. Poullet and A. Boag,
{\em Incremental unknowns preconditioning for solving the Helmholtz equation},
Numer. Meth. for PDEs {\bf 23 (6)}, (2007) 1396-1410.

\bibitem{PB08} P. Poullet and A. Boag,
{\em Equation-based interpolation and incremental unknowns for solving
the Helmholtz equation}, {\it submitted}

\bibitem{SW07} L. Song and Y. Wu,
{\em Incremental unknowns in three-dimensional stationary problem},
Numer. Algor. {\bf 46}, (2007) 153-171.

\end{references}


\end{document}
