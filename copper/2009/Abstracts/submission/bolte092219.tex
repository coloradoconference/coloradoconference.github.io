\documentclass{report}
\usepackage{amsmath,amssymb}
\setlength{\parindent}{0mm}
\setlength{\parskip}{1em}
\begin{document}
\begin{center}
\rule{6in}{1pt} \
{\large Matthias Bolten \\
{\bf Local Fourier analysis of interpolation operators for problems with certain complex stencils -- preliminary results}}

Department of Mathematics and Science \\ University of Wuppertal \\ D-42097 Wuppertal \\ Germany
\\
{\tt bolten@math.uni-wuppertal.de}\end{center}

Local Fourier analysis (LFA) (see e.g.~\cite{art:BRAN77,boo:TROT01}) is
known to be a valuable tool for the development of geometric multigrid
methods for PDEs allowing an effective tuning of multigrid components.
The main concept of LFA is the idea of keeping local stencils fixed and
treating the (local) problems as if they where part of the associated
infinitely large constant coefficient problems. This allows for an
optimal choice of smoothers as well as restriction operators.

The infinitely large problem with constant coefficients corresponds to a
($l$-level) Toeplitz operator. The Toeplitz operators are completely
described by their generating symbol, a ($l$-variate) $2 \pi$-periodic
function. For second order elliptic problems the generating symbol has a
unique zero of order two at the origin. The multigrid methods for
Toeplitz matrices and circulant matrices that have been developed in the
last years (c.f.~\cite{art:FIOR96,art:FIOR96a,art:SERR02,art:SERR04}),
work well for these problems, and they do not depend on the location of
the zero. In fact, the zero of the generating symbol just influences the
choice of the interpolation operator in the multigrid method.

Combining these developments we are able to provide local interpolation
operators for matrices with non-constant stencils with complex entries,
consider e.g. the stencil
\begin{equation*}
\frac{1}{h^{2}} \left[ \begin{array}{ccc}
& -e^{2 \pi i \varphi_{x,y+\mu}} & \\
-e^{-2 \pi i \varphi_{x-\mu,y}} & 4 & -e^{2 \pi i \varphi_{x+\mu,y}} \\
& -e^{-2 \pi i \varphi_{x,y-\mu}} &
\end{array} \right]
\end{equation*}
where $\varphi_{x-\mu,y}, \varphi_{x+\mu,y}, \varphi_{x,y-\mu},
\varphi_{x,y+\mu} \in [0,1]$ are some random parameters. This can be
considered as a non-shifted variant of the 2d Gauge Laplace matrix that
arises in a simplified model of lattice quantum chromodynamics. The idea
is similar as in LFA, namely take the local stencil as a constant stencil
in an infinitely large system. Using results of Serra-Capizzano and
Tablino-Possio in~\cite{art:SERR04} we are able to provide interpolation
operators for these matrices, yielding in our local definition of the
interpolation.

In this talk we will give a short overview over the used results,
introduce the concept in larger detail and present some numerical results
for the two-grid case.

\begin{thebibliography}{1}

\bibitem{art:BRAN77}
A.~Brandt.
\newblock Multi-level adaptive solutions to boundary-value problems.
\newblock {\em Math. Comp.}, 31(138):333--390, 1977.

\bibitem{art:FIOR96}
G.~Fiorentino and S.~Serra.
\newblock Multigrid methods for indefinite {Toeplitz} matrices.
\newblock {\em Calcolo}, 33:223--236, 1996.

\bibitem{art:FIOR96a}
G.~Fiorentino and S.~Serra.
\newblock Multigrid methods for symmetric positive definite block {Toeplitz}
matrices with nonnegative generating functions.
\newblock {\em SIAM J. Sci. Comput.}, 17(5):1068--1081, 1996.

\bibitem{art:SERR02}
S.~Serra-Capizzano.
\newblock Convergence analysis of two-grid methods for elliptic Toeplitz and
PDEs matrix sequences.
\newblock {\em Numer. Math.}, 92:433--465, 2002.

\bibitem{art:SERR04}
S.~Serra-Capizzano and C.~Tablino-Possio.
\newblock Multigrid methods for multilevel circulant matrices.
\newblock {\em SIAM J. Sci. Comput.}, 26(1):55--85, 2004.

\bibitem{boo:TROT01}
U.~Trottenberg, C.~Oosterlee, and A.~Sch{\"u}ller.
\newblock {\em Multigrid}.
\newblock Academic Press, San Diego, 2001.

\end{thebibliography}


\end{document}
