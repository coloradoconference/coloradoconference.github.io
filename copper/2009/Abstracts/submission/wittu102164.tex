\documentclass{report}
\usepackage{amsmath,amssymb}
\setlength{\parindent}{0mm}
\setlength{\parskip}{1em}
\begin{document}
\begin{center}
\rule{6in}{1pt} \
{\large Gabriel Wittum \\
{\bf Modeling and Numerical Simulation of Biological Systems by Multigrid Methods}}

Kettenhofweg 139 \\ 60325 Frankfurt am Main
\\
{\tt wittum@g-csc.de}\\
Gabriel Wittum\\
{\em G-CSC, University of Frankfurt}\end{center}

A typical feature of biological systems is their high complexity and
variability. This makes modelling and computation very difficult, in
particular for detailed models based on first principles. The problem
starts with modelling geometry, which has to extract the essential
features from those highly complex and variable phenotypes and at the
same time has to take in to account the stochastic variability. Moreover,
models of the highly complex processes which are going on these
geometries are far from being well established, since those are highly
complex too and often couple on a hierarchy of scales in space and time.
Simulating such systems always puts the whole approach to test, including
modeling, numerical methods and software implementations. In combination
with validation based on experimental data, all components have to be
enhanced to reach a reliable solving strategy.

To handle problems of this complexity, new mathematical methods and
software tools are required. In recent years, new approaches such as
parallel adaptive multigrid methods and corresponding software tools have
been developed allowing to treat problems of huge complexity. In the
lecture we report on the numerical simulation of the diffusion of
xenobiotics through human skin. First computations for this problem were
made in he last decade, yielding new insight into permeation pathways
through human skin, which were confirmed experimentally ten years later.


\end{document}
