\documentclass[11pt]{article}
\usepackage{amsfonts}
\setlength{\parskip}{1.2ex}      % space between paragraphs
\setlength{\parindent}{0em}      % amount of indention
\setlength{\textwidth}{165mm}    % default = 6.5in
\setlength{\oddsidemargin}{0mm}  % default = 0mm
\setlength{\textheight}{225mm}   % default = 9in
\setlength{\topmargin}{-1mm}     % default = 0mm
\date{ ~ \hspace{-4mm}}


\title{Neumann-Neumann and Feti Preconditioners for Edge Element Approximations of Maxwell's Equations  }

\author{Andrea Toselli \\ {\tt  toselli@cims.nyu.edu} \\ Courant Institute, 251 Mercer Street \\ New York, NY 10012}

\begin{document}
\maketitle
\thispagestyle{empty}





 



In this talk, we report on some new results on two iterative
substructuring methods of Neumann-Neumann and FETI type 
for some edge element approximations
of Maxwell equations in two dimensions. 



Iterative substructuring methods provide powerful
preconditioners for the solution of
linear systems arising from the finite element approximation
of partial differential equations.
In these methods, the computational domain is
partitioned into non-overlapping subdomains and a
preconditioner is built by solving local problems on the subdomains.



The Neumann-Neumann and FETI preconditioners that we consider have many 
algorithmic components in common, such as the solution of local
Dirichlet and Neumann problems, a set of local functions needed
to build a coarse space correction, and a set of scaling functions
that employ the values of the coefficients on the subdomains
and ensure that the condition number of the resulting linear
system is independent of the jumps of the coefficients.
In addition, FETI methods are more easily generalizable to approximations
on non-matching grids.



For both methods, we show  bounds for the condition number of the
corresponding preconditioned systems which grow logarithmically with 
the number of unknowns in each subdomain. Such bounds are independent of 
possibly large jumps of the coefficients across the boundaries of the 
subdomains.
For a Neumann-Neumann and a FETI method, 
we present some numerical results for conforming
approximations. We will also show some results for a FETI method
for a mortar approximation on non-matching grids.







\end{document}
