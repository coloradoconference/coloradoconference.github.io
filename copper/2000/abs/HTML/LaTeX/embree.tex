\documentclass[11pt]{article}
\usepackage{amsfonts}
\setlength{\parskip}{1.2ex}      % space between paragraphs
\setlength{\parindent}{0em}      % amount of indention
\setlength{\textwidth}{165mm}    % default = 6.5in
\setlength{\oddsidemargin}{0mm}  % default = 0mm
\setlength{\textheight}{225mm}   % default = 9in
\setlength{\topmargin}{-1mm}     % default = 0mm
\date{ ~ \hspace{-4mm}}


\title{Invariant Subspace Convergence for Arnoldi's Method  }

\author{Mark Embree \\ {\tt mark.embree@comlab.ox.ac.uk} \\ Oxford University Computing Laboratory  \\  Wolfson Building  \\  Parks Road, Oxford OX1 3QD  \\  England}

\begin{document}
\maketitle
\thispagestyle{empty}





 



Krylov subspace eigenvalue algorithms typically aim to efficiently 
calculate some subset of the spectrum with a special property, such
as the rightmost eigenvalues or minimum modulus eigenvalues.
What matrix properties determine the rate at which the Krylov subspace 
captures the invariant subspace associated with the desired eigenvalues?
When a matrix is normal, convergence only depends upon the dimension
of the invariant subspace, the distance between desired and undesired
eigenvalues, and biases of the starting vector.




Non-normality leads to a more complicated situation.
In this talk, we develop bounds on the angle between the Krylov 
subspace and the desired invariant subspace for general
(non-derogatory) matrices.  
These bounds take the form of multiplicative constants based 
on the quality of the starting vector and the non-normality 
of the matrix, along with a polynomial approximation problem over the pseudospectra.
In the context of restarted iterations, the bounds inform 
shift selection and choice of the basis dimension.
They also describe convergence when the desired invariant subspace
is associated with a defective eigenvalue, and
when the desired eigenvalues are stable but 
the undesired eigenvalues are ill-conditioned.
Examples illustrate the applicability of these bounds
to a variety of situations.




This talk describes joint research with Christopher Beattie and
John Rossi.







\end{document}
