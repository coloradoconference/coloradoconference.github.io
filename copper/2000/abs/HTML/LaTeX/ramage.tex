\documentclass[11pt]{article}
\usepackage{amsfonts}
\setlength{\parskip}{1.2ex}      % space between paragraphs
\setlength{\parindent}{0em}      % amount of indention
\setlength{\textwidth}{165mm}    % default = 6.5in
\setlength{\oddsidemargin}{0mm}  % default = 0mm
\setlength{\textheight}{225mm}   % default = 9in
\setlength{\topmargin}{-1mm}     % default = 0mm
\date{ ~ \hspace{-4mm}}


\title{Iterative convergence for stabilised finite element discretisations of advection-diffusion problems.  }

\author{Alison Ramage \\ {\tt  A.Ramage@strath.ac.uk} \\ Department of Mathematics,  University of Strathclyde \\ 26 Richmond Street, Glasgow G1 1XH, Scotland, UK}

\begin{document}
\maketitle
\thispagestyle{empty}





 



This talk is concerned with the design of robust and 
efficient iterative methods for solving advection-diffusion
equations. Specifically, we consider the stabilisation
of discrete finite element approximations on uniform
grids which do not resolve boundary layers. Such
stabilisation is necessary when the ratio of convection to 
diffusion is large, in order to prevent oscillations in the discrete 
solutions. 



The particular stabilisation method featured in this talk
is the commonly-used streamline diffusion approach, where
a certain amount of additional diffusion is added in the
direction of the flow streamlines  in order to damp
these oscillations. One issue which immediately arises is
how much  diffusion should be added: adding too little will not
damp the oscillations sufficiently, but adding too much will result in
an overly smooth and inaccurate solution. In the first part
of the talk, we will use Fourier analysis
to establish an appropriate value for the streamline
diffusion parameter in terms of producing accurate solutions.



In addition, we face the equally important question of how to solve the resulting
matrix equation efficiently. The large size of the linear systems
arising from discretisations of practical problems, particularly
in three dimensions, means that iterative solution methods are often
the only feasible option. In the second part of the talk, we consider
the choice of streamline diffusion parameter from the point of view
of ensuring fast convergence of smoothers based on the standard
GMRES iteration, and show that there is a symbiotic relationship between 
`optimal' solution approximation (in the above sense) and efficient
GMRES performance.





\end{document}
