\documentclass[11pt]{article}
\usepackage{amsfonts}
\setlength{\parskip}{1.2ex}      % space between paragraphs
\setlength{\parindent}{0em}      % amount of indention
\setlength{\textwidth}{165mm}    % default = 6.5in
\setlength{\oddsidemargin}{0mm}  % default = 0mm
\setlength{\textheight}{225mm}   % default = 9in
\setlength{\topmargin}{-1mm}     % default = 0mm
\date{ ~ \hspace{-4mm}}


\title{Aggregation-Based Domain Decomposition Methods for Unsaturated Flow I: Motivation and Design  }

\author{C. T. Kelley \\ {\tt tim\_kelley@ncsu.edu} \\ Center for Research in Scientific Computation \\ Department of Mathematics, Box 8205 \\ North Carolina State University \\ Raleigh, NC 27695-8205}

\begin{document}
\maketitle
\thispagestyle{empty}





 



Newton-Krylov-Schwarz methods solve nonlinear equations by using
Newton's method with a Schwarz domain decomposition preconditioned
Krylov method to approximate the Newton step. In this talk we will discuss
the design and implementation of Newton-Krylov-Schwarz solvers in the
context of the implicit temporal integration on an unstructured
three-dimensional spatial mesh of Richards' equation for flow in the
unsaturated zone. The issues include nonsmooth nonlinearities, construction
and efficient implementation of the coarse-mesh problem, and temporal
integration. The second part of the talk, given by E. W. Jenkins,
will discuss the convergence theory for the method.





\end{document}
