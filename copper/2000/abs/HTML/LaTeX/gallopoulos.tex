\documentclass[11pt]{article}
\usepackage{amsfonts}
\setlength{\parskip}{1.2ex}      % space between paragraphs
\setlength{\parindent}{0em}      % amount of indention
\setlength{\textwidth}{165mm}    % default = 6.5in
\setlength{\oddsidemargin}{0mm}  % default = 0mm
\setlength{\textheight}{225mm}   % default = 9in
\setlength{\topmargin}{-1mm}     % default = 0mm
\date{ ~ \hspace{-4mm}}


\title{{\bf  On the computational effectiveness of transfer function approximations to the matrix pseudospectrum. }  }

\author{C. Bekas and E. Gallopoulos ({\tt stratis@hpclab.ceid.upatras.gr}) \\ {\em  Department of Computer Engineering and Informatics \\ University of Patras, Greece } \\ \\ V. Simoncini  \\  {\em  IAN-CNR Pavia, Italy }}

\begin{document}
\maketitle
\thispagestyle{empty}





The $\epsilon$-
{\em 
pseudospectrum
}
of a matrix,
defined for example as

$$
\Lambda_{\epsilon} (A) = \{ z: z \in \Lambda(A+E)
\mbox{~~for some~} E\in C^{n\times n}
\mbox{~~with~} \| E \| \leq \epsilon \}
$$

where $\Lambda(A)$ denotes the spectrum of $A$, is acknowledged
to be a powerful mechanism for investigating the behavior of several (nonnormal)
matrix-dependent algorithms, ranging from iterative methods for large linear
systems to time-stepping algorithms; note the inclusion of specific functions
to that effect in the popular
Test Matrix Toolbox
(\verb2http://www.ma.man.ac.uk/~higham/testmat.html2)
of MATLAB.
The standard workhorse method for computing pseudospectra is the following:
(i) Discretize the region of interest in the complex plane and (ii) compute
the minimum singular value of the matrix $zI-A$ at every gridpoint $z$.
This
algorithm is simple, robust, embarassingly parallel and extremely expensive
even for medium sized matrices: Much more expensive than having to compute
matrix spectral information, such as eigenvalues and singular values.
In other words, pseudospectra are powerful but we must find ways
for obtaining them at lesser cost. ``Domain oriented'' methods for computing
pseudospectra attempt to reduce the number of gridpoints while ``matrix
oriented'' methods attempt to obtain the singular values faster.
Toh and Trefethen (SIAM J. Sci. Comput. 17(1)1-15, 1996)
have studied the following matrix oriented
approach: Apply Arnoldi to compute from $A$ a Hessenberg matrix of
smaller dimension and approximate the pseudospectrum from the
singular values of $H$ or the corresponding augmented Hessenberg
matrix. This has the potential to offer great computational
savings as it only requires SVD calculations for a Hessenberg
matrix of smaller size, but as shown in the aforementioned article, there
are cases where the method does not perform well.
In recent work
(\verb3http://etna.mcs.kent.edu/vol.7.1998/pp190-201.dir/pp190-201.html3)
we have shown, based on the equivalent definition
of the matrix pseudospectrum  that is based on resolvents
$ \Lambda_{\epsilon} (A) = \{ z: z \in C^{n\times n}:
\| \left(zI-A\right)^{-1} \| \geq \epsilon^{-1} \} $,
that  one promising improvement
to the Arnoldi approaches is to utilize a transfer
function formulation that provides a better approximation
to the pseudospectrum.
In this talk we will present
computational issues related to this method and investigate its
effectiveness compared  to existing approaches for computing
pseudospectra.




\end{document}
