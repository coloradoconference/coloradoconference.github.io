\documentclass{article}
\usepackage{amsfonts}
\setlength{\parskip}{1.2ex}      % space between paragraphs
\setlength{\parindent}{0em}      % amount of indention
\setlength{\textwidth}{165mm}    % default = 6.5in
\setlength{\oddsidemargin}{0mm}  % default = 0mm
\setlength{\textheight}{225mm}   % default = 9in
\setlength{\topmargin}{-1mm}     % default = 0mm
\date{ ~ \hspace{-4mm}}


\title{Element-free AMGe: General algorithms for computing interpolation weights  }

\author{Van Emden Henson, speaker ({\tt vhenson@llnl.gov}) \\ Panayot S. Vassilevski \\ \\ Center for Applied Scientific Computing \\ Lawrence Livermore National Laboratory \\ Box 808, L-560, Livermore CA 94551}

\begin{document}
\maketitle
\thispagestyle{empty}





We propose a new algorithm for constructing the interpolation weights
in algebraic multigrid (AMG). The rule we propose is related to the interpolation
described in [1] for AMGe, an element-based algebraic multigrid. There,
the interpolation is based on an energy-minimization principle for finite
element applications.



We first review the classical algorithm for constructing interpolation
weights in AMG, examining it from an energy-minimization perspective. Our
purpose is to extend the classical algorithm in a more general setting
than is possible for the traditional
{\em 
M-matrix
}
applications of classical
AMG.



Based on this discussion, we next present the new interpolation rule.
However, like the method outlined in [1], applying the rule requires access
to the entries of the individual element matrices. Further, these element
matrices must be built on all coarse levels, which is a nontrivial and
expensive task [2].



We then propose an approach that does not require access to the element
matrices even on the fine-grid. However, the additional information we
require instead is knowledge of the so-called
{\em 
rigid body modes,
}
that is, the vectors that span the null-space of the global (assembled)
{\em 
Neumann
}
-type
matrix. In the simplest case of scalar elliptic PDE discretization matrices,
this is just the constant vector. For the elasticity problem, this comprises
certain linear functions, which, in practice, means that one needs the
coordinates of the fine-grid nodes. In some cases this information may
not be available; we use only constant vectors for these cases.



Based on the rigid body modes, we specify appropriate boundary conditions
which are imposed on a
{\em 
local neighborhood matrix
}
associated with
a fine degree of freedom. This produces a sort of Neumann-type local (or
element) matrix. Finally, we can simply apply the rule from the AMGe methods
based on the Neumann local matrix.



Some numerical results are given, illustrating the application of the
method on discretized elliptic problems.


{\bf 
References
}

[1] M. Brezina, A. J. Cleary, R. D. Falgout, V. E. Henson, J. E. Jones, T.
A. Manteuffel, S. F. McCormick, and J. W. Ruge,
{\em 
Algebraic multigrid
based on element interpolation (AMGe),
}
SIAM J. Sci. Comput. (submitted),
1998. \newline
[2] J. E. Jones and P. S. Vassilevski,
{\em 
AMGe based on element agglomeration,
}
(1999) submitted.

This work was performed under the auspices of the U.S. Department of Energy
by University of California Lawrence Livermore National Laboratory under
contract No. W-7405-Eng-48.





\end{document}
