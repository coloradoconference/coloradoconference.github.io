\documentclass[11pt]{article}
\usepackage{amsfonts}
\setlength{\parskip}{1.2ex}      % space between paragraphs
\setlength{\parindent}{0em}      % amount of indention
\setlength{\textwidth}{165mm}    % default = 6.5in
\setlength{\oddsidemargin}{0mm}  % default = 0mm
\setlength{\textheight}{225mm}   % default = 9in
\setlength{\topmargin}{-1mm}     % default = 0mm
\date{ ~ \hspace{-4mm}}


\title{Accurate Numeric Methods for Interface Problems with  Strongly Discontinuous Coefficients  }

\author{Daoqi Yang \\ {\tt yang@math.wayne.edu} \\ Department of Mathematics \\ Wayne State University \\ Detroit, MI 48202, USA \\ {\tt http://www.math.wayne.edu/\~{}yang}}

\begin{document}
\maketitle
\thispagestyle{empty}






Standard finite difference, finite element,
finite volume, or spectral methods are designed for problems with small and
moderate coefficient discontinuities. Their accuracy deteriorates very
rapidly when coefficient jumps are increased and can be arbitrarily
bad for very large jumps.
In this paper, 
an iterative numeric method is proposed and studied 
for solving interface problems modeled by second order
partial differential equations with strongly discontinuous coefficients.
The accuracy of this method does not deteriorate as the jump in the
coefficient discontinuity goes to infinity.
Numeric experiments in the object-oriented paradigm using C++
will be presented.





\end{document}
