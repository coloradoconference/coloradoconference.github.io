\documentclass[11pt]{article}
\usepackage{amsfonts}
\setlength{\parskip}{1.2ex}      % space between paragraphs
\setlength{\parindent}{0em}      % amount of indention
\setlength{\textwidth}{165mm}    % default = 6.5in
\setlength{\oddsidemargin}{0mm}  % default = 0mm
\setlength{\textheight}{225mm}   % default = 9in
\setlength{\topmargin}{-1mm}     % default = 0mm
\date{ ~ \hspace{-4mm}}


\title{First-order System Least Squares (FOSLS) for Elliptic Grid Generation (EGG)   }

\author{Andrea Codd \\ {\tt andrea.codd@colorado.edu} \\ Department of Applied Mathematics  \\  University of Colorado at Boulder  \\  Boulder Colorado 80309-0526  \\}

\begin{document}
\maketitle
\thispagestyle{empty}




 
 


 
Elliptic Grid Generation (EGG), using the Winslow generator, defines a map
between a simple computational region and a potentially complicated
physical region.  It can be used numerically to create a mesh for use with
a finite element or other discretization method to solve the system of
equations posed on the physical domain.  Alternatively, it can be used to
transform equations posed on the physical region to a logical region,
where the transformed equations are then solved.  EGG allows complete
specification of the boundary maps.  Moreover, by choosing the
computational region to be convex, we can ensure that the Jacobian of the
map is positive, and that the map is one-to-one and onto.



Here we explore a new numerical method for solving the Winslow EGG
equations.  The approach is based on first-order system least squares
(FOSLS) for formulating the problem, Newton's method for linearization,
and algebraic multigrid for the matrix solver.  The basic purpose is to
provide a fully variational principle for EGG that facilitates accurate
discretization and fast solution methods.  We develop theoretical
estimates and demonstrate the potential of the method with numerical
experiments.    

 
 


\end{document}
