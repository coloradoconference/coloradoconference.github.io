\documentclass[11pt]{article}
\usepackage{amsfonts}
\setlength{\parskip}{1.2ex}      % space between paragraphs
\setlength{\parindent}{0em}      % amount of indention
\setlength{\textwidth}{165mm}    % default = 6.5in
\setlength{\oddsidemargin}{0mm}  % default = 0mm
\setlength{\textheight}{225mm}   % default = 9in
\setlength{\topmargin}{-1mm}     % default = 0mm
\date{ ~ \hspace{-4mm}}


\title{Iterative Determination of Kinetic Constants from PET Data Utilizing Total Least Squares Estimates  }

\author{Rosemary Renaut ({\tt  renaut@asu.edu}) \\ Kewei Chen \\ Cristina Negoita \\ \\ Department of Mathematics, Arizona State University \\ PO Box 87 1804, Tempe, AZ 85287-1804}

\begin{document}
\maketitle
\thispagestyle{empty}





 



Time series data acquired at the pixel level from positron
emission tomography images can be utilized to provide
visual images of parameters defining the physiological and
biochemical function of differing structures/tissues within
the brain. In studies of Alzheimer's disease it is of
interest to quantify the data such that statistical studies
identifying significant change can be performed.It is thus
important that quantification can be made both reliable and
computationally efficient.

The standard approach for the
solution of this problem assumes a single input/single
output model, in which parameter estimates are made
independently for each pixel. Even at this level, there are
still several questions concerning accuracy of the
estimation that need to be addressed, and in particular how
to address the contamination of the results due to the
various measurement errors.

The first approach reported in
the medical literature uses nonlinear least squares data
fitting for the parameters defining the model. This method
does not account at all for the error and has been found to
be inefficient computationally. More recent works discuss
the linear least squares problem which results from the
Laplace transform of the underlying differential equation.
Although more efficient, this approach also does not
account for the measurement error.

To overcome this problem
a generalized least squares method can be used. The method
is further refined by the incorporation of a penalty term
which enforces the parameters to physiologically realistic
values. This latter approach, while requiring the
determination of an appropriate penalty parameter, can be
incorporated in both the linear and nonlinear models. The
need to estimate the penalty can be avoided by the
utilization of a total least squares formualation.

In this
paper we evaluate the use of the total least squares method
in this context as compared to generalized least squares
with penalty, and investigate the formulation of the
problem as single-input multiple output. The latter model
is designed to take advantage of the locality of the data
in the plane, hence treating the image as a whole, rather
than on the pixel by pixel level. A multilevel iterative
algorithm for the solution of this problem is proposed. Its
effectiveness for the appropriate account of measurement
error and its computational efficiency will also be
discussed.





\end{document}
