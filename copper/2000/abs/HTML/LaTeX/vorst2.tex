\documentclass[11pt]{article}
\usepackage{amsfonts}
\setlength{\parskip}{1.2ex}      % space between paragraphs
\setlength{\parindent}{0em}      % amount of indention
\setlength{\textwidth}{165mm}    % default = 6.5in
\setlength{\oddsidemargin}{0mm}  % default = 0mm
\setlength{\textheight}{225mm}   % default = 9in
\setlength{\topmargin}{-1mm}     % default = 0mm
\date{ ~ \hspace{-4mm}}


\title{Ritz and Harmonic Ritz Approximations  }

\author{Henk van der Vorst \\ {\tt vorst@math.uu.nl} \\ Mathematical Institute, Utrecht University  \\  P.O. Box 80 010, 3508 TA UTRECHT  \\  The Netherlands  \\  {\tt http://www.math.uu.nl/people/vorst/}}

\begin{document}
\maketitle
\thispagestyle{empty}





 



It is well-known that one can extract Ritz approximations for eigenpairs
from Krylov subspace information. In the 1990s much attention has been
given to the so-called harmonic Ritz approximations. The harmonic Ritz
pairs can be interpreted as to correspond with the inverse of A, restricted
to A times the Krylov subspace. These new approximations have been
suggested in the context of the methods of Lanczos, Arnoldi, and
Jacobi-Davidson. From the two subspaces generated with the two-sided Lanczos 
algorithm we can also extract approximations for the eigenpairs associated
with the inverse of A: the harmonic Petrov approximations. In this case,
we obtain approximations for left and right eigenvectors of A.




We will give an overview of the various harmonic approximations for
eigenpairs and we we will discuss how they can be computed. It turns
out that the harmonic approximations may have an advantage for the
computation of eigenpairs corresponding to interior eigenvalues.    

 




\end{document}
