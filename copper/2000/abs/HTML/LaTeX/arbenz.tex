\documentclass[11pt]{article}
\usepackage{amsfonts}
\setlength{\parskip}{1.2ex}      % space between paragraphs
\setlength{\parindent}{0em}      % amount of indention
\setlength{\textwidth}{165mm}    % default = 6.5in
\setlength{\oddsidemargin}{0mm}  % default = 0mm
\setlength{\textheight}{225mm}   % default = 9in
\setlength{\topmargin}{-1mm}     % default = 0mm
\date{ ~ \hspace{-4mm}}


\title{On Computing Some of the Principal Eigenfrequencies of Cavity Resonators  }

\author{Peter Arbenz \\ {\tt arbenz@inf.ethz.ch} \\    Swiss Federal Institute of Technology (ETH)  \\  Institute of Scientific Computing  \\  CH-8092 Zurich, Switzerland}

\begin{document}
\maketitle
\thispagestyle{empty}





The computation of steady state electromagnetic waves in resonant cavities
is governed by the Maxwell equations. Their finite element (FEM) discretization
with Ned\'{e}l\'{e}c's edge elements leads to constrained symmetric
matrix eigenvalue problems of the form
$$ A \mbox{\boldmath $x$} = \lambda M \mbox{\boldmath $x$}
~~~~~~~~
C'\mbox{\boldmath $x$}=\mbox{\boldmath $0$} $$




Here, M is positive definite, and A, the discretization of the double-curl,
is positive semidefinite. The constraint corresponds to the divergence-free
condition in the continuous problem. We present and compare various approaches
to compute a few of the smallest eigenvalues of this problem. The approaches
differ in how the constraints are dealt with. All the approaches employ
the shift-and-invert spectral transformation which implies that large sparse
indefinite systems of equations have to be solved in the course of the
solution of the eigenvalue problem. We use the preconditioned conjugate
gradient method to solve the systems of equations. Two level preconditioners
evolve naturally from the hierarchical formulation of the FEM. The theoretical
behavior of the algorithms is verified by the numerical experiments.




\end{document}
