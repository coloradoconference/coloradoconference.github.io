\documentclass[11pt]{article}
\usepackage{amsfonts}
\setlength{\parskip}{1.2ex}      % space between paragraphs
\setlength{\parindent}{0em}      % amount of indention
\setlength{\textwidth}{165mm}    % default = 6.5in
\setlength{\oddsidemargin}{0mm}  % default = 0mm
\setlength{\textheight}{225mm}   % default = 9in
\setlength{\topmargin}{-1mm}     % default = 0mm
\date{ ~ \hspace{-4mm}}


\title{Mathematical Optimization Methods         for the Remediation of         Groundwater Contaminations  }

\author{A. Battermann \\ {\tt batt@uni-trier.de} \\ FB IV, Abt. Mathematik, University of Trier, 54286 Trier, Germany}

\begin{document}
\maketitle
\thispagestyle{empty}





 



An optimal control problem governed by 
partial differential equations is 
formulated for a practical application. 
The partial differential equations model 
groundwater flow and heat transfer in an aquifer. 
The black box setting and a large error margin in the available 
data lead to inaccurate function evaluations, 
these causing failure of 
standard optimization routines. 
Numerical results will be presented for the 
Implicit Filtering and Nelder--Mead algorithms.






\end{document}
