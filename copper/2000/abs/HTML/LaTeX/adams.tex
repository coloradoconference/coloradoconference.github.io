\documentclass[11pt]{article}
\usepackage{amsfonts}
\setlength{\parskip}{1.2ex}      % space between paragraphs
\setlength{\parindent}{0em}      % amount of indention
\setlength{\textwidth}{165mm}    % default = 6.5in
\setlength{\oddsidemargin}{0mm}  % default = 0mm
\setlength{\textheight}{225mm}   % default = 9in
\setlength{\topmargin}{-1mm}     % default = 0mm
\date{ ~ \hspace{-4mm}}


\title{Evaluation of a Geometric and an Algebraic Multigrid Method on 3D Finite Element Problems in Solid Mechanics  }

\author{Mark Adams \\ {\tt madams@cs.berkeley.edu} \\ 531 Soda Hall University of California, Berkeley Berkeley CA 94720}

\begin{document}
\maketitle
\thispagestyle{empty}





 



The availability of large high performance computers is providing
scientists and engineers with the opportunity to simulate a variety of
complex physical systems with ever more accuracy and thereby exploit the
advantages of computer simulations over laboratory experiments.  The
finite element method is widely used for these simulations.   The finite
element method requires that one or several linearized system of sparse
unstructured algebraic equations (the stiffness matrix) be solved at
each time step when implicit time integration is used.  The linear
system solves become the computational bottleneck (once the simulation
has been setup and before the results are interpreted) as the scale of
problems increases.  Direct solution methods have been (and still are)
popular as they are dependable, though the asymptotic complexity of
direct methods, or any fixed level method, is high in comparison to
optimal iterative methods (ie, multigrid).  Multigrid has been a popular
method for solving finite element (and finite difference) systems on
regular grids for over 30 years; the application of multigrid to
unstructured problems is, however, not well understood and has been an
active area of research in recent years.  We are aware of two categories
of promising unstructured multigrid methods: 1) ``geometric'' methods
that use standard finite element coarse grid function spaces (and hence
have a concrete geometric interpretation), and 2) rigid body mode coarse
grid space (``algebraic'') methods which we call algebraic as the coarse
grid function spaces can in principle be deduced from the stiffness
matrix or its component element stiffness matrices and the method that
we use uses the stiffness matrix to modify the initial rigid body mode
coarse grid spaces.  This paper evaluates the effectiveness of two
promising unstructured multigrid methods (one geometric and one
algebraic) on several challenging unstructured problems in solid
mechanics with up to 56 million degrees of freedom.






\end{document}
