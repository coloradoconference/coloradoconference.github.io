\documentclass[11pt]{article}
\usepackage{amsfonts}
\setlength{\parskip}{1.2ex}      % space between paragraphs
\setlength{\parindent}{0em}      % amount of indention
\setlength{\textwidth}{165mm}    % default = 6.5in
\setlength{\oddsidemargin}{0mm}  % default = 0mm
\setlength{\textheight}{225mm}   % default = 9in
\setlength{\topmargin}{-1mm}     % default = 0mm
\date{ ~ \hspace{-4mm}}


\title{The Weighted Distributed Modification ILU Factorisation  }

\author{Victor Eijkhout \\ {\tt eijkhout@cs.utk.edu} \\ Department of Computer Science, 107 Ayres Hall \\ University of Tennessee, Knoxville TN 37996}

\begin{document}
\maketitle
\thispagestyle{empty}





 



The quest for the ultimate incomplete factorisation is a struggle
between the demand for accuracy, and the demand for the well-definedness
of the factorisation. It is a well known phenomenon that ordinary
ILU can break down, even on SPD matrices. Various strategies to guarantee
a well-defined factorisation exist, but they often reduce the accuracy
of the factorisation. I will present a new factorisation that has
strong guarantees for its existence, and exhibits a reasonable
robustness in practical tests. The main idea behind this factorisation
is that fill-in in $(i,j)$ is scaled, and moved to both the
$(i,j)$ and $(j,j)$
diagonal elements.





\end{document}
