\documentclass[11pt]{article}
\usepackage{amsfonts}
\setlength{\parskip}{1.2ex}      % space between paragraphs
\setlength{\parindent}{0em}      % amount of indention
\setlength{\textwidth}{165mm}    % default = 6.5in
\setlength{\oddsidemargin}{0mm}  % default = 0mm
\setlength{\textheight}{225mm}   % default = 9in
\setlength{\topmargin}{-1mm}     % default = 0mm
\date{ ~ \hspace{-4mm}}


\title{CFD code convergence diagnostics and acceleration using the Recursive Projection Method  }

\author{Mihai Dorobantu ({\tt dorobam@utrc.utc.com}), \\ Kurt Lust, Alexander Khibnik, and Joakim Moller \\ United Technologies Research Center  \\  411 Silver Lane, MS 129-15  \\  East Hartford, CT - 06108}

\begin{document}
\maketitle
\thispagestyle{empty}






Under the general paradigm of the Recursive Projection Method (RPM) developed by H.
Keller as a stabilization method for stabilizing unstable nonlinear oscillations, we have
created a framework for analyzing and diagnosing convergence problems encountered in flow
simulations. The RPM framework contains three independent modules: a procedure to identify
leading eigenvalues of the iteration operator, a nonlinear Newton-type solver for small
systems, and the interface to the specific CFD code used. In this talk we present a series
of applications that illustrate how a combination of eigenvalue solvers and nonlinear
iterations built around existing black-box CFD codes can be used to (1) improve
convergence rates, (2) compute unstable steady-state solutions, (3) recover from
divergence due to unsuitable initializations, and (4) diagnose convergence stagnation
problems. The applications use both university and production codes applied to CFD
problems arising in jet engine design. 






\end{document}
