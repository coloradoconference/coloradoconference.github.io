\documentclass[11pt]{article}
\usepackage{amsfonts}
\setlength{\parskip}{1.2ex}      % space between paragraphs
\setlength{\parindent}{0em}      % amount of indention
\setlength{\textwidth}{165mm}    % default = 6.5in
\setlength{\oddsidemargin}{0mm}  % default = 0mm
\setlength{\textheight}{225mm}   % default = 9in
\setlength{\topmargin}{-1mm}     % default = 0mm
\date{ ~ \hspace{-4mm}}


\title{A Levenberg-Marquardt Method for Nonlinear Least-Squares Finite Element Computations  }

\author{Gerhard Starke \\ {\tt  starke@ing-math.uni-essen.de} \\ Fachbereich Mathematik \\ University of Essen}

\begin{document}
\maketitle
\thispagestyle{empty}





 



We consider the least-squares mixed finite element formulation of
nonlinear elliptic boundary value problems using a combination of
standard conforming finite elements with Raviart-Thomas spaces for
the flux. For the solution of the
arising nonlinear algebraic least-squares problems suitable variants of the
Levenberg-Marquardt method are proposed. In particular, for elliptic
problems arising from an implicit time discretization
of a variably saturated subsurface flow model, we study an appropriate
norm for the definition of the trust-region. 
We study an inexact version of
this approach where the linear least-squares problems in each step are
approximately solved by an adaptive multilevel method.
For realistic variably saturated subsurface flow problems, the results
of computational experiments are presented.




\end{document}
