\documentclass[11pt]{article}
\usepackage{amsfonts}
\setlength{\parskip}{1.2ex}      % space between paragraphs
\setlength{\parindent}{0em}      % amount of indention
\setlength{\textwidth}{165mm}    % default = 6.5in
\setlength{\oddsidemargin}{0mm}  % default = 0mm
\setlength{\textheight}{225mm}   % default = 9in
\setlength{\topmargin}{-1mm}     % default = 0mm
\date{ ~ \hspace{-4mm}}


\title{An Indefinite Preconditioner for Quadratic Control Problems  }

\author{E.W. Sachs \\ {\tt  sachs@uni-trier.de} \\ FB IV, Abt.\ Mathematik \\ University of Trier \\ 54286 Trier \\ Germany}

\begin{document}
\maketitle
\thispagestyle{empty}





 



The numerical solution of linear quadratic control problems 
governed by partial differential equations 
gives rise to linear systems of large dimension.
They have to be solved repeatedly in the course of optimization.
Crucial for their iterative solution 
is the use of efficient preconditioners
that exploit the structure of the problem. 
An indefinite preconditioner will be analysed in detail.







\end{document}
