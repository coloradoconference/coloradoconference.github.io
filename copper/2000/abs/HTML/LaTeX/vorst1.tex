\documentclass[11pt]{article}
\usepackage{amsfonts}
\setlength{\parskip}{1.2ex}      % space between paragraphs
\setlength{\parindent}{0em}      % amount of indention
\setlength{\textwidth}{165mm}    % default = 6.5in
\setlength{\oddsidemargin}{0mm}  % default = 0mm
\setlength{\textheight}{225mm}   % default = 9in
\setlength{\topmargin}{-1mm}     % default = 0mm
\date{ ~ \hspace{-4mm}}


\title{A Domain Decomposition Approach For Eigenvalue Problems  }

\author{Henk van der Vorst \\ {\tt vorst@math.uu.nl} \\ Mathematical Institute, Utrecht University  \\  P.O. Box 80 010, 3508 TA UTRECHT  \\  The Netherlands  \\  {\tt http://www.math.uu.nl/people/vorst/}}

\begin{document}
\maketitle
\thispagestyle{empty}







 



    The Jacobi-Davidson method leads to a so-called
    correction equation that has to be solved.
    This can be done with an appropriate iterative solver, but
    to make this really efficient, one needs an effective
    preconditioner. Unfortunately, the operator associated with
    the correction equation is usually indefinite and there
    are no really successful preconditioners for general indefinite
    matrices. The popular ILU preconditioners require much
    tuning in order to be occasionally effective, but it is
    worthwhile to invest some effort, because the preconditioner
    can be used for many correction equations, even for a number
    of successive eigenvalues.




    For eigenproblems stemming from PDEs, we try to adapt domain
    decomposition approaches. Following ideas from W.-P. Tang, who
    has proposed to work with an expanded matrix, Tan has suggested
    interface conditions for virtually overlapping  domains, so that
    the matrix can be split into parallel parts.
    The interface conditions can be tuned to the given problem, which
    makes this splitting very effective as a preconditioners for
    methods as GMRES, for classes of linear systems
    For eigenproblems, it turns out that the interface conditions
    have to be selected differently. 

    We will report on our efforts to construct effective preconditioners,
    and show some instructive examples.




This is joint work with 
Gerard Sleijpen
(\verb9http://www.math.uu.nl/people/sleijpen/9)
and Menno Genseberger.






\end{document}
