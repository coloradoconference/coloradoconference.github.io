\documentclass{article}
\usepackage{amsfonts}
\setlength{\parskip}{1.2ex}      % space between paragraphs
\setlength{\parindent}{0em}      % amount of indention
\setlength{\textwidth}{165mm}    % default = 6.5in
\setlength{\oddsidemargin}{0mm}  % default = 0mm
\setlength{\textheight}{225mm}   % default = 9in
\setlength{\topmargin}{-1mm}     % default = 0mm
\date{ ~ \hspace{-4mm}}


\title{On the solution of the linear systems which evolve from  large scale inverse problems  }

\author{Eldad Haber \\ {\tt haber@cs.ubc.ca} \\ Dept of Computer Science \\ University of British Columbia \\ Vancouver, Canada}

\begin{document}
\maketitle
\thispagestyle{empty}





 
In this talk we consider an inverse problem where the forward
problem is governed by a (discrete) Partial Differential Equation of the
form


 
(1)    $A(m) u = f$



where $A(m)$ is a matrix which depends on the ``model'' $m$, $u$ is some
physical field and $f$ is the source and boundary conditions. 
Such problems include the DC resistivity (or impedance tomography),
inverse Maxwell's equations, hydrology and many other problems.
The goal of the inverse problem is to recover the model $m$
based on the measurements of the field $u$.



In the inverse problem we consider recovering $m$ from
measurements of $u$ in a discrete number
of locations. 
The problem is usually solved by minimizing an objective
function of the form


 
(2)    $\min \phi =  \|A(m)^{-1}f - b \|^2 + \beta \| Wm\|^2$



where $b$ is the measured data, $\beta$ is a penalty parameter
and $W$ has the a priori information.




The optimization problem leads to
the solution of the Gauss-Newton (or quasi-Newton)
step at each iteration for the perturbation dm.



(3)     $Ldm =  (J(m)'J(m) + \beta W'W) dm = -r(m)$



where the matrix $J$ is a dense sensitivity matrix



(4)     $J(m) =  db(m)/dm$



and $r(m)$ is the gradient of the objective function in (2).
 



The solution of this system is complicated because it involves
large dense matrices. Furthermore, preconditioning the iteration
is not simple due to the different nature of the sensitivity matrix
$J$ (which can be viewed as an integral operator)
and the weighting matrix $W$
(which is usually a difference matrix).



In this talk we present a new approach for the solution of such systems.
First, we show that this system is equivalent to the solution
of a sparse KKT matrix. Second, we use the KKT matrix in order to
construct a sparse approximation approximate to $J$
based on  the reduced Hessian method applied to the sparse KKT system.
Thirdly, we show that using the sparse approximation
we can use standard preconditioners in order to solve
the approximate system.


 
We demonstrate the efficiency of the method on two model problems
from electromagnetics.





\end{document}
