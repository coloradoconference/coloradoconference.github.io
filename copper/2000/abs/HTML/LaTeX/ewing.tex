\documentclass[11pt]{article}
\usepackage{amsfonts}
\setlength{\parskip}{1.2ex}      % space between paragraphs
\setlength{\parindent}{0em}      % amount of indention
\setlength{\textwidth}{165mm}    % default = 6.5in
\setlength{\oddsidemargin}{0mm}  % default = 0mm
\setlength{\textheight}{225mm}   % default = 9in
\setlength{\topmargin}{-1mm}     % default = 0mm
\date{ ~ \hspace{-4mm}}


\title{Iterative Methods in Simulation of Applications of Fluid Flow in Porous Media  }

\author{Richard Ewing  \\  Institute for Scientific Computation  \\  Texas A\&M University}

\begin{document}
\maketitle
\thispagestyle{empty}


 {\bf 



Abstract



}



Understanding the fate and transport of contaminants to determine water quality and to develop remediation strategies or optimizing the recovery of hydrocarbons in petroleum applications each require the ability to model multiphase flow in heterogeneous three-dimensional reservoirs. Effective simulators require accurate numerical methods on general geometries. Use of mixed finite element methods and local grid refinement will be discussed. Iterative methods and domain decomposition algorithms for these problems will be presented. Example calculations for field simulations in aquifers or reservoirs with complex boundaries will be presented. Parallelization of the codes will also be discussed.




\end{document}
