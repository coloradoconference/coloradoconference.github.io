\documentclass[11pt]{article}
\usepackage{amsfonts}
\setlength{\parskip}{1.2ex}      % space between paragraphs
\setlength{\parindent}{0em}      % amount of indention
\setlength{\textwidth}{165mm}    % default = 6.5in
\setlength{\oddsidemargin}{0mm}  % default = 0mm
\setlength{\textheight}{225mm}   % default = 9in
\setlength{\topmargin}{-1mm}     % default = 0mm
\date{ ~ \hspace{-4mm}}


\title{Finite Volume Methods for Systems of Viscous Conservation Laws: Numerical Analysis and Software  }

\author{J\"{u}rgen Fuhrmann \\ {\tt fuhrmann@wias-berlin.de} \\ Weierstrass Institute for Applied Analysis and Stochastics \\ Mohrenstr. 39, ~  10117 Berlin.}

\begin{document}
\maketitle
\thispagestyle{empty}





 




Using the flux function notation developed for the description of time
explicite   finite  volume schemes, we    present   an exsistence  and
stability analysis for time implicite finite volume schemes for scalar
nonlinear viscous  conservation  laws.   The  generalization  of  this
notation  to  systems leads  to  an  easy  to   understand application
programming interface and an implementation of a solver for this class
of problems.  It uses various variants of Newton's method as nonlinear
solvers and --- among other possibilities --- agglomeration type algebraic
multigrid with point block ILU smoothers as preconditioners for Krylov
subspace methods for the linear systems. 


The corresponding code - \verb2sysconlaw2 - is implemented on
top   of the
{\bf 
pdelib
}
toolbox \linebreak
(\verb2http://www.wias-berlin.de/~pdelib2)
for the numerical  solution of partial  differential equations
developed    at the
Weierstrass Institute Berlin
(\verb2http://www.wias-berlin.de2).


Numerical  examples    like   two   phase  flow  in     porous  media,
thermoconvective  flow in porous   media, or the Brusselator equations
demonstrate the flexibility and usefulness of this approach. 











\end{document}
