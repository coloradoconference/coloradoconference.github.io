\documentclass[11pt]{article}
\usepackage{amsfonts}
\setlength{\parskip}{1.2ex}      % space between paragraphs
\setlength{\parindent}{0em}      % amount of indention
\setlength{\textwidth}{165mm}    % default = 6.5in
\setlength{\oddsidemargin}{0mm}  % default = 0mm
\setlength{\textheight}{225mm}   % default = 9in
\setlength{\topmargin}{-1mm}     % default = 0mm
\date{ ~ \hspace{-4mm}}


\title{A penalized finite element method for pure Neumann problem  }

\author{Rich Lehoucq \\ {\tt rblehou@sandia.gov} \\ Sandia National Laboratories, P.O.Box 5800, MS 87185-1110  \\ \\  Pavel Bochev \\ {\tt bochev@uta.edu} \\ Department of Mathematics, Box 19408 \\ University of Texas at Arlington, Arlington, TX 76019-0408}

\begin{document}
\maketitle
\thispagestyle{empty}





 



A conforming finite element discretization of the Poisson problem with
a pure Neumann boundary condition produces an algebraic problem with a
singular stiffness matrix. Typically, the singularity is removed by
specifying the solution at some point or by perturbing the Neumann
condition to a Robin type condition.

Here we propose and analyze a simple and efficient
least-squares penalization method that ensures nonsingularity of the
alegbraic problem and at the same time appears to minimize the
condition number of the matrix. Numerical results illustrating our
method are also included.





\end{document}
