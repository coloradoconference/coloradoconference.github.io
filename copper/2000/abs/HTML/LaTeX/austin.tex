\documentclass[11pt]{article}
\usepackage{amsfonts}
\setlength{\parskip}{1.2ex}      % space between paragraphs
\setlength{\parindent}{0em}      % amount of indention
\setlength{\textwidth}{165mm}    % default = 6.5in
\setlength{\oddsidemargin}{0mm}  % default = 0mm
\setlength{\textheight}{225mm}   % default = 9in
\setlength{\topmargin}{-1mm}     % default = 0mm
\date{ ~ \hspace{-4mm}}


\title{A Divergence-Free Relaxation Scheme in an $H^1$-Finite Element Space  }

\author{Travis Austin \\ {\tt  travis.austin@colorado.edu} \\ Department of Applied Mathematics       Campus Box 526, CU Boulder, CO 80309-0526}

\begin{document}
\maketitle
\thispagestyle{empty}





 



An effective multilevel solver for (I - grad div) must account for the
presence of divergence-free error components.  As a result, the Raviart-Thomas
(RT) finite element spaces, which have locally computable divergence-free 
subspaces, are often used in the discretization of (I - grad div).  
The presence of an epsilon-sized Laplacian term, leading to
(I - $\epsilon$~Laplacian - grad div),
results in poor approximation properties for the
discontinuous RT finite element spaces.  We present a new, 
continuous, RT-like finite element space for the discretization of 
(I - $\epsilon$~Laplacian - grad div)
and a corresponding multilevel solver which 
includes divergence-free relaxation.






\end{document}
