\documentclass[11pt]{article}
\usepackage{amsfonts}
\setlength{\parskip}{1.2ex}      % space between paragraphs
\setlength{\parindent}{0em}      % amount of indention
\setlength{\textwidth}{165mm}    % default = 6.5in
\setlength{\oddsidemargin}{0mm}  % default = 0mm
\setlength{\textheight}{225mm}   % default = 9in
\setlength{\topmargin}{-1mm}     % default = 0mm
\date{ ~ \hspace{-4mm}}


\title{Newton's Structured Matrix Iteration  }

\author{Victor Y.Pan \\ Department of Mathematics \& Computer Science \\ Lehman College, CUNY \\ Bronx, NY 10468, USA \\ vpan@alpha.lehman.cuny.edu}

\begin{document}
\maketitle
\thispagestyle{empty}


 



Newton's Iteration rapidly improves an initial approximation
to matrix inverse by performing two matrix multiplications per
recursive step. The iteration is particularly suitable for the
inversion of structured matrices because in this case matrix
operations are performed with $O(n)$ entries of short generators
of structured matrices rather than with order of $n^2$ entries. A
major problem is to control the length of the generator, which
tends to grow in the iterative process. Some techniques proposed
to control the growth in the Toeplitz-like case [Pan92], [Pan93],
[Pan93a] relied on the concept of approximate (orthogonal)
displacement rank, formally coined only in 1999. We study the
iteration for the more general class of structured matrices,
proposed some variations of the iterative process and techniques
for controlling the length of the generators and investigate the
convergence rate as well as computational complexity.







\end{document}
