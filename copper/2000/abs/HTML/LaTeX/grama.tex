\documentclass[11pt]{article}
\usepackage{amsfonts}
\setlength{\parskip}{1.2ex}      % space between paragraphs
\setlength{\parindent}{0em}      % amount of indention
\setlength{\textwidth}{165mm}    % default = 6.5in
\setlength{\oddsidemargin}{0mm}  % default = 0mm
\setlength{\textheight}{225mm}   % default = 9in
\setlength{\topmargin}{-1mm}     % default = 0mm
\date{ ~ \hspace{-4mm}}


\title{Error Controlled Multipole Methods for Arbitrary Particle Distributions  }

\author{Ananth Grama \\ {\tt http://www.cs.purdue.edu/people/ayg} \\ Department of Computer Sciences, Purdue University  \\  W. Lafayette, IN 47907 \\ ayg@cs.purdue.edu}

\begin{document}
\maketitle
\thispagestyle{empty}






Hierarchical multipole methods such as the Fast Multipole Method (FMM)
and Barnes-Hut (BH) method find extensive application in particle
dynamics problems and boundary element solvers. The computational complexity
of the dense matrix-vector product in these solvers is effectively reduced
by hierarchical multipole approximations. It has been shown that the
computational complexity of FMM can be bounded by $O(n \log n)$
for an arbitrarily
distributed set of n particles using box-collapsing and fair-split techniques.
The overall error introduced can be shown to grow linearly in the magnitude
of total charge in the system. This error behavior can be reduced to
logarithmic in total charge magnitude by using variable degree multipoles
for uniform distributions. In this paper, we demonstrate the use of alternate
multipole acceptance criteria to reduce the error in a single matrix-vector
product for arbitrarily unstructured domains. These results are particularly
important for boundary element solvers that typically mesh only surfaces of
3-D objects, resulting in highly unstructured domain models.
Experimental results are provided on a variety of distributions to
illustrate the error and computational characteristics of our method.





\end{document}
