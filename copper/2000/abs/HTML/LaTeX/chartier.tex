\documentclass[11pt]{article}
\usepackage{amsfonts}
\setlength{\parskip}{1.2ex}      % space between paragraphs
\setlength{\parindent}{0em}      % amount of indention
\setlength{\textwidth}{165mm}    % default = 6.5in
\setlength{\oddsidemargin}{0mm}  % default = 0mm
\setlength{\textheight}{225mm}   % default = 9in
\setlength{\topmargin}{-1mm}     % default = 0mm
\date{ ~ \hspace{-4mm}}


\title{Element Based Algebraic Multigrid (AMGe)  }

\author{Tim Chartier \\ {\tt chartier@colorado.edu} \\ Department of Applied Mathematics  \\  Box 526  \\  University of Colorado at Boulder  \\  Boulder   CO   80309-0526}

\begin{document}
\maketitle
\thispagestyle{empty}





 



Generic iterative methods can be robust in the sense that they converge
for a large class of matrix equations, but they typically do not obtain
full efficiency without taking the origin of the problem into account.
Multigrid and other methods that do take the problem origin into account
can obtain full efficiency, but they are usually tailored to specific
aspects of the target application. Algebraic multigrid (AMG) is a method
that attempts to take the middle ground by providing a matrix equation
solver that is removed from problem details, but that attains efficiency
comparable to conventional multigrid. In essence, AMG abstracts multigrid
principles to the matrix level by developing 'coarser' matrices and
interlevel transfer operators based solely on the original matrix
entries.



AMGe is a special version of AMG designed to exploit the special nature
of matrices that arise in finite element discretization of partial differential
equations. Assuming access to the element stiffness matrices, AMGe uses
a measure derived from multigrid theory to determine local representations
of algebraically ``smooth'' error components. We believe that this measure
and the resulting representations provide the basis for effective design of
coarse grids, coarse grid matrices, and interlevel transfers.  Our aim is to
use this measure for defining multigrid components in order to create a
more robust AMG scheme.  This talk will discuss this measure and recent
results in AMGe on difficult problems that arise in linear elasticity.






\end{document}
