\documentclass[11pt]{article}
\usepackage{amsfonts}
\setlength{\parskip}{1.2ex}      % space between paragraphs
\setlength{\parindent}{0em}      % amount of indention
\setlength{\textwidth}{165mm}    % default = 6.5in
\setlength{\oddsidemargin}{0mm}  % default = 0mm
\setlength{\textheight}{225mm}   % default = 9in
\setlength{\topmargin}{-1mm}     % default = 0mm
\date{ ~ \hspace{-4mm}}


\title{Smoother dependent interpolation for AMG  }

\author{Menno Verbeek, {\tt verbeek@math.uu.nl} \\ Mathematical Institute, Utrecht University \\ P.O.Box 80.010, 3508 TA Utrecht \\ The Netherlands \\ \\ Jane Cullum, {\tt cullumj@lanl.gov} \\ MS B256, CIC-3, Los Alamos National Laboratory \\ Los Alamos, NM 87545, USA}

\begin{document}
\maketitle
\thispagestyle{empty}






In the Ruge-St\"{u}ben variants of Algebraic Multigrid (AMG), the
interpolation matrix $P$ is determined from the linear system
matrix $A$ only. However, for AMG to work efficiently, $P$
should be chosen such, that the coarse grid correction damps the
errors that are not damped by the smoother, the so called
algebraically smooth errors. Therefore, not only the matrix $A$, but also
the choice of smoother determines what is algebraically smooth. This suggests
using not only $A$, but also the smoother, to determine the
interpolation $P$.




Errors in the span of the columns in $P$ are removed exactly by
a (2-level) coarse grid correction. Thus, we would like this span to
contain the algebraically smooth errors. To
achieve this, we assume that algebraically smooth vectors have the
same local behaviour with different, geometrically smooth, amplitudes
or modes. This is not an uncommon situation for partial differential
equations with simple smoothers. We use this local behaviour to
generate a sparse $P$ that represents this local behaviour, and
thus approximately spans the algebraically smooth vectors.




To obtain the local behaviour, we use an approximation to the
largest eigenvector of the smoother error operator.  We correct a
``linear'' interpolation to make it exact for this algebraically smooth
vector. We also explore the possibilities of using multiple smooth
vectors to construct an interpolation matrix $P$.




Results of numerical experiments, including transport
problems from the ASCI program and electromagnetic boundary integral
problems, will be presented.





\end{document}
