\documentclass[11pt]{article}
\usepackage{amsfonts}
\setlength{\parskip}{1.2ex}      % space between paragraphs
\setlength{\parindent}{0em}      % amount of indention
\setlength{\textwidth}{165mm}    % default = 6.5in
\setlength{\oddsidemargin}{0mm}  % default = 0mm
\setlength{\textheight}{225mm}   % default = 9in
\setlength{\topmargin}{-1mm}     % default = 0mm
\date{ ~ \hspace{-4mm}}


\title{First-Order Systems Least Squares for Electrical Impedance Tomography  }

\author{Hugh MacMillan \\ {\tt  hugh.macmillan@colorado.edu} \\ Dept. Applied Mathematics, CB 526 \\ University of Colorado, Boulder, CO 80309-0526}

\begin{document}
\maketitle
\thispagestyle{empty}





 


In electrical impedance tomography (EIT), the impedance within an
object is estimated from the surface voltage patterns that are
induced by a sequence of known current flux patterns. The
mathematical formulation of the EIT problem is a set of elliptic
equations (one for each boundary pattern) where both the scalar
variables (interior voltages) and the coefficient (interior
impedance) are unknown, and both Dirichlet (voltage) and Neumann
(current) boundary conditions are known. We develop a new
first-order systems least squares (FOSLS) formulation to determine
a suitable impedance and investigate the sense in which it is
suitable. For each set of boundary patterns, the full domain
FOSLS formulation is tailored to balance enforcement of the pair
of boundary conditions and the interior PDE .  The nonlinearity is
mild enough (polynomial) to admit a realizable nonlinear multigrid
scheme that yields successive approximations to the impedance and
to the scaled interior currents for each test.




\end{document}
