\documentclass[11pt]{article}
\usepackage{amsfonts}
\setlength{\parskip}{1.2ex}      % space between paragraphs
\setlength{\parindent}{0em}      % amount of indention
\setlength{\textwidth}{165mm}    % default = 6.5in
\setlength{\oddsidemargin}{0mm}  % default = 0mm
\setlength{\textheight}{225mm}   % default = 9in
\setlength{\topmargin}{-1mm}     % default = 0mm
\date{ ~ \hspace{-4mm}}


\title{Algebraic Multigrid for Problems with Discontinuous Coefficients   }

\author{Marian Brezina \\ {\tt brezina@colorado.edu} \\ Dept. of Applied Mathematics \\ University of Colorado at Boulder \\ Boulder, CO 80309-0526}

\begin{document}
\maketitle
\thispagestyle{empty}





 



Algebraic multigrid has long been established as a practical solver 
for a variety of problems, particularly elliptic differential
equations discretized on unstructured grids. 



Classical multigrid theories, however, are often carried out under 
assumptions not satisfied in computational practice. 
We will report on the results of an ongoing joint research done together 
with P. Vanek, C. Heberton and N. Neuss.   The research focuses on 
relaxing some of these assumptions and extending the multigrid theory 
to the cases more accurately reflecting practical environment. 



Our multilevel method is of smoothed-aggregation type. 
Modifications of the aggregation-based coarsening algorithm and of the 
smoothing procedures allow us to prove a convergence result for the 
problems with discontinuous coefficients with the same rate as for 
problems having no coefficient discontinuities.  



The theoretical results will be illustrated by computational experiments. 







\end{document}
