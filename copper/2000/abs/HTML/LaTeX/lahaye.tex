\documentclass{article}
\usepackage{amsfonts}
\setlength{\parskip}{1.2ex}      % space between paragraphs
\setlength{\parindent}{0em}      % amount of indention
\setlength{\textwidth}{165mm}    % default = 6.5in
\setlength{\oddsidemargin}{0mm}  % default = 0mm
\setlength{\textheight}{225mm}   % default = 9in
\setlength{\topmargin}{-1mm}     % default = 0mm
\date{ ~ \hspace{-4mm}}


\title{Solving Magnetic Field - Electrical Circuit Coupled systems   }

\author{Domenico Lahaye \\ {\tt  domenico.lahaye@cs.kuleuven.ac.be} \\ Dept. of Computer Science \\ Celestijnenlaan 200A B-3001 \\ Heverlee, Belgium}

\begin{document}
\maketitle
\thispagestyle{empty}





 




We consider preconditioned Krylov subspace solvers for large, 
sparse, complex symmetric matrices bordered by dense blocks. Such systems 
arise in computational electro magnetics when e.g. the differential problem 
describing the magnetic field is coupled with external electrical circuit 
relations. In low-frequent time-harmonic Maxwell formulations in two 
dimensions, the governing partial differential equation is the Helmholtz 
equation with a complex shift. The finite element discretization of this 
equation leads to sparse, complex symmetric systems. The external electrical 
circuits constitute relations for global quantities such as the voltage drop 
accross or the current through the computational domain. These quantities act 
as a source term in the Helmholtz equation. As the external circuit relations 
are global relations involving integrals over (part of) the computational 
domain, their discretization yields dense blocks. The field-circuit coupling 
can be accomplished in such a way that the overall matrix remains complex 
symmetric. 



 

The presence of dense blocks affects the convergence of Krylov subspace 
methods and hampers the immediate application of algebraic multigrid codes. 
This motivates the study of iterative schemes that allow to treat the 
discretized partial differential problem and the circuit realtions seperately. 



 

The complex symmetry of the hybrid field-circuit coupled system allows to 
solve it by the Conjugate Orthogonal Conjugate Gradient (COCG) method or the 
Symmetric Quasi Minimal Residual (SQMR) method. A possible symmetric 
preconditioner
is a block Jacobi scheme in which the field and circuit relations are treated 
as seperate blocks. In this scheme an algebraic multigrid code can be used 
for the discretized field equations. As the size of the circuit block is much 
smaller than that of the field block, the cost of solving the former is 
negligable compared to solving the latter. Solving the circuit block can 
therefore be done by a direct solver. By switching to a Gauss-Seidel scheme, 
the lower diagonal block in the matrix can be taken into account. The 
application of the Gauss-Seidel scheme as a preconditioner is more expensive 
than the Jacobi scheme by a multiplication of this lower diagonal block only. 
The cost of the block Jacobi and Gauss-Seidel schemes can be reduced by making 
the tolerance to which the field equations are solved dependent on the 
accuracy reached in the outer Krylov iteration. The resulting preconditioner
chances at every Krylov iteration step. A Krylov subspace method that
allows a variable preconditioner is for example Flexible GMRES
(FGMRES). 



 

Numerical experiments have shown that both the block Jacobi and the block 
Gauss-Seidel schemes diverge when applied as solver. The number of block 
Jacobi preconditioned COCG and block Gauss-Seidel preconditioned GMRES 
iterations is independent of the finite element mesh width. Block Gauss-Seidel
preconditioned GMRES requires less iterations to converge than block 
Jacobi preconditioned COCG, but both are equivalent in terms of CPU time. 
Further research is necessary to validate the proposed schemes on models of 
engineering relevance. 





\end{document}
