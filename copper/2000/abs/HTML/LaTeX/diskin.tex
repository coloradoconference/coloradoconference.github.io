\documentclass[11pt]{article}
\usepackage{amsfonts}
\setlength{\parskip}{1.2ex}      % space between paragraphs
\setlength{\parindent}{0em}      % amount of indention
\setlength{\textwidth}{165mm}    % default = 6.5in
\setlength{\oddsidemargin}{0mm}  % default = 0mm
\setlength{\textheight}{225mm}   % default = 9in
\setlength{\topmargin}{-1mm}     % default = 0mm
\date{ ~ \hspace{-4mm}}


\title{Efficient Highly Parallel Multigrid Methods for the Convection Equation.  }

\author{Boris Diskin \\ {\tt bdiskin@icase.edu} \\ Institute for Computer Applications in Science and Engineering (ICASE) \\ Mail Stop 132C, NASA Langley Research Center \\ Hampton, VA 23681-2199.}

\begin{document}
\maketitle
\thispagestyle{empty}





 



The convergence properties of multigrid algorithms are defined by
two factors: (1) the smoothing rate which describes the reduction
of high-frequency error components and (2) the quality of the coarse-grid
correction which is responsible for dumping of smooth error components.
In elliptic problems, all the fine-grid smooth components are
well approximated on the coarse grid built by standard (full) coarsening.
�In nonelliptic problems, however, some fine-grid
�components that are much smoother in the
characteristic direction than in other directions, cannot be approximated
with standard multigrid methods.

We present a novel multigrid approach to the solution of nonelliptic
problems.
This approach is based on semicoarsening and
well-balanced explicit correction terms, added to coarse-grid operators
to maintain on coarse grids the same cross-characteristic interaction as
on the target (fine) grid. Multicolor relaxation schemes are used on all
the levels, allowing a very efficient parallel implementation.
Applications to the 2-D constant-coefficient convection operator
discretized on vertex and cell centered grids are demonstrated.
The resulting multigrid algorithms demonstrate the ``textbook
multigrid efficiency'', meaning the solution to the governing problem
is attained in a computational work which is a small multiple of
the operation count in the discretized problem itself.
Extensions to the three dimensions, to variable coefficients,
and to convection-diffusion
problems are discussed.





\end{document}
