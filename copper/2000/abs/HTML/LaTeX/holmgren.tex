\documentclass[11pt]{article}
\usepackage{amsfonts}
\setlength{\parskip}{1.2ex}      % space between paragraphs
\setlength{\parindent}{0em}      % amount of indention
\setlength{\textwidth}{165mm}    % default = 6.5in
\setlength{\oddsidemargin}{0mm}  % default = 0mm
\setlength{\textheight}{225mm}   % default = 9in
\setlength{\topmargin}{-1mm}     % default = 0mm
\date{ ~ \hspace{-4mm}}


\title{Convergence Acceleration for the Euler Equations  }

\author{Sverker Holmgren ({\tt sverker@tdb.uu.se} \\ Henrik Brand\'{e}n \\ Box 120  \\  SE-754 01 Uppsala  \\  Sweden}

\begin{document}
\maketitle
\thispagestyle{empty}





 



The iterative solution of non-linear systems of equations arising from
systems of hyperbolic, time-independent partial differential 
equations (PDEs) is studied. As an important application, the 
non-linear Euler equations describing inviscid fluid flow in a
channel are solved. The PDE is discretized using a finite 
difference approximation on a structured grid, and the iteration to
steady-state is performed using standard explicit Runge-Kutta methods.



A convergence acceleration technique where a semicirculant (SC)
approximation of the spatial difference operator or Jacobian is
employed as preconditioner is considered. Numerical experiments
show that, for advancing the solution in pseudo-time, the single-stage
Runge-Kutta method (the forward Euler scheme) is the most efficient. 
Furthermore, the step in pseudo-time can be chosen as a constant, 
independent of the number of grid points and the artificial viscosity
parameter. These results are also consistent with analytical results
for a linear, scalar model problem.



The results for the SC method is compared to those of a standard
multigrid (MG) scheme. The number of iterations and the arithmetic
complexities are considered, and it is clear that the SC method is
more efficient for the problems studied. Also, the performance for
the MG scheme is sensitive to the amount of artificial dissipation
added to the difference approximation, while the SC method is not.  









\end{document}
