\documentclass[11pt]{article}
\usepackage{amsfonts}
\setlength{\parskip}{1.2ex}      % space between paragraphs
\setlength{\parindent}{0em}      % amount of indention
\setlength{\textwidth}{165mm}    % default = 6.5in
\setlength{\oddsidemargin}{0mm}  % default = 0mm
\setlength{\textheight}{225mm}   % default = 9in
\setlength{\topmargin}{-1mm}     % default = 0mm
\date{ ~ \hspace{-4mm}}


\title{On the perturbation of the Incomplete Cholesky factorization  }

\author{Gerard Meurant \\ {\tt  meurant@bruyeres.cea.fr} \\ CEA/DIF \\ DCSA/EDD \\ BP12 \\ 91680 Bruyeres le Chatel \\ France}

\begin{document}
\maketitle
\thispagestyle{empty}





 



In this paper we are concerned with the incomplete Cholesky
decomposition of symmetric M-matrices $C=e I +A$ where
$A$ arises from the discretization of an elliptic partial
differential equation, e being a ``small'' positive real
parameter. Such problem arise, for instance, from discretizing
parabolic equations.

As a model problem we can use the two-dimensional heat equation in the
unit
square with Dirichlet boundary conditions and an initial condition
$u(x,0)=u_0(t)$.
We discretize in space with finite differences
with a stepsize  h and a time implicit scheme. Then, we obtain
$(I/k+(\frac1{h^2})A) u^{n+1}=u^n/ k+f^{n+1}$, where
$k$ is the time step and  $A$ the matrix of the corresponding
elliptic problem. For some problems it makes sense to choose
$k=h$. After multiplication by  $h^2$ the matrix of the problem is
$C=kI+A$ where $k$ is ``small''.

The matrix C being symmetric positive definite, we would like to
solve the linear system at each time step with the preconditioned
conjugate gradient algorithm. A very popular preconditioner is the
incomplete Cholesky decomposition without any fill-in  IC(1,1). 
Usually the time step $k$ is
small. Therefore, it is interesting to know if one can compute an
approximation of the incomplete decomposition of C knowing the one
of $A$. Moreover, most of the time the time step is not constant,
therefore one cannot compute the decomposition of C once for
all. It has to be recomputed at each time step. Hence, it would be
interesting to find a way to cheaply update the incomplete
decomposition from one time step to the next.

We will describe answers to this problem and algorithms corresponding to
perturbations of order 0 and 1 and give numerical
examples for several problems with comparisons between our perturbed 
preconditioners and the
incomplete decomposition of  C, the one of $A$ and the SSOR
preconditioning showing that our perturbed factorizations give better
results. 
Then  we will apply the previous results to the solution
of the heat equation. Finally we will deal with another problem where
$C=I+e A$.





\end{document}
