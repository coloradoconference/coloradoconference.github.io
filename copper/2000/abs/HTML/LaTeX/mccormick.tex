\documentclass[11pt]{article}
\usepackage{amsfonts}
\setlength{\parskip}{1.2ex}      % space between paragraphs
\setlength{\parindent}{0em}      % amount of indention
\setlength{\textwidth}{165mm}    % default = 6.5in
\setlength{\oddsidemargin}{0mm}  % default = 0mm
\setlength{\textheight}{225mm}   % default = 9in
\setlength{\topmargin}{-1mm}     % default = 0mm
\date{ ~ \hspace{-4mm}}


\title{Least-Squares Methods for Partial Differential Equations  }

\author{Steve McCormick \\ {\tt  stevem@colorado.edu} \\ Department of Applied Mathematics  \\  Campus Box 526  \\  University of Colorado at Boulder  \\  Boulder, CO 80309-0526}

\begin{document}
\maketitle
\thispagestyle{empty}





 



Least-squares methods have become increasingly popular for solving a wide
variety of partial differential equations (PDEs). First-order system least
squares (FOSLS) is a special type of least-squares method that attempts to
reformulate the PDE so that it is self-adjoint with an associated energy
functional that is product H$^1$ equivalent. Some of the compelling
features of the FOSLS methodology include:




   * self-adjoint equations, stemming from the minimization principle;
 \newline 
   * good operator 'conditioning', stemming from the use of first-order
     formulations of the PDE or inverse norms in the minimization
     principle;
 \newline 
   * finite element and multigrid performance that is optimal and uniform
     in certain parameters (e.g., Reynolds number, Poisson ratio, and wave
     number), stemming from uniform product norm equivalence results.
 \newline 




Unfortunately, one of its limitations is that these benefits come only when
the original problem exhibits sufficient smoothness (it typically requires
H$^2$ regularity). An alternative to the usual L$^2$ form
of FOSLS is to use inverse norms, but this is generally very awkward and
it comes with substantially reduced efficiency.




This talk will describe an alternative, called FOSLL*, that exhibits the
generality of the inverse-norm FOSLS, but retains the efficiency and
simplicity of the L$^2$-norm FOSLS.










\end{document}
