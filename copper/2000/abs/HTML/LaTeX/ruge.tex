\documentclass[11pt]{article}
\usepackage{amsfonts}
\setlength{\parskip}{1.2ex}      % space between paragraphs
\setlength{\parindent}{0em}      % amount of indention
\setlength{\textwidth}{165mm}    % default = 6.5in
\setlength{\oddsidemargin}{0mm}  % default = 0mm
\setlength{\textheight}{225mm}   % default = 9in
\setlength{\topmargin}{-1mm}     % default = 0mm
\date{ ~ \hspace{-4mm}}


\title{FOSPACK: A FOSLS/AMG Package  }

\author{John Ruge \\ {\tt  jruge@colorado.edu} \\ 1005 Gillaspie Dr. Boulder, CO  80303}

\begin{document}
\maketitle
\thispagestyle{empty}





 



First order system least squares (FOSLS) is a recent development by Manteuffel 
and McCormick for expressing and discretizing systems of PDEs. In this method,
the original second-order system is reduced to a first-order system by 
introducing auxiliary unknowns, generally derivatives of the primary unknowns.
Some ``redundant'' equations can also be introduced in order to make the final
problem more well-posed. A FOSLS functional, consisting of the sum of the 
integrals of the residuals of the first-order equations, is then formed, so 
that the problem is recast as a minimization problem for this functional over 
the permissible spaces. Boundary conditions are either incorporated into the
functional, or imposed on the spaces involved. Now, the discrete problem is 
then formed by choosing suitable finite element spaces for the unknowns, and 
setting up the minimization equations. This approach has a number of advantages,
including a natural error measure (the functional value), and more freedom in 
choosing the finite element spaces than with traditional methods. In addition, 
when properly formulated, the equations for the different unknowns are 
somewhat decoupled, and the diagonal blocks (the operators coupling each
unknown with itself) are nicely elliptic. Together, these say that a block-
diagonal preconditioner would work well, and multigrid or multigrid-like methods
sould be very effective for inversion of the diagonal blocks. This is
supported by both theory and numerical tests. 

Algebraic multigrid (AMG) has been shown to work quite well on elliptic 
problems, and is particularly useful with unstructured meshes, complicated
domains, and varying or discontinuous problem coefficients. In the development 
of test codes over the past few years, it has become obvious that combining 
FOSLS and AMG could be very useful, and FOSPACK (First-Order System PACKage)
is the result. The intent is to allow a user to quickly specify a domain (and
initial mesh) and the first-order system, and have the package do the rest. 
The current 2D package allows for fairly arbitrary (user-generated) domains/
meshes or more simple (FOSPACK-generated) domains. A flexible interpreter 
allows for a simple specification of the first-order system (which itself can 
be quite complicated, involving spatially varying coefficients, matrix/vector
notation, and more) in a data file. Other features are automatic linearization 
of some nonlinear first-order problems (with iteration), and the use of ``zones'' 
to allow different types of equations to be specified in different subdomains.
Current work includes development of a graphical user interface and local
refinement capabilities, while future work includes more extensive mesh-
generation options and extension to 3D.

This talk will include a quick overview of FOSLS and AMG, a description of the
different components of the code, and sample problem specifications and
results. 





\end{document}
