\documentclass[11pt]{article}
\usepackage{amsfonts}
\setlength{\parskip}{1.2ex}      % space between paragraphs
\setlength{\parindent}{0em}      % amount of indention
\setlength{\textwidth}{165mm}    % default = 6.5in
\setlength{\oddsidemargin}{0mm}  % default = 0mm
\setlength{\textheight}{225mm}   % default = 9in
\setlength{\topmargin}{-1mm}     % default = 0mm
\date{ ~ \hspace{-4mm}}


\title{Preconditioners for Newton-Krylov Solvers of Richards' Equation  }

\author{Jim E. Jones ({\tt jjones@llnl.gov}) \\ Carol S. Woodward ({\tt cswoodward@llnl.gov}) \\ \\ Center for Applied Scientific Computing \\ Lawrence Livermore National Laboratory \\ P.O. Box 808, L-561 \\ Livermore, CA 94551}

\begin{document}
\maketitle
\thispagestyle{empty}





 



Accurate simulation of water resource management problems
requires the solution of large problems with many spatial zones.
In the case of variably saturated flow problems, we need to
develop scalable and highly efficient algorithms for solving
the large nonlinear systems of equations that arise from the
discretization of Richards' equation.
 \newline 

One approach to solving these systems is to apply a Newton method
requiring a linear Jacobian system solve at each iteration.  Although
Newton's method can have very fast convergence properties, it can
run slowly if the linear system solver is not efficient.
In this work, we solve the linear systems using a multigrid
preconditioned Krylov method. Properly designed multigrid solvers
are optimally efficient in that the work grows linearly with problem
size while the convergence rate is constant.
 \newline 

In this talk, we compare various strategies for solving the
Jacobian system with multigrid. One approach is to base the
multigrid preconditioner on the full Jacobian matrix, which
is non-symmetric. To save storage, and simplify the preconditioner
computation, we also consider multigrid preconditioners based
on symmetric approximations to the full Jacobian matrix.
We compare the efficiency of these various multigrid preconditioning
strategies within the context of
a Newton-Krylov method to solve the nonlinearities.
 \newline 

This work was performed under the
auspices of the U.S. Department of Energy by the
Lawrence Livermore National Laboratory under Contract W-7405-Eng-48.
 \newline 

UCRL-JC-137246 abs





\end{document}
