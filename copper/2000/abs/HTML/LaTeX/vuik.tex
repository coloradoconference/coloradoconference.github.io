\documentclass{article}
\usepackage{amsfonts}
\setlength{\parskip}{1.2ex}      % space between paragraphs
\setlength{\parindent}{0em}      % amount of indention
\setlength{\textwidth}{171mm}    % default = 6.5in
\setlength{\oddsidemargin}{-3mm}  % default = 0mm
\setlength{\textheight}{231mm}   % default = 9in
\setlength{\topmargin}{-9mm}     % default = 0mm
\date{ ~ \hspace{-4mm}}


\title{Deflated ICCG applied to problems with extreme contrasts in the coefficients  }

\author{C. Vuik ({\tt c.vuik@math.tudelft.nl}) \\ A. Segal \\ J. A. Meijerink \\ \\ Delft University of Technology  \\  Faculty of Information Technology and Systems  \\  Department of Applied Mathematical Analysis  \\  P.O. Box 5031  \\  2600 GA Delft  \\  The Netherlands}

\begin{document}
\maketitle
\thispagestyle{empty}





 




Knowledge of the fluid pressure history in the subsurface is important
for an
oil company to predict the presence of oil and natural gas in reservoirs
and a
key factor in safety and environmental aspects of drilling a well.
A mathematical  model for the prediction of fluid pressures in a
geological
time scale is based on conservation of mass and Darcy's law. The
resulting time-dependent three-dimensional non-linear
diffusion equation is linearized and
integrated in time by the Euler backward method. For the space
discretization the finite element method is applied. As a consequence in
each time-step a linear system of equations has to be
solved.



The matrix itself is sparse, but due to fill-in a
direct method requires too much memory to fit in core. Therefore only
iterative methods are
acceptable candidates for the solution of the linear systems of
equations.
Since the coefficient matrix of this system is symmetric and positive
definite, a
preconditioned Conjugate Gradient method (ICCG)
seems to be a suitable iterative method. Unfortunately the earth's crust
consists of layers with large contrasts in permeability. Hence a large
difference of the extreme eigenvalues
 is common in the system of equations to be solved. This leads
to slow convergence of ICCG and conventional termination criteria
are no longer reliable.



We prove that the number of small eigenvalues of the IC preconditioned
matrix is equal to the number of high-permeability
domains, which are not connected to a Dirichlet boundary.
The bad effect of these eigenvalues on the convergence of ICCG
can be annihilated by using the corresponding ``small'' eigenvectors as
projection vectors in Deflated  ICCG.
We describe an approximation of these ``small'' eigenvectors and prove
that the span of these vectors approximates the
``small'' eigenspace of the IC preconditioned matrix.
This implies that the convergence behavior of the
resulting DICCG is independent
of the size of the jump in the coefficients.



We investigate the effect of perturbations of the projection vectors. It
appears that large perturbations in the low-permeability parts
have only a limited influence on
the convergence properties of DICCG. 
This implies that small components of the projection vectors 
in these parts can be neglected to
save work and memory requirements.
Small perturbations of the projection vectors in high-permeability parts
have only a small effect on DICCG.



We observe that
 the resulting method is robust for elliptic problems with highly
discontinuous coefficients, and a robust stopping
criterion is available, which is not the case for the standard 
ICCG method. For high accuracies the DICCG method converges
considerably faster than the  ICCG method. However, for practical
accuracies the gain is enormous. This means that in the context of
non-linear problems or time-dependent problems DICCG is far
superior above ICCG.





\end{document}
