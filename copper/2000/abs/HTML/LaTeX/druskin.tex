\documentclass[11pt]{article}
\usepackage{amsfonts}
\setlength{\parskip}{1.2ex}      % space between paragraphs
\setlength{\parindent}{0em}      % amount of indention
\setlength{\textwidth}{165mm}    % default = 6.5in
\setlength{\oddsidemargin}{0mm}  % default = 0mm
\setlength{\textheight}{225mm}   % default = 9in
\setlength{\topmargin}{-1mm}     % default = 0mm
\date{ ~ \hspace{-4mm}}


\title{Gaussian spectral rules for second order finite-difference schemes  }

\author{Vladimir Druskin \\ {\tt druskin@ridgefield.sdr.slb.com} \\ Schlumberger-Doll Research, Old Quarry Rd. \\ Ridgefield CT 06877-4108}

\begin{document}
\maketitle
\thispagestyle{empty}





 



The subject of this talk is  targeted grid optimization  for second order 
FD approximations of elliptic and hyperbolic problems arising in remote 
sensing, control theory, etc., where the solution is needed only at few 
receiver points.

The optimization can be viewed as an extension of the conception of the 
Gaussian quadrature rules to the second order finite-difference schemes.  A 
standard Gaussian $k$-point quadrature for numerical integration is chosen to 
be exact for  $2k$ polynomials, and an optimal grid with $k$ nodes  is chosen 
to match the impedance at the receiver points for some $2k$ frequencies. To 
solve this problem we employ  methods of rational approximation and linear 
algebra traditionally used for optimization of iterative methods. The 
optimization yields   exponential convergence of the impedance, i.e., the 
standard second order scheme with the three-point stencil exhibits spectral 
superconvergence.

The optimized scheme is applied to  two- and three- dimensional  problems 
in electromagnetic and acoustic well logging. Our numerical experiments 
exhibit  exponential superconvergence at prescribed points (receivers), 
where the cost per grid node is close to that of the  standard second order 
finite-difference scheme. We observe more than one order speedup for 
practically important problems.

Collaborators: Sergey Asvadurov (SLB), David Ingerman (Princeton-MIT), 
Shari Moskow (UFL) and Leonid Knizhnerman (CGE).






\end{document}
