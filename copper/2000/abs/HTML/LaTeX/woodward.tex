\documentclass[11pt]{article}
\usepackage{amsfonts}
\setlength{\parskip}{1.2ex}      % space between paragraphs
\setlength{\parindent}{0em}      % amount of indention
\setlength{\textwidth}{165mm}    % default = 6.5in
\setlength{\oddsidemargin}{0mm}  % default = 0mm
\setlength{\textheight}{225mm}   % default = 9in
\setlength{\topmargin}{-1mm}     % default = 0mm
\date{ ~ \hspace{-4mm}}


\title{Implicit Solution of Radiation-Diffusion Problems Using  Newton-Krylov-Multigrid Methods  }

\author{Carol S. Woodward ({\tt cswoodward@llnl.gov}) \\ Peter Brown \\ Britton Chang \\ Frank Graziani  \\ Craig Kapfer \\ \\ Lawrence Livermore National Laboratory \\ P.O. Box 808, L-561 \\ Livermore, CA  94551}

\begin{document}
\maketitle
\thispagestyle{empty}





 



Modeling of radiation-diffusion processes has traditionally been 
accomplished through simulations based on decoupling and linearizing
the basic physics equations.  By applying these techniques, 
physicists have simplified their model enough that problems of
moderate sizes could be solved.  However, new applications demand
the simulation of larger problems for which the inaccuracies and 
nonscalability of current algorithms prevent solution.  Recent work 
in iterative methods has provided computational scientists with 
new tools for solving these problems.  



In this talk, we present an algorithm for the implicit solution 
of the diffusion approximation coupled to a material temperature 
equation.  This algorithm uses a stiff ODE solver coupled with 
Newton's method for solving the implicit equations arising at 
each time step.  The Jacobian systems are solved by applying 
GMRES preconditioned with a semicoarsening multigrid algorithm.  
By combining the nonlinear Newton iteration with a multigrid 
preconditioner, we hope to take advantage of the fast, robust 
nonlinear convergence of Newton's method and the scalability 
of the linear multigrid method.  Numerical results will be 
shown indicating that the method is accurate.  Furthermore, 
both algorithmic and parallel scalabilities will be explored.



This work was performed under the auspices of the U.S. Department of 
Energy by the Lawrence Livermore National Laboratory under 
Contract W-7405-Eng-48. UCRL-JC-135152 abs







\end{document}
