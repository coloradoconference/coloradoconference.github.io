\documentclass[11pt]{article}
\usepackage{amsfonts}
\setlength{\parskip}{1.2ex}      % space between paragraphs
\setlength{\parindent}{0em}      % amount of indention
\setlength{\textwidth}{165mm}    % default = 6.5in
\setlength{\oddsidemargin}{0mm}  % default = 0mm
\setlength{\textheight}{225mm}   % default = 9in
\setlength{\topmargin}{-1mm}     % default = 0mm
\date{ ~ \hspace{-4mm}}


\title{Parallelizable Preconditioners for Multigroup Flux-limited Diffusion Problems   }

\author{Doug Swesty \\ {\tt dswesty@mail.astro.sunysb.edu} \\ Department of Physics and Astronomy \\ SUNY at Stony Brook \\ Stony Brook, NY 11733}

\begin{document}
\maketitle
\thispagestyle{empty}





 



Numerical multi-group flux-limited diffusion problems are found in a
wide variety of contexts in many fields.  One common aspect of these
problems are the large linear systems that arise from the implicit
finite-difference techniques that are used to solve the underlying
integro-PDEs that describe the flow of radiation.  These problems
define an important class of linear systems worthy of further study.
In order to lay the groundwork for parallel approaches to solving
these systems we investigate the efficiency of a number of
combinations of parallelizable preconditioners and Krylov subspace
methods as applied to a series of test problems with one spatial and
one spectral dimension.  These test problems are designed to span the
range of physical conditions that one would typically encounter in in
a radiation transport simulation.  In turn these test problems produce
linear systems which span the range of possible properties for these
systems.  We describe the results found for several
sparse-approximate-inverse preconditioners when applied to these
problems.  We also discuss the effects of row scaling and symmetric
scaling on these methods.  Finally, we offer some suggestions for
further areas of investigation.






\end{document}
