\documentclass[11pt]{article}
\usepackage{amsfonts}
\setlength{\parskip}{1.2ex}      % space between paragraphs
\setlength{\parindent}{0em}      % amount of indention
\setlength{\textwidth}{165mm}    % default = 6.5in
\setlength{\oddsidemargin}{0mm}  % default = 0mm
\setlength{\textheight}{225mm}   % default = 9in
\setlength{\topmargin}{-1mm}     % default = 0mm
\date{ ~ \hspace{-4mm}}


\title{Adaptive ILU Preconditioning Strategies in Semiconductor Process Modeling  }

\author{Jun Zhang \\ {\tt http://www.cs.uky.edu/~jzhang} \\ Department of Computer Science  \\  University of Kentucky  \\  773 Anderson Hall  \\  Lexington, KY 40506-0046 \\ \\ Anand L. Pardhanani and Graham F. Carey  \\  CFD Laboratory, WRW 301  \\  University of Texas at Austin  \\  Austin, Texas 78712  \\}

\begin{document}
\maketitle
\thispagestyle{empty}





 


Preconditioning strategies based on incomplete factorization
using thresholding with dual dropping (ILUT) are
investigated for iterative solution of sparse linear systems
arising in semiconductor dopant diffusion modeling. Of
particular interest are questions associated with selection and
adaption of threshold parameters with spatial resolution, timestep 
in the adaptive ODE integrator and the problem physics. It is shown
that the convergence rate of the preconditioned iterative solvers
is severely affected by the time steps used in adaptive ODE 
integrator. Thus the accuracy and quality of the ILU preconditioner
have to be adaptively adjusted so that iterative convergence is achieved
and the overall computation cost is kept low.



Adaptive ILU preconditioners are compared with a fixed block Jacobi 
preconditioner and a direct band solver in terms of robustness and
total simulation cost. A few ILU adaptive preconditioning strategies 
are discussed and their performance is compared.





\end{document}
