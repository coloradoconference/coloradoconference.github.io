\documentclass[11pt]{article}
\usepackage{amsfonts}
\setlength{\parskip}{1.2ex}      % space between paragraphs
\setlength{\parindent}{0em}      % amount of indention
\setlength{\textwidth}{165mm}    % default = 6.5in
\setlength{\oddsidemargin}{0mm}  % default = 0mm
\setlength{\textheight}{225mm}   % default = 9in
\setlength{\topmargin}{-1mm}     % default = 0mm
\date{ ~ \hspace{-4mm}}


\title{Computation of Pseudospectra Using ARPACK and EIGS  }

\author{Thomas Wright \\ {\tt tgw@comlab.ox.ac.uk} \\ Oxford University Computing Laboratory,  \\  Wolfson Building, Parks Road,  \\  Oxford, OX1 3QD, England.}

\begin{document}
\maketitle
\thispagestyle{empty}





 



For highly non-normal matrices, pseudospectra may
provide useful
information in addition to eigenvalues. Unfortunately, computing
pseudospectra is expensive for large matrices; the straightforward
algorithm for an
$N\times N$
matrix using a grid of
$v\times v$ takes
$O(v^2 N^3)$ time. This can
often be improved by a 
factor of around $N/4$ using a variety of methods, but even with
this
saving the algorithm is still too slow to make it a practical tool for
large matrices.




In light of the above, one might think that
pseudospectra could be 
computed more efficiently using iterative rather than direct
methods. It has already been shown that the upper Hessenberg matrix
constructed during an Arnoldi iteration can be used to give a
reasonable approximation to the pseudospectra of the original
matrix. Here, I consider an extension to the Matlab iterative
eigenvalue solver `eigs' (based on the Fortran code ARPACK), which
allows the pseudospectra to be computed in a region around
the eigenvalues calculated. This will provide the user of eigs with
graphical information that may highlight cases in which the accuracy
of the eigenvalues returned should be carefully investigated. Due to
the size of the Hessenberg matrices 
constructed, this method is much more efficient than the direct method
(locally, at least), and for large matrices an approximation to the
pseudospectra can be computed along with the eigenvalues for a
relatively small additional cost.




We plan to demonstrate the above idea using an
interactive Matlab GUI. 





\end{document}
