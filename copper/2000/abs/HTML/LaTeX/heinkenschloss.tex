\documentclass[11pt]{article}
\usepackage{amsfonts}
\setlength{\parskip}{1.2ex}      % space between paragraphs
\setlength{\parindent}{0em}      % amount of indention
\setlength{\textwidth}{165mm}    % default = 6.5in
\setlength{\oddsidemargin}{0mm}  % default = 0mm
\setlength{\textheight}{225mm}   % default = 9in
\setlength{\topmargin}{-1mm}     % default = 0mm
\date{ ~ \hspace{-4mm}}


\title{Inexact Sequential Quadratic Programming Methods for the Solution of Inverse Problems Governed by PDEs  }

\author{Matthias Heinkenschloss \\ {\tt  heinken@caam.rice.edu} \\ Rice University  \\  Department of Computational and Applied Mathematics  \\  MS-134  \\  Houston TX 77005-1892}

\begin{document}
\maketitle
\thispagestyle{empty}





 



The numerical solution of inverse problems often leads to
large-scale optimization problems.
In this talk we discuss the solution of such problems
using sequential quadratic programming (SQP) methods.
While SQP methods are well established and successfully
used for constrained optimization, there are still open
issues when SQP subproblems, in particular the linearized
constraints, are not solved by direct linear algebra but
by iterative methods. Critical issues are the control of inexactness
to achieve global convergence of the SQP method and the design
of preconditioners for the solution of saddle point SQP subproblems.
 \newline 
In this talk we will first provide the necessary background on SQP
methods
and motivate the issues that arise when SQP subproblems are solved
iteratively.
Then we will propose ways to control inexactness to maintain global
and local convergence, and we will discuss the construction of
preconditioners
for the saddle point SQP subproblems. Theoretical results will be
supported by applications to inverse problem testexamples.






\end{document}
