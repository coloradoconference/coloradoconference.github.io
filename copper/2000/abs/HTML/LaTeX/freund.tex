\documentclass[11pt]{article}
\usepackage{amsfonts}
\setlength{\parskip}{1.2ex}      % space between paragraphs
\setlength{\parindent}{0em}      % amount of indention
\setlength{\textwidth}{165mm}    % default = 6.5in
\setlength{\oddsidemargin}{0mm}  % default = 0mm
\setlength{\textheight}{225mm}   % default = 9in
\setlength{\topmargin}{-1mm}     % default = 0mm
\date{ ~ \hspace{-4mm}}


\title{Krylov-Subspace Iterations for Reduced-Order Modeling  \newline  in VLSI Circuit Simulation  }

\author{Roland W. Freund \\ {\tt freund@research.bell-labs.com} \\ Bell Laboratories  \\  Room 2C-525  \\  700 Mountain Avenue  \\  Murray Hill, NJ 07974-0636}

\begin{document}
\maketitle
\thispagestyle{empty}





 



Traditional VLSI circuit simulation is based on the numerical 
solution of large-scale stiff nonlinear systems of 
differential-algebraic equations (DAEs).  Numerical techniques 
for such DAEs require the repeated solution of large sparse 
systems of linear equations, and iterative methods, such as 
Krylov-subspace algorithms, seem to be predestined to the 
solution of these linear systems.  However, for various reasons, 
standard SPICE-like circuit simulators employ direct methods, 
rather than iterative methods.  Nevertheless, the continuing 
evolution of VLSI circuits has now come to a point where 
Krylov-subspace methods are finally becoming mainstream tools
in circuit simulation.  More precisely, today, Krylov-subspace 
iterations are employed to generate reduced-order models of large 
linear subsystems of DAEs that describe large linear subcircuits of
the VLSI circuit to be simulated.




In this talk, we explain why and how Krylov-subspace methods 
are used for reduced-order modeling in VLSI circuit simulation.
We discuss various desirable and in part conflicting properties
of the reduced-order models, such as high approximation
accuracy, stability, and passivity, and show how to achieve 
these properties by means of Krylov-subspace iterations.
Numerical results for a variety of circuit examples are
presented.






\end{document}
