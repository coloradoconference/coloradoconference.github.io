\documentclass[11pt]{article}
\usepackage{amsfonts}
\setlength{\parskip}{1.2ex}      % space between paragraphs
\setlength{\parindent}{0em}      % amount of indention
\setlength{\textwidth}{165mm}    % default = 6.5in
\setlength{\oddsidemargin}{0mm}  % default = 0mm
\setlength{\textheight}{225mm}   % default = 9in
\setlength{\topmargin}{-1mm}     % default = 0mm
\date{ ~ \hspace{-4mm}}


\title{Preconditioning Techniques for the Generalized Stokes Problem  }

\author{Vivek Sarin \\ {\tt  sarin@cs.tamu.edu} \\ Department of Computer Science,  \\  Texas A\&M University,  \\  College Station, TX 77843-3112  \\}

\begin{document}
\maketitle
\thispagestyle{empty}





 



A discrete divergence-free basis for fluid velocity has the advantage
of automatically satisfying the continuity constraint in the
Navier-Stokes equations for incompressible fluids. When using local
solenoidal functions to represent such a basis, the resulting linear
system can be very ill-conditioned, in part due to the conditioning of
the basis itself. In an earlier paper [1], we have described an
algebraic multilevel technique to construct a well-conditioned
hierarchical basis that implicitly preconditions the linear system. A
careful analysis of this approach has provided further insights into
nature of the resulting linear system.  In this paper, we outline a
preconditioning technique that exploits the structure of the linear
system to achieve near-optimal convergence for each instance of the
generalized Stokes problem.  We present experiments to show that the
convergence of the preconditioned conjugate gradients method is almost
insensitive to the parameters of the problem.  In conjunction with the
algebraic multilevel scheme for a well-conditioned divergence-free
basis, the proposed technique appears to be a competitive approach for
fluid problems.  




[1] ``An Efficient Iterative Method for the Generalized Stokes 
Problem'', Vivek Sarin and Ahmed Sameh, 
SIAM J. Sci. Comput., Vol. 19, No. 1, pp. 206-226.





\end{document}
