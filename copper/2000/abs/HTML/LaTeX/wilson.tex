\documentclass[11pt]{article}
\usepackage{amsfonts}
\setlength{\parskip}{1.2ex}      % space between paragraphs
\setlength{\parindent}{0em}      % amount of indention
\setlength{\textwidth}{165mm}    % default = 6.5in
\setlength{\oddsidemargin}{0mm}  % default = 0mm
\setlength{\textheight}{225mm}   % default = 9in
\setlength{\topmargin}{-1mm}     % default = 0mm
\date{ ~ \hspace{-4mm}}


\title{Efficient Solver for Mixed and Control-Volume Mixed Finite Element Methods in Three Dimensions  }

\author{John D. Wilson ({\tt jwilson@carbon.cudenver.edu}) \\ Thomas F. Russell \\ \\ Department of Mathematics  \\  University of Colorado at Denver  \\  P.O. Box 173364, Campus Box 170  \\  Denver, CO 80217-3364, U.S.A.}

\begin{document}
\maketitle
\thispagestyle{empty}





 



In simulations of groundwater flow in heterogeneous aquifers, accuracy of
velocities can be substantially enhanced by discretizations based on mixed
or control-volume mixed finite element methods.  A potential side benefit is
improved accuracy of transport calculations that depend on the flow field.  
A practical difficulty, especially acute in 3-D, is that these methods lead
to discrete linear systems that are not positive definite and cannot be
solved by straightforward applications of well-known iterative solvers.

We present a specialized iterative solver for mixed methods in 3-D, built
around a decomposition of the discrete velocity space that leads to a
symmetric positive definite system (the pressure variable is eliminated).  
The decomposition relies on a local basis for the the divergence-free
subspace of the velocity space.  The positive definite system is then solved
using a preconditioned conjugate gradient method.  The unpreconditioned
system is ill-conditioned with respect to the dimension of the discrete
velocity space, but a preconditioner based on additive or multiplicative
domain decomposition makes the condition number independent of this
dimension.  Numerical results verify uniform convergence of the additive
preconditioned conjugate gradient method.  We also study numerically the
effects of heterogeneity and anisotropy of the conductivity coefficients on
the convergence rate.

For distorted hexahedral elements, which are trilinear images of cubes that
can represent irregular subsurface geology, we consider a control-volume
mixed discretization.  A similar decomposition of the velocity space is
used, hoping to obtain a positive definite system, although the system is
not, in general, symmetric.  We also present numerical results for the
convergence of the preconditioned conjugate gradient method applied to
this system.





\end{document}
