\documentclass[11pt]{article}
\usepackage{amsfonts}
\setlength{\parskip}{1.2ex}      % space between paragraphs
\setlength{\parindent}{0em}      % amount of indention
\setlength{\textwidth}{165mm}    % default = 6.5in
\setlength{\oddsidemargin}{0mm}  % default = 0mm
\setlength{\textheight}{225mm}   % default = 9in
\setlength{\topmargin}{-1mm}     % default = 0mm
\date{ ~ \hspace{-4mm}}


\title{Iterative algorithms for least squares problems with multiple right-hand sides  }

\author{Rasmus Munk Larsen \\ {\tt rmunk@solen.stanford.edu} \\   HEPL Annex A206,  Stanford University   \\ Stanford, CA 94305-4085, USA}

\begin{document}
\maketitle
\thispagestyle{empty}





 



We discuss iterative algorithms for solving linear least squares
problems with multiple right-hand sides when the coefficient matrix is
large, ill-conditioned and sparse or structured.




The algorithms presented are all based on the Lanczos
bidiagonalization process, but different implementations allow
significant trade-off between the number of operations, required
amount of storage, parallel efficiency and fraction of BLAS level 3
operations involved. We study the performance of three algorithms:
 \newline 


{\bf 
1.
}
The (block-) conjugate gradient algorithm applied to the normal
equations (CGLS) augmented with a Galerkin projection step.
 \newline 

{\bf 
2.
}
A projection algorithm based on (block-) Lanczos bidiagonalization
with partial reorthogonalization.
 \newline 

{\bf 
3.
}
Solving for each right-hand side independently by means of the LSQR
algorithm. 





As an example, ill-conditioned least squares problems with
multiple right-hand sides arise in image deblurring when restoring
multiple frames distorted by the same point spread function or when
implementing the Backus-Gilbert (BG) inversion method. The BG method
is used extensively in seismology, helioseismology, astronomy and
other areas involving inverse problems.

We present numerical results for examples from image deblurring and
helioseismology, where BG implementations based on the three algorithms
above were used to solve the 2D inverse problem of the solar rotation.





\end{document}
