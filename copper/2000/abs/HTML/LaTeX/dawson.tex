\documentclass[11pt]{article}
\usepackage{amsfonts}
\setlength{\parskip}{1.2ex}      % space between paragraphs
\setlength{\parindent}{0em}      % amount of indention
\setlength{\textwidth}{165mm}    % default = 6.5in
\setlength{\oddsidemargin}{0mm}  % default = 0mm
\setlength{\textheight}{225mm}   % default = 9in
\setlength{\topmargin}{-1mm}     % default = 0mm
\date{ ~ \hspace{-4mm}}


\title{Time-splitting methods for chemically reactive transport  }

\author{Clint Dawson \\ {\tt  clint@ticam.utexas.edu} \\ TICAM \\ University of Texas \\ Austin, TX  78712}

\begin{document}
\maketitle
\thispagestyle{empty}





 



In this talk we will focus on systems of
advection-diffusion-reaction equations which arise when modeling the
transport of chemical species in groundwater.  In these equations, one
can have kinetic and/or equilibrium reactions.  Typically, the
transport (advection-diffusion) is split from the reactions, thus one
takes a transport step for each component, then combines all the
components into a reaction step, which can be solved locally.  This
type of approach, while efficient, gives rise to time truncation
errors.  On the other hand, one can try to solve the system fully
implicitly, which reduces the time truncation errors but gives rise to
huge nonlinear systems.  We will discuss various time-stepping
approaches and examine the errors associated with each.  Examples for
both kinetic and equilibrium reactions will be discussed.




\end{document}
