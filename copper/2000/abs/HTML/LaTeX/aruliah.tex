\documentclass[11pt]{article}
\usepackage{amsfonts}
\setlength{\parskip}{1.2ex}      % space between paragraphs
\setlength{\parindent}{0em}      % amount of indention
\setlength{\textwidth}{165mm}    % default = 6.5in
\setlength{\oddsidemargin}{0mm}  % default = 0mm
\setlength{\textheight}{225mm}   % default = 9in
\setlength{\topmargin}{-1mm}     % default = 0mm
\date{ ~ \hspace{-4mm}}


\title{Preconditioning Strategies for Low Frequency Electromagnetic Simulations   }

\author{Dhavide Aruliah \\ {\tt  dhavide@cs.ubc.ca} \\ Dept. of Computer Science  \\  University of British Columbia  \\  Vancouver, BC, Canada  \\  V6T 1Z4  \\}

\begin{document}
\maketitle
\thispagestyle{empty}





 



  We consider three-dimensional electromagnetic problems in
  parameter regimes where the quasi-static approximation applies, the
  permeability is constant, the conductivity may vary significantly, and
  the range of frequencies is moderate. The difficulties encountered
  include handling solution discontinuities across interfaces and
  accelerating convergence of traditional iterative methods for the
  solution of the linear systems of algebraic equations that arise when
  discretizing Maxwell's equations in the frequency domain.




A finite volume discretization on a staggered grid derived from a
potential-current formulation leads to a large, sparse, complex linear
system of equations with a block structure that is diagonally dominant.
We compare various strategies based on block-ILU, SSOR, and multigrid
preconditioners used within a Krylov space iteration. A Fourier analysis
suggests a block preconditioner that leads to mesh independent rates of
convergence. Numerical experiments support theoretical predictions and
demonstrate the efficacy of this approach.







\end{document}
