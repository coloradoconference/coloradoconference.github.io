\documentclass[11pt]{article}
\usepackage{amsfonts}
\setlength{\parskip}{1.2ex}      % space between paragraphs
\setlength{\parindent}{0em}      % amount of indention
\setlength{\textwidth}{165mm}    % default = 6.5in
\setlength{\oddsidemargin}{0mm}  % default = 0mm
\setlength{\textheight}{225mm}   % default = 9in
\setlength{\topmargin}{-1mm}     % default = 0mm
\date{ ~ \hspace{-4mm}}


\title{Algebraic Multigrid: Accelerating the Convergence  }

\author{Jane Cullum \\ {\tt cullumj@lanl.gov} \\ MS B256, CIC-3, Los Alamos National Laboratory \\ Los Alamos, NM 87545, USA \\ \\ Menno Verbeek \\ {\tt verbeek@math.uu.nl} \\ Mathematical Institute, Utrecht University \\ P.O.Box 80.010, 3508 TA Utrecht, The Netherlands}

\begin{document}
\maketitle
\thispagestyle{empty}





Algebraic multigrid methods
{\em 
(AMG)
}
for
solving large systems of linear equations,
{\em 
Ax=b
}
,
are  matrix-based analogs of geometric 
multigrid methods. 
Both types of methods are multi-level, and
at each level utilize a smoothing procedure
which is applied to residual vectors, and  a coarse
{\em 
grid
} 
computation which is designed
to reduce smoothed residual errors. 




In this paper we consider
{\em 
AMG
}
procedures
which are based upon papers of Ruge and St\"{u}ben and
present a simple modification for accelerating the
convergence of such procedures. We also illustrate
interesting phenomena which are important in
studying the convergence of multigrid procedures, 
including significant differences in convergence rate 
which may be obtained using different choices of starting configurations.





\end{document}
