\documentclass[11pt]{article}
\usepackage{amsfonts}
\setlength{\parskip}{1.2ex}      % space between paragraphs
\setlength{\parindent}{0em}      % amount of indention
\setlength{\textwidth}{165mm}    % default = 6.5in
\setlength{\oddsidemargin}{0mm}  % default = 0mm
\setlength{\textheight}{225mm}   % default = 9in
\setlength{\topmargin}{-1mm}     % default = 0mm
\date{ ~ \hspace{-4mm}}


\title{Multilevel Schwarz Preconditioners for Two- and Three-phase Flow in Porous Media.  }

\author{Chris Kees \\ {\tt  chris\_kees@unc.edu} \\ Department of Environmental Sciences and Engineering  \\  University of North Carolina \\ Chapel Hill, NC 27599-7400}

\begin{document}
\maketitle
\thispagestyle{empty}





 



Differential-algebraic systems can be derived from multiphase model
equations via the method of lines. These systems are solved
effectively with variable order, variable time-step backward
difference formula methods.  Maintaining efficiency in two and three
dimensions, however, requires the parallel iterative solution of
large nonlinear systems of equations.

For this purpose we investigate Newton-Krylov solvers and domain
decomposition preconditioners. We present model formulations, method
implementation details, and results from numerical experiments along
with
a discussion of the trade-offs between efficiency and robustness for
a variety of preconditioners.

We use cell-centered finite differences to discretize the model
equations in space and then advance the solution in time using an
adaptive backward difference scheme that controls the local truncation
error of the temporal discretization. This method produces, at each
step, a nonlinear system of equations, which is solved with a modified
Newton-Krylov method. The Krylov method is scaled, preconditioned,
Bi-CGSTAB. We compare two-level additive and multiplicative Schwarz
preconditioners to one-level additive preconditioners for the scaled
Bi-CGSTAB iteration.




\end{document}
