\documentclass[11pt]{article}
\usepackage{amsfonts}
\setlength{\parskip}{1.2ex}      % space between paragraphs
\setlength{\parindent}{0em}      % amount of indention
\setlength{\textwidth}{165mm}    % default = 6.5in
\setlength{\oddsidemargin}{0mm}  % default = 0mm
\setlength{\textheight}{225mm}   % default = 9in
\setlength{\topmargin}{-1mm}     % default = 0mm
\date{ ~ \hspace{-4mm}}


\title{{\bf  Rational Approximation Preconditioners for General Sparse Linear Systems }  }

\author{Philippe Guillaume, {\tt guillaum@gmm.insa-tlse.fr} \\ UMR MIP 5640, D\'{e}partement de Math\'{e}matiques, INSA \\ Complexe Scientifique de Rangueil, 31077 Toulouse Cedex, France \\ \\ Yousef Saad, {\tt saad@cs.umn.edu} \\ Department of Computer Science and Engineering \\ University of Minnesota \\ 200 Union Street S.E., Minneapolis, MN 55455 \\ \\ Masha Sosonkina, {\tt masha@d.umn.edu} \\ Department of Computer Science \\ University of Minnesota - Duluth, 320 Heller Hall \\ 10 University Drive, Duluth, MN 55812-2496}

\begin{document}
\maketitle
\thispagestyle{empty}





 



We present a class of preconditioning techniques which exploit
rational function approximations to the original matrix. The matrix is
first shifted and then an incomplete LU factorization of the resulting matrix
is computed. The resulting factors are then used to compute a better
preconditioner to the original matrix. Since the incomplete
factorization is made on a shifted matrix, a good LU factorization 
is obtained without
allowing much fill-in. The result needs to be extrapolated to the
non-shifted matrix. Thus, the main motivation for this process is to
save memory. The method is useful for matrices whose incomplete LU
factorizations are poor, e.g., unstable. An error analysis for the 
conjugate gradient
algorithm gives some guidance for choosing the shift of the matrix, in
the special case where the shifted system is solved exactly.
For several diffcult real-world problems, we provide a few examples of the 
practical application of the proposed techniques. 





\end{document}
