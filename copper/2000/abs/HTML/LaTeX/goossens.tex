\documentclass[11pt]{article}
\usepackage{amsfonts}
\setlength{\parskip}{1.2ex}      % space between paragraphs
\setlength{\parindent}{0em}      % amount of indention
\setlength{\textwidth}{165mm}    % default = 6.5in
\setlength{\oddsidemargin}{0mm}  % default = 0mm
\setlength{\textheight}{225mm}   % default = 9in
\setlength{\topmargin}{-1mm}     % default = 0mm
\date{ ~ \hspace{-4mm}}


\title{Mortar Projection in Overlapping Composite Mesh Difference Methods  }

\author{Serge Goossens \\ {\tt http://www.cs.kuleuven.ac.be/\~{}serge/} \\ {\tt Serge.Goossens@cs.kuleuven.ac.be} \\ Katholieke Universiteit Leuven, Department of Computer Science  \\  Celestijnenlaan 200A, B-3001 Heverlee, BELGIUM}

\begin{document}
\maketitle
\thispagestyle{empty}




 


The main topic addressed in this talk is the use of the mortar projection
in an overlapping composite mesh difference method.  We summarise the
overlapping nonmatching mortar method described by Cai et al.  This method
has several desirable properties: it has a full theory, it is consistent,
the accuracy is of optimal order and the error is independent of the size
of the overlap.  The disadvantage of this method is that it needs weights,
which makes it impossible to use fast solvers for the subdomain problems.
The Composite Mesh Difference Method (CMDM) has a full theory, and the
accuracy is of optimal order.  The main advantage of this method is that no
weights are necessary and therefore it allows for fast subdomain solvers.
The disadvantage is that low order interpolation may lead to a local
inconsistent discretisation, resulting in an error that depends on the size
of the overlap.  We discus the easy to implement combination of bilinear
interpolation with the standard five-point stencil, showing the
disadvantages of a CMDM due to the local inconsistency when the low order
interpolaton is used.  



The goal is to take the mortar approach, drop the weights and compare its
results to the non-mortar methods.  In the ideal scheme the accuracy has to
be of optimal order and the error has to be independent of the size of
overlap.  For the purpose of preconditioning and the use of fast solvers it
is desirable not to have weights in the discretisation equations on the
overlapping parts of the domain.



The standard stencils (corresponding to P1 or Q1 finite elements on a
uniform mesh) with bicubic interpolation have these properties.  But these
schemes have several disadvantages.  First of all a bicubic interpolation
formula uses 16 points, which is a lot.  Second, they fit in the CMDM
theory, but this theory requires a large overlap since the interpolation
constant is quite large, i.e. 25/16 in 2D.  The numerical results show that
there is no dependency on the amount of overlap since the scheme is fully
consistent.  The modified stencil with 1D cubic interpolation proposed by
Goossens and Cai also fits in the CMDM theory.  In this case the
interpolation constant is smaller, i.e. 5/4 for 1D cubic interpolation.
Consequently the theory calls for less overlap.  The scheme is fully
consistent so the numerical results show that there is no dependency on the
amount of overlap.



The P1 and Q1 finite element discretisations on a uniform mesh can be
considered as finite difference stencils for which the local truncation
error is second order.  The mortar projection can be used for the
interpolation.  These schemes satisfy all the assumptions made in the CMDM
theory and the resulting method is second order.  However we need to
consider 2 interpolations for the mortar projection.  The first
interpolation is the actual projection from the master to the slave side of
the mortar on the interface.  In the case of overlapping nonmatching grids
we also need to compute the master side of the mortar, which requires
evaluating the P1 or Q1 finite element function.  This boils down to linear
interpolation.  As a result for P1 and Q1 finite elements a linear
interpolation is done in the direction normal on interface.


 
Based on our experience with bilinear interpolation we can estimate the
effect of doing linear interpolation in the direction normal on interface.
This interpolation gives rise to an extra term in the bound on the error in
the extended subdomain just as in the error bound for the P1 stencil with
bilinear interpolation.  A final point we are considering is the dependency
on the overlap.  A large overlap may be required since the mortar
projection does not satisfy the maximum principle (an example will be
shown).  We will present numerical results illustrating the influence of
the size of the overlap both on the accuracy of the discretisation and on
the performance of the iterative solver.





\end{document}
