\documentclass{report}
\usepackage{amsmath,amssymb}

\def\mathbi#1{\textbf{\em #1}}
\def\BA{\bf{A}}
\def\BB{\bf{B}}
\def\bn{\bf{n}}
\def\CP{\mathcal{P}}
\def\CV{\mathcal{V}}
\def\CI{\mathcal{I}}

\newcommand{\gradt}{\nabla\cdot}
\newcommand{\Reals}{\mathbb{R}}
\newcommand{\Cplex}{\mathbb{C}}

\newcommand{\Vecc}[2]{ \left(
	\begin{array}{c}
	#1 \\ #2
	\end{array}
	\right) }

\newcommand{\Matrr}[4]{ \left(
	\begin{array}{cc}
	#1 & #2 \\
	#3 & #4
	\end{array}
	\right) }

\newcommand{\calB}{{\mathcal B}}
\newcommand{\calBB}{{\mathcal B}^{\Box}}
\newcommand{\calK}{\mathcal{K}}
\newcommand{\bfA}{{\mathbf A}}
\newcommand{\bfb}{{\mathbf b}}
\newcommand{\bfr}{{\mathbf r}}
\newcommand{\bfx}{{\mathbf x}}
\newcommand{\bfxex}{{\mathbf x}_{\mbox{\scriptsize $\star$}}}
\newcommand{\Dim}{\mathop{\mathrm{dim\ }}}

\begin{document}
	%%%%%%%%%%%%%%%%%%%%%%%%%%%%%%%%%%%%%%%%%%%

\begin{center}
{\large
{\bf Computing dominant poles of transfer functions}}

	Joost Rommes \\
	Mathematical Institute, Utrecht University \\
	P.O.Box 80.010, 3508 TA Utrecht, The Netherlands \\
	http://www.math.uu.nl/people/rommes \\
	{\tt rommes@math.uu.nl} \\
	Nelson Martins
\end{center}
Recent work on power system
stability, controller design and electromagnetic transients
has used several advanced model reduction techniques.
Although these techniques, such as balanced truncation,
produce good results, they impose high computational costs
and hence are only applicable to moderately sized systems.
Modal model reduction is a cost-effective alternative for
large-scale systems, when only a fraction of the system pole
spectrum is controllable-observable for the transfer
function of interest. Modal reduction produces transfer
function modal equivalents from the knowledge of the
dominant poles and their corresponding residues. In this
talk a specialized eigenvalue method will be presented that
computes the most dominant poles and corresponding residues
of a SISO transfer function.

The transfer function of a
single input single output (SISO) system is defined as
$$
\qquad
\qquad
\qquad
\qquad
H(s) = \mathbf{c}^T (sI - A)^{-1}\mathbf{b} + d,
\qquad
\qquad
\qquad
\qquad
(1)
$$
where
$A\in\Reals^{n\times n}$,
$\mathbf{b},\mathbf{c}\in\Reals^n$, $d\in\Reals$ and
$I\in\Reals^{n\times n}$ is the identity
matrix and $s\in\Cplex$.
Without loss of generality, $d=0$ in
the following.

Let the eigenvalues (poles) of $A$ and the
corresponding right and left eigenvectors be given by the
triplets $(\lambda_j,\mathbf{x}_j,\mathbf{v}_j)$,
and let the right and
left eigenvectors be scaled so that
$\mathbf{v}_j^*\mathbf{x}_j=1$.
It is
assumed that $\mathbf{v}_j^*\mathbf{x}_k=0$ for $j\neq k$.
The transfer function $H(s)$ in equation (1)
can be expressed as a
sum of residues $R_j$ over first order poles:
$$
H(s) = \sum_{j=1}^n
\frac{R_j}{s - \lambda_j},
$$
where the residues $R_j$ are
$$
R_j = (\mathbf{x}_j^T\mathbf{c})(\mathbf{v}_j^*\mathbf{b}).
$$

A \textit{dominant} pole is a pole $\lambda_j$ that
corresponds to a residue $R_j$ with large magnitude
$|R_j| / |\mbox{Re}(\lambda_j)|$ , i.e.,
a pole that is well
observable and controllable in the transfer function. This
can also be observed from the corresponding Bode magnitude
plot of $H(s)$, where peaks occur at frequencies close to
the imaginary parts of the dominant poles of $H(s)$. An
approximation of $H(s)$ that consists of $k<n$ terms with
$|R_j|/|\mbox{Re}(\lambda_j)|$ above some value, determines
the effective transfer function behavior and is called the
transfer function modal equivalent:
$$ H_k(s) = \sum_{j=1}^k \frac{R_j}{s - \lambda_j}. $$
The problem of concern can now be formulated as:
\begin{quote}
Given a SISO linear, time invariant,
dynamical system
$(A,\mathbf{b},\mathbf{c},d)$,
compute $k\ll n$ dominant
poles $\lambda_j$ and the corresponding right and left
eigenvectors $\mathbf{x}_j$ and $\mathbf{v}_j$.
\end{quote}

The algorithm to be presented, called Subspace Accelerated
Dominant Pole Algorithm (SADPA) [1],
combines a Newton
algorithm [2]
with subspace
acceleration, a clever selection strategy and deflation to
efficiently compute the dominant poles and corresponding
residues. It can easily be extended to handle MIMO systems
as well [3].
The performance of the algorithm will be illustrated by
numerical examples of large scale power systems.

[1] J.~Rommes, N.~Martins, {\em Efficient computation
of transfer function dominant poles using subspace
acceleration}, UU Preprint (2005) 1340.

[2] N.~Martins, L.T.G.~Lima, H.J.C.P.~Pinto,
{\em Computing dominant poles of power system transfer
functions}, IEEE Trans.~Power Syst. {\bf 11}(1)
(1996) 162--170.

[3] J.~Rommes, N.~Martins,
{\em Efficient computation of multivariable transfer
function dominant poles using subspace acceleration},
UU Preprint (2006) 1344.



% 	%%%%%%%%%%%%%%%%%%%%%%%%%%%%%%%%%%%%%%%%%%%


\end{document}

