\documentclass{report}
\usepackage{amsmath,amssymb}

\def\mathbi#1{\textbf{\em #1}}
\def\BA{\bf{A}}
\def\BB{\bf{B}}
\def\bn{\bf{n}}
\def\CP{\mathcal{P}}
\def\CV{\mathcal{V}}
\def\CI{\mathcal{I}}

\newcommand{\gradt}{\nabla\cdot}
\newcommand{\Reals}{\mathbb{R}}
\newcommand{\Cplex}{\mathbb{C}}

\newcommand{\Vecc}[2]{ \left(
	\begin{array}{c}
	#1 \\ #2
	\end{array}
	\right) }

\newcommand{\Matrr}[4]{ \left(
	\begin{array}{cc}
	#1 & #2 \\
	#3 & #4
	\end{array}
	\right) }

\newcommand{\calB}{{\mathcal B}}
\newcommand{\calBB}{{\mathcal B}^{\Box}}
\newcommand{\calK}{\mathcal{K}}
\newcommand{\bfA}{{\mathbf A}}
\newcommand{\bfb}{{\mathbf b}}
\newcommand{\bfr}{{\mathbf r}}
\newcommand{\bfx}{{\mathbf x}}
\newcommand{\bfxex}{{\mathbf x}_{\mbox{\scriptsize $\star$}}}
\newcommand{\Dim}{\mathop{\mathrm{dim\ }}}

\begin{document}
	%%%%%%%%%%%%%%%%%%%%%%%%%%%%%%%%%%%%%%%%%%%

\begin{center}
{\large
{\bf A note on GMRES preconditioned by a perturbed \\
$LDL^T$ decomposition with static pivoting}}

	Iain S Duff \\
	Rutherford Appleton Laboratory, Chilton, Didcot \\
	Oxfordshire, OX11 0QX, UK \\
	{\tt i.s.duff@rl.ac.uk} \\
	Mario Arioli, Serge Gratton, Pralet, St\'ephane
\end{center}
This paper is concerned with solving the set of linear
equations $A x = b$ (1)
where the coefficient matrix $A$ is a
symmetric indefinite sparse matrix. Our hope is to solve
this system using a direct method that uses an accurate
factorization of $A$ but sometimes the cost of doing this is
too high in terms of time or memory. We have therefore
looked at the possibility of using static pivoting to avoid
these problems which are particularly acute if the matrix is
highly indefinite as for example can happen for saddle-point
problems.

As our direct method we will use a multifrontal
approach. In this approach we first determine an order for
choosing pivots based on the sparsity structure of $A$
(called the analysis step), and we then accommodate further
pivoting for numerical stability during the subsequent
numerical factorization phase. The problem when the matrix
is highly indefinite is that the resulting pivot sequence
used in the numerical factorization can differ substantially
from that predicted by the analysis step. In the
multifrontal context, the factorization can be represented
by a tree at each node of which elimination operations are
performed on a partially summed frontal matrix
$\Matrr{F_{11}}{F_{12}}{F_{12}^T}{F_{22}}$ (2)
and pivots at that stage can only
be chosen from within the fully summed block $F_{11}$. The
problem occurs when it is impossible or numerically suicidal
to eliminate all of $F_{11}$ resulting in more work and
storage (sometimes dramatically more) than forecast. A
simple way to avoid this problem is to force the elimination
of all of $F_{11}$ through static pivoting.

We thus
assume that the matrix $A$ has been factorized using the HSL
package {\tt MA57} with the option of using static pivoting
[1]. The static pivoting strategy will set the
diagonal entry to $\pm\tau$ when it is impossible to find a
suitable pivot in the fully summed blocks. It is common to
choose $\tau \approx \sqrt{{{\bf \varepsilon }\,}} ||A||$
(${{\bf \varepsilon }\,}$ machine precision).

Therefore, the computed factors $\hat{L}$ and
$\hat{D}$ are, in exact arithmetic, the exact factorization
of the perturbed problem
$A + E = \hat{L} \hat{D} \hat{L}^T$ (3),
where the matrix
$|E| \le \tau I$ is a diagonal matrix of rank equal to the
number of static pivots used during the factorization. The
nonzero diagonal entries in $E$ correspond to the positions
at which static pivoting was performed and they are all
equal to $\tau$ in modulus. Note that if $\tau$ is chosen
too small then the factorization could be very unstable
whereas if it is chosen too large, the factorization will be
stable but will not be an accurate factorization of the
original matrix (that is, $|E|$ will be large).

Equation (3)
gives a splitting of $A$ in terms of
$M = \hat{L} \hat{D} \hat{L}^T$ and $E$,
$A = M - E$,
and the solution of (1) can be expressed as the
solution of the equivalent system
$(I - M^{-1} E) x = M^{-1}b$ (4).
If the spectral radius of the matrix
$I - M^{-1} E $
is less than one, the system (4) can be
solved using iterative refinement. This has been used by
many authors, including [1] and is successful
over a wide range of matrices although is somewhat sensitive
to the value of $\tau$. If, however, the spectral radius is
greater or equal to one (or $\approx 1$), it is necessary to
switch to a more powerful method like GMRES. Although the
matrix is symmetric, we choose GMRES since it gives us much
more freedom to work with a wide range of preprocessors and
preconditionings.

We have found experimentally that using
the factorization (3) as a preconditioning for
GMRES works in most cases and is, as expected much more
robust than iterative refinement. Indeed GMRES gives
normwise backward stability in most cases, which is not the
case for iterative refinement. However, there are cases
where we do not get convergence to a scaled residual at
machine precision.

We have, however, found that restarted
GMRES performs better and that using FGMRES, even though our
preconditioner remains constant, does even better.

We
illustrate this through numerical experiment and then show
theoretically that, under reasonable assumptions, FGMRES
preconditioned by our static pivoting factorization is
backward stable so that a small scaled residual can be
achieved. Our analysis also holds for the case of restarted
FGMRES that we advocate as a measure to control the memory
requirement while still achieving the desired accuracy.
Indeed we give theoretical arguments why the restarting
often greatly improves the convergence.


[1] I.~S.~Duff and S.~Pralet, {\em Towards a stable static
pivoting strategy for the sequential and parallel solution
of sparse symmetric indefinite systems},
{T}echnical {R}eport TR/PA/05/26, CERFACS,
Toulouse, France, 2005. (Also available as RAL Report
RAL-TR-2005-007 and IRIT Report RT/TLSE/05/04.)



	%%%%%%%%%%%%%%%%%%%%%%%%%%%%%%%%%%%%%%%%%%%


\end{document}

