\documentclass{report}
\usepackage{amsmath,amssymb}

\def\mathbi#1{\textbf{\em #1}}
\def\BA{\bf{A}}
\def\BB{\bf{B}}
\def\bn{\bf{n}}
\def\CP{\mathcal{P}}
\def\CV{\mathcal{V}}
\def\CI{\mathcal{I}}

\newcommand{\gradt}{\nabla\cdot}
\newcommand{\Reals}{\mathbb{R}}
\newcommand{\Cplex}{\mathbb{C}}

\newcommand{\Vecc}[2]{ \left(
	\begin{array}{c}
	#1 \\ #2
	\end{array}
	\right) }

\newcommand{\Matrr}[4]{ \left(
	\begin{array}{cc}
	#1 & #2 \\
	#3 & #4
	\end{array}
	\right) }

\newcommand{\calB}{{\mathcal B}}
\newcommand{\calBB}{{\mathcal B}^{\Box}}
\newcommand{\calK}{\mathcal{K}}
\newcommand{\bfA}{{\mathbf A}}
\newcommand{\bfb}{{\mathbf b}}
\newcommand{\bfr}{{\mathbf r}}
\newcommand{\bfx}{{\mathbf x}}
\newcommand{\bfxex}{{\mathbf x}_{\mbox{\scriptsize $\star$}}}
\newcommand{\Dim}{\mathop{\mathrm{dim\ }}}

\begin{document}
	%%%%%%%%%%%%%%%%%%%%%%%%%%%%%%%%%%%%%%%%%%%

\begin{center}
{\large
{\bf A fast iterative solver for acoustic scattering by objects \\
	in layered media}}

	Jari Toivanen \\
	Center for Research in Scientific Computation \\
	Box 8205, North Carolina State University, Raleigh NC 27695-8205 \\
	{\tt jatoivan@ncsu.edu} \\
	Kazufumi Ito
\end{center}
We consider the computation of time-harmonic acoustic
scattering by objects in layered media. An example of such
problem is the scattering by a mine buried in sediment. The
computational domain can be tens or hundreds of meters long
while the target requires modeling of details smaller than
one centimeter. A discretized problem can have several
billion degrees of freedom.

We decompose the
computational domain into far-field and near-field domains
and then we perform a finite element discretization. For the
vastly larger far-field domain we use an orthogonal mesh and
a preconditioner based on a fast direct solver. For the
near-field domain a more standard preconditioner can be
used. The combination of these two defines a preconditioner
for the GMRES method. An essential implementation detail is
the reduction of iterations into a small sparse subspace.
Due to this the memory usage is essentially reduced and the
GMRES method can be used without restarts.

We present
numerical results with two-dimensional and three-dimensional
problems demonstrating the efficiency of the proposed
approach. For example, we show that it is possible to solve
problems with more than a billion degrees of freedom on a
modern PC.



	%%%%%%%%%%%%%%%%%%%%%%%%%%%%%%%%%%%%%%%%%%%


\end{document}

