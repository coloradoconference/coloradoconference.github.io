\documentclass{report}
\usepackage{amsmath,amssymb}

\def\mathbi#1{\textbf{\em #1}}
\def\BA{\bf{A}}
\def\BB{\bf{B}}
\def\bn{\bf{n}}
\def\CP{\mathcal{P}}
\def\CV{\mathcal{V}}
\def\CI{\mathcal{I}}

\newcommand{\gradt}{\nabla\cdot}
\newcommand{\Reals}{\mathbb{R}}
\newcommand{\Cplex}{\mathbb{C}}

\newcommand{\Vecc}[2]{ \left(
	\begin{array}{c}
	#1 \\ #2
	\end{array}
	\right) }

\newcommand{\Matrr}[4]{ \left(
	\begin{array}{cc}
	#1 & #2 \\
	#3 & #4
	\end{array}
	\right) }

\newcommand{\calB}{{\mathcal B}}
\newcommand{\calBB}{{\mathcal B}^{\Box}}
\newcommand{\calK}{\mathcal{K}}
\newcommand{\bfA}{{\mathbf A}}
\newcommand{\bfb}{{\mathbf b}}
\newcommand{\bfr}{{\mathbf r}}
\newcommand{\bfx}{{\mathbf x}}
\newcommand{\bfxex}{{\mathbf x}_{\mbox{\scriptsize $\star$}}}
\newcommand{\Dim}{\mathop{\mathrm{dim\ }}}

\begin{document}
	%%%%%%%%%%%%%%%%%%%%%%%%%%%%%%%%%%%%%%%%%%%

\begin{center}
{\large
{\bf A family of generalized Gauss-Newton methods for 2D inverse \\
	gravimetry problem}}

	Alexandra Smirnova \\
	Dept. of Mathematics and Statistics \\
	Georgia State University, 30 Pryor St.,  Atlanta GA 30303 \\
	{\tt smirn@mathstat.gsu.edu}
\end{center}
We consider a generalized Gauss-Newton's scheme
$$ x_{n+1}=\xi-\theta(F^{\prime*}(x_n)F'(x_n),\alpha_n)
F^{\prime*}(x_n)\{F(x_n)-f-F'(x_n)(x_n-\xi)\} $$ for
solving nonlinear unstable operator equation
$\,F(x)=f\,$ in a Hilbert space. In case of noisy data
we propose a novel a posteriori stopping rule $$
||F(x_N)-f_\delta||^2\le \tau \delta <
||F(x_n)-f_\delta||^2,\quad 0\le n< N,\quad
\tau >1, $$
and
prove a convergence theorem under a source type
condition on the solution. As a consequence of this
theorem we obtain convergence rates for five different
generating functions,
$\,\theta=\theta(\lambda,\alpha),\,$ of a spectral
parameter $\lambda$ and $\alpha>0$.

The new algorithms are tested on the 2D inverse
gravimetry problem reduced to a nonlinear integral
equation of the first kind: $$ F(x):=g \,\triangle
\sigma \int^b_a\int^d_c\left\{
\frac{1}{[\,(\xi-t)^2+(\nu-s)^2+x^2(\xi,\nu)\,]^{1/2}}
\right.  \\ \left.  -
\frac{1}{[\,(\xi-t)^2+(\nu-s)^2+h^2\,]^{1/2}}
\right\}\,d\xi\,d\nu = f(t,s), $$ where $g$ is the
gravitational constant, $\triangle \sigma$ is the
density jump on the interface, and $f(t,s)$ is the
gravitational strength anomaly.  The results of
numerical simulations are presented and some practical
recommendations on the choice of parameters are given.



	%%%%%%%%%%%%%%%%%%%%%%%%%%%%%%%%%%%%%%%%%%%


\end{document}

