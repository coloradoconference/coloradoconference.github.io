\documentclass{report}
\usepackage{amsmath,amssymb}

\def\mathbi#1{\textbf{\em #1}}
\def\BA{\bf{A}}
\def\BB{\bf{B}}
\def\bn{\bf{n}}
\def\CP{\mathcal{P}}
\def\CV{\mathcal{V}}
\def\CI{\mathcal{I}}

\newcommand{\gradt}{\nabla\cdot}
\newcommand{\Reals}{\mathbb{R}}
\newcommand{\Cplex}{\mathbb{C}}

\newcommand{\Vecc}[2]{ \left(
	\begin{array}{c}
	#1 \\ #2
	\end{array}
	\right) }

\newcommand{\Matrr}[4]{ \left(
	\begin{array}{cc}
	#1 & #2 \\
	#3 & #4
	\end{array}
	\right) }

\newcommand{\calB}{{\mathcal B}}
\newcommand{\calBB}{{\mathcal B}^{\Box}}
\newcommand{\calK}{\mathcal{K}}
\newcommand{\bfA}{{\mathbf A}}
\newcommand{\bfb}{{\mathbf b}}
\newcommand{\bfr}{{\mathbf r}}
\newcommand{\bfx}{{\mathbf x}}
\newcommand{\bfxex}{{\mathbf x}_{\mbox{\scriptsize $\star$}}}
\newcommand{\Dim}{\mathop{\mathrm{dim\ }}}

\begin{document}
	%%%%%%%%%%%%%%%%%%%%%%%%%%%%%%%%%%%%%%%%%%%

\begin{center}
{\large
{\bf A constrained minimization algorithm for solving nonlinear \\
	eigenvalue problems in electronic structure calculation}}

	Chao Yang \\
	Lawerence Berkeley National Laboratory \\
	1 Cyclotron Rd, MS-50F, Berkeley CA 94720 \\
	{\tt cyang@lbl.gov} \\
	Juan Meza, Ling-wang Wang
\end{center}
One of the fundamental problems in electronic structure
calculation is to determine electron orbitals associated
with the minimum total energy of large atomistic systems.
The total energy minimization problem is often formulated as
a nonlinear eigenvalue problem and solved by an iterative
scheme called Self Consistent Field (SCF) iteration. In this
talk, a new direct constrained optimization algorithm for
minimizing the Kohn-Sham (KS) total energy functional is
presented.

The key ingredients of this algorithm involve
projecting the total energy functional into a sequences of
subspaces of small dimensions and seeking the minimizer of
total energy functional within each subspace. The minimizer
of the projected energy functional not only provides a
search direction along which the KS total energy functional
decreases but also gives an optimal ``step-length'' to move
along this search direction. Due to the small dimension of
the projected problem, the minimizer of the projected energy
functional can be computed by several different methods.
These methods will be examined and compared in this talk.
Numerical examples will be provided to demonstrate that this
new direct constrained optimization algorithm can be more
efficient and robust than the SCF iteration.



	%%%%%%%%%%%%%%%%%%%%%%%%%%%%%%%%%%%%%%%%%%%


\end{document}

