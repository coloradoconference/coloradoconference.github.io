\documentclass{report}
\usepackage{amsmath,amssymb}

\def\mathbi#1{\textbf{\em #1}}
\def\BA{\bf{A}}
\def\BB{\bf{B}}
\def\bn{\bf{n}}
\def\CP{\mathcal{P}}
\def\CV{\mathcal{V}}
\def\CI{\mathcal{I}}

\newcommand{\gradt}{\nabla\cdot}
\newcommand{\Reals}{\mathbb{R}}
\newcommand{\Cplex}{\mathbb{C}}

\newcommand{\Vecc}[2]{ \left(
	\begin{array}{c}
	#1 \\ #2
	\end{array}
	\right) }

\newcommand{\Matrr}[4]{ \left(
	\begin{array}{cc}
	#1 & #2 \\
	#3 & #4
	\end{array}
	\right) }

\newcommand{\calB}{{\mathcal B}}
\newcommand{\calBB}{{\mathcal B}^{\Box}}
\newcommand{\calK}{\mathcal{K}}
\newcommand{\bfA}{{\mathbf A}}
\newcommand{\bfb}{{\mathbf b}}
\newcommand{\bfr}{{\mathbf r}}
\newcommand{\bfx}{{\mathbf x}}
\newcommand{\bfxex}{{\mathbf x}_{\mbox{\scriptsize $\star$}}}
\newcommand{\Dim}{\mathop{\mathrm{dim\ }}}

\begin{document}
	%%%%%%%%%%%%%%%%%%%%%%%%%%%%%%%%%%%%%%%%%%%

\begin{center}
{\large
{\bf Coupled Gauss-Seidel algorithm in multigrid mode for the \\
	thin film equation}}

	Stephen Langdon \\
	Dept.~of Mathematics, University of Reading \\
	Whiteknights, PO Box 220, Reading RG6 2AX United Kingdom \\
	{\tt s.langdon@reading.ac.uk} \\
	John W Barrett
\end{center}
In this talk we consider the iterative solution of a
nonlinear system arising from a finite element
discretisation of the fourth order equation \[
\frac{\partial u}{\partial t} + \nabla \cdot
(|u|^{\gamma}\nabla\Delta u) = 0, \] where $\gamma>0$. This
equation models a thin liquid film spreading on a solid
surface, with $u$ the height of the film. It is well known
that for nonnegative initial data, the solution $u$ remains
nonnegative for all time. However, this nonnegativity of $u$
is not guaranteed if the equation is discretised in a naive
way. Imposing the nonnegativity of $u$ as a constraint leads
to a discrete variational inequality to be solved at each
time step. Specifically defining $S^h$ to be the space of
piecewise linear functions on a uniform mesh and $K^h\subset
S^h$ to be the space of nonnegative functions in $S^h$,
given $U^{n-1}\in K^h$ we seek $U^{n} \in K^h$ and $W^n \in
S^h$ such that \begin{eqnarray*} &(U^{n},\chi)^h +
\tau\,(|U^{n-1}|^{\gamma}\nabla W^n,\nabla \chi) =
(U^{n-1},\chi)^h \qquad \forall \ \chi \in S^h,& \\ &(\nabla
U^{n},\nabla (\chi-U^{n})) \geq (W^{n}, \chi-U^n)^h \qquad
\forall \ \chi \in K^h,& \end{eqnarray*} where $\tau$
represents the time step, and $(\cdot,\cdot)$ and
$(\cdot,\cdot)^h$ represent the $L^2$ inner product and its
trapezoidal rule discretisation respectively.


Well-posedness, stability, unique solvability, and
convergence of $U^n$ to $u$ and $W^n$ to $w=-\Delta u$ were
established by Barrett, Blowey and Garcke in 1998. To solve
the nonlinear system they used an Uzawa algorithm, for which
they were able to demonstrate convergence of
$U^{n,p}\rightarrow U^n$ and of $\int_{\Omega}
|U^{n-1}|^{\gamma} |\nabla(W^n-W^{n,p})|^2 dx \rightarrow
0$, as the number of iterations $p\rightarrow\infty$.
However, the convergence of this algorithm was found to be
extremely slow. Here, we propose instead a coupled
Gauss-Seidel algorithm in multigrid mode for the iterative
solution of the nonlinear system. Proving convergence for
the multigrid algorithm remains an open question, but
numerical results indicate mesh independent convergence to
the same solution as that achieved with the Uzawa algorithm
in most cases tested, with a greatly reduced computational
cost compared to iterating on a single grid.



	%%%%%%%%%%%%%%%%%%%%%%%%%%%%%%%%%%%%%%%%%%%


\end{document}

