\documentclass{report}
\usepackage{amsmath,amssymb}

\def\mathbi#1{\textbf{\em #1}}
\def\BA{\bf{A}}
\def\BB{\bf{B}}
\def\bn{\bf{n}}
\def\CP{\mathcal{P}}
\def\CV{\mathcal{V}}
\def\CI{\mathcal{I}}

\newcommand{\gradt}{\nabla\cdot}
\newcommand{\Reals}{\mathbb{R}}
\newcommand{\Cplex}{\mathbb{C}}

\newcommand{\Vecc}[2]{ \left(
	\begin{array}{c}
	#1 \\ #2
	\end{array}
	\right) }

\newcommand{\Matrr}[4]{ \left(
	\begin{array}{cc}
	#1 & #2 \\
	#3 & #4
	\end{array}
	\right) }

\newcommand{\calB}{{\mathcal B}}
\newcommand{\calBB}{{\mathcal B}^{\Box}}
\newcommand{\calK}{\mathcal{K}}
\newcommand{\bfA}{{\mathbf A}}
\newcommand{\bfb}{{\mathbf b}}
\newcommand{\bfr}{{\mathbf r}}
\newcommand{\bfx}{{\mathbf x}}
\newcommand{\bfxex}{{\mathbf x}_{\mbox{\scriptsize $\star$}}}
\newcommand{\Dim}{\mathop{\mathrm{dim\ }}}

\begin{document}
	%%%%%%%%%%%%%%%%%%%%%%%%%%%%%%%%%%%%%%%%%%%

\begin{center}
{\large
{\bf Finite-difference solution of the 3D EM problem
using integral equation type preconditioners}}

	Mikhail Zaslavsky \\
	320 Bent St., Cambridge MA 02141 \\
	{\tt mzaslavsky@ridgefield.oilfield.slb.com} \\
	S Davydychdeva, V Druskin,
	L.~Knizhnerman, A.~Abubakar, T.~Habashy
\end{center}
The electromagnetic prospecting problem requires fine
gridding to account for sea bottom and to model complicated
targets. This results in large computational costs using
conventional finite-difference solvers. To circumvent these
problems, we employ a volume integral equation approach for
preconditioning and to eliminate the background, thus
significantly reducing the condition number and
dimensionality of the problem. Since the problem should be
solved in unbounded domain we use so-called optimal grids to
truncate error of approximation at infinity. Special
averaging procedure is proposed to account for
inhomogenuity. Theory and numerical results will be
presented.


	%%%%%%%%%%%%%%%%%%%%%%%%%%%%%%%%%%%%%%%%%%%


\end{document}

