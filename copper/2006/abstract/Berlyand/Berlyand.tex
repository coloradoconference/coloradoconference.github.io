\documentclass{report}
\usepackage{amsmath,amssymb}

\def\mathbi#1{\textbf{\em #1}}
\def\BA{\bf{A}}
\def\BB{\bf{B}}
\def\bn{\bf{n}}
\def\CP{\mathcal{P}}
\def\CV{\mathcal{V}}
\def\CI{\mathcal{I}}

\newcommand{\gradt}{\nabla\cdot}
\newcommand{\Reals}{\mathbb{R}}
\newcommand{\Cplex}{\mathbb{C}}

\newcommand{\Vecc}[2]{ \left(
	\begin{array}{c}
	#1 \\ #2
	\end{array}
	\right) }

\newcommand{\Matrr}[4]{ \left(
	\begin{array}{cc}
	#1 & #2 \\
	#3 & #4
	\end{array}
	\right) }

\newcommand{\calB}{{\mathcal B}}
\newcommand{\calBB}{{\mathcal B}^{\Box}}
\newcommand{\calK}{\mathcal{K}}
\newcommand{\bfA}{{\mathbf A}}
\newcommand{\bfb}{{\mathbf b}}
\newcommand{\bfr}{{\mathbf r}}
\newcommand{\bfx}{{\mathbf x}}
\newcommand{\bfxex}{{\mathbf x}_{\mbox{\scriptsize $\star$}}}
\newcommand{\Dim}{\mathop{\mathrm{dim\ }}}

\begin{document}
	%%%%%%%%%%%%%%%%%%%%%%%%%%%%%%%%%%%%%%%%%%%

\begin{center}
{\large
{\bf Discrete network approximation for singular behavior of \\
the effective viscosity of concentrated suspensions}}

	Leonid Berlyand \\
	Mathematics Department, Penn State University \\
	University Park  PA 16801  \\
	{\tt berlyand@math.psu.edu} \\
	Yuliya Gorb, Alexei Novikov
\end{center}
We present a new approach for calculation of effective
properties of high contrast random composites and illustrate
it by considering highly packed suspensions of rigid
particles in a Newtonian fluid.

The main idea of this
approach is a reduction of the original continuum problem,
which is described by PDE with rough coefficients, to a
discrete random network. This reduction is done in two steps
which constitute the ``fictitious fluid'' approach. In Step 1
we introduce a ``fictitious fluid'' continuum problem when
fluid flows only in narrow channels between closely spaced
particles, which reflects physical fact that the dominant
contribution to the dissipation rate comes from these
channels. In Step 2 we derive a discrete network
approximation for the latter continuum problem.

Next we
use this approach to calculate the effective viscous
dissipation rate in a 2D model of a suspension. We show that
under certain boundary conditions the model exhibits an
anomalously strong rate of blow up when the concentration of
particles tends to maximal. We explore physical ramification
of this phenomenon.

We will also discuss how an
iterative procedure of the network construction which may be
used in the study of dynamics of highly packed suspensions.



	%%%%%%%%%%%%%%%%%%%%%%%%%%%%%%%%%%%%%%%%%%%


\end{document}

