\documentclass{report}
\usepackage{amsmath,amssymb}

\def\mathbi#1{\textbf{\em #1}}
\def\BA{\bf{A}}
\def\BB{\bf{B}}
\def\bn{\bf{n}}
\def\CP{\mathcal{P}}
\def\CV{\mathcal{V}}
\def\CI{\mathcal{I}}

\newcommand{\gradt}{\nabla\cdot}
\newcommand{\Reals}{\mathbb{R}}
\newcommand{\Cplex}{\mathbb{C}}

\newcommand{\Vecc}[2]{ \left(
	\begin{array}{c}
	#1 \\ #2
	\end{array}
	\right) }

\newcommand{\Matrr}[4]{ \left(
	\begin{array}{cc}
	#1 & #2 \\
	#3 & #4
	\end{array}
	\right) }

\newcommand{\calB}{{\mathcal B}}
\newcommand{\calBB}{{\mathcal B}^{\Box}}
\newcommand{\calK}{\mathcal{K}}
\newcommand{\bfA}{{\mathbf A}}
\newcommand{\bfb}{{\mathbf b}}
\newcommand{\bfr}{{\mathbf r}}
\newcommand{\bfx}{{\mathbf x}}
\newcommand{\bfxex}{{\mathbf x}_{\mbox{\scriptsize $\star$}}}
\newcommand{\Dim}{\mathop{\mathrm{dim\ }}}

\begin{document}
	%%%%%%%%%%%%%%%%%%%%%%%%%%%%%%%%%%%%%%%%%%%

\begin{center}
{\large
{\bf Splittings for iterative solution of linear systems}}

	Marko Huhtanen \\
	Institute of Mathematics, Helsinki University of Technology \\
	Box 1100, FIN-02015, Finland \\
	{\tt marko.huhtanen@tkk.fi} \\
	Mikko Byckling
\end{center}
Consider iteratively solving a
linear system
$Ax=b$,
with invertible $A\in \Cplex^{n \times n}$ and
$b\in \Cplex^n$, by splitting the matrix $A$ as
$A=L+R$, where $L$
and $R$ are both readily invertible. In such a case the
recently introduced residual minimizing Krylov subspace
method [1] can be executed, allowing, in a certain
sense, preconditioning simultaneously with $L$ and $R$.

Splittings satisfying $A=L+R$ result either form the
structure of the problem, or are algebraic. Splittings of
Gauss-Seidel type belong to the latter category. In this
talk we discuss such splittings of $A$.

[1] M.~Huhtanen and O.~Nevanlinna,
{\em A minimum residual algorithm for
solving linear systems},
submitted manuscript available at
\verb9www.math.hut.fi/~mhuhtane/index.html9.
% \texttt{www.math.hut.fi/$\sim$mhuhtane/index.html}.



	%%%%%%%%%%%%%%%%%%%%%%%%%%%%%%%%%%%%%%%%%%%


\end{document}

