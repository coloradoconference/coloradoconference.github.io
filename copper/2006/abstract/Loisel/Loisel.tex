\documentclass{report}
\usepackage{amsmath,amssymb}

\def\mathbi#1{\textbf{\em #1}}
\def\BA{\bf{A}}
\def\BB{\bf{B}}
\def\bn{\bf{n}}
\def\CP{\mathcal{P}}
\def\CV{\mathcal{V}}
\def\CI{\mathcal{I}}

\newcommand{\gradt}{\nabla\cdot}
\newcommand{\Reals}{\mathbb{R}}
\newcommand{\Cplex}{\mathbb{C}}

\newcommand{\Vecc}[2]{ \left(
	\begin{array}{c}
	#1 \\ #2
	\end{array}
	\right) }

\newcommand{\Matrr}[4]{ \left(
	\begin{array}{cc}
	#1 & #2 \\
	#3 & #4
	\end{array}
	\right) }

\newcommand{\calB}{{\mathcal B}}
\newcommand{\calBB}{{\mathcal B}^{\Box}}
\newcommand{\calK}{\mathcal{K}}
\newcommand{\bfA}{{\mathbf A}}
\newcommand{\bfb}{{\mathbf b}}
\newcommand{\bfr}{{\mathbf r}}
\newcommand{\bfx}{{\mathbf x}}
\newcommand{\bfxex}{{\mathbf x}_{\mbox{\scriptsize $\star$}}}
\newcommand{\Dim}{\mathop{\mathrm{dim\ }}}

\begin{document}
	%%%%%%%%%%%%%%%%%%%%%%%%%%%%%%%%%%%%%%%%%%%

\begin{center}
{\large
{\bf A domain decomposition method that converges in two steps \\
	for three subdomains}}

	S\'ebastien Loisel \\
	2~4, rue du Li\`evre, Case postale 64 \\
	1211 Gen\`eve 4 (Suisse) \\
	{\tt loisel@math.unige.ch}
\end{center}
In Schwarz-like domain decomposition methods, a domain
$\Omega$ is broken into two or more subdomains and
Dirichlet, Neumann, Robin or pseudo-differential problems
are iteratively solved on each subdomain. For certain
problems, it is well-known that the Dirichlet-Neumann
iteration for two subdomains will converge in two steps. Let
$\Omega$ be an open domain and
$\Omega_{1},\Omega_{2},\Omega_{3}$ a domain decomposition of
$\Omega$ such that each pair of subdomains shares an
interface (for instance,
$\Omega=\{ z\in\Cplex \;|\; |z|<1\}$
and $\Omega_{j}=\{ re^{i\theta} \;|\; 0<r<1$ and
$\theta\in(2j\pi/3,2(j+1)\pi/3)\}$,
$j=1,2,3$).
We will show a new Schwarz-like domain decomposition
method that converges in two iterations in this situation.



	%%%%%%%%%%%%%%%%%%%%%%%%%%%%%%%%%%%%%%%%%%%


\end{document}

