\documentclass{report}
\usepackage{amsmath,amssymb}

\def\mathbi#1{\textbf{\em #1}}
\def\BA{\bf{A}}
\def\BB{\bf{B}}
\def\bn{\bf{n}}
\def\CP{\mathcal{P}}
\def\CV{\mathcal{V}}
\def\CI{\mathcal{I}}

\newcommand{\gradt}{\nabla\cdot}
\newcommand{\Reals}{\mathbb{R}}
\newcommand{\Cplex}{\mathbb{C}}

\newcommand{\Vecc}[2]{ \left(
	\begin{array}{c}
	#1 \\ #2
	\end{array}
	\right) }

\newcommand{\Matrr}[4]{ \left(
	\begin{array}{cc}
	#1 & #2 \\
	#3 & #4
	\end{array}
	\right) }

\newcommand{\calB}{{\mathcal B}}
\newcommand{\calBB}{{\mathcal B}^{\Box}}
\newcommand{\calK}{\mathcal{K}}
\newcommand{\bfA}{{\mathbf A}}
\newcommand{\bfb}{{\mathbf b}}
\newcommand{\bfr}{{\mathbf r}}
\newcommand{\bfx}{{\mathbf x}}
\newcommand{\bfxex}{{\mathbf x}_{\mbox{\scriptsize $\star$}}}
\newcommand{\Dim}{\mathop{\mathrm{dim\ }}}

\begin{document}
	%%%%%%%%%%%%%%%%%%%%%%%%%%%%%%%%%%%%%%%%%%%

\begin{center}
{\large
{\bf An asynchronous parallel derivative-free algorithm for \\
	handling general constraints}}

	Josh D Griffin \\
	P.O.~Box 969 MS 9159,  Livermore CA 94551 \\
	{\tt jgriffi@sandia.gov} \\
	Tamara G Kolda
\end{center}
We will discuss an asynchronous parallel implementation of a
derivative-free augmented Lagrangian algorithm for handling
general nonlinear constraints recently proposed by Kolda,
Lewis, and Torczon. The method solves a series of linearly
constrained subproblems, seeking to approximately minimize
the augmented Lagrangian which involves the nonlinear
constraints. Each subproblem is solved using a generating
set search algorithm capable of handling degenerate linear
constraints. We use APPSPACK to solve the
linearly-constrained subproblems, enabling the objective and
nonlinear constraint functions to be computed asynchronously
in parallel. A description and theoretical analysis of the
algorithm will be given followed by numerical results.



	%%%%%%%%%%%%%%%%%%%%%%%%%%%%%%%%%%%%%%%%%%%


\end{document}

