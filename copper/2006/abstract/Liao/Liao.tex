\documentclass{report}
\usepackage{amsmath,amssymb}

\def\mathbi#1{\textbf{\em #1}}
\def\BA{\bf{A}}
\def\BB{\bf{B}}
\def\bn{\bf{n}}
\def\CP{\mathcal{P}}
\def\CV{\mathcal{V}}
\def\CI{\mathcal{I}}

\newcommand{\gradt}{\nabla\cdot}
\newcommand{\Reals}{\mathbb{R}}
\newcommand{\Cplex}{\mathbb{C}}

\newcommand{\Vecc}[2]{ \left(
	\begin{array}{c}
	#1 \\ #2
	\end{array}
	\right) }

\newcommand{\Matrr}[4]{ \left(
	\begin{array}{cc}
	#1 & #2 \\
	#3 & #4
	\end{array}
	\right) }

\newcommand{\calB}{{\mathcal B}}
\newcommand{\calBB}{{\mathcal B}^{\Box}}
\newcommand{\calK}{\mathcal{K}}
\newcommand{\bfA}{{\mathbf A}}
\newcommand{\bfb}{{\mathbf b}}
\newcommand{\bfr}{{\mathbf r}}
\newcommand{\bfx}{{\mathbf x}}
\newcommand{\bfxex}{{\mathbf x}_{\mbox{\scriptsize $\star$}}}
\newcommand{\Dim}{\mathop{\mathrm{dim\ }}}

\begin{document}
	%%%%%%%%%%%%%%%%%%%%%%%%%%%%%%%%%%%%%%%%%%%

\begin{center}
{\large
{\bf An iterative projection method for solving large-scale nonlinear \\
eigenproblems with application to next-generation accelerator design}}

	Ben-Shan Liao \\
	Dept.~of Mathematics \\
	University of California, Davis CA 95616 \\
	{\tt liao@math.ucdavis.edu} \\
	Lie-Quan Lee, Zhaojun Bai, Kwok Ko
\end{center}
The emerging needs to solve large-scale nonlinear eigenvalue
problems arising in many engineering applications have come
into notice. More researches have been conducted on
efficient algorithm development and computational theory.
However, the nonlinearity varying greatly from problem to
problem results in a challenging computational task. Instead
of considering arbitrary nonlinear eigenvalue problems, we
consider a certain type of problems for robust and efficient
algorithm developments. This particular type nonlinear
eigenvalue problems consist of a dominated linear and
positive definite pencil and a ``small'' nonlinear component.
A number of applications give rise of nonlinear eigenvalue
problems of such type.  Examples include vibration study of
fluid solid structures and eigencomputation problem from
fiber optic design.

In this talk, a nonlinear eigenvalue problem we particular
interested in is from the finite element analysis of the
resonant frequencies and external Q of a waveguide loaded
cavity, as currently be studied by researchers for
next-generation accelerator design. We study iterative
subspace projection methods, such as nonlinear Arnoldi
method. We focus on the critical stages of algorithms, such
as the choice of initial projection subspace, and the
expansion and the refinement of projection subspace. We
present a notable improvement over the early iterative
projection methods in our case study.


	%%%%%%%%%%%%%%%%%%%%%%%%%%%%%%%%%%%%%%%%%%%


\end{document}

