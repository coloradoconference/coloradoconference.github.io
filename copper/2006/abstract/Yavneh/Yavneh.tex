\documentclass{report}
\usepackage{amsmath,amssymb}

\def\mathbi#1{\textbf{\em #1}}
\def\BA{\bf{A}}
\def\BB{\bf{B}}
\def\bn{\bf{n}}
\def\CP{\mathcal{P}}
\def\CV{\mathcal{V}}
\def\CI{\mathcal{I}}

\newcommand{\gradt}{\nabla\cdot}
\newcommand{\Reals}{\mathbb{R}}
\newcommand{\Cplex}{\mathbb{C}}

\newcommand{\Vecc}[2]{ \left(
	\begin{array}{c}
	#1 \\ #2
	\end{array}
	\right) }

\newcommand{\Matrr}[4]{ \left(
	\begin{array}{cc}
	#1 & #2 \\
	#3 & #4
	\end{array}
	\right) }

\newcommand{\calB}{{\mathcal B}}
\newcommand{\calBB}{{\mathcal B}^{\Box}}
\newcommand{\calK}{\mathcal{K}}
\newcommand{\bfA}{{\mathbf A}}
\newcommand{\bfb}{{\mathbf b}}
\newcommand{\bfr}{{\mathbf r}}
\newcommand{\bfx}{{\mathbf x}}
\newcommand{\bfxex}{{\mathbf x}_{\mbox{\scriptsize $\star$}}}
\newcommand{\Dim}{\mathop{\mathrm{dim\ }}}

\begin{document}
	%%%%%%%%%%%%%%%%%%%%%%%%%%%%%%%%%%%%%%%%%%%

\begin{center}
{\large
{\bf New algorithms for vector quantization}}

	Irad Yavneh \\
	Dept.~of Computer Science \\
	Technion - Israel Institute of Technology, Haifa 32000, Israel \\
	{\tt irad@cs.technion.ac.il} \\
	Yair Koren
\end{center}
Vector quantization is the classical problem of representing
continuum with only a finite number of representatives or
representing an initially rich amount of discrete data with
a lesser amount of representatives. This problem has
numerous applications. The objective of achieving a
quantization with minimal distortion leads to a hard
non-convex optimization problem, typically with many local
minima. The main problem is thus to find an initial
approximation that is close to a ``good'' local minimum.
Once such an approximation is found, the well-known
Lloyd-Max iterative algorithm may be used to converge to the
nearby a local minimum. In this talk we will describe the
problem and present two new approaches to its approximate
solution.



	%%%%%%%%%%%%%%%%%%%%%%%%%%%%%%%%%%%%%%%%%%%


\end{document}

