\documentclass{report}
\usepackage{amsmath,amssymb}

\def\mathbi#1{\textbf{\em #1}}
\def\BA{\bf{A}}
\def\BB{\bf{B}}
\def\bn{\bf{n}}
\def\CP{\mathcal{P}}
\def\CV{\mathcal{V}}
\def\CI{\mathcal{I}}

\newcommand{\gradt}{\nabla\cdot}
\newcommand{\Reals}{\mathbb{R}}
\newcommand{\Cplex}{\mathbb{C}}

\newcommand{\Vecc}[2]{ \left(
	\begin{array}{c}
	#1 \\ #2
	\end{array}
	\right) }

\newcommand{\Matrr}[4]{ \left(
	\begin{array}{cc}
	#1 & #2 \\
	#3 & #4
	\end{array}
	\right) }

\newcommand{\calB}{{\mathcal B}}
\newcommand{\calBB}{{\mathcal B}^{\Box}}
\newcommand{\calK}{\mathcal{K}}
\newcommand{\bfA}{{\mathbf A}}
\newcommand{\bfb}{{\mathbf b}}
\newcommand{\bfr}{{\mathbf r}}
\newcommand{\bfx}{{\mathbf x}}
\newcommand{\bfxex}{{\mathbf x}_{\mbox{\scriptsize $\star$}}}
\newcommand{\Dim}{\mathop{\mathrm{dim\ }}}

\begin{document}
	%%%%%%%%%%%%%%%%%%%%%%%%%%%%%%%%%%%%%%%%%%%

\begin{center}
{\large
{\bf Theory of molecular fluids}}

	B.~Montgomery Pettitt \\
	Dept.~of Chemistry \\
	University of Houston, Houston TX 77204-5003 \\
	{\tt pettitt@uh.edu} \\
\end{center}
A quantitative theory of the structure of molecular fluids
described by atoms or sites has remained elusive in soft
condensed matter theory. Many-body and field theoretic
approaches to the correlations liquids have advanced slowly
since the 1930s. The qualitative and quantitative
inconsistencies of the many-body integral equation theory
for predicting the structure and thermodynamic properties of
model molecular fluids have been understood for some time.

Several means have been proposed to correct these
inconsistencies, many concentrating on the Goldstone
theorem. A formally distinct method for constructing a
diagrammatically proper theory eliminates terms in the
expansion which correspond to unphysical intramolecular
interactions, or so-called bad graphs. Unfortunately, while
certain qualitative advances using the proper theory have
been successful the quantitative results appear to be
uniformly disappointing in comparison to simulation.

We present a new derivation from a topological expansion of
a model for the single atom activity followed by a
topological reduction and low order truncation. This leads
to an approximate numerical value for the new density
coefficient.  The resulting equations give a substantial
improvement over the standard construction as shown with
a series of simple diatomic model simulations.



	%%%%%%%%%%%%%%%%%%%%%%%%%%%%%%%%%%%%%%%%%%%


\end{document}

