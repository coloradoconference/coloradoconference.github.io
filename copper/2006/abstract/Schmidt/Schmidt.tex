\documentclass{report}
\usepackage{amsmath,amssymb}

\def\mathbi#1{\textbf{\em #1}}
\def\BA{\bf{A}}
\def\BB{\bf{B}}
\def\bn{\bf{n}}
\def\CP{\mathcal{P}}
\def\CV{\mathcal{V}}
\def\CI{\mathcal{I}}

\newcommand{\gradt}{\nabla\cdot}
\newcommand{\Reals}{\mathbb{R}}
\newcommand{\Cplex}{\mathbb{C}}

\newcommand{\Vecc}[2]{ \left(
	\begin{array}{c}
	#1 \\ #2
	\end{array}
	\right) }

\newcommand{\Matrr}[4]{ \left(
	\begin{array}{cc}
	#1 & #2 \\
	#3 & #4
	\end{array}
	\right) }

\newcommand{\calB}{{\mathcal B}}
\newcommand{\calBB}{{\mathcal B}^{\Box}}
\newcommand{\calK}{\mathcal{K}}
\newcommand{\bfA}{{\mathbf A}}
\newcommand{\bfb}{{\mathbf b}}
\newcommand{\bfr}{{\mathbf r}}
\newcommand{\bfx}{{\mathbf x}}
\newcommand{\bfxex}{{\mathbf x}_{\mbox{\scriptsize $\star$}}}
\newcommand{\Dim}{\mathop{\mathrm{dim\ }}}

\begin{document}
	%%%%%%%%%%%%%%%%%%%%%%%%%%%%%%%%%%%%%%%%%%%

\begin{center}
{\large
{\bf Phenomenological comparison of different AMG approaches for \\
	the finite element analysis in surgery simulations}}

	Jens G.~Schmidt \\
	C\&C Research Labs, NEC Europe Ltd. \\
	Rathausallee 10 53757, Sankt Augustin, Germany \\
	{\tt schmidt@ccrl-nece.de}
\end{center}
We are dealing with the simulation of maxillo-facial
surgeries, especially with distraction osteogenesis. During
this treatment the surgeon cuts free the upper jaw
(maxilla), which is subsequently relocated into a new
position with a distraction device in the course of several
weeks. Our simulation tool is set up to predict the
displacements of the facial tissues during and after the
pulling process and is based on individual CT images of the
patient's head before treatment. Its purpose is to support
the surgeon in optimizing the treatment plan and avoiding
additional post operative plastic surgeries.

The input
data for the simulation task is generated by adding the
suggested cuts, the geometry of the distraction device and
the suggested forces to the CT data of the patient's head.
From these data we generate a Finite Element mesh of the
head and perform a Finite Element analysis of the
distraction process. In order to achieve sufficient accuracy
we have to resolve most of the geometrical features of the
human head, which leads to a large number of unknowns,
typically several millions. In addition to that the
computational costs are significantly increased by the
finite displacements which can only be properly approximated
by non-linear modeling. The resulting systems of equations
must be solved on high performance computing resources, such
as parallel or vector machines.

Focusing on the
efficiency of the Finit Element simulation, the linear
solver used to solve the arising systems of equations plays
the most crucial role. In our case standard solvers like
Krylov methods or ILU methods fail, as we will show in our
presentation. Therefore we will focus on the use of
Multigrid solvers.

But the complex geometry of the human
head in combination with large jumps of the material
parameters -- Young's modulus jumps about 5 orders of
magnitude between bone and soft tissues -- prevents standard
multigrid approaches to converge at a sufficient rate. In
theory they can be dramatically improved by computing the
``near null space'' of the systems, consisting of the all
quasi-rigid body modes, and treat it separately. But we will
show, that for our problems a basis of this space needs
approximately 10 times the memory of the linear system
itself, which rules out this approach.

In our
presentation we will demonstrate the performance of the only
two solvers we have found to work on our problems so far,
which are BoomerAMG from LLNL's HYPRE package and ML from
Sandia's Trilinos package. We will show the results of our
intensive parameter studies and discuss the extensibility of
our performance results for elasto-mechanical Finite Element
simulations in general.



	%%%%%%%%%%%%%%%%%%%%%%%%%%%%%%%%%%%%%%%%%%%


\end{document}

