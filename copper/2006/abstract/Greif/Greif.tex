\documentclass{report}
\usepackage{amsmath,amssymb}

\def\mathbi#1{\textbf{\em #1}}
\def\BA{\bf{A}}
\def\BB{\bf{B}}
\def\bn{\bf{n}}
\def\CP{\mathcal{P}}
\def\CV{\mathcal{V}}
\def\CI{\mathcal{I}}

\newcommand{\gradt}{\nabla\cdot}
\newcommand{\Reals}{\mathbb{R}}
\newcommand{\Cplex}{\mathbb{C}}

\newcommand{\Vecc}[2]{ \left(
	\begin{array}{c}
	#1 \\ #2
	\end{array}
	\right) }

\newcommand{\Matrr}[4]{ \left(
	\begin{array}{cc}
	#1 & #2 \\
	#3 & #4
	\end{array}
	\right) }

\newcommand{\calB}{{\mathcal B}}
\newcommand{\calBB}{{\mathcal B}^{\Box}}
\newcommand{\calK}{\mathcal{K}}
\newcommand{\bfA}{{\mathbf A}}
\newcommand{\bfb}{{\mathbf b}}
\newcommand{\bfr}{{\mathbf r}}
\newcommand{\bfx}{{\mathbf x}}
\newcommand{\bfxex}{{\mathbf x}_{\mbox{\scriptsize $\star$}}}
\newcommand{\Dim}{\mathop{\mathrm{dim\ }}}

\begin{document}
	%%%%%%%%%%%%%%%%%%%%%%%%%%%%%%%%%%%%%%%%%%%

\begin{center}
{\large
{\bf Preconditioners for the discretized time-harmonic \\
Maxwell equations in mixed form}}

	Chen Greif \\
	University of British Columbia, 2366 Main Mall \\
	Vancouver B.C.  V6T 1Z4 Canada \\
	{\tt greif@cs.ubc.ca} \\
	Dominik Sch\"otzau
\end{center}
We introduce a new preconditioning technique for iteratively
solving linear systems arising from finite element
discretizations of the mixed formulation of the
time-harmonic Maxwell equations. The preconditioners are
block diagonal with positive definite blocks and are based
on discrete augmentation using the scalar Laplacian. They
are motivated by spectral equivalence properties of the
discrete operators. Specifically, we show that augmenting
the curl-curl operator by a discrete grad-div operator,
weighed by the scalar Laplacian, yields almost immediate
convergence when preconditioned MINRES is used. We also show
that if the augmented term is replaced by the vector mass
matrix we still obtain fast convergence. Similar
(operator-independent) algebraic principles can be applied
in general settings and give rise to preconditioners that
work effectively for saddle-point linear systems whose (1,1)
block is highly singular. Analytical observations and
numerical results demonstrate the scalability and the
convergence properties of the proposed approach.



	%%%%%%%%%%%%%%%%%%%%%%%%%%%%%%%%%%%%%%%%%%%


\end{document}

