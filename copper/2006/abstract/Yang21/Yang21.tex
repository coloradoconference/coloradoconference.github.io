\documentclass{report}
\usepackage{amsmath,amssymb}

\def\mathbi#1{\textbf{\em #1}}
\def\BA{\bf{A}}
\def\BB{\bf{B}}
\def\bn{\bf{n}}
\def\CP{\mathcal{P}}
\def\CV{\mathcal{V}}
\def\CI{\mathcal{I}}

\newcommand{\gradt}{\nabla\cdot}
\newcommand{\Reals}{\mathbb{R}}
\newcommand{\Cplex}{\mathbb{C}}

\newcommand{\Vecc}[2]{ \left(
	\begin{array}{c}
	#1 \\ #2
	\end{array}
	\right) }

\newcommand{\Matrr}[4]{ \left(
	\begin{array}{cc}
	#1 & #2 \\
	#3 & #4
	\end{array}
	\right) }

\newcommand{\calB}{{\mathcal B}}
\newcommand{\calBB}{{\mathcal B}^{\Box}}
\newcommand{\calK}{\mathcal{K}}
\newcommand{\bfA}{{\mathbf A}}
\newcommand{\bfb}{{\mathbf b}}
\newcommand{\bfr}{{\mathbf r}}
\newcommand{\bfx}{{\mathbf x}}
\newcommand{\bfxex}{{\mathbf x}_{\mbox{\scriptsize $\star$}}}
\newcommand{\Dim}{\mathop{\mathrm{dim\ }}}

\begin{document}
	%%%%%%%%%%%%%%%%%%%%%%%%%%%%%%%%%%%%%%%%%%%

\begin{center}
{\large
{\bf On parallel algebraic multigrid preconditioners for systems of PDEs}}

	Ulrike Meier Yang \\
	Center for Applied Scientific Computing \\
	Lawrence Livermore National Laboratory \\
	Box 808, L-560, Livermore CA 94551 \\
	{\tt umyang@llnl.gov}
\end{center}
Algebraic multigrid (AMG) is a very efficient, scalable
algorithm for solving large linear systems on unstructured
grids. When solving linear systems derived from systems of
partial differential equations (PDEs) often a different
approach is required than for those derived from a scalar
PDE. There are mainly two approaches, the function approach
(also known as the ``unknown'' approach), and the nodal or
``point'' approach. The function approach defines coarsening
and interpolation separately for each function. The nodal
approach uses AMG in a block manner, where all variables
that correspond to the same grid node are coarsened,
interpolated and relaxed together. While the function
approach is much easier to implement and often more
efficient, there are problems for which this approach is not
sufficient and the more expensive nodal approach is needed.


Several parallel implementations of both approaches using
various coarsening schemes and interpolation operators are
investigated. Advantages and disadvantages of both
approaches are discussed, and numerical results are
presented.



	%%%%%%%%%%%%%%%%%%%%%%%%%%%%%%%%%%%%%%%%%%%


\end{document}

