\documentclass{report}
\usepackage{amsmath,amssymb}

\def\mathbi#1{\textbf{\em #1}}
\def\BA{\bf{A}}
\def\BB{\bf{B}}
\def\bn{\bf{n}}
\def\CP{\mathcal{P}}
\def\CV{\mathcal{V}}
\def\CI{\mathcal{I}}

\newcommand{\gradt}{\nabla\cdot}
\newcommand{\Reals}{\mathbb{R}}
\newcommand{\Cplex}{\mathbb{C}}

\newcommand{\Vecc}[2]{ \left(
	\begin{array}{c}
	#1 \\ #2
	\end{array}
	\right) }

\newcommand{\Matrr}[4]{ \left(
	\begin{array}{cc}
	#1 & #2 \\
	#3 & #4
	\end{array}
	\right) }

\newcommand{\calB}{{\mathcal B}}
\newcommand{\calBB}{{\mathcal B}^{\Box}}
\newcommand{\calK}{\mathcal{K}}
\newcommand{\bfA}{{\mathbf A}}
\newcommand{\bfb}{{\mathbf b}}
\newcommand{\bfr}{{\mathbf r}}
\newcommand{\bfx}{{\mathbf x}}
\newcommand{\bfxex}{{\mathbf x}_{\mbox{\scriptsize $\star$}}}
\newcommand{\Dim}{\mathop{\mathrm{dim\ }}}

\begin{document}
	%%%%%%%%%%%%%%%%%%%%%%%%%%%%%%%%%%%%%%%%%%%

\begin{center}
{\large
{\bf Adaptive mesh refinement: in the presence of discontinuities}}

	Alfonso Limon \\
	School of Mathematical Sciences, Claremont Graduate University \\
	710 N.~College Ave., Claremont CA 91711 \\
	{\tt alfonso.limon@cgu.edu} \\
	Hedley Morris
\end{center}
Classical multiresolution wavelet techniques have been used
successfully to simplify the computation of PDEs by
concentrating resources in places where the solution varies
quickly. However, classical techniques tend to fail near
solution discontinuities, as Gibb's effects contaminate the
wavelet coefficients used to refine the solution. Non-linear
adaptive stencil methods, such as the ENO scheme, can
reconstruct the solution accurately across jumps, but
possess neither the compression capabilities nor the
well-understood stability properties of wavelets. Expanding
on Harten's ideas, we construct an alternative to wavelets,
a multiresolution method that does not suffer from Gibb's
effects and has good compression properties. We will present
this alternative multiresolution method and compare its
performance to other methods by means of several examples.


	%%%%%%%%%%%%%%%%%%%%%%%%%%%%%%%%%%%%%%%%%%%


\end{document}

