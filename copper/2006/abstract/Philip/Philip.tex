\documentclass{report}
\usepackage{amsmath,amssymb}

\def\mathbi#1{\textbf{\em #1}}
\def\BA{\bf{A}}
\def\BB{\bf{B}}
\def\bn{\bf{n}}
\def\CP{\mathcal{P}}
\def\CV{\mathcal{V}}
\def\CI{\mathcal{I}}

\newcommand{\gradt}{\nabla\cdot}
\newcommand{\Reals}{\mathbb{R}}
\newcommand{\Cplex}{\mathbb{C}}

\newcommand{\Vecc}[2]{ \left(
	\begin{array}{c}
	#1 \\ #2
	\end{array}
	\right) }

\newcommand{\Matrr}[4]{ \left(
	\begin{array}{cc}
	#1 & #2 \\
	#3 & #4
	\end{array}
	\right) }

\newcommand{\calB}{{\mathcal B}}
\newcommand{\calBB}{{\mathcal B}^{\Box}}
\newcommand{\calK}{\mathcal{K}}
\newcommand{\bfA}{{\mathbf A}}
\newcommand{\bfb}{{\mathbf b}}
\newcommand{\bfr}{{\mathbf r}}
\newcommand{\bfx}{{\mathbf x}}
\newcommand{\bfxex}{{\mathbf x}_{\mbox{\scriptsize $\star$}}}
\newcommand{\Dim}{\mathop{\mathrm{dim\ }}}

\begin{document}
	%%%%%%%%%%%%%%%%%%%%%%%%%%%%%%%%%%%%%%%%%%%

\begin{center}
{\large
{\bf Resistive magnetohydrodynamics with implicit adaptive mesh refinement}}

	Bobby Philip \\
	MS B256, Computer and Computational Sciences Division \\
	Los Alamos National Laboratory, PO Box 1663, Los Alamos NM 87545 \\
	{\tt bphilip@lanl.gov} \\
	Michael Pernice, Luis Chacon
\end{center}
Implicit adaptive mesh refinement (AMR) is used to simulate
a model resistive magnetohydrodynamics problem. This
challenging multi-scale, multi-physics problem involves a
wide range of length and time scales. AMR is employed to
resolve extremely thin current sheets, essential for an
accurate macroscopic description. Implicit time integration
is used to step over fast Alfven time scales. At each time
step, large-scale systems of nonlinear equations are solved
using Jacobian-free Newton-Krylov methods together with a
physics-based preconditioner. The preconditioner is
implemented using optimal multilevel solvers such as the
Fast Adaptive Composite grid (FAC) method. We will describe
our initial results highlighting various aspects of problem
formulation, optimal preconditioning on AMR grids, and
efforts towards achieving parallelism.



	%%%%%%%%%%%%%%%%%%%%%%%%%%%%%%%%%%%%%%%%%%%


\end{document}

