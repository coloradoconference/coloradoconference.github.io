\documentclass{report}
\usepackage{amsmath,amssymb}

\def\mathbi#1{\textbf{\em #1}}
\def\BA{\bf{A}}
\def\BB{\bf{B}}
\def\bn{\bf{n}}
\def\CP{\mathcal{P}}
\def\CV{\mathcal{V}}
\def\CI{\mathcal{I}}

\newcommand{\gradt}{\nabla\cdot}
\newcommand{\Reals}{\mathbb{R}}
\newcommand{\Cplex}{\mathbb{C}}

\newcommand{\Vecc}[2]{ \left(
	\begin{array}{c}
	#1 \\ #2
	\end{array}
	\right) }

\newcommand{\Matrr}[4]{ \left(
	\begin{array}{cc}
	#1 & #2 \\
	#3 & #4
	\end{array}
	\right) }

\newcommand{\calB}{{\mathcal B}}
\newcommand{\calBB}{{\mathcal B}^{\Box}}
\newcommand{\calK}{\mathcal{K}}
\newcommand{\bfA}{{\mathbf A}}
\newcommand{\bfb}{{\mathbf b}}
\newcommand{\bfr}{{\mathbf r}}
\newcommand{\bfx}{{\mathbf x}}
\newcommand{\bfxex}{{\mathbf x}_{\mbox{\scriptsize $\star$}}}
\newcommand{\Dim}{\mathop{\mathrm{dim\ }}}

\begin{document}
	%%%%%%%%%%%%%%%%%%%%%%%%%%%%%%%%%%%%%%%%%%%

\begin{center}
{\large
{\bf Optimal additive Schwarz preconditioning for minimal residual \\
	methods with Euclidean and energy norms}}

	Daniel B Szyld \\
	Dept.~of Mathematics (038-16) \\
	Temple University, 1805 N Broad Street \\
	Philadelphia PA 19122-6094 \\
	{\tt szyd@temple.edu} \\
	Marcus Sarkis
\end{center}
For the solution of non-symmetric or indefinite linear
systems arising from discretizations of elliptic problems,
two-level additive Schwarz preconditioners are known to be
optimal in the sense that convergence bounds for the
preconditioned problem are independent of the mesh and the
number of subdomains. These bounds are based on some kind of
{\em energy norm}. However, in practice, iterative methods
which minimize the Euclidean norm of the residual are used,
despite the fact that the usual bounds are non-optimal,
i.e., the quantities appearing in the bounds may depend on
the mesh size; see [1].
In this paper,
iterative methods are presented which minimize the same
energy norm in which the optimal Schwarz bounds are derived,
thus maintaining the Schwarz optimality. As a consequence,
bounds for the Euclidean norm minimization are also derived,
thus providing a theoretical justification for the practical
use of Euclidean norm minimization methods preconditioned
with additive Schwarz. Both left and right preconditioners
are considered, and relations between them are derived.
Numerical experiments illustrate the theoretical
developments.

[1] X.-C.~Cai, J.~Zou,
Numer. Linear Algebra Appl. {\bf 9} (2002) 379--397.



	%%%%%%%%%%%%%%%%%%%%%%%%%%%%%%%%%%%%%%%%%%%


\end{document}

