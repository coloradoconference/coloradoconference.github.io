\documentclass{report}
\usepackage{amsmath,amssymb}

\def\mathbi#1{\textbf{\em #1}}
\def\BA{\bf{A}}
\def\BB{\bf{B}}
\def\bn{\bf{n}}
\def\CP{\mathcal{P}}
\def\CV{\mathcal{V}}
\def\CI{\mathcal{I}}

\newcommand{\gradt}{\nabla\cdot}
\newcommand{\Reals}{\mathbb{R}}
\newcommand{\Cplex}{\mathbb{C}}

\newcommand{\Vecc}[2]{ \left(
	\begin{array}{c}
	#1 \\ #2
	\end{array}
	\right) }

\newcommand{\Matrr}[4]{ \left(
	\begin{array}{cc}
	#1 & #2 \\
	#3 & #4
	\end{array}
	\right) }

\newcommand{\calB}{{\mathcal B}}
\newcommand{\calBB}{{\mathcal B}^{\Box}}
\newcommand{\calK}{\mathcal{K}}
\newcommand{\bfA}{{\mathbf A}}
\newcommand{\bfb}{{\mathbf b}}
\newcommand{\bfr}{{\mathbf r}}
\newcommand{\bfx}{{\mathbf x}}
\newcommand{\bfxex}{{\mathbf x}_{\mbox{\scriptsize $\star$}}}
\newcommand{\Dim}{\mathop{\mathrm{dim\ }}}

\begin{document}
	%%%%%%%%%%%%%%%%%%%%%%%%%%%%%%%%%%%%%%%%%%%

\begin{center}
{\large
{\bf Moment-of-fluid interface reconstruction}}

	Vadim Dyadechko \\
	MS B284 Los Alamos National Laboratory \\
	Los Alamos, NM 87545 \\
	{\tt vdyadechko@lanl.gov} \\
	Mikhail Shashkov
\end{center}
Volume-of-fluid~(VoF) methods [2] are widely used in
Eulerian simulations of multi-phase flows with mutable
interface topology. The popularity of VoF methods is
explained by their unique ability to preserves the mass of
each fluid component on the discrete level. The strategy of
VoF methods consists in calculating the interface location
at each discrete moment of time from the volumes of the cell
fractions occupied by different materials. Most VoF methods
use a single linear interface to divide two materials in a
mixed cell ~(Piecewise-Linear Interface Calculation~(PLIC))
[3,4,5]. Once the direction of the interface normal is know,
the location of the interface is uniquely identified by the
volumes of the cells fraction. Unfortunately the interface
normal can not be evaluated without the volume fraction data
from the surrounding cells, which prohibits the resulting
approximation to resolve any interface details smaller than
a characteristic size of the cell cluster involved in
evaluation of the normal.

To overcome this limitation, we
designed a new \emph{mass-conservative} interface
reconstruction method [1], which calculates the interface
based on both \emph{volumes and centroids} of the cell
fractions. This choice of the input data allows to evaluate
the interface normal in a mixed cell \emph{even without the
information from the adjacent elements}. The location of the
linear interface in each mixed cell is determined by
\emph{fitting the centroid of the cell fraction behind the
interface to the reference one}, which leads to
($d\!-\!1$)-variate optimization problem in $\Reals^d$. The
technique proposed, called Moment-of-Fluid~(MoF) interface
reconstruction, results in a \emph{second order accurate}
interface approximation~ (linear interfaces are
reconstructed exactly), has higher resolution, and is shown
to be \emph{more accurate than VoF-PLIC methods}.

We
present a detailed description of MoF interface
reconstruction algorithm in 2D, which includes iterative
procedure for centroid fitting and a new algorithm for
cutting appropriate volume fractions from polygonal cells.


[1] V.~Dyadechko and M.~Shashkov, {\em
Moment-of-fluid interface reconstruction},
Technical Report LA-UR-05-7571, Los Alamos
National Laboratory, Los Alamos, NM, Oct 2005.
{\tt http://math.lanl.gov/~vdyadechko/doc/2005-mof.pdf}

[2] C.~W.~Hirt and B.~D.~Nichols, {\em Volume of
fluid (VOF) method for the dynamics of free boundaries},
J.~Comp. Physics {\bf 39}(1) (1981) 201--25.

[3] J.~E.~Pilliod and E.~G.~Puckett, {\em Second-order
accurate volume-of-fluid algorithms for tracking
material interfaces}, J.~Comp. Physics {\bf 199}(2)
(2004) 465--502.

[4] B.~Swartz, {\em The second-order sharpening
of blurred smooth borders}, Mathematics of Computation,
{\bf 52}(186) (1989) 675--714

[5] D.~L.~Youngs, {\em An interface tracking method for a
3D Eulerian hydrodynamics code}, Technical Report
AWRE/44/92/35, Atomic Weapon Research Establishment,
Aldermaston, Berkshire, UK, Apr 1987.


	%%%%%%%%%%%%%%%%%%%%%%%%%%%%%%%%%%%%%%%%%%%


\end{document}

