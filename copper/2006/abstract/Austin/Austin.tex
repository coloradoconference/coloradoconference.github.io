\documentclass{report}
\usepackage{amsmath,amssymb}

\def\mathbi#1{\textbf{\em #1}}
\def\BA{\bf{A}}
\def\BB{\bf{B}}
\def\bn{\bf{n}}
\def\CP{\mathcal{P}}
\def\CV{\mathcal{V}}
\def\CI{\mathcal{I}}

\newcommand{\gradt}{\nabla\cdot}
\newcommand{\Reals}{\mathbb{R}}
\newcommand{\Cplex}{\mathbb{C}}

\newcommand{\Vecc}[2]{ \left(
	\begin{array}{c}
	#1 \\ #2
	\end{array}
	\right) }

\newcommand{\Matrr}[4]{ \left(
	\begin{array}{cc}
	#1 & #2 \\
	#3 & #4
	\end{array}
	\right) }

\newcommand{\calB}{{\mathcal B}}
\newcommand{\calBB}{{\mathcal B}^{\Box}}
\newcommand{\calK}{\mathcal{K}}
\newcommand{\bfA}{{\mathbf A}}
\newcommand{\bfb}{{\mathbf b}}
\newcommand{\bfr}{{\mathbf r}}
\newcommand{\bfx}{{\mathbf x}}
\newcommand{\bfxex}{{\mathbf x}_{\mbox{\scriptsize $\star$}}}
\newcommand{\Dim}{\mathop{\mathrm{dim\ }}}

\begin{document}
	%%%%%%%%%%%%%%%%%%%%%%%%%%%%%%%%%%%%%%%%%%%

\begin{center}
{\large
{\bf Multilevel homogenization techniques for the cardiac bidomain equations}}

	Travis M Austin \\
	Bioengineering Institute University of Auckland \\
	Private Bag 92019, \quad Auckland, New Zealand \\
	{\tt t.austin@auckland.ac.nz} \\
	Mark L Trew, Andrew J Pullan
\end{center}
The cardiac bidomain equations are a set of nonlinear
partial differential equations that are used to model the
flow of current within cardiac tissue by treating
intracellular and extracellular space as two
interpenetrating domains. In the past 10 years research
groups around the world have been using the bidomain
equations in a variety of sophisticated ways, from modeling
fibrillation in the human heart to understanding how plunge
electrodes affect potential fields during in vitro
experiments on cardiac tissue. The Bioengineering Institute
at the University of Auckland has been a world leader in
imaging cardiac tissue at a microscale resolution, and
discovering specialized features that can be incorporated
into the bidomain model.

In this talk, I will briefly
introduce the Bioengineering Institute's imaging work, and
then focus on how we are using these imaging results in our
modeling framework. This discussion will focus on how we
are using Black Box Multigrid to generate homogenized models
that take into account the discontinuities found at the
mezoscale. Such models allow the effect of the discontinuous
cardiac structures to be seen in the potential fields at a
reduced cost. This work is founded upon the multilevel
upscaling approach of MacLachlan and Moulton (Water
Resources Research, 2005), and begins exploring their ideas
in three-dimensions and in a time-dependent framework.



	%%%%%%%%%%%%%%%%%%%%%%%%%%%%%%%%%%%%%%%%%%%


\end{document}

