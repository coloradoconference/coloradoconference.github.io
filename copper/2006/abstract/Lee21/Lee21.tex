\documentclass{report}
\usepackage{amsmath,amssymb}

\def\mathbi#1{\textbf{\em #1}}
\def\BA{\bf{A}}
\def\BB{\bf{B}}
\def\bn{\bf{n}}
\def\CP{\mathcal{P}}
\def\CV{\mathcal{V}}
\def\CI{\mathcal{I}}

\newcommand{\gradt}{\nabla\cdot}
\newcommand{\Reals}{\mathbb{R}}
\newcommand{\Cplex}{\mathbb{C}}

\newcommand{\Vecc}[2]{ \left(
	\begin{array}{c}
	#1 \\ #2
	\end{array}
	\right) }

\newcommand{\Matrr}[4]{ \left(
	\begin{array}{cc}
	#1 & #2 \\
	#3 & #4
	\end{array}
	\right) }

\newcommand{\calB}{{\mathcal B}}
\newcommand{\calBB}{{\mathcal B}^{\Box}}
\newcommand{\calK}{\mathcal{K}}
\newcommand{\bfA}{{\mathbf A}}
\newcommand{\bfb}{{\mathbf b}}
\newcommand{\bfr}{{\mathbf r}}
\newcommand{\bfx}{{\mathbf x}}
\newcommand{\bfxex}{{\mathbf x}_{\mbox{\scriptsize $\star$}}}
\newcommand{\Dim}{\mathop{\mathrm{dim\ }}}

\begin{document}
	%%%%%%%%%%%%%%%%%%%%%%%%%%%%%%%%%%%%%%%%%%%

\begin{center}
{\large
{\bf A novel algebraic multigrid-based approach for solving \\
	Maxwell's equations}}

	Barry Lee \\
	CASC L-561 LLNL, P.O.~Box 808 \\
	Livermore CA 94551-0909 \\
	{\tt lee123@llnl.gov} \\
	Charles Tong
\end{center}
This talk presents a new algebraic multigrid-based method
for solving the curl-curl formulation of Maxwell's equations
discretized with edge elements. The ultimate goal of this
approach is two-fold. The first is to produce a
multiple-coarsening multigrid method with two approximately
decoupled hierarchies branching off at the initial coarse
level, one resolving the divergence-free error and the other
resolving the curl-free error, i.e., a multigrid method that
couples only on the finest level and mimics a Helmholtz
decomposition on the coarse levels. The second consideration
is to produce the hierarchies using a non-agglomerate
coarsening scheme.

To roughly attain this two-fold goal,
this new approach constructs the first coarse level using
topological properties of the mesh. In particular, a
discrete orthogonal decomposition of the finest edges is
constructed by dividing them into two sets, those forming a
minimum spanning tree and the complement set forming the
cotree. Since the cotree edges do not form closed cycles,
these edges cannot support ``complete'' near-nullspace
gradient functions of the curl-curl Maxwell operator. Thus,
partitioning the finest level matrix using this tree/cotree
decomposition, the cotree-cotree submatrix does not have a
large near-nullspace. Hence, a non-agglomerate algebraic
multigrid method (AMG) that can handle strong positive and
negative off-diagonal elements can be applied to this
submatrix. This cotree operator is related to the initial
coarse-grid operator for the divergence-free hierarchy.

The
curl-free hierarchy is generated by a nodal Poisson operator
obtained by restricting the Maxwell operator to the space of
gradients. Unfortunately, because the cotree operator itself
is not the initial coarse-grid operator for the
divergence-free hierarchy, the multiple-coarsening scheme
composed of the cotree matrix and its coarsening, and the
nodal Poisson operator and its coarsening does not give an
overall efficient method. Algebraically, the tree/cotree
coupling on the finest level, which is accentuated through
smooth divergence-free error, is too strong to be handled
sufficiently only on the finest level. In this new approach,
these couplings are handled using oblique/orthogonal
projections onto the space of discretely divergence-free
vectors. In the multigrid viewpoint, the initial coarsening
from the target fine level to the divergence-free subspace
is obtained using these oblique/orthogonal
restriction/interpolation operators in the Galerkin
coarsening procedure. The resulting coarse grid operator can
be preconditioned with a product operator involving a
cotree-cotree submatrix and a topological matrix related to
a discrete Poisson operator.

The overall iteration is then a
multigrid cycle for a nodal Poisson operator (the curl-free
branch) coupled on the finest grid to a preconditioned
Krylov iteration for the fine grid Maxwell operator
restricted to the subspace of discretely divergence-free
vectors. Numerical results are presented to verify the
effectiveness and difficulties of this new approach for
solving the curl-curl formulation of Maxwell's equations.



	%%%%%%%%%%%%%%%%%%%%%%%%%%%%%%%%%%%%%%%%%%%


\end{document}

