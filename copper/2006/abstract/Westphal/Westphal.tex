\documentclass{report}
\usepackage{amsmath,amssymb}

\def\mathbi#1{\textbf{\em #1}}
\def\BA{\bf{A}}
\def\BB{\bf{B}}
\def\bn{\bf{n}}
\def\CP{\mathcal{P}}
\def\CV{\mathcal{V}}
\def\CI{\mathcal{I}}

\newcommand{\gradt}{\nabla\cdot}
\newcommand{\Reals}{\mathbb{R}}
\newcommand{\Cplex}{\mathbb{C}}

\newcommand{\Vecc}[2]{ \left(
	\begin{array}{c}
	#1 \\ #2
	\end{array}
	\right) }

\newcommand{\Matrr}[4]{ \left(
	\begin{array}{cc}
	#1 & #2 \\
	#3 & #4
	\end{array}
	\right) }

\newcommand{\calB}{{\mathcal B}}
\newcommand{\calBB}{{\mathcal B}^{\Box}}
\newcommand{\calK}{\mathcal{K}}
\newcommand{\bfA}{{\mathbf A}}
\newcommand{\bfb}{{\mathbf b}}
\newcommand{\bfr}{{\mathbf r}}
\newcommand{\bfx}{{\mathbf x}}
\newcommand{\bfxex}{{\mathbf x}_{\mbox{\scriptsize $\star$}}}
\newcommand{\Dim}{\mathop{\mathrm{dim\ }}}

\begin{document}
	%%%%%%%%%%%%%%%%%%%%%%%%%%%%%%%%%%%%%%%%%%%

\begin{center}
{\large
{\bf A least-squares finite element method for viscoelastic flow}}

	Chad Westphal \\
	Dept.~of Mathematics, Wabash College \\
	300 W.~Wabash Ave., Crawfordsville IN 47933 \\
	{\tt westphac@wabash.edu}
\end{center}
The equations describing viscoelastic fluid flow are
inherently nonlinear and pose a continuing challenge to most
numerical approximation methods. Even the addition of a
small component of elastic behavior to Newtonian fluid can
introduce instabilities. In this talk we present progress
toward a multilevel least-squares finite element method for
steady viscoelastic fluid flow of Oldroyd-B type. Our
approach is to combine an outer iteration consisting of
linearization steps on a nested sequence of grids with an
inner iteration of a least-squares finite element
discretization and algebraic multigrid linear solver. One of
the challenges arising from this system is in treating the
convective nature of the constitutive equation. We discuss
an analogous, but simplified, problem that illustrates the
difficulties, and suggest a strategy for least-squares
discretizations to treat equations with strong convective
terms.



	%%%%%%%%%%%%%%%%%%%%%%%%%%%%%%%%%%%%%%%%%%%


\end{document}

