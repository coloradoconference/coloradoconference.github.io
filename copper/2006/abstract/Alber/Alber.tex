\documentclass{report}
\usepackage{amsmath,amssymb}

\def\mathbi#1{\textbf{\em #1}}
\def\BA{\bf{A}}
\def\BB{\bf{B}}
\def\bn{\bf{n}}
\def\CP{\mathcal{P}}
\def\CV{\mathcal{V}}
\def\CI{\mathcal{I}}

\newcommand{\gradt}{\nabla\cdot}
\newcommand{\Reals}{\mathbb{R}}
\newcommand{\Cplex}{\mathbb{C}}

\newcommand{\Vecc}[2]{ \left(
	\begin{array}{c}
	#1 \\ #2
	\end{array}
	\right) }

\newcommand{\Matrr}[4]{ \left(
	\begin{array}{cc}
	#1 & #2 \\
	#3 & #4
	\end{array}
	\right) }

\newcommand{\calB}{{\mathcal B}}
\newcommand{\calBB}{{\mathcal B}^{\Box}}
\newcommand{\calK}{\mathcal{K}}
\newcommand{\bfA}{{\mathbf A}}
\newcommand{\bfb}{{\mathbf b}}
\newcommand{\bfr}{{\mathbf r}}
\newcommand{\bfx}{{\mathbf x}}
\newcommand{\bfxex}{{\mathbf x}_{\mbox{\scriptsize $\star$}}}
\newcommand{\Dim}{\mathop{\mathrm{dim\ }}}

\begin{document}
	%%%%%%%%%%%%%%%%%%%%%%%%%%%%%%%%%%%%%%%%%%%


\begin{center}
{\large
{\bf Parallel coarse grid selection strategies}}

	David Alber \\
	Siebel Center for Computer Science \\
	University of Illinois at Urbana-Champaign \\
	201 North Goodwin Avenue, Urbana IL 61801 \\
	{\tt alber@uiuc.edu} \\
	Luke Olson
\end{center}
Traditional coarse grid selection algorithms for algebraic
multigrid use a strength of connection measure to select
coarse degrees of freedom. The strength of connection is a
heuristic used to determine the influences between degrees
of freedom in M-matrices. Coarsening algorithms using a
strength of connection are known to select ineffective
coarse grids for some cases where the operator is not an
M-matrix. Additionally, these methods do not consider other
information such as the smoother to be used in the solve
phase. Alternatively, compatible relaxation selects coarse
grids without explicitly using a strength of connection
measure. Instead, a smoother is applied to identify degrees
of freedom where the smooth error is large. This information
is then used to select the coarse grid. Recent work on
compatible relaxation has produced viable serial
implementations and useful theoretical results. The goal of
this work is to produce effective and efficient parallel
compatible relaxation methods. In this talk, parallel
compatible relaxation implementations will be introduced and
discussed, along with results from experiments on both
structured and unstructured problems.



	%%%%%%%%%%%%%%%%%%%%%%%%%%%%%%%%%%%%%%%%%%%


\end{document}

