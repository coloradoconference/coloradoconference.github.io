\documentclass{report}
\usepackage{amsmath,amssymb}

\def\mathbi#1{\textbf{\em #1}}
\def\BA{\bf{A}}
\def\BB{\bf{B}}
\def\bn{\bf{n}}
\def\CP{\mathcal{P}}
\def\CV{\mathcal{V}}
\def\CI{\mathcal{I}}

\newcommand{\gradt}{\nabla\cdot}
\newcommand{\Reals}{\mathbb{R}}
\newcommand{\Cplex}{\mathbb{C}}

\newcommand{\Vecc}[2]{ \left(
	\begin{array}{c}
	#1 \\ #2
	\end{array}
	\right) }

\newcommand{\Matrr}[4]{ \left(
	\begin{array}{cc}
	#1 & #2 \\
	#3 & #4
	\end{array}
	\right) }

\newcommand{\calB}{{\mathcal B}}
\newcommand{\calBB}{{\mathcal B}^{\Box}}
\newcommand{\calK}{\mathcal{K}}
\newcommand{\bfA}{{\mathbf A}}
\newcommand{\bfb}{{\mathbf b}}
\newcommand{\bfr}{{\mathbf r}}
\newcommand{\bfx}{{\mathbf x}}
\newcommand{\bfxex}{{\mathbf x}_{\mbox{\scriptsize $\star$}}}
\newcommand{\Dim}{\mathop{\mathrm{dim\ }}}

\begin{document}
	%%%%%%%%%%%%%%%%%%%%%%%%%%%%%%%%%%%%%%%%%%%

\begin{center}
{\large
{\bf Approximate finite-differences in matrix-free Newton-Krylov methods}}

	Homer Walker \\
	Mathematical Sciences Department, Worcester Polytechnic Institute \\
	100 Institute Road, Worcester MA 01609-2280 \\
	{\tt walker@wpi.edu} \\
	Peter N.~Brown, Rebecca D.~Wasyk, Carol S.~Woodward
\end{center}
Newton-Krylov methods are often implemented in
``matrix-free'' form, in which the Jacobian-vector products
required by the Krylov solver are approximated by finite
differences. We consider using approximate function values
in these finite differences. We first formulate a
finite-difference Arnoldi process that uses approximate
function values and give backward-error results for it. We
then outline a Newton--Krylov method that uses an
implementation of the GMRES or Arnoldi method based on this
process and develop a local convergence analysis for it,
giving sufficient conditions on the approximate function
values for desirable local convergence properties to hold.
We conclude with numerical experiments involving particular
function-value approximations suitable for nonlinear
diffusion problems. For this case, conditions are given for
meeting the convergence assumptions for both lagging and
linearizing the nonlinearity in the function evaluation.



	%%%%%%%%%%%%%%%%%%%%%%%%%%%%%%%%%%%%%%%%%%%


\end{document}

