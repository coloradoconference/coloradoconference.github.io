\documentclass{report}
\usepackage{amsmath,amssymb}

\def\mathbi#1{\textbf{\em #1}}
\def\BA{\bf{A}}
\def\BB{\bf{B}}
\def\bn{\bf{n}}
\def\CP{\mathcal{P}}
\def\CV{\mathcal{V}}
\def\CI{\mathcal{I}}

\newcommand{\gradt}{\nabla\cdot}
\newcommand{\Reals}{\mathbb{R}}
\newcommand{\Cplex}{\mathbb{C}}

\newcommand{\Vecc}[2]{ \left(
	\begin{array}{c}
	#1 \\ #2
	\end{array}
	\right) }

\newcommand{\Matrr}[4]{ \left(
	\begin{array}{cc}
	#1 & #2 \\
	#3 & #4
	\end{array}
	\right) }

\newcommand{\calB}{{\mathcal B}}
\newcommand{\calBB}{{\mathcal B}^{\Box}}
\newcommand{\calK}{\mathcal{K}}
\newcommand{\bfA}{{\mathbf A}}
\newcommand{\bfb}{{\mathbf b}}
\newcommand{\bfr}{{\mathbf r}}
\newcommand{\bfx}{{\mathbf x}}
\newcommand{\bfxex}{{\mathbf x}_{\mbox{\scriptsize $\star$}}}
\newcommand{\Dim}{\mathop{\mathrm{dim\ }}}

\begin{document}
	%%%%%%%%%%%%%%%%%%%%%%%%%%%%%%%%%%%%%%%%%%%

\begin{center}
{\large
{\bf Incomplete LU preconditioning enhancement strategies for sparse matrices}}

	Eun-Joo Lee \\
	Laboratory for High Performance Scientific Computing
		\& Computer Simulation \\
	Dept. of Computer Science, University of Kentucky \\
	Lexington KY 40506-00046 \\
	{\tt elee3@csr.uky.edu}
	\\ Jun Zhang
\end{center}
Several preconditioning enhancement strategies for improving
inaccurate preconditioners produced by the incomplete LU
factorizations of sparse matrices are presented.  The strategies
employ the elements that are dropped during the incomplete LU
factorization and utilize them in different ways by separate
algorithms.

 The first strategy (error compensation) applies the
dropped elements to the lower and upper parts of the LU
factorization to computer a new error compensated LU factorization.
Another strategy (inner-outer iteration), which is a variant of the
incomplete LU factorization, embeds the dropped elements in its
iteration process.

Experimental results show that the presented
enhancement strategies improve the accuracy of the incomplete LU
factorization when the initial factorizations found to be
inaccurate.  Furthermore, the convergence cost of the preconditioned
Krylov subspace methods is reduced on solving the original sparse
matrices with the proposed strategies.


	%%%%%%%%%%%%%%%%%%%%%%%%%%%%%%%%%%%%%%%%%%%


\end{document}

