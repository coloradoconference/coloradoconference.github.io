\documentclass{report}
\usepackage{amsmath,amssymb}

\def\mathbi#1{\textbf{\em #1}}
\def\BA{\bf{A}}
\def\BB{\bf{B}}
\def\bn{\bf{n}}
\def\CP{\mathcal{P}}
\def\CV{\mathcal{V}}
\def\CI{\mathcal{I}}

\newcommand{\gradt}{\nabla\cdot}
\newcommand{\Reals}{\mathbb{R}}
\newcommand{\Cplex}{\mathbb{C}}

\newcommand{\Vecc}[2]{ \left(
	\begin{array}{c}
	#1 \\ #2
	\end{array}
	\right) }

\newcommand{\Matrr}[4]{ \left(
	\begin{array}{cc}
	#1 & #2 \\
	#3 & #4
	\end{array}
	\right) }

\newcommand{\calB}{{\mathcal B}}
\newcommand{\calBB}{{\mathcal B}^{\Box}}
\newcommand{\calK}{\mathcal{K}}
\newcommand{\bfA}{{\mathbf A}}
\newcommand{\bfb}{{\mathbf b}}
\newcommand{\bfr}{{\mathbf r}}
\newcommand{\bfx}{{\mathbf x}}
\newcommand{\bfxex}{{\mathbf x}_{\mbox{\scriptsize $\star$}}}
\newcommand{\Dim}{\mathop{\mathrm{dim\ }}}

\begin{document}
	%%%%%%%%%%%%%%%%%%%%%%%%%%%%%%%%%%%%%%%%%%%

\begin{center}
{\large
{\bf New iterative eigensolvers and preconditioners for electronic structure calculations in nano and materials science}}

	Andrew Canning \\
	LBNL MS-50F, One Cyclotron Road, Berkeley CA 94720 \\
	{\tt acanning@lbl.gov} \\
	Julien Langou, Christof Voemel, Osni Marques \\
	Stanimire Tomov, Gabriel Bester, Lin-Wang Wang
\end{center}
Density functional based electronic structure calculations
have become the most heavily used approach in materials
science to calculate materials properties with the accuracy
of a full quantum mechanical treatment of the electrons.
This approach results in a single particle form of the
Schr\"{o}dinger equation which is a non-linear eigenfunction
problem. The standard self-consistent solution of this
problem involves solving for the lowest eigenpairs
corresponding to the electrons in the system. In
non-self-consistent formulations the problem becomes one of
determining the interior eigenpairs corresponding to the
electrons of interest which are typically around the gap in
the eigenvalue spectrum for non-metallic systems.

In this
talk I will present results for new iterative eigensolvers
based on conjugate gradients (the LOBPCG method) in the
context of plane wave electronic structure calculations.
This new method gives significant speedup over existing
conjugate gradient methods used in electronic structure
calculations.
I will also present results for a new
preconditioner based on first solving the bulk structure
corresponding to a given nanosystem and then using that as a
preconditioner to solve the nanosystem.

This new
preconditioner gives significant speedup compared to
previously used preconditioners based on the diagonal of the
matrix. These new methods will be demonstrated for CdSe
quantum dots as well as quantum wires constructed from
layers of InP and InAs.  This work was supported by the
Director, Office of Advanced Scientific Computing Research,
Division of Mathematical, Information and Computational
Sciences of the U.S. Dept.~of Energy and the Laboratory
Directed Research and Development Program of Lawrence
Berkeley National Laboratory under contract number
DE-AC03-76SF00098.



	%%%%%%%%%%%%%%%%%%%%%%%%%%%%%%%%%%%%%%%%%%%


\end{document}

