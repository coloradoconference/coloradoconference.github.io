\documentclass{report}
\usepackage{amsmath,amssymb}

\def\mathbi#1{\textbf{\em #1}}
\def\BA{\bf{A}}
\def\BB{\bf{B}}
\def\bn{\bf{n}}
\def\CP{\mathcal{P}}
\def\CV{\mathcal{V}}
\def\CI{\mathcal{I}}

\newcommand{\gradt}{\nabla\cdot}
\newcommand{\Reals}{\mathbb{R}}
\newcommand{\Cplex}{\mathbb{C}}

\newcommand{\Vecc}[2]{ \left(
	\begin{array}{c}
	#1 \\ #2
	\end{array}
	\right) }

\newcommand{\Matrr}[4]{ \left(
	\begin{array}{cc}
	#1 & #2 \\
	#3 & #4
	\end{array}
	\right) }

\newcommand{\calB}{{\mathcal B}}
\newcommand{\calBB}{{\mathcal B}^{\Box}}
\newcommand{\calK}{\mathcal{K}}
\newcommand{\bfA}{{\mathbf A}}
\newcommand{\bfb}{{\mathbf b}}
\newcommand{\bfr}{{\mathbf r}}
\newcommand{\bfx}{{\mathbf x}}
\newcommand{\bfxex}{{\mathbf x}_{\mbox{\scriptsize $\star$}}}
\newcommand{\Dim}{\mathop{\mathrm{dim\ }}}

\begin{document}
	%%%%%%%%%%%%%%%%%%%%%%%%%%%%%%%%%%%%%%%%%%%

\begin{center}
{\large
{\bf Inexact solves in Krylov-based model reduction of large-scale systems}}

	Serkan Gugercin \\
	Dept.~of Mathematics, Virginia Tech. \\
	460 McBryde Hall, Blacksburg VA 24061-0123 \\
	{\tt gugercin@math.vt.edu}
\end{center}
Dynamical systems are the basic framework for modeling and
control of an enormous variety of complex systems. Direct
numerical simulation of the associated models has been one
of the few available means when goals include accurate
prediction or control of complex physical phenomena.
However, the ever increasing need for improved accuracy
requires the inclusion of ever more detail in the modeling
stage, leading inevitably to ever larger-scale, ever more
complex dynamical systems.

Simulations in such
large-scale settings can be overwhelming and make
unmanageably large demands on computational resources, which
is the main motivation for model reduction. The goal of
model reduction is to produce a much lower dimensional
system having the same input/output characteristics as the
original. Recently, Krylov-based methods have emerged as
promising candidates for reduction of large-scale dynamical
systems.

The main cost in Krylov-based model reduction
is due to solving a set of linear systems of the form
$(s_o \mathbf{I}_n - \mathbf{A}) \mathbf{v} = \mathbf{b}$
where
$\mathbf{A}$ is an ${n \times n}$ matrix, $\mathbf{b}$ is an
$n$-dimensional vector, $s_0$ is a complex number called the
{\it interpolation point} and $\mathbf{I}_n$ is the identity
matrix of size $n$. Since the need for more detail and
accuracy in the modeling stage causes the system dimension,
$n$, to reach levels on the order of millions, direct
solvers for the linear system
$(s_o \mathbf{I}_n - \mathbf{A}) \mathbf{v} = \mathbf{b}$
are no longer feasible;
hence inexact solves need to be employed in Krylov-based
model reduction.

In this talk, we investigate the use of
inexact solves in a Krylov-based model reduction setting and
present the resulting perturbation effects on the underlying
model reduction problem. We show that for a \emph{good}
selection of interpolation points, Krylov-based model
reduction is robust with respect to the perturbations due to
inexact solves. On the other hand, when the interpolation
points are \emph{poorly} selected, these perturbations are
magnified through the model reduction process. We also
examine stopping criteria, effective preconditioning, and
restarting techniques in the particular context of model
reduction. Finally, we incorporate inexact solves for the
Krylov-based optimal ${\cal H}_2$ approximation. The result
is an effective optimal model reduction algorithm applicable
in realistic large-scale settings.



	%%%%%%%%%%%%%%%%%%%%%%%%%%%%%%%%%%%%%%%%%%%


\end{document}

