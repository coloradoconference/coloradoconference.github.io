\documentclass{report}
\usepackage{amsmath,amssymb}

\def\mathbi#1{\textbf{\em #1}}
\def\BA{\bf{A}}
\def\BB{\bf{B}}
\def\bn{\bf{n}}
\def\CP{\mathcal{P}}
\def\CV{\mathcal{V}}
\def\CI{\mathcal{I}}

\newcommand{\gradt}{\nabla\cdot}
\newcommand{\Reals}{\mathbb{R}}
\newcommand{\Cplex}{\mathbb{C}}

\newcommand{\Vecc}[2]{ \left(
	\begin{array}{c}
	#1 \\ #2
	\end{array}
	\right) }

\newcommand{\Matrr}[4]{ \left(
	\begin{array}{cc}
	#1 & #2 \\
	#3 & #4
	\end{array}
	\right) }

\newcommand{\calB}{{\mathcal B}}
\newcommand{\calBB}{{\mathcal B}^{\Box}}
\newcommand{\calK}{\mathcal{K}}
\newcommand{\bfA}{{\mathbf A}}
\newcommand{\bfb}{{\mathbf b}}
\newcommand{\bfr}{{\mathbf r}}
\newcommand{\bfx}{{\mathbf x}}
\newcommand{\bfxex}{{\mathbf x}_{\mbox{\scriptsize $\star$}}}
\newcommand{\Dim}{\mathop{\mathrm{dim\ }}}

\begin{document}
	%%%%%%%%%%%%%%%%%%%%%%%%%%%%%%%%%%%%%%%%%%%

\begin{center}
{\large
{\bf The role of optimization in the validation \& verification, \\
calibration \& validation processes}}

	Genetha Gray \\
	Sandia National Labs, P.O.~Box 969, MS 9159 \\
	Livermore CA 94551-0969 \\
	{\tt gagray@sandia.gov} \\
	Monica Martinez-Canales, Chuck Hembree
\end{center}
A comprehensive study of many of the complex systems in
science and engineering may demand physical experimentation.
However, in many cases, physical experiments can be
prohibitively expensive or even impossible to perform.
Fortunately, the behavior of many of these systems can be
imitated by computer models, and thus, computer simulation
may be a viable alternative or augmentation.

 The
inclusion of computer simulations does introduce many new
challenges. For example, code verification should be used to
confirm that the underlying equations are being solved
correctly. Calibration or parameter estimation must be
incorporated to update the computational model in response
to experimental data. In addition, validation is an
important process that can answer questions of correctness
of the equations and models for the physics being modeled
and the application being studied. Moreover, validation
metrics must be carefully chosen in order to explicitly
compare experimental and computational results and quantify
the uncertainties in these comparisons. Overall, the
validation and verification verification, calibration, and
validation processes for computational experimentation can
provide the best estimates of what can happen and the
likelihood of it happening when uncertainties are taken into
account.

 In this talk, we will focus on the validation
calibration under uncertainty process. In particular, we
will focus on the problem of electrical simulations and
elucidate the role of optimization in the parameter
extraction problem. We will describe the process, an example
problem and our results, and how the inclusion of
sensitivity analysis can improve the optimization procedure.



	%%%%%%%%%%%%%%%%%%%%%%%%%%%%%%%%%%%%%%%%%%%


\end{document}

