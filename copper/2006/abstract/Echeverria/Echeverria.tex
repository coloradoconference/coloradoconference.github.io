\documentclass{report}
\usepackage{amsmath,amssymb}

\def\mathbi#1{\textbf{\em #1}}
\def\BA{\bf{A}}
\def\BB{\bf{B}}
\def\bn{\bf{n}}
\def\CP{\mathcal{P}}
\def\CV{\mathcal{V}}
\def\CI{\mathcal{I}}

\newcommand{\gradt}{\nabla\cdot}
\newcommand{\Reals}{\mathbb{R}}
\newcommand{\Cplex}{\mathbb{C}}

\newcommand{\Vecc}[2]{ \left(
	\begin{array}{c}
	#1 \\ #2
	\end{array}
	\right) }

\newcommand{\Matrr}[4]{ \left(
	\begin{array}{cc}
	#1 & #2 \\
	#3 & #4
	\end{array}
	\right) }

\newcommand{\calB}{{\mathcal B}}
\newcommand{\calBB}{{\mathcal B}^{\Box}}
\newcommand{\calK}{\mathcal{K}}
\newcommand{\bfA}{{\mathbf A}}
\newcommand{\bfb}{{\mathbf b}}
\newcommand{\bfr}{{\mathbf r}}
\newcommand{\bfx}{{\mathbf x}}
\newcommand{\bfxex}{{\mathbf x}_{\mbox{\scriptsize $\star$}}}
\newcommand{\Dim}{\mathop{\mathrm{dim\ }}}

\begin{document}
	%%%%%%%%%%%%%%%%%%%%%%%%%%%%%%%%%%%%%%%%%%%

\begin{center}
{\large
{\bf On the manifold-mapping optimization technique}}

	David Echeverr\'{i}a \\
	Centre for Mathematics and Computer Science (CWI)\\
	Kruislaan 413, NL-1098 SJ Amsterdam, The Netherlands \\
	{\tt d.echeverria@cwi.nl} \\
	Pieter W.~Hemker, {\tt p.w.hemker@cwi.nl}
\end{center}
Optimization problems in practice
often need cost-function evaluations that are very expensive
to compute. Examples are, e.g., optimal design problems
based on complex finite element simulations. As a
consequence, many optimizations may require very long
computing times. The space-mapping (SM) technique
[1,2] was developed as an alternative
in these situations.

In SM terminology, the accurate but
expensive-to-evaluate models are called {\it fine} models,
${\mathbf f}:X \subset \Reals^n \to \Reals^m$.
The SM method also
needs a second, simpler and cheaper and computationally
faster model, the {\it coarse} model,
${\mathbf c}:Z \subset \Reals^n \to \Reals^m$,
in order to speed-up the optimization
process. The key element in this technique is a
right-preconditioning for the coarse model, known as the
{\it SM function}
${\mathbf p}:X \to Z$, that aligns the two
model responses.
The function ${\mathbf c}({\mathbf p}({\mathbf x}))$
corrects the coarse model and
can be used as a surrogate for the fine
model in the accurate optimization. In most cases the SM
function is much simpler than the fine model, in the sense
that it is easier to approximate. This fact endows the SM
technique with its well-reported efficiency. However, it
does not always converge to the right solution.


Defect-correct theory [3] helps to see that, in order
to achieve the accurate optimum, the SM function is
generally insufficient and also left-preconditioning is
needed.  In [4] we introduce the mapping
${\mathbf s}:{\mathbf c}(Z) \to {\mathbf f}(X)$
and the associated
manifold-mapping (MM) algorithm. MM employs
${\mathbf s}({\mathbf c}({\bar{{\mathbf p}}}({\mathbf x})))$
as the fine model surrogate. Here, the
function ${\bar{{\mathbf p}}}:X \to Z$
is not the above SM function but an
arbitrary simple bijection, often the identity.
The MM algorithm is as efficient as SM but converges
to the accurate optimal solution [4,5].

In the first part of the presentation the MM algorithm will
be briefly introduced and a proof of convergence will be
given. The use of more than two models (multi-level
approach) and the possibility of having a coarse model with
a different dimension than the fine one
($X \subset \Reals^{n_{\mathbf f}}$
and $Z \subset \Reals^{n_{\mathbf c}}$ with
$n_{\mathbf f} \neq n_{\mathbf c}$)
will be the issues dealt with in the second part of
the talk.

[1]
J.~W.~Bandler, R.~M.~Biernacki,
C.~H.~Chen, P.~A.~Grobelny and R.~H.~Hemmers,
{\em Space
Mapping Technique for Electromagnetic Optimization},
IEEE Trans. on Microwave Theory and Techniques,
{\bf 42}(12) (1994) 2536--2544.

[2]
J.~W.~Bandler, Q.~S.~Cheng, S.~A.~Dakroury, A.~S.~Mohamed,
M.~H.~Bakr, K.~Madsen and J.~S{\o}ndergaard,
{\em Space Mapping: The State of the Art},
IEEE Trans. on
Microwave Theory and Techniques 52({\bf 1}) (2004)
337--361.

[3]
K.~B{\"o}hmer and P.~W.~Hemker and H.~J.~Stetter,
{\em Defect Correction Methods: Theory and
Applications}, The defect correction approach,
Computing Suppl. {\bf 5} 1--32,
K.~B{\"o}hmer and H.~J.~Stetter ed.,
Springer-Verlag, Berlin, Heidelberg, New York, Tokyo, 1984.

[4]
D.~Echeverr\'{\i}a, and P.~W.~Hemker,
{\em Space mapping and defect correction},
Comp. Methods in Appl. Math. {\bf 5}(2) (2005) 107--136.

[5]
D.~Echeverr\'{i}a, D.~Lahaye, L.~Encica,
E.~A.~Lomonova, P.~W.~Hemker and A.~J.~A.~Vandenput,
{\em Manifold-Mapping Optimization Applied to Linear
Actuator Design},
accepted for publication,
IEEE Transactions on Magnetics, 2006.



	%%%%%%%%%%%%%%%%%%%%%%%%%%%%%%%%%%%%%%%%%%%


\end{document}

