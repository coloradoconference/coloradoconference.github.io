\documentclass{report}
\usepackage{amsmath,amssymb}

\def\mathbi#1{\textbf{\em #1}}
\def\BA{\bf{A}}
\def\BB{\bf{B}}
\def\bn{\bf{n}}
\def\CP{\mathcal{P}}
\def\CV{\mathcal{V}}
\def\CI{\mathcal{I}}

\newcommand{\gradt}{\nabla\cdot}
\newcommand{\Reals}{\mathbb{R}}
\newcommand{\Cplex}{\mathbb{C}}

\newcommand{\Vecc}[2]{ \left(
	\begin{array}{c}
	#1 \\ #2
	\end{array}
	\right) }

\newcommand{\Matrr}[4]{ \left(
	\begin{array}{cc}
	#1 & #2 \\
	#3 & #4
	\end{array}
	\right) }

\newcommand{\calB}{{\mathcal B}}
\newcommand{\calBB}{{\mathcal B}^{\Box}}
\newcommand{\calK}{\mathcal{K}}
\newcommand{\bfA}{{\mathbf A}}
\newcommand{\bfb}{{\mathbf b}}
\newcommand{\bfr}{{\mathbf r}}
\newcommand{\bfx}{{\mathbf x}}
\newcommand{\bfxex}{{\mathbf x}_{\mbox{\scriptsize $\star$}}}
\newcommand{\Dim}{\mathop{\mathrm{dim\ }}}

\begin{document}
	%%%%%%%%%%%%%%%%%%%%%%%%%%%%%%%%%%%%%%%%%%%

\begin{center}
{\large
{\bf Adaptive selection of face coarse degrees of freedom in the \\
	BDDC and the FETI-DP iterative substructuring methods}}

	Jan Mandel \\
	Dept.~of Mathematical Sciences \\
	University of Colorado at Denver and Health Sciences Center \\
	Campus Box 170, P.O.~Box 173364, Denver CO 80217-3364 \\
	{\tt jmandel@math.cudenver.edu} \\
	B.~Sousedik
\end{center}
We propose adaptive selection of the coarse space of the
BDDC and FETI-DP iterative substructuring methods by adding
coarse degrees of freedom with support on selected
intersections of adjacent substructures. The coarse degrees
of freedom are constructed using eigenvectors associated
with the intersections. The minimal number of coarse degrees
of freedom on the selected intersections is added to
decrease a heuristic indicator of the the condition number
under a target value specified a priori. It is assumed that
the starting coarse degrees of freedom are already
sufficient to prevent relative rigid body motions of any
selected pair of adjacent substructures. It is shown
numerically on 2D elasticity problems that the indicator
based on pairs of substructures with common edges predicts
reasonably well the actual condition number, and that the
method can select adaptively the hard part of the problem
and concentrate computational work there to achieve good
convergence of the iterations at a modest cost.



	%%%%%%%%%%%%%%%%%%%%%%%%%%%%%%%%%%%%%%%%%%%


\end{document}

