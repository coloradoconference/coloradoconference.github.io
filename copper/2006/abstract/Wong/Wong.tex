\documentclass{report}
\usepackage{amsmath,amssymb}

\def\mathbi#1{\textbf{\em #1}}
\def\BA{\bf{A}}
\def\BB{\bf{B}}
\def\bn{\bf{n}}
\def\CP{\mathcal{P}}
\def\CV{\mathcal{V}}
\def\CI{\mathcal{I}}

\newcommand{\gradt}{\nabla\cdot}
\newcommand{\Reals}{\mathbb{R}}
\newcommand{\Cplex}{\mathbb{C}}

\newcommand{\Vecc}[2]{ \left(
	\begin{array}{c}
	#1 \\ #2
	\end{array}
	\right) }

\newcommand{\Matrr}[4]{ \left(
	\begin{array}{cc}
	#1 & #2 \\
	#3 & #4
	\end{array}
	\right) }

\newcommand{\calB}{{\mathcal B}}
\newcommand{\calBB}{{\mathcal B}^{\Box}}
\newcommand{\calK}{\mathcal{K}}
\newcommand{\bfA}{{\mathbf A}}
\newcommand{\bfb}{{\mathbf b}}
\newcommand{\bfr}{{\mathbf r}}
\newcommand{\bfx}{{\mathbf x}}
\newcommand{\bfxex}{{\mathbf x}_{\mbox{\scriptsize $\star$}}}
\newcommand{\Dim}{\mathop{\mathrm{dim\ }}}

\begin{document}
	%%%%%%%%%%%%%%%%%%%%%%%%%%%%%%%%%%%%%%%%%%%

\begin{center}
{\large
{\bf An embedding method for simulation of immobilized enzyme kinetics \\
	and transport in sessile hydrogel drops}}

	Chao-Jen Wong \\
	School of Mathematical Sciences, Claremont Graduate University \\
	710 N.~College Ave., Claremont CA 91711 \\
	{\tt wongc@cgu.edu} \\
	Ali Nadim
\end{center}
We
present a new numerical method, termed the embedding method,
to solve a system of nonlinear PDEs for multi-phase problems
in asymmetric 3-D domains. The main feature of this method
is its ability to perform interface calculation and account
for conditions relating solution properties across phase
interface using a finite difference / volume-fraction-based
front-capturing hybrid technique.

The approach begins by
considering the computational domain as physically separated
phases. A finite difference method with a Cartesian grid is
employed on the whole domain while modifications are applied
to correct boundary conditions at the interfaces. The
volume-fraction-based front-capturing algorithm is used to
capture each interface in terms of the volume fraction in
each cell. The major aspect of this method is its
implementation simplicity, which results in code generation
that can be highly optimized. To highlight this method, an
application is presented for simulation and investigation of
enzyme reactions within a sessile hydrogel drop, where the
Michaelis-Menten kinetics is used to model the reaction
mechanism.



	%%%%%%%%%%%%%%%%%%%%%%%%%%%%%%%%%%%%%%%%%%%


\end{document}

