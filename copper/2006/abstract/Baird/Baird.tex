\documentclass{report}
\usepackage{amsmath,amssymb}

\def\mathbi#1{\textbf{\em #1}}
\def\BA{\bf{A}}
\def\BB{\bf{B}}
\def\bn{\bf{n}}
\def\CP{\mathcal{P}}
\def\CV{\mathcal{V}}
\def\CI{\mathcal{I}}

\newcommand{\gradt}{\nabla\cdot}
\newcommand{\Reals}{\mathbb{R}}
\newcommand{\Cplex}{\mathbb{C}}

\newcommand{\Vecc}[2]{ \left(
	\begin{array}{c}
	#1 \\ #2
	\end{array}
	\right) }

\newcommand{\Matrr}[4]{ \left(
	\begin{array}{cc}
	#1 & #2 \\
	#3 & #4
	\end{array}
	\right) }

\newcommand{\calB}{{\mathcal B}}
\newcommand{\calBB}{{\mathcal B}^{\Box}}
\newcommand{\calK}{\mathcal{K}}
\newcommand{\bfA}{{\mathbf A}}
\newcommand{\bfb}{{\mathbf b}}
\newcommand{\bfr}{{\mathbf r}}
\newcommand{\bfx}{{\mathbf x}}
\newcommand{\bfxex}{{\mathbf x}_{\mbox{\scriptsize $\star$}}}
\newcommand{\Dim}{\mathop{\mathrm{dim\ }}}

\begin{document}
	%%%%%%%%%%%%%%%%%%%%%%%%%%%%%%%%%%%%%%%%%%%

\begin{center}
{\large
{\bf The representer method for data assimilation
of two phase flow in porous media}}

	John Baird \\
	Institute for Computational Engineering and Sciences \\
	The University of Texas at Austin \\
	1 Texas Longhorns, \#C0200, \quad  Austin TX 78712 \\
	{\tt jbaird@ices.utexas.edu} \\
	Clint Dawson
\end{center}
Advances in instrumenting and imaging the subsurface are
yielding large data sets, making data assimilation vital to
modeling flow through porous media. We derive and implement
the representer method applied to the oil/water model for
reservoirs, a nonlinear model. The representer method, like
the Kalman filter, solves the Euler-Lagrange (E-L) system
for the minimizer of a least-squares functional of the
misfit between the model and measurements. Because the
representer method uses the superposition principle, a
nonlinear model requires linearization of the E-L system. A
key concern is finding a linearization that converges
appropriately. We show that convergence is strongly affected
by the choice of weights in the least-squares functional. We
also compare the effects of linearization and the
computational costs of the representer method with the
ensemble Kalman filter.



	%%%%%%%%%%%%%%%%%%%%%%%%%%%%%%%%%%%%%%%%%%%


\end{document}

