\documentclass{report}
\usepackage{amsmath,amssymb}

\def\mathbi#1{\textbf{\em #1}}
\def\BA{\bf{A}}
\def\BB{\bf{B}}
\def\bn{\bf{n}}
\def\CP{\mathcal{P}}
\def\CV{\mathcal{V}}
\def\CI{\mathcal{I}}

\newcommand{\gradt}{\nabla\cdot}
\newcommand{\Reals}{\mathbb{R}}
\newcommand{\Cplex}{\mathbb{C}}

\newcommand{\Vecc}[2]{ \left(
	\begin{array}{c}
	#1 \\ #2
	\end{array}
	\right) }

\newcommand{\Matrr}[4]{ \left(
	\begin{array}{cc}
	#1 & #2 \\
	#3 & #4
	\end{array}
	\right) }

\newcommand{\calB}{{\mathcal B}}
\newcommand{\calBB}{{\mathcal B}^{\Box}}
\newcommand{\calK}{\mathcal{K}}
\newcommand{\bfA}{{\mathbf A}}
\newcommand{\bfb}{{\mathbf b}}
\newcommand{\bfr}{{\mathbf r}}
\newcommand{\bfx}{{\mathbf x}}
\newcommand{\bfxex}{{\mathbf x}_{\mbox{\scriptsize $\star$}}}
\newcommand{\Dim}{\mathop{\mathrm{dim\ }}}

\begin{document}
	%%%%%%%%%%%%%%%%%%%%%%%%%%%%%%%%%%%%%%%%%%%

\begin{center}
{\large
{\bf Improving mass-conservation of least-squares finite element methods}}

	Jeff Heys \\
	Chemical and Materials Engineering, Arizona State University \\
	Box 876006, Tempe AZ 85287-6006 \\
	{\tt heys@asu.edu} \\
	T.~A.~Manteuffel, S.~F.~McCormick, E.~Lee
\end{center}
Interest in least-squares finite element methods continues
to grow due to at least two of its major strengths. First,
the linear systems obtained after discretization are SPD and
often $H^1$ elliptic so that they can be efficiently solved
(often with optimal scalability) using a number of iterative
methods, including conjugate gradients and multigrid.
Second, the least-squares functional provides a sharp
measure of the local error with negligible computational
costs. Despite these two major advantages, the methods have
not gained widespread use, largely because they are
perceived as not providing accurate approximations to the
true solution, especially with regards to conservation of
mass.
The least-squares finite element method is not
discretely conservative, but the approximate solutions given
by the method are the \emph{most accurate approximate
solutions possible in the functional norm for a given
finite-dimensional space}. The method converges to the
approximation that minimizes the functional, so it gives a
relatively accurate solution in the functional norm.

However, anyone would agree that an approximate solution to
the Navier-Stokes equations that has an inflow rate that is
100 times the outflow rate is not an acceptable
approximation, even if it is `accurate' in some norm.
Unfortunately, some combinations of common least-square
functionals and finite element spaces for the Navier-Stokes
equations generate solutions that lose 99\% of the mass
between the inflow and outflow boundaries for some
particular boundary conditions. Herein lies the challenge
for least-squares methods: how do we formulate a functional
and boundary conditions that better represents the type of
accuracy we desire?

In this talk, two new first-order
system reformulations of the Navier-Stokes equations are
presented that admit a wider range of mass conserving
boundary conditions. It is common with least-squares methods
to rewrite the Navier-Stokes equations as a system of
first-order equations using the velocity-vorticity form. The
two new first-order systems are based on the
velocity-vorticity form, but they include a new variable,
$\mathbf{r}$, representing the pressure gradient plus all
or part of the convective term. As we will demonstrate, the
resulting operator problem can be solved very efficiently
using a multigrid or an algebraic multigrid solver, and
excellent mass conservation is observed for multiple test
problems. A difficulty with the new formulations is
obtaining boundary conditions for the new variable,
$\mathbf{r}$, but we will demonstrate at least three
different methods for overcoming this difficulty.



	%%%%%%%%%%%%%%%%%%%%%%%%%%%%%%%%%%%%%%%%%%%


\end{document}

