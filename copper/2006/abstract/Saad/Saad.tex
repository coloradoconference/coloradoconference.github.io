\documentclass{report}
\usepackage{amsmath,amssymb}

\def\mathbi#1{\textbf{\em #1}}
\def\BA{\bf{A}}
\def\BB{\bf{B}}
\def\bn{\bf{n}}
\def\CP{\mathcal{P}}
\def\CV{\mathcal{V}}
\def\CI{\mathcal{I}}

\newcommand{\gradt}{\nabla\cdot}
\newcommand{\Reals}{\mathbb{R}}
\newcommand{\Cplex}{\mathbb{C}}

\newcommand{\Vecc}[2]{ \left(
	\begin{array}{c}
	#1 \\ #2
	\end{array}
	\right) }

\newcommand{\Matrr}[4]{ \left(
	\begin{array}{cc}
	#1 & #2 \\
	#3 & #4
	\end{array}
	\right) }

\newcommand{\calB}{{\mathcal B}}
\newcommand{\calBB}{{\mathcal B}^{\Box}}
\newcommand{\calK}{\mathcal{K}}
\newcommand{\bfA}{{\mathbf A}}
\newcommand{\bfb}{{\mathbf b}}
\newcommand{\bfr}{{\mathbf r}}
\newcommand{\bfx}{{\mathbf x}}
\newcommand{\bfxex}{{\mathbf x}_{\mbox{\scriptsize $\star$}}}
\newcommand{\Dim}{\mathop{\mathrm{dim\ }}}

\begin{document}
% 	%%%%%%%%%%%%%%%%%%%%%%%%%%%%%%%%%%%%%%%%%%%
% 
% \begin{center}
% {\large
% {\bf MIQR: a multilevel incomplete QR preconditioner
% for large sparse least-squares problems}}
% 
% 	Yousef Saad \\
% 	Dept.~of Computer Science and Engineering \\
% 	University of Minnesota, 4-196 EE/CSci Building \\
% 	200 Union Street S.E., Minneapolis, MN 55455 \\
%         {\tt saad@cs.umn.edu} \\
%         Na Li
% \end{center}
% 
% We present a Multilevel Incomplete QR (MIQR) factorization
% for solving large sparse least-squares problems. The
% algorithm builds the factorization by exploiting structural
% orthogonality in general sparse matrices. At any given step,
% the algorithm finds an independent set of columns, i.e., a
% set of columns that have orthogonal patterns. The other
% columns are then block orthogonalized against columns of the
% independent set and the process is repeated recursively for
% a certain number of levels on these remaining columns. The
% final level matrix is processed with a standard QR or
% Incomplete QR factorization. Dropping strategies are
% employed throughout the levels in order to maintain a good
% level of sparsity. A few improvements to this basic scheme
% are explored. Among these is the relaxation of the
% requirement of independent sets of columns. Numerical tests
% are proposed which compare this scheme with the standard
% incomplete QR preconditioner, the robust incomplete
% factorization (RIF) preconditioner, and ARMS (on the normal
% equations).



	%%%%%%%%%%%%%%%%%%%%%%%%%%%%%%%%%%%%%%%%%%%


\end{document}

