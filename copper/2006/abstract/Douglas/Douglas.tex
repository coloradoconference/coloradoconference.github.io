\documentclass{report}
\usepackage{amsmath,amssymb}

\def\mathbi#1{\textbf{\em #1}}
\def\BA{\bf{A}}
\def\BB{\bf{B}}
\def\bn{\bf{n}}
\def\CP{\mathcal{P}}
\def\CV{\mathcal{V}}
\def\CI{\mathcal{I}}

\newcommand{\gradt}{\nabla\cdot}
\newcommand{\Reals}{\mathbb{R}}
\newcommand{\Cplex}{\mathbb{C}}

\newcommand{\Vecc}[2]{ \left(
	\begin{array}{c}
	#1 \\ #2
	\end{array}
	\right) }

\newcommand{\Matrr}[4]{ \left(
	\begin{array}{cc}
	#1 & #2 \\
	#3 & #4
	\end{array}
	\right) }

\newcommand{\calB}{{\mathcal B}}
\newcommand{\calBB}{{\mathcal B}^{\Box}}
\newcommand{\calK}{\mathcal{K}}
\newcommand{\bfA}{{\mathbf A}}
\newcommand{\bfb}{{\mathbf b}}
\newcommand{\bfr}{{\mathbf r}}
\newcommand{\bfx}{{\mathbf x}}
\newcommand{\bfxex}{{\mathbf x}_{\mbox{\scriptsize $\star$}}}
\newcommand{\Dim}{\mathop{\mathrm{dim\ }}}

\begin{document}
	%%%%%%%%%%%%%%%%%%%%%%%%%%%%%%%%%%%%%%%%%%%

\begin{center}
{\large
{\bf Actually doing dynamic data-driven application simulations (in 4 parts)}}

	Craig C.~Douglas \\
	Computer Science Department, 773 Anderson Hall \\
	University of Kentucky, Lexington KY 40506-0046 \\
	and Computer Science Department, P.O.~Box 208285 \\
	Yale University,  New Haven CT 06520-8285 \\
	{\tt douglas-craig@cs.yale.edu}

	Chris R.~Johnson and Steven G.~Parker, University of Utah \\
	Janice Coen, National Center for Atmospheric Research \\
	Jan Mandel and Jonathan Beezley, Univ.~of Colorado at Denver
\end{center}
This is a four part talk to introduce DDDAS concepts before
the Wednesday night workshop.

{\bf Part I:} Introduction to DDDAS and Its Impact on High
Performance Computing Environments \\
Craig Douglas

DDDAS is a paradigm whereby an application (or simulation)
and measurements become a symbiotic feedback control system.
DDDAS entails the ability to dynamically incorporate
additional data into an executing application, and in
reverse, the ability of an application to dynamically steer
the measurement process.  Such capabilities promise more
accurate analysis and prediction, more precise controls, an
d more reliable outcomes.  The ability of an application to
control and guide the measurement process and determine
when, where, and how it is best to gather additional data
has itself the potential of enabling more effective
measurement methodologies.  Furthermore, the incorporation
of dynamic inputs into an executing application invokes new
system modalities and helps create application software
systems that can more accurately describe real world,
complex systems.  This enables the development of
applications that intelligently adapt to evolving conditions
and that infer new knowledge in ways that are not
predetermined by the initialization parameters and initial
static data.

DDDAS creates a rich set of new challenges for applications,
algorithms, systems software, and measurement methods.
DDDAS research typically requires strong, systematic
collaborations between applications domain researchers and
mathematics, statistics, and computer sciences researchers,
as well as researchers involved in the design and
implementation of measurement methods and instruments.
Consequently, most DDDAS projects involve multidisciplinary
teams of researchers.

In addition, DDDAS enabled applications run in a different
manner than many traditional applications.  They place
different strains on high performance systems and centers.
In this talk, we will also categorize some of these
differences.

{\bf Part 2:} Problem Solving Environments for DDDAS
\\
Chris R.~Johnson and Steven G.~Parker

One of the significant challenges for DDDAS is to create
software infrastructure and tools that help DDDAS
researchers tackle the multidisciplinary, often large-scale,
dynamically coupled problems described in the previous
presentation.

DDDAS problems often require using multiple software
frameworks and packages, which leads to the significant
software architecture challenge of integrating and providing
interoperability of different software frameworks, packages,
and libraries.  Our approach to this challenge is  to create
software ``bridges'' using a meta-component model that allows
the user to easily connect one software framework or package
to another.

The new system (currently called SCIRun2, but that will
change very soon) support the entire life cycle of
scientific applications by allowing scientific programmers
to quickly and easily develop new techniques, debug new
implementations, and apply known algorithms to solve novel
problems.  SCIRun2 also contain many powerful visualization
algorithms for scalar,  vector, and tensor field
visualization, as well as image processing tools.

In this presentation, we will provide examples of DDDAS
software integration.

{\bf Part 3:} The Impact of DDDAS on Wildland Fire Modeling and
Fire Front Tracking
\\
Janice Coen and Jan Mandel

In this talk, we will describe an application to which DDDAS
concepts are being applied.  Wildland fire modeling involves
a numerical weather prediction model that is two-way coupled
to a fire behavior model, so that the fire can create its
own weather.  This is an extremely challenging computational
problem with limited predictability because of the
uncertainty in fire behavior in addition to uncertainties in
weather modeling.  It is also difficult to obtain
observations near a wildfire.  Thus, DDDAS concepts have
great potential in advancing this area.  In this work, we
will describe techniques we have been applying to introduce
DDDAS concepts into what was a traditional modeling
approach.  We define a novel partial differential equation
based model for wildland fires instead of the usual
stochastic based model.  We will show where a DDDAS approach
can provide a breakthrough in fire front tracking that can
be transmitted to the people on the mountainsides through
both computational science and high tech advances.

{\bf Part 4:} Out of Time Order Kalman Filtering for DDDAS Data
Assimilation
\\
Jan Mandel and Jonathan Beezley

We present the basic principles of data assimilation by
ensemble filtering.  These methods run a collection of
randomly perturbed simulations, called an ensemble.  From
time to time, the ensemble is modified by creating linear
combinations found by solving a least squares problem to
match the data.  We describe new developments needed to
accommodate strongly nonlinear wildfire models and sparse
data.  The new methods include Tikhonov-like regularization
in a Bayesian probabilistic framework and new hybrid
deterministic/stochastic ensemble methods for non-Gaussian
distributions, based on the theory of probability measures
on Sobolev spaces.  We also discuss a space-time ensemble
approach to assimilate data arriving out of order.

{\bf Wednesday night, April 5},
Workshop on Dynamic Data Driven Liquid Flows
\\ Douglas, Johnson, Coen, Mandel

This workshop will be strictly hands on.  It will introduce
DDDAS techniques (see http://www.dddas.org) including
dynamic modeling, errors, sensor operation, and the
symbiotic relations between the sensors and the application.

We will simulate the level of a liquid in media that is
porous in one boundary edge only and design from scratch an
algorithm to maintain it at a fixed level on average even
though the liquid is disappearing through the open boundary
using a random step function.

We will develop convergence results initially using a
semi-direct method, but some of the participants may end up
with a random walk by the end of the workshop.  We will
iterate on the liquid problem until we develop a fast
iterative (and convergent) algorithm that we have thoroughly
tested.  We will use the data from experiments to drive the
entire methodology and the algorithms will drive how and
when data is collected.

This workshop will be held in one of the local watering
holes, not in the conference center.  Sensor oversight and
correction will be provided at the tables.



	%%%%%%%%%%%%%%%%%%%%%%%%%%%%%%%%%%%%%%%%%%%


\end{document}

