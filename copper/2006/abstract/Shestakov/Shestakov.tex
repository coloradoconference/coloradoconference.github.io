\documentclass{report}
\usepackage{amsmath,amssymb}

\def\mathbi#1{\textbf{\em #1}}
\def\BA{\bf{A}}
\def\BB{\bf{B}}
\def\bn{\bf{n}}
\def\CP{\mathcal{P}}
\def\CV{\mathcal{V}}
\def\CI{\mathcal{I}}

\newcommand{\gradt}{\nabla\cdot}
\newcommand{\Reals}{\mathbb{R}}
\newcommand{\Cplex}{\mathbb{C}}

\newcommand{\Vecc}[2]{ \left(
	\begin{array}{c}
	#1 \\ #2
	\end{array}
	\right) }

\newcommand{\Matrr}[4]{ \left(
	\begin{array}{cc}
	#1 & #2 \\
	#3 & #4
	\end{array}
	\right) }

\newcommand{\calB}{{\mathcal B}}
\newcommand{\calBB}{{\mathcal B}^{\Box}}
\newcommand{\calK}{\mathcal{K}}
\newcommand{\bfA}{{\mathbf A}}
\newcommand{\bfb}{{\mathbf b}}
\newcommand{\bfr}{{\mathbf r}}
\newcommand{\bfx}{{\mathbf x}}
\newcommand{\bfxex}{{\mathbf x}_{\mbox{\scriptsize $\star$}}}
\newcommand{\Dim}{\mathop{\mathrm{dim\ }}}

\begin{document}
% 	%%%%%%%%%%%%%%%%%%%%%%%%%%%%%%%%%%%%%%%%%%%
% 
% \begin{center}
% {\large
% {\bf Solution of the nonlinear multifrequency radiation
% diffusion equation in a multiphysics,
% high energy density, AMR code}}
% 
% 	Aleksei Shestakov \\
% 	Lawrence Livermore National Laboratory, POB 808, L-38 \\
% 	Livermore CA 94551 \\
%         {\tt shestakov1@llnl.gov}
% \end{center}
% 
% We describe a scheme to
% solve the multifrequency radiation diffusion equation
% which is intended for a
% multiphysics, high energy density computer code with
% adaptive mesh refinement (AMR). In our code, AMR is
% implemented by refining in both space and time [1].
% There may be several levels of refinement,
% which, going from fine to coarse, are nested within each
% other.
% 
% We time-advance as follows. Assume there are only
% two levels, one coarse, with domain $\Omega_c$ and boundary
% $\partial \Omega_c$, and one fine $\Omega_f$ with boundary $\partial \Omega_f$. Since the
% domains are nested, $\Omega_f \subseteq \Omega_c$. At the start of the
% time cycle, the equations are first updated on $\Omega_c$ using a
% timestep $\Delta t_c$, a process defined as a {\em level solve\/}
% on $\Omega_c$. If the spatial grid on $\Omega_f$ is a twofold
% refinement of that discretizing $\Omega_c$, we need two level
% solves on $\Omega_f$, each with timestep $\Delta t_c/2$, in order to
% bring the $\Omega_f$ solution up to the advanced coarse level
% time. Boundary conditions (BC) are required on $\partial \Omega_f$. On
% parts of $\partial \Omega_f$ which do not extend to the physical
% boundary, BC are obtained by interpolating the coarse grid
% solution.
% 
% For diffusion equations, e.g., $u_t = (D u_x)_x$,
% conventionally, one supplies Dirichlet data. This ensures
% that the coarse and fine grid solution is continuous across
% $\partial \Omega_f.$ However, the flux $-D u_x$ may be discontinuous,
% which is unacceptable since this results in a loss of
% conservation. To remedy the defect, after the level solves,
% the coarse and fine grid solutions are {\em synced.}
% One solves a related, nearly homogeneous, problem for
% corrections on the union of discretizations of $\Omega_f$ and
% $\Omega_c$. The sole non-homogeneity of the system for the
% corrections is the miss-match of the fluxes on $\partial \Omega_f$. When
% the corrections are added to the result of the level solves,
% one obtains a conservative solution, continuous and with
% continuous flux.
% 
% This paper describes the AMR
% implementation for the multigroup radiation diffusion and
% matter energy balance equations,
% \begin{eqnarray*}
% \qquad \qquad \qquad \qquad
% \partial_t u_g
% & = &
% \nabla \cdot D_g \nabla u_g + \kappa_g \, ( \, B_g - u_g \, )
% \, , \;\; g = 1, \, \ldots, \, G
% \qquad \qquad \qquad \qquad (1)
% \\
% \rho \, c_v \partial_t T
% & = &
% - \sum_{k=1}^G
% \Delta_k \, \kappa_k \, ( \, B_k - u_k \, ) \, .
% \qquad \qquad \qquad \qquad (2)
% \end{eqnarray*}
% In
% equations (1-2),
% $u_g$ is the radiation energy density of the $g$th
% {\em group}.
% Groups arise by discretizing the frequency domain
% $0 \leq \nu \leq \infty$ into $G$ intervals.
% In equations (1-2),
% $D_g$ and $\kappa_g$ are the
% diffusion and coupling coefficients, $B_g$ is the Planck
% function, $\rho$ the mass density, $c_v$ the specific heat,
% and $\Delta_k = \nu_k - \nu_{k-1}$. The system is nonlinear;
% $D_g$ and $\kappa_g$, which in addition to being strong
% functions of frequency, depend on $\rho$ and $T$. For
% non-ideal gases, $c_v$ depends on $\rho$ and $T$.
% Equations (1-2)
% describe the evolution of
% the $G+1$ unknowns $\{u_k\}_{k=1}^G$ and $T$.
% 
% A single level solve of
% equations (1-2)
% is a formidable task
% in itself. For the advance, we use the procedure described
% by Shestakov [3], generalized for ``real,'' multiple
% materials whose properties ($c_v$, $k_g$, etc.) are given in
% tabular form.
% 
% For simulations using AMR, after advancing
% on two levels, $\Omega_c$ and $\Omega_f$, the solutions are synced
% using a generalization of the Howell and Greenough procedure
% (HG) [1], which may be directly applied to
% equations (1-2)
% if $G = 1$. However, if $G > 1$,
% the situation is more complicated since the energies $u_g$
% are coupled. We resolve the difficulty by applying concepts
% of the ``Partial Temperature'' scheme (PT) of Lund and
% Wilson [2]. As in PT, we cycle through the groups
% in random order. Each group is synced as in HG, but the
% correction to $T$ is only a partial change. Only after all
% the groups have been addressed, do we obtain the final
% correction.
% 
% Our AMR procedure is implemented in a
% multiphysics code. Results will be presented. We simulate
% effects of strong explosions in air and compare multigroup
% results with runs where the frequency domain is not
% discretized, so-called gray diffusion. The simulations also
% use the hydro and heat conduction modules. In addition to
% the coarse level, there are two levels of refinement.
% 
% [1]
% L.~H.~Howell and J.~A.~Greenough,
% J.~Comp. Phys. {\bf 184}(1) (2003)
% 53--78.
% 
% [2]
% C.~M.~Lund and J.~R.~Wilson,
% Lawrence Livermore Natl. Lab. report UCRL-84678,
% July 29, 1980.
% 
% [3]
% A.~I.~Shestakov,
% {\em 8th Copper Mountain Conference on Iterative Methods},
% March 2004.



	%%%%%%%%%%%%%%%%%%%%%%%%%%%%%%%%%%%%%%%%%%%


\end{document}

