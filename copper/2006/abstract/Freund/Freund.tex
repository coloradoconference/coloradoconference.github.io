\documentclass{report}
\usepackage{amsmath,amssymb}

\def\mathbi#1{\textbf{\em #1}}
\def\BA{\bf{A}}
\def\BB{\bf{B}}
\def\bn{\bf{n}}
\def\CP{\mathcal{P}}
\def\CV{\mathcal{V}}
\def\CI{\mathcal{I}}

\newcommand{\gradt}{\nabla\cdot}
\newcommand{\Reals}{\mathbb{R}}
\newcommand{\Cplex}{\mathbb{C}}

\newcommand{\Vecc}[2]{ \left(
	\begin{array}{c}
	#1 \\ #2
	\end{array}
	\right) }

\newcommand{\Matrr}[4]{ \left(
	\begin{array}{cc}
	#1 & #2 \\
	#3 & #4
	\end{array}
	\right) }

\newcommand{\calB}{{\mathcal B}}
\newcommand{\calBB}{{\mathcal B}^{\Box}}
\newcommand{\calK}{\mathcal{K}}
\newcommand{\bfA}{{\mathbf A}}
\newcommand{\bfb}{{\mathbf b}}
\newcommand{\bfr}{{\mathbf r}}
\newcommand{\bfx}{{\mathbf x}}
\newcommand{\bfxex}{{\mathbf x}_{\mbox{\scriptsize $\star$}}}
\newcommand{\Dim}{\mathop{\mathrm{dim\ }}}

\begin{document}
	%%%%%%%%%%%%%%%%%%%%%%%%%%%%%%%%%%%%%%%%%%%

\begin{center}
{\large
{\bf A flexible conjugate gradient method and its \\
application in power grid analysis of VLSI circuits}}

	Roland W.~Freund \\
	Dept.~of Mathematics, University of California \\
	Davis One Shields Avenue, Davis CA 95616 \\
	{\tt freund@math.ucdavis.edu}
\end{center}
The design and verification of today's very large-scale
integrated (VLSI) circuits involve some extremely
challenging numerical problems. One of the truly large-scale
problems in this area is power grid analysis. Power grids
are modeled as networks with up to 10 millions nodes.
Steady-state analysis of power grids requires the solution
of correspondingly large sparse symmetric positive definite
linear systems. The coefficient matrices of these systems
have the structure of weighted Laplacians on
three-dimensional grids, but with `boundary' conditions
given on a subset of the interior grid points.
Strongly-varying weights and the interior boundary
conditions have the effect that solutions of these linear
systems are often very localized, with components of the
solution being near zero in large parts of the grid.

In this
talk, we present a flexible conjugate gradient method that
is tailored to the solution of the truly large-scale linear
systems arising in VLSI power grid analysis. The algorithm
allows changing preconditioners and sparsification of the
search directions at each iteration. These are the key
features to exploit the local nature of the solutions. We
also discuss the problem of constructing efficient
preconditioners for the linear systems in VLSI power grid
analysis, and we present results of numerical experiments.


	%%%%%%%%%%%%%%%%%%%%%%%%%%%%%%%%%%%%%%%%%%%


\end{document}

