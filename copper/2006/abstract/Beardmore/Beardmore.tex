\documentclass{report}
\usepackage{amsmath,amssymb}

\def\mathbi#1{\textbf{\em #1}}
\def\BA{\bf{A}}
\def\BB{\bf{B}}
\def\bn{\bf{n}}
\def\CP{\mathcal{P}}
\def\CV{\mathcal{V}}
\def\CI{\mathcal{I}}

\newcommand{\gradt}{\nabla\cdot}
\newcommand{\Reals}{\mathbb{R}}
\newcommand{\Cplex}{\mathbb{C}}

\newcommand{\Vecc}[2]{ \left(
	\begin{array}{c}
	#1 \\ #2
	\end{array}
	\right) }

\newcommand{\Matrr}[4]{ \left(
	\begin{array}{cc}
	#1 & #2 \\
	#3 & #4
	\end{array}
	\right) }

\newcommand{\calB}{{\mathcal B}}
\newcommand{\calBB}{{\mathcal B}^{\Box}}
\newcommand{\calK}{\mathcal{K}}
\newcommand{\bfA}{{\mathbf A}}
\newcommand{\bfb}{{\mathbf b}}
\newcommand{\bfr}{{\mathbf r}}
\newcommand{\bfx}{{\mathbf x}}
\newcommand{\bfxex}{{\mathbf x}_{\mbox{\scriptsize $\star$}}}
\newcommand{\Dim}{\mathop{\mathrm{dim\ }}}

\begin{document}
	%%%%%%%%%%%%%%%%%%%%%%%%%%%%%%%%%%%%%%%%%%%

\begin{center}
{\large
{\bf A numerical bifurcation analysis of the Ornstein-Zernike equation}}

	Robert E Beardmore \\
	Dept.~of Mathematics, Imperial College London \\
	South Kensington Campus,  London SW7 2AZ UK \\
	{\tt r.beardmore@ic.ac.uk} \\
	A Peplow, F Bresme
\end{center}
The isotropic Ornstein-Zernike (OZ) equation
$$
\qquad
\qquad
\qquad
h(r) = c(r) + \rho \int_{\mathbb{R}^{3}}
h(\|{\bf x}-{\bf y}\|)c(\|{\bf y}\|)d{\bf y},
\qquad
\qquad
\qquad
(1)
$$
that is the subject of this
paper was presented almost a century ago to model the
molecular structure of a fluid at varying densities. In
order to form a well-posed mathematical system of equations
from (1) that can be solved, at least in
principle, we assume the existence of a closure
relationship. This is an algebraic equation that augments
(1) with a pointwise constraint that is deemed to
hold throughout the fluid and it forces a relationship
between the total and direct correlation functions ($h$ and
$c$ respectively).

Some closures have a mathematically
appealing structure in the sense that the total correlation
function is posed as a perturbation of the {\em Mayer
f-function} given by \[ f(r)=-1+e^{-\beta u(r)}.\] This
perturbation depends on the potential $u$, temperature
(essentially $1/\beta$) and the indirect correlation
function through a nonlinear function that we denote $G$:
$$
\qquad
\qquad
\qquad
h = f(r) + e^{-\beta u(r)}G(h-c),
\qquad
(G(0)= 0),
\qquad
\qquad
\qquad
(2)
$$
so that (1-2) are solved together with $\beta$
and $\rho$ as bifurcation parameters. There are many
closures in use and if we write $\exp_1(z) = -1+e^z$ then
the hyper-netted chain (HNC) closure
$$
\qquad
\qquad
\qquad
\qquad
\qquad
G(\gamma) = \exp_1(\gamma)
\qquad
\qquad
\qquad
\qquad
\qquad
(3)
$$
has the form of
(2) and is popular in the physics and chemistry
literature.

The purpose of the talk is show that {\em any
reasonable} discretisation method applied to
(1-2) suffers from an inherent defect if
the HNC closure is used that can be summarised as follows:
phase transitions lead to fold bifurcations. The existence
of a phase transition is characterised by the existence of a
bifurcation at infinity with respect to $h$ in an $L^1$ norm
at a certain density, such that boundedness of $h$ is
maintained in a certain $L^p$ norm. This behaviour is
difficult to mimic computationally by projecting onto a
space of fixed and finite dimension and, as a result,
projections of (1-2) can be shown to
undergo at least one fold bifurcation if such a bifurcation
at infinity is present. However, other popular closure
relations do not necessarily suffer from the same defect.



	%%%%%%%%%%%%%%%%%%%%%%%%%%%%%%%%%%%%%%%%%%%


\end{document}

