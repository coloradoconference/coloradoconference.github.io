\documentclass{report}
\usepackage{amsmath,amssymb}

\def\mathbi#1{\textbf{\em #1}}
\def\BA{\bf{A}}
\def\BB{\bf{B}}
\def\bn{\bf{n}}
\def\CP{\mathcal{P}}
\def\CV{\mathcal{V}}
\def\CI{\mathcal{I}}

\newcommand{\gradt}{\nabla\cdot}
\newcommand{\Reals}{\mathbb{R}}
\newcommand{\Cplex}{\mathbb{C}}

\newcommand{\Vecc}[2]{ \left(
	\begin{array}{c}
	#1 \\ #2
	\end{array}
	\right) }

\newcommand{\Matrr}[4]{ \left(
	\begin{array}{cc}
	#1 & #2 \\
	#3 & #4
	\end{array}
	\right) }

\newcommand{\calB}{{\mathcal B}}
\newcommand{\calBB}{{\mathcal B}^{\Box}}
\newcommand{\calK}{\mathcal{K}}
\newcommand{\bfA}{{\mathbf A}}
\newcommand{\bfb}{{\mathbf b}}
\newcommand{\bfr}{{\mathbf r}}
\newcommand{\bfx}{{\mathbf x}}
\newcommand{\bfxex}{{\mathbf x}_{\mbox{\scriptsize $\star$}}}
\newcommand{\Dim}{\mathop{\mathrm{dim\ }}}

\begin{document}
	%%%%%%%%%%%%%%%%%%%%%%%%%%%%%%%%%%%%%%%%%%%

\begin{center}
{\large
{\bf Iterative solver for density functional theory calculations \\
	on composite meshes by quadratic finite elements}}

	J.-L.~Fattebert \\
	PO Box 808, L-561, Center for Applied Scientific Computing \\
	Lawrence Livermore National Laboratory, Livermore CA 94551 \\
	{\tt fattebert1@llnl.gov} \\
	R.~Hornung, A.~Wissink
\end{center}
Density Functional Theory (DFT) is a simplified quantum
model that has proved very successful in real applications.
It introduces an independent particles description of the
electronic structure of molecules or materials which is much
simpler to treat that the original Schr\"{o}dinger equations.
Simulating realistic physical systems by DFT however is
still computationally very demanding. More efficient
numerical algorithms to reduce computer time and enable
larger simulations are always in demand by chemists and
physicists who are studying phenomenon at the molecular
level.

The finite element (FE) method, a very popular
approach to solve partial differential equations, has only
recently started being used for solving the Kohn-Sham
equations of Density Functional Theory for realistic 3D
applications. Traditionally, the pseudo-spectral approach
has been the most popular in the field under the
denomination Plane Waves method. The regular usage of
periodic boundary conditions with simple geometries explains
this preference. However with the increase in computer power
and the growing interest in studying larger and more diverse
systems, more flexible real-space discretizations by finite
difference or finite elements have recently attracted more
interest.

In this work we focus on FE discretizations
with local mesh refinement for the Kohn-Sham equations, and
propose an efficient iterative solver and preconditioner for
this problem. We present a hierarchical quadratic Finite
Elements approach to discretize the equations on structured
non-uniform meshes. A multigrid FAC preconditioner is
proposed to iteratively minimize the energy functional
associated to the Kohn-Sham equations. It is based on an
accelerated steepest descent-based scheme. The method has
been implemented using SAMRAI, a parallel software
infrastructure for general AMR applications. Numerical
results of electronic structure calculations on small atomic
clusters show in particular a mesh-independent convergence
rate for the iterative solver.



	%%%%%%%%%%%%%%%%%%%%%%%%%%%%%%%%%%%%%%%%%%%


\end{document}

