\documentclass{report}
\usepackage{amsmath,amssymb}

\def\mathbi#1{\textbf{\em #1}}
\def\BA{\bf{A}}
\def\BB{\bf{B}}
\def\bn{\bf{n}}
\def\CP{\mathcal{P}}
\def\CV{\mathcal{V}}
\def\CI{\mathcal{I}}

\newcommand{\gradt}{\nabla\cdot}
\newcommand{\Reals}{\mathbb{R}}
\newcommand{\Cplex}{\mathbb{C}}

\newcommand{\Vecc}[2]{ \left(
	\begin{array}{c}
	#1 \\ #2
	\end{array}
	\right) }

\newcommand{\Matrr}[4]{ \left(
	\begin{array}{cc}
	#1 & #2 \\
	#3 & #4
	\end{array}
	\right) }

\newcommand{\calB}{{\mathcal B}}
\newcommand{\calBB}{{\mathcal B}^{\Box}}
\newcommand{\calK}{\mathcal{K}}
\newcommand{\bfA}{{\mathbf A}}
\newcommand{\bfb}{{\mathbf b}}
\newcommand{\bfr}{{\mathbf r}}
\newcommand{\bfx}{{\mathbf x}}
\newcommand{\bfxex}{{\mathbf x}_{\mbox{\scriptsize $\star$}}}
\newcommand{\Dim}{\mathop{\mathrm{dim\ }}}

\begin{document}
	%%%%%%%%%%%%%%%%%%%%%%%%%%%%%%%%%%%%%%%%%%%

\begin{center}
{\large
{\bf A comparison of eigensolvers for large electronic structure calculations}}

	Osni Marques \\
	Lawrence Berkeley National Laboratory, 1 Cyclotron Road \\
	MS 50F-1650, Berkeley CA 94720-8139 \\
	{\tt oamarques@lbl.gov} \\
	Andrew Canning, Julien Langou, Stanimire Tomov, \\
	Christof Voemel, Lin-Wang Wang
\end{center}
The solution of the single particle Schr\"{o}dinger equation
that arises in electronic structure calculations often
requires solving for interior eigenstates of a large
Hamiltonian. The states at the top of the valence band and
at the bottom of the conduction band determine the band gap
that relates to important physical characteristics such as
optical or transport properties.

In order to avoid the
explicit computation of all eigenstates, a folded spectrum
method has been usually employed to compute only the
eigenstates near the band gap. In this talk, we compare the
conjugate gradient minimization and the optimal block
preconditioned conjugate gradient (LOBPCG) applied to the
folded spectrum matrix with the Jacobi-Davidson algorithm.



	%%%%%%%%%%%%%%%%%%%%%%%%%%%%%%%%%%%%%%%%%%%


\end{document}

