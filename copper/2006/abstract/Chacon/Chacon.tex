\documentclass{report}
\usepackage{amsmath,amssymb}

\def\mathbi#1{\textbf{\em #1}}
\def\BA{\bf{A}}
\def\BB{\bf{B}}
\def\bn{\bf{n}}
\def\CP{\mathcal{P}}
\def\CV{\mathcal{V}}
\def\CI{\mathcal{I}}

\newcommand{\gradt}{\nabla\cdot}
\newcommand{\Reals}{\mathbb{R}}
\newcommand{\Cplex}{\mathbb{C}}

\newcommand{\Vecc}[2]{ \left(
	\begin{array}{c}
	#1 \\ #2
	\end{array}
	\right) }

\newcommand{\Matrr}[4]{ \left(
	\begin{array}{cc}
	#1 & #2 \\
	#3 & #4
	\end{array}
	\right) }

\newcommand{\calB}{{\mathcal B}}
\newcommand{\calBB}{{\mathcal B}^{\Box}}
\newcommand{\calK}{\mathcal{K}}
\newcommand{\bfA}{{\mathbf A}}
\newcommand{\bfb}{{\mathbf b}}
\newcommand{\bfr}{{\mathbf r}}
\newcommand{\bfx}{{\mathbf x}}
\newcommand{\bfxex}{{\mathbf x}_{\mbox{\scriptsize $\star$}}}
\newcommand{\Dim}{\mathop{\mathrm{dim\ }}}

\begin{document}
	%%%%%%%%%%%%%%%%%%%%%%%%%%%%%%%%%%%%%%%%%%%

\begin{center}
{\large
{\bf A fully implicit extended 3D MHD solver}}

	Luis Chacon \\
	MS K717, Los Alamos National Laboratory, Los Alamos NM 87545 \\
	{\tt chacon@lanl.gov} \\
	Dana A.~Knoll
\end{center}
We present results from our research on Jacobian-free
Newton-Krylov (JFNK) methods applied to the time-dependent,
primitive-variable, 3D extended magnetohydrodynamics (MHD)
equations. MHD is a fluid description of the plasma state.
While plasma is made up of independent (but coupled) ion and
electron species, the standard MHD description of a plasma
only includes ion time and length scales (one-fluid model).
Extended MHD (XMHD) includes nonideal effects such as
nonlinear, anisotropic transport and two-fluid (Hall and
diamagnetic) effects. XMHD supports so-called dispersive
waves (whistler, ion acoustic), which feature a quadratic
dispersion relation $\omega \sim k^{2}$. In explicit time
integration methods, this results in a stringent CFL limit
$\Delta t_{CFL}\propto \Delta x^{2}$, which severely limits
their applicability to the study of long-frequency phenomena
in XMHD.

A fully implicit implementation promises
efficiency (by removing the CFL constraint) without
sacrificing numerical accuracy [1].
However, the nonlinear
nature of the XMHD system and the numerical stiffness of its
fast waves make this endeavor very difficult. Newton-Krylov
methods can meet the challenge provided suitable
preconditioning is available.

We propose a successful
preconditioning strategy for the 3D primitive-variable XMHD
formalism. It is based on ``physics-based'' ideas [2,3],
in which a hyperbolic system of
equations (which is diagonally submissive for $\Delta
t>\Delta t_{CFL}$) is {}``parabolized'' to arrive to a
diagonally dominant approximation of the original system,
which is multigrid-friendly. The use of approximate
multigrid (MG) techniques to invert the {}``parabolized''
operator is a crucial step in the effectiveness of the
preconditioner and the scalability of the overall algorithm.
The parabolization procedure can be properly generalized
using the well-known Schur decomposition of a 2$\times $2
block matrix. In the context of XMHD, the resulting Schur
complement is a system of PDEs that couples the three
plasma velocity components, and needs to be inverted in a
coupled manner. Nevertheless, a system MG treatment is still
possible since, when properly discretized, the XMHD Schur
complement is block diagonally dominant by construction, and
block smoothing is effective.

In this presentation, we
will discuss the derivation and validity of the
physics-based preconditioner for resistive MHD and its
generalization to XMHD, the connection with Schur complement
analysis, and the system-MG treatment of the associated
systems. A novel second-order, cell-centered, conservative
finite-volume discretization has been recently developed [4]
for the XMHD system above, and will be used
in this work. It is suitable for general curvilinear
geometries, solenoidal in $\mathbf{B}$ and $\mathbf{J}$,
numerically non-dissipative, and linearly and nonlinearly
stable. We will demonstrate the algorithm using the GEM
challenge configuration [5].
Grid convergence
studies will demonstrate that CPU time scales scale
optimally as $\mathcal{O}(N)$, where $N$ is the number of
unknowns, and that the number of Krylov iterations scales as
$\mathcal{O}(N^{0})$. Time convergence studies will
demonstrate a favorable scaling with time step
$\mathcal{O}(\Delta t^{\alpha })$, with $\alpha <1.0$.

[1] D.~A.~Knoll, L.~Chac\'{o}n, L.~G.~Margolin, and V.~A.~Mousseau,
J.~Comput. Phys. \textbf{185} (2003) 583.

[2] L.~Chac\'{o}n, D.~A.~Knoll, and J.~M.~Finn,
J.~Comput. Phys., \textbf{178} (2002) 15.

[3] L.~Chac\'{o}n and D.~A.~Knoll, J.~Comput. Phys.,
\textbf{188} (2003) 573.

[4] L.~Chac\'{o}n, Comp. Phys. Comm., \textbf{163} (2004) 143.

[5] J.~Birn et al.,
J.~Geophys. Res., \textbf{106} (2001) 3715.


	%%%%%%%%%%%%%%%%%%%%%%%%%%%%%%%%%%%%%%%%%%%


\end{document}

