\documentclass{report}
\usepackage{amsmath,amssymb}

\def\mathbi#1{\textbf{\em #1}}
\def\BA{\bf{A}}
\def\BB{\bf{B}}
\def\bn{\bf{n}}
\def\CP{\mathcal{P}}
\def\CV{\mathcal{V}}
\def\CI{\mathcal{I}}

\newcommand{\gradt}{\nabla\cdot}
\newcommand{\Reals}{\mathbb{R}}
\newcommand{\Cplex}{\mathbb{C}}

\newcommand{\Vecc}[2]{ \left(
	\begin{array}{c}
	#1 \\ #2
	\end{array}
	\right) }

\newcommand{\Matrr}[4]{ \left(
	\begin{array}{cc}
	#1 & #2 \\
	#3 & #4
	\end{array}
	\right) }

\newcommand{\calB}{{\mathcal B}}
\newcommand{\calBB}{{\mathcal B}^{\Box}}
\newcommand{\calK}{\mathcal{K}}
\newcommand{\bfA}{{\mathbf A}}
\newcommand{\bfb}{{\mathbf b}}
\newcommand{\bfr}{{\mathbf r}}
\newcommand{\bfx}{{\mathbf x}}
\newcommand{\bfxex}{{\mathbf x}_{\mbox{\scriptsize $\star$}}}
\newcommand{\Dim}{\mathop{\mathrm{dim\ }}}

\begin{document}
	%%%%%%%%%%%%%%%%%%%%%%%%%%%%%%%%%%%%%%%%%%%

\begin{center}
{\large
{\bf The convergence of Krylov subspaces methods with recycling}}

	Eric de Sturler \\
	Department of Mathematics, 460 McBryde \\
        Virginia Tech, Blacksburg VA 24061-0123 \\
        {\tt sturler@vt.edu} \\
        Michael L.~Parks \\
\end{center}
Many problems in science and engineering require the
solution of a long sequence of linear systems, with small
changes from one matrix to the next but substantial changes
over multiple systems.  We are particularly interested in
cases where both the matrix and the right hand side change
and systems are not available simultaneously.  Such
sequences arise in time-dependent iterations, nonlinear
systems of equations and optimization, (distributed)
parameter identification and inverse problems, and many
other problems.

In recent papers [1,2,3] we have proposed methods to
recycle selected subspaces from the Krylov spaces generated
for previous linear systems to improve the convergence of
subsequent linear systems.  In this presentation, we
discuss several important convergence issues:
\begin{itemize}
\item the convergence of Krylov methods
that recycle approximate solution spaces, approximate
invariant subspaces, and other relevant spaces,
\item the relevant perturbation theory for the spaces
mentioned above for sequences of matrices arising in a
range of applications,
\item how fast our proposed methods learn to adapt to a
changing problem.
\end{itemize}
We provide experimental
results for a range of problems from tomography, nonlinear
mechanics, large-scale design optimization, and statistical
mechanics.

[1] Michael L. Parks, Eric de Sturler, Greg Mackey, Duane
D. Johnson, and Spandan Maiti,
{\em Recycling Krylov Subspaces for Sequences of Linear
Systems}, SIAM Journal on Scientific Computing
(accepted with minor revisions), 2006, available as
Tech. Report UIUCDCS-R-2004-2421, March 2004, from
\verb9http://www-faculty.cs.uiuc.edu/~sturler9.

[2]
Misha Kilmer and Eric de Sturler,
{\em Recycling Subspace Information for Diffuse Optical
Tomography},
SIAM Journal on Scientific Computing (accepted for
publication),
2006, available from
\verb9http://www-faculty.cs.uiuc.edu/~sturler9.

[3]
Shun Wang, Eric de Sturler, and Glaucio H. Paulino,
{\em Large-Scale Topology Optimization using Preconditioned
Krylov Subspace Methods with Recycling},
International Journal for Numerical Methods in
Engineering (submitted), 2006,
available as Technical Report UIUCDCS-R-2006-2678
from
\verb9http://www-faculty.cs.uiuc.edu/~sturler9.



	%%%%%%%%%%%%%%%%%%%%%%%%%%%%%%%%%%%%%%%%%%%


\end{document}

