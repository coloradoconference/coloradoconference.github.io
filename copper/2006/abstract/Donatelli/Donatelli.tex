\documentclass{report}
\usepackage{amsmath,amssymb}

\def\mathbi#1{\textbf{\em #1}}
\def\BA{\bf{A}}
\def\BB{\bf{B}}
\def\bn{\bf{n}}
\def\CP{\mathcal{P}}
\def\CV{\mathcal{V}}
\def\CI{\mathcal{I}}

\newcommand{\gradt}{\nabla\cdot}
\newcommand{\Reals}{\mathbb{R}}
\newcommand{\Cplex}{\mathbb{C}}

\newcommand{\Vecc}[2]{ \left(
	\begin{array}{c}
	#1 \\ #2
	\end{array}
	\right) }

\newcommand{\Matrr}[4]{ \left(
	\begin{array}{cc}
	#1 & #2 \\
	#3 & #4
	\end{array}
	\right) }

\newcommand{\calB}{{\mathcal B}}
\newcommand{\calBB}{{\mathcal B}^{\Box}}
\newcommand{\calK}{\mathcal{K}}
\newcommand{\bfA}{{\mathbf A}}
\newcommand{\bfb}{{\mathbf b}}
\newcommand{\bfr}{{\mathbf r}}
\newcommand{\bfx}{{\mathbf x}}
\newcommand{\bfxex}{{\mathbf x}_{\mbox{\scriptsize $\star$}}}
\newcommand{\Dim}{\mathop{\mathrm{dim\ }}}

\begin{document}
	%%%%%%%%%%%%%%%%%%%%%%%%%%%%%%%%%%%%%%%%%%%

\begin{center}
{\large
{\bf Filter factor analysis of an iterative multilevel regularizing method}}

	Marco Donatelli \\
	Dipartimento de Fisica e Matematica,
	Universit\`a dell'Insubria \\
	Sede di Como, Via Valleggio 11, 22100 Como, Italy \\
	{\tt marco.donatelli@uninsubria.it}
\end{center}
Recent results have shown that iterative methods of multigrid
type are very effective for regularizing purposes.
In short, the reconstruction quality is of the same level
or slightly better than that related to well-known
regularizing procedures such as Landweber or conjugate
gradient (CG) for normal equations, but the associated
computational cost is highly reduced.

Here we analyze the filter features of one of these
multigrid techniques in order to provide a theoretical
motivation for the excellent regularizing characteristics
experimentally observed in the discussed methods.


	%%%%%%%%%%%%%%%%%%%%%%%%%%%%%%%%%%%%%%%%%%%


\end{document}

