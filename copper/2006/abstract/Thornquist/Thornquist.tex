\documentclass{report}
\usepackage{amsmath,amssymb}

\def\mathbi#1{\textbf{\em #1}}
\def\BA{\bf{A}}
\def\BB{\bf{B}}
\def\bn{\bf{n}}
\def\CP{\mathcal{P}}
\def\CV{\mathcal{V}}
\def\CI{\mathcal{I}}

\newcommand{\gradt}{\nabla\cdot}
\newcommand{\Reals}{\mathbb{R}}
\newcommand{\Cplex}{\mathbb{C}}

\newcommand{\Vecc}[2]{ \left(
	\begin{array}{c}
	#1 \\ #2
	\end{array}
	\right) }

\newcommand{\Matrr}[4]{ \left(
	\begin{array}{cc}
	#1 & #2 \\
	#3 & #4
	\end{array}
	\right) }

\newcommand{\calB}{{\mathcal B}}
\newcommand{\calBB}{{\mathcal B}^{\Box}}
\newcommand{\calK}{\mathcal{K}}
\newcommand{\bfA}{{\mathbf A}}
\newcommand{\bfb}{{\mathbf b}}
\newcommand{\bfr}{{\mathbf r}}
\newcommand{\bfx}{{\mathbf x}}
\newcommand{\bfxex}{{\mathbf x}_{\mbox{\scriptsize $\star$}}}
\newcommand{\Dim}{\mathop{\mathrm{dim\ }}}

\begin{document}
	%%%%%%%%%%%%%%%%%%%%%%%%%%%%%%%%%%%%%%%%%%%

\begin{center}
{\large
{\bf Fixed-polynomial approximate spectral transformations for \\
	preconditioning the eigenvalue problem}}

	Heidi K.~Thornquist \\
	Sandia National Laboratories, P.O.~Box 5800, MS 0316 \\
	Albuquerque NM 87185-0316 \\
	{\tt hkthorn@sandia.gov} \\
	Danny C.~Sorensen
\end{center}
Arnoldi's method is often used to compute a few eigenvalues
and eigenvectors of large, sparse matrices. When the
eigenvalues of interest are not dominant or well-separated,
this method may suffer from slow convergence. Spectral
transformations are a common acceleration technique that
address this issue by introducing a modified eigenvalue
problem that is easier to solve than the original. This
modified problem accentuates the eigenvalues of interest,
but requires solving a linear system, which is
computationally expensive for large-scale eigenvalue
problems.

In this talk we will show how this expense can
be reduced through a preconditioning scheme that uses a
fixed-polynomial operator to approximate the spectral
transformation. Three different constructions for a
fixed-polynomial operator are derived from some common
iterative methods for non-Hermitian linear systems. The
implementation details and numerical behavior of these three
operators are compared. Numerical experiments will be
presented demonstrating that this preconditioning scheme is
a competitive approach for solving large-scale eigenvalue
problems. The results illustrate the effectiveness of this
technique using several practical eigenvalue problems from
science and engineering ranging from hundreds to more than a
million unknowns.



	%%%%%%%%%%%%%%%%%%%%%%%%%%%%%%%%%%%%%%%%%%%


\end{document}

