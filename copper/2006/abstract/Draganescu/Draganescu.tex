\documentclass{report}
\usepackage{amsmath,amssymb}

\def\mathbi#1{\textbf{\em #1}}
\def\BA{\bf{A}}
\def\BB{\bf{B}}
\def\bn{\bf{n}}
\def\CP{\mathcal{P}}
\def\CV{\mathcal{V}}
\def\CI{\mathcal{I}}

\newcommand{\gradt}{\nabla\cdot}
\newcommand{\Reals}{\mathbb{R}}
\newcommand{\Cplex}{\mathbb{C}}

\newcommand{\Vecc}[2]{ \left(
	\begin{array}{c}
	#1 \\ #2
	\end{array}
	\right) }

\newcommand{\Matrr}[4]{ \left(
	\begin{array}{cc}
	#1 & #2 \\
	#3 & #4
	\end{array}
	\right) }

\newcommand{\calB}{{\mathcal B}}
\newcommand{\calBB}{{\mathcal B}^{\Box}}
\newcommand{\calK}{\mathcal{K}}
\newcommand{\bfA}{{\mathbf A}}
\newcommand{\bfb}{{\mathbf b}}
\newcommand{\bfr}{{\mathbf r}}
\newcommand{\bfx}{{\mathbf x}}
\newcommand{\bfxex}{{\mathbf x}_{\mbox{\scriptsize $\star$}}}
\newcommand{\Dim}{\mathop{\mathrm{dim\ }}}

\begin{document}
	%%%%%%%%%%%%%%%%%%%%%%%%%%%%%%%%%%%%%%%%%%%

\begin{center}
{\large
{\bf An abstract method for extending two-level preconditioners \\
to multilevel preconditioners of comparable quality}}

        Andrei Dr{\u a}g{\u a}nescu \\
        Sandia National Laboratories,
        Optimization/Uncertainty Estimation Department \\
        P.O. Box 5800, MS. 1110, Albuquerque NM 87185-1110 \\
                        \hspace*{9mm}\hfill
        {\tt aidraga@sandia.gov}
                        \hfill  \hyperlink{index}{\small {\em
index}} \\
\end{center}

We present an abstract method for designing a
multilevel
preconditioner given a two-level preconditioner for
an
operator with positive definite symmetric part.

If we denote by $A_{\mathrm{fine}}$ and
$A_{\mathrm{coarse}}$ two discrete versions of a continuous
operator, then a two-level preconditioner
$T_{\mathrm{fine}}$ for $A_{\mathrm{fine}}$ can be
described in general by a function
$T_{\mathrm{fine}} = {\mathcal F}(A_{\mathrm{coarse}}^{-1}$,
$A_{\mathrm{fine}})$,
where it is assumed that the evaluation of ${\mathcal F}$
requires a level-independent  number of applications of
$A_{\mathrm{fine}}$ and $k$  applications of
$A_{\mathrm{coarse}}^{-1}$  ($k=1$ or $2$).  The natural
extension to a multilevel preconditioner, consisting in
replacing in $T_{\mathrm{fine}}$ the call to
$A_{\mathrm{coarse}}^{-1}$ with a recursive call to
${\mathcal F}$, is known to sometimes produce multilevel
preconditioners of lower quality (e.g., for certain types
of inverse problems).  Based on the idea that inverting
$A_{\mathrm{fine}}$ essentially means to solve the
nonlinear  equation \mbox{$X^{-1} - A_{\mathrm{fine}}= 0$},
we define our multigrid preconditioner to be the first
Newton iterate of the map $X\mapsto X^{-1} -
A_{\mathrm{fine}}$ starting at the ``natural'' multilevel
preconditioner.  For $k=1$, the resulting algorithm has a
W-cycle structure, and differs only slightly from the
textbook version of the W-cycle.  Moreover, the method
guarantees that the resulting preconditioner maintains the
approximation quality of the initial two-level
preconditioner.  The quality of approximation is measured
using a certain distance function, which determines the
degree to which two operators with positive definite 
symmetric parts are spectrally equivalent.  We apply this
method to designing and analyzing a multigrid
preconditioner for a linear advection-diffusion-reaction
equation. 

{\small Sandia is a multiprogram laboratory operated by
Sandia Corporation, a Lockheed-Martin Company, for the
United States Department of Energy under 
Contract DE-AC04-94AL85000.}




	%%%%%%%%%%%%%%%%%%%%%%%%%%%%%%%%%%%%%%%%%%%


\end{document}

