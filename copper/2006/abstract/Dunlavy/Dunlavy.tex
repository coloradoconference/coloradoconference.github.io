\documentclass{report}
\usepackage{amsmath,amssymb}

\def\mathbi#1{\textbf{\em #1}}
\def\BA{\bf{A}}
\def\BB{\bf{B}}
\def\bn{\bf{n}}
\def\CP{\mathcal{P}}
\def\CV{\mathcal{V}}
\def\CI{\mathcal{I}}

\newcommand{\gradt}{\nabla\cdot}
\newcommand{\Reals}{\mathbb{R}}
\newcommand{\Cplex}{\mathbb{C}}

\newcommand{\Vecc}[2]{ \left(
	\begin{array}{c}
	#1 \\ #2
	\end{array}
	\right) }

\newcommand{\Matrr}[4]{ \left(
	\begin{array}{cc}
	#1 & #2 \\
	#3 & #4
	\end{array}
	\right) }

\newcommand{\calB}{{\mathcal B}}
\newcommand{\calBB}{{\mathcal B}^{\Box}}
\newcommand{\calK}{\mathcal{K}}
\newcommand{\bfA}{{\mathbf A}}
\newcommand{\bfb}{{\mathbf b}}
\newcommand{\bfr}{{\mathbf r}}
\newcommand{\bfx}{{\mathbf x}}
\newcommand{\bfxex}{{\mathbf x}_{\mbox{\scriptsize $\star$}}}
\newcommand{\Dim}{\mathop{\mathrm{dim\ }}}

\begin{document}
	%%%%%%%%%%%%%%%%%%%%%%%%%%%%%%%%%%%%%%%%%%%

\begin{center}
{\large
{\bf Global optimization: for some problems, there's HOPE}}

	Daniel Dunlavy \\
	Sandia National Laboratories \\
	P.O.~Box 5800 M/S 1110, Albuquerque NM 87185-1110 \\
	{\tt dmdunla@sandia.gov} \\
	Dianne O'Leary
\end{center}
We present a new method for solving unconstrained
minimization problems---Homotopy Optimization with
Perturbations and Ensembles (HOPE). HOPE is a homotopy
optimization method that finds a sequence of minimizers of a
homotopy function mapping a template function to the target
function, the objective function of our minimization
problem. To increase the likelihood of finding a global
minimizer, points in the sequence are perturbed and used as
starting points to find other minimizers. Points in the
resulting ensemble of minimizers are used as starting points
to find minimizers of the homotopy function as it deforms
the template function into the target function.

We show
that certain choices of the parameters used in HOPE lead to
instances of existing methods: probability-one homotopy
methods, stochastic search methods, and simulated annealing.
We use these relations and further analysis to demonstrate
the convergence properties of HOPE.

The development of
HOPE was motivated by the protein folding problem, the
problem of predicting the structure of a protein as it
exists in nature, given its amino acid sequence. However, we
demonstrate that HOPE is also successful as a general
purpose minimization method for nonconvex functions.


Numerical experiments performed to test HOPE include solving
several standard test problems and the protein folding
problem using two different protein models. In most of these
experiments, standard homotopy functions are used in HOPE.
Additionally, several new homotopy functions are introduced
for solving the protein folding problems to demonstrate how
HOPE can be used to exploit the properties or structure of
particular problems.

Results of experiments demonstrate
that HOPE outperforms several methods often used for solving
unconstrained minimization problems---a quasi-Newton method
with BFGS Hessian update, a globally convergent variant of
Newton's method, and ensemble-based simulated annealing.



	%%%%%%%%%%%%%%%%%%%%%%%%%%%%%%%%%%%%%%%%%%%


\end{document}

