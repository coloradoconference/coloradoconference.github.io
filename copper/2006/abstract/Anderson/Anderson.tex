\documentclass{report}
\usepackage{amsmath,amssymb}

\def\mathbi#1{\textbf{\em #1}}
\def\BA{\bf{A}}
\def\BB{\bf{B}}
\def\bn{\bf{n}}
\def\CP{\mathcal{P}}
\def\CV{\mathcal{V}}
\def\CI{\mathcal{I}}

\newcommand{\gradt}{\nabla\cdot}
\newcommand{\Reals}{\mathbb{R}}
\newcommand{\Cplex}{\mathbb{C}}

\newcommand{\Vecc}[2]{ \left(
	\begin{array}{c}
	#1 \\ #2
	\end{array}
	\right) }

\newcommand{\Matrr}[4]{ \left(
	\begin{array}{cc}
	#1 & #2 \\
	#3 & #4
	\end{array}
	\right) }

\newcommand{\calB}{{\mathcal B}}
\newcommand{\calBB}{{\mathcal B}^{\Box}}
\newcommand{\calK}{\mathcal{K}}
\newcommand{\bfA}{{\mathbf A}}
\newcommand{\bfb}{{\mathbf b}}
\newcommand{\bfr}{{\mathbf r}}
\newcommand{\bfx}{{\mathbf x}}
\newcommand{\bfxex}{{\mathbf x}_{\mbox{\scriptsize $\star$}}}
\newcommand{\Dim}{\mathop{\mathrm{dim\ }}}

\begin{document}
	%%%%%%%%%%%%%%%%%%%%%%%%%%%%%%%%%%%%%%%%%%%

\begin{center}
{\large
{\bf An additive Schwarz parallel approach to space-time \\
finite elements for hyperbolic equations}}

	Matthew Anderson \\
	Louisiana State University Dept.~of Physics \& Astronomy \\
	202 Nicholson Hall Tower Drive,  Baton Rouge LA 70803-4001 \\
	{\tt matt@phys.lsu.edu} \\
	Jung-Han Kimn
\end{center}
We study a time parallel space-time finite element approach
for the nonhomogeneous wave equation using a continuous time
Galerkin method and a time decomposition strategy for
preconditioning. Space-time finite elements provide some
natural advantages for numerical relativity in black hole
simulations. With space-time elements, time-varying
computational domains are straightforward, higher-order
approaches are easily formulated, and both time and spatial
domains can be discretized using a more general mesh. We
present fully implicit examples in $1+1$, $2+1$, and $3+1$
dimensions using linear quadrilateral, hexahedral, and
tesseractic elements. Krylov solvers with additive Schwarz
preconditioning are used for solving the linear system. We
introduce a time decomposition strategy in preconditioning
which significantly improves performance when compared with
unpreconditioned cases. Parallel performance results are
also given.




	%%%%%%%%%%%%%%%%%%%%%%%%%%%%%%%%%%%%%%%%%%%


\end{document}

