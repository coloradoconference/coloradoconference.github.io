\documentclass{report}
\usepackage{amsmath,amssymb}

\def\mathbi#1{\textbf{\em #1}}
\def\BA{\bf{A}}
\def\BB{\bf{B}}
\def\bn{\bf{n}}
\def\CP{\mathcal{P}}
\def\CV{\mathcal{V}}
\def\CI{\mathcal{I}}

\newcommand{\gradt}{\nabla\cdot}
\newcommand{\Reals}{\mathbb{R}}
\newcommand{\Cplex}{\mathbb{C}}

\newcommand{\Vecc}[2]{ \left(
	\begin{array}{c}
	#1 \\ #2
	\end{array}
	\right) }

\newcommand{\Matrr}[4]{ \left(
	\begin{array}{cc}
	#1 & #2 \\
	#3 & #4
	\end{array}
	\right) }

\newcommand{\calB}{{\mathcal B}}
\newcommand{\calBB}{{\mathcal B}^{\Box}}
\newcommand{\calK}{\mathcal{K}}
\newcommand{\bfA}{{\mathbf A}}
\newcommand{\bfb}{{\mathbf b}}
\newcommand{\bfr}{{\mathbf r}}
\newcommand{\bfx}{{\mathbf x}}
\newcommand{\bfxex}{{\mathbf x}_{\mbox{\scriptsize $\star$}}}
\newcommand{\Dim}{\mathop{\mathrm{dim\ }}}

\begin{document}
	%%%%%%%%%%%%%%%%%%%%%%%%%%%%%%%%%%%%%%%%%%%

\begin{center}
{\large
{\bf Multi-level optimization for image registration using \\
	local refinement on octrees}}

	Stefan Heldman \\
	Dept.~of Mathematics and Computer Science, Emory University \\
	400 Dowman Drive, Atlanta Georgia 30322 \\
	{\tt heldmann@mathcs.emory.edu}
	\\ Eldad Haber, Jan Modersitzki
\end{center}
We present a new multi-level approach for non-linear image
registration using local refinement techniques.

Standard multi-level approaches for this problem discretize
the domain starting with a regular coarse grid and refine
every cell from level to level. In our approach, we also
start with a regular coarse grid but its refinement for
higher levels is done locally. Using local refinement is
motivated by the observation that changes in the solution at
higher levels appear mainly locally and large areas stay
unchanged such that there is no need for a finer resolution.
The local refinement in our approach is done by subdividing
cells into four (2D) or eight (3D) resulting quad (2D) and
octree-grids (3D), respectively.

Compared with the standard multi-level approach, our method
requires substantially less memory and arithmetic
operations. Therefore, it is in particular well-suited for
large-scale problems.



	%%%%%%%%%%%%%%%%%%%%%%%%%%%%%%%%%%%%%%%%%%%


\end{document}

