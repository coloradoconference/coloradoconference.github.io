\documentclass{report}
\usepackage{amsmath,amssymb}

\def\mathbi#1{\textbf{\em #1}}
\def\BA{\bf{A}}
\def\BB{\bf{B}}
\def\bn{\bf{n}}
\def\CP{\mathcal{P}}
\def\CV{\mathcal{V}}
\def\CI{\mathcal{I}}

\newcommand{\gradt}{\nabla\cdot}
\newcommand{\Reals}{\mathbb{R}}
\newcommand{\Cplex}{\mathbb{C}}

\newcommand{\Vecc}[2]{ \left(
	\begin{array}{c}
	#1 \\ #2
	\end{array}
	\right) }

\newcommand{\Matrr}[4]{ \left(
	\begin{array}{cc}
	#1 & #2 \\
	#3 & #4
	\end{array}
	\right) }

\newcommand{\calB}{{\mathcal B}}
\newcommand{\calBB}{{\mathcal B}^{\Box}}
\newcommand{\calK}{\mathcal{K}}
\newcommand{\bfA}{{\mathbf A}}
\newcommand{\bfb}{{\mathbf b}}
\newcommand{\bfr}{{\mathbf r}}
\newcommand{\bfx}{{\mathbf x}}
\newcommand{\bfxex}{{\mathbf x}_{\mbox{\scriptsize $\star$}}}
\newcommand{\Dim}{\mathop{\mathrm{dim\ }}}

\begin{document}
	%%%%%%%%%%%%%%%%%%%%%%%%%%%%%%%%%%%%%%%%%%%

\begin{center}
{\large
{\bf A mathematical framework for equivalent real formulations}}

	Sarah M.~Knepper \\
	37 N College Ave, Box 787 \\
	Saint Joseph MN 56374 \\
	{\tt smknepper@csbsju.edu} \\
	Michael A.~Heroux
\end{center}
Equivalent real formulations (ERFs) are useful for solving
complex linear systems using real solvers. Using the four
ERFs discussed by Day and Heroux [1], each can be
expressed by multiplying the \emph{canonical} $K$ form of
the complex matrix by certain diagonal and permutation
matrices on either side. This will allow, for instance, one
ERF to be used as a preconditioner and another ERF to be
used to iteratively solve the linear system by simply
switching back and forth between the forms through scaling
and permuting.

Many real world problems result in a
complex-valued linear system of the form
$C w = d$, where $C$ is a known
$m\times n$ complex matrix, $d$ is a known $m\times 1$
complex vector, and $w$ is an unknown $n\times 1$
complex vector. We
can re-write $C$ as a real matrix of size $2m\times 2n$
called the \emph{canonical} $K$ form. If we let matrix $A$
contain the real parts of the complex matrix $C$ and let
matrix $B$ consist of the corresponding imaginary parts, we
can write $A + i B = C$.
The \emph{canonical} $K$ form is created by
forming the matrix
$K = \Matrr{A}{-B}{B}{A}$.

To preserve the sparsity pattern of $C$,
each complex value $c_{pq} = a_{pq} + ib_{pq}$ is converted
into a $2\times 2$ sub-block with the structure
$ \Matrr{a_{pq}}{-b_{pq}}{b_{pq}}{a_{pq}}.$
For instance, if
$$C = \Matrr{a_{11} + i b_{11}}{a_{12} + i b_{12}}{0}{a_{22} + i b_{22}}$$
then the \emph{permuted canonical} $K$ form is
$$K = \left(
\begin{array}{cccc} a_{11} & -b_{11} & a_{12} & -b_{12} \\
b_{11} & a_{11} & b_{12} & a_{12} \\ 0 & 0 & a_{22} &
-b_{22} \\ 0 & 0 & b_{22} & a_{22} \end{array} \right).
$$

The four ERFs that we will concern
ourselves with are:
$K_1 = \Matrr{A}{-B}{B}{A}$,
$K_2 = \Matrr{A}{B}{B}{-A}$,
$K_3 = \Matrr{B}{A}{A}{-B}$, and
$K_4 = \Matrr{B}{-A}{A}{B}$.
Each of the ERFs can be obtained from the
\emph{permuted canonical} $K$ form by multiplying by
diagonal and permutation matrices on both sides.
In other words,
$K_i = D_l P_l K P_r D_r$,
where $D_l$, $P_l$, $P_r$, and $D_r$ are
certain matrices depending on the size of the complex matrix
and which ERF we desire. Three diagonal matrices and two
permutation matrices (together with their transposes) exist
for the ERFs we are considering.

The talk will describe
the specific diagonal and permutation matrices needed as
well as how to transform from one ERF to another.

[1] David Day and Michael A.~Heroux,
{\em Solving Complex-Valued Linear Systems
via Equivalent Real Formulations},
SIAM J.~Sci. Comput. {\bf 23}(2) (2001) 480--498.



	%%%%%%%%%%%%%%%%%%%%%%%%%%%%%%%%%%%%%%%%%%%


\end{document}

