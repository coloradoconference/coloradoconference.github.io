\documentclass{report}
\usepackage{amsmath,amssymb}

\def\mathbi#1{\textbf{\em #1}}
\def\BA{\bf{A}}
\def\BB{\bf{B}}
\def\bn{\bf{n}}
\def\CP{\mathcal{P}}
\def\CV{\mathcal{V}}
\def\CI{\mathcal{I}}

\newcommand{\gradt}{\nabla\cdot}
\newcommand{\Reals}{\mathbb{R}}
\newcommand{\Cplex}{\mathbb{C}}

\newcommand{\Vecc}[2]{ \left(
	\begin{array}{c}
	#1 \\ #2
	\end{array}
	\right) }

\newcommand{\Matrr}[4]{ \left(
	\begin{array}{cc}
	#1 & #2 \\
	#3 & #4
	\end{array}
	\right) }

\newcommand{\calB}{{\mathcal B}}
\newcommand{\calBB}{{\mathcal B}^{\Box}}
\newcommand{\calK}{\mathcal{K}}
\newcommand{\bfA}{{\mathbf A}}
\newcommand{\bfb}{{\mathbf b}}
\newcommand{\bfr}{{\mathbf r}}
\newcommand{\bfx}{{\mathbf x}}
\newcommand{\bfxex}{{\mathbf x}_{\mbox{\scriptsize $\star$}}}
\newcommand{\Dim}{\mathop{\mathrm{dim\ }}}

\begin{document}
	%%%%%%%%%%%%%%%%%%%%%%%%%%%%%%%%%%%%%%%%%%%

\begin{center}
{\large
{\bf A priori error bounds for eigenvalues approximated by the Ritz values}}

	Merico E.~Argentati \\
	Dept.~of Mathematical Sciences \\
	 University of Colorado at Denver and Health Sciences Center \\
	 P.O.~Box 173364, Campus Box 170, Denver CO 80217-3364 \\
	{\tt rargenta@math.cudenver.edu} \\
	Andrew V.~Knyazev
\end{center}
The Rayleigh-Ritz method finds the stationary values of the
Rayleigh quotient, called Ritz values, on a given trial
subspace as optimal, in some sense, approximations to
eigenvalues of a Hermitian operator $A$. When a trial subspace
is invariant with respect to $A$, the Ritz values are some of
the eigenvalues of $A$. Given two finite dimensional subspaces
$X$ and $Y$ of the same dimension, such that $X$ is an invariant
subspace of $A$, the absolute changes in the Ritz values of $A$
with respect to $X$ compared to the Ritz values with respect
to $Y$ represent the absolute eigenvalue approximation error.

We estimate the error in terms of the principal angles
between $X$ and $Y$. There are several known results of this
kind, e.g., for the largest (or the smallest) eigenvalues of
$A$, the maximal error is bounded by a constant times the sine
squared of the largest principal angle between $X$ and $Y$. The
constant is the difference between the largest and the
smallest eigenvalues of $A$, called the spread of the spectrum
of $A$.

We prove that the absolute eigenvalue error is
majorized by a constant times the squares of the sines of
the principal angles between the subspaces $X$ and $Y$, where
the constant is proportional to the spread of the spectrum
of $A$, e.g., for Ritz values that are the largest or smallest
contiguous set of eigenvalues of $A$, we show that the
proportionality factor is simply one. Our majorization
results imply a very general set of inequalities, and some
of the known error bounds follow as special cases.
Majorization results of this kind are not apparently known
in the literature and can be used, e.g., to derive novel
convergence rate estimates of the block Lanczos method.




	%%%%%%%%%%%%%%%%%%%%%%%%%%%%%%%%%%%%%%%%%%%


\end{document}

