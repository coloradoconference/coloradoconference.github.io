\documentclass{report}
\usepackage{amsmath,amssymb}

\def\mathbi#1{\textbf{\em #1}}
\def\BA{\bf{A}}
\def\BB{\bf{B}}
\def\bn{\bf{n}}
\def\CP{\mathcal{P}}
\def\CV{\mathcal{V}}
\def\CI{\mathcal{I}}

\newcommand{\gradt}{\nabla\cdot}
\newcommand{\Reals}{\mathbb{R}}
\newcommand{\Cplex}{\mathbb{C}}

\newcommand{\Vecc}[2]{ \left(
	\begin{array}{c}
	#1 \\ #2
	\end{array}
	\right) }

\newcommand{\Matrr}[4]{ \left(
	\begin{array}{cc}
	#1 & #2 \\
	#3 & #4
	\end{array}
	\right) }

\newcommand{\calB}{{\mathcal B}}
\newcommand{\calBB}{{\mathcal B}^{\Box}}
\newcommand{\calK}{\mathcal{K}}
\newcommand{\bfA}{{\mathbf A}}
\newcommand{\bfb}{{\mathbf b}}
\newcommand{\bfr}{{\mathbf r}}
\newcommand{\bfx}{{\mathbf x}}
\newcommand{\bfxex}{{\mathbf x}_{\mbox{\scriptsize $\star$}}}
\newcommand{\Dim}{\mathop{\mathrm{dim\ }}}

\begin{document}
	%%%%%%%%%%%%%%%%%%%%%%%%%%%%%%%%%%%%%%%%%%%

\begin{center}
{\large
{\bf Compatible relaxation and coarsening in algebraic multigrid}}

	James Brannick \\
	Institute for Scientific Computing Research \\
	Lawrence Livermore National Laboratory \\
	Box 808, L551, Livermore CA 94551 \\
	{\tt brannick@colorado.edu} \\
	Steve McCormick, Marian Brezina,
	Rob Falgout, Tom Manteuffel, John Ruge, Ludmil Zikatanov
\end{center}
Algebraic multigrid (AMG) has been shown to be an efficient
iterative solver for many of the large and sparse linear
systems arising from the discretization of partial
differential equations. There remain, however, many classes
of problems for which conventional AMG setup algorithms,
based on the properties of M-matrices, are inappropriate. In
this talk, we consider the use of compatible relaxation (CR)
as a tool for extending the applicability of AMG. A CR-based
coarsening algorithm is presented along with numerical
results demonstrating that the variational multigrid solver
resulting from the proposed approach maintains
multigrid-like optimality, without the need for parameter
tuning, for some problems where current algorithms exhibit
degraded performance.



	%%%%%%%%%%%%%%%%%%%%%%%%%%%%%%%%%%%%%%%%%%%


\end{document}

