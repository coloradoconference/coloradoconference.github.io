\documentclass{report}
\usepackage{amsmath,amssymb}

\def\mathbi#1{\textbf{\em #1}}
\def\BA{\bf{A}}
\def\BB{\bf{B}}
\def\bn{\bf{n}}
\def\CP{\mathcal{P}}
\def\CV{\mathcal{V}}
\def\CI{\mathcal{I}}

\newcommand{\gradt}{\nabla\cdot}
\newcommand{\Reals}{\mathbb{R}}
\newcommand{\Cplex}{\mathbb{C}}

\newcommand{\Vecc}[2]{ \left(
	\begin{array}{c}
	#1 \\ #2
	\end{array}
	\right) }

\newcommand{\Matrr}[4]{ \left(
	\begin{array}{cc}
	#1 & #2 \\
	#3 & #4
	\end{array}
	\right) }

\newcommand{\calB}{{\mathcal B}}
\newcommand{\calBB}{{\mathcal B}^{\Box}}
\newcommand{\calK}{\mathcal{K}}
\newcommand{\bfA}{{\mathbf A}}
\newcommand{\bfb}{{\mathbf b}}
\newcommand{\bfr}{{\mathbf r}}
\newcommand{\bfx}{{\mathbf x}}
\newcommand{\bfxex}{{\mathbf x}_{\mbox{\scriptsize $\star$}}}
\newcommand{\Dim}{\mathop{\mathrm{dim\ }}}

\begin{document}
	%%%%%%%%%%%%%%%%%%%%%%%%%%%%%%%%%%%%%%%%%%%

\begin{center}
{\large
{\bf Iterative solution techniques for flexible approximation schemes \\
	in multiparticle simulations}}

	Masha Sosonkina \\
	Ames Laboratory DOE, 236 Wilhelm Hall \\
	Ames IA 50011 \\
	{\tt masha@scl.ameslab.gov} \\
	Igor Tsukerman, Elena Ivanova, Sergey Voskoboynikov
\end{center}
The paper examines various parallel iterative solvers
for the new Flexible Local Approximation MEthod (FLAME)
[5,6] applied to colloidal
systems. The electrostatic potential in such  systems
can be described, at least for monovalent salts in the
solvent, by the Poisson-Boltzmann equation
(see, e.g., [2]).
Classical Finite-Difference (FD)
schemes would require unreasonably fine meshes to
represent the boundaries of multiple spherical
particles at arbitrary locations with sufficient
accuracy. In the Finite Element Method, mesh generation
for a large number of particles becomes impractical.
The Fast Multipole Method works well only if the
particle sizes are neglected and the Poisson-Boltzmann
equation is linearized [1].  Classical
FD schemes rely on Taylor expansions that break down
near material interfaces (such as particle boundaries)
due to the lack of smoothness  of the field. In FLAME,
Taylor expansions in the vicinity of the particles  are
replaced with much more accurate approximations.
Namely, the local FLAME bases are constructed by
matching (via the boundary conditions) the spherical
harmonics for the electrostatic potential inside and
outside the particle;
see [5,6] for details.

The system matrices of FLAME and classical FD have the
same sparsity structure for the same grid stencil on a
regular Cartesian grid; for example, the standard
seven-point stencil leads to a seven-diagonal matrix.
However, the FLAME matrix is generally nonsymmetric.
Several parallel iterative solution techniques have
been tested with an emphasis on suitable parallel
preconditioning for the nonsymmetric system matrix. In
particular, flexible GMRES [3]
preconditioned with the distributed Schur Complement
[4] has been considered and
compared with Additive Schwarz and global incomplete
ILU(0) preconditionings.  It has been observed that
Schur Complement preconditioning with a small amount of
fill and a few inner iterations scales well and
exhibits good solution times while attaining almost
linear speedup. The number of iterations and the
computational time depends only mildly on the Debye
parameter of the electrolyte.

[1]
L.~F.~Greengard, J.~Huang,
{\em A new version of the {Fast Multipole Method}
for screened {Coulomb} interactions in three
dimensions}, J.~Comput. Phys. {\bf 180}(23)
(2002) 642--658.

[2]
A.~Yu.~Grosberg, T.~T.~Nguyen, B.~I.~Shklovskii,
{\em Colloquium: The physics of charge inversion
in chemical and biological systems},
Reviews of Modern Physics {\bf 74}(2)
(2002) 329--345.

[3]
Y.~Saad, {\em Iterative Methods for Sparse Linear
Systems, 2nd edition}, SIAM (2002) Philadelphia PA.

[4]
Y.~Saad, M.~Sosonkina, {\em Distributed
{Schur Complement} techniques for general sparse
linear systems}, SIAM J.~Sci. Comp. {\bf 21}(4)
(1999) 1337--1356.

[5]
I.~Tsukerman, {\em Electromagnetic applications
of a new finite-difference calculus},
IEEE Trans. Magn. {\bf 41}(7) (2005) 2206--2225.

[6]
I.~Tsukerman, {\em A class of difference schemes with
flexible local approximation}, J.~Comp. Phys.
{\bf 211}(2) (2006) 659--699.



	%%%%%%%%%%%%%%%%%%%%%%%%%%%%%%%%%%%%%%%%%%%


\end{document}

