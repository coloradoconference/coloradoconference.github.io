\documentclass{report}
\usepackage{amsmath,amssymb}

\def\mathbi#1{\textbf{\em #1}}
\def\BA{\bf{A}}
\def\BB{\bf{B}}
\def\bn{\bf{n}}
\def\CP{\mathcal{P}}
\def\CV{\mathcal{V}}
\def\CI{\mathcal{I}}

\newcommand{\gradt}{\nabla\cdot}
\newcommand{\Reals}{\mathbb{R}}
\newcommand{\Cplex}{\mathbb{C}}

\newcommand{\Vecc}[2]{ \left(
	\begin{array}{c}
	#1 \\ #2
	\end{array}
	\right) }

\newcommand{\Matrr}[4]{ \left(
	\begin{array}{cc}
	#1 & #2 \\
	#3 & #4
	\end{array}
	\right) }

\newcommand{\calB}{{\mathcal B}}
\newcommand{\calBB}{{\mathcal B}^{\Box}}
\newcommand{\calK}{\mathcal{K}}
\newcommand{\bfA}{{\mathbf A}}
\newcommand{\bfb}{{\mathbf b}}
\newcommand{\bfr}{{\mathbf r}}
\newcommand{\bfx}{{\mathbf x}}
\newcommand{\bfxex}{{\mathbf x}_{\mbox{\scriptsize $\star$}}}
\newcommand{\Dim}{\mathop{\mathrm{dim\ }}}

\begin{document}
	%%%%%%%%%%%%%%%%%%%%%%%%%%%%%%%%%%%%%%%%%%%

\begin{center}
{\large
{\bf Solving bordered systems of linear equations for large-scale \\
	continuation and bifurcation analysis}}

	Eric T Phipps \\
	Sandia National Laboratories, Applied Computational Methods Dept. \\
	P.O.~Box 5800, MS-0316, Albuquerque NM 87185 \\
	{\tt etphipp@sandia.gov} \\
	Andrew G Salinger
\end{center}
Solving bordered systems of linear equations where the
matrix is augmented by a small number of additional rows and
columns is ubiquitous in continuation and bifurcation
analysis. Examples include pseudo-arclength continuation,
constraint following, and turning point location. However
solving these systems in a large-scale setting where the
original matrix is large and sparse is difficult. Directly
augmenting the matrix destroys the sparsity structure of the
original matrix since the additional rows and columns are
usually dense, while block elimination methods have
difficulty when the original matrix is nearly singular and
result in additional linear solves.

In this talk we
discuss a simple method for solving systems of this form
using Krylov iterative linear solvers based on computing the
$QR$ factorization of the augmented rows, and is an
extension of the Householder pseudo-arclength continuation
method developed by H.~Walker. It allows solutions of the
bordered system to be computed with a cost roughly
equivalent to solving the original matrix and is
well-conditioned even when the original matrix is singular.


We then apply this technique to the problem of computing
turning point bifurcations in large-scale nonlinear systems.
The $QR$ approach allows turning point algorithms that are
faster, more robust and scale better to millions of unknowns
compared to traditional block elimination schemes. Examples
of applying these techniques to large-scale structural and
fluid mechanics problems will be presented. These techniques
have been implemented in a continuation and bifurcation
software package called LOCA, short for The Library of
Continuation Algorithms, developed by the authors and
publicly available as a part of Trilinos, a set of scalable
linear and nonlinear solvers.



	%%%%%%%%%%%%%%%%%%%%%%%%%%%%%%%%%%%%%%%%%%%


\end{document}

