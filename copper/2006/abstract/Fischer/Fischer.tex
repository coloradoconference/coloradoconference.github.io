\documentclass{report}
\usepackage{amsmath,amssymb}

\def\mathbi#1{\textbf{\em #1}}
\def\BA{\bf{A}}
\def\BB{\bf{B}}
\def\bn{\bf{n}}
\def\CP{\mathcal{P}}
\def\CV{\mathcal{V}}
\def\CI{\mathcal{I}}

\newcommand{\gradt}{\nabla\cdot}
\newcommand{\Reals}{\mathbb{R}}
\newcommand{\Cplex}{\mathbb{C}}

\newcommand{\Vecc}[2]{ \left(
	\begin{array}{c}
	#1 \\ #2
	\end{array}
	\right) }

\newcommand{\Matrr}[4]{ \left(
	\begin{array}{cc}
	#1 & #2 \\
	#3 & #4
	\end{array}
	\right) }

\newcommand{\calB}{{\mathcal B}}
\newcommand{\calBB}{{\mathcal B}^{\Box}}
\newcommand{\calK}{\mathcal{K}}
\newcommand{\bfA}{{\mathbf A}}
\newcommand{\bfb}{{\mathbf b}}
\newcommand{\bfr}{{\mathbf r}}
\newcommand{\bfx}{{\mathbf x}}
\newcommand{\bfxex}{{\mathbf x}_{\mbox{\scriptsize $\star$}}}
\newcommand{\Dim}{\mathop{\mathrm{dim\ }}}

\begin{document}
	%%%%%%%%%%%%%%%%%%%%%%%%%%%%%%%%%%%%%%%%%%%

\begin{center}
{\large
{\bf Structured matrices, multigrid methods, and the Helmholtz equation}}

	Ranier Fischer \\
	Technical University of Munich \\
	Boltzmannstr. 3, 85748 Garching, Germany \\
	{\tt rainer.fischer@mytum.de} \\
	Thomas Huckle
\end{center}
We wish to present a different approach to the multigrid
solution of the Helmholtz equation with constant
coefficients. It is primarily based on certain classes of
structured matrices and their strong correspondence to
generating functions. Discretization of the Helmholtz
equation with certain boundary conditions results in
structured linear systems which are associated with
generating functions. Depending on the kind of boundary
conditions, the discretized Helmholtz equation is a linear
system of Toeplitz, tau, circulant, or DCT-III type. By
solving these systems with normal equations, we have the
advantage that the corresponding generating functions are
nonnegative, although they have a whole curve of zeros.


The multigrid methods we develop are especially designed for
structured matrix classes, making heavy use of the
associated generating functions. Over the last ten years, a
specific theory of multigrid methods has been developed for
structured matrices whose generating functions have isolated
zeros. It is based on the AMG approach and the convergence
theory of Ruge and St\"uben. In this work, we extend some of
these theoretical results to the case of generating
functions with whole zero curves, and apply these modified
multigrid methods to the solution of the Helmholtz equation.
We propose two different strategies how this can be done.
\begin{itemize}
\item The first strategy is based on the
idea of representing the whole zero curve on all grids. For
a multigrid method based on the Galerkin approach, we can
prove optimal two-grid convergence, but such a method is
computationally too expensive. Therefore we propose a
rediscretization technique where the zero curve is
approximated on each grid. This results in fast convergence
and in coarse grid matrices with the same banded structure
as the given matrix. The only disadvantage of this approach
is that zero curves become larger on coarser levels, and
therefore the number of grids is limited.
\item The second
strategy consists of splitting the original problem into a
fixed number of coarse grid problems. Corresponding to a
generating function with isolated zeros, each of these
problems locally represents one part of the zero curve. Each
coarse grid problem is solved with a standard multigrid
method. We combine this splitting technique with the first
strategy to construct a faster and more robust multigrid
solver.
\end{itemize}
We wish to discuss the use of our
multigrid strategies not only as a solver, but also as a
preconditioner for Krylov subspace methods. Moreover, the
methods can also be applied to anisotropic linear systems
whose generating functions have a whole zero curve.



	%%%%%%%%%%%%%%%%%%%%%%%%%%%%%%%%%%%%%%%%%%%


\end{document}

