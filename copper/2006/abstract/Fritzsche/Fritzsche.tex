\documentclass{report}
\usepackage{amsmath,amssymb}

\def\mathbi#1{\textbf{\em #1}}
\def\BA{\bf{A}}
\def\BB{\bf{B}}
\def\bn{\bf{n}}
\def\CP{\mathcal{P}}
\def\CV{\mathcal{V}}
\def\CI{\mathcal{I}}

\newcommand{\gradt}{\nabla\cdot}
\newcommand{\Reals}{\mathbb{R}}
\newcommand{\Cplex}{\mathbb{C}}

\newcommand{\Vecc}[2]{ \left(
	\begin{array}{c}
	#1 \\ #2
	\end{array}
	\right) }

\newcommand{\Matrr}[4]{ \left(
	\begin{array}{cc}
	#1 & #2 \\
	#3 & #4
	\end{array}
	\right) }

\newcommand{\calB}{{\mathcal B}}
\newcommand{\calBB}{{\mathcal B}^{\Box}}
\newcommand{\calK}{\mathcal{K}}
\newcommand{\bfA}{{\mathbf A}}
\newcommand{\bfb}{{\mathbf b}}
\newcommand{\bfr}{{\mathbf r}}
\newcommand{\bfx}{{\mathbf x}}
\newcommand{\bfxex}{{\mathbf x}_{\mbox{\scriptsize $\star$}}}
\newcommand{\Dim}{\mathop{\mathrm{dim\ }}}

\begin{document}
	%%%%%%%%%%%%%%%%%%%%%%%%%%%%%%%%%%%%%%%%%%%

\begin{center}
{\large
{\bf Extensions of certain graph-based algorithms for preconditioning}}

	David Fritzsche \\
	Dept.~of Mathematics, Temple University \\
	Philadelphia PA 19122-6094 \\
	{\tt david.fritzsche@math.temple.edu} \\
	Andreas Frommer, Daniel B.~Szyld
\end{center}
The original TPABLO algorithms are a collection of algorithms
which compute a symmetric permutation of a linear system such
that the permuted system has a relatively full block
diagonal with relatively large nonzero entries. This block
diagonal can then be used as a preconditioner.
We propose and analyze three extensions of this approach: we
incorporate a nonsymmetric
permutation to obtain a large diagonal, we use a more general
parametrization for TPABLO, and we
use a block Gauss-Seidel preconditioner which can be implemented
to have the same execution time
as the corresponding block Jacobi preconditioner. Since our
approach allows for efficient use of level 3
BLAS operations, it outperforms direct solvers and rivals
standard ILU preconditioners on many test
problems on a single processor system, while having good
potential for efficient parallelization.


	%%%%%%%%%%%%%%%%%%%%%%%%%%%%%%%%%%%%%%%%%%%


\end{document}

