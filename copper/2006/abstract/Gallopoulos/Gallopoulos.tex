\documentclass{report}
\usepackage{amsmath,amssymb}

\def\mathbi#1{\textbf{\em #1}}
\def\BA{\bf{A}}
\def\BB{\bf{B}}
\def\bn{\bf{n}}
\def\CP{\mathcal{P}}
\def\CV{\mathcal{V}}
\def\CI{\mathcal{I}}

\newcommand{\gradt}{\nabla\cdot}
\newcommand{\Reals}{\mathbb{R}}
\newcommand{\Cplex}{\mathbb{C}}

\newcommand{\Vecc}[2]{ \left(
	\begin{array}{c}
	#1 \\ #2
	\end{array}
	\right) }

\newcommand{\Matrr}[4]{ \left(
	\begin{array}{cc}
	#1 & #2 \\
	#3 & #4
	\end{array}
	\right) }

\newcommand{\calB}{{\mathcal B}}
\newcommand{\calBB}{{\mathcal B}^{\Box}}
\newcommand{\calK}{\mathcal{K}}
\newcommand{\bfA}{{\mathbf A}}
\newcommand{\bfb}{{\mathbf b}}
\newcommand{\bfr}{{\mathbf r}}
\newcommand{\bfx}{{\mathbf x}}
\newcommand{\bfxex}{{\mathbf x}_{\mbox{\scriptsize $\star$}}}
\newcommand{\Dim}{\mathop{\mathrm{dim\ }}}

\begin{document}
	%%%%%%%%%%%%%%%%%%%%%%%%%%%%%%%%%%%%%%%%%%%

\begin{center}
{\large
{\bf Can information retrieval aid iterative methods?}}

	Efstratios Gallopoulos \\
	Computer Engineering and Informatics \\
	University of Patras, 26500 Patras, Greece \\
	{\tt stratis@ceid.upatras.gr} \\
	Spyros Hatzimihail
\end{center}
Recently, Computational Linear Algebra techniques have
started playing an important role in Information Retrieval
(IR) research. Indeed, the Vector Space Model and its
derivatives, such as Latent Semantic Indexing, are heavily
used and investigated. In this presentation we consider the
problem in ``reverse mode'', namely using IR techniques to
help in linear algebra and iterative methods in particular.
The candidate problem is the solution of large linear
systems with multiple right hand sides. The efficiency of
solvers is known to depend on the amount of information
shared amongst the right hand sides. We specifically
investigate the combination of clustering algorithms and
schemes from the existing literature to solve such problems.

This research is supported in part by a University of
Patras KARATHEODORI grant.



	%%%%%%%%%%%%%%%%%%%%%%%%%%%%%%%%%%%%%%%%%%%


\end{document}

