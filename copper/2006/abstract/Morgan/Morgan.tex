\documentclass{report}
\usepackage{amsmath,amssymb}

\def\mathbi#1{\textbf{\em #1}}
\def\BA{\bf{A}}
\def\BB{\bf{B}}
\def\bn{\bf{n}}
\def\CP{\mathcal{P}}
\def\CV{\mathcal{V}}
\def\CI{\mathcal{I}}

\newcommand{\gradt}{\nabla\cdot}
\newcommand{\Reals}{\mathbb{R}}
\newcommand{\Cplex}{\mathbb{C}}

\newcommand{\Vecc}[2]{ \left(
	\begin{array}{c}
	#1 \\ #2
	\end{array}
	\right) }

\newcommand{\Matrr}[4]{ \left(
	\begin{array}{cc}
	#1 & #2 \\
	#3 & #4
	\end{array}
	\right) }

\newcommand{\calB}{{\mathcal B}}
\newcommand{\calBB}{{\mathcal B}^{\Box}}
\newcommand{\calK}{\mathcal{K}}
\newcommand{\bfA}{{\mathbf A}}
\newcommand{\bfb}{{\mathbf b}}
\newcommand{\bfr}{{\mathbf r}}
\newcommand{\bfx}{{\mathbf x}}
\newcommand{\bfxex}{{\mathbf x}_{\mbox{\scriptsize $\star$}}}
\newcommand{\Dim}{\mathop{\mathrm{dim\ }}}

\begin{document}
	%%%%%%%%%%%%%%%%%%%%%%%%%%%%%%%%%%%%%%%%%%%

\begin{center}
{\large
{\bf Restarted Lanczos for nonsymmetric eigenvalue problems and \\
	linear equations}}

	Ron Morgan \\
	Math Department, Baylor University \\
	Waco TX 76798-7328 \\
	{\tt Ronald\_Morgan@baylor.edu} \\
	Dywayne Nicely
\end{center}
First, a restarted nonsymmetric Lanczos method will be
discussed. It can be used to compute both left and right
eigenvectors even when storage is limited. Approximate
eigenvectors are retained at the restart as with implicitly
restarted Arnoldi. It uses a three-term recurrence, but some
reorthogonalization is needed.

The next topic is a
related method called BiCG with deflated restarting (QMR
with deflated restarting may also be discussed). It
simultaneously solves linear equations and computes left and
right eigenvectors. For the case of multiple right-hand
sides, the eigenvector information from solving the first
right-hand side can help efficiently solve subsequent
right-hand sides. A deflated BiCGStab can be used for this.
Deflated BiCGStab has a projection over the eigenvectors
followed by regular BiCGStab.



	%%%%%%%%%%%%%%%%%%%%%%%%%%%%%%%%%%%%%%%%%%%


\end{document}

