\documentclass{report}
\usepackage{amsmath,amssymb}

\def\mathbi#1{\textbf{\em #1}}
\def\BA{\bf{A}}
\def\BB{\bf{B}}
\def\bn{\bf{n}}
\def\CP{\mathcal{P}}
\def\CV{\mathcal{V}}
\def\CI{\mathcal{I}}

\newcommand{\gradt}{\nabla\cdot}
\newcommand{\Reals}{\mathbb{R}}
\newcommand{\Cplex}{\mathbb{C}}

\newcommand{\Vecc}[2]{ \left(
	\begin{array}{c}
	#1 \\ #2
	\end{array}
	\right) }

\newcommand{\Matrr}[4]{ \left(
	\begin{array}{cc}
	#1 & #2 \\
	#3 & #4
	\end{array}
	\right) }

\newcommand{\calB}{{\mathcal B}}
\newcommand{\calBB}{{\mathcal B}^{\Box}}
\newcommand{\calK}{\mathcal{K}}
\newcommand{\bfA}{{\mathbf A}}
\newcommand{\bfb}{{\mathbf b}}
\newcommand{\bfr}{{\mathbf r}}
\newcommand{\bfx}{{\mathbf x}}
\newcommand{\bfxex}{{\mathbf x}_{\mbox{\scriptsize $\star$}}}
\newcommand{\Dim}{\mathop{\mathrm{dim\ }}}

\begin{document}
	%%%%%%%%%%%%%%%%%%%%%%%%%%%%%%%%%%%%%%%%%%%

\begin{center}
{\large
{\bf Constraint-style preconditioners for regularized saddle point problems}}

	Sue Dollar \\
	Dept.~of Mathematics, University of Reading \\
	Whiteknights, P.O.Box 220, Reading, Berkshire, RG6 6AX, UK
	\\ {\tt h.s.dollar@reading.ac.uk}
\end{center}
The problem of finding good preconditioners for the numerical
solution of a certain important class of indefinite linear
systems is considered. These systems are of a saddle point
structure
$$ \Matrr{A}{B^T}{B}{-C} \Vecc{x}{y} = \Vecc{c}{d}, $$
where $A\in\Reals^{n\times n}$,
$C\in\Reals^{m\times m}$ are symmetric,
and $B\in\Reals^{m\times n}$ has full rank.

In {\em
Constraint preconditioning for indefinite linear systems},
SIAM J.~Matrix Anal. Appl. {\bf 21} (2000),
Keller, Gould and Wathen analyzed the idea of
using constraint preconditioners that have a
specific 2 by 2 block structure for the case of
$C$ being zero. We
shall extend this idea by allowing the (2,2) block to be
non-zero. Results concerning the spectrum and form of the
eigenvectors are presented, as are numerical results to validate
our conclusions.
We will also introduce the idea of
implicit-factorization constraint preconditioners which allow us
to efficiently carry out the required preconditioning steps.



	%%%%%%%%%%%%%%%%%%%%%%%%%%%%%%%%%%%%%%%%%%%


\end{document}

