\documentclass{report}
\usepackage{amsmath,amssymb}

\def\mathbi#1{\textbf{\em #1}}
\def\BA{\bf{A}}
\def\BB{\bf{B}}
\def\bn{\bf{n}}
\def\CP{\mathcal{P}}
\def\CV{\mathcal{V}}
\def\CI{\mathcal{I}}

\newcommand{\gradt}{\nabla\cdot}
\newcommand{\Reals}{\mathbb{R}}
\newcommand{\Cplex}{\mathbb{C}}

\newcommand{\Vecc}[2]{ \left(
	\begin{array}{c}
	#1 \\ #2
	\end{array}
	\right) }

\newcommand{\Matrr}[4]{ \left(
	\begin{array}{cc}
	#1 & #2 \\
	#3 & #4
	\end{array}
	\right) }

\newcommand{\calB}{{\mathcal B}}
\newcommand{\calBB}{{\mathcal B}^{\Box}}
\newcommand{\calK}{\mathcal{K}}
\newcommand{\bfA}{{\mathbf A}}
\newcommand{\bfb}{{\mathbf b}}
\newcommand{\bfr}{{\mathbf r}}
\newcommand{\bfx}{{\mathbf x}}
\newcommand{\bfxex}{{\mathbf x}_{\mbox{\scriptsize $\star$}}}
\newcommand{\Dim}{\mathop{\mathrm{dim\ }}}

\begin{document}
	%%%%%%%%%%%%%%%%%%%%%%%%%%%%%%%%%%%%%%%%%%%

\begin{center}
{\large
{\bf How a step toward wider class of matrices could help improve \\
	convergence area of relaxation methods}}

	Vladimir Kostic \\
	Kralja Petra I 69/44 \\
	21000 Novi Sad, Serbia and Montenegro \\
	{\tt vkostic@im.ns.ac.yu} \\
	Ljiljana Cvetkovic
\end{center}
Investigations are related to several different relaxation
methods for solving systems of linear equations, but the
main idea is always the same: knowing that system matrix is
strictly diagonally dominant (SDD), we can consider it as an
S-SDD (see Lj. Cvetkovic, V.~Kostic and R.~S.~Varga,
{\em A new Gerschgorin-type eigenvalue inclusion area},
ETNA 18, 2004)
matrix for every nonempty proper subset S of the set
of indices, and from this fact we can derive, in some sense,
an optimal convergence area for relaxation parameter(s).
This convergence area is usually significantly wider than
the corresponding one, obtained from the knowledge of SDD
property, only. Instead of S-SDD class, some more subclasses
of H-matrices can be used for the same purposes.



	%%%%%%%%%%%%%%%%%%%%%%%%%%%%%%%%%%%%%%%%%%%


\end{document}

