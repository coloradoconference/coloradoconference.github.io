\documentclass{report}
\usepackage{amsmath,amssymb}

\def\mathbi#1{\textbf{\em #1}}
\def\BA{\bf{A}}
\def\BB{\bf{B}}
\def\bn{\bf{n}}
\def\CP{\mathcal{P}}
\def\CV{\mathcal{V}}
\def\CI{\mathcal{I}}

\newcommand{\gradt}{\nabla\cdot}
\newcommand{\Reals}{\mathbb{R}}
\newcommand{\Cplex}{\mathbb{C}}

\newcommand{\Vecc}[2]{ \left(
	\begin{array}{c}
	#1 \\ #2
	\end{array}
	\right) }

\newcommand{\Matrr}[4]{ \left(
	\begin{array}{cc}
	#1 & #2 \\
	#3 & #4
	\end{array}
	\right) }

\newcommand{\calB}{{\mathcal B}}
\newcommand{\calBB}{{\mathcal B}^{\Box}}
\newcommand{\calK}{\mathcal{K}}
\newcommand{\bfA}{{\mathbf A}}
\newcommand{\bfb}{{\mathbf b}}
\newcommand{\bfr}{{\mathbf r}}
\newcommand{\bfx}{{\mathbf x}}
\newcommand{\bfxex}{{\mathbf x}_{\mbox{\scriptsize $\star$}}}
\newcommand{\Dim}{\mathop{\mathrm{dim\ }}}

\begin{document}
	%%%%%%%%%%%%%%%%%%%%%%%%%%%%%%%%%%%%%%%%%%%

\begin{center}
{\large
{\bf The block grade of a block Krylov space}}

	Martin H.~Gutknecht \\
	Seminar for Applied Mathematics, ETH Zurich \\
	8092 Zurich, Switzerland \\
	{\tt mhg@math.ethz.ch} \\
	Thomas Schmelzer
\end{center}
The so-called grade of a
vector $b$ with respect to a nonsingular matrix $\bfA$ is
the dimension of the (largest) Krylov (sub)space generated
by $\bfA$ from $b$. It determines in particular, how many
iterations a Krylov space method with linearly independent
residuals requires for finding in exact arithmetic the
solution of $\bfA x=b$ (if the initial approximation $x_0$
is the zero vector). In this talk we generalize the grade
notion to block Krylov spaces and show that this and other
fundamental properties carry over to block Krylov space
methods for solving linear systems with multiple right-hand
sides.

We consider $s$ linear systems with the same
nonsingular coefficient matrix $\bfA$, but different
right-hand sides $b^{(i)}$, which we gather in a
\textit{block vector}
$\bfb := (b^{(1)},\dots,b^{(s)})$.
The $s$ systems are then written as
$$
\bfA \bfx = \bfb \qquad
\text{with} \quad \bfA \in \Cplex^{N\times N}\,,
\quad \bfb \in
\Cplex^{N\times s}\,,\quad
\bfx \in \Cplex^{N\times s}\,.
$$
Standard block Krylov space methods
construct in the $n$th iteration approximate solutions
gathered in a block vector $\bfx_n$ chosen such that
$$
\bfx_n - \bfx_0 \in \calBB_n
(\bfA, \bfr_0) \,,
$$
where $\bfx_0$
contains the $s$ initial approximations and $\bfr_0$ the
corresponding initial residuals, while
$\calBB_n$ is the
Cartesian product
$$
\calBB_n = \underbrace{\calB_n
\times \cdots \times
\calB_n}_{s\text{ times}}
$$
with
$$
\calB_n = \calK_n (\bfA,
r_0^{(1)}) + \cdots + \calK_n
(\bfA, r_0^{(s)}) \,.
$$
Here, $\calK_n (\bfA, r_0^{(i)})$
is the
usual $n$th Krylov (sub)space of the $i$th system. It is
important, that, in general, the sum in the last formula is
not a direct sum, that is, the Krylov spaces may have
nontrivial intersections.

The \textit{block grade of\/
$\bfr_0$ with respect to\/ $\bfA$} or, the
\textit{block grade of\/ $\bfA$
with respect to\/ $\bfr_0$}
is the positive integer
$\bar\nu := \bar\nu(\bfr_0,\bfA)$
defined by
$$
\bar\nu(\bfr_0,\bfA) = \min \left\{n \big\vert
\Dim
\calB_n(\bfA, \bfr_0) =
\Dim
\calB_{n+1}(\bfA, \bfr_0)
\right\}\,.
$$

Among the results we have
established for the block grade are the following ones.


{\sc Lemma 1} ~
For $n \geq \bar\nu(\bfr_0,\bfA)$,
$$
\calB_n(\bfA, \bfr_0) =
\calB_{n+1}(\bfA, \bfr_0)\,,\qquad
\calBB_n(\bfA, \bfr_0) =
\calBB_{n+1}(\bfA, \bfr_0)\,.
$$


{\sc Lemma 2} ~
The block grade of the
block Krylov space and the grades of the individual Krylov
spaces contained in it are related by
$$
\calB_{\bar\nu(\bfr_0,\bfA)}(\bfA, \bfr_0) =
\calK_{\bar\nu(r_0^{(1)},\bfA)}(\bfA, r_0^{(1)})
+ \dots +
\calK_{\bar\nu(r_0^{(s)},\bfA)}(\bfA, r_0^{(s)})\,.
$$

{\sc Lemma 3} ~
The block grade $\bar\nu(\bfr_0,\bfA)$ is characterized by
$$
\bar\nu(\bfr_0,\bfA)
= \min\left\{ n \big\vert \bfA^{-1} \bfr_0 \in
\calBB_n(\bfA, \bfr_0) \right\}.
$$


{\sc Theorem} ~
Let $\bfxex$ be the block
solution of $\bfA\bfx = \bfb$ and let $\bfx_0$ be any
initial block approximation of it and
$\bfr_0 := \bfb - \bfA \bfx_0$
the corresponding block residual. Then
$$
\bfxex \in \bfx_0 +
\calBB_{\bar\nu(\bfr_0,\bfA)}(\bfA,\bfr_0)
\,.
$$

We also discuss the effects of
the size of the block grade on the efficiency of a block
Krylov space method.



	%%%%%%%%%%%%%%%%%%%%%%%%%%%%%%%%%%%%%%%%%%%


\end{document}

