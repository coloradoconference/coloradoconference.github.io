\documentclass{report}
\usepackage{amsmath,amssymb}

\def\mathbi#1{\textbf{\em #1}}
\def\BA{\bf{A}}
\def\BB{\bf{B}}
\def\bn{\bf{n}}
\def\CP{\mathcal{P}}
\def\CV{\mathcal{V}}
\def\CI{\mathcal{I}}

\newcommand{\gradt}{\nabla\cdot}
\newcommand{\Reals}{\mathbb{R}}
\newcommand{\Cplex}{\mathbb{C}}

\newcommand{\Vecc}[2]{ \left(
	\begin{array}{c}
	#1 \\ #2
	\end{array}
	\right) }

\newcommand{\Matrr}[4]{ \left(
	\begin{array}{cc}
	#1 & #2 \\
	#3 & #4
	\end{array}
	\right) }

\newcommand{\calB}{{\mathcal B}}
\newcommand{\calBB}{{\mathcal B}^{\Box}}
\newcommand{\calK}{\mathcal{K}}
\newcommand{\bfA}{{\mathbf A}}
\newcommand{\bfb}{{\mathbf b}}
\newcommand{\bfr}{{\mathbf r}}
\newcommand{\bfx}{{\mathbf x}}
\newcommand{\bfxex}{{\mathbf x}_{\mbox{\scriptsize $\star$}}}
\newcommand{\Dim}{\mathop{\mathrm{dim\ }}}

\begin{document}
	%%%%%%%%%%%%%%%%%%%%%%%%%%%%%%%%%%%%%%%%%%%

\begin{center}
{\large
{\bf HP local refinement using FOSLS}}

	Josh Nolting \\
	2467 E 127th Court,  Thornton CO 80241 \\
	{\tt josh.nolting@colorado.edu} \\
	Thomas Manteuffel
\end{center}
Local refinement enables us to concentrate computational
resources in areas that need special attention, for example,
near steep gradients and singularities. In order to use
local refinement efficiently, it is important to be able to
quickly estimate local error. FOSLS is an ideal method to
use for this because the FOSLS functional yields a sharp a
posteriori error measure for each element. This talk will
discuss a strategy for determining which elements to refine
in order to optimize the accuracy/computational cost. Set in
the context of a full multigrid algorithm, our strategy
leads to a refinement pattern with nearly equal error on
each element. Further refinement is essentially uniform,
which allows for an efficient parallel implementation.
Numerical experiments will be presented.



	%%%%%%%%%%%%%%%%%%%%%%%%%%%%%%%%%%%%%%%%%%%


\end{document}

