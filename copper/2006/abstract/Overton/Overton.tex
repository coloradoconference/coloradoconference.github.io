\documentclass{report}
\usepackage{amsmath,amssymb}

\def\mathbi#1{\textbf{\em #1}}
\def\BA{\bf{A}}
\def\BB{\bf{B}}
\def\bn{\bf{n}}
\def\CP{\mathcal{P}}
\def\CV{\mathcal{V}}
\def\CI{\mathcal{I}}

\newcommand{\gradt}{\nabla\cdot}
\newcommand{\Reals}{\mathbb{R}}
\newcommand{\Cplex}{\mathbb{C}}

\newcommand{\Vecc}[2]{ \left(
	\begin{array}{c}
	#1 \\ #2
	\end{array}
	\right) }

\newcommand{\Matrr}[4]{ \left(
	\begin{array}{cc}
	#1 & #2 \\
	#3 & #4
	\end{array}
	\right) }

\newcommand{\calB}{{\mathcal B}}
\newcommand{\calBB}{{\mathcal B}^{\Box}}
\newcommand{\calK}{\mathcal{K}}
\newcommand{\bfA}{{\mathbf A}}
\newcommand{\bfb}{{\mathbf b}}
\newcommand{\bfr}{{\mathbf r}}
\newcommand{\bfx}{{\mathbf x}}
\newcommand{\bfxex}{{\mathbf x}_{\mbox{\scriptsize $\star$}}}
\newcommand{\Dim}{\mathop{\mathrm{dim\ }}}

\begin{document}
	%%%%%%%%%%%%%%%%%%%%%%%%%%%%%%%%%%%%%%%%%%%

\begin{center}
{\large
{\bf Nonsmooth, nonconvex optimization: theory, algorithms, and applications}}

	Michael Overton \\
	Courant Institute \\
	251 Mercer St, New York NY 10012 \\
	{\tt overton@cs.nyu.edu}
\end{center}
Theory: there are two standard approaches to generalizing
derivatives to nonsmooth, nonconvex optimization: the Clarke
subdifferential (or generalized gradient), and the MIRW
subdifferential (or subgradient sets), as expounded in
Rockafellar and Wets (Springer, 1998). We briefly discuss
these and mention their advantages and disadvantages. They
coincide for an important class of functions: those that are
locally Lipschitz and regular, which includes continuously
differentiable functions and convex functions.


Algorithms: the usual approach is bundle methods, which are
complicated. We describe some alternatives: BFGS (a new look
at an old method), and Gradient Sampling (a simply stated
method that, although computationally intensive, has solved
some previously unsolved problems and has a nice convergence
theory).

 Applications: these abound in control, but
surely in other areas too. Of particular interest to me are
applications involving eigenvalues and singular values of
nonsymmetric matrices. Sometimes even easily stated problems
in a few variables are hard. Our new code HIFOO (H-Infinity
Fixed-Order Optimization) is intended for use by practicing
control engineers and has solved some open problems in
control.

 This is all joint work with James Burke and
Adrian Lewis. HIFOO is also joint with Didier Henrion.



	%%%%%%%%%%%%%%%%%%%%%%%%%%%%%%%%%%%%%%%%%%%


\end{document}

