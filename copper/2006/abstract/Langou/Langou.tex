\documentclass{report}
\usepackage{amsmath,amssymb}

\def\mathbi#1{\textbf{\em #1}}
\def\BA{\bf{A}}
\def\BB{\bf{B}}
\def\bn{\bf{n}}
\def\CP{\mathcal{P}}
\def\CV{\mathcal{V}}
\def\CI{\mathcal{I}}

\newcommand{\gradt}{\nabla\cdot}
\newcommand{\Reals}{\mathbb{R}}
\newcommand{\Cplex}{\mathbb{C}}

\newcommand{\Vecc}[2]{ \left(
	\begin{array}{c}
	#1 \\ #2
	\end{array}
	\right) }

\newcommand{\Matrr}[4]{ \left(
	\begin{array}{cc}
	#1 & #2 \\
	#3 & #4
	\end{array}
	\right) }

\newcommand{\calB}{{\mathcal B}}
\newcommand{\calBB}{{\mathcal B}^{\Box}}
\newcommand{\calK}{\mathcal{K}}
\newcommand{\bfA}{{\mathbf A}}
\newcommand{\bfb}{{\mathbf b}}
\newcommand{\bfr}{{\mathbf r}}
\newcommand{\bfx}{{\mathbf x}}
\newcommand{\bfxex}{{\mathbf x}_{\mbox{\scriptsize $\star$}}}
\newcommand{\Dim}{\mathop{\mathrm{dim\ }}}

\begin{document}
	%%%%%%%%%%%%%%%%%%%%%%%%%%%%%%%%%%%%%%%%%%%

\begin{center}
{\large
{\bf A fast a-posteriori reorthogonalization scheme for the classical \\
	Gram-Schmidt orthogonalization in the context of iterative methods}}

	Julien Langou \\
	1122 Volunteer Blvd, Claxton Bldg. Room 233 \\
	Knoxville TN 37996-3450 \\
	{\tt langou@cs.utk.edu}
\end{center}
The year 2005 have been marked by two new papers on the
Classical Gram-Schmidt algorithm (see [1,2]). These results
offer a better understanding of the Classical Gram-Schmidt
algorithm. It is finally proved that the Classical
Gram-Schmidt algorithm generates a loss of orthogonality
bounded by the square of the condition number of the initial
matrix. In the first part of the talk, I will quickly review
the proof, explain its key points and its implication in the
context of iterative methods. In the second part, I will
focus on the new results that we have found related to the
Classical Gram-Schmidt algorithm. In particular an
a-posteriori reorthogonalization scheme extremely efficient
is given in the context of iterative methods. (We borrow
ideas developed in [3] in the context of GMRES-MGS.)

[1] A.~Smoktunowicz and J.~Barlow,
{\em A note on the error analysis of Classical Gram Schmidt},
submitted to Numerische Mathematik (2005).

[2] Luc Giraud, Julien Langou, Miroslav Rozlo\v{z}n\'{\i}k,
Jasper van den Eshof, {\em Rounding error analysis of
the classical Gram-Schmidt orthogonalization process},
Numerische Mathematik {\bf 101}(1) (July 2005) 87--100.

[3] Luc Giraud, Serge Gratton, Julien Langou,
{\em A rank-$k$ update procedure for
reorthogonalizing the orthogonal factor from modified
Gram-Schmidt}, SIAM J.~Matrix An. Appl. {\bf 25}(4)
(August 2004) 1163--1177.



	%%%%%%%%%%%%%%%%%%%%%%%%%%%%%%%%%%%%%%%%%%%


\end{document}

