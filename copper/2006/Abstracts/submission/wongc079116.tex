\documentclass{report}
\usepackage{amsmath,amssymb}
\setlength{\parindent}{0mm}
\setlength{\parskip}{1em}
\begin{document}
\begin{center}
\rule{6in}{1pt} \
{\large Chao-Jen Wong \\
{\bf An Embedding Method for Simulation of Immobilized Enzyme Kinetics and Transport in Sessile Hydrogel Drops }}

School of Mathematical Sciences \\ Claremont Graduate University: 2005 \\ 710 N College Ave Claremont \\ CA 91711
\\
{\tt wongc@cgu.edu}\\
Ali Nadim\end{center}

\baselineskip 1.0pc
\begin{center}
Abstract
\end{center}
We present a new numerical method, termed the embedding method, to solve
a system of nonlinear PDEs for multi-phase problems in asymmetric 3-D
domains. The main feature of this method is its ability to perform
interface calculation and account for conditions relating solution
properties across phase interface using a finite difference /
volume-fraction-based front-capturing hybrid technique. The approach
begins by considering the computational domain as physically separated
phases. A finite difference method with a Cartesian grid is employed on
the whole domain while modifications are applied to correct boundary
conditions at the interfaces. The volume-fraction-based front-capturing
algorithm is used to capture each interface in terms of the volume
fraction in each cell. The major aspect of this method is its
implementation simplicity, which results in code generation that can be
highly optimized. To highlight this method, an application is presented
for simulation and investigation of enzyme reactions within a sessile
hydrogel drop, where the Michaelis-Menten kinetics is used to model the
reaction mechanism.


\end{document}
