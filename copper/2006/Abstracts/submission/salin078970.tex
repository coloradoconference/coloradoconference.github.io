\documentclass{report}
\usepackage{amsmath,amssymb}
\setlength{\parindent}{0mm}
\setlength{\parskip}{1em}
\begin{document}
\begin{center}
\rule{6in}{1pt} \
{\large Andrew Salinger \\
{\bf Space-Time Solution of Large-Scale PDE Applications}}

Sandia National Labs \\ PO Box 5800 \\ MS-1111 \\ Albuquerque \\ NM 87185
\\
{\tt agsalin@sandia.gov}\\
Daniel Dunalvy\\
Eric Phipps\end{center}

Software and algorithms are being developed to efficiently
formulate and solve transient PDE problems as "steady"
problems in a space-time domain. In this way, sophisticated
design and analysis tools for steady problems, such as
continuation methods, can be brought to bear on transient
(and eventually periodic) problems. This new capability is
being developed in the Trilinos solver framework, and is
designed to present a simple interface to application codes.
The software allows for parallelism over both the space and
time domains.

The main hurdle to make this approach viable is to be able
to efficiently solve the very large Jacobian matrix for very
large the space-time system, with its characteristic structure.
We will present results for a number of preconditioners and
solution methods that we have developed for this linear system.
Numerical results for a PDE reacting flow application will be
presented. These results will shed some light on the underlying
question of whether it can pay to parallelize the time domain.


\end{document}
