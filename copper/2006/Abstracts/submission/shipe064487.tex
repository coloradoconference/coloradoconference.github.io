\documentclass{report}
\usepackage{amsmath,amssymb}
\setlength{\parindent}{0mm}
\setlength{\parskip}{1em}
\begin{document}
\begin{center}
\rule{6in}{1pt} \
{\large Peter L. Shi \\
{\bf A Large Scale Nonlinear Finite Element Solver Algorithm with Optimal Speed and Robustness}}

Dept of Math and Statistics \\ Oakland University \\ Rochester MI 48323
\\
{\tt pshi@oakland.edu}\end{center}

The speaker will present a new theory, the related algorithmic and
programming architecture for solving nonlinear boundary value problems
with {\it optimal speed and robustness} by using large numbers of finite
elements.


The new methodology will achieve the desired optimality
over a singularly large spectrum of nonlinear finite element models on
bounded domains of $R^n$ with $n=2$ and 3.
For example, the algorithm will cover the Galerkin formulation
of well-posed nonlinear elliptic systems whose principle part is
Lipschitz continuous and strongly monotone in a Sobolev space and certain
problems that lack unique solutions such as stationary Navier-Stokes
equations. Large variations of stiffness will also be permitted in both
magnitude and frequency.


The merit of the new algorithm is not only its speed and scope,
but also its mathematically rigorous theory,
the elegance in algorithmic design, and simplicity
in implementation. The approach will be
based on the proven success of the speaker's long time effort starting
from early 1990s, recently reported in the article\footnote{This article
is submitted to Advances in Computational Mathematics in August, 2005 (95
pages). } titled ``Foundation of
nonlinear finite element solvers, Part I", which successfully establishes
the corresponding result for
second order {\it quasi-linear} elliptic systems
with non-negative lower order terms.
Central to the algorithm, the speaker will reformulate a finite element
model by generalized Wiener-Hopf equations. This will
make element-wise conditioning an inexpensive process,
whereby reducing the solution procedure to the straightforward
Banach contraction mapping principle: given $f$, $y_0$ and $m$, compute
$$
y_{k+1} = (I-R^* T R) y_k+ R^*f,
\quad k=0,1,2\dots m.
$$
Here the operator $I- R^* T R$ is strictly contractive with
the contraction constant
independent of the number of unknowns in the system; $T$
is a scaled direct sum of the local stiffness operators;
$R$ and $R^*$ are linear operators conjugate to each other,
and I is the identity mapping. Computing $Ry$ and $R^* y^*$
for arbitrary $y$ and $y^*$ is equivalent to solving a linear system
defined by a fixed class of sparse $M$-matrices and their close variants,
which
can be accomplished by algebraic multi-grid method (AMG) in linear
computational count. For a large class of practical problems, they can
also be
accomplished by a variety of other linear solver techniques,
showing the robustness of
the algorithm. From the numerical point of view, $R$ and $R^*$
are optimal conditioners of $T$ in terms of cost, efficiency and
robustness. They depend only on a discrete function space modulo the
kernel of an appropriate linear analog of $T$.
Throughout the algorithmic design and analysis,
non-traditional tools such as
topological spaces and discrete measures will be systematically deployed
for representing and handling the
data structure. This is another novelty of the speaker's approach from
the standard methodology.


The speaker's new approach is related neither to Newton-Krylov method and
its variants, nor to FAS.
At the philosophical level, it is a natural extension of AMG
to its fully nonlinear analog without using FAS. It can also be viewed as
an extreme exercise
of the {\it finite element tearing and inter-connection} (FETI)
philosophy coupled with
a novel treatment of degrees of freedom.


\end{document}
