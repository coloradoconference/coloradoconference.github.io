Good morning Bruce,
Here is another one that could not submit.
Thanks
Cathy

---------- Forwarded message ----------
Date: Wed, 1 Feb 2006 23:23:54 GMT
From: Iain Duff <I.Duff@rl.ac.uk>
To: Copper.Conference@colorado.edu
Cc: elman@cs.umd.edu, isd@brora.cis.rl.ac.uk, tmanteuf@boulder.colorado.edu
Subject: Do we have a problem?

Cathy/Howard/Tom

We could have a potential problem with the abstract submission.  Like I
am sure many people, I tried to submit my abstract at the eleventh hour
and the software bombed me out for a claimed too long abstract.  It also
did it rather abruptly clearing all fields.  I append my whole LaTeX
file below which a word count will show to be less than 1000 words
and which prints comfortably in less than two pages.  Also
I only submitted the main text (in the section called introduction).

I would like this to be taken as my submission but, more importantly,
it would be good to check if there is a problem with the site and, if so,
to correct it and extend the deadline.

Of course, I have just returned from the village pub so the problem
could be with me ... if so please let me know what I did wrong.

Best wishes

Iain

++++++++++++++++++++++++++
%\NeedsTeXFormat{LaTeX2e}[1994/06/01]

\documentclass{siamltex}

\usepackage{latexsym}
\usepackage{eufrak }
\usepackage{epsfig}
\usepackage{graphicx}
\usepackage{rotating}

\usepackage{comment}

\graphicspath{{figures/}{:figures:}}

%MACRO Mario%%%%%%%%%%%%%%%%%%%%%%%%%%%%%%%%%%%%%%%%%

\newtheorem{remark}{Remark}{\itshape}{\rmfamily}
\renewcommand{\arraystretch}{1.5}

\newcommand{\RR}{\mbox{I\hspace{-.23em}R}}

\newcommand{\Mh}{\widehat{M}}
\newcommand{\Hh}{\hat{H}}
\newcommand{\Vh}{\hat{V}}
\newcommand{\Vb}{\bar{V}}
\newcommand{\Lb}{\bar{L}}
\newcommand{\Lh}{\hat{L}}
\newcommand{\Dh}{\hat{D}}
\newcommand{\Db}{\bar{D}}
\newcommand{\Rb}{\overline{R}}

\newcommand{\Hb}{\bar{H}}
\newcommand{\Zb}{\bar{Z}}
\newcommand{\Zh}{\hat{Z}}
\newcommand{\Wh}{\hat{W}}
\newcommand{\Zt}{\widetilde{Z}}
\newcommand{\It}{\widetilde{I}}
\newcommand{\ybk}{\bar{y}_k}
\newcommand{\xbk}{\bar{x}_k}
\newcommand{\zb}{\bar{z}}
\newcommand{\vb}{\bar{v}}
\newcommand{\xb}{\bar{x}}

\newtheorem{Hy}{Hypothesis}[section]
\newtheorem{Th}{Theorem}[section]
\newtheorem{Alg}{Algorithm}[section]
\newtheorem{Co}{Corollary}[section]
\newtheorem{Lm}{Lemma}[section]
\newcommand{\rbox}{\nolinebreak~\hfill{$\Box$}}


\newcommand{\um}{{\bf \varepsilon }\,}
\newcommand{\oepsq}{{\cal{O}}(\um2)}
\newcommand{\oeps}{{\cal{O}}(\um)}

\begin{document}

%%%%%%%%%%%%%%%%%%%%%%%%%%%%%%%%%%%%%%%%%%%%%%%%%%%%%%%%%%

\title{A note on GMRES preconditioned by a perturbed $L D L^T$ decomposition with static pivot}
\author{M. Arioli$^{1}$ and I.~S. Duff$^{1,2}$ and S. Gratton$^{2}$ and S. Pralet$^{2}$}


\maketitle
%%%%%%%%%%%%%%%%%%%%

\footnotetext[1]{  Rutherford Appleton Laboratory, Chilton,
Didcot, Oxfordshire, OX11 0QX, UK. email: m.arioli@rl.ac.uk,Phone: +441235 445332}

\footnotetext[2]{  CERFACS, rue G. Coriolis,
31057 Toulouse, France}
%\centerline{\today}

%%%%%%%%%%%%%%%%%% ABSTRACT %%%%%%%%%%%%%%%%%%%%%%%%%%%%%%%%
\begin{abstract}

\end{abstract}
%%%
\begin{keywords}
  augmented systems, saddle-point problems, sparse matrices, GMRES, FGMRES,
  static pivoting, roundoff error.
\end{keywords}
%%%
\begin{AMS}
65F05,
\end{AMS}

\pagestyle{myheadings}
\thispagestyle{plain}
\markboth{M. ARIOLI, I.~S. DUFF, S. GRATTON, AND S. PRALET}
{GMRES AND $L^TDL$ WITH STATIC PIVOTING}

\section{Introduction}
This paper is concerned with solving the set of linear equations
\begin{eqnarray}\label{Ax=b}
A x = b.
\end{eqnarray}
where the coefficient matrix $A$ is a symmetric indefinite sparse matrix.
Our hope is to solve this system using a direct method that uses an
accurate factorization of $A$  but sometimes the
cost of doing this is too high in terms of time or memory.  We have therefore
looked
at the possibility of using static pivoting to avoid these problems which
are particularly acute if the matrix is highly indefinite as for example can
happen for saddle-point problems.

As our direct method we will use a multifrontal approach.  In this approach
we first determine an order for choosing pivots based on the sparsity
structure of $A$ (called the analysis step), and we then accommodate further
pivoting for numerical
stability during the subsequent numerical factorization phase.  The problem
when the matrix is highly indefinite is that the resulting pivot sequence
used in the numerical factorization can differ substantially from that
predicted by the analysis step.  In the multifrontal context, the
factorization can be represented by a tree at each node of which elimination
operations are performed on a partially summed frontal matrix
\begin{eqnarray}\label{multifrontal}
\left ( {\begin{array} {ll} F_{11} & F_{12} \\ F_{12}^T & F_{22} \end{array}}
\right ),
\end{eqnarray}
and pivots at that stage can only be chosen from within the fully summed block
$F_{11}$. The problem occurs when it is impossible or numerically suicidal
to eliminate all of $F_{11}$ resulting in more work and storage (sometimes
dramatically more) than forecast.  A simple way to avoid this problem
is to force the elimination of all of $F_{11}$ through static pivoting.

We thus assume that the matrix $A$ has been factorized using the HSL package
{\tt MA57}
with the option of using static pivoting \cite{dupr:05}. The static pivoting
strategy will set the diagonal entry to +- $\tau$
when it is impossible to find a suitable pivot in the fully summed blocks.
It is common to choose $\tau \approx \sqrt{\um} ||A||$ ($\um$ machine
precision).

Therefore, the computed factors $\hat{L}$ and $\hat{D}$ are, in exact
arithmetic, the exact factorization
of the perturbed problem
\begin{eqnarray}\label{A_per}
A + E = \hat{L} \hat{D} \hat{L}^T,
\end{eqnarray}
where the matrix $|E| \le \tau  I$ is a diagonal matrix of rank equal to the
number of static pivots used during the factorization. The nonzero diagonal
entries in $E$
correspond to the positions at which  static pivoting was performed and they
are all equal to $\tau$ in modulus. Note that if $\tau$ is chosen too small
then the
factorization could be very unstable whereas if it is chosen too large, the
factorization will be stable but will not be an accurate factorization of the
original matrix (that is, $|E|$ will be large).

Equation (\ref{A_per}) gives a splitting of $A$ in terms of $M = \hat{L} \hat{D} \hat{L}^T$ and $E$
\begin{eqnarray}\label{A_split}
A = M - E,
\end{eqnarray}
and the solution of (\ref{Ax=b}) can be expressed as the solution of the equivalent system
\begin{eqnarray}\label{I-H}
(I - M^{-1} E) x = M^{-1}b.
\end{eqnarray}
If the spectral radius of the matrix $I - M^{-1} E $ is less than one, the system (\ref{I-H}) can be solved using
iterative refinement. This has been used by many authors, including \cite{dupr:05} and is successful over a wide range of matrices although is somewhat
sensitive to the value of $\tau$.  If, however, the
spectral radius is
greater or equal to one (or $\approx 1$), it is necessary to switch to a
more powerful
method like  GMRES. Although the matrix is symmetric, we choose
GMRES since it gives us much more freedom to work with a wide range of
preprocessors and preconditionings.

We have found experimentally that using the factorization (\ref{A_per}) as
a preconditioning for GMRES works in most cases and is, as expected much
more robust than iterative refinement.  Indeed GMRES gives
normwise backward stability in most cases, which is
not the case
for iterative refinement. However, there are cases where we
do not get convergence to a scaled residual at machine precision.

We have, however, found that restarted GMRES performs better and that using
FGMRES,
even though our preconditioner remains constant, does even
better.

We illustrate this through numerical experiment and then show theoretically
that, under reasonable assumptions, FGMRES preconditioned by our static
pivoting
factorization is backward stable so that a small scaled residual can
be achieved.  Our analysis also holds for the case of restarted FGMRES that we advocate as a measure to control the memory requirement while still
achieving the desired accuracy.  Indeed we give theoretical arguments why
the restarting often greatly improves the convergence.


\begin{thebibliography}{10}

\bibitem{dupr:05}
{\sc I.~S. Duff and S.~Pralet}, {\em Towards a stable static pivoting strategy
  for the sequential and parallel solution of sparse symmetric indefinite
  systems}, {T}echnical {R}eport TR/PA/05/26, CERFACS, Toulouse, France, 2005.
\newblock Also available as RAL Report RAL-TR-2005-007 and IRIT Report
  RT/TLSE/05/04.

\end{thebibliography}

\end{document}

