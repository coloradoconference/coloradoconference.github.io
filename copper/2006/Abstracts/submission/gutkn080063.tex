\documentclass{report}
\usepackage{amsmath,amssymb}
\setlength{\parindent}{0mm}
\setlength{\parskip}{1em}
\begin{document}
\begin{center}
\rule{6in}{1pt} \
{\large Martin H. Gutknecht \\
{\bf The Block Grade of a Block Krylov Space}}

Seminar for Applied Mathematics \\ ETH Zurich \\ 8092 Zurich \\ Switzerland
\\
{\tt mhg@math.ethz.ch}\\
Thomas Schmelzer\end{center}

\newtheorem{Thm}{\sc Theorem}
\newtheorem{Cor}[Thm]{\sc Corollary}
\newtheorem{Lem}[Thm]{\sc Lemma}

\newcommand{\CC}{{\mathbb C}}
\newcommand{\RR}{{\mathbb R}}

\newcommand{\q}{\quad}
\newcommand{\qq}{\quad\quad}

\newcommand{\Dim}{\mathop{\mathrm{dim\ }}}
\newcommand{\Span}{\mathsf{span}\,}

\newcommand{\calB}{{\mathcal B}}
\newcommand{\calBB}{{\mathcal B}^{\Box}}
\newcommand{\calK}{\mathcal{K}}

\newcommand{\bfA}{{\mathbf A}}
\newcommand{\bfb}{{\mathbf b}}
\newcommand{\bfr}{{\mathbf r}}
\newcommand{\bfx}{{\mathbf x}}
\newcommand{\bfxex}{{\mathbf x}_{\mbox{\scriptsize $\star$}}}

The so-called grade of a vector $b$ with respect to a nonsingular
matrix $\bfA$ is the dimension of the (largest) Krylov (sub)space generated
by $\bfA$ from $b$. It determines in particular, how many iterations a
Krylov space method with linearly independent residuals requires
for finding in exact arithmetic the solution of $\bfA x=b$ (if the
initial approximation $x_0$ is the zero vector).
In this talk we generalize the grade notion to block Krylov spaces and
show that this and other fundamental properties carry over to block
Krylov space methods for solving linear systems with multiple right-hand
sides.

We consider $s$ linear systems with the same nonsingular coefficient
matrix $\bfA$, but different right-hand sides $b^{(i)}$, which we
gather in a \textit{block vector} $\bfb := (b^{(1)},\dots,b^{(s)})$.
The $s$ systems are then written as
\[
\bfA \bfx = \bfb
\qq \text{with} \q
\bfA \in \CC^{N\times N}\,,\q
\bfb \in \CC^{N\times s}\,,\q
\bfx \in \CC^{N\times s}\,.
\]
Standard block Krylov space methods construct in the $n$th iteration
approximate solutions gathered in a block vector $\bfx_n$ chosen
such that
\[
\bfx_n - \bfx_0 \in \calBB_n (\bfA, \bfr_0) \,,
\]
where $\bfx_0$ contains the $s$ initial approximations and
$\bfr_0$ the corresponding initial residuals, while $\calBB_n$
is the Cartesian product
\[
\calBB_n = \underbrace{\calB_n \times \cdots \times \calB_n}%
_{s\text{ times}}
\]
with
\[
\calB_n =
\calK_n (\bfA, r_0^{(1)}) + \cdots + \calK_n (\bfA, r_0^{(s)}) \,.
\]
Here, $\calK_n (\bfA, r_0^{(i)})$ is the usual $n$th Krylov (sub)space
of the $i$th system. It is important, that, in general, the sum in
the last formula is not a direct sum, that is, the Krylov spaces
may have nontrivial intersections.

The \textit{block grade of\/ $\bfr_0$ with respect to\/ $\bfA$} or,
the \textit{block grade of\/ $\bfA$ with respect to\/ $\bfr_0$} is
the positive integer $\bar\nu := \bar\nu(\bfr_0,\bfA)$
defined by
\[
\bar\nu(\bfr_0,\bfA)
= \min \left\{n \big\vert \Dim \calB_n(\bfA, \bfr_0) =
\Dim \calB_{n+1}(\bfA, \bfr_0) \right\}\,.
\]

Among the results we have established for the block grade are the
following ones.

\begin{Lem}\label{lemBI-BlKSS1}
For $n \geq \bar\nu(\bfr_0,\bfA)$,
\[
\calB_n(\bfA, \bfr_0) = \calB_{n+1}(\bfA, \bfr_0)\,,\qq
\calBB_n(\bfA, \bfr_0) = \calBB_{n+1}(\bfA, \bfr_0)\,.
\]
\end{Lem}

\begin{Lem}\label{lemBI-BlKSS5}
The block grade of the block Krylov space and the grades of
the individual Krylov spaces contained in it are related by
\[
\calB_{\bar\nu(\bfr_0,\bfA)}(\bfA, \bfr_0)
= \calK_{\bar\nu(r_0^{(1)},\bfA)}(\bfA, r_0^{(1)}) + \dots
+ \calK_{\bar\nu(r_0^{(s)},\bfA)}(\bfA, r_0^{(s)})\,.
\]
\end{Lem}

\begin{Lem}\label{lemBI-BlKSS2}
The block grade $\bar\nu(\bfr_0,\bfA)$ is characterized by
\[
\bar\nu(\bfr_0,\bfA) =
\min\left\{ n \big\vert \bfA^{-1} \bfr_0 \in \calBB_n(\bfA, \bfr_0) \right\}
\,.
\]
\end{Lem}

\begin{Thm}\label{corBI-BlKSS2'}
Let $\bfxex$ be the block solution of $\bfA\bfx = \bfb$ and let $\bfx_0$ be
any initial block approximation of it and $\bfr_0 := \bfb - \bfA \bfx_0$
the corresponding block residual. Then
\[
\bfxex \in \bfx_0 + \calBB_{\bar\nu(\bfr_0,\bfA)}(\bfA,\bfr_0) \,.
\]
\end{Thm}

We also discuss the effects of the size of the block grade on the
efficiency of a block Krylov space method.


\end{document}
