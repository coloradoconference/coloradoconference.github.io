\documentclass{report}
\usepackage{amsmath,amssymb}
\setlength{\parindent}{0mm}
\setlength{\parskip}{1em}
\begin{document}
\begin{center}
\rule{6in}{1pt} \
{\large Hector Klie \\
{\bf A KRYLOV-KARHUNEN-LOEVE MOMENT EQUATION (KKME) APPROACH FOR SOLVING STOCHASTIC POROUS MEDIA FLOW EQUATIONS}}

Center for Subsurface Modeling \\ Institute for Computational Engineering and Science \\ The University of Texas \\ ACES 5 326 \\ Mailcode C0200 \\ Austin \\ TX \\ 78712
\\
{\tt klie@ices.utexas.edu}\\
Adolfo Rodriguez \\
Mary F. Wheeler\end{center}

The need to quantify and manage uncertainty in several science and
engineering problems has motivated an increasing rate of development in
the field of stochastic equations as viable representative models for
predicting the behavior of associated physical systems. In contrast to
the widespread use of Monte Carlo simulations (MCS), stochastic equations
have shown remarkable potential to develop efficient, accurate and
physical insightful models. The main challenge of solving stochastic
equations is to develop tractable descriptions of the system response in
terms of stochastic differential operators and random fields. In
particular, the solution of stochastic PDE's in porous media flow raises
unexplored challenges to the solution of the discretized system of
equations due to the significant size of standard reservoir problems
combined with the uncertainty induced by error measurements,
discontinuities, nonlinearities in the reservoir parameters at different
spatial and temporal scales.

This work introduces a Krylov-Karhenun-Loeve moment equation (KKLME)
approach for the solution of stochastic PDE's arising in large-scale
porous media flow applications. The approach combines recent developments
in Karhenun-Loeve moment equation methods with a block deflated Krylov
iterative solution of a sequence of deterministic linear algebraic
equations sharing the same matrix operator but different right-hand sides
(RHSs).

In this approach, the log of the random field (i.e., log of permeability)
describing the transmissibility coefficients is decomposed as the sum of
a deterministic average log field plus a mean-zero random fluctuation.
The covariance function associated with the fluctuation component is
further decomposed by means of the Karhunen-Loeve (KL) expansion scheme.
This expansion consists of modes (i.e. stochastic orders) of increasing
frequency but decreasing magnitude. It has been shown that the KL
expansion is of mean square convergence and summation thus implying
significant computational savings. Subsequently, a mixed finite element
procedure (equivalent to a cell-centered finite difference scheme under a
suitable quadrature rule) is employed to derive a system of linear random
algebraic equations.

In order to compute higher-order approximations for the different
pressure moments, a perturbation approach followed by an expansion of
orthogonal random variables is performed to express the variability of
pressures with respect to the random field. The algebraic manipulation of
modes and moments results in a sequence of deterministic linear systems
with multiple RHSs sharing the same matrix operator. This matrix operator
corresponds to the discretization of the average random field that, in
general, has better algebraic properties than the operator associated
with the original random field. A set of independent RHSs becomes
available when a lower moment is computed, and each moment involves the
solution of several modes or combinations of them with previous moment
solutions.

Since the associated average system is generally large and sparse, a
Krylov subspace iterative solver is suitable in this case. The
availability of RHSs describes a particular computational pattern that is
amenable for highly scalable implementations and, at the same time,
imposes particular challenges for an efficient Krylov subspace
implementation. The nature of the algebraic system may range from SPD
systems to highly indefinite non-symmetric systems depending on the
phases considered in the flow model and on the discretization of the
primary variables involved (e.g., pressures and saturations of some given
phases).

With regard to the solution of systems with multiple RHSs, significant
advances have been made in recycling the information generated by Krylov
iterative methods. These advances have been primarily concerned with the
need to improve preconditioning, to develop effective truncation and
restarting strategies, and to perform efficient solutions for a sequence
of linear systems when vectors contained in the Krylov basis are kept
through iterations or different restart cycles. Deflation methods have
been shown to be very effective in these circumstances as they ``remove''
components of the solution associated with extreme eigenvalues that
prevent or slow down convergence. To perform deflation the Krylov
subspace is selectively augmented with approximate eigenvectors that are
either computed during the iteration or somehow known \emph{a priori}
from some geometrical/physical mean.

On the other hand, in order to reduce the negative effects of sequential
inner products on memory performance, it is advisable to rely on block
implementations or, in other words, process a subset of right-hand-sides
in a simultaneous fashion. It is worth noting that there are important
considerations as to the maximum number of RHSs that may be processed
simultaneously, due primarily to memory limitations and the loss of
orthogonality induced by nearly dependent vectors contained in the
(augmented) Krylov subspace. Consequently, a novel block deflation Krylov
iterative method is proposed. The main feature of this implementation is
to enable changes both in the size of the block and in the augmented
subspace as the iteration proceeds. This ensures both numerical stability
and flexibility in replacing deflated RHSs (i.e., associated with
converging solutions) with new RHSs.

Numerical experiments show that KKLME requires significant less computing
time than MCS to converge to the different statistical moments for the
pressure response. Results are shown for single phase flow and for
pressure system arising in two-phase fully implicit formulations using a
block deflation CG iterative method. In light of these preliminary
results, extensions to relate random fields with spectrum information are
required in order to exploit further efficiencies.


\end{document}
