\documentclass{report}
\usepackage{amsmath,amssymb}
\setlength{\parindent}{0mm}
\setlength{\parskip}{1em}
\begin{document}
\begin{center}
\rule{6in}{1pt} \
{\large Ilya Lahuk \\
{\bf STEEPEST DESCENT AND CONJUGATE GRADIENT METHODS WITH VARIABLE PRECONDITIONING}}

Department of Mathematical Sciences \\ University of Colorado at Denver and Health Sciences Center \\ P O Box 173364 \\ Campus Box 170 \\ Denver \\ CO 80217-3364
\\
{\tt ilashuk@math.cudenver.edu}\\
Andrew Knyazev\end{center}

We show that the conjugate gradient method with variable preconditioning
in certain situations cannot give any improvement compared to the
steepest descent method for solving a
linear system with a symmetric positive definite (SPD) matrix of
coefficients. We assume that the
preconditioner is SPD on each step, and that the condition number of the
preconditioned system
matrix is bounded from above by a constant independent of the step
number. Our proof is geometric
and is based on the simple fact that a nonzero vector multiplied by all
SPD matrices with a condition
number bounded by a constant generates a circular cone.

(Submitted to the Student Paper Competition).


\end{document}
