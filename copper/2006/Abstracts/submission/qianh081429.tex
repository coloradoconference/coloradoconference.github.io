\documentclass{report}
\usepackage{amsmath,amssymb}
\setlength{\parindent}{0mm}
\setlength{\parskip}{1em}
\begin{document}
\begin{center}
\rule{6in}{1pt} \
{\large Haifeng Qian \\
{\bf Stochastic Preconditioning for Iterative Linear Equation Solvers}}

1000 8th Street SE \\ Apt 107 \\ Minneapolis \\ MN 55414
\\
{\tt qianhf@ece.umn.edu}\\
Sachin Sapatnekar\end{center}

Stochastic techniques, namely random walks, have been used to form linear
equation solvers since 1940's, but have not been used for
preconditioning, at least not effectively, to the best of our knowledge.
In this talk, We present a new stochastic preconditioning approach: we
prove that for symmetric diagonally-dominant M-matrices, an incomplete
LDL factorization can be obtained from random walks, and used as a
preconditioner for an iterative solver, e.g., conjugate gradient. The
theory can be extended to general matrices with nonzero diagonal entries.

The stochastic preconditioning is performed by a random walk ``game''
defined as follows. Given a finite undirected connected graph
representing a street map, a walker starts from one of the nodes, and
goes to one of the adjacent nodes every day with a certain probability.
The walker pays an amount of money, $m_i$ at node $i$, to a motel for
lodging everyday, until he/she reaches one of the homes, which are a
subset of the nodes. Then the journey is complete and he/she will be
rewarded a certain amount of money, $m_0$. The problem is to determine
the gain function:
\begin{equation}
f(i)=E[\textrm{money earned }|\textrm{walk starts at node }i] \label{eq:fdef}
\end{equation}
These gain values satisfy the following linear equations:
\begin{eqnarray}
& & f(i)=\sum_{j \in \textrm{neighbors of }i}{p_{i,j}f(j)}-m_i
\quad,\quad \forall i \nonumber \\
& & f(\textrm{a home node}) = m_0 \label{eq:game}
\end{eqnarray}
where $p_{i,j}$ is the transition probability of going from node $i$ to
node $j$, and note that $j$ can be a home node. Thus a random walk game
is mapped onto a system of linear equations. Conversely, it can be
verified that given $A \mathbf{x} = \mathbf{b}$, where $A$ is a symmetric
diagonally-dominant M-matrix, we can always construct a random walk game
that is mathematically equivalent, in which the set of $f$ values is
equal to the solution vector $\mathbf{x}$. To find the $i^{\rm th}$ entry
of $\mathbf{x}$, one may run a number of walks from node $i$ and take the
average of the results; to get the complete solution, one may repeat the
process for every entry of $\mathbf{x}$. We alter this normal procedure
by adding the following rule: every calculated node becomes a new home
in the game with an award amount equal to its calculated $f$ value.
Without loss of generality, suppose the nodes are in the natural ordering
$1,2,\cdots,N$, then for walks starting from node $i$, the node set
$\{1,2,\cdots,i-1\}$ are homes where walks terminate (in addition to
homes generated from
the strictly-diagonally-dominant rows of $A$), while the node set
$\{i,i+1,\cdots,N\}$ are motels where walks pass by.

Define the operator ${\rm rev}(\cdot)$ on square matrices such that it
reverses the ordering of the rows and reverses the ordering of the
columns: ${\rm rev}(A)_{i,j}=A_{N+1-i,N+1-j} , \quad \forall i,j \in \{
1,2,\cdots,N \}$.
Let the exact LDL factorization of ${\rm rev}(A)$ be
${\rm rev}(A) = L_{{\rm rev}(A)} D_{{\rm rev}(A)} \left( L_{{\rm rev}(A)}
\right)^{\rm T}$.
Again, in the random walk game, assume that the nodes are in the natural
ordering $1,2,\cdots,N$, and that node $i$ corresponds to the $i^{\rm
th}$ row of $A$.
We prove the following relations:
\begin{eqnarray}
\left( L_{{\rm rev}(A)} \right)_{i,j} & \approx & -
\frac{M_{N+1-j,N+1-i}}{W_{N+1-j}}, \quad\forall i>j
\\
\left( D_{{\rm rev}(A)} \right)_{i,i} & \approx & \frac{W_{N+1-i}
A_{N+1-i,N+1-i}} {J_{N+1-i}}
\end{eqnarray}
where $W_k$ is the total number of walks that are carried out from node
$k$, $M_{k_1,k_2}$ is the number of walks that start from node $k_1$ and
end at $k_2$, and $J_{k}$ is the number of times that the $W_k$ walks
from node $k$ pass node $k$ itself.
These equations show that we can approximate an LDL factorization by
collecting information from random walks.
We further prove that if $\left( L_{{\rm rev}(A)} \right)_{i,j} = 0$ then
$M_{N+1-j,N+1-i} = 0$; in other words, the nonzero pattern of the $L$
factor produced by random walks is a subset of nonzero pattern of the
exact $L_{{\rm rev}(A)}$. Therefore, we conclude that an incomplete LDL
factorization can be obtained from random walks.

We argue that the obtained incomplete LDL factors have better quality,
i.e., better accuracy-size tradeoffs, than the incomplete Cholesky factor
obtained by a traditional method based on Gaussian elimination. Our
argument is based on the
fact that each row in the $L$ factor is independently alculated and has
no correlation with the computation of other rows. Therefore we avoid the
error accumulation in traditional incomplete factorization procedure.

We also discuss, by defining a new set of game rules, how this theory can
be extended to general matrices, given that the diagonal entries are
nonzero.

To evaluate the proposed approach, a set of benchmark matrices are
generated by Y. Saad's SPARSKIT by finite-difference discretization of
the 3D Laplace's equation $\nabla ^2 u = 0$ with Dirichlet boundary
condition. The matrices correspond
to 3D grids with sizes 50-by-50-by-50, 60-by-60-by-60, up to
100-by-100-by-100, and a right-hand-side vector with all entries being 1
is used with each of them. We compare the proposed solver, i.e., random
walk preconditioned conjugate gradient, against ICCG with ILU(0) and ICCG
with ILUT. The complexity metric is the number of double-precision
multiplications needed at the iterative solving stage, in order
to converge with an error tolerance of $10^{-6}$. The results show up to
2.1 times speedup over ICCG, and a trend that the larger and denser a
matrix is, the more the proposed solver outperforms ICCG.

This talk is partially based on \cite{1}, and the implementation is
available to the public \cite{2}.

\begin{thebibliography}{2}

\bibitem{1}
H. Qian and S. S. Sapatnekar,
``A hybrid linear equation solver and its application in quadratic placement,''
\emph{IEEE/ACM International Conference on Computer Aided Design Digest
of Technical Papers}, pp. 905-909, 2005.

\bibitem{2}
H. Qian and S. S. Sapatnekar,
The Hybrid Linear Equation Solver Binary Release, available at\\
http://www.ece.umn.edu/users/qianhf/hybridsolver

\end{thebibliography}


\end{document}
