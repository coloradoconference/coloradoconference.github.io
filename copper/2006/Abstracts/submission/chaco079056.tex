\documentclass{report}
\usepackage{amsmath,amssymb}
\setlength{\parindent}{0mm}
\setlength{\parskip}{1em}
\begin{document}
\begin{center}
\rule{6in}{1pt} \
{\large Luis Chacon \\
{\bf A fully implicit extended 3D MHD solver}}

MS K717 \\ Los Alamos National Laboratory \\ Los Alamos \\ NM 87545
\\
{\tt chacon@lanl.gov}\\
Dana A. Knoll\end{center}

We present results from our research on Jacobian-free Newton-Krylov
(JFNK) methods applied to the time-dependent, primitive-variable,
3D extended magnetohydrodynamics (MHD) equations. MHD is a fluid description
of the plasma state. While plasma is made up of independent (but coupled)
ion and electron species, the standard MHD description of a plasma
only includes ion time and length scales (one-fluid model). Extended
MHD (XMHD) includes nonideal effects such as nonlinear, anisotropic
transport and two-fluid (Hall and diamagnetic) effects. XMHD supports
so-called dispersive waves (whistler, ion acoustic), which feature
a quadratic dispersion relation $\omega \sim k^{2}$. In explicit
time integration methods, this results in a stringent CFL limit $\Delta
t_{CFL}\propto \Delta x^{2}$,
which severely limits their applicability to the study of long-frequency
phenomena in XMHD.

A fully implicit implementation promises efficiency (by removing the
CFL constraint) without sacrificing numerical accuracy.%
\footnote{D. A. Knoll, L. Chac\'{o}n, L. G. Margolin, and V. A. Mousseau, J.
Comput. Phys. \textbf{185}, 583 (2003)
} However, the nonlinear nature of the XMHD system and the numerical
stiffness of its fast waves make this endeavor very difficult. Newton-Krylov
methods can meet the challenge provided suitable preconditioning is
available.

We propose a successful preconditioning strategy for the 3D primitive-variable
XMHD formalism. It is based on {}``physics-based'' ideas,%
\footnote{L. Chac\'{o}n, D. A. Knoll, and J. M. Finn, J. Comput. Phys., \textbf{178},
15 (2002)%
}$^{,}$%
\footnote{L. Chac\'{o}n and D. A. Knoll, J. Comput. Phys., \textbf{188}, 573
(2003)
} in which a hyperbolic system of equations (which is diagonally submissive
for $\Delta t>\Delta t_{CFL}$) is {}``parabolized'' to arrive to
a diagonally dominant approximation of the original system, which
is multigrid-friendly. The use of approximate multigrid (MG) techniques
to invert the {}``parabolized'' operator is a crucial step in the
effectiveness of the preconditioner and the scalability of the overall
algorithm. The parabolization procedure can be properly generalized
using the well-known Schur decomposition of a 2$\times $2 block matrix.
In the context of XMHD, the resulting Schur complement is a system
of PDE's that couples the three plasma velocity components, and needs
to be inverted in a coupled manner. Nevertheless, a system MG treatment
is still possible since, when properly discretized, the XMHD Schur
complement is block diagonally dominant by construction, and block
smoothing is effective.

In this presentation, we will discuss the derivation and validity
of the physics-based preconditioner for resistive MHD and its generalization
to XMHD, the connection with Schur complement analysis, and the system-MG
treatment of the associated systems. A novel second-order, cell-centered,
conservative finite-volume discretization has been recently developed%
\footnote{L. Chac\'{o}n, Comp. Phys. Comm., \textbf{163}, 143 (2004)
} for the XMHD system above, and will be used in this work. It is suitable
for general curvilinear geometries, solenoidal in $\mathbf{B}$ and
$\mathbf{J}$, numerically non-dissipative, and linearly and nonlinearly
stable. We will demonstrate the algorithm using the GEM challenge
configuration.%
\footnote{J. Birn et al., J. Geophys. Res., \textbf{106}, 3715 (2001)%
} Grid convergence studies will demonstrate that CPU time scales scale
optimally as $\mathcal{O}(N)$, where $N$ is the number of unknowns,
and that the number of Krylov iterations scales as $\mathcal{O}(N^{0})$.
Time convergence studies will demonstrate a favorable scaling with
time step $\mathcal{O}(\Delta t^{\alpha })$, with $\alpha <1.0$.


\end{document}
