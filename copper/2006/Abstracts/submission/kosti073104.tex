\documentclass{report}
\usepackage{amsmath,amssymb}
\setlength{\parindent}{0mm}
\setlength{\parskip}{1em}
\begin{document}
\begin{center}
\rule{6in}{1pt} \
{\large Vladimir Kostic \\
{\bf How a step toward wider class of matrices could help improve convergence area of relaxation methods}}

Kralja Petra I 69/44 \\ 21000 Novi Sad \\ Serbia and Montenegro
\\
{\tt vkostic@im.ns.ac.yu}\\
Ljiljana Cvetkovic\end{center}

Investigations are related to several different relaxation methods for
solving systems of linear equations, but the main idea is always the
same: knowing that system matrix is strictly diagonally dominant (SDD),
we can consider it as an S-SDD (see Lj. Cvetkovic, V. Kostic and R. S.
Varga, {\em A new Ger\v sorin-type eigenvalue inclusion area}, ETNA 18,
2004) matrix for every nonempty proper subset S of the set of indices,
and from this fact we can derive, in some sense, an optimal convergence
area for relaxation parameter(s). This convergence area is usually
significantly wider than the corresponding one, obtained from the
knowledge of SDD property, only. Instead of S-SDD class, some more
subclasses of H-matrices can be used for the same purposes.


\end{document}
