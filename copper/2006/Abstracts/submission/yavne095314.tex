\documentclass{report}
\usepackage{amsmath,amssymb}
\setlength{\parindent}{0mm}
\setlength{\parskip}{1em}
\begin{document}
\begin{center}
\rule{6in}{1pt} \
{\large Irad Yavneh \\
{\bf New Algorithms for Vector Quantization}}

Department of Computer Science \\ Technion - Israel Institute of Technology \\ Haifa 32000 \\ Israel
\\
{\tt irad@cs.technion.ac.il}\\
Yair Koren\end{center}

Vector quantization is the classical problem of representing continuum
with only a finite number of representatives or representing an initially
rich amount of discrete data with a lesser amount of representatives.
This problem has numerous applications. The objective of achieving a
quantization with mininmal distortion leads to a hard non-convex
optimization problem, typically with many local minima. The main problem
is thus to find an initial approximation that is close to a
``good'' local minimum. Once such an approximation is found, the
well-known Lloyd-Max iterative algorithm may be used to converge to the
nearby a local minimum. In this talk we will describe the problem and
present two new approaches to its approximate solution.


\end{document}
