\documentclass{report}
\usepackage{amsmath,amssymb}
\setlength{\parindent}{0mm}
\setlength{\parskip}{1em}
\begin{document}
\begin{center}
\rule{6in}{1pt} \
{\large Jonathan Hu \\
{\bf Alternatives to Smoothed Aggregation Multigrid Prolongators for Anisotropic Problems}}

Sandia National Laboratories \\ 7011 East Avenue MS9159 \\ Livermore \\ CA 94551
\\
{\tt jhu@sandia.gov}\\
Ray Tuminaro\end{center}

We consider alternatives to the traditional methods in smoothed
aggregation for generating prolongator operators. The goal is to produce
prolongators that are appropriate for anisotropic operators.

Consider the linear problem $Ax=b$. In the first approach, smoothed
aggregation with basis function shifting, we begin with the standard
smoothed prolongator, $P^{(sm)} = (I - \omega D^{-1} A) P^{(t)},$ where
$P^{(t)}$ is the tentative prolongator. Given a prescribed prolongator
sparsity pattern, this method moves
basis function support (columns of $P^{(sm)}$) from one aggregate to
another to produce a new prologator, $P^{(shift)}$. The shifting is done
in such a way that null space interpolation is maintained.

In the second approach, we consider extensions to building prolongator
operators via energy optimization (Vanek, Mandel, Brezina). In this
method, the sum of the energies of the prolongator basis functions is
minimized, subject to a fixed prolongator nonzero pattern and
interpolation of low-energy
modes. We consider modifications such as filtering $A$ to account for
anisotropies and applying a modified CG method to solve the optimization
problem.

We present numerical experiments that compare the two methods to
traditional smoothed aggregation on a variety of model problems.


\end{document}
