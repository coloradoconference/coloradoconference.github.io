\documentclass{report}
\usepackage{amsmath,amssymb}
\setlength{\parindent}{0mm}
\setlength{\parskip}{1em}
\begin{document}
\begin{center}
\rule{6in}{1pt} \
{\large Merico E. Argentati \\
{\bf A priori error bounds for eigenvalues approximated by the Ritz values}}

Department of Mathematical Sciences \\ University of Colorado at Denver and Health Sciences Center \\ Denver \\ P O Box 173364 \\ Campus Box 170 \\ Denver \\ CO 80217-3364
\\
{\tt rargenta@math.cudenver.edu}\\
Andrew V. Knyazev\end{center}

The Rayleigh-Ritz method finds the stationary values of the Rayleigh
quotient, called Ritz values, on a given trial subspace as optimal, in
some sense, approximations to eigenvalues of a Hermitian operator A. When
a trial subspace is invariant with respect to A, the Ritz values are some
of the eigenvalues of A. Given two finite dimensional subspaces X and Y
of the same dimension, such that X is an invariant subspace of A, the
absolute changes in the Ritz values of A with respect to X compared to
the Ritz values with respect to Y represent the absolute eigenvalue
approximation error. We estimate the error in terms of the principal
angles between X and Y. There are several known results of this kind,
e.g., for the largest (or the smallest) eigenvalues of A, the maximal
error is bounded by a constant times the sine squared of the largest
principal angle between X and Y. The constant is the difference between
the largest and the smallest eigenvalues of A, called the spread of the
spectrum of A.

We prove that the absolute eigenvalue error is majorized by a constant
times the squares of the sines of the principal angles between the
subspaces X and Y, where the constant is proportional to the spread of
the spectrum of A, e.g., for
Ritz values that are the largest or smallest contiguous set of
eigenvalues of A, we show that the proportionality factor is simply one.
Our majorization results imply a very general set of inequalities, and
some of the known error bounds follow as special cases. Majorization
results of this kind are not apparently known in the literature and can
be used, e.g., to derive novel convergence rate estimates of the block
Lanczos method.


\end{document}
