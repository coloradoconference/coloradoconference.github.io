\documentclass{report}
\usepackage{amsmath,amssymb}
\setlength{\parindent}{0mm}
\setlength{\parskip}{1em}
\begin{document}
\begin{center}
\rule{6in}{1pt} \
{\large Andrew Canning \\
{\bf New Iterative Eigensolvers and Preconditioners for Electronic Structure Calculations in Nano and Materials Science}}

LBNL MS-50F \\ One Cyclotron Road \\ Berkeley \\ CA94720
\\
{\tt acanning@lbl.gov}\\
Julien Langou\\
Christof Voemel\\
	Osni Marques, Stanimire Tomov, Gabriel Bester, Lin-Wang Wang \end{center}

Density functional based electronic structure calculations
have become the most heavily used approach in materials science to
calculate materials properties with the accuracy of a full quantum
mechanical treatment of the electrons. This approach results in a
single particle form of the Schrodinger equation which is a non-linear
eigenfunction problem. The standard selfconsistent solution of this problem
involves solving for the lowest eigenpairs corresponding to the electrons in
the system. In non-selfconsistent formulations the problem becomes one of
determining the interior eigenpairs corresponding to the electrons of interest
which are typically around the gap in the eigenvalue spectrum for non-metallic systems.
In this talk I will present results for new iterative eigensolvers based on conjugate
gradients (the LOBPCG method) in the context of plane wave electronic
structure calculations.
This new method gives significant speedup over existing conjugate gradient methods
used in electronic structure calculations. I will also
present results for a new preconditioner based on first solving the bulk structure
corresponding to a given nanosystem and then using that as a
preconditioner to solve the nanosystem.
This new preconditioner gives significant speedup compared to previously
used preconditioners
based on the diagonal of the matrix. These new methods will be demonstrated for
CdSe quantum dots as well as quantum wires constructed from layers of InP and InAs.
(This work was supported by the Director, Office of
Advanced Scientific Computing Research, Division of Mathematical,
Information and Computational Sciences of the U.S. Department
of Energy and the Laboratory Directed Research and Development
Program of Lawrence Berkeley National Laboratory
under contract number DE-AC03-76SF00098)


\end{document}
