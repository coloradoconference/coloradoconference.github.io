\documentclass{report}
\usepackage{amsmath,amssymb}
\setlength{\parindent}{0mm}
\setlength{\parskip}{1em}
\begin{document}
\begin{center}
\rule{6in}{1pt} \
{\large Julien LANGOU \\
{\bf A fast a-posteriori reorthogonalization scheme for the Classical Gram-Schmidt orthogonalization in the context of iterative methods }}

1122 Volunteer Blvd \\ Claxton Bldg Room 233 \\ Knoxville \\ TN 37996-3450
\\
{\tt langou@cs.utk.edu}\end{center}


The year 2005 have been marked by two new papers on the Classical
Gram-Schmidt algorithm (see [1,2]). These results offer a better
understanding of the Classical Gram-Schmidt algorithm. It is finally
proved that the Classical Gram-Schmidt algorithm generates a loss of
orthogonality bounded by the square of the condition number of the
initial matrix. In the first part of the talk,
I will quickly review the proof, explain its key points and its
implication in the context of iterative methods. In the second part, I
will focus on the new results that we have found related to the Classical
Gram-Schmidt algorithm. In particular an a-posteriori reorthogonalization
scheme extremely efficient is given in the context of iterative methods.
(We borrow ideas developped in [3] in the context of GMRES-MGS.)


\begin{itemize}
\item A. Smoktunowicz and J. Barlow (2005).
A note on the error analysis of Classical Gram Schmidt.
Submitted to Numerische Mathematik.
\item Luc Giraud, Julien Langou, Miroslav Rozlo\v{z}n\'{\i}k, and Jasper van den Eshof.
Rounding error analysis of the classical Gram-Schmidt orthogonalization process.
Numerische Mathematik, 101(1):87-100, July 2005.
\item Luc Giraud, Serge Gratton, and Julien Langou.
A rank-$k$ update procedure for reorthogonalizing the orthogonal factor
from modified Gram-Schmidt.
SIAM J. Matrix Analysis and Applications, 25(4):1163-1177, August 2004.
\end{itemize}


\end{document}
