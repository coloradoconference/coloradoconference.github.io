\documentclass{report}
\usepackage{amsmath,amssymb}
\setlength{\parindent}{0mm}
\setlength{\parskip}{1em}
\begin{document}
\begin{center}
\rule{6in}{1pt} \
{\large Jeff Butler \\
{\bf Improving coarsening and interpolation for algebraic multigrid}}

Department of Applied Mathematics \\ University of Waterloo \\ Waterloo Ontario Canada N2L 3G1
\\
{\tt jsbutler@math.uwaterloo.ca}\\
Hans De Sterck\end{center}

Algebraic multigrid (AMG) is one of the most efficient algorithms for
solving large sparse linear systems on unstructured grids. Classical
coarsening schemes such as the standard Ruge-Steuben method [1] can lead
to adverse effects on computation time and memory usage that affect
scalability. Memory and execution time complexity growth is remedied for
various large three-dimensional (3D) problems using the parallel modified
independent set (PMIS) coarsening strategy
developed by De Sterck, Yang, and Heys [2]. However, this
coarsening strategy often leads to erratic grids without a regular
structure that diminish convergence. This talk looks at a modification of
the PMIS algorithm that consistently produces more uniform coarsenings,
and significantly improves convergence properties over the original PMIS
algorithm. Improvements of existing interpolation schemes and application
of the resulting methods to several problems are also examined.


[1] J. Ruge, K. Stueben, Algebraic multigrid (AMG), in: S. McCormick,
ed., Multigrid Methods, vol. 3 of Frontires in Applied Mathematics (SIAM,
1987) 73-130.

[2] H. De Sterck, U. M. Yang, and J. J. Heys, Reducing Complexity in
Parallel Algebraic Multigrid Preconditioners, SIAM Journal on Matrix
Analysis and Applications, to appear, 2006.


\end{document}
