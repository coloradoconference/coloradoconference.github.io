\documentclass{report}
\usepackage{amsmath,amssymb}
\setlength{\parindent}{0mm}
\setlength{\parskip}{1em}
\begin{document}
\begin{center}
\rule{6in}{1pt} \
{\large Andy Wathen \\
{\bf Converging in the right norm}}

Oxford University Computing Laboratory \\ Wolfson Building \\ Parks Road \\ Oxford OX1 3QD \\ UK
\\
{\tt wathen@comlab.ox.ac.uk}\end{center}

For PDE problems, Numerical Analysts would always wish to
establish error estimates in `natural' norms for a given problem.
In the context of iterative solution methods there is similarly
the issue of the right norm for convergence: discrete norms are
equivalent, but measuring a convergence tolerance in a norm in
which half of the variables are scaled by $h^{-5}$ or $h^5$ is
definitely not the right thing to do in general!

In particular this issue arises when preconditioning with
minimum residual methods because then monotonic residual reduction
occurs in a norm based on the preconditioner. (For SPD problems
and Conjugate Gradients it is well known that any SPD
preconditioning does not affect the relevant norm).

In this talk we will discuss this issue in the context of models
of incompressible flow - Poisson, Stokes, Advection-Diffusion and
Navier-Stokes problems - and show how the optimal block
preconditioners developed by Silvester and the author for the
Stokes problem give convergence in the right norm; comments will
also be made regarding Navier-Stokes preconditioning.


This is joint work with Howard Elman and David Silvester


\end{document}
