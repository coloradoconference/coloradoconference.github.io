\documentclass{report}
\usepackage{amsmath,amssymb}
\setlength{\parindent}{0mm}
\setlength{\parskip}{1em}
\begin{document}
\begin{center}
\rule{6in}{1pt} \
{\large Arie, A. Verhoeven \\
{\bf Error analysis of BDF Compound-Fast multirate method for differential-algebraic equations}}

HG 8 47 (Technische Universiteit Eindhoven) \\ Den Dolech 2 \\ 5600 MB Eindhoven \\ The Netherlands
\\
{\tt averhoev@win.tue.nl}\\
Jan, E.J.W. ter Maten\\
Bob, R.M.M.  Mattheij\\
	Beelen        Theo,      T.G.J.
	El Guennouni  Ahmed,     A.
	Tasi\'c       Bratislav  B.\end{center}

Analogue electrical circuits are usually modeled by differential-algebraic
equations (DAE) of type:
\begin{equation}\label{daeqj}\frac{d}{dt}\left[\mathbf{q}(t,\mathbf{x})\right]+\mathbf{j}(t,\mathbf{x})=\mathbf{0},\end{equation}
where $\mathbf{x} \in \mathbb{R}^d$ represents the state of the circuit.
A common analysis is the transient analysis, which computes the solution
$\mathbf{x}(t)$ of this non-linear DAE along the time interval
$[0,T]$ for a given initial state.
Often, parts of electrical circuits have latency or multirate behaviour which
implies the need for multirate integration methods.

For a multirate method it is necessary to partition the variables and
equations into an active (A) and a latent (L) part.
The active and latent parts can be expressed by
$\mathbf{x}_A = \mathbf{B}_A \mathbf{x},\mathbf{x}_L = \mathbf{B}_L \mathbf{x}$ where
$\mathbf{B}_A,\mathbf{B}_L$ are permutation matrices.
Then equation (\ref{daeqj}) is written as the following partitioned system:
\begin{equation}\label{part_DAE}
\begin{array}{c}
\frac{d}{dt}\left [\mathbf{q}_{A}(t,\mathbf{x}_A,\mathbf{x}_L)\right
]+\mathbf{j}_{A}(t,\mathbf{x}_A,\mathbf{x}_L)=\mathbf{0},\\
\frac{d}{dt}\left [\mathbf{q}_{L}(t,\mathbf{x}_A,\mathbf{x}_L)\right
]+\mathbf{j}_{L}(t,\mathbf{x}_A,\mathbf{x}_L)=\mathbf{0}.
\end{array}
\end{equation}
In contradiction to classical integration methods, multirate methods
integrate both parts with different stepsizes or even with different
schemes.
Besides the coarse time-grid $\{T_n,0\leq n \leq N\}$ with stepsizes $H_n
= T_n - T_{n-1}$, also a refined time-grid $\{t_{n-1,m},1\leq n \leq N, 0
\leq m \leq q_n\}$ is used with stepsizes $h_{n,m} = t_{n,m} - t_{n,m-1}$
and multirate factors $q_n$. If the two time-grids are synchronized,
$T_n= t_{n,0} = t_{n-1,q_n}$ holds for all $n$.
There are a lot of multirate approaches for partitioned systems
but we will consider the Compound-Fast version of the BDF methods.
This method performs the following four steps:
\begin{enumerate}
\item The complete system is integrated at the coarse time-grid.
\item The latent interface variables are interpolated at the refined time-grid.
\item The active part is integrated at the refined time-grid, using the
interpolated values at the latent interface.
\item The active solution at the coarse time-grid is updated.
\end{enumerate}
The above methods can be shown to be stable under reasonable conditions.
In this paper we will concentrate on error control.

The local discretization error $\delta^n$ of the compound phase still has
the same behaviour $\delta^n = O(H_n^{K+1})$.
Let $\bar{\mathbf{P}}^n,\bar{\mathbf{Q}}^n$ be the Nordsieck vectors
which correspond to the predictor and corrector polynomials of $\mathbf{q}$.
Then th error $\delta^n$ can be estimated by $\hat{\delta}^n$:
\begin{equation}
\hat{\delta}^n = \frac{-H_n}{T_n - T_{n-K-1}} \left
[\bar{\mathbf{Q}}_{1}^{n}-\bar{\mathbf{P}}_{1}^{n} \right ].
\end{equation}
Now $\hat r_C^n = \|\mathbf{B}_L \hat{\delta}^n\| + \tau\|\mathbf{B}_A
\hat{\delta}^n\| $ is the used weighted error norm, which must satisfy
$\hat r_C^n <$ TOL$_C$.\\
At the refined time-grid the DAE has been perturbed by
the interpolated latent variables. The local discretization error $\delta^{n,m}$
is defined as the residue after inserting the exact solution
in the BDF scheme of the refinement phase.
However during the refinement instead of $\delta^{n,m}$ the perturbed
local error $\tilde{\delta}^{n,m}$ is estimated.
During the refinement each step $\mathbf{x}_A^{n-1,m}$ is computed from
the following scheme:
\begin{equation}\alpha_{n-1,m}
\mathbf{q}_A(t_{n-1,m},\mathbf{x}_A^{n-1,m},\hat{\mathbf{x}}_L^{n-1,m}) +
h_{n-1,m}
\mathbf{j}_A(t_{n-1,m},\mathbf{x}_A^{n-1,m},\hat{\mathbf{x}}_L^{n-1,m}) +
\tilde\beta_{n-1,m} = \mathbf{0}.\end{equation}
Here $\tilde\beta_{n-1,m}$ is a constant which depends
on the previous values of $\mathbf{x}_A$ and $\hat{\mathbf{x}}_L$.
A tedious analysis yields the following asymptotic behaviour:
\begin{equation}\label{deltarelation}
\mathbf{B}_A\delta^{n-1,m} \doteq
\mathbf{B}_A\tilde{\delta}^{n-1,m} + \frac{1}{4}h
\mathbf{K}_{n-1,m}\mathbf{B}_L\rho^{n-1,m}.
\end{equation}
Here $\rho^{n-1,m}$ is the interpolation error at the refined grid and
$\mathbf{K}_{n-1,m}$ is
the coupling matrix.
The perturbed local discretization error $\mathbf{B}_A\tilde{\delta}^{n,m}$ behaves as
$O(h_{n-1,m}^{k+1})$ and
can be estimated
in a similar way as $\delta^n$.
Thus the active error estimate $\mathbf{B}_A\hat\delta^{n-1,m}$
satisfies $\mathbf{B}_A\hat\delta^{n-1,m} \doteq
\mathbf{B}_A\hat{\tilde{\delta}}^{n-1,m} + \frac{1}{4}h
\hat{\mathbf{K}}_{n-1,m}\mathbf{B}_L\hat\rho^{n-1,m}$.
Let $L$ be the interpolation order, then it can be shown that
$\frac{1}{4}\|\hat{\mathbf{K}}_{n}\mathbf{B}_L\rho^{n-1,m}\|$ is less than
\begin{equation}
\hat{r}_I^{n} = \frac{1}{4}\frac{H_n}{T_n - T_{n-L-1}}
\|\hat{\mathbf{K}}_{n}\mathbf{B}_L\left[\bar{\mathbf{X}}_{1}^n-\bar{\mathbf{Y}}_{1}^n\right]
\|.
\end{equation}
Here $\bar{\mathbf{Y}}^n,\bar{\mathbf{X}}^n$ are the Nordsieck vectors
which correspond to the predictor and corrector polynomials of $\mathbf{x}$.
This error estimate $\hat{r}_I^n$ has the asymptotic behaviour
$\hat{r}_I^{n} = O(H_n^{L+1})$.
It follows that
$\|\mathbf{B}_A\hat{\delta}^{n,m}\|$ satisfies:
\begin{equation}\begin{array}{rcl}
\|\mathbf{B}_A\hat{\delta}^{n-1,m}\| &\leq& \hat{\tilde{r}}_A^{n-1,m} +
h\hat{r}_I^{n} =: \hat{r}_A^{n-1,m}.
\end{array}\end{equation}
If $\hat{r}_I^{n} \leq $ TOL$_I = \sigma$TOL$_A$ and
$\hat{\tilde{r}}_A^{n-1,m} \leq \tilde{\mbox{TOL}}_A = (1-\sigma h)$TOL$_A$
then $\hat{r}_A^{n-1,m} \leq \tilde{\mbox{TOL}}_A + h$TOL$_I =$ TOL$_A$.
The weighting factor
$0 < \sigma < \frac{1}{h_{n-1,m}}$ is chosen such that
$(\hat r_C^n/\mbox{TOL}_C) \leq (\hat{r}_I^{n}/\mbox{TOL}_I)$.
Adaptive stepsize control of $H_n,h_{n,m}$ can be used to keep
$\hat{r}_I^{n}=O(H_n^{L+1})$ and $\hat{r}_A^{n-1,m}= O(h_{n-1,m}^{k+1})$
close to
$\theta \mbox{TOL}_I$ and $\theta \tilde{\mbox{TOL}}_A$ respectively,
where $0<\theta<1$ is a safety factor.

We tested a circuit with $5 \times 10$ inverters. The circuit is driven
by $5$ voltage sources which can have
different frequencies. The location of the active part is controlled by the connecting
elements and the voltage sources. The connecting elements were chosen such that
the active part consists of 3 inverters.
We did a Euler Backward Compound-Fast multirate simulation on
$[0,10^{-8}]$ with $\sigma = 0.5,\tau = 0$. It appears that the multirate
method is able to produce accurate results while the computational time
is decreased with factor 13.


\end{document}
