\documentclass{report}
\usepackage{amsmath,amssymb}
\setlength{\parindent}{0mm}
\setlength{\parskip}{1em}
\begin{document}
\begin{center}
\rule{6in}{1pt} \
{\large Aleksei Shestakov \\
{\bf Solution of the nonlinear multifrequency radiation diffusion equation in a multiphysics, high energy density, AMR code}}

Lawrence Livermore National Laboratory \\ POB 808 \\ L-38 \\ Livermore CA 94551
\\
{\tt shestakov1@llnl.gov}\end{center}


\newcommand{\be}{\begin{equation}}
\newcommand{\ee}{\end{equation}}
\newcommand{\bey}{\begin{eqnarray}}
\newcommand{\eey}{\end{eqnarray}}

\newcommand{\pref}[1]{(\ref{#1})}

\newcommand{\cv}{c_v }
\newcommand{\doo}{\partial \Omega }
\newcommand{\doc}{\partial \Omega_c }
\newcommand{\dof}{\partial \Omega_f }
\newcommand{\dfcf}{\delta F_{cf} }
\newcommand{\dfi}{\delta F_i }
\newcommand{\dg}{\Delta_g}
\newcommand{\dk}{\Delta_k}
\newcommand{\dt}{\Delta t}
\newcommand{\dtc}{\Delta t_{c}}
\newcommand{\dtf}{\Delta t_{f}}
\newcommand{\fIm}{F_{I-1/2} }
\newcommand{\fim}{F_{i-1/2} }
\newcommand{\fip}{F_{i+1/2} }
\newcommand{\fjm}{F_{j-1/2} }
\newcommand{\fjp}{F_{j+1/2} }
\newcommand{\fJp}{F_{J+1/2} }
\newcommand{\gb}{\beta }
\newcommand{\gbf}{\beta_f }
\newcommand{\gk}{\kappa }
\newcommand{\gl}{\lambda }
\newcommand{\glf}{\lambda_f }
\newcommand{\oc}{\Omega_c }
\newcommand{\of}{\Omega_f }
\newcommand{\zl}{\zeta_{\ell}}


We describe a scheme to solve the
multifrequency radiation diffusion equation
which is intended for a
multiphysics, high energy density computer code
with adaptive mesh refinement (AMR). In our code,
AMR is implemented by refining in both space and time
\cite{HowGre}. There may be several levels of refinement,
which, going from fine to coarse, are nested
within each other.

We time-advance as follows. Assume there are
only two levels, one coarse, with domain $\oc$ and boundary
$\doc$, and one fine $\of$ with boundary $\dof$. Since the
domains are nested, $\of \subseteq \oc$. At the start of the
time cycle, the equations are first updated
on $\oc$ using a
timestep $\dtc$, a process defined as a
{\em level solve\/} on $\oc$.
If the spatial grid on $\of$
is a twofold refinement of that discretizing $\oc$, we need
two level solves on $\of$, each with
timestep $\dtc/2$, in order to bring the $\of$ solution up
to the advanced coarse level time.
Boundary conditions (BC) are required on $\dof$.
On parts of $\dof$ which do not extend to the physical
boundary, BC are obtained by interpolating the
coarse grid solution. For diffusion equations,
e.g., $u_t = (D u_x)_x$, conventionally, one supplies
Dirichlet data. This ensures
that the coarse and fine grid solution is
continuous across $\dof.$ However,
the flux $-D u_x$ may be discontinuous,
which is unacceptable since
this results in a loss
of conservation. To remedy the defect, after the
level solves,
the coarse and fine grid solutions are {\em synced.} One solves a related,
nearly homogeneous, problem for corrections on the
union of discretizations of $\of$ and $\oc$. The sole
non-homogeneity of the system for the corrections is the
miss-match of the fluxes on $\dof$. When the corrections
are added to the result of the level solves, one obtains a
conservative solution, continuous and with continuous flux.

This paper describes the AMR implementation
for the multigroup radiation diffusion and matter
energy balance equations,
\bey
\partial_t u_g & = & \nabla \cdot D_g \nabla u_g +
\gk_g \, ( \, B_g - u_g \, ) \, ,
\;\; g = 1, \, \ldots, \, G \label{ugeq} \\
\rho \, \cv \partial_t T & = & - \sum_{k=1}^G \dk \,
\gk_k \, ( \, B_k - u_k \, ) \, . \label{emeq}
\eey
In \pref{ugeq}--\pref{emeq}, $u_g$ is the
radiation energy density of the $g$th {\em group.}
Groups arise by
discretizing the frequency domain
$0 \leq \nu \leq \infty$ into $G$ intervals.
In \pref{ugeq}--\pref{emeq},
$D_g$ and $\gk_g$ are the diffusion and
coupling coefficients, $B_g$ is the Planck function,
$\rho$ the mass density, $\cv$ the specific heat,
and $\dk = \nu_k - \nu_{k-1}$.
The system is nonlinear; $D_g$ and $\gk_g$,
which in addition to being strong functions of frequency,
depend on $\rho$ and $T$. For non-ideal gases, $\cv$
depends on $\rho$ and $T$. Equations~\pref{ugeq}--\pref{emeq}
describe the evolution of the
$G$ + 1 unknowns $\{u_k\}_{k=1}^G$ and $T$.

A single level solve of \pref{ugeq}--\pref{emeq}
is a formidable task in itself. For the advance,
we use the procedure described
by Shestakov \cite{She}, generalized for ``real,''
multiple materials whose properties ($\cv$, $k_g$, etc.) are
given in tabular form.

For simulations using AMR,
after advancing on two levels, $\oc$ and $\of$, the
solutions are synced using a generalization of the
Howell and Greenough procedure (HG) \cite{HowGre},
which may be directly applied to \pref{ugeq}--\pref{emeq}
if $G = 1$. However, if $G > 1$, the situation is
more complicated since the energies $u_g$
are coupled. We
resolve the difficulty by applying concepts of the ``Partial
Temperature'' scheme (PT) of Lund and Wilson \cite{LunWil}.
As in PT, we cycle through the groups in random order.
Each group is synced as in HG, but the correction to
$T$ is only a partial change. Only after all the groups
have been addressed, do we obtain the final correction.

Our AMR procedure is implemented in a multiphysics code. Results
will be presented. We simulate effects of strong
explosions in air and compare multigroup results
with runs where the frequency domain is not
discretized, so-called gray diffusion. The simulations
also use the hydro and heat conduction modules. In addition
to the coarse level, there are two levels of refinement.

\begin{thebibliography}{99}

\bibitem{HowGre}
L.\ H.\ Howell and J.\ A. Greenough,
{\em J. Comp.\ Phys.,} 2003, 184 (1), p.\ 53-78.

\bibitem{LunWil}
C.\ M.\ Lund and J.\ R.\ Wilson,
Lawrence Livermore Natl. Lab. report UCRL-84678, July 29, 1980.

\bibitem{She}
A.\ I.\ Shestakov, {\em
8th Copper Mountain Conference
on Iterative Methods,}
March 2004.


\end{thebibliography}


\end{document}
