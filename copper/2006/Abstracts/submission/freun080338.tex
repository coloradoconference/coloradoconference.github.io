\documentclass{report}
\usepackage{amsmath,amssymb}
\setlength{\parindent}{0mm}
\setlength{\parskip}{1em}
\begin{document}
\begin{center}
\rule{6in}{1pt} \
{\large Roland W. Freund \\
{\bf A Flexible Conjugate Gradient Method and its Application in Power Grid Analysis of VLSI Circuits }}

Department of Mathematics \\ University of California \\ Davis \\ One Shields Avenue \\ Davis \\ CA 95616
\\
{\tt freund@math.ucdavis.edu}\end{center}

The design and verification of today's very large-scale
integrated (VLSI) circuits involve some extremely challenging
numerical problems. One of the truly large-scale problems
in this area is power grid analysis. Power grids are modeled
as networks with up to 10 millions nodes. Steady-state analysis
of power grids requires the solution of correspondingly large
sparse symmetric positive definite linear systems. The
coefficient matrices of these systems have the structure of
weighted Laplacians on three-dimensional grids, but with `boundary'
conditions given on a subset of the interior grid points.
Strongly-varying weights and the interior boundary conditions
have the effect that solutions of these linear systems are often
very localized, with components of the solution being near zero
in large parts of the grid. In this talk, we present a flexible
conjugate gradient method that is tailored to the solution of the
truly large-scale linear systems arising in VLSI power grid
analysis. The algorithm allows changing preconditioners and
sparsification of the search directions at each iteration.
These are the key features to exploit the local nature of the
solutions. We also discuss the problem of constructing
efficient preconditioners for the linear systems in VLSI power grid
analysis, and we present results of numerical experiments.


\end{document}
