\documentclass{report}
\usepackage{amsmath,amssymb}
\setlength{\parindent}{0mm}
\setlength{\parskip}{1em}
\begin{document}
\begin{center}
\rule{6in}{1pt} \
{\large Ulrike Meier Yang \\
{\bf On Parallel Algebraic Multigrid Preconditioners for Systems of PDEs}}

Center for Applied Scientific Computing \\ Lawrence Livermore National Laboratory \\ Box 808 \\ L-560 \\ Livermore \\ CA 94551
\\
{\tt umyang@llnl.gov}\end{center}

Algebraic multigrid (AMG) is a very efficient, scalable algorithm for
solving large linear systems on unstructured grids.
When solving linear systems derived from systems of partial differential
equations (PDEs) often a different approach is
required than for those derived from a scalar PDE.
There are mainly two approaches, the function approach (also known
as the ``unknown" approach), and the nodal or ``point" approach.
The function approach defines coarsening and interpolation separately for
each function. The nodal approach uses AMG in a
block manner, where all variables that correspond to the same grid
node are coarsened, interpolated and relaxed together.
While the function approach is much easier to implement and often
more efficient, there are problems for which this approach is not
sufficient and the more expensive nodal approach is needed.

Several parallel implementations of both approaches using
various coarsening schemes and interpolation operators are investigated.
Advantages and disadvantages of both approaches are discussed,
and numerical results are presented.


\end{document}
