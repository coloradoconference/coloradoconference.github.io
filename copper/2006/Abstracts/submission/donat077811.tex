\documentclass{report}
\usepackage{amsmath,amssymb}
\setlength{\parindent}{0mm}
\setlength{\parskip}{1em}
\begin{document}
\begin{center}
\rule{6in}{1pt} \
{\large Marco, M. Donatelli \\
{\bf Regularization by multigrid-type algorithms}}

Universit� dell'Insubria \\ Dip Fisica e Matematica \\ Via Valleggio \\ 11 \\ 22100 Como (CO) - Italy
\\
{\tt marco.donatelli@uninsubria.it}\\
Stefano, S. Serra-Capizzano\end{center}

We consider the de-blurring problem of noisy and blurred images in
the case of space invariant point spread functions. The use of
appropriate boundary conditions leads to linear systems with
structured coefficient matrices related to space invariant
operators like Toeplitz, circulants, trigonometric matrix algebras
etc. We combine an algebraic multigrid (which is designed ad hoc
for structured matrices) with the low-pass projectors typical of
the classical geometrical multigrid employed in a PDEs context.
Thus, using an appropriate smoother, we obtain an iterative
regularizing method (see \cite{mgmreg, Dphd}) based on: projection
in a subspace where it is easier to distinguish between the signal
and the noise and then application of an iterative regularizing
method in the projected subspace. Therefore any iterative
regularizing method like conjugate gradient (CG), conjugate
gradient for normal equation (CGNE), Landweber etc., can be used
as smoother in our multigrid algorithm. The projector is chosen in
order to maintain the same algebraic structure at each recursion
level and having a low-pass filter property, which is very useful
in order to reduce the noise effects. In this way, we obtain a
better restored image with a flatter restoration error curve and
also in less time than the auxiliary method used as smoother.

Like any multigrid algorithm, the resulting technique is
parameterized in order to have more degrees of freedom: a simple
choice of the parameters allows to devise a powerful regularizing
method whose main features are the following:
\begin{enumerate}
\item it is used with early stopping
like any regularizing iterative method
and its cost per iteration is about $1/3$ of the cost of the method used
as smoother (CG, Landweber, CGNE);
\item it can be adapted to work with all the boundary conditions used in literature
(Dirichlet, periodic, Neumann or anti-reflective);
\item the minimal relative restoration error with respect to the true image is
significantly lower with regard to the method used as smoother and
the associated curve of the relative restoration errors with respect
to the iterations is ``flatter'' (therefore the quality of the reconstruction is not
critically dependent on the stopping iteration);
\item when it is applied to the system $A{\bf f}=\bf g$ the minimal relative
error is comparable with regard to all the best known
techniques for the normal equations $A^TA{\bf f}=A^T\bf g$,
but in this case the convergence is much faster;
\end{enumerate}
As direct consequence of point 3, the choice of the exact
iteration where to stop is less critical than in other
regularizing iterative methods while, as a consequence of point 4,
we can choose multigrid procedures which are extremely more
efficient than classical techniques without losing accuracy in the
restored image. Several numerical experiments show the
effectiveness of our proposal. A Theoretical analysis of multigrid
methods is usually a difficult task and a first largely used
approach considers a two grid method. In the same way, to proving
the regularizing properties of our multigrid methods, we provide
some estimations on the filter factor of the two level strategy.

Finally, it can be easily (by using a simple projection at every
step) combined with nonnegativity constraints. Moreover we propose
a possible generalization where the multigrid regularization is
applied as a one-step method: now the only parameter to choose is
the number of recursive calls.

\begin{thebibliography}{biblio}
\bibitem{Dphd}
{\sc M. Donatelli}. {\em Image Deconvolution and Multigrid Methods},
PhD Thesis in Applied and Computational Mathematics, Univ. of Milano, December 2005.
\bibitem{mgmreg}
\textsc{M. Donatelli and S. Serra Capizzano},
\emph{On the regularizing power of multigrid-type algorithms},
SIAM J. Sci. Comput., in press.
\end{thebibliography}


\end{document}
