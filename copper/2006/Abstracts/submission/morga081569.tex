\documentclass{report}
\usepackage{amsmath,amssymb}
\setlength{\parindent}{0mm}
\setlength{\parskip}{1em}
\begin{document}
\begin{center}
\rule{6in}{1pt} \
{\large Ron Morgan \\
{\bf Restarted Lanczos for Nonsymmetric Eigenvalue Problems and Linear Equations}}

Math Department \\ Baylor University \\ Waco \\ TX 76798-7328
\\
{\tt Ronald\_Morgan@baylor.edu}\\
Dywayne Nicely\end{center}

First, a restarted nonsymmetric Lanczos method will be discussed. It can
be used to compute both left and right eigenvectors even when storage is
limited. Approximate eigenvectors are retained at the restart as with
implicitly restarted Arnoldi. It uses a three-term recurrence, but some
reorthogonalization is needed.

The next topic is a related method called BiCG with deflated restarting
(QMR with deflated restarting may also be discussed). It simultaneously
solves linear equations and computes left and right eigenvectors. For the
case of multiple right-hand sides, the eigenvector information from
solving the first right-hand side can help efficiently solve subsequent
right-hand sides. A deflated BiCGStab can be used for this. Deflated
BiCGStab has a projection over the eigenvectors followed by regular
BiCGStab.


\end{document}
