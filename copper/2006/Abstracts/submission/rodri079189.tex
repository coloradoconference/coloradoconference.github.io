\documentclass{report}
\usepackage{amsmath,amssymb}
\setlength{\parindent}{0mm}
\setlength{\parskip}{1em}
\begin{document}
\begin{center}
\rule{6in}{1pt} \
{\large Adolfo Rodriguez \\
{\bf Krylov-Secant methods for estimating subsidence parameters in hydrocarbon reservoirs}}

Center for Subsurface Modeling \\ The University of Texas at Austin \\ Texas 78712
\\
{\tt adolfo@ices.utexas.edu}\\
Hector Klie\\
Mary F. Wheeler\end{center}

It is a well known fact that under primary fluid production
conditions, pore pressures of hydrocarbon reservoirs and aquifers
tend to decline. This decline may lead to severe deformations
inside and around the reservoir. The deformation observed at the
surface is known as subsidence and can produce negative
environmental effects as well as damage to surface facilities and
infrastructures. Positively speaking, subsidence observations can
be combined with inverse algorithms in order to assess reservoir
behavior and detect non-depleted regions that may be subject to
further exploitation. The estimation of subsidence usually
requires the performance of flow simulations coupled to mechanical
deformation. These simulations are computationally intensive and,
for this reason, they are seldom performed.

In this work we introduce a Krylov-secant inversion framework for
estimating the deformation produced as a consequence of pore
pressure reduction in the reservoir. The formulation is based on
a solution of the equilibrium equation where perturbations due to
pore pressure reduction and elastic modulus contrasts are
introduced. The resulting equation for the strain is given in the
form of the Lippmann-Schwinger integral, i.e.,

\begin{equation}
{\bf{e}}\left( {\bf{x}} \right) = - \int\limits_\Omega {\Gamma
\left( {{\bf{x - x'}}} \right)} \left\{ {\Delta {\bf{Ce}}\left(
{{\bf{x'}}} \right) - {\bf{\alpha }}\Delta p} \right\}d{\bf{x'}},
\label{eq1}
\end{equation}
where $\Gamma \left( {{\bf{x - x'}}} \right)$ is the modified half
space Green's function; $\Delta\bf{C}$ is the elastic module
contrast; $\Delta p$ is the magnitude of the pore pressure drop;
and $\alpha$ is the Biot's poroelastic constant, assumed here to
be a tensor. Both $\Delta\bf{C}$ and $\Delta p$ are zero outside
the reservoir but can be functions of $\bf{x'}$. The integration
in (\ref{eq1}) is performed over the reservoir domain.

A Born-type approximation was implemeted, where the total field it is
assumed to be the incident field by analogy to electromagnetic
theory.
The first order Born approximation is given by inserting equation
(\ref{eq2}) in the right hand side of equation (\ref{eq1}).

Based on the discretization of (\ref{eq2}) the resulting inverse
problem can be stated as the minimization of the following
mistmatch functional:
\begin{equation}
\phi =\frac{1}{2} \|\hat{d}-d\|_W^2, \label{eq3}
\end{equation}

where, $d$ and $\hat{d}$ $\in C^n$ are the predicted and observed
data vectors respectively, and $W$ is a diagonal weighting matrix,
based on the inverse of the covariance of the measurements that
compensates for the noise present in the data. In the problem
addressed here, the observed data are the subsidence observations
at some points, while the predicted data vector $d$ is determined
through solution of the forward model (\ref{eq1}). The
minimization problem (\ref{eq3}) is large and ill-posed,
especially in the event of high heterogeneities and few
measurements. To perform the inversion, a Newton-Krylov procedure
based on krylov-scant updates is proposed.

The Krylov-secant framework entails a recycling or extrapolation
of the Krylov information generated for the solution of the
current Jacobian equation to perform a sequence of secant steps
restricted to the Krylov basis. In other words, the Newton step is
recursively composed with Broyden updates constrained to the
reduced Krylov subspace. This is repeated until no further
decrease of the nonlinear residual can be delivered, in which case
a new nonlinear step yielding another Jacobian system is
performed.

The proposed framework includes dynamic control of linear
tolerances (i.e., forcing terms), preconditioning, and
regularization to achieve both efficiency and robustness.
Furthermore, this approach may optionally accommodate the latest
deflation or augmented Krylov basis strategies for further
efficiency. The framework has been previously applied for the
solution of several nonlinear PDE's under Newton-Krylov
implementations, but the present work explores further issues with
respect to inexact Gauss-Newton methods based on Krylov iterative
solutions such as LSQR.

Numerical experiments indicate that the current approach is a
viable option for performing fast inversion implementations.
Comparisons are made against traditional quasi-Newton and
gradient-based implementations. It is concluded that the proposed
approach has the potential for application to electromagnetic,
radar, seismic and medical technologies.


\end{document}
