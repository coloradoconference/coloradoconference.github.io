\documentclass{report}
\usepackage{amsmath,amssymb}
\setlength{\parindent}{0mm}
\setlength{\parskip}{1em}
\begin{document}
\begin{center}
\rule{6in}{1pt} \
{\large C. Doucet \\
{\bf Scaling models and data for solving large sparse linear systems : a comparison of methods}}

Cedric Doucet \\ CEDRAT SA \\ 15 \\ chemin de Malacher \\ 38240 Meylan
\\
{\tt cedric.doucet@cedrat.com}\\
I. Charpentier\\
J.-L. Coulomb\\
	Guerin C.\end{center}

As industrial problems may involve different kinds of physical parameters
and different types of coupled equations, ill-conditionned sparse linear
systems may arise from the discretization method. Let $Au=f$ be a
nonsingular sparse linear system where $A\in \mathbb{C}^{n\times n}$, and
$u,f\in \mathbb{C}^n$. If the spectral condition number $\kappa(A)$ is
too far from one, direct solvers can lack of accuracy and iterative
methods can fail to converge. An economical way of avoiding these
difficulties is to find two diagonal matrices $D_r$ and $D_c$ such that
$\kappa(D_rAD_c) \approx \underset{D_1,D_2}{min}{\kappa(D_1A,D_2)}$.
Then, the solving process becomes
\begin{enumerate}
\item compute $\hat{u}$ such that $\hat{A}\hat{u}=\hat{f}$
\item compute $u = D_c\hat{u}$
\end{enumerate}
where $\hat{A}= D_rAD_c$ and $\hat{f}=D_rf$. Numerical properties of
$\hat{A}$ differ according to the scaling method : it can have normalized
rows/columns[1,5] or it can be approximately doubly stochastic[3,7].
Other methods make the matrix have arbitrary row/column sums[4,6]. In
this paper, we propose to make clear the interests of scaling corrections
for supernodal and multifrontal direct solvers and for preconditioned
iterative methods on industrial applications based on Maxwell equations
(coupled problems, nonlinear materials, moving structures, transient
problems) and discretized by means of nodal or edge finite elements.
\newline
\newline
\text{References}
\newline
[1] J. R. Bunch, Equilibration of symmetric matrices in the max-norm,
ACM, 18, pp. 566-572, 1971.
\newline
[2] P. Butkovic, H. Schneider, Applications of max algebra to diagonal
scaling of matrices, Electronic Journal of Linear Algebra, 13, pp.
262-273, 2005.
\newline
[3] M. F\"{u}rer, Quadratic Convergence for Scaling of Matrices, ANALCO, 2004.
\newline
[4] N. Linial, A. Samorodnitsky, A. Wigderson, A deterministic strongly
polynomial algorithm for matrix scaling and approximate permanents,
Combinatorica, 20, pp. 531-544, 2000.
\newline
[5] O. E. Livine, G. H. Golub, Scaling by Binormalization.
\newline
[6] D. P. O'Leary, Scaling symmetric positive definite matrices to
prescribed row sums, Linear Algebra and its Applications, 370, pp.
185-191, 2003.
\newline
[7] D. Ruiz, A Scaling Algorithm to Equilibrate Both Rows and Columns
Norms in Matrices, RAL-TR-2001-034


\end{document}
