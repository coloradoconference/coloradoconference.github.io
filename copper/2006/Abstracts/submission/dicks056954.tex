\documentclass{report}
\usepackage{amsmath,amssymb}
\setlength{\parindent}{0mm}
\setlength{\parskip}{1em}
\begin{document}
\begin{center}
\rule{6in}{1pt} \
{\large Kelly I. Dickson \\
{\bf Uniformly Well-Conditioned Pseudo-Arclength Continuation}}

243 Harrelson Hall \\ Campus Box 8205 \\ North Carolina State University \\ Raleigh \\ NC 27695
\\
{\tt kidickso@unity.ncsu.edu}\\
C.T. Kelley\\
I. C. F. Ipsen\\
	I. G. Kevrekidis\end{center}

Numerical continuation is the process of solving nonlinear equations of the form
\[G(x,\lambda)=0\]
for various real number parameter values, $\lambda$. The obvious
approach, called natural parameterization, is to perturb $\lambda$ with
each continuation step and find the corresponding solution $x$ via a
nonlinear solver (Newton's method). While this approach is reasonable for
paths containing only regular points (points $(x,\lambda)$ where the
Jacobian matrix of $G$ is nonsingular), the approach breaks down at
simple fold points where the Jacobian matrix of $G$ becomes singular and
Newton's method fails. In order to remedy this, one may implement
pseudoarclength continuation (PAC) which introduces a new parameter based
on the arclength $s$ of the solution path. In order to implement PAC, one
converts the old problem $G(x,\lambda)=0$ to a new problem
\[F(x(s),\lambda(s))=0.\]
Using PAC on the new problem requires the Jacobian matrix of $F$, $F'$,
which ought to be nonsingular at both regular points and simple folds if
we have indeed bypassed the problem that natural parameterization
presents. While the nonsingularity of $F'$ at regular points and simple
folds is a known fact, we present a theorem that gives conditions under
which $F'$ is uniformly nonsingular for a path containing simple folds.
We do this by bounding the smallest singular value of $F'$ from below.
The theorem justifies the use of PAC in a practical way for solution
curves containing nothing ``worse'' than a simple fold.


\end{document}
