\documentclass{report}
\usepackage{amsmath,amssymb}
\setlength{\parindent}{0mm}
\setlength{\parskip}{1em}
\begin{document}
\begin{center}
\rule{6in}{1pt} \
{\large Jonathan W Boyle \\
{\bf Algebraic multigrid and convection diffusion problems}}

School of Mathematics \\ The University of Manchester \\ Sackville Street \\ Manchester \\ M60 1QD \\ UK
\\
{\tt j.boyle@manchester.ac.uk}\\
David J Silvester\end{center}

The talk discusses the use of algebraic multigrid (AMG) when solving
discrete convection diffusion problems. For practical applications, e.g.
modelling incompressible fluid flow, the interest is in using AMG as a
preconditioning component within a Krylov subspace solver like GMRES. To
be effective in this role, the AMG method should be 'black box', i.e.
require no user tuning, and be effective in the sense that if applied as
a solver for a convection-diffusion subproblem then iteration counts are
independent of mesh size and other problem parameters.

When designing AMG for convection diffusion, it is necessary to consider
the interplay between wind direction and AMG smoothing scheme. Numerical
test data is presented showing performance for AMG applied to streamline
diffusion stabilized finite element discretizations of model problems.
The results demonstrate the crucial importance of choosing an appropriate
smoothing
scheme. The performance of a "well-chosen" AMG method in the context of
pressure-convection diffusion preconditioning of the Navier-Stokes
equations will also be discussed.

This is ongoing work. It is funded by EPSRC grant EP/C000528/1.


\end{document}
