\documentclass{report}
\usepackage{amsmath,amssymb}
\setlength{\parindent}{0mm}
\setlength{\parskip}{1em}
\begin{document}
\begin{center}
\rule{6in}{1pt} \
{\large Jan Mandel \\
{\bf Adaptive Selection of Face Coarse Degrees of Freedom in the BDDC and the FETI-DP Iterative Substructuring Methods}}

Department of Mathematical Sciences \\ University of Colorado at Denver \\ and Health Sciences Center \\ Campus Box 170 \\ P O Box 173364 \\ Denver \\ CO 80217-3364
\\
{\tt jmandel@math.cudenver.edu}\\
B. Sousedik\end{center}

We propose adaptive selection of the coarse space of the BDDC and
FETI-DP iterative substructuring methods by adding coarse degrees
of freedom with support on selected intersections of
adjacent substructures. The coarse degrees of freedom are constructed using
eigenvectors associated with the intersections. The minimal number
of coarse degrees of freedom on the selected intersections is added to decrease
a heuristic indicator of the the condition number under a target
value specified a priori. It is assumed that the starting coarse
degrees of freedom are already sufficient to prevent relative rigid body motions
of any selected pair of adjacent substructures. It is shown
numerically on 2D elasticity problems that the indicator based on
pairs of substructures with common edges predicts reasonably well
the actual condition number, and that the method can select
adaptively the hard part of the problem and concentrate
computational work there to achieve good convergence of the
iterations at a modest cost.


\end{document}
