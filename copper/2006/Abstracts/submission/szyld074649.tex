\documentclass{report}
\usepackage{amsmath,amssymb}
\setlength{\parindent}{0mm}
\setlength{\parskip}{1em}
\begin{document}
\begin{center}
\rule{6in}{1pt} \
{\large Daniel B Szyld \\
{\bf Optimal Additive Schwarz Preconditioning for Minimal Residual Methods with Euclidean and Energy Norms}}

Department of Mathematics (038-16) \\ Temple University \\ 1805 N Broad Street \\ Philadelphia PA 19122-6094 \\ USA
\\
{\tt szyd@temple.edu}\\
Marcus Sarkis\end{center}

For the solution of non-symmetric or indefinite linear
systems arising from discretizations of elliptic problems,
two-level additive Schwarz preconditioners are known
to be optimal in the sense that convergence bounds
for the preconditioned problem are
independent of the mesh and the number of subdomains.
These bounds are based on some kind of {\em energy norm}.
However, in practice, iterative methods which
minimize the Euclidean norm of the residual are used,
despite the fact that the usual bounds are non-optimal, i.e.,
the quantities appearing in the bounds may depend on
the mesh size; see [X.-C.\ Cai and J.\ Zou,
{\em Numer.\ Linear Algebra Appl.}, 9:379--397, 2002].
In this paper,
iterative methods are presented which minimize the same
energy norm in which the optimal Schwarz bounds are derived,
thus maintaining the Schwarz optimality.
As a consequence, bounds for the Euclidean norm minimization
are also derived, thus providing a theoretical justification
for the practical use of Euclidean norm minimization methods
preconditioned with additive Schwarz.
Both left and right preconditioners are considered, and
relations between them are derived. Numerical experiments
illustrate the theoretical developments.


\end{document}
