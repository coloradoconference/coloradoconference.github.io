\documentclass{report}
\usepackage{amsmath,amssymb}
\setlength{\parindent}{0mm}
\setlength{\parskip}{1em}
\begin{document}
\begin{center}
\rule{6in}{1pt} \
{\large Travis M Austin \\
{\bf Multilevel Homogenization Techniques for the Cardiac Bidomain Equations}}

Bioengineering Institute \\ University of Auckland \\ Private Bag 92019 \\ Auckland \\ New Zealand
\\
{\tt t.austin@auckland.ac.nz}\\
Mark L Trew\\
Andrew J Pullan\end{center}

The cardiac bidomain equations are a set of nonlinear partial
differential equations that are used to model the flow of current within
cardiac tissue by treating intracellular and extracellular space as two
interpenetrating domains. In the past 10 years research groups around the
world have been using the bidomain equations in a variety of
sophisticated ways, from modelling fibrillation in the human heart to
understanding how plunge electrodes affect potential fields during in
vitro experiments on cardiac tissue. The Bioengineering Institute at the
University of Auckland has been a world leader in imaging cardiac tissue
at a microscale resolution, and discovering specialized features that can
be incorporated into the bidomain model.

In this talk, I will briefly introduce the Bioengineering
Institute's imaging work, and then focus on how we are using these
imaging results in our modelling framework. This discussion will focus on
how we are using Black Box Multigrid to generate homogenized models that
take into account the discontinuities found at the mezoscale. Such models
allow the effect of the discontinuous cardiac structures to be seen in
the potential fields at a reduced cost. This work is founded upon the
multilevel upscaling approach of MachLachlan and Moulton (Water Resources
Research, 2005), and begins exploring their ideas in three-dimensions and
in a time-dependent framework.



\end{document}
