\documentclass{report}
\usepackage{amsmath,amssymb}
\setlength{\parindent}{0mm}
\setlength{\parskip}{1em}
\begin{document}
\begin{center}
\rule{6in}{1pt} \
{\large David Echeverr\'{i}a \\
{\bf On the Manifold-Mapping \\Optimization Technique}}

Centre for Mathematics and Computer Science (CWI)\\ \\ Kruislaan 413 \\ NL-1098 SJ Amsterdam \\ The Netherlands
\\
{\tt d.echeverria@cwi.nl p.w.hemker@cwi.nl}\\
Pieter W. Hemker\end{center}

\newcommand{\ff}{{\mathbf f}}
\newcommand{\pp}{{\mathbf p}}
\newcommand{\ppb}{\bar{{\mathbf p}}}
\newcommand{\xx}{{\mathbf x}}
\renewcommand{\ss}{{\mathbf s}}
\newcommand{\yy}{{\mathbf y}}
\newcommand{\cc}{{\mathbf c}}
\newcommand{\RR}{\ensuremath{\mathbb{R}}}
\thispagestyle{empty}

Optimization problems in practice often need cost-function
evaluations that are very expensive to compute. Examples are, e.g.,
optimal design problems based on complex finite element simulations.
As a consequence, many optimizations may require very long computing
times. The space-mapping (SM) technique \cite{smlist010,survey2004}
was developed as an alternative in these situations.

In SM terminology, the accurate but expensive-to-evaluate models are
called {\it fine} models, \mbox{$\ff:X \subset \RR^n \to \RR^m$}. The
SM method also needs a second, simpler and cheaper
model, the {\it
coarse} model, \mbox{$\cc:Z \subset \RR^n \to \RR^m$}, in order to
speed-up the optimization process. The key element in this technique
is a right-preconditioning for the coarse model, known as the {\it SM function}
\mbox{$\pp:X \to Z$}, that aligns the two model responses. The
function $\cc(\pp(\xx))$ corrects the coarse model and can be used
as a surrogate for the fine model in the accurate optimization. In
most cases the SM function is much simpler than the fine model, in
the sense that it is easier to approximate. This fact endows the SM
technique with its well-reported efficiency. However, it does not always
converge to the right solution.

Defect-correct theory \cite{dcp} helps to see that, in order to
achieve the accurate optimum, the SM function is generally
insufficient and also left-preconditioning is needed. \mbox{In
\cite{paper01}} we introduce the mapping \mbox{$\ss:\cc(Z) \to
\ff(X)$} and the associated manifold-mapping (MM) algorithm. MM
employs $\ss(\cc(\ppb(\xx)))$ as the fine model surrogate.
Here, the function $\ppb:X \to Z$ is not the above SM function
but an arbitrary simple bijection, often the identity. The MM algorithm is as
efficient as SM but converges to the accurate \mbox{optimal solution
\cite{paper01,paper02}}.

In the first part of the presentation the MM algorithm will be
briefly introduced and a proof of convergence will be given. The use
of more than two models (multi-level approach) and the possibility
of having a coarse model with a different dimension than the fine
one ($X \subset \RR^{n_\ff}$ and $Z \subset \RR^{n_\cc}$ with $n_\ff
\neq n_\cc$) will be the issues dealt with in the second part of the
talk.

\begin{thebibliography}{1}
\bibitem{smlist010} J.~W.~Bandler, R.~M.~Biernacki,
C.~H.~Chen, P.~A.~Grobelny and R.~H.~Hemmers,
{\em Space Mapping Technique for Electromagnetic Optimization},
pp. 2536--2544, IEEE Trans. on Microwave Theory and Techniques,
\mbox{42 ({\bf 12}), 1994}.
\bibitem{survey2004} J.~W.~Bandler, Q.~S.~Cheng, S.~A.~Dakroury,
A.~S.~Mohamed, M.~H.~Bakr, K.~Madsen and J.~S{\o}ndergaard,
{\em Space Mapping: The State of the Art}, pp. 337--361, IEEE Trans. on Microwave
Theory and Techniques, \mbox{52 ({\bf 1}), 2004}.
\bibitem{dcp} K. B{\"o}hmer and P. W. Hemker and H. J. Stetter, {\em
Defect Correction Methods: Theory and Applications}, The defect
correction approach, Computing Suppl. 5, pp. 1--32, K. B{\"o}hmer and H.
J. Stetter ed., Springer-Verlag, Berlin, Heidelberg, New York, Tokyo,
1984.
\bibitem{paper01} D. Echeverr\'{\i}a, and P.W. Hemker, {\em Space mapping
and defect correction}, Comp. Methods in Appl. Math., 5 ({\bf 2}), pp.
107--136, 2005.
\bibitem{paper02} D. Echeverr\'{i}a, D. Lahaye, L. Encica, E.A. Lomonova,
P.W. Hemker and A.J.A. Vandenput,
{\em Manifold-Mapping Optimization Applied to Linear Actuator Design},
IEEE Transactions on Magnetics, 2006. Accepted for publication.
\end{thebibliography}
\thispagestyle{empty}


\end{document}
