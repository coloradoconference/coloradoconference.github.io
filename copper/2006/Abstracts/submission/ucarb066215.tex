\documentclass{report}
\usepackage{amsmath,amssymb}
\setlength{\parindent}{0mm}
\setlength{\parskip}{1em}
\begin{document}
\begin{center}
\rule{6in}{1pt} \
{\large Bora Ucar \\
{\bf Partitioning sparse matrices for parallel preconditioned iterative methods}}

Department of Mathematics and Computer Science \\ Emory University \\ Suite W401 \\ 400 Dowman Drive \\ Atlanta \\ Georgia 30322
\\
{\tt ubora@mathcs.emory.edu}\\
Cevdet Aykanat\end{center}

We will discuss parallelization techniques for the preconditioned
iterative methods that use explicit preconditioners such as approximate
inverses or factored approximate inverses. Applications of these
preconditioners require sparse matrix-vector multiplication (SpMxV)
operations. Roughly speaking, the problem reduces to partitioning two or
more matrices together in order to efficiently parallelize the
computations of the form $y\gets ABCx$. Note that the computations
$y\gets ABCx$ are performed as successive SpMxV operations, and hence
there are dependencies among the input and output vectors of these SpMxV
operations. Additional dependencies are imposed by the linear vector
operations that take part in a full step of the chosen iterative method.
We will first discuss how to analyze the preconditioned iterative methods
to determine the dependencies between the inputs and outputs of the SpMxV
operations. We will give a short account of such dependencies for a
number of widely used methods including BiCGStab, preconditioned
conjugate gradients, and GMRES. Next, we will develop hypergraph models
which capture the dependencies among the input and output vectors of the
SpMxV operations with different matrices. We will show that partitioning
a single hypergraph amounts to simultaneous partitioning of the matrices
in
$y\gets ABCx$ computations in such a way that the total volume of
communication is minimized and an appropriate balance criterion among the
processor loads is maintained. We will present experimental results
obtained using a parallel implementation of the right preconditioned
BiCGStab method on a PC cluster.

This is a joint work with Prof C. Aykanat of Bilkent University,
Ankara, Turkey.


\end{document}
