\documentclass{report}
\usepackage{amsmath,amssymb}
\setlength{\parindent}{0mm}
\setlength{\parskip}{1em}
\begin{document}
\begin{center}
\rule{6in}{1pt} \
{\large Kees Vuik \\
{\bf Complex shifted-Laplace preconditioners for the Helmholtz equation}}

Delft University of Technology \\ Faculty of Electrical Engineering \\ Mathematics and Computer Science \\ Delft Institute of Applied Mathematics \\ Mekelweg 4 \\ 2628 CD Delft \\ The Netherlands
\\
{\tt c.vuik@tudelft.nl}\\
Yogi Erlangga\\
Kees Oosterlee\\
	Martin Van Gijzen\end{center}

Complex shifted-Laplace preconditioners for the Helmholtz equation
\\[2ex]
C. Vuik, Y.A. Erlangga, C.W. Oosterlee and M.B. van Gijzen\\
Delft University of Technology\\
Faculty of Electrical Engineering,
Mathematics and Computer Science\\
Delft Institute of Applied Mathematics\\
Mekelweg 4,
2628 CD Delft,
The Netherlands \\
{\bf e-mail: c.vuik@tudelft.nl}\\
[2ex]
In this paper, the time-harmonic wave equation in heterogeneous media
is solved
numerically. The underlying equation governs wave propagations and
scattering
phenomena arising in many area's, e.g. in aeronautics, geophysics, and
optical
problems. In particular, we look for efficient iterative
solution techniques for the Helmholtz equation
discretized by finite difference discretizations. Since the number of
grid points per wavelength should be sufficiently large for accurate
solutions, for very high wavenumbers the discrete problem becomes
extremely large, thus prohibiting the use of direct solvers.
However, since the coefficient matrix is sparse,
iterative solvers are an interesting alternative.

In many geophysical applications that are of our interest,
an unbounded domain is used. In our model we
approximate such a domain by a bounded domain, where appropriate
boundary conditions are used to prevent spurious reflections. As
boundary conditions we compare the following possibilities:
Dirichlet, Neumann, Sommerfeld, Absorbing Layer and Perfect Matched
Layer. Due to the boundary conditions and damping in the heterogeneous
medium, the coefficient
matrix is complex-valued.

It appears that standard iterative solvers (ILU preconditioned
Krylov solver, Multigrid, etc.) fail for the Helmholtz equation, if the
wavenumber becomes sufficiently high. In this paper we
present a Bi-CGSTAB solution method combined with a
novel preconditioner for high wavenumbers. The preconditioner is based
on the inverse of an
Helmholtz operator, where an artificial damping term is added to the
operator. This preconditioner can
be approximated by multigrid. This is somewhat surprising as
multigrid, without
enhancements, has convergence troubles for the original Helmholtz
operator at high wavenumbers.

Currently, we are investigating the best choice of the damping
term. If the damping term is small Bi-CGSTAB
converges fast, but it is difficult to use multigrid for the
preconditioner. On the other hand, if the damping
term is large the multigrid approximation is very good, but the
convergence of Bi-CGSTAB is slow. So a compromise is required to
obtain an efficient solver. To find a good value of the damping term
we study the spectral properties of the preconditioned
matrix. It appears that an eigenvalue analysis of this matrix
can be used to predict
the convergence of GMRES. In practice it appears that these insights
can also be used for the convergence of Bi-CGSTAB.
We conclude that Bi-CGSTAB combined with the novel preconditioner
converges satisfactorily for all choices of the boundary conditions.


\end{document}
