\documentclass{report}
\usepackage{amsmath,amssymb}
\setlength{\parindent}{0mm}
\setlength{\parskip}{1em}
\begin{document}
\begin{center}
\rule{6in}{1pt} \
{\large Heidi K. Thornquist \\
{\bf Fixed-Polynomial Approximate Spectral Transformations for Preconditioning the Eigenvalue Problem}}

Sandia National Laboratories \\ P O Box 5800 \\ MS 0316 \\ Albuquerque \\ NM 87185-0316
\\
{\tt hkthorn@sandia.gov}\\
Danny C. Sorensen\end{center}

Arnoldi's method is often used to compute a few eigenvalues and
eigenvectors of large, sparse matrices. When the eigenvalues of interest
are not dominant or well-separated, this method may suffer from slow
convergence. Spectral transformations are a common acceleration technique
that address this issue by introducing a modified eigenvalue problem that
is easier to solve than the original. This modified problem accentuates
the eigenvalues of interest, but requires solving a linear system, which
is
computationally expensive for large-scale eigenvalue problems.

In this talk we will show how this expense can be reduced through a
preconditioning scheme that uses a fixed-polynomial operator to
approximate the spectral transformation. Three different constructions
for a fixed-polynomial operator are derived from some
common iterative methods for non-Hermitian linear systems.
The implementation details and numerical behavior of these three
operators are compared. Numerical experiments will be presented
demonstrating that this preconditioning scheme is a competitive approach
for solving large-scale eigenvalue problems. The results
illustrate the effectiveness of this technique using several practical
eigenvalue problems from science and engineering ranging from hundreds to
more than a million unknowns.


\end{document}
