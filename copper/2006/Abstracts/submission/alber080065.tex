\documentclass{report}
\usepackage{amsmath,amssymb}
\setlength{\parindent}{0mm}
\setlength{\parskip}{1em}
\begin{document}
\begin{center}
\rule{6in}{1pt} \
{\large David Alber \\
{\bf Parallel Coarse Grid Selection Strategies}}

Siebel Center for Computer Science \\ University of Illinois at Urbana-Champaign \\ 201 North Goodwin Avenue \\ Urbana \\ IL 61801
\\
{\tt alber@uiuc.edu}\\
Luke Olson\end{center}

Traditional coarse grid selection algorithms for algebraic multigrid use
a strength of connection measure to select coarse degrees of freedom. The
strength of connection is a heuristic used to determine the influences
between degrees of freedom in M-matrices. Coarsening algorithms using a
strength of connection are known to select ineffective coarse grids for
some cases where the operator is not an M-matrix. Additionally, these
methods do not consider other information such as the smoother to be used
in the solve phase.

Alternatively, compatible relaxation selects coarse grids without
explicitly using a strength of connection measure. Instead, a smoother is
applied to identify degrees of freedom where the smooth error is large.
This information is then used to select the coarse grid. Recent work on
compatible relaxation has produced viable serial implementations and
useful theoretical results.

The goal of this work is to produce effective and efficient parallel
compatible relaxation methods. In this talk, parallel compatible
relaxation implementations will be introduced and discussed, along with
results from experiments on both structured and unstructured problems.


\end{document}
