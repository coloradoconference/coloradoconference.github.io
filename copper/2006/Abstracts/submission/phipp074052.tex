\documentclass{report}
\usepackage{amsmath,amssymb}
\setlength{\parindent}{0mm}
\setlength{\parskip}{1em}
\begin{document}
\begin{center}
\rule{6in}{1pt} \
{\large Eric T Phipps \\
{\bf Solving Bordered Systems of Linear Equations for Large-Scale Continuation and Bifurcation Analysis}}

Sandia National Laboratories \\ Applied Computational Methods Department \\ PO Box 5800 MS-0316 \\ Albuquerque \\ NM 87185
\\
{\tt etphipp@sandia.gov}\\
Andrew G Salinger\end{center}

Solving bordered systems of linear equations where the matrix is
augmented by a small number of additional rows and columns is ubiquitous
in continuation and bifurcation analysis. Examples include
pseudo-arclength continuation, constraint following, and turning point
location. However solving these systems in a large-scale setting where
the original matrix is large and sparse is difficult. Directly augmenting
the matrix destroys the sparsity structure of the original matrix since
the additional rows and columns are usually dense, while block
elimination methods have difficulty when the original matrix is nearly
singular and result in additional linear solves.

In this talk we discuss a simple method for solving systems of this form
using Krylov iterative linear solvers based on computing the $QR$
factorization of the augmented rows, and is an extension of the
Householder pseudo-arclength continuation method developed by H. Walker.
It allows solutions of the bordered system to be computed with a cost
roughly equivalent to solving the original matrix and is well-conditioned
even when the original matrix is singular.

We then apply this technique to the problem of computing turning point
bifurcations in large-scale nonlinear systems. The $QR$ approach allows
turning point algorithms that are faster, more robust and scale better to
millions of unknowns compared to traditional block elimination schemes.
Examples of applying these techniques to large-scale structural and fluid
mechanics problems will be presented. These techniques have been
implemented in a continuation and bifurcation software package called
LOCA, short for The Library of Continuation Algorithms, developed by the
authors and publicly available as a part of Trilinos, a set of scalable
linear and nonlinear solvers.


\end{document}
