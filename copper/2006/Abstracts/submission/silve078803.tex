\documentclass{report}
\usepackage{amsmath,amssymb}
\setlength{\parindent}{0mm}
\setlength{\parskip}{1em}
\begin{document}
\begin{center}
\rule{6in}{1pt} \
{\large David Silvester \\
{\bf Least Squares Preconditioners for Stabilized Mixed Aproximation of the Navier-Stokes Equations }}

School of Mathematics \\ University of Manchester \\ Sackville Street \\ Manchester M60 1QD \\ UK
\\
{\tt d.silvester@manchester.ac.uk}\\
Howard Elman\\
Victoria Howle\\
	Shadid, John
	Tuminaro, Ray\end{center}

We consider the Navier--Stokes equations
\begin{equation} \label{NS}
\begin{array}{rcl}
-\nu \nabla^2 {\bf u}
+ ({\bf u} \cdot {\rm grad})\, {\bf u} +
{\rm grad}\, p &= &{\bf f} \\
-{\rm div}\, {\bf u} &= &0
\end{array}
\end{equation}
on $\Omega \subset \mathbb{R}^d$, $d=2$ or $3$. Here, ${\bf u}$
is the $d$-dimensional velocity field, which is assumed to satisfy
suitable boundary conditions on $\partial \Omega$, $p$ is the
pressure, and $\nu$ is the kinematic viscosity, which is
inversely proportional to the Reynolds number.

Linearization and discretization of (\ref{NS}) by finite elements, finite
differences or finite volumes leads to a sequence of linear systems of
equations of the form
\begin{equation} \label{NSdiscretelinearStabilized}
\left[
\begin{array}{lc}
F & \ B^T \\ B & -\frac{1}{\nu}C
\end{array}
\right]
\begin{bmatrix}
{\bf u}\\ p
\end{bmatrix}
=
\begin{bmatrix}
{\bf f}\\ g
\end{bmatrix}.
\end{equation}
These systems, which are the focus of this talk, must be solved
at each step of a nonlinear (Picard or Newton) iteration. Here, $B$ and
$B^T$ are matrices corresponding to discrete divergence and gradient
operators, respectively and $F$ operates on the discrete velocity space.
For {\em div-stable} discretizations, $C=0$. For mixed approximation
methods that do not
uniformly satisfy a discrete inf-sup condition, the matrix $C$ is
a nonzero {\em stabilization operator}.
Examples of finite element methods that require stabilization are
the mixed approximations using linear or bilinear velocities
(trilinear in three-dimensions) coupled with constant pressures,
as well as any discretization in which equal order discrete velocities and
pressures are specified using a common set of nodes.

The focus of this talk is the Least Squares Commutator (LSC)
preconditioner developed by Elman, Howle, Shadid, Shuttleworth and
Tuminaro, and
unveiled at the Copper Mountain Conference in 2004.
This preconditioning methodology is one of several choices that are
effective for Navier-Stokes equations, and it has the advantage of being
defined from strictly algebraic considerations.
The resulting preconditioning methodology is competitive with the
pressure convection-diffusion preconditioner of Kay, Loghin and Wathen,
and in some cases its performance is superior.
However, the LSC approach has so far only been shown to be applicable to
the case where $C=0$ in (\ref{NSdiscretelinearStabilized}).
In this talk we show that the least squares commutator
preconditioner can be extended to cover the case of mixed
approximation that require stabilization.
This closes a gap in the derivation of these ideas, and a version of the
method can be also formulated from algebraic considerations, which enables
the fully automated algebraic construction of effective preconditioners
for the Navier-Stokes equations by essentially using only properties of the
matrices in (\ref{NSdiscretelinearStabilized}).

Our focus in this work is on steady flow problems although the ideas discussed
generalize in a straightforward manner to unsteady flow.


\end{document}
