\documentclass{report}
\usepackage{amsmath,amssymb}
\setlength{\parindent}{0mm}
\setlength{\parskip}{1em}
\begin{document}
\begin{center}
\rule{6in}{1pt} \
{\large S\'ebastien Loisel \\
{\bf A domain decomposition method that converges in two steps for three subdomains.}}

2�4 \\ rue du Li\`evre \\ Case postale 64 \\ 1211 Gen\`eve 4 (Suisse)
\\
{\tt loisel@math.unige.ch}\end{center}

In Schwarz-like domain decomposition methods, a domain $\Omega$ is
broken into two or more subdomains and Dirichlet, Neumann, Robin or
pseudo-differential problems are iteratively solved on each subdomain.
For certain problems, it is well-known that the Dirichlet-Neumann
iteration for two subdomains will converge in two steps. Let $\Omega$
be an open domain and $\Omega_{1},\Omega_{2},\Omega_{3}$ a domain
decomposition of $\Omega$ such that each pair of subdomains shares
an interface (for instance, $\Omega=\{ z\in\Bbb C|\;|z|<1\}$ and
$\Omega_{j}=\{ re^{i\theta}|\;0<r<1\textrm{ and }\theta\in(2j\pi/3,2(j+1)\pi/3)\}$,
$j=1,2,3$.) We will show a new Schwarz-like domain decomposition
method that converges in two iterations in this situation.


\end{document}
