\documentclass{report}
\usepackage{amsmath,amssymb}
\setlength{\parindent}{0mm}
\setlength{\parskip}{1em}
\begin{document}
\begin{center}
\rule{6in}{1pt} \
{\large Leonid Berlyand \\
{\bf  Discrete Network Approximation for Singular Behavior of the Effective Viscosity of Concentrated Suspensions}}

Mathematocs Department \\ Penn State University \\ University Park \\ PA 16801 \\ USA
\\
{\tt berlyand@math.psu.edu}\\
Yuliya Gorb\\
 Alexei Novikov\end{center}

We present a new approach for calculation of effective properties of high
contrast random composites
and illustrate it by considering highly packed suspensions of rigid
particles in a Newtonian fluid.

The main idea of this approach is a reduction of the original continuum
problem, which is described by PDE with rough coefficients, to a discrete
random network. This reduction is done in two steps which constitute the
"fictitious fluid" approach. In Step 1 we introduce a "fictitious fluid"
continuum
problem when fluid flows only in narrow channels between closely spaced
particles, which reflects physical fact that the dominant contribution to
the dissipation rate comes from these channels. In Step 2 we derive a
discrete network approximation for the latter continuum problem.

Next we use this approach to calculate the effective viscous dissipation
rate in a 2D model of a suspension. We show that that under certain
boundary conditions the model exhibits an anomalously strong rate of blow
up when the concentration of particles tends to maximal. We explore
physical ramification of this phenomenon.


We will also discuss how an iterative procedure of the network
construction which may be used in the study of dynamics of highly packed
suspensions.

The work was done jointly with Y. Gorb and A. Novikov.


\end{document}
