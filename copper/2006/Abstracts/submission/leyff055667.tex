\documentclass{report}
\usepackage{amsmath,amssymb}
\setlength{\parindent}{0mm}
\setlength{\parskip}{1em}
\begin{document}
\begin{center}
\rule{6in}{1pt} \
{\large Sven Leyffer \\
{\bf The Return of the Filter Method }}

9700 South Cass Ave \\ Argonne \\ IL 60439
\\
{\tt leyffer@mcs.anl.gov}\\
Michael Friedlander\end{center}

Filters have been introduced as an alternative to penalty
functions to promote global convergence for nonlinear
optimization algorithms. A filter borrows ideas from multi-
objective optimization and accepts a trial point whenever
the objective or the constraint violation is improved
compared to previous iterates. We present new filter active
set approaches to nonlinear optimization based on a two-
phase methodology. The first finds an estimate of the optimal
active set, and the second phase performs a Newton step on
the corresponding equality constrained problem. The approach
allows inexact subsystem solves, making it suitable for PDE
constrained optimization. Time permitting we present numerical
experience on large structured optimization problems.


\end{document}
