\documentclass{report}
\usepackage{amsmath,amssymb}
\setlength{\parindent}{0mm}
\setlength{\parskip}{1em}
\begin{document}
\begin{center}
\rule{6in}{1pt} \
{\large Rosemary A Renaut \\
{\bf Parameter Decomposition for Iteratively Regularized Gauss Newton Solutions in Optical Tomography}}

Department of Mathematics and Statistics \\ 871804 \\ Arizona State University \\ Tempe \\ AZ 85287 1804
\\
{\tt renaut@asu.edu}\\
Taufiquar Khan\\
Alexandra Smirnova\end{center}


We extend evaluation of the iteratively regularized Gauss Newton method
for the solution of the parameter estimation problem in Optical
Tomography. The general problem of optical tomography requires the
estimation of the underlying model parameters ${\mathbf q}$, for example
the coefficient of diffusion $D$ and the coefficient of absorption
$\mu_a$, (i.e. ${\mathbf q}=(D,\mu_a)^T$) that belong to a parameter set
$Q$. The conditioning of the problem with respect to each parameter set
is different. We investigate the use of an alternating parameter
decomposition approach for solution of the nonlinear inverse problem with
regularization. Contrary to statements on the general nonlinear least
squares problem in standard references eg Bjorck 1996 , we find that
decomposition with respect to the parameter set allows solution of the
regularized problem with the use of appropriately chosen weighting
schemes.


\section{Problem Formulation}
Suppose that ${\mathbf q}$ is obtained as the solution of the inverse problem :
\begin{equation}
-\nabla\cdot D({\mathbf x})\nabla u({\mathbf x};{\mathbf q})
+\mu_a({\mathbf x})u({\mathbf x};{\mathbf q})=s({\mathbf x}),
\label{eq:forwardprob}
\end{equation}
for which data $\mathbf g$ are given on the boundary of domain $\Omega$,
where for ${\mathbf x}$ defined on the domain $\Omega$, $u({\mathbf q})$
is in an appropriate abstract space $H$, and $s$ represents the forcing
function, or source. To find the model parameters ${\mathbf q}$ we seek
to minimize the residuals $F_{ij}=C_{ij}-g_{ij}$ where $C_{ij}$ are
approximations to $g_{ij}$ over all sources and measurements. This is a
typical nonlinear least squares problem
\begin{eqnarray}
{\mathbf q}^* = {\mathrm argmin}{{\mathbf q}}\frac{1}{2}\|F\|_F^2
={\mathrm argmin}{{\mathbf q}}\frac{1}{2}\sum \limits_{j=1}^{n_m}\sum
\limits_{i=1}^{n_s} (C_{ij}(u({\mathbf q}))-g_{ij})^2.
\label{eq:nlscost}
\end{eqnarray}
We consider solution of this nonlinear least squares problem accompanied
with regularization
\begin{equation}
{\mathbf q}^* = {\rm argmin}{{\mathbf q}}\frac{1}{2}\|F\|_F^2 + \tau_1
R_1({\mathbf q}_1) +\tau_2 R_2({\mathbf q}_2 ),
\label{eq:decreg}
\end{equation}
where $R_1$ and $R_2$ could be any regularization operator and need not
be the same in each case, and regularization parameters $\tau_1, \tau_2$
are allowed to be of different scale. This problem was solved in
Babushinsky, Khan and Smirnova ( 2005) using iteratively regularized
Gauss-Newton methods, without separation or investigation of the
conditioning of the parameter components. Our results will show the
impact of parameter separation and improvements in the algorithm through
use of a parameter decomposition approach, motivated by appropriate
theoretical considerations


\end{document}
