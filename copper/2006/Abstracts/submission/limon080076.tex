\documentclass{report}
\usepackage{amsmath,amssymb}
\setlength{\parindent}{0mm}
\setlength{\parskip}{1em}
\begin{document}
\begin{center}
\rule{6in}{1pt} \
{\large Alfonso Limon \\
{\bf Adaptive Mesh Refinement: in the presence of discontinuities }}

School of Mathematical Sciences \\ Claremont Graduate University \\ 710 N College Ave \\ Claremont \\ CA 91711
\\
{\tt alfonso.limon@cgu.edu}\\
Hedley  Morris\end{center}

Classical multiresolution wavelet techniques have been used successfully
to simplify the computation of PDEs by concentrating resources in places
where the solution varies quickly. However, classical techniques tend to
fail near solution discontinuities, as Gibb�s effects contaminate the
wavelet coefficients used to refine the solution. Non-linear adaptive
stencil methods, such as the ENO scheme, can reconstruct the solution
accurately across jumps, but possess neither the compression capabilities
nor the well-understood stability properties of wavelets. Expanding on
Harten�s ideas, we construct an alternative to wavelets, a
multiresolution method that does not suffer from Gibb�s effects and has
good compression properties. We will present this alternative
multiresolution method and compare its performance to other methods by
means of several examples.


\end{document}
