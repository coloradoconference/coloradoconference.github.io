\documentclass{report}
\usepackage{amsmath,amssymb}
\setlength{\parindent}{0mm}
\setlength{\parskip}{1em}
\begin{document}
\begin{center}
\rule{6in}{1pt} \
{\large J. J. Heys \\
{\bf Improving Mass-Conservation of Least-Squares Finite Element Methods}}

Chemical and Materials Engineering \\ Arizona State University \\ Box 876006 \\ Tempe \\ Arizona 85287-6006
\\
{\tt jheys@asu.edu}\\
T. A. Manteuffel\\
S. F. McCormick\\
	Lee, E.\end{center}

Interest in least-squares finite element methods continues to grow due to
at least two of its major strengths. First, the linear systems obtained
after discretization are SPD and often $H^1$ elliptic so that they can be
efficiently solved (often with optimal scalability) using a number of
iterative methods, including conjugate gradients and multigrid. Second,
the least-squares functional provides a sharp measure of the local error
with negligible computational costs. Despite these two major advantages,
the methods have not gained widespread use, largely because they are
perceived as not providing accurate approximations to the true solution,
especially with regards to conservation of mass. The least-squares finite
element method is not discretely conservative, but the approximate
solutions given by the method are the \emph{most accurate approximate
solutions possible in the functional norm for a given finite-dimensional
space}. The method converges to the approximation that minimizes the
functional, so it gives a relatively accurate solution in the functional
norm. However, anyone would agree that an approximate solution to the
Navier-Stokes equations that has an inflow rate that is 100 times the
outflow rate is not an acceptable approximation, even if it is `accurate'
in some norm. Unfortunately, some combinations of common least-square
functionals and finite element spaces for the Navier-Stokes equations
generate solutions that lose 99\% of the mass between the inflow and
outflow boundaries for some particular boundary conditions. Herein lies
the challenge for least-squares methods: how do we formulate a functional
and boundary conditions that better represents the type of accuracy we
desire?

In this talk, two new first-order system reformulations of the
Navier-Stokes equations are presented that admit a wider range of mass
conserving boundary conditions. It is common with least-squares methods
to rewrite the Navier-Stokes equations as a system of first-order
equations using the velocity-vorticity form. The two new first-order
systems are based on the velocity-vorticity form, but they include a new
variable, {$\mathbf{r}$}, representing the pressure gradient plus all or
part of the convective term. As we will demonstrate, the resulting
operator problem can be solved very efficiently using a multigrid or an
algebraic multigrid solver, and excellent mass conservation is observed
for multiple test problems. A difficulty with the new formulations is
obtaining boundary conditions for the new variable, {$\mathbf{r}$}, but
we will demonstrate at least three different methods for overcoming this
difficulty.


\end{document}
