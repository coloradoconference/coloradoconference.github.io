\documentclass{report}
\usepackage{amsmath,amssymb}
\setlength{\parindent}{0mm}
\setlength{\parskip}{1em}
\begin{document}
\begin{center}
\rule{6in}{1pt} \
{\large Misha Kilmer \\
{\bf A Regularized Gauss-Newton Method for Nonlinear Imaging Problems in Diffuse Optical Tomography}}

Mathematics Dept  \\ Tufts Univ \\ 503 Boston Ave \\ Medford \\ MA 02155
\\
{\tt misha.kilmer@tufts.edu}\\
Eric de Sturler\end{center}

The goal in diffuse optical tomography for medical imaging is the joint
reconstruction of parameterized images of absorption and scattering of
light in the body. The reconstruction requires the approximate solution
of a nonlinear least squares problem for the image parameters. While
traditional approaches such as damped Gauss-Newton (GN) and
Levenberg-Marquardt (LM) have been shown to be effective at solving the
imaging problem in its various forms, a considerable number of additional
function and Jacobian evaluations are involved in determining the correct
step length and/or damping parameter. In 3D imaging problems, depending
on the particular image model, the cost of one function evaluation is, at
a minimum, the cost of a dense matrix-vector product and in the worst,
but more realistic, case requires the solution of several large-scale
PDE�s. A Jacobian evaluation is even slightly more expensive. Therefore,
it is crucial to keep the number of function and Jacobian evaluations to
a minimum.

The ill-conditioning of the Jacobian, together with the presence of noise
in the data, motivates us to devise a regularized, trust-region-based
Gauss-Newton approach for determining search directions. Although LM can
be thought of as a regularized analogue to determining the GN direction,
LM has the property of damping possibly important contributions to the
search direction in spectral components corresponding to small singular
values. On the other hand, the Gauss-Newton direction is too influenced
by components due to small singular values early on, causing the line
search to work hard to refine the step length. We propose a method that
systematically evaluates the potential contribution of each of the
spectral components corresponding to the GN-direction and constructs the
new direction relative to this contribution within the confines of a
trust-region. Examples show the success of our method in minimizing
function evaluations with respect to other well-known methods.


\end{document}
