\documentclass{report}
\usepackage{amsmath,amssymb}
\setlength{\parindent}{0mm}
\setlength{\parskip}{1em}
\begin{document}
\begin{center}
\rule{6in}{1pt} \
{\large Marko Huhtanen \\
{\bf SPLITTINGS FOR ITERATIVE SOLUTION OF LINEAR SYSTEMS}}

Institute of Mathematics \\ Helsinki University of Technology \\ Box 1100 \\ FIN-02015 \\ Finland
\\
{\tt marko.huhtanen@tkk.fi}\end{center}

\def\setC{\mathbb{C}}

Consider iteratively solving
a linear system
\begin{equation}\label{eku}
Ax=b,
\end{equation}
with invertible $A\in \setC^{n \times n}$ and $b\in \setC^n$,
by splitting the matrix $A$ as
\begin{equation}\label{toku}
A=L+R,
\end{equation}
where $L$ and $R$ are both readily invertible.
In such a case the recently introduced residual minimizing
Krylov subspace method \cite{HN} can be executed, allowing,
in a certain sense, preconditioning simultaneously with $L$ and $R$.

Splittings satisfying \eqref{toku} result either form
the structure of the problem, or are algebraic.
Splittings of Gauss-Seidel type belong to the latter category.
In this talk we discuss such splittings of $A$.

\medskip
This is joint work with Mikko Byckling.

\begin{thebibliography}{1}
\bibitem{HN} {\sc M. Huhtanen and O. Nevanlinna}, {\em A minimum residual
algorithm for solving linear systems,} submitted manuscript available at
\texttt{www.math.hut.fi/$\sim$mhuhtane/index.html}.
\end{thebibliography}


\end{document}
