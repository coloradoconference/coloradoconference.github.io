\documentclass{report}
\usepackage{amsmath,amssymb}
\setlength{\parindent}{0mm}
\setlength{\parskip}{1em}
\begin{document}
\begin{center}
\rule{6in}{1pt} \
{\large Giovanni Lapenta \\
{\bf A Preconditioned Newton-Krylov Strategy for Moving Mesh Adaptation}}

MS: K717 \\ Los Alamos National Laboratory \\ Los Alamos NM87545
\\
{\tt lapenta@lanl.gov}\\
Luis Chac\'on\end{center}


We propose a new approach to adaptive mesh motion based on solving a
coupled self-consistent system for the physics equations and for the
grid generation equations. The key aspect of the new method is the
solution of the non-linear coupled system with a preconditioned
Newton-Krylov method. The present work described the approach and
focuses on the preconditioning techniques.

Adaptive grids are becoming an ever more common tool for high performance
scientific computing. We focus here on the type of adaptation achieved
by moving a constant number of points according to appropriate rules,
an approach termed moving mesh adaptation (MMA). The approach we consider
here is based specifically on retaining a finite volume approach but
allowing the grid to evolve in time according to a grid evolution
equation obtained from minimization principles.

In the present paper we consider the fundamental question in the application
of MMA. Is it worth the effort? The literature is very rich and considerable
results have been obtained in designing MMA approaches that provide
grids that can indeed present the desired properties. But the question
of whether once the adaptive grids are used the simulations are actually
more cost effective remains largely unanswered.

We have revisited the question and have reached the conclusion that
in order to obtain an effective MMA strategy, three ingredients need
to be considered.

First is the effective formulation of the moving grid equations. In
1D the problem is benign, as error minimization leads to error
equiditribution and to a rigorous and simple minimization procedure.
In 2D and 3D the problem is more challenging but we have derived an
effective approach based on the classic approach by
Brackbill-Saltzmann-Winslow~\cite{brackbill-jcp-82-adgrid}.
The crucial ingredients of our approach are the formulation of the
physics equations in a conservative form and of the formulation of
the grid generation equations using harmonic
mapping~\cite{chacon-lapenta}. The independent
variables of the physics equations are changed from the physical to
the logical space and the equations are rewritten in the logical
space in a fully conservative form~\cite{chacon-lapenta}.

Second is the solution algorithm for the MMA method. Here we bring
a new development. The moving mesh equations and the physics equations,
derived by discretizing the problem under investigation on a moving
grid, form a tightly coupled system of algebraic non-linear equations.
Traditionally, the coupling is broken, the physics and grid equations
being solved separately in a lagged time-splitting approach.
However, in presence of sharp fronts or other moving features, breaking
such coupling can lead to grid lagging with respect of the physics
equations, with adaptation resulting behind rather than on the moving
feature.

We avoid breaking the coupling and solve the full non-linear set of
physics and grid equation using the preconditioned Newton-Krylov (NK)
approach.

Third ingredient in a cost-effective grid adaptation is an efficient
preconditioning technique. In 1D a simple block tridiagonal approach
works effectively~\cite{lapenta-chacon}. Each set of equations, for
physics and for grid generation, is preconditioned with a
tridiagonal matrix obtained by numerically approximating the
corresponding diagonals in the Jacobian. In 2D, we rely on a
multigrid preconditioning strategy where a crucial innovation is how
to coarsen the information relative to the adaptivity in the
harmonic grid generation equations~\cite{chacon-lapenta}..

In the present paper we describe the approach followed and we report
a number of examples to illustrate the performance of the new
approach.


\bibitem{brackbill-jcp-82-adgrid}J.~U. Brackbill, J.~S. Saltzman,
Adaptive zoning for singular problems
in 2 dimensions, J. Comput. Phys. 46~(3) (1982) 342--368.
\bibitem{chacon-lapenta}L.~Chac\'{o}n, G.~Lapenta, A fully implicit, nonlinear adaptive
grid strategy, J. Comput. Phys., 212 (2006) 703--717.
\bibitem{lapenta-chacon}G.~Lapenta, L.~Chac\'{o}n, Cost-e�ectiveness of
fully implicit moving
mesh adaptation: a practical investigation in 1D, J. Comput. Phys.,
submitted.


\end{document}
