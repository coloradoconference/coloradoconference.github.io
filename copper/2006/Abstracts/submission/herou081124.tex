\documentclass{report}
\usepackage{amsmath,amssymb}
\setlength{\parindent}{0mm}
\setlength{\parskip}{1em}
\begin{document}
\begin{center}
\rule{6in}{1pt} \
{\large Michael A Heroux \\
{\bf A Multiscale Perspective on Density Functional Theory for Inhomogeneous Fluids}}

18125 Kreigle Lake Rd \\ Avon \\ MN 56310
\\
{\tt maherou@sandia.gov}\\
Laura J D Frink\\
Andrew G Salinger\end{center}

\newcommand{\dfts}{Fluid-DFTs}
\newcommand{\fdft}{Fluid-DFT}
\newcommand{\fdfts}{Fluid-DFTs}
\newcommand{\dft}{Fluid-DFT}
\newcommand{\Aul}{$A_{11}$}
\newcommand{\Aur}{$A_{12}$}
\newcommand{\All}{$A_{21}$}
\newcommand{\Alr}{$A_{22}$}
\newcommand{\smsc}{Segregated Multi-level Schur Complement Methods}
\newcommand{\SMSC}{SMSC methods}

Some modeling and simulations efforts have historically encompassed
multiscale phenomena without explicitly handling multiscale features. One
such area is density functional theories for inhomogeneous fluids
(\dfts). In this presentation we look at \dfts\ from a fresh perspective,
calling out the fact that \dfts\ incorporate multiple length scales that
are introduced in a way such that each longer scale increases the
fidelity of the model. By viewing \dfts\ from this perspective, we
develop a mathematical framework and a collection of solution algorithms
that have a dramatic impact on the robustness, performance and
scalability of the implicit equations generated by \dfts.

The basic framework for all of our solver algorithms reflects the
importance of inter-physics coupling in the extended variable formulation
of the \dfts. This physics coupling led us to a physics-based block
matrix formulation in order to partition critical and nonlocal ancillary
variables. The idea is to partition the data into blocks that can be
optimally managed or solved. The general $2\times 2$ block matrix is

$$\label{equ:BlockEquation}
\left(
\begin{array}{ccc|ccc}
A_{11}^{11} & \cdots & A_{11}^{1j} & A_{12}^{1,j+1} & \cdots & A_{12}^{1k} \\
\vdots & \ddots & \vdots & \vdots & \ddots & \vdots \\
A_{11}^{j1} & \cdots & A_{11}^{jj} & A_{11}^{j1} & \cdots & A_{11}^{jj} \\\hline

A_{21}^{j+1,1} & \cdots & A_{21}^{j+1,j} & A_{22}^{j+1,j+1} & \cdots & A_{22}^{j+1,k} \\
\vdots & \ddots & \vdots & \vdots & \ddots & \vdots \\
A_{21}^{k1} & \cdots & A_{21}^{kj} & A_{22}^{k,j+1} & \cdots & A_{22}^{kk} \\
\end{array}
\right)
\left(
\begin{array}{c}
x_1^1 \\
\vdots \\
x_1^j \\\hline
x_2^{j+1} \\
\vdots \\
x_2^k \\
\end{array}
\right)
=
\left(
\begin{array}{c}
b_1^1 \\
\vdots \\
b_1^j \\\hline
b_2^{j+1} \\
\vdots \\
b_2^k \\
\end{array}
\right)
$$

where $k$ is the number of DOFs tracked per node. The superscript
$(p, q)$ denotes the block of coefficients generated by DOF $p$
interactions with DOF $q$. The subscripts and partition lines
impose a coarser partitioning of the matrix into a 2-by-2 block
system that will be used with a Schur complement approach. We
denote by $A_{11}$, $A_{12}$, $A_{21}$ and $A_{22}$ the upper left,
upper right, lower left and lower right submatrix of the coarse
2-by-2 block matrix, respectively. Similarly $x_1$ and $x_2$, and
$b_1$ and $b_2$ are the upper and lower parts of $x$ and $b$,
respectively.

Given this two-level structure, the basic strategy for solving each
global linear system generated by Newton's method is as follows:
\begin{enumerate}
\item Identify and reorder DOFs $1$ through $j$ such that
$A_{11}^{-1}$ (the inverse of $A_{11}$) is easy to apply (in
parallel).
\item Determine a preconditioner $P$ for $S = A_{22} -
A_{21}A_{11}^{-1}A_{12}$, the Schur complement of $A$ with respect
to $A_{22}$.
\item Solve $Sx_2 = (b_2 - A_{21}b_1)$ using a preconditioned
Krylov method such as GMRES, with preconditioner $P$. Note that
$S$ may or may not be explicitly formed, depending on other problem details.
\item Finally, solve
for $x_1 = A_{11}^{-1}(b_1 - A_{12}x_2)$.
\end{enumerate}

Given this basic framework, we will describe specific solvers for
special categories of \dft\ problems, including 2 and 3 dimensional
hard-sphere problems and polymer chains. We give results for
several problem areas including nanopore and lipid bi-layer models
where this Schur complement approach provides one to two orders of
magnitude improvement in performance and an order of magnitude
reduction in memory requirements.


\end{document}
