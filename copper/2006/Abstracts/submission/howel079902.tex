\documentclass{report}
\usepackage{amsmath,amssymb}
\setlength{\parindent}{0mm}
\setlength{\parskip}{1em}
\begin{document}
\begin{center}
\rule{6in}{1pt} \
{\large Jason S. Howell \\
{\bf Implementation and performance of a two-grid method for nonlinear reaction-diffusion equations}}

Department of Mathematical Sciences \\ O-106 Martin Hall \\ Box 340975 \\ Clemson \\ SC 29634-0975
\\
{\tt jshowel@clemson.edu}\\
Carol S. Woodward\\
Todd S. Coffey\end{center}

Two-grid methods have been developed by Xu (SIAM J. Num. Anal, 1996) for
application to linear and nonlinear PDEs. Of particular interest are
methods that can be applied to large nonlinear problems that arise in the
simulation of physical processes, such as a method due to Dawson,
Wheeler, and Woodward (SIAM J. Num. Anal, 1998). This scheme solves the
original nonlinear problem on a mesh coarser than originally specified to
capture the nonlinear behavior of the solution, then utilizes a
linearized version of the problem to correct the coarse approximation on
the original problem mesh. The two-grid method potentially reduces the
overall computational cost by requiring the solution of a smaller
nonlinear system and a large linear system in place of the original large
nonlinear problem. In this talk we investigate the application of this
method to nonlinear reaction-diffusion equations. In particular, we
discuss issues that arise in the implementation of the algorithm, and
perform numerical experiments on problems designed to gauge the
performance of the two-grid method relative to a standard Newton
iterative nonlinear solver.


\end{document}
