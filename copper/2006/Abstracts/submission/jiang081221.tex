\documentclass{report}
\usepackage{amsmath,amssymb}
\setlength{\parindent}{0mm}
\setlength{\parskip}{1em}
\begin{document}
\begin{center}
\rule{6in}{1pt} \
{\large Lianjun Jiang \\
{\bf A preconditioned L-BFGS algorithm with application to protein structure prediction}}

1475 Folsom st \\ apt 171 \\ Boulder \\ CO 80302
\\
{\tt lianjunj@gmail.com}\\
Richard Byrd\\
Elizabeth Eskow\\
	Robert B. Schnabel\end{center}

The limited-memory BFGS method has been widely used in large scale
unconstrained optimization problems, including protein
structure prediction. A major weakness of the L-BFGS method is that it
may converge very slowly for ill-conditioned problems. We propose a
preconditioned L-BFGS method, where we form the preconditioner from parts
of the partially separable objective function. We report results of
experiments in the context of the protein structure prediction problem
for four different proteins, using a protein energy model as the
objective function and multiple initial configurations for each protein.
The results show speed-ups with factors between 3 and 10 in terms of
function evaluations and with factors between 2 and 7 in terms of CPU
time. The difference between CPU time and function evaluation speed-up is
due to the extra overhead of calculating and applying the preconditioner.
We also compare our results to a method from the other competitive class
of large-scale methods, preconditioned truncated Newton method (TNPACK).
The limited results indicate that the preconditioned L-BFGS method may be
more efficient.

We also tried this approach in a more general context, using limited
memory with an incomplete Cholesky factorization of the Hessian as a
preconditioner. We compared the performance with the truncated Newton
algorithm TRON, which uses a similar preconditioner. Tested on a subset
of the CUTER test problems, our results show that, using this
preconditioner, the preconditioned LBFGS method is competitve with TRON
and much better than the LBFGS algorithm without preconditioner.


\end{document}
