\documentclass{report}
\usepackage{amsmath,amssymb}
\setlength{\parindent}{0mm}
\setlength{\parskip}{1em}
\begin{document}
\begin{center}
\rule{6in}{1pt} \
{\large Gavino Pautasso \\
{\bf Performance of the Predictor-Corrector Preconditioners for Newton-Krylov Solvers }}

MS: K717 \\ Los Alamos National Laboratory \\ Los Alamos \\ NM 87545
\\
{\tt valsusa@gmail.com}\\
Giovanni Lapenta\\
Jianwei Ju\end{center}

We investigate an alternative implementation of preconditioning techniques for the
solution of non-linear problems. Within the framework of Newton-Krylov methods,
preconditioning techniques are needed to improve the performance of the solvers.
We use a different implementation approach to re-utilize existing semi-implicit
methods to precondition fully implicit non-linear schemes. We use a predictor-
corrector approach where the fully non-linear scheme is the corrector and the pre-
existing semi-implicit scheme is the predictor [1,2]. The advantage of
the proposed approach is that it allows to retrofit existing codes, with
only minor modifications, in particular avoiding the need to reformulate
existing methods in terms of variations,
as required instead by other approaches now currently used.

A classic problem of computational science and engineering is the
search for an efficient numerical scheme for solving non-linear
time-dependent partial differential equations. Explicit and
semi-implicit methods can provide simple solution techniques but are
seriously limited by time step limitations for stability (explicit
methods) and accuracy (semi-implicit methods).

Recently, significant progress has been made in the development of
fully implicit approaches for solving nonlinear problems: the
Newton-Krylov (NK) method. The method is developed
from the Newton iterative method, by applying a linear iterative
solver to the Jacobian equation for the Newton step and terminating
that iteration when a suitable convergence criterion
holds.

For the solution of the linear Jacobian equation, Krylov methods are
often the choice, leading to the Newton-Krylov (NK) approach.
However, for most cases, Krylov solvers can be extremely
inefficient. The need for good preconditioners techniques becomes a
constraining factor in the development of NK solvers.

In a number of fields, recent work based on multi-grid and
physics-based preconditioners[3] have demonstrated
extremely competitive performances.

In the present study, we discuss a different implementation of
preconditioning: the predictor-corrector (PC) preconditioner [1,2]. The
approach has two novelties. First, it preconditions directly the
non-linear equations rather than the linear Jacobian equation for
the Newton step. The idea is not new~\cite{kelley}, but it is
implemented here in a new way that leads to great simplifications of
the implementation. We note that this simplification is designed
also to minimize the effort in refitting existing semi-implicit
codes into full fledged implicit codes, representing perhaps a
greater advance in software engineering than in computational
science. Second, we test new ways of preconditioning the equations
by using a combination of predictor-corrector semi-implicit
preconditioning.

The fundamental idea is to use a predictor to advance a
semi-implicit discretization of the governing equations and use a
corrector Newton step to correct for the initial state of the
predictor step. The typical NK solver is used to compute the unknown
value of the state vector at the end of the time step ${\bf x}^{1}$
from its known value at the previous time step ${\bf x}^0$. Instead,
we use the Newton method to iterate for a modification of the actual
known state $ {\bf x}^{*}$ from the previous time step to find a
modified {\it "previous"} state that makes the semi-implicit
predictor step give the solution of the fully implicit method.

Two advantages are obvious. First, the actual previous state ${\bf
x}^0$ is likely to be a better first guess for the modified initial
state $ {\bf x}^{*}$ of the predictor than it is for the final
state of the corrector step. Second, by modifying the non-linear
function and consequently modifying the Jacobian equation, the PC
preconditioner modifies the spectral properties of the Jacobian
matrix in the same way as preconditioners applied directly to the
Jacobian equation. Indeed, as shown below the PC preconditioner
gives the same type of speed-up of the Krylov convergence without
requiring to formulate an actual preconditioning of the Krylov
solver.

We use a suite of problems, including non-lnear diffusion [2] and the
standard driven cavity flow problem [1], as benchmarks to demonstrate the
preformance and the reliability of the PC preconditioning method.

[1] J. Ju, G. Lapenta, Predictor-Corrector Preconditioned Newton-Krylov
Method For Cavity Flow, Lecture Notes in Computer Science, 3514, 82,
2005.

[2] G. Lapenta, J. Ju, Predictor-Corrector Preconditioners for
Newton-Krylov Solvers, J. Compuatat. Phys., submitted.

[3] D. A. Knoll, D. Keyes, Jacobian-free Newton-Krylov methods: a survey of
approaches and applications, J. Comput. Phys. 193 (2004) 357�397.


\end{document}
