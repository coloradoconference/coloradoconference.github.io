\documentclass{report}
\usepackage{amsmath,amssymb}
\setlength{\parindent}{0mm}
\setlength{\parskip}{1em}
\begin{document}
\begin{center}
\rule{6in}{1pt} \
{\large Bobby Philip \\
{\bf Resistive Magnetohydrodynamics with Implicit Adaptive Mesh Refinement}}

MS B256 \\ Computer and Computational Sciences Division \\ Los Alamos National Laboratory \\ PO Box 1663 \\ Los Alamos \\ NM 87545
\\
{\tt bphilip@lanl.gov}\\
Michael Pernice\\
Luis Chacon\end{center}

Implicit adaptive mesh refinement (AMR) is used to simulate a model
resistive magnetohydrodynamics problem. This challenging multi-scale,
multi-physics problem involves a wide range of length and time scales.
AMR is employed to resolve extremely thin current sheets, essential for
an accurate macroscopic description. Implicit time integration is used to
step over fast Alfven time scales. At each time step, large-scale systems
of nonlinear equations are solved using Jacobian-free Newton-Krylov
methods together with a physics-based preconditioner. The preconditioner
is implemented using optimal multilevel solvers such as the Fast Adaptive
Composite grid (FAC) method. We will describe our initial results
highlighting various aspects of problem formulation, optimal
preconditioning on AMR grids, and efforts towards achieving parallelism.

This work was performed under the auspices of the U.S. Department of
Energy by Los Alamos National Laboratory under contract W-7405-ENG-36.
Los Alamos National Laboratory does not endorse the viewpoint of a
publication or guarantee its technical correctness. LA-UR 04-7312,
04-9024, 05-2645


\end{document}
