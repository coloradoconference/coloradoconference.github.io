\documentclass{report}
\usepackage{amsmath,amssymb}
\setlength{\parindent}{0mm}
\setlength{\parskip}{1em}
\begin{document}
\begin{center}
\rule{6in}{1pt} \

\title{
Iterative solution techniques for flexible approximation schemes in
multiparticle simulations\thanks{The work was supported in part
by the U.S.  Department of Energy under Contract W-7405-ENG-82, 
in part by NERSC, in part by NSF under grants 0305120, 0304453, 9812895
and 9702364 and in part by the University of Minnesota Duluth.}}

{\large Masha Sosonkina \\
{\bf Iterative solution techniques for flexible approximation schemes in
multiparticle simulations}}

Ames Laboratory DOE \\ Ames IA 50011 \\
{\tt  masha@scl.ameslab.gov} \\
Igor Tsukerman,
Elena Ivanova,
Sergey Voskoboynikov
\end{center}


The paper examines various parallel iterative solvers for the new Flexible
Local Approximation MEthod (FLAME) \cite{Tsukerman05, Tsukerman06}
applied to colloidal systems. The electrostatic potential in such  systems
can be described, at least for monovalent salts in the solvent, by the
Poisson-Boltzmann equation (see e.g. \cite{Grosberg02}). Classical
Finite-Difference (FD) schemes would require unreasonably fine meshes to
represent the boundaries of multiple spherical particles at arbitrary
locations with sufficient accuracy. In the Finite Element Method, mesh
generation for a large number of particles becomes impractical. The Fast
Multipole Method works well only if the particle sizes are neglected and 
the Poisson-Boltzmann equation is linearized \cite{Greengard02}.

Classical FD schemes rely on Taylor expansions that break down near 
material interfaces (such as particle boundaries) due to the lack of 
smoothness  of the field. In FLAME, Taylor expansions in the vicinity 
of the particles  are
replaced with much more accurate approximations. Namely, the local FLAME
bases are constructed by matching (via the boundary conditions) the
spherical harmonics for the electrostatic potential inside and outside 
the particle; see \cite{Tsukerman05, Tsukerman06} for details.

The system matrices of FLAME and classical FD have the same sparsity
structure for the same grid stencil on a regular Cartesian grid; for
example, the standard seven-point stencil leads to a seven-diagonal
matrix.  However, the FLAME matrix is generally nonsymmetric.
Several parallel iterative solution techniques have been tested with an
emphasis on suitable parallel preconditioning for the nonsymmetric 
system matrix. In particular, flexible GMRES \cite{saad03iterative}
preconditioned with the distributed Schur Complement
\cite{saad-sosonkina-pschur} has been considered and compared with
Additive Schwarz and global incomplete ILU(0) preconditionings.
It has been observed that Schur Complement preconditioning with a small 
amount of fill and a few inner iterations scales well and exhibits good 
solution times while attaining
almost linear speedup. The number of iterations and the computational time
depends only mildly on the Debye parameter of the electrolyte.


\begin{thebibliography}{1}

\bibitem{Greengard02}
L.~F. Greengard and J.~Huang.
\newblock A new version of the {Fast Multipole Method} for screened {Coulomb}
  interactions in three dimensions.
\newblock {\em J. Comput. Phys.}, 180(23):642--658, 2002.

\bibitem{Grosberg02}
A.~Yu. Grosberg, T.~T. Nguyen, and B.~I. Shklovskii.
\newblock Colloquium: The physics of charge inversion in chemical and
  biological systems.
\newblock {\em Reviews of Modern Physics}, 74(2):329--345, 2002.

\bibitem{saad03iterative}
Y.~Saad.
\newblock {\em Iterative Methods for Sparse Linear Systems, 2nd edition}.
\newblock SIAM, Philadelpha, PA, 2003.

\bibitem{saad-sosonkina-pschur}
Y.~Saad and M.~Sosonkina.
\newblock Distributed {Schur Complement} techniques for general sparse linear
  systems.
\newblock {\em SIAM J. Scientific Computing}, 21(4):1337--1356, 1999.

\bibitem{Tsukerman05}
I.~Tsukerman.
\newblock Electromagnetic applications of a new finite-difference calculus.
\newblock {\em IEEE Trans. Magn.}, 41(7):2206--2225, 2005.

\bibitem{Tsukerman06}
I.~Tsukerman.
\newblock A class of difference schemes with flexible local approximation.
\newblock {\em J. Comput. Phys.}, 211(2):659--699, 2006.

\end{thebibliography}


\end{document}
