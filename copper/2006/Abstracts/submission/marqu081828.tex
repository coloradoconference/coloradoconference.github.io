\documentclass{report}
\usepackage{amsmath,amssymb}
\setlength{\parindent}{0mm}
\setlength{\parskip}{1em}
\begin{document}
\begin{center}
\rule{6in}{1pt} \
{\large Osni Marques \\
{\bf A Comparison of Eigensolvers for Large Electronic Structure Calculations}}

Lawrence Berkeley National Laboratory \\ 1 Cyclotron Road \\ MS 50F-1650 \\ Berkeley \\ CA 94720-8139 \\ USA
\\
{\tt oamarques@lbl.gov}\\
	Andrew Canning
	Julien Langou
	Stanimire Tomov
	Christof Voemel
	Lin-Wang Wang
	\end{center}

The solution of the single particle Schroedinger equation that arises in
electronic structure calculations often requires solving for interior
eigenstates of a large Hamiltonian. The states at the top of the valence
band and at the bottom of the conduction band determine the band gap
that relates to important physical characteristics such as optical or
transport properties.

In order to avoid the explicit computation of all eigenstates, a folded
spectrum method has been usually employed to compute only the eigenstates
near the band gap. In this talk, we compare the conjugate gradient
minimization and the optimal block preconditioned conjugate gradient
(LOBPCG) applied to the folded spectrum matrix with the Jacobi-Davidson
algorithm.


\end{document}
