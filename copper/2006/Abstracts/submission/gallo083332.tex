\documentclass{report}
\usepackage{amsmath,amssymb}
\setlength{\parindent}{0mm}
\setlength{\parskip}{1em}
\begin{document}
\begin{center}
\rule{6in}{1pt} \
{\large Efstratios Gallopoulos \\
{\bf Can Information Retrieval aid Iterative Methods? }}

Computer Engineering and Informatics \\ University of Patras \\ 26500 Patras \\ Greece
\\
{\tt stratis@ceid.upatras.gr}\\
Spyros Hatzimihail\end{center}

Recently, Computational Linear Algebra techniques have started playing an
important role in Information Retrieval (IR) research. Indeed, the Vector
Space Model and its derivatives, such as Latent Semantic Indexing, are
heavily used and investigated. In this presentation we consider the
problem in ``reverse mode'', namely using IR techniques to help in linear
algebra and iterative
methods in particular. The candidate problem is the solution of large
linear systems with multiple right hand sides. The
efficiency of solvers is known to depend on the amount of information
shared amongst the right hand sides.
We specifically investigate the combination of clustering algorithms and
schemes from the existing literature to solve such problems.

This research is supported in part by a University of Patras KARATHEODORI grant.


\end{document}
