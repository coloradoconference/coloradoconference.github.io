\documentclass{report}
\usepackage{amsmath,amssymb}
\setlength{\parindent}{0mm}
\setlength{\parskip}{1em}
\begin{document}
\begin{center}
\rule{6in}{1pt} \
{\large John Baird \\
{\bf The Representer Method for Data Assimilation of Two-Phase Flow in Porous Media}}

Institute for Computational Engineering and Sciences \\ The University of Texas at Austin \\ 1 Texas Longhorns \\ #C0200 \\ Austin \\ TX 78712
\\
{\tt jbaird@ices.utexas.edu}\\
Clint Dawson\end{center}

Advances in instrumenting and imaging the subsurface are yielding large
data sets, making data assimilation vital to modeling flow through porous
media. We derive and implement the representer method applied to the
oil/water model for reservoirs, a nonlinear model. The representer
method, like the Kalman filter, solves the Euler-Lagrange (E-L) system
for the minimizer of a least-squares functional of the misfit between the
model and measurements. Because the representer method uses the
superposition principle, a nonlinear model requires linearization of the
E-L system. A key concern is finding a linearization that converges
appropriately. We show that convergence is strongly affected by the
choice of weights in the least-squares functional. We also compare the
effects of linearization and the computational costs of the representer
method with the ensemble Kalman filter.


\end{document}
