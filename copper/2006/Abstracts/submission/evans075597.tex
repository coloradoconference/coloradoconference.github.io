\documentclass{report}
\usepackage{amsmath,amssymb}
\setlength{\parindent}{0mm}
\setlength{\parskip}{1em}
\begin{document}
\begin{center}
\rule{6in}{1pt} \
{\large Katherine J. Evans \\
{\bf Development of a 2-D model to simulate convection and phase transition efficiently}}

MS B216 \\ PO Box 1663 \\ Los Alamos National Laboratory \\ Los Alamos \\ NM 87545
\\
{\tt kevans@lanl.gov}\\
D. A. Knoll\\
Michael Pernice \end{center}

We present a two-dimensional convection phase change model using the
incompressible Navier-Stokes equation set and enthalpy as the energy
conservation variable. Significant algorithmic challenges are posed for
problems of phase change interfaces within convective flow regimes. The
equation set is solved with the Jacobian-Free Newton-Krylov (JFNK)
nonlinear inexact Newton's method. SIMPLE, a pressure-correction
algorithm, is used as a physics-based preconditioner. This algorithm is
compared to solutions using SIMPLE as the main solver.

Algorithm performance is assessed for a benchmark problem, phase change
convection within a square cavity of a solid pure material cooled below
the melting temperature. A time step convergence analysis demonstrates
that the JFNK model with second order discretization is second order
accurate in time. A Gallium melting simulation is also performed and
evaluated; in this configuration multiple roll cells develop in the
melted region at early times when the aspect ratio is high. The
JFNK-SIMPLE method is shown to be more efficient per time step and more
robust at larger
time steps when compared to SIMPLE as the main solution algorithm.
Overall CPU savings of more than an order of magnitude are realized.

As a further analysis of JFNK-SIMPLE, multigrid is wrapped around SIMPLE,
so SIMPLE acts as the smoother within a multigrid preconditioner to JFNK.
For phase change conduction, additional gains in efficiency can be
accomplished. When SIMPLE is incorporated as a preconditioner and
smoother within JFNK, the ability to model more
complex and realistic phase change convection problems with increased
robustness and efficiency is achieved.


\end{document}
