\documentclass{report}
\usepackage{amsmath,amssymb}
\setlength{\parindent}{0mm}
\setlength{\parskip}{1em}
\begin{document}
\begin{center}
\rule{6in}{1pt} \
{\large Peter, R. Graves-Morris \\
{\bf BiCGStab, VPAStab and an adaptation to mildly nonlinear systems.}}

Computing Department \\ University of Bradford \\ Bradford \\ West Yorks BD7 1DP \\ England
\\
{\tt p.r.graves-morris@bradford.ac.uk}\end{center}


BiCGStab [Van der Vorst, 1992] and GMRes [Saad and Schultz, 1986] are the
names of two famous families of iterative methods
for the solution of large linear systems. Subsequently, much research has focussed on
appropriate generalizations for large nonlinear systems [Kelley, 1995].
Here, we will discuss a
nonlinear generalization of BiCGStab, and see its performance on the $\phi$-2 equation
and Bratu equations in 2D, and for Burghers' equation.

The key equations of BiCGStab will be summarized, and then its connections with
vector Pad\'{e} approximation of the Richardson series will be briefly reviewed.
These considerations lead naturally to the algorithm called VPAStab for
stabilised vector-Pad\'{e} approximation of a vector-valued function
whose coefficients are linearly generated [Graves-Morris, 2003].
VPAStab, when applied to systems of linear equations, is very similar to BiCGStab.
A generalization of the algorithm for the acceleration of convergence of a
nonlinearly generated system of equations is proposed here using recursions of the form
\begin{eqnarray}
\mathbf{x}^{(2k)} & := &
(1+\alpha_{k})\mathbf{x}^{(2k-1)}-\alpha_{k}\mathbf{x}^{(2k-2)} \nonumber
\\
& & \ \ \
+(1-\theta_{k})[(1+\alpha_{k})\mathbf{r}^{(2k-1)}-\alpha_{k}\mathbf{r}^{(2k-2)}],
\nonumber \\
\mathbf{x}^{(2k+1)} & := & (1+\beta_{k})\mathbf{x}^{(2k)} -
\beta_{k}\mathbf{x}^{(2k-1)} + (1+\beta_{k})\mathbf{r}^{(2k)} \nonumber
\\
& & \ \ \ - \beta_{k}(1-\theta_{k})\mathbf{r}^{(2k-1)}. \nonumber
\end{eqnarray}
In these formulas, $\mathbf{x}^{(i)}$ are accelerated estimates of the
solution of the equations,
and $\mathbf{r}^{(i)}$ are their (preconditioned) residuals. The
initialization of the recursions
and the other parameters $\alpha_{k}$, $\beta_{k}$ and $\theta_{k}$
are common to those of VPAStab. Some encouraging comparative numerical
results will be shown.


\end{document}
