\documentclass{report}
\usepackage{amsmath,amssymb}
\setlength{\parindent}{0mm}
\setlength{\parskip}{1em}
\begin{document}
\begin{center}
\rule{6in}{1pt} \
{\large Michael Overton \\
{\bf Nonsmooth, Nonconvex Optimization: Theory, Algorithms and Applications}}

Courant Institute \\ 251 Mercer St \\ New York \\ NY 10012
\\
{\tt overton@cs.nyu.edu}\end{center}

Theory: there are two standard approaches to generalizing derivatives
to nonsmooth, nonconvex optimization: the Clarke subdifferential (or
generalized gradient), and the MIRW subdifferential (or subgradient sets),
as expounded in Rockafellar and Wets (Springer, 1998). We briefly discuss
these and mention their advantages and disadvantages. They coincide for
an important class of functions: those that are locally Lipschitz and regular,
which includes continuously differentiable functions and convex functions.

Algorithms: the usual approach is bundle methods, which are complicated.
We describe some alternatives: BFGS (a new look at an old method), and
Gradient Sampling (a simply stated method that, although computationally
intensive, has solved some previously unsolved problems and has a nice
convergence theory).

Applications: these abound in control, but surely in other areas too.
Of particular interest to me are applications involving eigenvalues
and singular values of nonsymmetric matrices. Sometimes even
easily stated problems in a few variables are hard. Our new code
HIFOO (H-Infinity Fixed-Order Optimization) is intended for use by
practicing control engineers and has solved some open problems in control.

This is all joint work with James Burke and Adrian Lewis.
HIFOO is also joint with Didier Henrion.


\end{document}
