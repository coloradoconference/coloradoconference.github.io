\documentclass{report}
\usepackage{amsmath,amssymb}
\setlength{\parindent}{0mm}
\setlength{\parskip}{1em}
\begin{document}
\begin{center}
\rule{6in}{1pt} \
{\large Chao Yang \\
{\bf A Constrained Minimization Algorithm for Solving Nonlinear Eigenvalue Problems in Electronic Structure Calculation}}

Lawerence Berkeley National Laboratory \\ 1 Cyclotron Rd \\ MS-50F \\ Berkeley \\ CA 94720
\\
{\tt cyang@lbl.gov}\\
Juan Meza\\
Ling-wang Wang\end{center}

One of the fundamental problems in electronic structure calculation
is to determine electron orbitals associated with the minimum total
energy of large atomistic systems. The total energy minimization problem
is often formulated as a nonlinear eigenvalue problem and
solved by an iterative scheme called Self Consistent Field (SCF)
iteration. In this talk, a new direct constrained optimization
algorithm for minimizing the Kohn-Sham (KS) total energy functional
is presented. The key ingredients of this algorithm involve projecting
the total energy functional into a sequences of subspaces of small
dimensions and seeking the minimizer of total energy functional within
each subspace. The minimizer of the projected energy functional not only
provides a search direction along which the KS total energy functional
decreases but also gives an optimal ``step-length" to move along this
search direction. Due to the small dimension of the projected problem,
the minimizer of the projected energy functional can be computed by
several different methods. These methwill be examined and compared in
this talk. Numerical examples will be provided to demonstrate that this
new direct constrained optimization algorithm can be more efficient and
robust than the SCF iteration.



\end{document}
