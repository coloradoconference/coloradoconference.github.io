\documentclass{report}
\usepackage{amsmath,amssymb}
\setlength{\parindent}{0mm}
\setlength{\parskip}{1em}
\begin{document}
\begin{center}
\rule{6in}{1pt} \
{\large Joost Rommes \\
{\bf Computing dominant poles of transfer functions}}

Joost Rommes \\ Mathematical Institute \\ Utrecht University \\ P O Box 80 010 \\ 3508 TA Utrecht \\ The Netherlands
\\
{\tt rommes@math.uu.nl}\\
Nelson Martins\end{center}

\newcommand{\vx}{\mathbf{x}}
\newcommand{\vv}{\mathbf{v}}
\newcommand{\vtx}{\tilde{\mathbf{x}}}
\newcommand{\vtv}{\tilde{\mathbf{v}}}
\newcommand{\vb}{\mathbf{b}}
\newcommand{\vc}{\mathbf{c}}
\newcommand{\mR}{\mathbb{R}}
\newcommand{\mRn}{\mathbb{R}^n}
\newcommand{\mRnn}{\mathbb{R}^{n\times n}}
\newcommand{\mC}{\mathbb{C}}

Recent work on power system stability, controller design and
electromagnetic transients has used several advanced
model reduction techniques. Although these
techniques, such as balanced truncation, produce good results, they impose high
computational costs and hence are only applicable to moderately sized
systems. Modal model reduction is a cost-effective alternative for
large-scale systems, when only a fraction of the system pole spectrum is
controllable-observable for the transfer function of interest. Modal reduction
produces transfer function modal equivalents from the knowledge of the dominant
poles and their corresponding residues. In this talk a specialized eigenvalue
method will be presented that computes the most dominant poles and corresponding
residues of a SISO transfer function.

The transfer function of a single input single output (SISO) system is defined as
\begin{equation}\label{eq:transfer}
H(s) = \vc^T (sI - A)^{-1}\vb + d,
\end{equation}
where $A\in\mRnn$, $\vb,\vc\in\mRn$, $d\in\mR$ and $I\in\mRnn$ is the
identity matrix and $s\in\mC$. Without loss of generality, $d=0$ in the
following.

Let the eigenvalues (poles) of $A$ and the corresponding right and left eigenvectors be
given by the triplets $(\lambda_j,\vx_j,\vv_j)$, and let the right and left
eigenvectors be scaled so that $\vv_j^*\vx_j=1$. It is assumed that $\vv_j^*\vx_k=0$ for
$j\neq k$. The transfer function
$H(s)$ (\ref{eq:transfer}) can be expressed as a sum of residues $R_j$
over first order poles:
\begin{equation}
H(s) = \sum_{j=1}^n \frac{R_j}{s - \lambda_j},\nonumber
\end{equation}
where the residues $R_j$ are
\begin{equation}
R_j = (\vx_j^T\vc)(\vv_j^*\vb).\nonumber
\end{equation}

A \textit{dominant} pole is a pole $\lambda_j$ that corresponds to a
residue $R_j$ with large
magnitude $|R_j| / |\mbox{Re}(\lambda_j)|$ , i.e.~a pole that is well observable and
controllable in the transfer function. This can also be observed from the
corresponding Bode magnitude plot of $H(s)$, where peaks occur at
frequencies close to the
imaginary parts of the dominant poles of $H(s)$. An approximation of $H(s)$ that
consists of $k<n$ terms with $|R_j|/|\mbox{Re}(\lambda_j)|$ above some
value, determines the effective
transfer function behavior and is called the transfer function modal equivalent:
\begin{equation}
H_k(s) = \sum_{j=1}^k \frac{R_j}{s - \lambda_j},\nonumber
\end{equation}
The problem of concern can now be formulated as:
\begin{quote}
Given a SISO linear, time
invariant, dynamical system $(A,\vb,\vc,d)$, compute $k\ll n$ dominant poles
$\lambda_j$ and the corresponding right and left
eigenvectors $\vx_j$ and $\vv_j$.
\end{quote}

The algorithm to be presented, called Subspace
Accelerated Dominant Pole Algorithm (SADPA)\footnote{J.~Rommes and N.~Martins,
\textit{Efficient computation of transfer function dominant poles using subspace
acceleration}, 2005, UU Preprint 1340}, combines a
Newton algorithm\footnote{N.~Martins, L.T.G.~Lima and H.J.C.P.~Pinto,
\textit{Computing dominant poles of power system transfer functions},IEEE
Trans.~Power Syst.",
vol.~11, nr.~1, pp 162--170, 1996} with subspace acceleration, a clever
selection strategy and deflation to efficiently compute the dominant poles and
corresponding residues. It can easily be extended to handle MIMO systems as
well\footnote{J.~Rommes and N.~Martins,
\textit{Efficient computation of multivariable transfer function dominant
poles using subspace
acceleration}, 2006, UU Preprint 1344}. The performance of the algorithm
will be illustrated by
numerical examples of large scale power systems.


\end{document}
