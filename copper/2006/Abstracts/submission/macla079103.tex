\documentclass{report}
\usepackage{amsmath,amssymb}
\setlength{\parindent}{0mm}
\setlength{\parskip}{1em}
\begin{document}
\begin{center}
\rule{6in}{1pt} \
{\large Scott MacLachlan \\
{\bf A greedy strategy for coarse-grid selection}}

Department of Computer Science and Engineering \\ University of Minnesota \\ 4-192 EE/CS Building \\ 200 Union Street SE \\ Minneapolis \\ MN 55455
\\
{\tt maclach@cs.umn.edu}\\
Yousef Saad\end{center}

In recent years, substantial effort has been focused on developing
methods capable of solving the large linear systems that arise from the
discretization of partial differential equations. This research has been
driven by application scientists' demands of higher accuracy, in both
mathematical modelling and computational solution. Efficient solution of
many of these linear systems is possible only through the use of
multilevel solution techniques. While highly optimized algorithms may be
developed using knowledge about the origins of the matrix problem to be
considered, much current interest is in the development of purely
algebraic approaches that may be applied in many situations, without
extensive problem-specific tuning.

In this talk, we present a new algebraic approach to finding the
fine/coarse partitions needed in multilevel approaches. The algorithm is
motivated by theoretical analysis of the performance of algebraic
multigrid (AMG) and algebraic recursive multilevel solvers (ARMS). From
the AMG point of view, the coarsening is consistent with the ideas of
compatible relaxation, while it may also be motivated by the algebraic
criteria central to ARMS. While no guarantee on the rate of coarsening is
given, the splitting is shown to always yield an effective preconditioner
in the two-level sense. Numerical performance of two-level and multilevel
variants of this approach is demonstrated.


\end{document}
