\documentclass{report}
\usepackage{amsmath,amssymb}
\setlength{\parindent}{0mm}
\setlength{\parskip}{1em}
\begin{document}
\begin{center}
\rule{6in}{1pt} \
{\large Vadim Dyadechko \\
{\bf Moment-of-fluid interface reconstruction}}

MS B284 \\ Los Alamos National Laboratory \\ Los Alamos \\ NM 87545
\\
{\tt vdyadechko@lanl.gov}\\
Mikhail Shashkov\end{center}

\newcommand{\R}{\mathbb R}
\renewcommand{\tau}{\Delta t}
\newcommand{\centroid}[1]{\x_c(#1)}
\renewcommand{\vec}[1]{{\bf#1}}
\newcommand{\x}{\vec{x}}
Volume-of-fluid~(VoF) methods
[2]
are widely used in
Eulerian simulations
of multi-phase flows
with mutable
interface topology.
The popularity of VoF methods
is explained
by their unique ability to preserves
the mass of each fluid component
on the discrete level.
The strategy of VoF methods consists in
calculating the interface location
at each discrete moment of time
from the
volumes of the cell fractions
occupied by different materials.
Most
VoF methods
use a single linear interface
to divide two materials in a mixed cell%
~(Piecewise-Linear Interface Calculation~(PLIC))
[3,4,5].
Once the direction of the interface
normal
is know,
the location of the interface is uniquely
identified by the
volumes of the cells fraction.
Unfortunately the interface normal can not be evaluated
without the volume fraction data from the
surrounding cells,
which prohibits
the resulting approximation
to resolve any interface details
smaller than a characteristic size
of the cell cluster involved in
evaluation of the normal.

To overcome this limitation, we designed a new \emph{mass-conservative}
interface reconstruction metho
[1],
which calculates the interface based
on both \emph{volumes and centroids} of the cell fractions.
This choice of the input data
allows to evaluate the interface normal in a mixed cell
\emph{even without the information from the adjacent elements}.
The location of the linear interface in each mixed cell
is determined by \emph{fitting the centroid of the cell fraction
behind the interface to the reference one},
which leads to ($d\!-\!1$)-variate optimization problem in $\R^d$.
The technique proposed,
called Moment-of-Fluid~(MoF) interface reconstruction,
results in a \emph{second order accurate}
interface approximation~%
(linear interfaces are reconstructed exactly),
has higher resolution,
and is shown to be \emph{more accurate than VoF-PLIC methods}.

We present a detailed description of
MoF interface reconstruction algorithm in 2D,
which includes iterative procedure for centroid fitting
and a new algorithm for cutting appropriate volume
fractions from polygonal cells.
\clearpage
\begin{center}
{\bf REFERENCES}\\[-3mm]
\end{center}
\begin{tabular}{lp{110mm}}
\hspace{-3mm}{[1]}&
V.~Dyadechko and M.~Shashkov.
\newblock Moment-of-fluid interface reconstruction.
\newblock Technical Report LA-UR-05-7571, Los Alamos National Laboratory, Los
Alamos, NM, Oct 2005.\\[1mm]
\hspace{-3mm}{[2]}&
C.~W. Hirt and B.~D. Nichols.
Volume of fluid {(VOF)} method for the dynamics of free boundaries.
{\em Journal of Computational Physics}, 39(1):201--25, Jan 1981.\\[1mm]
\hspace{-3mm}{[3]}&
J.~E. Pilliod and E.~G. Puckett.
Second-order accurate volume-of-fluid algorithms for tracking
material interfaces.
{\em Journal of Computational Physics}, 199(2):465--502, Sep 2004.\\[1mm]
\hspace{-3mm}{[4]}&
B.~Swartz.
The second-order sharpening of blurred smooth borders.
{\em Mathematics of Computation}, 52(186):675--714, Apr 1989.\\[1mm]
\hspace{-3mm}{[5]}&
D.~L. Youngs.
An interface tracking method for a {3D} {E}ulerian hydrodynamics
code.
Technical Report AWRE/44/92/35, Atomic Weapon Research Establishment,
Aldermaston, Berkshire, UK, Apr 1987.
\end{tabular}


\end{document}
