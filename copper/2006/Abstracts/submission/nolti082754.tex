\documentclass{report}
\usepackage{amsmath,amssymb}
\setlength{\parindent}{0mm}
\setlength{\parskip}{1em}
\begin{document}
\begin{center}
\rule{6in}{1pt} \
{\large Josh nolting \\
{\bf HP Local Refinement Using FOSLS }}

2467 East 127th Court \\ Thornton \\ co 80241
\\
{\tt josh.nolting@colorado.edu}\\
Thomas Manteuffel\end{center}

Local refinement enables us to concentrate computational resources in areas
that need special attention, for example, near steep gradients and
singularities.  In order to use local refinement efficiently, it is important
to be able to quickly estimate local error. FOSLS is an ideal method to use for
this because the FOSLS functional yields a sharp a posteriori error measure for
each element. This talk will discuss a strategy for determining which elements
to refine in order to optimize the accuracy/computational cost. Set in the
context of a full multigrid
algorithm, our strategy leads to a refinement pattern with nearly equal error on
each element. Further refinement is essentially uniform, which allows for an
efficient parallel implementation. Numerical experiments will be presented. 


\end{document}
