\documentclass{report}
\usepackage{amsmath,amssymb}
\setlength{\parindent}{0mm}
\setlength{\parskip}{1em}
\begin{document}
\begin{center}
\rule{6in}{1pt} \
{\large Victoria Howle \\
{\bf The effect of boundary conditions and inner solver accuracy within pressure convection-diffusion preconditioners for the incompressible Navier-Stokes equations.}}

Sandia National Labs \\ PO Box 969 \\ MS9159 \\ Livermore \\ CA 94551
\\
{\tt vehowle@sandia.gov}\\
Ray Tuminaro\\
Jacob Schroder\end{center}

{\it Pressure convection--diffusion} preconditioners for solving the
incompressible Navier--Stokes equations were first proposed by Kay,
Loghin, and Wathen and Silvester, Elman, Kay, and Wathen. While numerous
theoretical and numerical studies have demonstrated mesh independent
convergence for these block methods on several problems and their overall
efficacy, there are several potential weaknesses remaining in the
practical use of these methods.

Perhaps one of the most poorly understood topics within this block
preconditioner family is the influence of boundary conditions on
overall algorithm convergence. The notion of differential commuting is
the basis for all {\it pressure convection-diffusion} preconditioners.
The main mathematical difficulty is that differential commuting does not
hold at the boundaries. Thus, it is unclear what boundary conditions
should be enforced in subblocks of the preconditioner. Heuristics have
been developed that roughly account for boundary conditions associated
with inflow, outflow, and no--slip. However, these rough heuristics often
do not properly capture the boundary interactions, and can in fact lead
to a degradation in convergence rates as the mesh is refined. We first
explore the effect of having ``ideal'' boundary conditions within the
preconditioner. While not computationally feasible, the ideal boundary
condition results highlight the importance of choosing suitable boundary
conditions. We then explore somewhat more practical approximations to the
ideal conditions based on ILU factorizations and probing [Siefert and de
Sturler, 2005].

Another important issue is the relationship between the accuracy of
inner sub-problem solves to the overall convergence rate of the outer
iteration. It has been known for quite some time that when
mesh-independent sub-problem solvers are used inexactly within this
block preconditioner, the overall convergence of the outer iteration
remains mesh independent for the Stokes equations [Silvester and Wathen,
1994]. The situation is not well understood for the Navier-Stokes
equations, and it is also not well understood when the inner subblock
solver has less than ideal behavior. We have observed a noticeable
degradation in the outer iteration convergence rate when a sub-solver is
terminated too quickly. We discuss possible solutions to this issue
including reusing information from repeated linear solves.


\end{document}
