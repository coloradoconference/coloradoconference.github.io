\documentclass{report}
\usepackage{amsmath,amssymb}
\setlength{\parindent}{0mm}
\setlength{\parskip}{1em}
\begin{document}
\begin{center}
\rule{6in}{1pt} \
{\large Robert E Beardmore \\
{\bf A Numerical Bifurcation Analysis of the Ornstein-Zernike equation with Hypernetted Chain Closure}}

Department of Mathematics \\ Imperial College London \\ South Kensington Campus \\ London SW7 2AZ \\ UK
\\
{\tt r.beardmore@ic.ac.uk}\\
A Peplow\\
F Bresme\end{center}

The isotropic Ornstein-Zernike (OZ) equation
\begin{equation}
\label{eq:oz} h(r) = c(r) + \rho \int_{\mathbb{R}^{3}} h(\|{\bf x}-{\bf
y}\|)c(\|{\bf y}\|)d{\bf y},
\end{equation}
that is the subject of this paper was presented almost a century ago to
model the molecular
structure of a fluid at varying densities.
In order to form a well-posed mathematical system of equations from
(\ref{eq:oz}) that can be solved, at least in principle, we assume the
existence of a closure relationship. This is an algebraic equation that
augments (\ref{eq:oz}) with a pointwise constraint that is deemed to hold
throughout the fluid and it forces a relationship between the total and
direct correlation functions ($h$ and $c$ respectively).

Some closures have a mathematically appealing structure in the sense that
the total correlation function is posed as a perturbation of the {\em
Mayer f-function} given by
\[ f(r)=-1+e^{-\beta u(r)}.\]
This perturbation depends on the potential $u$, temperature (essentially
$1/\beta$) and the
indirect correlation function through a nonlinear function that we denote
$G$: \begin{equation} h = f(r) + e^{-\beta u(r)}G(h-c),\qquad
(G(0)= 0), \label{eq:G} \end{equation} so that (\ref{eq:oz}-\ref{eq:G})
are solved together with $\beta$ and $\rho$ as bifurcation parameters.
There are many closures in use and if we
write $\exp_1(z) = -1+e^z$ then the hyper-netted chain (HNC) closure
\begin{equation} G(\gamma) = \exp_1(\gamma) \end{equation} has the form
of
(\ref{eq:G}) and is popular in the physics and chemistry literature.

The purpose of the talk is show that {\em any reasonable} discretisation
method applied to (\ref{eq:oz}-\ref{eq:G}) suffers from an inherent
defect if the HNC closure is used that can be summarised as follows:
phase transitions lead to fold bifurcations. The existence of a phase
transition is characterised by the existence of a bifurcation at infinity
with respect to $h$ in an $L^1$ norm at a certain density, such that
boundedness of $h$ is maintained in a certain $L^p$ norm. This behaviour
is difficult to mimic computationally by projecting onto a space of fixed
and finite dimension and, as a result, projections of
(\ref{eq:oz}-\ref{eq:G}) can be shown to undergo at least one fold
bifurcation if such a bifurcation at infinity is present. However, other
popular closure relations do not necessarily suffer from the same defect.


\end{document}
