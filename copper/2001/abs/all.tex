\documentclass[11pt]{article}

\setlength{\baselineskip}{11.4pt}
\setlength{\parskip}{0.9ex}
\setlength{\parindent}{0mm}

\setlength{\textwidth}{176mm}
\setlength{\oddsidemargin}{-5mm}
\setlength{\evensidemargin}{-5mm}

\setlength{\textheight}{236mm}
\setlength{\topmargin}{-19mm}

\newcommand{\nextab}[4]{
	\section{#2}
	{\bf #1} \\ \nopagebreak
	{#3} \\ \nopagebreak
	{\tt #4} \nopagebreak
	}


\title{Copper Mountain Conference Abstracts}
\author{~ }
\date{March 2001}

\begin{document}

\maketitle


%	*******************************************
%	*************  adams.html  ************
%	*******************************************

\nextab
	{Mark Adams}
	{Parallel algorithms for unstructured Gauss-Seidel multigrid smoothers}
	{MS 9217, Sandia National Laboratory, PO Box 969, Livermore CA 94551}
	{mfadams@ca.sandia.gov}

Gauss-Seidel is the most commonly used multigrid smoother on
serial machines as it is provably optimal on structured grids
and exhibits superior performance on unstructured grids as
well.  Gauss-Seidel is not commonly (or ever to our knowledge)
used on parallel machines as it is difficult to parallelize.
Jacobi, block Jacobi and additive Schwarz parallelize well but
require damping when used as multigrid smoothers; these damping
parameters are well know for say Poisson on regular grids but we
are not aware of an effective method to pick this parameter for
unstructured elasticity problems.  We have found that Krylov
solvers (eg, conjugate gradients (CG) for symmetric positive
definite problems) preconditioned with the above mentioned
methods are effective and to are knowledge are university used
for unstructured elasticity problems.  Gauss-Seidel does however
have some very attractive properties as a multigrid smoother,
namely: fast convergence, no global communication and fewer
flops per iteration as the ``Eisenstat'' trick allows for a
incorporating an initial guess whereas CG requires that a
residual be computed first and thus doubles the work required if
only one iteration is used (as is common for smoothers).  This
talk will discuss some preliminary work on one approach for
parallelizing a few iterations (ie, one or two) of Gauss-Seidel
with unstructured linear elasticity example problems with up to
76 million degrees of freedom.



%	*******************************************
%	*************  austin.html  ************
%	*******************************************

\nextab
{Travis Austin}
{A Divergence-Free Relaxation Scheme in an
	$H^1$-Finite Element Space: Numerical Results and Theory}
{Dept. of Applied Mathematics, 526 UCB, Boulder CO 80309-0526}
{austint73@yahoo.com}


An effective multilevel solver for (I - grad div) must account for the
presence of divergence-free error components.  As a result, the Raviart-Thomas
(RT) finite element spaces, with locally computable divergence-free
subspaces, are often used in the discretization of (I - grad div).
The presence of an epsilon-sized Laplacian term, leading to
(I - eps*Laplacian - grad div),
results in poor approximation properties for the
discontinuous RT finite element spaces.  With a new continuous, RT-like
finite element space, we can define a multilevel solution algorithm which
achieves optimal convergence factors.  In this talk, we will present numerical
results validating our claim and introduce theory to firmly establish the
algorithms effectiveness.

%	*******************************************
%	*************  bader.html  ************
%	*******************************************

\nextab
{Michael Bader}
{A robust and parallel Multigrid Method for Convection Diffusion Equations}
{Institut f\"ur Informatik, TU M\"unchen, 80290 M\"unchen, Germany}
{bader@in.tum.de}

We present a multigrid method that is based on the combination of
recursive substructuring of the computational domain and the
discretization on hierarchical bases and generating systems.
The substructuring approach makes the algorithm inherently parallel.
The use of hierarchical generating systems allows an optimal complexity
of the resulting method.
The robustness of the resulting method for convection problems is
achieved by a partial elimination between certain ``coarse grid unknowns''.

This additional elimination is performed such that in the resulting
matrices of the local systems of equations any couplings between coarse
grid unknowns on the separator and coarse grid unknowns on the subdomain
boundary are zero.
The coarse grid unknowns themselves are chosen with respect to the
physical properties of the underlying differential equation.

The resulting coarse grid meshes are not uniformly coarsened but are
refined on the edges and separators of the coarse grid cells.
This refinement of the coarse grid unknowns is chosen such that the
number of coarse grid unknowns grows with the square root of the total
number of boundary and separator unknowns.
This choice is due to the fact that for convection diffusion problems
the isolines of the transported matter are typically parabolic.

We will present some results for the application of this method to several
convection diffusion problems.
For these benchmark problems the algorithm shows an $O(N)$ complexity with
respect to the required computing time and storage.
In particular, the number of iterations needed does not depend on the
geometry of the flow field.
Moreover it is also independent of the strength of convection as long
as a certain amount of diffusion is present.
More precisely, it requires that on the finest grid a certain mesh Peclet
number is not exceeded.

The algorithm can also be applied to convection problems on complicated
computational domains with reasonably complex boundaries and obstacles.
It is especially suited to discretization techniques that are based on
a quadtree or octree representation of the computational domain.




%	*******************************************
%	*************  bennighof.html  ************
%	*******************************************

\nextab
{Jeff Bennighof and Richard B. Lehoucq}
{Automated Multi-Level Substructuring: Numerical Results}
{Dept. Aerospace Engineering \& Engineering Mechanics,
University of Texas at Austin,
W.R. Woolrich Laboratories (WRW), Room 117D - Mail Code: C0600,
Austin TX 78712-1085
\\ and \\
Sandia National Laboratories, PO Box 5800, MS 1110, Albuquerque NM 87185-1110
}{}


In the Automated Multi-Level Substructuring (AMLS) method for
frequency response analysis of complex structures, very large
finite element models are divided into thousands of
substructures on many levels, and frequency response is
represented in terms of substructure modes. The algorithm
approximates the fundamental global mode subspace.  Numerical
examples demonstrate how multi-million degree of freedom FEM
models are efficiently resolved and how excellent eigenvalue and
frequency response accuracy can be obtained.




%	*******************************************
%	*************  bochev.html  ************
%	*******************************************

\nextab
{Pavel Bochev}
{Constrained transport remap algorithms on unstructured quadrilateral and hexahedral grids}
{Dept. of Mathematics, Box 19408, University of Texas at Arlington, Arlington TX 76019-0408}
{bochev@math.uta.edu}


The constrained transport (CT) algorithm introduced by Evans and Hawley
(1988) on structured meshes provides a mechanism for advection of magnetic
flux density which also preserves a discrete div(B)=0 property. CT is
extended to unstructured meshes and algorithms for computing high-order
upwind fluxes on edges are compared. The natural association of edge and
face elements with CT on unstructured meshes is discussed.




%	*******************************************
%	*************  brandt.html  ************
%	*******************************************

\nextab
{Achi Brandt}
{Multiscale Computation: Recent Methodological Developments}
{Dept. of Computer Science and Applied Mathematics, The Weizmann Institute of Science, Rehovot 76100, Israel}
{achi@wisdom.weizmann.ac.il}


A brief account of new multiscale computational approaches in
the following areas: efficient algebraic multigrid, Dirac solvers,
image processing, clustering, tomography, inverse hyperbolic problems,
optimal control, collective computation of many eigenfunctions,
global optimization, statistical mechanics and molecular dynamics.
A full account of these and other developments, with detailed
reference list, will appear in the Proceeding of the Yosemite
Educational Symposium (Springer Series Lecture Notes in Computational
Science and Engineering) and on the Internet at the time of the
Copper Mountain Conference.


%	*******************************************
%	*************  brenner.html  ************
%	*******************************************

\nextab
{Susanne C. Brenner}
{An Additive Theory for Multigrid V-Cycle Algorithms}
{Dept. of Mathematics, University of South Carolina, Columbia SC 29208 USA}
{brenner@math.sc.edu}



In this talk we will present a result on the asymptotic
behavior of the contraction numbers of multigrid V-cycle
algorithms with respect to the number of smoothing steps.
This result is obtained using an additive convergence theory
and new estimates for mesh-dependent norms.  When combined
with earlier results of Zhang, Bramble and Pasciak, it yields
a generalization of the classical convergence theorem of
Braess and Hackbusch to problems without full elliptic regularity.



%	*******************************************
%	*************  brezina.html  ************
%	*******************************************

\nextab
{Marian Brezina}
{An extension of convergence theory for the smoothed aggregation multigrid.}
{University of Colorado, Boulder CO 80309-0526}
{brezina@colorado.edu}


We have recently presented modified version of the smoothed aggregation
multigrid suitable for application on problems featuring large variation of
coefficients. For the modified version, we have been able to prove convergence
independent of the size of the coefficient jump under certain assumptions
on the jump distribution.

However, numerical experiments suggest that our assumptions may be too
restrictive and one may expect good convergence rates even
in situations where our assumptions do not hold.
In this talk, we present an extension of the convergence theory to certain
configurations excluded by our theory to date.



%	*******************************************
%	*************  chartier.html  ************
%	*******************************************

\nextab
{Tim Chartier}
{Spectral AMGe}
{Dept. Applied Mathematics, 526 UCB, University of Colorado, Boulder CO 80309-0526}
{chartier@colorado.edu}


AMGe is an algebraic multigrid method for solving discretizations that
arise in Ritz-type finite element methods for partial differential
equations.  Assuming access to the element stiffness matrices, Spectral
AMGe uses the spectral decomposition of small collections of element
stiffness matrices to determine local representations of algebraically
``smooth'' error components.  This decomposition provides the basis for
generating a coarse grid and for defining effective interpolation.  Recent
results will be presented for problems such as linear elasticity.





%	*******************************************
%	*************  codd.html  ************
%	*******************************************

\nextab
{Andrea Codd}
{First-Order System Least Squares (FOSLS) for Elliptic Grid Generation (EGG)}
{Dept. of Applied Mathematics, University of Colorado at Boulder, Boulder CO 80309-0526}
{andrea.codd@colorado.edu}


Elliptic Grid Generation (EGG), using the Winslow generator, defines a map
between a simple computational region and a potentially complicated
physical region. It can be used numerically to create meshes for
discretizing equations directly on the physical domain or indirectly on
the computational domain by way of the transforming map. EGG allows
complete specification of the boundary, and it guarantees a one-to-one and
onto transformation when the computational region is convex.

A new fully variational approach is developed for solving the Winslow
equations that enables accurate discretization and fast solution methods.
The EGG equations are converted to a first-order system that is then
linearized via Newton's method. First-order system least squares (FOSLS)
is used to formulate and discretize the Newton step, and the resulting
matrix equation is solved using algebraic multigrid (AMG). The approach is
coupled with nested iteration to provide an accurate initial guess for
finer levels using coarse-level computation. Theoretical and numerical
results confirm the usual full multigrid efficiency: accuracy comparable
to the finest-level discretization is achieved at a cost proportional to
the number of finest-level degrees of freedom.




%	*******************************************
%	*************  cullum.html  ************
%	*******************************************

\nextab
{Jane Cullum}
{A numerical exploration of algebraic multigrid}
{MS B256, CCS-3, Los Alamos National Laboratory, Los Alamos NM 87545}
{cullumj@lanl.gov}


Algebraic multigrid (AMG) methods for
solving large systems of linear equations, $Ax=b$,
are  matrix-based analogs of geometric
multigrid methods.
Both types of methods are multi-level,
with the method being applied recursively
at each level except at the bottom level.


In this talk we
focus on AMG procedures which are
based upon papers of Ruge and St\"uben.
At each level, except the bottom level,
the procedure requires the choice of the
\begin{enumerate}
\item {\em Coarse} and {\em Fine} variables
\item {\em Prolongation} and  {\em Restriction} operators
\item {\em Coarse} problem
\item {\em Smoother} operator
\end{enumerate}

The choice of each of these components
places constraints upon the choice of each of
the other components.
Numerically, we explore the effects of
different choices upon the observed convergence.
We consider example problems where AMG
has been shown to work well and other example
problems where  AMG converges
slowly.



%	*******************************************
%	*************  diskin.html  ************
%	*******************************************

\nextab
{Boris Diskin}
{Textbook Multigrid Efficiency for High-Reynolds-Number Navier-Stokes Equations}
{ICASE, MS 132C, NASA Langley Research Center, Hampton VA 23681}
{bdiskin@icase.edu}


Full multigrid (FMG) algorithms are the fastest solvers for elliptic problems.
These algorithms can solve a general discretized elliptic problem
to the discretization accuracy in a computational work  that is a small
(less than 10) multiple of the operation count in one target-grid residual
evaluation. Such efficiency is known as textbook multigrid efficiency
(TME). The difficulties associated with extending
TME for solution of the Reynolds-averaged
Navier-Stokes (RANS) equations relate to the fact that the
RANS equations are a system of coupled nonlinear equations
that is not, even for subsonic Mach numbers, fully elliptic, but
contain hyperbolic partitions. TME for the RANS simulations
can be achieved if the different factors contributing to
the system could be separated and treated
optimally, e.g., by multigrid for elliptic factors
and by downstream marching for hyperbolic factors.
One of the ways to separate the factors is the
{\em distributed relaxation} approach proposed by A. Brandt.
Earlier demonstrations of TME solvers with
distributed relaxation has been performed for
the incompressible free-stream inviscid and viscous
flows without boundary layers.


In this talk, I am going to outline a general framework
for achieving TME in solution of the  high-Reynolds-number Navier-Stokes
equations. TME distributed-relaxation solvers will be demonstrated for
the viscous incompressible and subsonic compressible flows with
boundary layers and for the inviscid compressible flow with shock.


%	*******************************************
%	*************  douglas.html  ************
%	*******************************************

\nextab
{Craig C. Douglas}
{Subgrid Substructuring Techniques for a Multilayered Ocean Model}
{Dept. Computer Science, University of Kentucky,
325 McVey Hall - CCS, Lexington KY 40506-0045}
{douglas@ccs.uky.edu}


We simulate the wind driven circulation in oceans.  Our aim
is to study the impact of short scale wind forcing on the
oceanic circulation.  We use the isopycnal version of the
Spectral Element Ocean Model (SEOM).  The modeling of these
flows and their climatic impact is complicated by the
inherent range of spatial scales involved, which extend
from the global scale of $O(10,000)$ km down to the local
scale of $O(1)$ km, and by the intrinsic three dimensionality
of the dynamics.

SEOM offers an elegant solution to these difficulties.  It
features advanced algorithms, based on $h-p$ type finite
element methods, allowing accurate representation of
complex coastline and oceanic bathymetry, variable lateral
resolution, and high order solution of the three
dimensional oceanic equations of motion.

SEOM's geometrical flexibility permits highly inhomogeneous
horizontal grids.  An added advantage of the technique is
its scalability.  Most of the computations are carried out
at the element level; only interface information needs to
be exchanged between elements.  The dual characteristic of
dense and structured local computations, and sparse and
unstructured communication enhances the locality of the
computations.  The dual characteristic of dense and
structured local computations, and sparse and unstructured
communication enhances the locality of the computations,
and makes SEOM ideally suited for parallel computers.

In this talk, we define a novel set of techniques that
allow us to store relevant matrices for 1/200-th the normal
amount that would be expected using normal sparse matrix
techniques.  A combination of a Schur complement and a
parallel algebraic multigrid method is described and
compared to a traditional matrix-free preconditioned
conjugate gradient solution methodology.


%	*******************************************
%	*************  farrell.html  ************
%	*******************************************

\nextab
{Paul A. Farrell}
{Factors Involved in the Performance of Computations on Beowulf Clusters}
{Dept. of Mathematics \& Computer Science,
Kent State University, Kent OH 44242}
{farrell@mcs.kent.edu}


PC (Beowulf) clusters represent a cost-effective platform
for large scale scientific computations. In this talk,
we examine the effects of some possible configuration,
hardware, and software choices on
the communications latency and throughput attainable, and the consequent
impact on scalability and performance of codes.
We compare performance currently attainable using Gigabit Ethernet with
that of Fast Ethernet.
We discuss the effects of various versions of the Linux kernel,
and the approaches to tuning it to improve TCP/IP performance.


We evaluate and compare the performance of LAM, MPICH, and MVICH on a
Linux cluster connected by a Gigabit Ethernet network.
Since LAM and MPICH use the
TCP/IP socket interface for communicating messages, it is critical to
have high TCP/IP performance for these to give satisfactory results.
Despite many efforts to improve TCP/IP
performance, the performance graphs presented here indicate that the
overhead incurred in protocol stack processing is still high.
We discuss the Virtual Interface Architecture (VIA) which is intended to
provide low latency, high bandwidth message-passing between user processes.
Developments such as the VIA-based MPI implementation MVICH can improve
communication throughput and thus give the promise of enabling
distributed applications to improve performance.

Finally we present some examples of how these various choices can
impact the performance of an example multigrid code.




%	*******************************************
%	*************  fulton.html  ************
%	*******************************************

\nextab
{Scott R. Fulton}
{On the Accuracy of Multigrid Truncation Error Estimates}
{Dept. of Mathematics and Computer Science,
Clarkson University, Potsdam NY 13699-5815}
{fulton@clarkson.edu}


In solving boundary-value problems, multigrid methods can provide
computable estimates of the truncation error, which can then be used in
adaptive grid refinement algorithms or in extrapolation to higher-order
accuracy (tau-extrapolation).  To be useful, these estimates must be
accurate, i.e., if the truncation error itself is order p, then the
computed estimate must differ from it by a term of order p+m for some
positive m.



This paper analyzes the accuracy of multigrid truncation error
estimates, examining how m depends on the grid transfers employed.  In
particular, we compare two definitions of the relative local truncation
error (a computable estimate of the truncation error difference between
two grids) found in the literature.  One definition requires a careful
choice of high-order grid transfers to achieve accurate estimates (e.g.,
Bernert, 1997), while the other (e.g., Schaffer, 1984) can utilize
simpler grid transfers (and is itself simpler to compute).  Our
analytical results are illustrated with numerical calculations for
several model problems.


%	*******************************************
%	*************  gopalakrishnan.html  ************
%	*******************************************

\nextab
{Jayadeep Gopalakrishnan}
{Multigrid and Schwarz methods for time-harmonic Maxwell equations}
{Institute for Mathematics and its Applications,
514 Vincent Hall, 206 Church St SE, Minneapolis MN 55414}
{jayg@ima.umn.edu}


Time-harmonic Maxwell equations in a lossless cavity lead to a second
order differential equation for electric field involving a
differential operator that is neither elliptic nor definite.  A
Galerkin method using Nedelec's curl-conforming finite elements can be
employed to get approximate solutions numerically. In this talk,
results of [4, 5] on the suitability of multigrid and Schwarz methods
for efficiently solving the resulting indefinite linear system will be
presented.


The analysis of the present work builds on techniques in [1, 6] to
overcome the difficulties caused by the non-ellipticity of the
operator. As in their analysis, discrete Helmholtz decompositions play
a crucial role.  A new and critical ingredient of our analysis is an
estimate on discrete solution operators of Maxwell equations.  This
estimate allows analysis of overlapping Schwarz methods in the spirit
of perturbation arguments in [3], even though our operator is not
elliptic.  It also allows analysis of a `\\' multigrid cycle
using techniques of [2], although prima facie it may appear that
ellipticity is required for application of these techniques. In both
cases, the analysis involves comparison of operators with their
analogues in positive definite case.

Of practical importance is the fact that some algorithms that work
well in the elliptic case, fails in our application. It has now been
known for some time that a multigrid V-cycle with a point-Jacobi
smoother is not appropriate for our problem.  However, we show that a
multigrid algorithm with block Jacobi or block Gauss-Seidel smoother,
if the blocks are chosen as in [1], leads to good convergence rates,
provided the coarse mesh used is sufficiently fine.  Our results also
indicate that, in contrast to the elliptic case, replacing subdomain
solves by equivalent operations in overlapping Schwarz methods does
not lead to good results.



\begin{enumerate}

\item
D. N. Arnold, R. S. Falk and R. Winther,
{\em Multigrid in H(div) and H(curl)},
Numer. Math. {\bf 85}(2):197--217, 2000.


\item
J. H. Bramble, D. Y. Kwak, and J. E. Pasciak.
{\em Uniform convergence of multigrid V-cycle
iterations for indefinite and nonsymmetric
problems},  SIAM J. Numer. Anal.,
{\bf 31}(6):1746--1763, 1994.


\item
X.-C. Cai and O. B. Widlund.
{\em Domain decomposition algorithms for
indefinite elliptic problems},
SIAM J. Sci. Stat. Comput.,
{\bf 13}(1):243--258, 1992.


\item
R. Hiptmair,
{\em Multigrid method for Maxwell's equations},
SIAM J. Numer. Anal.,
{\bf 36}(1):204--225, 1999.

\item
J. Gopalakrishnan and J. E. Pasciak,
{\em Overlapping Schwarz preconditioners for indefinite
time harmonic Maxwell equations}.
Submitted. Available as IMA Preprint 1711, June 2000.


\item
J. Gopalakrishnan, J. E. Pasciak and L. Demkowicz.
{\em A multigrid algorithm for time harmonic
Maxwell equations}. In preparation.


\end{enumerate}



%	*******************************************
%	*************  haase.html  ************
%	*******************************************

\nextab
{Gundolf Haase}
{Simultaneous Iterations in Ocean Modeling}
{Johannes Kepler University Linz,
Institute for Analysis and Computational Mathematics,
Dept. of Computational Mathematics and Optimization,
Altenberger Strasse 69,
A-4040 Linz, Austria}
{ghaase@numa.uni-linz.ac.at}


The investigated ocean modeling code bases on a
five layer model using higher order spectral elements
for discretization
and a special filtering technique
for the vortex and divergence field.

These filterings contain the solving of the Laplace equation
for the velocity components in x- and y-direction
cause on each layer most of the arithmetic work per time step.
The system matrix is never accumulated and so all
matrix operations are performed
on an element level with the small dense
element matrices.

We have the special situation that the system matrix and the
filtering matrix are the same for both directions and
for all layers. This gave us the chance
to redesigning the code such that all layers and directions
can be handled simultaneously.
The general reduction of memory accesses together with
cache aware data structures improve the run time significantly.

These techniques can be also used very successfully when
the pcg solver uses multigrid as preconditioner.


Ocean Modeling Session


%	*******************************************
%	*************  haber.html  ************
%	*******************************************

\nextab
{E. Haber, U. Ascher}
{Multigrid all-at-once methods for large scale inverse problems}
{Dept of Computer Science UBC, Vancouver BC, Canada}
{}


The problem of recovering a parameter function based on
measurements of solutions of a system of partial differential
equations in several space variables leads to a number of
computational challenges. Upon discretization of a regularized
formulation a large, sparse constrained optimization problem is
obtained.
Typically in the literature, the constraints are eliminated and
the resulting unconstrained formulation is solved by some variant
of Newton's method, usually the Gauss-Newton method. A
preconditioned conjugate gradient algorithm is applied at each
iteration for the resulting reduced Hessian system.

In this talk we apply instead a multigrid method directly to the
KKT system arising from a Newton-type method for the constrained
formulation (an ``all-at-once'' approach). Since the reduced
Hessian system presents significant expense already in forming
a matrix-vector product, the savings are substantial.

Numerical experiments are performed for the DC-resistivity
problem in 3D, comparing the two approaches for solving the
linear system at each Gauss-Newton iteration and a substantial
efficiency gain is demonstrated.
The relative efficiency of our proposed method is even higher in
the context of inexact Newton-type methods, where the linear
system at each iteration is solved less accurately. .



%	*******************************************
%	*************  howle.html  ************
%	*******************************************

\nextab
{Victoria E. Howle}
{Experiences with Parallel Block Preconditioning of
	the Linearized Incompressible Navier-Stokes Equations}
{Sandia National Laboratory, MS 9217, PO Box 969, Livermore CA 94551}
{vehowle@ca.sandia.gov}
\\ {\bf Ray Tuminaro}, Sandia National Laboratory
\\ {\bf John Shadid}, Sandia National Laboratory
\\ {\bf Howard Elman}, Univ. of Maryland


We consider the use of parallel block preconditioning techniques
for the Navier-Stokes equations.  These block techniques correspond
to those of Kay \& Loghin and result in methods that can be effective
and robust over a variety of Reynolds numbers. The basic idea is to
approximate the Schur complement operator for the Pressure equation
based on the notion that certain differential operators approximately
commute. The resulting preconditioner requires two block `solves' at each
invocation: one for the pressure unknowns and the other for the velocities.
A multigrid V cycle is then used to approximate each of these block solves,
which makes the cost per iteration proportional to the number of unknowns.
Numerical studies on the ASCI Red machine demonstrate that the preconditioner
is effective in terms of both convergence rate and parallel performance.


%	*******************************************
%	*************  hu.html  ************
%	*******************************************

\nextab
{Jonathan Hu}
{Parallel Algebraic Aggregation for Maxwell's Equations}
{Sandia National Laboratories, P.O. Box 969, MS 9217, Livermore CA 94551}
{jhu@ca.sandia.gov}


{\bf Ray Tuminaro} \\
Sandia National Laboratories, P.O. Box 969, MS 9217, Livermore CA 94551
\\ \verb9tuminaro@ca.sandia.gov9

{\bf Allen C. Robinson} \\
Sandia National Laboratories, Computational Physics R\&D,
	P.O. Box 5800, MS 0819, Albuquerque NM 87185-0819
\\ \verb9acrobin@sandia.gov9


{\bf Pavel B. Bochev} \\
Dept. of Mathematics, Box 19408, University of Texas, Arlington TX 76019-0408
\\ \verb9bochev@uta.edu9


We consider the use of parallel algebraic multigrid for the
solution of Maxwell's equations which are discretized via
edge elements.  The key difficulty is properly mapping the
Curl operator's null space on to the coarse grids via a
prolongation operator that is constructed using only
algebraic information (i.e. matrix coefficients and a
minimal amount of element information).  The scheme that we
consider is based on the work of Reitzinger and Sch\"oberl
as well as that of Hiptmair.  This parallel multilevel
preconditioner is implemented within the ML framework which
already contains similar techniques like smoothed
aggregation.  The resulting iteration scheme is being
integrated into a large complex parallel code that requires
the repeated solution of the eddy current approximation to
Maxwell's equations in a heterogeneous material properties
environment as part of an overall Arbitrary
Lagrangian-Eulerian magnetohydrodynamics algorithm.
Numerical experiments are presented illustrating the
efficiency of the approach on the ASCI Red machine in terms
of both convergence and parallel speed-up.


%	*******************************************
%	*************  iskandarani.html  ************
%	*******************************************

\nextab
{Mohamed Iskandarani}
{Overview of Ocean Modeling Using Spectral Elements}
{Rosenstiel School of Marine and Atmospheric Science,
4600 Rickenbacker Causeway, Miami FL 33149-1098}
{miskandarani@rsmas.miami.edu}

Finite element are relatively uncommon for basin scale simulations of
the oceanic circulation. Here, we present the spectral element ocean
model, SEOM, an h-p type FEM model that solves the Boussinesq
hydrostatic primitive equations. The model exist in two versions that
differ primarily in their choice of vertical coordinate system: a
``sigma''-like terrain-following coordinate system, and a density
(isopycnal) coordinate. We discuss briefly the pro's and cons of each
version and present a calculation of the wind-driven circulation over
the Pacific Basin using the isopycnal model.




%	*******************************************
%	*************  israeli.html  ************
%	*******************************************

\nextab
{Moshe Israeli, Elena Braverman}
{A Hierarchical Domain Decomposition Method for the Solution of Helmholtz and Biharmonic Equations}
{Technion, Computer Science Dept., Haifa 32000, Israel}
{israeli@cs.technion.ac.il ~ ~ maelena@cs.technion.ac.il}

{\bf Amir Averbuch} \\
School of Mathematical Sciences, Tel Aviv University, Tel Aviv 69978, Israel
\\ \verb9amir@math.tau.ac.il9

Implicit discretization of time dependent problems
in computational physics, semiconductor device simulation,
electromigration and
fluid dynamics often gives rise to elliptic equations.
Helmholtz type equations usually appear in acoustics
or electromagnetics and also as a result of time discretization of
the Navier-Stokes equations.
Biharmonic problems appear in elasticity and viscous flows.

We solve the Helmholtz and the biharmonic equation by the Domain
Decomposition (DD) methods.
Previously [1] we adopted a DD method where the equation was solved in
each subdomain with assumed boundary conditions,
resulting in jumps in function or derivative on subdomain boundaries.
These jumps were removed by the introduction of singularity layers.
In order to account for the global effect of the layers we computed
first the influence of each layer on each subdomain boundary,
taking into account the decay or smoothing out of the influence
as a function of the distance from the layer.
To reduce the communication load,  compression in a multiwavelet
basis was applied.
Nevertheless, this part of the  procedure can  become expensive
as the number of subdomains grows considerably.
An  algorithm for a fast solution of the Poisson equation
by decomposition of the domain into square domains and the subsequent
matching of these solutions by the fast multipole method was
developed in [2].

The algorithm developed in [1] incorporates the following steps:

1. In each subdomain
a particular solution of the non-homogeneous equation
with arbitrary Neumann (Dirichlet) boundary conditions is found.

2. The collection of particular solutions usually has
discontinuities (or discontinuities in the derivatives) on the
boundaries of the subdomains. We introduce double (single) layers on the
boundaries to match
the solutions from different domains to have continuous global solution.
The effect of these layers on other boundaries is calculated.

3. The solutions obtained at the first step are patched
by adding the solutions of the Laplace
equation with the boundary conditions that were computed in the previous
step.

4. An additional solution of the corresponding homogeneous equation is
added to satisfy the global boundary conditions.


Considerable improvement in the efficiency of
the interface jump removal step can be achieved if at each step only
adjacent boxes are matched. This is the basis of the hierarchical
approach which was proposed in [4].
The ``elementary step'' of the hierarchical algorithm is the following.

1. First, for each two adjacent subdomains some boundary conditions
are defined.
These conditions should not contradict the given right hand side
at the junctions (see [4]).
The Poisson equation is solved with these boundary
conditions by a fast spectral algorithm [3].

2. The solutions have a discontinuity in the first derivative.
We match the subdomains by adding certain discontinuous
functions. In fact we only evaluate these functions at the boundaries
of two adjacent subdomains and then
solve the homogeneous equation in each subdomain with the cumulative
boundary conditions.

3. The global homogeneous equation is solved in such a way that
it satisfies
the assumed conditions at the ``global boundaries'' of the merged
subdomains.
The solution of the non-homogeneous equation is expensive
if compared to the homogeneous equation where efficient algorithms
are available [3].

This step is repeated hierarchically.
For example, first the smallest adjacent domains are matched:
box 1 with box 2, box 3 with box 4, then the merged box 1,2 is
matched with 3,4, afterwards the box which is a union of boxes 1,2,3,4
is patched with the adjacent merged box etc.

Here we extend the results of [1,4] in the following directions.

1. In addition to the Poisson equation, we solve
the Helmholtz or modified Helmholtz equation.
One of the problems that arise is
resonance and non-resonance situations
for Dirichlet and Neumann boundary conditions.
This can be avoided by introducing
``transparent'' boundary conditions which are
approximations to the Sommerfeld radiation
boundary condition.
This means that the scattered wave behaves
asymptotically like a diverging spherical wave. Some modifications
of the non-reflecting conditions were considered in [5].

2. We solve the biharmonic equation with either a free boundary
as in [2] or
boundary conditions of one type only
(Dirichlet or Neumann).
The biharmonic equation is split into a coupled system of
the Poisson equations. The unknown function satisfies the Poisson equation
with a certain right hand side which will be called a vorticity
function.
The solution  is also performed by DD.
Here the matching algorithm which was developed in [1,4]
is applied to find the smooth vorticity function which matches
the given right hand side. Then we obtain a solution of the
biharmonic equation. The global boundary conditions of either
Dirichlet or Neumann
type can be satisfied by the addition of an appropriate
harmonic function, as was done for the solution of the Poisson equation.

\begin{enumerate}

\item
A. Averbuch, E. Braverman and M. Israeli,
{\em Parallel adaptive solution of
a Poisson equation with multiwavelets},
SIAM J. Sci. Comput. {\bf 22} (2000), 1053--1086.

\item
L. Greengard and J.-Y. Lee,
{\em A direct adaptive Poisson solver of arbitrary order
accuracy}, J. Comput. Phys. {\bf 125} (1996), 415--424.

\item
A. Averbuch, M. Israeli, L. Vozovoi,
{\em A fast Poisson solver of arbitrary order accuracy
in rectangular regions},
SIAM J. of Sci. Comput.
{\bf 19} (1998), 933--952.

\item
M. Israeli, E. Braverman and A. Averbuch,
{\em A Hierarchical domain decomposition method with
low communication overhead}, submitted to Proceedings of
the 13th Domain Decomposition Conference, Lyon, 2000.

\item
A. Bamberger, P. Joly and J.E. Roberts,
{\em Second-order absorbing boundary conditions for the wave
equation: a solution for the corner},
SIAM J. Numer. Anal. {\bf 27} (1990), 323--352.

\end{enumerate}

%	*******************************************
%	*************  jones.html  ************
%	*******************************************

\nextab
{Jim Jones, Panayot Vassilevski, Carol S. Woodward}
{Computational Issues in the Application of Nonlinear Multigrid to Nonlinear Diffusion Problems}
{Center for Applied Scientific Computing, Lawrence Livermore National Laboratory, Box 808, L-561 Livermore CA 94551}
{jjones@llnl.gov}



Our discussion concerns the application of nonlinear multigrid,
or FAS, to nonlinear diffusion problems with the eventual target
application of variably saturated flow.  We will focus on
computational issues related to calculation of efficient
interpolation and restriction operators for the multigrid
V-cycle, as well as appropriate nonlinear coarse grid operators.
Our numerical results compare this nonlinear multigrid method
with traditional Newton techniques.

This work was performed under the auspices of the U.S. Dept. of
Energy by University of California
Lawrence Livermore National Laboratory under
Contract W-7405-Eng-48


%	*******************************************
%	*************  kenigsberg.html  ************
%	*******************************************

\nextab
{Avraham Kenigsberg}
{A Multigrid Approach for Fast Geodesic Active Contours}
{Dept. of Computer Science, Technion, Haifa 32000, Israel}
{}

Image segmentation is a basic and important problem in the field
of computer vision. A recent geometric approach for image
segmentation is the geodesic active contour based on the level-set
method. One drawback of the method, is the extended numerical
support that makes its solution time consuming. We propose to solve an
implicit system of the geodesic active contour model using the
computationally efficient multigrid method.

This work is a part of the M.Sc. thesis research of the author,
under the supervision of Dr. Ron Kimmel and Assoc. Prof. Irad Yavneh.




%	*******************************************
%	*************  kiefer.html  ************
%	*******************************************

\nextab
{Michael Griebel, Frank Kiefer}
{AMG-based wavelet-like multiscale solvers for convection-diffusion problems}
{Dept. of Applied Mathematics, Division for Scientific Computing and Numerical Simulation, Wegelerstr. 6, D-53115 Bonn, Germany}
{griebel@iam.uni-bonn.de, kiefer@iam.uni-bonn.de}


We consider the efficient solution of discrete convection
dominated convection-diffusion problems.  It is well known that
the standard hierarchical basis multigrid method (HBMG) leads
for discrete operators arising from singularly perturbed
convection-diffusion problems neither to optimal (w.r.t.
mesh size) nor to robust solvers, i.e. the performance still
depends strongly on the coefficients in the differential
equation (e.g. strength of convection).

(Pre-)Wavelet splittings of the underlying function spaces allow
efficient algorithms that can be viewed as generalized HBMG
methods.  They can be interpreted as ordinary multigrid methods
that use a special kind of multiscale smoother and show for the
respective non-perturbed equations an optimal convergence
behavior similar to classical multigrid.  In our general
Petrov--Galerkin multiscale approach we apply problem-dependent
coarsening strategies known from robust multigrid techniques
(matrix-dependent prolongations, algebraic coarsening) together
with certain (pre-)wavelet-like and hierarchical multiscale
decompositions of the trial- and test-spaces on the finest
grid.  We demonstrate through extensive experiments that by this
choice we are able to construct generalized HBMG methods, that
result in possibly robust solvers.



%	*******************************************
%	*************  kim.html  ************
%	*******************************************

\nextab
{Chisup Kim}
{A Two-level Preconditioner for an Anisotropic Mixed Finite Element Problem}
{Dept. of Mathematics, Texas A\&M University, College Station TX 77843-3368}
{cskim@math.tamu.edu}


We consider a mixed finite element method for a model second anisotropic
elliptic equation on the unit square $\Omega$ of the following form
\begin{eqnarray}
- ( p_xx + \varepsilon p_yy ) &=& f  ~{\rm in} ~ \Omega,
\nonumber \\
p &=& 0  ~{\rm in} ~ \Gamma,
\nonumber
\end{eqnarray}
where $\Gamma$ is the boundary of $\Omega$ and $\varepsilon$
is a small constant.
We study a multilevel preconditioner for this problem on uniform
rectangular and triangular meshes.

We use a two-level result by Bramble, Pasciak, and Zhang [1]
in which no approximation or regularity
conditions are required.  Our ``coarse'' level problem will be the finite
element problem on the same mesh as in the mixed finite element problem.

In the rectangular case, we use the Schur complement of the mixed problem
at the ``fine'' level.  A mesh dependent form is used in the analysis and
implementation.  In the triangular case, we take the equivalent
nonconforming finite element problem to the mixed finite element problem
as the ``fine'' level problem.  In both cases, line Jacobi smoothers are
used to effectively reduce the error in the ``fine'' level.


[1] Bramble, Pasciak, and Zhang, {\em East-West J.~Numer.~Math.},
	{\bf 4} 1996, pp.99--120


%	*******************************************
%	*************  kirk.html  ************
%	*******************************************

\nextab
{B. Kirk, K. Lipnikov, G. Carey}
{Cascadic Multigrid Simulation of Incompressible Viscous Flow Problems:  Performance Analysis and Parallel Workstation and Cluster Implementation}
{Aerospace Enineering Dept, C0600, University of Texas, Austin TX 78712}
{benkirk@cfdlab.ae.utexas.edu}


In this investigation we examine the use of one-way cascadic
multigrid strategies CMG for solution of incompressible viscous
flow problems using the finite element method.  The content of
the presentation is as follows:  First we describe the basic CMG
approach for representative elliptic boundary value problems and
summarize the theoretical error estimates from approximation
theory, desired smoother properties, and arithmetic complexity
of the method as a function of iteration parameters at each
level.  The extension of these error and complexity estimates to
adaptive grids is also given.


In the numerical experiments, performance of the algorithm on
both serial and distributed parallel systems is examined.  We
carry out a series of comparison studies between the CMG scheme
and a standard bi-conjugate gradient solve strategy on the fine
level grid.  Other issues such as the treatment of the diagonal
elements in the zero block corresponding to the pressure
variables are also considered and the results of associated
numerical studies are presented.  Finally, parallel performance
studies on a distributed parallel PC cluster in the CFDLab
(http://cfdlab.ae.utexas.edu) are described and both performance
and flow simulation results are given for 3D applications to
linear Stokes flow, low Reynolds number Navier Stokes problems,
and coupled fluid-thermal problems.


\begin{enumerate}


\item Bornemann, F. and P. Deuflhard,
{\em The Cascadic Multigrid Method for Elliptic Problems},
Num. Math {\bf 75}, 135--152, 1996.

\item Bramble, J., J. Pasciak, J. Wang and J. Xu,
{\em Convergence Estimates for Multigrid Algorithms
without Regularity Assumptions}, Math. Comp
{\bf 57}, 23--45, 1991.

\item Carey, G. F.,
{\em Computational Grids:  Generation, Adaptation
and Solution Strategies}, Taylor \& Francis, 1997.

\item Carey, G. F.,
{\em Adaptive Techniques and Related Issues in
Finite Element Modeling of Heat and Fluid Flow},
To appear in Proceedings of CHT'01 (Australia May 2001).

\item Carey, G. F., R. McLay, G. Bicken, B. Barth, S. Swift and
A. Ardelea,
{\em Parallel Finite Element Solution of 3D
Rayleigh-Benard-Marangoni Flows}, IJNMF {\bf 31},
37--52, 1999.

\item Carey, G. F., R. McLay, W. Barth, S. Swift and B. Kirk,
{\em Distributed Parallel Simulation of Surface Tension Driven
Viscous Flow and Transport Processes}, Submitted to World
Scientific, June, 2000.

\item Ciarlet, P. J.,
{\em The Finite Element Method for Elliptic Problems},
North-Holland, Amsterdam, 1978.

\item Deuflhard, P.,
{\em Cascadic Conjugate Gradient Methods for
Elliptic Partial Differential Equations: Algorithm and Numerical
Results}, in D. Keyes and J. Xu (eds).,
{\em Domain Decomposition
Methods in Scientific and Engineering Computing},
AMS Series, 180, 29--42, 1994.

\end{enumerate}



%	*******************************************
%	*************  knyazev.html  ************
%	*******************************************


\section{Eigensolvers with multigrid preconditioners}
{\bf Andrew Knyazev, Klaus Neymeyr} \\
Center for Computational Mathematics, University of Colorado at Denver,
P.O. Box 173364, Campus Box 170, Denver CO 80217-3364,
\verb9http://www-math.cudenver.edu/~aknyazev/9 \\
{\tt andrew.knyazev@cudenver.edu, neymeyr@na.uni-tuebingen.de}


We describe the Locally Optimal Block Preconditioned Conjugate
Gradient (LOBPCG) Method for symmetric eigenvalue problems,
based on a local optimization of a three-term recurrence.
We suggest using the same multigrid preconditioner in the LOBPCG method
for eigenproblems as that in the preconditioned conjugate
gradient method for the corresponding system of linear equations.
We provide new convergence rate estimates and
numerical results, which show effectiveness of such an approach.

A MATLAB code of the LOBPCG method is available at \\
\verb9http://www-math.cudenver.edu/~aknyazev/software/CG/9.

The talk is partially based on the papers:

\begin{itemize}

\item
Andrew Knyazev,
Toward the Optimal Preconditioned Eigensolver: Locally Optimal Block
Preconditioned Conjugate Gradient Method, Report
UCD-CCM 149, 2000, at the Center for Computational Mathematics,
University of Colorado at Denver (a revised version accepted to SIAM SISC).

\item
Klaus Neymeyr,
Solving mesh eigenproblems with multigrid efficiency
SFB 382, Report Nr. 157, October 2000

\end{itemize}



%	*******************************************
%	*************  korsawe.html  ************
%	*******************************************

\nextab
{Johannes Korsawe}
{Nonlinear Convergence via Linear Multilevel Performance: Gauss-Newton-Multilevel}
{University of Hannover, Dept. for Applied Mathematics, 30167 Hannover, Germany}
{jkorsawe@ifam.uni-hannover.de}


This talk is about multilevel techniques for solving a nonlinear
problem $f(u)=0!$ in a Hilbert space $H$, arising from a least squares
formulation
for a system of nonlinear partial differential equations
of first order.
After discretizing the space $H$, this is a minimization problem that may
be solved using a linear multilevel method for successive Gauss-Newton
problems or using a nonlinear multilevel method like FAS.

Although these approaches both yield nice numerical results, there are
some difficulties in the convergence theory.
Since the discretized problem is not consistent, the core problem turns
from finding a zero of $f(u)$
to finding a zero of $df/du$ which requires second order information about
$f$ for the theory.
One more problem for describing theory for the linear multilevel case is
the lack
of a direct relation between the error due to the discretization
and the algebraical error due to the iterative method in order to check
on overall convergence.
For the FAS-case, the Gauss-Newton method does not fit into existing
nonlinear multilevel theory with respect
to smoothing properties.

These problems disappear if the (inexact) Gauss-Newton solver is applied
directly to the infinite
dimensional problem in H and the emerging linear problems are
discretized and solved
using linear multilevel after that.
The problem remains consistent and the above-mentioned errors both
contribute to
the inaccuracy of the Gauss-Newton method.
Consequently, the number of linear multilevel sweeps may be monitored by
easily computable
error bounds via the multiplicative multilevel norm ensuring convergence
of the overall method.

In this talk, these results will be presented in more detail. Numerical
examples show the competitiveness of this method.




%	*******************************************
%	*************  kowarschik.html  ************
%	*******************************************

\section{Enhancing the Cache Performance of Multigrid Codes on Structured Grids}
{\bf Markus Kowarschik} \\
System Simulation Group, Dept. of Computer Science,
	University of Erlangen-Nuremberg, Germany.
	\verb9http://www10.informatik.uni-erlangen.de/~markus9
\\ {\tt markus.kowarschik@cs.fau.de}


Modern computer architectures use hierarchical memory designs in order
to hide the latencies of accesses to main memory components, which are
dramatically slow in contrast to the floating-point performance of the
CPUs. Current memory designs commonly involve several levels of cache
memories, which can be accessed up to a hundred times faster than main
memory.
It is well-known that efficient program execution can merely be achieved,
if the codes respect the hierarchical memory design.
Unfortunately, today's compilers are still far away from automatically
performing code transformations like the ones we apply in order to achieve
remarkable speedups.
As a consequence, much of this optimization effort is left to the programmer.

Another observation is that iterative methods, like e.g. multigrid, are
characterized by
successive sweeps over data sets, which - for representative problems in
science and engineering - are much too large to fit in cache.
Therefore, conservative implementations of such algorithms often reach
disappointing execution speeds, which are far away from the
theoretically available peak performances of the machines under
consideration.

In this paper we present techniques in order to enhance the cache
efficiency of multigrid methods for variable-coefficient problems
on regular mesh structures.
These techniques comprise both {\em data layout optimizations}
and {\em data access optimizations}.
Data layout optimizations are code transformation techniques which
change the data arrangement in memory, e.g. array padding and the
introduction of cache-aware data structures.
In contrast, data access optimizations modify the order in which
data are retrieved in the course of the computation. Loop blocking
(tiling), for example, belongs to the class of data access optimizations.

A variety of performance results are provided, showing both profiling data
and the speedups which can be achieved by applying these kinds of
optimization techniques.
It can be observed that in most cases speedup factors of 2-3 can be
obtained.

For further details, we would like to refer the reader to
\verb9http://wwwbode.in.tum.de/Par/arch/cache9,
the web page of our {\em DiME} project
(Data-local iterative methods for the efficient solution of
partial differential equations).


%	*******************************************
%	*************  kraus.html  ************
%	*******************************************

\nextab
{Van Emden Henson}
{Element-Free AMG Interpolation Based on Multilevel Extension Mappings}
{Center for Applied Scientific Computing, Lawrence Livermore National Laboratory, Livermore CA}
{}
\\
{\bf Johannes Kraus} \\
Center for Applied Scientific Computing, Lawrence Livermore
	National Laboratory, Livermore CA, and
	University for Mining and Metallurgy, Leoben, Austria
\\ {\tt kraus1@llnl.gov}


{\bf Panayot S. Vassilevski} \\
Center for Applied Scientific Computing, Lawrence Livermore
National Laboratory, Livermore CA


Recently a new general algorithm, denoted ``element-free AMGe,'' was
proposed [1], for constructing the interpolation weights in algebraic
multigrid.  The method uses an extension mapping to provide boundary
values outside a neighborhood about a fine-grid degree of freedom
(dof) to which interpolation is desired. The interpolated value is
obtained by a matrix-dependent harmonic extension of these boundary
values into the interior of the neighborhood. In essence, this method
is designed to capture information that can be obtained from
individual finite-element stiffness matrices, as is done in the
so-called ``element-based AMG method'' (AMGe) [2], for problems in which
such matrices may not be available.

The object of such a method is to characterize, on a local scale, the
nature of the smooth modes of the global operator, insuring that they
can be represented by the interpolation. We propose here a
modification to this method, in which we use a more global method of
localizing the nature of the smooth modes. In essence, the ``extended
neighborhood'' of the fine-grid dof is defined globally, by looking at
the coarse dofs on every level that feed information to the specified
fine dof.  The harmonic extension is then generated from a
``multilevel'' extended neighborhood, which can be constructed to allow
for recursive improvement of the interpolation operators as each
coarser level is addressed.

We describe the basic algorithm and methods of implementation, and
discuss some early experimental results.

This work was performed under the auspices of the U.S. Dept.
of Energy by University of California Lawrence Livermore National
Laboratory under contract No. W-7405-Eng-48.


\begin{enumerate}

\item
Henson, V. E., and Vassilevski, P. S.,
{\em Element Free AMGe: general algorithms for computing
interpolation weights in AMG}, to appear in
SIAM Journal on Scientific Computing.

\item
Brezina, M., Cleary, A. J., Falgout, R. D., Henson, V. E., Jones,
J. E., Manteuffel, T. A., McCormick, S. F., and Ruge, J. W.,
{\em Algebraic multigrid based on element interpolation (AMGe)},
SIAM Journal on Scientific Computing
{\bf 22}, 1570--1592, 2000.


\end{enumerate}



%	*******************************************
%	*************  kremenetsky.html  ************
%	*******************************************

\nextab
{Mark Kremenetsky}
{Considerations for Parallel CFD Enhancements on SGI ccNUMA and Cluster Architectures}
{Supercomputer Applications, Silicon Graphics, Inc.,M/S 41L-932, Mountain View, California 94043}
{mdk@sgi.com}


The maturity of Computational Fluid Dynamics (CFD) methods and
the increasing computational power of contemporary computers has
enabled industry to incorporate CFD technology in several stages
of design processes.  As the application of the CFD technology
grows from component level analysis to system level, the
complexity and the size of models increase continuously.
Successful simulation requires synergy between CAD, grid
generation and solvers.

The requirement for shorter design cycles has put severe
limitations on the turnaround time of the numerical
simulations.  The time required for (1) mesh generation for
computational domains of complex geometry and (2) obtaining
numerical solutions for flows with complex physics has
traditionally been the pacing item for CFD applications.
Unstructured grid generation techniques and parallel algorithms
have been instrumental in making such calculations affordable.
Availability of these algorithms in commercial packages has
grown in the last few years and parallel performance has become
a very important factor in the selection of such methods for
production work.

Although extensive research has been devoted in determining the
optimum parallel paradigm, in practice the best parallel
performance can be obtained only when algorithm and paradigms
take into consideration the architectural design of the target
computer system they are intended for.  This paper addresses the
issues related to efficient performance of the commercial CFD
software FLUENT (based on AMG linear solver) on a cache coherent
Non Uniform Memory (ccNUMA) Architecture.  Also presented are
results from implementation of FLUENT on cluster systems of
workstation for both the Linux and SGI IRIX operating systems.
Issues related to performance of the message passing system and
memory-processor affinity are investigated for efficient
scalability of FLUENT when applied to a variety of industrial
problems.


%	*******************************************
%	*************  lahaye.html  ************
%	*******************************************

\nextab
{Domenico Lahaye}
{A Multilevel Preconditioner for Field-Circuit Coupled Problems}
{Dept. of Computer Science, Celestijnenlaan 200A, B-3001 Heverlee Belgium}
{domenico.lahaye@cs.kuleuven.ac.be}


In simulating time-varrying magnetic fields in electromagnetic devices like
motors and transformers, it is often necessary to couple the partial
differential equation for the magnetic field with a model for the external
electrical circuit connections. The electrical circuit is a system of
linear equations relating the unknown currents and voltages of the
electrical conductors present in the device to known voltage and current
sources. Time-varrying sources give rise to magnetically induced currents
and voltages in the conductors. The partial differential equation and the
circuit are coupled by these magnetically induces quantities.



In low-frequent time-harmonic Maxwell formulations in two dimensions,
the partial differential governing the magnetic vector potential is the
Helmholtz equation with a complex shift. The finite element discretization
of this equation results in sparse, complex symmetric system matrices.
Discretized
field-circuit coupled problems yield two by two block structured matrices
whose diagonal is formed by the discretized partial differential equation
and the electrical circuit matrix. The size of the second diagonal block is
typically several orders of magnitude smaller than that of the first. The
coupling is performed in such a way that no fill-in occurs in the
discretized differential equation matrix and that the two by two
block matrices are again complex symmetric.


Solving the linear system is the computational bottleneck in simulating
technically relevant engineering problems. Motivated by previous experience
[1,2],
we want to alleviate this bottleneck by the application of algebraic multigrid
(AMG) techniques. The straightforward application of AMG is hampered by the
presence of the electrical circuit. We developed a multigrid cycle that takes
the circuit into account.


Our multigrid technique is a generalization of a method by Hackbush for
solving an elliptic problem augmented by an algebraic equation. We base the
AMG setup on the differential equation block of the matrix. The electrical
circuit is taking into account in the cycling phase. The resulting algorithm
is a black-box solver for general field-circuit coupled problems.


For the implementation of our multigrid technique, we developed an interface
between the GMD-AMG-code by Stueben and PETSc. This interace allows to call
the AMG setup on the differential equation block of the system matrix. After
the setup, the AMG coarser grid and interpolation operators are available as
PETSc matrices. The multigrid cycling is done by PETSc's multigrid components
that we extended to be able to treat the electrical circuit.


Our algorithm has been tested on a variety of engineering problems. It has
proven to be stable and to deliver a speedup by a factor between five and ten
compared to previously existing solvers.

\begin{enumerate}

\item
R. Mertens, H. De Gersem, R. Belmans, K. Hameyer, D. Lahaye,
S. Vandewalle and D. Roose,
{\em An Algebraic Multigrid Method for Solving Very
Large Electromagnetic Systems},
IEEE Trans. on Magn. {\bf 34} (5), 3327--3330 (1999).

\item
D. Lahaye, H. De Gersem, S. Vandewalle and K. Hameyer,
{\em Algebraic Multigrid for Complex Symmetric Systems},
IEEE Trans. on Magn. {\bf 36} (4), 1535--1538 (2000).

\end{enumerate}



%	*******************************************
%	*************  lee.html  ************
%	*******************************************

\nextab
{Lee}
{FOSLS for Neutron/Photon Transport at LLNL}
{Lawrence Livermore National Laboratory, CASC L-661, P.O. Box 808, Livermore CA 94551}
{lee123@llnl.gov}


We give an overview of our FOSLS codes for
neutron/photon transport at LLNL. A brief description of the Boltzmann
transport equation, the finite element discretizations, and the
numerical algorithms will be given. In particular, we describe
the least-squares finite element formulations (trilinear in space
and spherical harmonic in angle, trilinear in space and piecewise
constant in angle, trilinear/macro-element in space and spherical
harmonic in angle) and the multigrid algorithms implemented in
our codes. Numerical examples illustrating discretization accuracy,
multigrid scalability, and processor scalability will be presented.


%	*******************************************
%	*************  lehoucq.html  ************
%	*******************************************

\nextab
{Richard B. Lehoucq and Jeff Bennighof}
{Automated Multi-Level Substructuring: Theory}
{Sandia National Laboratories,
PO Box 5800 MS 1110,
Albuquerque NM 87185-1110
\\ and \\
The University of Texas at Austin,
Dept. of Aerospace Engineering and Engineering Mechanics,
W.R. Woolrich Laboratories (WRW),
Room 117D - Mail Code: C0600,
Austin TX 78712-1085}
{}

A new method for frequency
response analysis of very large finite element models of structures,
known as Automated Multi-Level Substructuring (AMLS),
has recently been developed.
Large FEM models typically having over one million degrees of freedom are
automatically subdivided into thousands of substructures. The response of the
structure is represented in terms of substructures eigenvectors. These
eigenvectors are much less expensive to obtain than the global eigenvectors
required for conventional modal frequency response analysis.

As a result, the computational cost of the
analysis is substantially reduced when AMLS is used in place of traditional
methods.
This presentation will give an
overview of an analysis of AMLS.



%	*******************************************
%	*************  macmillan.html  ************
%	*******************************************

\nextab
{Hugh R. MacMillan}
{First-order System Least-squares for Electrical Impedance Tomography}
{University of Colorado, Boulder}
{macmillh@colorado.edu}


The practical limitations of electrical impedance tomography
(EIT), due to both the necessarily finite set of inexact boundary
data and the diffusive nature of the current into the interior,
lead to the conclusion that reconstructing the interior impedance
is an ill-posed problem. Given a set of applied normal boundary
currents, a standard approach to EIT is to minimize the defect
between the known and the computed boundary voltages that are
associated, respectively, with the exact impedance and its
approximation. In minimizing a boundary functional, standard
least-squares approaches implicitly impose the interior partial
differential equation (PDE), the diffusion equation. We have
developed and implemented a first-order system least-squares
(FOSLS) formulation which incorporates the elliptic PDE in a
global multigrid minimization scheme. To place the new functional in
context, we establish its equivalence to existing least-squares
approaches, and to a novel norm on the error in the approximate
impedance. The effect of this equivalence is to guarantee that,
for each formulation, there is a unique minimizer in the topology
corresponding to this special norm.  Thus, the theory quantifies
the sense in which the fully nonlinear inverse problem is well
posed, and guides the discretization according to those components
which we can expect to reconstruct, regardless of the
computational objective. With this framework established, {\it a
priori} information, which might otherwise be used to regularize,
can be incorporated by introducing an additional term to the
functional.






%	*******************************************
%	*************  mandel.html  ************
%	*******************************************

\nextab
{Jan Mandel}
{Approximation and Coupling Estimators for Algebraic Multigrid}
{Center for Computational Mathematics, Dept. of Mathematics, University of Colorado at Denver, P.O. Box 173364, Campus Box 170, Denver CO 80217-3364}
{jmandel@math.cudenver.edu}


A-priori estimates of approximation of fine grid functions by
coarse grid functions are important for the design of robust
Algebraic Multigrid methods. A number of coarsening schemes will
work well on an easy problem, such as a the Laplace equation
discretized by linear elements on a reasonable unstructured
grid. Methods that incorporate rigid body modes, such as [1],
work also very well for elasticity. Realistic problems, however,
typically include elements violating shape limits, large jumps
of coefficients,  and special kinds of elements that destroy the
numerical relevance of  the underlying differential equations,
such as side constraints on the values of degrees of freedom,
enforced by large penalties (``stiff spring'' or ``contact''
elements in engineering parlance), or even arbitrary additional
equations that are eliminated before the matrix is passed to the
solver (``multiple point constraints''). Such problems are hard to
solve by Algebraic Multigrid even if they are symmetric and
positive semidefinite.  Without a-priori numerical estimates of
the rate of convergence, with  a rigorous foundation, an
Algebraic Multigrid algorithm is essentially based just on the
hope that the problem will not have anything unexpected and
things will work out in the end.

One common estimate that can be computed a-priori is the weak
approximation property, which bounds the error of the best
approximation in Euclidean norm of a fine grid vector by the
prolongation of a coarse grid vector in terms of the energy norm
of the fine grid vector. The weak approximation property is
known to imply a bound on the two-level convergence factor,
albeit a fairly pessimistic one. In the smoothed aggregation
method [1], two-level as well as multilevel convergence bounds
can be obtained [2] from the weak approximation property for the
so-called tentative prolongator, which is simply the transpose
of the matrix of a weighted aggregation of degrees of freedom.
The actual prolongation used in the multigrid algorithm is then
obtained by smoothing the tentative prolongation.


In the absence of multiple point constraints, the constant in
the weak approximation property can be bounded rigorously from
the solution of eigenvalue problems based on local element
matrices.  In [3], it was proposed to select the columns of the
tentative prolongator as the eigenvectors of the local problems
and to control the convergence of algebraic multigrid by
choosing the number of the eigenvectors and by selecting the
amount of smoothing of the prolongation.


In practice, the problem to be solved is most conveniently given
in terms of a single global stiffness matrix with all
constraints incorporated. Then the information contained in the
local stiffness matrices is lost, and to bound the constant in
the weak approximation property rigorously by the solution of
local eigenvalue problems, one would need to decompose the
global matrix into the sum of positive semidefinite local
matrices. We show that in general, such decomposition does not
exists if the dimension of the coarse space is more than two. To
estimate the contribution of a single aggregate (or,
equivalently, of a block of coarse basis functions) to the
constant in the weak approximation property, we decompose the
global matrix into local matrices corresponding to the
decomposition of the set of all nodes into the given aggregate
and its complement.  We also present further approximation
techniques to reduce the cost of the estimation.  The resulting
estimates are not rigorous but they are still practically
useful.

We show that bad constants in the weak approximation condition
will sometimes arize when the coarsening does not follow strong
couplings of the nodes. An analysis of the mechanism how this
happens shows when coarsening along weak couplings can still be
tolerated.  We observe that while strong couplings can be
usually quite well determined from the local stiffness matrices,
the computation of strong couplings from the global matrix is
not reliable.  We propose a new method to estimate the strength
of the coupling between two nodes from an approximate local
matrix corresponding to the pair of nodes.


\begin{enumerate}

\item
P. Vanek, J. Mandel, and M. Brezina,
{\em Algebraic Multigrid by Smoothed Aggregation for
Second and Fourth Order Elliptic Problems},
Computing {\bf 56} (1996) 179--196. \\
\verb9http://www-math.cudenver.edu/~jmandel/papers/meis.ps.gz9

\item
P. Vanek, M. Brezina, and J. Mandel,
{\em Convergence of Algebraic Multigrid Based on Smoothed
Aggregation}, UCD/CCM Report 126,
February 1998. Revised February 2000.
To appear in Numerische Mathematik.
\verb9http://www-math.cudenver.edu/~jmandel/papers/amg.pdf9

\item
M. Brezina, C. Heberton, J. Mandel, and P. Vanek,
{\em An Iterative Method with Convergence Rate Chosen a Priori},
UCD/CCM Report {\bf 140}, April 1999. \\
\verb9http://www-math.cudenver.edu/~jmandel/papers/bhmv.ps.gz9

\end{enumerate}



%	*******************************************
%	*************  mardal.html  ************
%	*******************************************

\nextab
{An efficient parallel iterative approach to the time-dependent Stokes problem}
{Kent-Andre Mardal}
{P.O. Box 1080, Blindern, N-0316 Oslo, Norway}
{kent-and@ifi.uio.no}



In this paper we consider an optimal multigrid/domain
decomposition preconditioner for the time-dependent Stokes
problem.  Preconditioners for this problem arise when using
fully implicit time stepping schemes for the Navier-Stokes
equations.  However, as the time stepping parameter decreases
against 0, the problem to be solved at each time step changes
from the Stokes problem to the mixed formulation of the Poisson
equation.  The same preconditioning techniques do not work in
both cases, even the finite elements typically used for Stokes
are not considered stable for the mixed Poisson equation.  We
will show that some typical Stokes elements are in fact stable
also for the Poisson equation in another norm, this leads us to
a proper preconditioner working uniformly in the time stepping
parameter.  Preconditioners for this problem have been studied
before in [Bramble, Pasciak 94], however our approach is
different.  In essence they constructed the preconditioner  by
assuming only that the finite elements satisfy the usual
Babuska-Brezzi condition for Stokes problem, requiring that the
time parameter $k<h^2$.  Our preconditioner is based on the
observation that the continuous time-dependent Stokes problem
satisfy  another Babuska-Brezzi condition, which also the Mini
element satisfy. We are therefore able to make a rather simple
preconditioner, where the time stepping parameter only enters
the preconditioner as a constant.  The efficiency of this
preconditioner will  be demonstrated by numerical experiments
done in parallel with  Diffpack, a C++ toolbox for finite
element simulations,  on a Beowulf cluster having roughly 50
CPUs.  It is established that the preconditioner works for the
Mini element.  Other elements like the Crouzeix-Raviart element
and the P2-P1 element will be considered.

This is joint work with Hans Petter Langtangen and  Ragnar Winther.

An extended abstract can be found
at \verb9http://www.ifi.uio.no/~kent-and/copper.ps9.




%	*******************************************
%	*************  mavriplis.html  ************
%	*******************************************

\nextab
{Dimitri Mavriplis}
{Comparisons of Unstructured Multigrid as a Non-linear Solver, a Linear Solver, or a Pre-Conditioner}
{ICASE MS 132C, NASA Langley Research Center, Hampton VA 23681}
{dimitri@icase.edu}


The relative efficiency of an unstructured multigrid
algorithm used as a non-linear solver (FAS)
is compared with the efficiency of the equivalent
multigrid algorithm used as a linear solver
on a Newton linearization of the non-linear problem,
and with the efficiency obtained
using either approach as a preconditioner
for a non-linear GMRES method.
Two types of problems are examined,
the solution of a transient radiation diffusion
problem, and the solution of the Navier-Stokes equations.
In the first case, the discretization employs
a nearest neighbor stencil and the linearized scheme
operates on an exact Newton linearization.
In the second case, the discretization is based
on a second-order form which involves neighbors of neighbors
while the linearization employed is based on a first-order
accurate nearest-neighbor stencil.
For the radiation diffusion case, the linear multigrid
approach is more efficient than the FAS approach,
mainly due to the expense involved in the more frequent
computation of the highly non-linear residuals required
in the non-linear multigrid case.
When the linear system is solved to sufficient
tolerance, quadratic convergence of the non-linear problem
is observed.
For the Navier-Stokes equations, quadratic convergence
of the non-linear problem is not possible, since an
inexact linearization is employed.
While a linear multigrid iteration is shown
to be substantially cheaper than the corresponding
non-linear multigrid iteration, the relative performance
of both methods depends largely on the tolerance
to which the linear system is solved, and the achievable
non-linear convergence rate.
Using either method as a preconditioner
rather than as a solver is also shown to provide
additional convergence acceleration.
Other issues such as memory requirements
and robustness of the various schemes will also be discussed.





%	*******************************************
%	*************  medeke.html  ************
%	*******************************************

\nextab
{Bj\"orn Medeke}
{On Algebraic Multilevel Preconditioners for Disordered Systems}
{Dept. of Mathematics, University of Wuppertal, D-42097 Wuppertal, Germany}
{medeke@math.uni-wuppertal.de}


Based on a Schur complement approach we present an algebraic multilevel
preconditioning technique for large, sparse and totally disordered systems.


We focus on the Schwinger model which is an ideal laboratory to
develop and to test preconditioners for discretizations of lattice
fermions since it has many features in common with lattice QCD. The
corresponding Schwinger matrices represent a nearest neighbor coupling
on a regular two-dimensional lattice. In general, the Schwinger matrices
are large, sparse and structured but totally disordered.



In contrast to the original system, an initial odd-even reduction enables us
to distinguish between strong and weak edges in the digraph induced by the
odd-even reduced system. The F-sites are chosen to be a maximal independent
vertex set of the reduced digraph consisting of the vertex set and the set
of strong edges.  Although the F-sites are selected algebraically,
the resulting F/C partitioning of the even lattice sites can be easily fixed
in advance such that costly coarsening strategies become obsolete.



To precondition the subsystem residing on the F-sites only,
standard incomplete LU preconditioners (ILU) are considered.
Rigorous bounds of the condition number of the preconditioned
submatrix are given. An appropriate and easy to construct Schur
complement approximation is proposed. With respect to the
odd-even reduced system no fill-in occurs so that the Schur
complement preconditioner allows a recursive procedure on coarse
lattices.


The algebraic multilevel preconditioner is compared with standard
odd-even preconditioning. Numerical experiments indicate that the
Schur complement preconditioner outperforms the standard odd-even
preconditioner.


%	*******************************************
%	*************  moulton.html  ************
%	*******************************************

\nextab
{David Moulton, Misha Shashkov}
{An Augmented Systems Approach to Preconditioning Mimetic Discretizations}
{Mathematical Modeling and Analysis Group, Theoretical Division, Los Alamos National Laboratory, Los Alamos NM 87545}
{moulton@lanl.gov}

{\bf Jim Morel} \\
Transport Methods Group,
Computer and Computational Sciences Division,
Los Alamos National Laboratory,
Los Alamos NM 87545



The diffusive component in many important application areas is
characterized by a discontinuous diffusion coefficient with fine-scale
anisotropic spatial structure; moreover, the underlying grid may be
severely distorted.  Thus, increasingly mixed discretizations, which
include mixed finite element methods (FEM) and support operator
methods (SOM), are employed because they explicitly enforce important
physical physical properties of the problem.  Unfortunately, these
discretizations are based on the first order form, and hence naturally
lead to an indefinite linear system.  In the case of orthogonal grids
and a diagonal diffusion tensor optimal preconditioners have been
developed.  However, the performance of these methods degrades with
increasing full tensor anisotropy or grid distortion.  Moreover,
the development of alternative preconditioners that are robust
with respect to general forms of anisotropy has been thwarted by
a significant loss of sparsity.

In this work we are motivated by one specific advantage that the
hybrid or local forms of mixed discretizations exhibit, namely, their
more localized sparsity structure.  Specifically, for the SOM we
consider augmentation of the flux (i.e., vector unknowns) such that an
appropriate ordering of the augmented flux leads to a new block
diagonal system for this component.  In contrast to the block diagonal
structure of the hybrid system this system has blocks centered about
vertices, and block elimination of the flux (i.e., formation of the
Schur complement) leads to a symmetric positive definite scalar
problem with a standard cell-based 9-point structure (in two
dimensions).  This reduced system is readily solved with existing
robust multigrid methods, such as Dendy's black box multigrid.  An
analogous approach is used to augment the hybrid system and derive the
equivalent preconditioner for this case.  We demonstrate the
effectiveness of this preconditioner for both GMRES iterations of the
full indefinite system as well as a CG iterations of the reduced
scalar problem.



%	*******************************************
%	*************  naegele.html  ************
%	*******************************************

\nextab
{Sandra N\"agele}
{Multigrid Method coupled with Large Eddy Simulation}
{Interdisciplinary Centre for Scientific Computing, University of Heidelberg, Im Neuenheimer Feld 368, 69120 Heidelberg, Germany}
{sandra.naegele@iwr.uni-heidelberg.de}




The use of multigrid methods in conjunction with a Large Eddy
Simulation (LES) is very proximate since both are based on
multiple scales. LES is a turbulence model which resolves large
turbulent scales and models the small ones. The scale separation
is performed by applying a spatial convolution operator (filter
operator) to the incompressible Navier-Stokes equations. As
filter operator a top hat filter is applied with a grid
dependent support size. Hence the application of the filter
results in a locally varying average in space. The LES model
itself also depends on the grid size since the filtering process
removes all subgrid scales. Some special dynamic LES models have
been developed by various researchers which apply two filters
with different support size at each point of the domain to the
governing equation system. By comparison of the two different
large scale resolutions the model term can be specified locally.
This is similar to the multigrid cycle where the defect is
restricted to the coarser grid and higher frequencies are
removed.  Another property of dynamic models is their ability to
adjust themselves to local flow structures. This adaptivity is
very useful since in some regions of the domain the flow can be
laminar and a turbulence model is not necessary at all. Hence
dynamic models were used in the simulations.


The simulations were carried out on the software platform
{\bf UG} which is based on unstructured grids
(see \verb9http://cox.iwr.uni-heidelberg.de/~ug9).
Thus the filter width varies in the
solution domain. Using unstructured grids the resolution of the
turbulent scales can be increased locally to decrease the
modeling effort. In the neighbourhood of walls for example it
is possible to use a smaller grid size. By this grid adaptation
the local flow structures can be better resolved and less
modeling with less modeling error is necessary.


A Krylov subspace method with linear multigrid as preconditioner
is used to solve the linearized system. Within the multigrid
cycle it is important to separate modeling and solving in the
sense that only on the finest grid the modeled part of the
equations is determined as described above. This is necessary
for an appropriate modeling of the subgrid turbulent scales.
Afterwards the model term is restricted to the coarse grids and
a standard linear multigrid cycle can be used.  This solution
strategy was applied to different flow problems which will be
presented.


%	*******************************************
%	*************  oosterlee.html  ************
%	*******************************************

\nextab
{C.W. Oosterlee}
{On multigrid methods for linear complementarity problems with application to American-style options}
{GMD-SCAI, Schloss Birlinghoven, 53754 Sankt Augustin, Germany}
{oosterlee@gmd.de}


In this talk, we will discuss multigrid methods for solving
time-dependent 2D partial differential equations (PDEs)
arising in option pricing theory.  The problem considered
is the computation of the value of an option of
American-style in a stochastic volatility setting. It leads
to the solution of a convection-diffusion type PDE
with a free, beforehand unknown boundary.
In [4] it has been shown that
for American-style options the theory of linear
complementarity, as it is developed in the 1970-ties
applies.  It is possible to rewrite the arising free
boundary problems to linear complementarity problems
(LCPs), of the form
\centerline{(equation)}
plus boundary
conditions, where $L$ is a differential operator.  This
formulation is beneficial for iterative solution, since the
unknown boundary does not appear explicitly and can be
obtained in a postprocessing step.  A slight disadvantage
of this formulation is that the linear partial differential
equation has been transformed into a nonlinear partial
differential inequality.


In 1983, Brandt and Cryer [1] proposed a multigrid method
for LCPs arising from free boundary problems.  The
algorithm is a multigrid generalization of the projected
SOR method. Due to the nonlinear character of the problem,
the multigrid method is based on the full approximation
scheme, FAS.  The solution method has therefore been called
the projected full approximation scheme, PFAS in [1].  In
the original paper, the operator $L$ in [1,3] was the
nicely elliptic Laplace operator and fast convergence was
presented.  PFAS has been successfully used in the
financial community for American options with stochastic
volatility in [2].  The smoother applied in [2] is,
however, somewhat involved.  It is based on the pointwise
PSOR method for the detection of the free boundary,
followed by a linewise version in order to deal with the
stretched numerical grid occurring in the financial
problem.  The need to change multigrid components like the
smoother for optimal convergence of a new problem at hand
is sometimes considered as unsatisfactory.

It is our aim to give some more insight in the multigrid
convergence for the problems from option pricing. At the
same time, we will introduce some recent developments in
multigrid techniques to the field of LCPs, making the
algorithms more robust, i.e. less sensitive to parameter
changes.

In the search for efficient solvers that are more generally
applicable, we consider multigrid as a preconditioner for
Krylov subspace methods.  Often different complications
such as anisotropies, nonlinearities or strong positive
off-diagonal stencil elements occur simultaneously, as is
the case in the discretization of the PDEs arising from
American-style option pricing, for example discretized by a
higher-order upwind scheme.

The fundamental idea of multigrid, to reduce the high
frequency components of the error by smoothing procedures
and to take care of the low frequency error components by
coarse grid corrections, might not work optimally if
straightforward multigrid approaches are used.  Certain
error components may remain large since they cannot be
reduced by pointwise Gauss-Seidel-type smoothing procedures
combined with a standard coarse grid approximation. These
specific error components (and the corresponding
eigenvectors/eigenvalues) are then responsible for the poor
multigrid convergence.

A commonly known solution approach for {\em nonlinear
equations} is to apply global (Newton) linearization,
solve the arising linear system with a Krylov subspace
method, like GMRES or BiCGSTAB and a multigrid
preconditioner.  This is not the approach followed here.
We apply to nonlinear problems a solution method based on
PFAS as the multigrid technique.  The Krylov subspace
acceleration can be interpreted as a technique, in which
intermediate iterants are recombined in order to obtain an
improved approximation, see, for example, [3].  We will
generalize this method to solving LCPs.

\begin{enumerate}

\item
A. Brandt and C.W. Cryer,
{\em Multigrid algorithms for the solution of linear
complementarity problems arising from free boundary problems},
SIAM J. Sci. Comput {\bf 4}, 655--684, 1983.


\item
N. Clarke and K. Parrot,
{\em Multigrid for American option pricing with stochastic
volatility}, Appl. Math. Finance {\bf 6}, 177--197, 1999.

\item
U. Trottenberg, C.W. Oosterlee and A. Sch\"uller,
{\em Multigrid}, Academic Press, London,  2000.

\item
P. Wilmott, J. Dewynne and S. Howison,
{\em Option pricing}. Oxford Financial Press, 1993.

\end{enumerate}




%	*******************************************
%	*************  park.html  ************
%	*******************************************

\nextab
{K. C. Park, Carlos A. Felippa  and Gert Rebel}
{Localized Construction of Non-Matching Interfaces That  Pass A-Priori  Patch Test}
{Center for Aerospace Structures and Dept. Aerospace Engineering Sciences, University of Colorado, Boulder CO 80309-0429}
{kcpark@titan.colorado.edu}


Interface coupling of independently modeled structural and other
multiphysics computational modules is emerging as a key procedure for
local refinements, contact-impact modeling, and parallel simulation
employing partitioned domains. In the first part of this paper, we
present a localized version of the method of Lagrange multipliers whose
key feature  is the use of frames that
lie between the partitioned domains and that facilitate localized
interface coupling as opposed to global coupling inherent when employing
the classical method of Lagrange multipliers.

In the second part of the paper we present a selection of its
applications to the coupling of non-matching interfaces and contact
problems. In particular, we present an interface patch test criterion
that can be employed a priori for the construction of patch test-passing
interfaces.  The resulting coupling procedure satisfies a priori the
interface patch test when the interfaces are discretized with different
non-matching meshes. Example are finally discussed, which demonstrate the
utility of the present method as compared to the master-slave method, the
mortar method and the so-called three-field method.



%	*******************************************
%	*************  pasciak.html  ************
%	*******************************************

\nextab
{Joseph Pasciak, R.D. Lazarov, J. Schoberl and P. Vassilevski}
{An penalty approach for elliptic problems on non-matching grids}
{Dept. of Mathematics, Texas A\&M University, College Station TX 77840}
{pasciak@math.tamu.edu}


In this talk I will consider approximations to second order elliptic
problems on non-matching grids which utilize penalty terms to
impose interface continuity.  This approach avoids the introduction and
construction of multiplier subspaces commonly associated with mortar
finite element techniques.
I will present results which
show that in the case of piecewise linear finite elements,
the penalty formulation leads to almost optimal approximation.


\verb9http://www.math.tamu.edu/~pasciak9



%	*******************************************
%	*************  pflaum.html  ************
%	*******************************************

\nextab
{C. Pflaum}
{Discretization of Neutron Transport by FOSLS and Piecewise Constant Functions on the Sphere}
{}
{}


The approximation of the function space on the sphere
plays an important role in the discretization of transport equation.
We will present a discretization that employs
piecewise constant functions on the sphere and the FOSLS formulation
of the Boltzmann transport equation. The discrete equation system is
solved by a multigrid process in space and on the sphere.
Numerical results show an amelorization of the ray effect.
For the implementation of the discretization and the multigrid
algorithm, we used the parallel version of the library EXPDE
(Expression Templates for Partial Differential Equations).


This work was performed under the auspices
of the U.S. Dept.  of Energy by University of
California Lawrence Livermore National Laboratory
under contract no. W-7405-Eng-48.




%	*******************************************
%	*************  philip.html  ************
%	*******************************************

\nextab
{Bobby Philip}
{Elliptic Solvers with Adaptive Mesh Refinement and First-Order System Least-Squares (FOSLS) Methodologies}
{L-551, Center for Applied Scientific Computing, Lawrence Livermore National Laboratory, Livermore CA 94550}
{bobbyp@llnl.gov}


The talk will describe current ongoing work on developing
elliptic solvers on adaptively refined curvilinear coordinate
grids. First-Order System Least-Square formulations are used for
elliptic systems to facilitate the development of reliable local
error estimators. These error estimators will be used to
determine where to place local fine grids during the adaptive
mesh refinement process.  The class of Fast Adaptive Composite
Grid (FAC) algorithms will be described. Details of implementing
AFACx, an asynchronous version of FAC, on adaptively refined
curvilinear grids will be described.



%	*******************************************
%	*************  poole.html  ************
%	*******************************************

\nextab
{Gene Poole and Yong-Cheng Liu}
{Advancing Analysis Capabilities in ANSYS through Solver Technology}
{ANSYS, Inc., 275 Technology Drive, Canonsburg PA 15317}
{}

This talk describes substantial improvements in analysis
capabilities in a large scale commercial finite element program
made possible by the implementation of solver technology.  The
ANSYS program is a commercial finite element analysis program
which has been in use for thirty years.  The original code,
developed around a direct frontal solver has been expanded over
the years to include full featured pre- and post- processing
capabilities which support a comprehensive list of analysis
capabilities including linear static analysis, multiple
nonlinear analyses, modal analysis and many other analysis
types.  The finite element models on which these analyses are
used have continued to grow in size and complexity.  This growth
in size and complexity has been both enabled by and dependent on
new solver technology along with increased computer memory and
CPU resources.  Beginning in 1994 ANSYS added a jacobi
preconditioned conjugate gradient solver (JCG) and subsequently
an incomplete cholesky preconditioned conjugate gradient solver
(ICCG) to improve thermal analysis capabilities.  In recent
years the addition of the Boeing sparse matrix library for modal
and static analysis, and a proprietary preconditioned conjugate
gradient solver as well as additional iterative solvers to
support new CFD capabilities have greatly increased the number
of solver options available in ANSYS.  Most recently, in version
5.7, ANSYS has added a new domain solver for solving very large
structural analysis solutions on distributed MPI-based computer
systems and the newest iterative solver option, an algebraic
multi-grid iterative solver (AMG).

This paper will describe implementation considerations for the
addition of new solver technology to a large legacy code,
compare resource requirements for the various solver choices and
present some comparative results from several customer generated
problems.  The AMG solver benefits, both in improved robustness
and parallel processing efficiency will be described.  The paper
will also discuss some of the implementation challenges that
have been overcome to add new solver technology to a large
existing code.  The role of solver technology in meeting current
and future demands of large scale commercial analysis codes will
be discussed.


%	*******************************************
%	*************  ruede.html  ************
%	*******************************************

\nextab
{Ulrich Ruede}
{Multigrid Solution of Bioelelectric Field Problems}
{Dept. of Applied Mathematics, Univ. of Colorado, Boulder CO 80309-0526
\\ (permanent address:
Lehrstuhl fuer Informatik X,
Universitaet Erlangen-Nuernberg,
Cauerstr. 6,
D-91058 Erlangen, Germany)}
{ruede@cs.fau.de}

{\bf C.R. Johnson, A. Samsonov, K. Zyp} \\
Scientific Computing and Imaging Institute, University of Utah

{\bf Marcus Mohr} \\
Lehrstuhl fuer Informatik X, Universitaet Erlangen-Nuernberg, Cauerstr. 6, D-91058 Erlangen, Germany


The reconstruction of bioelectric fields from non-invasive measurements
can be used as a powerful new diagnostic tool in cardiology and neurology.
Mathematically, the reconstruction of a bioelectric field can be modeled
as an inverse problem for a potential equation. This problem is ill-posed
and requires special treatment, in particular either regularization or an
otherwise suitable restriction of the solution space.


The differential equation itself can be discretized by finite differences or
finite elements and thus gives rise to large sparse linear systems for
which multigrid is one of the most efficient solvers, but regularization,
adaptive mesh refinement, and efficient solution techniques must be
combined to solve the inverse bioelectric field problem efficiently.
While
multigrid algorithms can reduce the compute times substantially, new local
regularization techniques can be used to improve the quality of the
reconstruction. Local mesh refinement can be used to increase the
resolution in domains of increased activity, but must be used with care
because refined meshes worsen the ill-conditioning of the inverse problem.



%	*******************************************
%	*************  ruge.html  ************
%	*******************************************

\nextab
{John Ruge}
{AMG: An Overview}
{1005 Gillaspie Dr., Boulder CO  80305}
{jruge@colorado.edu}

Over the years, there have been many variants and different approaches
to algebraic multigrid methods. In this talk, we attempt to describe
and classify a number of these. This is not meant to be all-inclusive,
but is a start of a framework for classification and comparison of
AMG variants, both new and old.


%	*******************************************
%	*************  schoeberl.html  ************
%	*******************************************

\nextab
{Joachim Schoeberl}
{Algebraic Multigrid for H(curl) and H(div)}
{Dept. of Mathematics, Texas A\&M University, College Station, Texas 77843-3368}
{joachim@math.tamu.edu}

{\bf Stefan Reizinger} \\
SFB013 ``Scientific Computing'', University Linz, 4020 Linz, Austria
\\ {\tt reitz@sfb013.uni-linz.ac.at}



There exists a variety of algebraic multigrid methods for elliptic
problems in Sobolev spaces. Recently, Arnold, Falk and Winther and
Hiptmair presented geometric multigrid methods for elliptic problems
in the spaces H(curl) and H(div). Their new multigrid components are
special smoothers, the grid transfer is canonical since the spaces are
nested. Their analysis is much more involved then the multigrid
estimates for problems in Sobolev spaces. The difficulty occurs since
the operator leads to a regularity pick-up only on a part of the
space. They overcome the difficulty by splitting the space due to the
discrete Helmholtz decomposition.  Heavily exploring the commuting
diagram property, they can approximate both components independently.



In our talk we will present a technique to maintain the commuting
diagram property during an algebraic coarsening process. The key is to
identify the geometric entities node, edge and face, and keep
their relationships on the coarser levels. As usual, we can give a two
level analysis, and present a couple of numerical results
demonstrating the good multigrid behavior. The algorithm is
implemented into Stefan Reitzinger's AMG code {\bf PEBBLES}
(\verb9http://www.sfb013.uni-linz.ac.at/~reitz/pebbles.html9).



%	*******************************************
%	*************  schorsch.html  ************
%	*******************************************

\nextab
{Jens Georg Schmidt}
{FOSLL* for Stokes and Elasticity}
{Dept. of Applied Math, University of Colorado, Campus Box 0526, Boulder CO 80309-0526}
{schorsch@colorado.edu}

The standard FOSLS Method only works well if the exact solution
of the problem is in $H^2$.
For many practical applications of Stokes or Elasticity this is not
true (Mixed BCs, reentrant corners, etc.). In our talk we
present the FOSLL* Method, which is an FOSLS-like method,
constructed to overcome this drawback.

We will present numerical results for Stokes and linear Elasticity, which
show optimal FE- and MG-convergence of the method.


%	*******************************************
%	*************  schweitzer.html  ************
%	*******************************************

\nextab
{Michael Griebel, Marc Alexander Schweitzer}
{A Multilevel Particle-Partition of Unity Method}
{Dept. of Applied Mathematics, Division for Scientific Computing and Numerical Simulation, Wegelerstr. 6, D-53115 Bonn, Germany}
{griebel@iam.uni-bonn.de, schweitz@iam.uni-bonn.de}


In this paper we present a multilevel solver for Galerkin
discretizations of elliptic PDEs using so-called meshless
methods (MM) [2, 3, 5, 6, 7, 8].  Meshless methods are promising
approaches to overcome the problem of mesh generation which
still is the most time-consuming part of any finite element (FE)
simulation. Meshless methods are based only on a (finite)
collection of independent points within the domain of interest,
i.e. there are no fixed connections between any two points like
in a conventional mesh.

In the design of our multilevel solver, we utilize a
hierarchical cover construction algorithm [2] for
Partition of Unity Methods, {\bf PUM}
(see \verb9http://wissrech.iam.uni-bonn.de/research/projects/schweitz/pum/9),
which induces a hierarchy
of nonnested PUM function spaces and a variational approach to
the approximation problem arising in the construction of
interlevel transfer operators.  The shape functions of our PUM
are piecewise rational functions and non-interpolatory, due to
the meshless construction which uses only a set of irregularly
spaced points with no fixed connection between them (unlike in a
mesh).  Hence, the usual interpolation approach to the
construction of interlevel transfer operators is not applicable.
One way of tackling this problem is the use of $L2$-projections as
interlevel transfer operators.  The linear systems arising from
a PUM discretization are sparse block systems, due to the fact
that the PUM shape functions are products of a partition of
unity function ($h$-component) and a higher order approximation
function ($p$-component).  Therefore, we can choose the second
main ingredient of our multilevel solver, the smoothing
operators, to utilize this block structure.  Here, we choose
block smoothers which operate on p-component blocks of the PUM
function space and therefore eliminate the p-dependence of the
convergence rate.

Since the meshless construction leads to a sequence of nonnested
function spaces, the variational assumption
$$
a_k\,({\bf I}_{k-1}^k \phi_{k-1}, {\bf I}_{k-1}^k \psi_{k-1}) \not= a_{k - 1}\,(\phi_{k-1}, \psi_{k-1})
$$
is no longer valid (even if we assume that the bilinear forms
are the same on all levels)
and the standard theory for variational multilevel algorithms
is not applicable. Along the lines of a general multilevel
theory presented in [9]
the (uniform) convergence of the V and W cycles can be shown if
$$
a_k\,({\bf I}_{k-1}^k \phi_{k-1}, {\bf I}_{k-1}^k \psi_{k-1}) \leq C a_{k - 1}\,(\phi_{k-1}, \psi_{k-1})
$$
holds (in addition to further regularity assumptions).
The results of our numerical experiments show that this is indeed
the case for our method (for $L = -\Delta + id$).
The V(1,1) cycle converges independently of the number of points
$N$
(which are generated by a Halton sequence) and the polynomial
degree $p$ used during the discretization of the PDE.
The average error reduction factor is independent
of the number of points $N$ and the polynomial degree $p$
between 0.1 and 0.25 for the V(1,1) cycle, i.e.
$$
0.2 \sim \rho \not= \rho\,(N, p)
$$
using block Gauss-Seidel smoothing and about 0.5 for block Jacobi smoothing.
Furthermore, these results also hold for highly irregular
point sets, which correspond to adaptive discretizations.
In summary, our multilevel Partition of Unity Method
converges independently of the polynomial degree
of the discretization and independently of the
number and position of the discretization points.

The overall computational costs C of our multilevel algorithm
is bounded by the number of points and the costs for the
p-block inversion, i.e. its worst case complexity in d dimensions is
$$
C = O(Np^{3d})
$$

In some cases, the computational costs C may even be reduced to
$$
C = O(Np^{d})
$$
leading to a multilevel solver for an hp discretization which would be
only linearly dependent on the total number of degrees of freedom.


\begin{enumerate}

\item M. Griebel and M. A. Schweitzer.
{\em A Particle-Partition of Unity Method:
Part III Parallelization}, in preparation.


\item M. Griebel and M. A. Schweitzer.
{\em A Particle-Partition of Unity Method:
Part II Efficient Cover Construction and Reliable
Integration}, in preparation.


\item M. Griebel and M. A. Schweitzer.
{\em A Particle-Partition of Unity Method for the solution
of Elliptic, Parabolic and Hyperbolic PDE}
SIAM J. Sci. Comp. {\bf 22}(3), 853--890, 2000.
Also as SFB Preprint 600, SFB 256, Institut fr Angewandte
Mathematik, Universitt Bonn.


\item A. Caglar, M. Griebel, M. A. Schweitzer, and G. Zumbusch,
{\em Dynamic load-balancing of hierarchical tree algorithms on a
cluster of multiprocessor PCs and on the Cray T3E},
in H.W.~Meuer, editor,
{\em Proceedings 14th Supercomputer Conference, Mannheim},
ISBN 3-932178-08-4, Mannheim, Germany, 1999.
Mateo.  SuParCup '99 Award Winning Paper,
also as SFB 256 report 27.


\item M. A. Schweitzer.
{\em Ein Partikel-Galerkin-Verfahren mit
Ansatzfunktionen der Partition of Unity Method},
Diplomarbeit, Institut fr Angewandte Mathematik,
Universitt Bonn, 1997.


\item I. Babuska and J. M. Melenk.
{\em The Partition of Unity Finite Element Method:
Basic Theory and Applications}, Comput. Meths.
Appl. Mech. Engrg., Special Issue on Meshless Methods 1996
{\bf 139}, 289--314.


\item C. A. M. Duarte and J. T. Oden,
{\em Hp clouds - A Meshless Method to solve Boundary
Value Problems}, Num Meth. for PDE {\bf 12}
(1996), 673--705.
Also as Tech Rep 95-05, TICAM, University of Texas, 1995.


\item C. A. M. Duarte, I. Babuska and J. T. Oden,
{\em Generalized Finite Element Methods for Three
Dimensional Structural Mechanics Problems},
Computers and Structures {\bf 77} (2000), 215--232.


\item J. H. Bramble, J. E. Pasciak and J. Xu.
{\em The Analysis of Multigrid Algorithms with
nonnested Spaces or noninherited quadratic Forms},
Math. Comp. {\bf 56} (1991), 1--34.


\end{enumerate}


%	*******************************************
%	*************  silva.html  ************
%	*******************************************

\nextab
{Malik Silva}
{Cache Aware Data Layouts}
{Dept. of Statistics and Computer Science, University of Colombo, Colombo, Sri Lanka}
{msilva@mail.cmb.ac.lk}


Feeding the CPU with data operands is the bottleneck in many
scientific computations.  This bottleneck is alleviated by means
of caches, small fast memories to keep data.  The performance of
a memory-intensive computation depends critically on whether
most of the data accesses can be performed within the cache.
This research is on cache aware computing, in which the
programmers make their code cache friendly.  In particular, we
have tested cache aware data laying which is a promising
technique as it gives significant performance improvements.  For
example, we recorded an improvement of about 64\% over the
standard matrix multiplication.


%	*******************************************
%	*************  stals.html  ************
%	*******************************************

\nextab
{Linda Stals}
{The Parallel Solution of Radiation Transport Equations}
{Dept. of Computer Science, Old Dominion University, Norfolk VA 23529-01
\\ and \\ ICASE, NASA Langley Res. Ctr, Hampton VA 23681-2199}
{stals@icase.edu}


Radiation transport equations arise in the study of many different fields,
such as combustion, astrophysics and hypersonic flow. The solution of
these equations presents interesting challenges due to large jumps in the
coefficients and strong non-linearities. In this talk we plan to compare
the parallel efficiency of several techniques that may be used to solve
these non-linear equations.

Under certain physical assumptions, such
as isotropic radiation, optically thick material and temperature
equilibrium, radiation transport may be modeled by a system of three
non-linear equations. As a first approach we consider the case where all
energies are in equilibrium and the system is reduced to a single highly
non-linear equation. The radiation may travel through inhomogeneous
material, which translates into large jumps in the coefficients.
Furthermore, a flux limiter is often included and it changes the equations
from being locally parabolic to hyperbolic. As a consequence of all of
these features, the equations are very difficult and time
consuming to solve.

To be able to solve the types of problems we are
interested in we need to look at algorithms that show both parallel
scalability and algorithmic scalability, ignoring one of these aspects
means that we can not fully exploit available resources and will fall
short of our goals. Techniques such as the multigrid method and adaptive
finite elements have good algorithmic scalability, we want to measure
their parallel scalability in relation to the radiation transport
equations. Specifically, we shall focus on the performance of an inexact
Newton multigrid scheme and compare it with the Full Approximation Scheme
(FAS). Furthermore, we will also look at how the use of adaptive
refinement affect the solution time.



%	*******************************************
%	*************  tiesinga.html  ************
%	*******************************************

\nextab
{Eite Tiesinga}
{Multigrid Modeling of Two Confined and Interacting Atoms}
{National Institute of Standards and Technology, 100 Bureau Drive stop 8423, Gaithersburg MD, 20899-8423}
{eite.tiesinga@nist.gov}

{\bf William F. Mitchell} \\
National Institute of Standards and Technology,
100 Bureau Drive stop 8910,
Gaithersburg MD, 20899-8910
\\ {\tt william.mitchell@nist.gov}


The 1994 discovery by P. Shor of a polynomial quantum factoring
algorithm has started initial efforts into the development of quantum
computers.  This opened new avenues of research for mathematicians,
computer scientists, and physicists alike.  The quantum paradigm of
computing induced renewed interest in complexity classes, (quantum)
error correction, searches for other algorithms, and the search for
physical implementations of the basic building blocks,
such as qubits and quantum gates, of a quantum computer.


We have applied multigrid methods to solve for a two-dimensional
Schr\"odinger equation in order to study the feasibility of a quantum
computer based on extremely-cold neutral alkali-metal atoms.  Qubits are
implemented as motional states of an atom trapped in a single well of
an optical lattice of counter-propagating laser beams.  Quantum gates
are constructed by bringing two atoms together in a single well leaving
the interaction between the atoms to cause entanglement.  For special
geometries of the optical lattices and thus shape of the wells, quantifying
the entanglement reduces to solving for eigenfunctions of a Schr\"odinger
equation that contains a two-dimensional Laplacian, a trapping potential
that describes the optical well, and a short-ranged interaction potential.


Length and energy scales of the trapping potential and interaction
potential are too disparate to allow for a naive discretization of
the differential operator. The size of the resulting linear systems would
prohibit reasonable calculations. An adaptive grid approach automatically
finds a spatial representation that is refined when the atoms are close
together and the interaction potential is very strong and is coarse
at larger separation where the trapping potential is the largest
energy scale.  The finite element discretization of the Schr\"odinger
equation on this grid produces an eigenproblem with the matrix
$A^{-1}B$ where $A$ is the usual finite
element discretization of a second order linear elliptic operator and
$B$ contains the $L_2$ inner products of the
finite element basis functions.  Selected eigenvalues and eigenvectors
are computed by an iterative method that requires only multiplication
of a vector by the matrix.  Multiplication by $A^{-1}$
is achieved by solving a linear system with the matrix $A$ by a
multigrid method.




%	*******************************************
%	*************  toivanen.html  ************
%	*******************************************

\nextab
{Janne Martikainen, Tuomo Rossi, Jari Toivanen}
{Multilevel Preconditioners for Boundary Lagrange Multipliers}
{Dept. of Mathematical Information Technology,
University of Jyv\"askyl\"a,
P.O. Box 35 (Agora), FIN-40351 Jyv\"askyl\"a, Finland,
http://www.mit.jyu.fi/en/}
{jamartik@mit.jyu.fi, tro@mit.jyu.fi, tene@mit.jyu.fi}


By embedding the original elliptic boundary value problem into a larger
simple-shaped domain and treating the Dirichlet boundary conditions by
using boundary Lagrange multipliers, the discretization and efficient
solution can become much easier. The generation of hierarchical meshes
required by the traditional multigrid methods for complicated domains
can be very difficult or even impossible. Furthermore, it much easier
to make fast cache aware multigrid implementations for simple-shaped
domains like rectangles. The use of Lagrange multipliers leads to
saddle-point problems which are here assumed to be symmetric.
Such problems can be solved using a preconditioned MINRES method.
A simple and natural idea is to construct separate preconditioners
for primal variables and Lagrange multipliers. Thus, the overall
preconditioner has a block diagonal form. The preconditioner for
the primal variables can be based on any symmetric multigrid method or,
for example, on fast direct solvers. It is well-known that a good
preconditioner for the Lagrange multipliers should approximate the
Schur complement of the diagonal block for primal variables in
the saddle-point matrix. One family of such preconditioners are based
on the square root of a boundary operator, but they are not robust with
respect to the geometry of domain. Furthermore, usual implementations
based on FFT can not be used for three-dimensional problems.



Here, preconditioners for Lagrange multipliers based on multilevel
techniques for Poisson-like problems are proposed. More precisely, the
employed techniques are the multilevel nodal basis (BPX) and multilevel
diagonal scaling (MDS) preconditioners. The idea is to approximate
the Schur complement using these techniques and then perform the
multiplication by its inverse by using a suitable iterative method.
The implementation must take an advantage of sparsity of vectors
in the different levels of multilevel preconditioners in order to be
affordable. To implement such a sparse version of BPX or MDS might seem
to be difficult, but in practice it is rather easy and straightforward task.
A natural assumption is that the number of Lagrange multipliers is the
order of square root of $N$ for two-dimensional problems and the order of
square of cubic root of $N$ for three-dimensional problems, where $N$ denotes
the number of primal variables. In this case, a simple analysis shows
that the computational cost of approximating the Schur complement using
BPX or MDS requires the same order of floating point operations as
the number of Lagrange multipliers is. Furthermore, based on a simple
condition number estimation it can be shown that the number of CG or
Chebyshev iterations required to perform a multiplication by the inverse
of the Schur complement approximation is the order of square root of $N$
for two-dimensional problems and the order of cubic root of $N$ for
three-dimensional problems. Thus, the total computational cost of
applying Lagrange preconditioner once is the order of $N$.



In the numerical experiments, the efficiency of proposed approach
is studied for two-dimensional and three-dimensional Poisson and diffusion
problems. The number of iterations and the condition number for several
problems are given. Also, the capability to solve easily problems in very
complicated domains is demonstrated.



%	*******************************************
%	*************  tuminaro.html  ************
%	*******************************************

\nextab
{A Parallel Multilevel Preconditioning Module for Unstructured Mesh Krylov Solvers}
{Ray Tuminaro}
{Sandia National Laboratories, PO Box 969, Livermore CA 94551}
{tuminaro@ca.sandia.gov}

{\bf Charles Tong} \\
Lawrence Livermore National Labs, Livermore CA

{\bf John Shadid, Karen Devine, David Day} \\
Sandia National Laboratories, Albuquerque NM 87185


Multilevel methods offer the best promise to attain both
fast convergence and parallel efficiency in the numerical
solution of partial differential equations. Unfortunately,
they have not been widely used, in part, due to
implementation difficulties for unstructured mesh solvers.
To facilitate use, a multilevel preconditioner software
module, ML, has been constructed. Several methods (e.g.
smoothed aggregation multigrid and two-level domain
decomposition) are provided requiring relatively modest
effort from the application developer. We discuss the
current status of this software module including the
parallel smoothed aggregation method, domain decomposition
scheme, and their use with other software packages (e.g.
Aztec, ParaSails, Metis, SuperLU, XYT) for outer
iterations, smoothers, and coarse grid solvers. We will
discuss our experiences in solving problems arising from a
few application areas in electromagnetics, elasticity, and
computational fluid dynamics (CFD) on the ASCI red
machine.



%	*******************************************
%	*************  vassilevski.html  ************
%	*******************************************

\nextab
{Panayot S. Vassilevski}
{Parallelizing AMGe using domain decomposition}
{Center for Applied Scientific Computing, Lawrence Livermore National Laboratory, 7000 East Avenue, Mail Stop 560, Livermore CA 94550}
{vassilevski1@llnl.gov}


In this talk we will present a natural domain decomposition
strategy for parallelizing agglomeration based AMGe
(algebraic multigrid for finite element problems exploiting
fine-grid element matrices).


The method consists of the following simple steps:


{\bf (1)}   local subdomain solvers (preconditioners); i.e., each
mesh subdomain (union of fine grid elements) is assigned to
a processor and a sequential AMGe solver is built in there.
That is, a sequence of coarse grids (coarse spaces) and
respective interpolation matrices are constructed. The
setup phase is based on the local bilinear (quadratic)
forms (assembled from the subdomain element matrices).
Since the local bilinear forms, in the case of model
Laplace operator, correspond to a problem with natural
(Neumann type) boundary conditions they are typically only
semi-definite. Therefore, one builds the sequential AMGe
solvers to be defined only for vectors in a space which is
a hierarchical complement to a certain coarse space. This
coarse space must contain the null-space components (so
called rigid body modes) of the local quadratic forms. The
version of the AMGe method which we exploit, called
spectral agglomerate AMGe, allows for explicit construction
of such hierarchical complements and ensures that the
null-space components are fully represented on the coarse
grid.



{\bf (2)}   a global coarse problem;
The overall preconditioner then is defined as the standard
Neumann-Neumann type domain decomposition preconditioner
(cf., [3], [2] or [4]), however, here the preconditioner is
defined not for the interface Schur complements (as is in
the original Neumann-Neumann case) but is applied to the
original bilinear forms. As already mentioned, in addition
to the subdomain solvers, one incorporates a global coarse
solver.


The sequential subdomain coarsening AMGe algorithm consists
of an agglomeration step and involves computing a few
minimal eigenvectors of the corresponding assembled
agglomerate stiffness matrix. The method (similarly to [1]
and [5]) requires access to the individual element matrices
(in order to assemble certain subdomain matrices). Based on
the topological agglomeration algorithm (cf. [5]) we
employed and the special interpolation rule chosen, one is
able to define coarse elements and coarse element matrices
thus allowing for recursion.


Numerical tests illustrating the parallel performance of
the algorithm will be presented.



\begin{enumerate}


\item M. Brezina A. J. Cleary, R. D. Falgout, V. E. Henson,
J.  E. Jones, T. A. Manteuffel, S. F. McCormick, and J. W. Ruge,
{\em Algebraic multigrid based on element interpolation (AMGe)},
SIAM J. Sci. Comput. {\bf 22} (2000), 1570--1592.

\item J. Mandel,
{\em Balancing domain decomposition},
Comm. Appl. Numer. Methods {\bf 9} (1993), 233--241.

\item Y.-H. De Roeck and P. Le Tallec,
{\em Analysis and a test of a local domain decomposition
preconditioner}, in: ``Fourth International Symposium on
Domain Decomposition Methods for Partial Differential
Equations'' (R. Glowinski, Y. Kuznetsov, G. Meurant, J.
Periaux, and O. Widlund, eds.),
SIAM, Philadelphia PA 1991.

\item M. Dryja and O. B. Widlund,
{\em Schwarz methods of Neumann-Neumann type for
three-dimensional elliptic finite element problems},
Comm. Pure Appl. Math. {\bf 42} (1995), 121--155.

\item J. E. Jones and P. S. Vassilevski,
{\em AMGe based on element agglomeration},
SISC (to appear).


\end{enumerate}


This work was performed under the auspices of the U. S.
Dept. of Energy by University of California Lawrence
Livermore National Laboratory under contract W-7405-Eng-48.




%	*******************************************
%	*************  wan.html  ************
%	*******************************************

\nextab
{Antony Jameson}
{Two Multigrid Time Stepping Schemes which Preserve Monotonicity and TVD}
{Dept. of Aeronautics and Astronautics, Stanford University, Stanford CA 94305}
{jameson@baboon.stanford.edu}

{\bf Justin W.L. Wan} \\
Dept. of Computer Science, University of Waterloo, Waterloo, Ontario, Canada N2L 3G1
\\ {\tt jwlwan@bryce1.uwaterloo.ca}


In this talk, we present two efficient multigrid time stepping schemes for
scalar linear and nonlinear wave equations based on a upwind biased
interpolation and restriction, and a nonstandard coarse grid update
formula. Furthermore, we prove that these schemes preserve monotonicity
and are total variation diminishing (TVD). Thus, no numerical oscillation
is introduced, resulting in fast wave propagation to the steady state.


Multigrid for solving elliptic partial differential equations (PDEs)
has been proven, both numerically and theoretically, to be a successful
and powerful techniques. Efficient multigrid methods have also been
proposed for solving non-elliptic equations, in particular, Euler
and Navier Stokes equations. One approach is to accelerate the evolution
of a hyperbolic system to a
steady state on multiple grids by taking larger time steps on
coarse grids without violating the stability condition. Thus, the low
frequency disturbances are rapidly expelled through the outer boundary
whereas the high frequency errors are locally damped. It can be proved
that an M-level multigrid cycle consisting of one smoothing
step on each coarse grid results in an effective time step of the sum
of the time steps of all the coarser grids.


While the above approach has achieved great success, for instance, in
Euler calculations, the understanding of the numerical property is
relatively limited and analyses are scarce.
In this paper, two efficient multigrid time stepping schemes proposed by
Jameson is studied and extended for the steady state solution of
scalar linear and nonlinear wave equations. One scheme
is multiplicative in nature and the other additive. Thus the former is
usually more effective but the latter is more parallel.
The upwind biased interpolation and restriction are defined based on the
characteristic directions. As opposed to elliptic multigrid,
they are not transpose of each other. For nonlinear wave equations, the
characteristic directions change at every grid points and in every time
steps. To define the appropriate upwind interpolation and restriction, we
solve a local Riemann problem which determines locally the characteristic
direction. Up to the authors' knowledge, Riemann solutions have not
been used in the context of multigrid interpolation/restriction.


In addition, we present the numerical analysis of these schemes;
the primary focus is on the monotonicity preserving and total variation
diminishing properties. Both concepts are fundamental in designing
discretization schemes for conservation laws, but nevertheless, have never
been used to analyze multigrid methods in the literature.
We prove that both the two-level multiplicative and additive schemes
preserve monotonicity and are TVD; and the same holds for the multilevel
additive scheme.
Finally, numerical results for solving the linear wave equation and the
nonlinear Burgers' equation in one and two dimensions are presented to
demonstrate the effectiveness of the proposed schemes and verify that no
oscillation occurs during the multigrid time stepping.






%	*******************************************
%	*************  wienands.html  ************
%	*******************************************

\nextab
{R. Wienands, C.W. Oosterlee}
{Local Fourier $k$-grid ($k$=1,2,3) analysis for Navier-Stokes-type systems}
{GMD - Institute for Algorithms and Scientific Computing (SCAI),
Schloss Birlinghoven, D-53754 Sankt Augustin, Germany}
{wienands@gmd.de, oosterlee@gmd.de}


Local Fourier one- (smoothing) and two-grid analysis (LFA) [1,2,6] are
well-known tools for the quantitative analysis and the design of
efficient multigrid methods for several problems.
The two-grid analysis is the basis
for classical asymptotic multigrid convergence estimates [6].
Moreover, it is the main analysis tool for nonsymmetric problems.
For several multigrid components or cycle variants, however, the
overall multigrid convergence cannot be approximated accurately
by two-grid factors. For example, if one is interested in the varying
performance of V- and W-cycles, of pre- and post-smoothing, of
different smoothers on different grids or of different discretizations
on different grids one has to consider at least one additional grid
leading to a three-grid analysis [8].

Although LFA is often applied to scalar equations,
two-grid values for systems of equations are rarely found
in the literature and three-grid values are completely missing.
To close this gap we developed a freely available, general Fourier
analysis program -- \verb9LFA00_2D9 --
which will be presented in the first part of the talk.
\verb9LFA00_2D9 is a set of Fortran77 subroutines to
perform Fourier $k$-grid ($k$=1,2,3)
analysis for two-dimensional systems of PDEs yielding measures of
h-ellipticity [1], smoothing factors, two- and three-grid convergence
factors and norms of the two- and three-grid operators.
Several discretizations of well-known
systems, like the biharmonic system, the Stokes equations, the Oseen
equations (linearized Navier-Stokes equations) have already been
implemented. Furthermore, an option to analyze your own linearized
system is available.
\verb9LFA00_2D9 supports a large variety of multigrid components including
\begin{itemize}
\item Coarsening strategy: standard (full), semi coarsening,
\item Coarse grid discretization: direct PDE based, Galerkin-type,
\item Prolongation: bilinear, cubic interpolation, matrix-dependent
interpolation (of Dendy[3]- and de Zeeuw[9]-type),
\item Restriction: full, half weighting, injection,
\item Relaxation: Jacobi, Gauss-Seidel, red-black-type smoothers,
point- and line-wise.
\end{itemize}
It is, furthermore, possible to estimate convergence factors if the
two- or three-grid operators are used as preconditioners for GMRES [7].

In the second part of the talk, we evaluate several non-staggered
discretizations for Navier-Stokes-type systems (w.r.t. an efficient
multigrid treatment) in order to demonstrate the usefulness of the
analysis program. Here, we focus on second order discretizations.
In particular, the Stokes and the Oseen equations are investigated,
discretized by central differences with an artificial pressure correction
term in the continuity equation (for low Reynolds numbers) or by higher
order upwind schemes like Dick's flux difference splitting [4]
using van Leer's kappa-scheme [5] (for high Reynolds numbers).
The Fourier results are confirmed by numerical experiments with the
incompressible Navier-Stokes equations.

\begin{enumerate}

\item A. Brandt,
{\em Multigrid techniques: 1984 guide with applications to
fluid dynamics}, GMD-Studie Nr. 85, Sankt Augustin, Germany (1984).

\item A. Brandt,
{\em Rigorous quantitative analysis of multigrid,
I: Constant coefficients two-level cycle with L2-norm},
SIAM J. Numer. Anal. {\bf 31} (1994), 1695--1730.

\item J.E. Dendy Jr.,
{\em Blackbox multigrid for nonsymmetric problems},
Appl. Math. Comput. {\bf 13} (1983), 261--283.

\item E. Dick and J. Linden,
{\em A multigrid method for steady incompressible
Navier-Stokes equations based on flux difference splitting},
Int. J. Num. Meth. Fluids {\bf 14} (1992), 1311--1323.

\item B. van Leer,
{\em Upwind-difference methods for aerodynamic problems
governed by the Euler equations}, In:
{\em Large Scale Computations in Fluid Mechanics},
Lectures in Appl. Math. {\bf 22}, II, B. Enquist, S. Osher,
and R. Somerville (eds.), AMS, Providence RI, 1985, 327--336.

\item U. Trottenberg, C.W. Oosterlee, and A. Sch\"uller,
{\em Multigrid}, Academic Press (2001).

\item R. Wienands, C.W. Oosterlee, and T. Washio,
{\em Fourier analysis of GMRES(m) preconditioned by multigrid},
SIAM J. Sci. Comput. {\bf 22} (2000), 582--603.

\item R. Wienands and C.W. Oosterlee,
{\em On three-grid Fourier analysis for multigrid},
to appear in SIAM J. Sci. Comput. (2001).

\item P.M. de Zeeuw,
{\em Matrix-dependent prolongations and restrictions in a
blackbox multigrid solver},
J. Comput. Appl. Math. {\bf 33} (1990), 1--27.

\end{enumerate}


%	*******************************************
%	*************  wittum.html  ************
%	*******************************************

\nextab
{Gabriel Wittum}
{Multigrid Methods for Porous Media Flow Problems}
{Institute for Computer Science, Heidelberg University, INF 368, D-69120 Heidelberg}
{wittum@iwr.uni-heidelberg.de}


Solvers play a crucial role in many simulations
determining the complexity of the whole process. Thus
solving often limits the obtainable accuracy and fast
solver even are the key to simulate new challenging
problems.

In the lecture multigrid methods for the simulation of
large porous media flow problems are discussed.  Problems
like heterogeneity, non-M-matrices etc. are addresseed.
Further adaptivity and parallelism is discussed. The software
system UG is presented which is based on these
strategies. In several application cases the efficiency of
the selected approach is shown.



%	*******************************************
%	*************  yang.html  ************
%	*******************************************

\nextab
{Ulrike Meier Yang}
{On the Use of Schwarz Smoothing in AMG}
{Center for Applied Scientific Computing, Lawrence Livermore National Laboratory, Box 808, L-560, Livermore CA 94551}
{umyang@llnl.gov}


The need to solve increasingly larger linear systems, with hundreds of
millions or billions of unknowns, has necessitated the use of massively
parallel computers and the investigation of scalable linear solvers, such as
multigrid. For unstructured grid problems, an attractive choice is
algebraic multigrid (AMG). Although AMG is a very effective method for
many applications, however for some applications, e.g. structural elasticity
problems where the governing equations are a system of PDEs with the unknowns
being the displacements in each coordinate direction, the choice of
conventional smoothers such as Jacobi or Gauss-Seidel within AMG is not
sufficient to achieve good convergence.


In this talk, we investigate the use of the Schwarz alternating method as a
smoother on the finest levels of AMG. Preliminary experiments show promising
results for such problems as mentioned above. We consider both multiplicative
and additive variants of the Schwarz method. The use of a multicoloring
technique for parallelization of the multiplicative variant is considered.
The choice of domains, efficiency and convergence behavior are discussed,
and numerical results are presented.


*This work was performed under the auspices of the U.S. Dept. of Energy
by University of California Lawrence Livermore National Laboratory under
contract number W-7405-Eng-48.



%	*******************************************
%	*************  yavneh.html  ************
%	*******************************************

\nextab
{Irad Yavneh, Ron Kimmel}
{An Algebraic Multigrid Approach for Shape from Photometric Stereo}
{Dept. of Computer Science, Technion---Israel Institute of Technology, Haifa, 32000, Israel}
{irad@cs.technion.ac.il}


We apply a new algebraic multigrid method for solving computer vision
problems with constraints, in particular ``shape from photometric
stereo''. A variational formulation is applied to the problem
of shape reconstruction from three or more images of an object with
the same viewing direction and different lighting conditions,
supplemented by some pointwise height constraints. In order to
obtain a smooth reconstruction, we choose a weight-function
that is singular at the constrained points, resulting in
an elliptic equation with constraints and
singular coefficients, which is
solved efficiently by our algebraic multigrid algorithm.



%	*******************************************
%	*************  zhang.html  ************
%	*******************************************

\section{Multigrid Method and High Order Compact Scheme
	for Solving Boundary Layer Problems on Nonuniform Grids}
{\bf Jun Zhang} \\
Laboratory for High Performance Scientific Computing and Computer Simulation,
	Dept. of Computer Science, University of Kentucky,
	Lexington KY 40506-0046 \\
\verb9www.cs.uky.edu/~jzhang9


A fourth order compact finite difference scheme (nine point compact
difference scheme) and a multigrid cycling algorithm are employed
to solve the two dimensional convection diffusion equations with
boundary layers. The computational domain is first discretized on a
nonuniform (stretched) grid to resolve the boundary layers. A grid
transformation technique is used to map the nonuniform grid to a
uniform one. The fourth order compact scheme is applied to the
transformed uniform grid. A multigrid method is then used to solve
the resulting linear system. We conduct experimental analyses to show
how the grid stretching may affect the computed accuracy of the solutions
from the fourth order compact scheme. We demonstrate that the grid
stretching may affect the convergence rate of the multigrid
method. Numerical experiments are used to show that a graded mesh
and a grid transform are necessary to compute high accuracy solution
with the fourth order compact scheme for convection diffusion problems
with boundary layers. Accuracy comparisons between the standard
(central and upwind) difference schemes and the present fourth order
compact scheme are given. Special properties of the transformed
convection diffusion equation that may affect multigrid
convergence are investigated.


%%%%%%%%%%%%%%%%%%%%%%%%%%%%

\end{document}
