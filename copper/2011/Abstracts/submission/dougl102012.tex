\documentclass{report}
\usepackage{amsmath,amssymb}
\setlength{\parindent}{0mm}
\setlength{\parskip}{1em}
\begin{document}
\begin{center}
\rule{6in}{1pt} \
{\large Craig C. Douglas \\
{\bf Isogeometric Multigrid }}

University of Wyoming Math Dept \\ 1000 E University Ave  \\ Dept 3036 \\ Laramie \\ WY 82072-3036
\\
{\tt craig.c.douglas@gmail.com}\\
Victor Calo\\
Nathan Collier\\
Hyoseop Lee\end{center}

In engineering design, geometry is a major bottleneck in
obtaining finite element solutions for a particular problem.
A design is usually born inside a CAD package and
subsequently needs to be tessellated (or meshed) such that
the geometry is approximated by a finite element space.
This process, while semi-automatic in some cases, is not
without pitfalls and often requires human interaction to
verify the resulting mesh. Furthermore, mesh refinements
require a return to the CAD system and a re-tessellation of
the CAD object. This means that convergence studies are
prohibitively expensive for complex geometries and seldom
performed.

Isogeometric analysis [1] has been developed as a solution
to this problem, simplifying, and in some cases eliminating,
the problem of converting geometric discretizations in the
engineering design process. Isogeometric analysis is an
isoparametric finite element method that uses the
Non-Uniform Rational B-spline basis (NURBS), which dominates
the CAD market. It is hoped that in using this basis form,
which is prevalent in the CAD community, that the bridge
between analysis and design can be bridged.

A NURBS basis may be $h$-refined via a process known as knot
insertion. This process can be likened to the splitting of
elements in traditional finite elements, however knot
insertion may also be used to control the continuity between
elements. More importantly the knot insertion process may
be captured as a linear operator that interpolates a vector
from one function space to a refined space exactly. We
exploit these operators to develop a multigrid approach for
isogeometric systems.

In this talk we introduce isogeometric multigrid analysis
and the NURBS basis in more detail as well as define the
interpolation operator and finally show numerical results.

[1] T. J. R. Hughes, J. Cottrell, Y. Bazilevs, Isogeometric
analysis: CAD, finite elements, NURBS, exact geometry and
mesh refinement, Computer Methods in Applied Mechanics and
Engineering, 194 (2005), pp. 4135-4195.


\end{document}
