\documentclass{report}
\usepackage{amsmath,amssymb}
\setlength{\parindent}{0mm}
\setlength{\parskip}{1em}
\begin{document}
\begin{center}
\rule{6in}{1pt} \
{\large Harald Koestler \\
{\bf A robust geometric multigrid solver within the WaLBerla framework}}

Cauerstr 6 \\ 91058 Erlangen \\ Germany
\\
{\tt harald.koestler@informatik.uni-erlangen.de}\end{center}

Recently, more and more GPU HPC clusters are installed and thus there is
a need to adapt existing software design concepts and algorithms to
multi-GPU environments. We have developed a
modular and easily extensible software framework called WaLBerla
working on block-structured domains.
It covers a wide range of applications ranging from particulate flows
over free surface flows to nano fluids coupled with temperature
simulations and medical imaging.

In this talk we report on our experiences to extend WaLBerla in order to
support geometric multigrid algorithms for the numerical solution of
partial differential equations (PDEs) on
multi-GPU clusters. As building blocks we use a damped Jacobi or
red-black Gauss-Seidel smoother,
collocation coarse approximation (CCA) to compute the coarse grid
stencils and standard or matrix-dependent intergrid transfer operators.
CCA allows us to force a certain size of the coarse grid stencil and
coincides with Galerkin coarsening, if both
have the same number of stencil entries.

We discuss the object-oriented software and
performance engineering concepts necessary to integrate efficient compute
kernels into our WaLBerla framework and show that a large fraction of the
high computational performance offered by current heterogeneous HPC
clusters can be sustained for geometric multigrid algorithms.


\end{document}
