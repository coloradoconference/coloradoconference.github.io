\documentclass{report}
\usepackage{amsmath,amssymb}
\setlength{\parindent}{0mm}
\setlength{\parskip}{1em}
\begin{document}
\begin{center}
\rule{6in}{1pt} \
{\large Wim, I Vanroose \\
{\bf A polynomial multigrid smoother for the iterative solution of the heterogeneous Helmholtz problem}}

Dept Mathematics and Computer Science \\ Universiteit Antwerpen \\ Middelheimlaan 1 \\ 2020 Antwerpen \\ Belgium
\\
{\tt wim.vanroose@ua.ac.be}\\
Bram Reps\\
Hisham bin Zubair\end{center}


This contribution is focused on the numerical solution of the
indefinite Helmholtz equation
\begin{equation}\label{eq:helmholtz}
Hu \equiv -(\Delta+k^2(x,y))u= f \mbox{ in } \Omega
\end{equation}
with a space a space dependent wave number $k$ and exterior complex
stretching as absorbing boundary conditions. The solution method is
based on a preconditioned Krylov subspace, however, the preconditioner
operator is the Helmholtz equation discretized on a grid with a
complex grid distance, which is approximately inverted with one
multigrid cycle.

The failure of multigrid on the Helmholtz problem have been widely
documented and is caused by the indefinite spectrum of the operator
$H$. Discretized on a grid with grid distance $h$, the negative
Laplacian, $-\Delta$, leads to a spectrum that is spread between $0$ and
$\mathcal{O}(1/h^2)$ on the real axis. However, the wave number will
shift this spectrum in the negative direction. This leads to an
indefinite matrix for wave numbers $k$ larger than a certain
threshold. This means that the spectrum is spread over both the
positive real part as the negative real part of the complex plane, not
necessarily excluding zero as an eigenvalue. Realistic problems have
absorbing boundary conditions, that move the eigenvalues slightly below
the real axis.

Two difficulties emerge when multigrid is applied. First, typical
smoothers like weighted Jacobi or Gauss-Seidel become unstable for
indefinite problems. However, the unstable modes are the smoothest
modes and should be stabilized by the coarse grid correction, for
problems that are slightly indefinite. The second problem is the
appearance of diverging coarse grid corrections for certain wave
numbers. This has been analyzed in detail by \cite{elman}. The peaks
in the convergence plots can be interpreted as resonances.

These difficulties are avoided when multigrid is applied to a complex
shifted Laplacian \cite{EVO06,EVO}. Then the wave number is complex
valued which cannot cause resonances in the coarse grid correction.
In \cite{JCP-paper}, we show that the complex shifting is equivalent with
discretizing the Laplacian with a complex valued grid distance.

It is important to stress that introducing the complex shift
suppresses and broadens the resonances during the coarse grid
correction, but it leaves the possibility that the smoother is
unstable. In this contribution we report on our analysis of replacing
the standard smoother such as weighted Jacobi or Gauss-Seidel with a
polynomial smoother.

In particular we have analysed the performance of polynomial
smoothers \cite{siam} on Helmholtz problems discretized with finite
difference and an exterior complex scaling as an absorbing boundary
conditions. The spectrum of the operator is bounded by a
triangle \cite{JCP-paper} where the first corner lies in $-k^2$ and
the second in $-k^2+4/h^2$ and a third corner lies in down in the
complex plane. This triangle lies entirely in the lower half of the
complex plane. The eigenvalues of the smooth modes lie near $-k^2$ while
in the other corners the modes are rapidly oscillating.

The smoothers we have studied have the form
\begin{equation}
f(t) = (1 -\omega_1 t) (1 -\omega_2 t)(1 -\omega_3 t),
\end{equation}
where the weights are choosen such that the error iteration,
$e^{(k+1)} = f(A) e^{(k)}$, fits the following requirements. It
satisfies the fixed point property since $f(0)=1$. The smoothest
modes near $-k^2$ are mapped close to the unit cirlcle since $f$ is
such that $f(-k^2)=e^{\imath\varphi}$ with $\varphi \in [0,2\pi[$. The
two other corners are
mapped to zero such that the oscillatory modes are
smoothed. In \cite{siam} we show that the coefficients $\omega_1$,
$\omega_2$ en $\omega_3$ can be choosen such that the $f$ is stable and
behaves as a smoother given certain conditions on the complex grid of
the preconditioner operator.

In practice, however, this hand tuned particular polynomial can be
replaced by GMRES(3), which optimizes the coefficients $\omega_i$ each
iteration. Since the particular polynomial provides a upper bound the
GMRES(3) convergence, the GMRES(3) smoother inherits all the
properties from the hand tuned polynomials. Note that since we use a
complex grid, we only need a restart of three, while \cite{elman}
requires a complicated smoothing strategy.

We have extensively tested this solution strategy and it gives a
satisfactory multigrid convergence that is independent of the wave
number. However, the krylov convergence is increases linearly in the
wave number.


\providecommand{\bysame}{\leavevmode\hbox to3em{\hrulefill}\thinspace}
\providecommand{\MR}{\relax\ifhmode\unskip\space\fi MR }
\providecommand{\MRhref}[2]{%
\href{http://www.ams.org/mathscinet-getitem?mr=#1}{#2}
}
\providecommand{\href}[2]{#2}
\begin{thebibliography}{1}

\bibitem{elman}
H.C. Elman, O.G. Ernst, and D.P. O'Leary, \emph{{A multigrid method enhanced by
Krylov subspace iteration for discrete Helmholtz equations}}, SIAM Journal of
Scientific Computing \textbf{23} (2002), no.~4, 1291--1315.

\bibitem{EVO06}
Y.A. Erlangga, CW~Oosterlee, and C.~Vuik, \emph{{A novel multigrid based
preconditioner for heterogeneous Helmholtz problems}}, SIAM Journal on
Scientific Computing \textbf{27} (2006), no.~4, 1471--1492.

\bibitem{EVO}
YA~Erlangga, C.~Vuik, and CW~Oosterlee, \emph{{On a class of preconditioners
for solving the Helmholtz equation}}, Applied Numerical Mathematics
\textbf{50} (2004), no.~3-4, 409--425.

\bibitem{JCP-paper}
B.~Reps, W.~Vanroose, and H.~Zubair, \emph{{On the indefinite Helmholtz
equation: Complex stretched absorbing boundary layers, iterative analysis,
and preconditioning}}, Journal of Computational Physics \textbf{229} (2010),
no.~22, 8384--8405.

\bibitem{siam}
W.~Vanroose, B.~Reps, and H.~Zubair, \emph{The discrete helmholtz equation
preconditioned with multigrid with a polynomial smoother}, Submitted to SIAM
journal of Numerical Analysis, http://arxiv.org/abs/1012.5379. (2010).

\end{thebibliography}


\end{document}
