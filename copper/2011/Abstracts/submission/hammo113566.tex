\documentclass{report}
\usepackage{amsmath,amssymb}
\setlength{\parindent}{0mm}
\setlength{\parskip}{1em}
\begin{document}
\begin{center}
\rule{6in}{1pt} \
{\large Jason F Hammond \\
{\bf Modeling and Simulating Biofilms in Fluid Flow }}

9661 Fox St \\ Northglenn \\ CO 80260
\\
{\tt hammonjf@colorado.edu}\\
David M Bortz\end{center}

In this talk we use the immersed boundary method to simulate the
interaction of fluid flowing in a tube with an attached biofilm on the
inner surface of the tube. We use the incompressible viscous
Navier-Stokes (N-S) equations to describe the motion of the flowing
fluid. In this simulation we can assign different density and viscosity
values to the biofilm than that of the surrounding fluid. Also included
in this simulation are breakable springs connecting the particles in the
biofilm which allow us to include erosion and sloughing detachment into
the model. We discretize the fluid equations using finite differences and
use a multigrid method to solve it at each time step. Multigrid is used
because the biofilm has a different density and viscosity than the
surrounding fluid which causes the coefficients in the N-S equations to
be non-constant in time. We apply this model in both two and three
dimensions.


\end{document}
