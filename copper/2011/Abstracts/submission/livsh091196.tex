\documentclass{report}
\usepackage{amsmath,amssymb}
\setlength{\parindent}{0mm}
\setlength{\parskip}{1em}
\begin{document}
\begin{center}
\rule{6in}{1pt} \
{\large Ira Livshits \\
{\bf Application of multigrid to indefinite Helmholtz equation}}

Department of Mathematical Sciences \\ Ball State University \\ Muncie \\ IN 47306
\\
{\tt ilivshits@bsu.edu}\end{center}

Helmholtz equation is a widely used model for many phenomena involving
wave propagation. Finding solutions to these equations is often
nontrivial especially for the ones with high wave numbers. For multigrid,
for example, the main bottleneck is a richness of the near kernel of the
Helmholtz operator. One of the ways to deal with such richness is to
use multiple descriptions of solution on each coarse enough grid. In this
talk, we discuss new results for the two-dimensional Helmholtz equation,
obtained by utilizing some (relatively) old ideas.


\end{document}
