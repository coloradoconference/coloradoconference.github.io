\documentclass{report}
\usepackage{amsmath,amssymb}
\setlength{\parindent}{0mm}
\setlength{\parskip}{1em}
\begin{document}
\begin{center}
\rule{6in}{1pt} \
{\large Kirk E. Jordan \\
{\bf A Glimpse at IBM�s Blue Gene/Q System and Implications for Applications and Solvers such as Multigrid}}

1 Rogers Street \\ Cambridge \\ MA 02142
\\
{\tt kjordan@us.ibm.com}\end{center}

High performance computing (hpc) is a tool frequently used to understand
complex problems in numerous areas such as aerospace, biology, climate
modeling and energy. Scientists and engineers working on problems in
these and other areas demand ever increasing compute power for their
problems. In order to satisfy the demand for increase performance to
achieve breakthrough science and engineering, we turn to parallelism
through large systems with multi-core chips. Over the next year, IBM will
be deploying its next generation Blue Gene/Q system. This talk will give
a glimpse into the Blue Gene/Q system with some preliminary results on
the initial Blue Gene/Q prototype system. I will comment on what the Blue
Gene/Q might mean for applications and solvers. This will include
possibly looking at Multigrid in a new light. I will comment on potential
trade-offs that may be encountered on a system such as the Blue Gene/Q
that will need to be investigated. There is tremendous potential with
this new system but new approaches may be needed to take full advantage
of this potential and follow on systems as we move ever closer to
Exascale.


\end{document}
