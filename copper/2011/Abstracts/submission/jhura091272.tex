\documentclass{report}
\usepackage{amsmath,amssymb}
\setlength{\parindent}{0mm}
\setlength{\parskip}{1em}
\begin{document}
\begin{center}
\rule{6in}{1pt} \
{\large Chetan Jhurani \\
{\bf Fast Construction of Sparse Preconditioners for High-Order Finite Element Problems}}

Tech-X Corporation \\ 5621 Arapahoe Ave \\ Suite A \\ Boulder \\ CO 80303
\\
{\tt jhurani@txcorp.com}\\
Travis Austin\\
Ben Jamroz\\
Marian Brezina\\
Thomas Manteuffel\\
John Ruge\end{center}

High-order finite elements provide better accuracy per degree of freedom
than the canonical first-order finite elements. However, their
implementation uses more computer memory because of denser element
stiffness matrices, which in turn also leads to longer run-times when
solving these systems with preconditioned Krylov methods. To reduce the
run-times, researchers have considered preconditioners based on matrices
assembled from sparser element stiffness matrices built from low-order
elements on a separate and finer discretization [1,2]. This leads to
higher accuracy due to high-order basis functions while avoiding the high
cost associated with dense operations.

In a similar vein, we introduce a fast algebraic method of constructing
sparse preconditioners for high-order finite element problems. This
method creates preconditioners on individual elements by approximately
solving a constrained quadratic optimization problem for non-zero entries
of the preconditioner matrix. Assembly of these local preconditioner
matrices results in the global preconditioner. The sparsity pattern is
chosen automatically in an algebraic manner and does not use any
geometric information. Our earlier work on this topic used the eigenvalue
decomposition of element stiffness matrix [3]. We have now generalized
the method and made it faster by using a set of randomized orthogonal
test vectors instead of eigenvectors. We present numerical results on
multiple mesh types and for stationary and transient problems that show
that using these preconditioners results in faster run-times and less
memory requirements.

1. S. ORSZAG. Spectral methods for problems in complex geometries. J.
Comp. Phys. 37 (1980), pp. 70-92.
2. J. HEYS, T. MANTEUFFEL, S. MCCORMICK, L. OLSON. Algebraic multigrid
for higher-order finite elements. J. Comp. Phys. 204 (2005), pp. 520-532.
3. T. AUSTIN, M. BREZINA, T. MANTEUFFEL, J. RUGE. Automatic construction
of sparse preconditioners for high-order finite element problems.
accepted for publication in Bentham eBook: Efficient Preconditioned
Solution Methods for Elliptic Partial Differential Equations.


\end{document}
