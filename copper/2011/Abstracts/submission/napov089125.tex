\documentclass{report}
\usepackage{amsmath,amssymb}
\setlength{\parindent}{0mm}
\setlength{\parskip}{1em}
\begin{document}
\begin{center}
\rule{6in}{1pt} \
{\large Artem Napov \\
{\bf An algebraic multigrid method \\ with guaranteed convergence rate I: \\The symmetric case}}

Universit� Libre de Bruxelles \\ Service de M�trologie Nucl�aire (CP 165/84) \\ Avenue F D Roosevelt 50 \\ B-1050 Bruxelles - Belgium
\\
{\tt anapov@ulb.ac.be}\\
Yvan Notay\end{center}

We present an algebraic multigrid method that has a guaranteed
convergence rate for the class of nonsingular symmetric M-matrices with
nonnegative row sum. A key ingredient is a new aggregation-based
coarsening algorithm which insures that the two-grid convergence factor
remains below a prescribed (user defined) parameter. This is achieved by
using the bound in [2] based on the worst aggregates' ``quality''. For a
sensible choice of this parameter, it is shown that the recursive use of
the two-grid procedure yields a convergence independent of the number of
levels, providing that one uses a proper AMLI-cycle. On the other hand,
the computational cost per iteration
step is of optimal order if the mean aggregates' size is large enough.
This point is addressed analytically for the model Poisson problem and,
further, numerically through a wide range of numerical experiments,
demonstrating the robustness of the method. The experiments are performed
on low order discretizations of
second order elliptic PDEs in two and three dimensions, with both
structured and unstructured grids, some of them with local refinement
and/or reentering corner, and possible jumps or anisotropies in the PDE
coefficients.

\textbf{References}\\
[1]~A. Napov and Y. Notay, {\em An algebraic multigrid method with
guaranteed convergence rate},
Report GANMN 10-03, \\
\verb+http://mntek3.ulb.ac.be/pub/docs/reports/pdf/ganmn1003.pdf+\,\\
[2]~A. Napov and Y. Notay, {\em Algebraic analysis of aggregation-based multigrid},
Numer. Lin. Alg. Appl., available in Wiley Online Library, DOI: 10.1002/nla.741\,


\end{document}
