\documentclass{report}
\usepackage{amsmath,amssymb}
\setlength{\parindent}{0mm}
\setlength{\parskip}{1em}
\begin{document}
\begin{center}
\rule{6in}{1pt} \
{\large ILYA LASHUK \\
{\bf Element Agglomeration Coarse Raviart--Thomas Spaces With Approximation Properties}}

Lawrence Livermore National Laboratory \\ 7000 East Avenue \\ L-560 \\ Livermore \\ CA 94550
\\
{\tt LASHUK2@LLNL.GOV}\\
PANAYOT VASSILEVSKI\end{center}

We propose a new technique based on element agglomeration for
constructing coarse subspaces of the lowest order tetrahedral
Raviart--Thomas finite element spaces. The coarse spaces are spanned by
local basis functions associated with each agglomerated face (i.e.,
interface between two agglomerated elements). Each such face is
associated with up to 4 coarse shape functions. The support of these
functions extends into the neighboring agglomerated elements and their
construction involves the solution of certain local mixed finite element
problem on each neighboring agglomerated element. In contrast to a
previous work, the thus constructed coarse subspace exhibits certain
approximation properties due to the fact that it contains (i.e.,
interpolates exactly) the global coordinate vector constants
$\mathbf{e}_1$, $\mathbf{e}_2$, $\mathbf{e}_3$ and the global position
vector $\mathbf{x}$. Our construction is general; in particular, we do
not assume that the agglomerated faces are planar. Possible applications
of the coarse Raviart--Thomas spaces are in constructing multigrid
methods for $H(\mathbf{div})$ bilinear forms, and (based on the
approximation properties of these spaces) in upscaling of mixed
formulation of diffusion problems. We provide some preliminary numerical
examples.


\end{document}
