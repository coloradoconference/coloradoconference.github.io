\documentclass{report}
\usepackage{amsmath,amssymb}
\setlength{\parindent}{0mm}
\setlength{\parskip}{1em}
\begin{document}
\begin{center}
\rule{6in}{1pt} \
{\large Armando Coco \\
{\bf Multigrid technique for non-eliminated boundary conditions in arbitrary domain}}

Via Procida \\ 54 \\ 96012 Avola (SR) \\ Italy
\\
{\tt coco@dmi.unict.it}\\
Giovanni Russo\end{center}

Poisson equation in arbitrary domain (possibly with moving boundary) is
central to many applications, such as diffusion phenomena,fluid dynamics,
charge transport in semiconductors, crystal growth, electromagnetism and
many others.
We present a rather simple iterative method~\cite{CocoRusso:Elliptic} to
solve the Poisson equation in arbitrary domain $\Omega$, identified by a
level set function $\phi$ in such a way $\Omega=\left\{ x \in
\mathbb{R}^d \colon \phi(x)<0 \right\}$, and mixed boundary conditions.
Such iterative scheme is just the building-block for a proper multigrid
approach~\cite{CocoRusso:MG}.

The method is based on ghost-cell technique for finite difference
discretization on a regular Cartesian grid. The structure of ghost points
is complex and elimination of discrete boundary conditions from the
system is hard to perform. In addition, a simple Gauss-Seidel scheme for
the whole system does not converge. Therefore, in order to provide a good
smoother for the multigrid approach, we relax the whole problem
introducing a fictitious time and looking for the steady-state solution.
The iterations on the boundary are performed in order to provide smooth
errors. The iterative scheme is proved to converge, at least for first
order accurate discretization.

Multigrid techniques for ghost points are well-studied in literature,
especially in the case of rectangular domain, where the restriction of
the boundary is performed using a restriction operator of codimension
$1$, since ghost points are aligned with the Cartesian axis.
In the case of arbitrary domain, ghost points have an irregular structure
and we provide a reasonable definition of the restriction operator for
the boundary.

We also show that a proper treatment of the boundary iterations can
improve the rate of convergence of the multigrid and the cost of this
extra computational work is negligible.

The method can also be extended to the interesting case of discontinuous
coefficients across an interface. Some preliminary numerical results are
provided in this talk.

\begin{thebibliography}{100}

\bibitem{Brandt:analysis}
A.~Brandt, Rigorous quantitative analysis of multigrid, I: constant
coefficients two-level cycle with L2-norm,
{\em SIAM Journal on Numerical Analysis}, 31 (1994), pp.~1695-1730.

\bibitem{CocoRusso:Elliptic}
A.~Coco and G.~Russo, A fictitious time method for the solution of
poisson equation in an arbitrary domain embedded in a square grid,
submitted to {\em Journal of Computational Physics}.

\bibitem{CocoRusso:MG}
A.~Coco and G.~Russo, Multigrid approach for poisson's equation with
mixed boundary condition in an arbitrary domain, {\em in preparation}.

\bibitem{CocoRusso:DiscoCoeff1D}
A.~Coco and G.~Russo, Second order multigrid methods for elliptic
problems with discontinuous coefficients on an arbitrary interface, I:
one dimensional problems, submitteed to {\em Numerical Mathematics:
Theory, Methods, Application}, proceedings to European Multi-grid
Conference EMG2010.

\bibitem{Gibou:second}
F.~Gibou and R.~Fedkiw, A second-order-accurate symmetric discratization
of the poisson equation on irregular domains,
{\em Journal of Computational Physics}, 176 (2002), pp.~205-227.

\bibitem{Gibou:Robin}
J.~Papac, F.~Gibou and C.~Ratsch, Efficient Symmetric Discretization for
the Poisson, Heat and Stefan-Type Problems with Robin Boundary
Conditions,
{\em Journal of Computational Physics}, 229 (2010), pp.~875-889.

\bibitem{Trottemberg:MG}
U.~Trottemberg, C.~Oosterlee, and A.~Schuller, {\em Mulitgrid}. Academic Press, (2000).

\end{thebibliography}


\end{document}
