\documentclass{report}
\usepackage{amsmath,amssymb}
\setlength{\parindent}{0mm}
\setlength{\parskip}{1em}
\begin{document}
\begin{center}
\rule{6in}{1pt} \
{\large Yao Chen \\
{\bf An Algebraic Multilevel Method for Anisotropic Elliptic Equations Based on Subgraph Matching}}

012 McAllister Building \\ Penn State University \\ University Park \\ PA 16802
\\
{\tt chen_y@math.psu.edu}\\
James  Brannick \\
Ludmil  Zikatanov  \end{center}

We present a strength of connection measure for algebraic multilevel
algorithms for a class of linear systems corresponding to the graph
Laplacian on a general graph. The coarsening in the multilevel algorithm
is based on matching in the underlying graph. Our main new idea is to
define a local measure of the quality of the matching whose maximum gives
an upper bound on the stability (energy norm) of the projection on the
coarse space. As an application, we focus on utilizing this measure as a
tool for constructing coarse spaces for anisotropic diffusion problems.
With appropriate boundary conditions, the discretized systems are
positive semidefinite but not necessarily M-matrices. Specifically, we
consider the diffusion equation with grid aligned as well as non-grid
aligned anisotropies in the diffusion coefficient and show that the
strength of connection measure is able to capture the correct anisotropic
behavior in both cases. We then study a coarsening algorithm that uses
this measure in a greedy strategy to find the subgraph matching (set of
aggregates). The process forms an initial set of subgraphs, each
consisting of a single vertex, and then adds vertices to these subgraphs
corresponding to the local direction of the anisotropy as determined by
the proposed measure.


\end{document}
