\documentclass{report}
\usepackage{amsmath,amssymb}
\setlength{\parindent}{0mm}
\setlength{\parskip}{1em}
\begin{document}
\begin{center}
\rule{6in}{1pt} \
{\large Minho Park \\
{\bf Relaxation-corrected Bootstrap Algebraic Multigrid (\textit{\MakeLowercase{r}}BAMG)}}

1300 30TH ST APT E1-14 Boulder \\ CO 80303
\\
{\tt parkmh@colorado.edu}\end{center}

Bootstrap Algebraic Multigrid (BAMG) is a multigrid-based solver for
matrix equations of the form $Ax=b$. Its aim is to automatically
determine the interpolation weights used in algebraic multigrid (AMG) by
locally fitting a set of test vectors that have been relaxed as solutions
to the corresponding homogeneous equation, $Ax=0$. This paper develops
another form of BAMG, called \textit{r}BAMG, that involves modifying the
least-squares process by temporarily relaxing on the test vectors at the
fine-grid interpolation points.

The \textit{r}BAMG setup process involves several components that are
developed in this paper. Besides the new least-squares principle
involving the residuals of the test vectors, a simple extrapolation
scheme is developed to accurately estimate the convergence factors of the
evolving AMG solver. Such a capability is essential to effective
development of a fast solver, and the approach introduced here proves to
be much more effective than the conventional approach of just observing
successive error reduction factors. Another component of the setup
process is the use of the current V-cycle to ensure its effectiveness or,
when poor convergence is observed, to expose error components that are
not being properly attenuated. Another related component is the scaling
and recombination Ritz process that targets the so-called weak
approximation property in an attempt to reveal the important elements of
these evolving error and test vector spaces.

The study of \textit{r}BAMG here is an attempt to systematically analyze
the behavior of the algorithm in terms relative to several parameters.
The focus here is on the number of test vectors, the number of
relaxations applied to them, and the dimension of the matrix to which the
scheme is applied. A large number of other parameters and options could
also be considered, including different cycling strategies, other
coarsening strategies (e. g., computing several eigenvector
approximations on coarse levels), different numbers of relaxation sweeps
on coarse levels, different possible strategies for combining test
vectors and error components produced by the current cycles, and so on.
Studying all of these options and parameters would not be feasible here.
Instead, reasonable choices are made based on some sample studies (that,
in the interest of space, we choose not to document here), with the hope
that the \textit{r}BAMG algorithm studied here is generally fairly
effective and robust. Our analysis is thus able to focus on how this
scheme behaves numerically in the face of increasing the numbers of test
vectors and relaxation sweeps performed on them, as well as the problem
sizes.


\end{document}
