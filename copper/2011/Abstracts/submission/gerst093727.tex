\documentclass{report}
\usepackage{amsmath,amssymb}
\setlength{\parindent}{0mm}
\setlength{\parskip}{1em}
\begin{document}
\begin{center}
\rule{6in}{1pt} \
{\large Axel Gerstenberger \\
{\bf Algebraic Multi-Grid techniques for the eXtended Finite Element Method }}

P O Box 5800 \\ MS-1320 \\ Albuquerque \\ NM 87185-1320 \\ USA
\\
{\tt agerste@sandia.gov}\\
Axel Gerstenberger\\
Raymond S. Tuminaro\end{center}

After more than a decade of research on the eXtended Finite Element
Method (XFEM), the method has developed into a valuable tool for
simulating crack propagation, fluid-structure interaction and
multiphase/multimaterial problems. As problem sizes grow, the wish to
apply iterative Algebraic Multi-Grid (AMG) methods to XFEM problem
arises. However, the introduction of additional degrees of freedom with
special approximation functions, which is the main ingredient of XFEM,
poses a number of challenges that hamper the straight forward application
of AMG. Examples of such difficulties are increased condition numbers of
the system matrix, varying number of unknowns per node, and matrix
graphs, that do not reflect physical properties of the system.

This presentation will illustrate the main difficulties of using AMG
methods for XFEM and will propose a number of strategies that eventually
allow one to reach the performance that one expects from using AMG for
standard Finite Element Methods.


\end{document}
