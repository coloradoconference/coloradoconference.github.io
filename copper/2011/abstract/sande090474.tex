\documentclass{report}
\usepackage{amsmath,amssymb}
\setlength{\parindent}{0mm}
\setlength{\parskip}{1em}
\begin{document}
\begin{center}
\rule{6in}{1pt} \
{\large Geoffrey, D. Sanders \\
{\bf Locally Supported Eigenvectors of Graph-Laplacian Matrices}}

Lawrence Livermore National Laboratory \\ Box 808 \\ L-561 \\ Livermore \\ CA 94551-0808
\\
{\tt sanders29@llnl.gov}\\
Allison Baker\\
Van E. Henson\\
Sarah Powers\\
Panayot Vassilevski\end{center}

Certain types of simple configurations in the periphery of an undirected
and unweighted graph yield graph Laplacian matrices (GLs) with some
eigenvectors that are locally-supported, or nonzero only on the vertices
within the individual configurations. For certain classes of scale-free
graphs, such configurations are highly likely and we use
locally-supported eigenvectors to solve eigenproblems associated with GLs
with improved efficiency. We demonstrate that the configurations are
locally detectable and that the associated locally supported eigenvectors
are locally computable. Additionally, we show that an aggregation-based
coarsening process yields a smaller graph whose GL represents all other
eigenpairs of the original GL. We provide an approximation result that
shows this representation is exact. In other words, the detection of such
configurations provides a first coarsening that introduces no spectral
approximation error. We discuss the scope for application of these
results and demonstrate their use on some examples.


\end{document}
