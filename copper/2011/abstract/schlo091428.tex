\documentclass{report}
\usepackage{amsmath,amssymb}
\setlength{\parindent}{0mm}
\setlength{\parskip}{1em}
\begin{document}
\begin{center}
\rule{6in}{1pt} \
{\large Nico Schl\"omer \\
{\bf Preconditioning the Jacobian~system of the extreme~type--II~Ginzburg--Landau~problem}}

Universiteit Antwerpen \\ Campus Middelheim \\ Gebouw G \\ Middelheimlaan 1 \\ 2020 Antwerpen \\ Belgium
\\
{\tt nico.schloemer@ua.ac.be}\\
Wim Vanroose\end{center}

The nonlinear Schr\"odinger equation is used in many areas of science
and technology and describes, for example, the propagation of
solutions in fiber optics or Bose-Einstein condensates in ultra cold
traps. The Ginzburg--Landau equations is a prototype of such an
equation that is used to model the state and the magnetic field inside
superconducting nanodevices. To understand the dynamics of these
systems, it is critical to have an efficient solver such
that the solution space of can be efficiently explored by, for
example, numerical continuation.

For an open, bounded domain $\Omega\subseteq\mathbb{R}^n$ with a
piecewise smooth boundary
$\partial\Omega$, the Ginzburg--Landau equations for extreme type--II
superconductors read
\begin{equation}\label{eq:GL}
\begin{cases}
0 = \left(-\text{i}\nabla - \mathbf{A}\right)^2 \psi - \psi \left(1 -
|\psi|^2\right) \quad \text{on } \Omega \\[3mm]
0 = \mathbf{n} \cdot ( -\text{i}\nabla - \mathbf{A}) \psi \quad \text{on } \partial\Omega
\end{cases}
\end{equation}
The unknown $\psi\in H^2_{\mathbb{C}}(\Omega)$ is commonly referred to as
\emph{order parameter}.
As opposed to the general case, where the magnetic vector potential
$\mathbf{A}\in H_{\mathbb{R}^n}^2(\Omega)$ is an
unknown of the system, $\mathbf{A}$ is given here by an external magnetic field
$\mathbf{H}_0$ via
\[
\begin{cases}
\nabla\times(\nabla\times \mathbf{A}) = 0,\\
\lim\limits_{\|x\|\to\infty} \nabla\times\mathbf{A} = \mathbf{H}_0;
\end{cases}
\]
the magnetic field fully penetrates the whole domain and is not altered
by the supercurrent.
The physical observables are the Cooper-pair density
$\rho_{\text{C}}=|\psi|^2$ and the induced magnetic field
$\mathbf{B}=\nabla\times\mathbf{A}$.

Extreme type--II superconductors are common in the domain of high-tem\-per\-a\-ture
superconductors, for example.

The trivial solution, $\psi=0$, is the normal non-superconducting
state which is the lowest energy state for fields above the critical
magnetic field strength. For weak magnetic fields, however, there are
non-zero solutions that have a lower energy. These are the famous vortex
solutions where the magnetic fields penetrate the sample.

This talk will be concerned with the numerical solution of~(\ref{eq:GL}). Applying
Newton's method, the focus will be on to solve linear equation systems with the
Jacobian operator of~(\ref{eq:GL}),
\begin{equation}\label{eq:jacobian}
\begin{cases}
\mathcal{J}(\psi)\varphi = \left( (-\text{i} \nabla - \mathbf{A})^2 - 1 +
2|\psi|^2 \right) \varphi + \psi^2 \overline{\varphi},\\
0 = \mathbf{n} \cdot ( -\text{i}\nabla - \mathbf{A}) \varphi \quad
\text{on } \partial\Omega.
\end{cases}
\end{equation}
Properties of $\mathcal{J}(\psi)$ will be outlined that restrict the use
of classical solvers.
For preconditioning those linear solves, special focus will be on the
first part of the operator
\begin{equation}\label{eq:keo}
\begin{cases}
\mathcal{K}\varphi = (-\text{i} \nabla - \mathbf{A})^2 \varphi,\\
0 = \mathbf{n} \cdot ( -\text{i}\nabla - \mathbf{A}) \varphi \quad
\text{on } \partial\Omega,
\end{cases}
\end{equation}
and, connected with that, multigrid strategies. Appropriate discretizations will
introduced briefly.

Eventually, a solver for $\mathcal{J}(\psi)$ will be constructed which allows
for a constant number of linear solver iterations independent of the problem size.
Numerical evidence will be presented along with scalability studies to provide
insight on the performance of the solver in high-performance computing environments.


\end{document}
