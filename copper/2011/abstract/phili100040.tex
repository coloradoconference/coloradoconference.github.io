\documentclass{report}
\usepackage{amsmath,amssymb}
\setlength{\parindent}{0mm}
\setlength{\parskip}{1em}
\begin{document}
\begin{center}
\rule{6in}{1pt} \
{\large Bobby Philip \\
{\bf Multilevel Block Preconditioned Methods for Multi-Domain Quasistatic Thermomechanics}}

Oak Ridge National Laboratory \\ Oak Ridge \\ TN 37831
\\
{\tt philipb@ornl.gov}\end{center}

The source-term for most of the physics in nuclear fuel is the heat
generated from nuclear fission and the resulting transmutation (which
includes fission) of the materials due to irradiation. The heat,
primarily generated in the fuel, is transported through the fuel, across
a gap, through a cladding material, and removed by a coolant. The
irradiation and temperature change in turn produce a mechanical response
in the solid bodies (fuel and cladding). This problem is described by a
multi-domain (pellet, gap, and clad) multi-physics (nonlinear
thermomechanics) nonlinear system of PDE's. This talk will describe
recent work on fully coupled quasistatic nonlinear thermomechanics
simulations for this problem. The solution method used is a Jacobian free
Newton-Krylov method preconditioned by a block diagonal preconditioner
consisting of thermal diffusion and mechanics. The talk will describe the
solution process, comparison to experiment, and implementation within a
new multi-physics framework being jointly developed by ORNL, LANL, and
INL.



\end{document}
