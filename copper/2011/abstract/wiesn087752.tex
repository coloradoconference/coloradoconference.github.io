\documentclass{report}
\usepackage{amsmath,amssymb}
\setlength{\parindent}{0mm}
\setlength{\parskip}{1em}
\begin{document}
\begin{center}
\rule{6in}{1pt} \
{\large Tobias, A. Wiesner \\
{\bf An improved AMG transfer operator for nonsymmetric positive-definite systems}}

Technical University of Munich \\ Institute for Computational Mechanics \\ Boltzmannstrasse 15 \\ 85748 Garching \\ Germany
\\
{\tt wiesner@lnm.mw.tum.de}\\
Michael, W. Gee\\
Raymond, S. Tuminaro\\
Wolfgang, A. Wall\end{center}

Algebraic multigrid methods are well-known and well-established for
symmetric positive-definite problems as they arise in many applications
in the area of engineering and applied sciences. However, nonsymmetric
positive definite systems are still challenging even though they have
received more attention in recent years (\cite{sala2008},
\cite{notay2010}, \cite{BrMaMcRuSa2010}).

This talk is devoted to the design of AMG preconditioners for
nonsymmetric (positive-definite) systems. In this case, the crucial point
is the construction of appropriate nonsymmetric multigrid transfer
operators. We extend the ideas of ``smoothed aggregation'' and ``energy
minimization'' (e.g. \cite{mandel1999}, \cite{brannick2006} and
\cite{brannick20062}) that are often used for symmetric positive-definite
systems. A new class of AMG transfer operators is proposed for
nonsymmetric
problems based on a constrained descent algorithm which is utilized to
iteratively improve prolongation and restriction operators while allowing
one to prescribe an optimized sparsity pattern.

A key feature of this talk is the algorithm for the effective
determination of prolongator and restrictor sparsity patterns. �Initial
patterns are obtained from either classical (nonsymmetric) aggregation
approaches or are based on an incomplete block LU decomposition within a
two level domain
decomposition method. �A heuristic filtering method is proposed which can
be applied to these initial patterns. Our filtering approach uses a
strong pattern filter for finer multigrid levels and a very weak pattern
filter for coarser levels. �We illustrate how an appropriate sparsity
pattern choice
results in a noticeable gain in convergence speed.

We compare our new transfer operator strategy with existing techniques
(e.g. Petrov-Galerkin from \cite{sala2008}) by means of examples arising
from finite element discretization of the convection-diffusion and
Navier-Stokes equations.

\begin{thebibliography}{100}
\bibitem{sala2008}
M.~Sala and R. S.~Tuminaro, {\em A new Petrov-Galerkin smoothed
aggregation preconditioner for nonsymmetric linear systems}, SIAM Journal
on Scientific Computing, 31 (2008), pp.~143--166.
\bibitem{notay2010}
Y.~Notay, {\em Algebraic analysis of two-grid methods: the nonsymmetric
case}, Numer. Linear Algebra Appl., 17 (2010), pp.~73-96
\bibitem{BrMaMcRuSa2010}
M.~Brezina, T.~Manteuffel, S.~Mc{C}ormick, J.~Ruge and G.~Sanders,�{\em
Towards Adaptive Smoothed Aggregation ($\alpha$SA) for Nonsymmetric
Problems}, SIAM Journal on Scientific Computing, 32 (2010), pp.~14--39
\bibitem{mandel1999}
J.~Mandel, M.~Brezina and P. Van\v{e}k, {\em Energy optimization of
algebraic multigrid bases}, Computing, 62 (1999), pp.~205--228
\bibitem{brannick2006}
J.~Brannick, M.~Brezina, S.~MacLachlan, T.~Manteuffel, S.~McCormick,
J.~Ruge, {\em An energy-based AMG coarsening strategy}, Num. Lin. Alg.
Appl., 13 (2006), pp.~133--148
\bibitem{brannick20062}
J.~Brannick, L.~Zikatanov, {\em Algebraic Multigrid Methods Based on
Compatible Relaxation and Energy Minimization}, Lecture Notes in
Computational Science and Engineering: Domain Decomposition Methods in
Science and Engineering XVI, 55 (2007), pp.~15--26
\end{thebibliography}


\end{document}
