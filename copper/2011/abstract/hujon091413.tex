\documentclass{report}
\usepackage{amsmath,amssymb}
\setlength{\parindent}{0mm}
\setlength{\parskip}{1em}
\begin{document}
\begin{center}
\rule{6in}{1pt} \
{\large Jonathan Hu \\
{\bf Coarse Grid Representations of the Near Nullspace in an Energy Minimizing Multigrid}}

Sandia National Labs \\ P O Box 969 MS 9159 \\ Livermore \\ CA 94551
\\
{\tt jhu@sandia.gov}\\
Jeremie Gaidamour\\
Chris Siefert\\
Ray Tuminaro\end{center}

Solving linear systems arising from multi-physics simulations is
challenging, especially when the underlying model has boundary layers,
stretching, or discontinuities. We have developed an algebraic multigrid
method to address such issues. This method can be viewed as solving a
constrained optimization problem. {\it A priori} sparsity patterns and
interpolation of low-energy error modes are constraints in an energy
minimization process that can yield an AMG preconditioner with low
operator complexity and good convergence properties. This method allows
for considerable freedom in choosing the number of coarse degrees of
freedom per node as well as the coarse representation of the low-energy
modes. This gives rise to certain questions, such as how to compress
near-nullspace information and how to enforce coarse near-nullspace
orthogonality in order to have an amenable minimization problem. We
discuss various answers to these questions and their implications for the
solver, and illustrate with numerical experiments that the resulting
multigrid methods can be effective, especially for linear equations
arising from systems of PDEs.


\end{document}
