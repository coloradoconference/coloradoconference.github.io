\documentclass{report}
\usepackage{amsmath,amssymb}
\setlength{\parindent}{0mm}
\setlength{\parskip}{1em}
\begin{document}
\begin{center}
\rule{6in}{1pt} \
{\large Badri Hiriyur \\
{\bf A quasi-algebraic multigrid approach based on Schur complements for linear systems associated with XFEM}}

626A Seeley W Mudd Hall \\ 500 West 120th Street \\ New York \\ NY 10027
\\
{\tt bkh2112@columbia.edu}\\
Ray Tuminaro\\
Haim Waisman\\
Erik Boman\\
David Keyes\end{center}

An algebraic multigrid method is proposed that is suitable for the linear
systems associated with modeling fracture via extended finite element
method (XFEM). The new method follows naturally from an energy minimizing
algebraic multigrid framework and is suitable to be used as a
preconditioner to accelerate Krylov subspace based iterative methods. The
key idea is the modification of the prolongator sparsity pattern to
prevent interpolation across cracks. The proposed method is
quasi-algebraic since the geometric levelset information relating to
cracks are used to modify the prolongator sparsity pattern. This levelset
information is readily available from the XFEM discretization process. It
is shown that algebraic multigrid using these modified transfer operators
is highly efficient when applied to the Schur complement of the XFEM
stiffness matrix in which the enriched degrees of freedom are condensed
out. In the proposed method, suitable modifications are made to avoid the
explicit computation of the Schur complement. Numerical experiments
illustrate that the resulting method is scalable since the convergence
properties are relatively insensitive to the mesh density and to the
number of cracks or their location.


\end{document}
