\documentclass{report}
\usepackage{amsmath,amssymb}
\setlength{\parindent}{0mm}
\setlength{\parskip}{1em}
\begin{document}
\begin{center}
\rule{6in}{1pt} \
{\large Jeremie Gaidamour \\
{\bf MueLu: Designing an extensible framework for multigrid algorithm research.}}

Sandia National Laboratories \\ Mailstop 1320 \\ P O Box 5800 \\ Albuquerque \\ NM \\ 87185-1320
\\
{\tt jngaida@sandia.gov}\\
Jonathan J. Hu\\
Christopher M. Siefert\\
Raymond S. Tuminaro\end{center}

As part of the Sandia Trilinos project, we are developing a new
object-oriented multigrid solver. This new library allows for great
flexibility on the choice of inter-grid operators (geometric,
aggregate-based, F/C) and can be tuned to choose strategies that are
tailored for specific problems. In particular, the library is intended to
make available new multigrid methods based on energy minimization. These
methods extend the applicability of AMG methods to challenging problems
like those arising from systems of PDEs. For instance, the sparsity
pattern of the intergrid operator can be explicitly specified to control
the cost and robustness of the method.

To take advantage of the flexibility of such algorithms, control over all
the components of the algorithm must be available via the high-level
interface. We want to enable advanced users to customize deeply all the
components of a multigrid solver in an easy way. Special attention was
paid to the code design during development to provide a large and
straight-forward extensibility.

In this talk, we first discuss the underlying design philosophies. We
then give an overview of the framework. We discuss how and why modern
programming concepts facilitate the development of such framework. We
illustrate how various multigrid methods can be assembled, with a
particular focus on energy minimization. We will conclude with some
numerical examples that demonstrate the new capabilities. Finally, we
discuss future plans.


\end{document}
