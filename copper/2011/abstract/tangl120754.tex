\documentclass{report}
\usepackage{amsmath,amssymb}
\setlength{\parindent}{0mm}
\setlength{\parskip}{1em}
\begin{document}
\begin{center}
\rule{6in}{1pt} \
{\large Lei Tang \\
{\bf Parallel Adaptive Mesh Refinement for FOSLS-AMG}}

Department of Applied Math \\ 526 UCB \\ Boulder CO \\ 80309-0526
\\
{\tt tangl@colorado.edu}\\
Marian Brezina\\
Tom Manteuffel\\
Steve McCormick\\
John Ruge\end{center}

This talk introduces new adaptive mesh refinement (AMR) strategies for first-order system
least-squares (FOSLS) finite elements in conjunction with algebraic
multigrid (AMG) methods in the context of nested iteration (NI). The goal
is to reach a certain error tolerance with the least amount of
computational cost and nearly uniform distribution of the error over all
elements. To accomplish this, the refinement decisions at each refinement
level are determined based on minimizing the
``accuracy-per-computational-cost" (ACE) efficiency measure that take
into account both error reduction and computational cost. The
NI-FOSLS-AMG-ACE approach
produces a sequence of refinement levels in which the error is equally
distributed across elements on a relatively coarse grid. Once the
solution is numerically resolved, refinement becomes nearly uniform.
Accommodations of the ACE approach to massively distributed memory
architectures involve a geometric binning strategy to reduce
communication cost. Load balancing begins at very coarse levels. Elements
and nodes are redistributed using parallel quadtree structures and a
space filling curve. This automatically ameliorates load balancing issues
at finer levels. Numerical results show that the NI-FOSLS-AMG-pACE
approach is able to provide highly accurate approximations to rapidly
varying solutions using a small number of work units. Excellent weak and
strong scalability are demonstrated on 4,096 processors for problems with
15 million biquadratic elements.


\end{document}
