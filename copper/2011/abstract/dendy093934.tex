\documentclass{report}
\usepackage{amsmath,amssymb}
\setlength{\parindent}{0mm}
\setlength{\parskip}{1em}
\begin{document}
\begin{center}
\rule{6in}{1pt} \
{\large Joel E Dendy \\
{\bf Using Refined Operators to Define A Cell-Structure-Preserving Multigrid Method for the Diffusion Equation}}

Los Alamos National Laboratory \\ P O Box 1663 \\ MS-B284 \\ Los Alamos \\ NM 87545
\\
{\tt jed@lanl.gov}\\
John D. Moulton\end{center}

We consider the standard cell-centered discretization of the diffusion
equation on a rectangle with discontinuous diffusion coefficient.
Multigrid methods that preserve this cell-based structure are desirable
in some applications, such as those that employ local grid refinement.
Previously we considered an extension of black box multigrid with a
coarsening factor of three, as this approach naturally preserves the
cell-based structure. However, a coarsening factor of two is more common
in grid refinement algorithms, and in this case the standard application
of black box multigrid ignores the cell-based structure and coarsens the
structured dual grid. Moreover, using a naive generalization of the
interpolation operator that preserves the cell-based structure in
conjunction with variational coarsening leads to a twenty-five point
coarse-grid
stencil.

In this work we develop a variational coarsening algorithm that
coarsens by a factor of two, generates nine-point coarse-grid
stencils, and preserves the cell-based structure. We first consider the
zero-removal case and extend the difference operator on the cell-centers
of the finest grid to be a difference operator on the underlying grid
consisting of cell-centers, cell-vertices and cell-edges. For this
extended operator we derive an operator-induced interpolation, which is
used to coarsen variationally to a difference operator on the grid of
cell-vertices. This second operator is coarsened in the standard black
box multigrid fashion to the grid of coarse-grid cell centers. The
resulting operator has a nine-point stencil, and the resulting
interpolation operator from fine-grid cell-centers to coarse-grid
cell-centers has the same stencil as the transpose of bilinear
interpolation. Next we consider non-zero-removal and show how to extend
the above method to handle this case. We consider anisotropic coefficient
problems and show how to modify the above method to handle this case. We
show that the method also works for grids with dimensions that are not
powers of two. We present some comparisons of the new method with classic
BoxMG.


\end{document}
