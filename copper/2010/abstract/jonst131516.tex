\documentclass{report}
\usepackage{amsmath,amssymb}
\setlength{\parindent}{0mm}
\setlength{\parskip}{1em}
\begin{document}
\begin{center}
\rule{6in}{1pt} \
{\large T.B. Jonsthovel \\
{\bf On the use of rigid body modes to the deflated preconditioned conjugate gradient method}}

Stevinweg 1 \\ 2628CN \\ Delft \\ the Netherlands
\\
{\tt t.b.jonsthovel@tudelft.nl}\\
M.B. van Gijzen\\
C. Vuik\end{center}

Finite element computations are indispensable for the simulation of
material behavior. Recent developments in visualization and meshing
software give rise to high-quality but very large meshes. As a result,
large systems with millions of degrees of freedom need to be solved. In
our application, the finite element stiffness matrix is symmetric,
positive definite and therefore the Preconditioned Conjugate Gradient
(PCG) method is our method of choice. The PCG method is also well suited
for parallel applications which are needed in practical applications.

Many finite element computations involve simulation of {\it inhomogenous}
materials. These materials lead to large jumps in the entries of the
stiffness matrix. We have shown that these jumps slow down the
convergence of the PCG method and that by decoupling of those regions
with a deflation technique a more robust PCG method can be constructed:
the Deflated Preconditioned Conjugate Gradient (DPCG) method.

The DPCG method uses deflation vectors that contain the rigid body modes
of sets of elements with similar properties. We will derive a cheap and
general applicable method to compute those rigid body modes. We also
provide a mathematical justification of our approach. Finally, we will
discuss numerical experiments on composite materials to validate our
results.


\end{document}
