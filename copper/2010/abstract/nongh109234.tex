\documentclass{report}
\usepackage{amsmath,amssymb}
\setlength{\parindent}{0mm}
\setlength{\parskip}{1em}
\begin{document}
\begin{center}
\rule{6in}{1pt} \
{\large Hung D Nong \\
{\bf Trilinos Interface in SEACISM}}

Sandia National Labs \\ CSRI/133 \\ P O Box 5800 \\ Albuquerque \\ NM 87185-1318
\\
{\tt hdnong@sandia.gov}\end{center}

SEACISM is an active and on-going project to develop a scalable,
efficient and accurate Community Ice Sheet Model for predictive ice sheet
simulation at the petascale. It will eventually be integrated into the
Community Climate System Model as the ice sheet component.
With Glimmer, an existing "shallow-ice approximation" model, as the
starting point, SEACISM will incorporate to the model improved ice
dynamics and representations of subglacial hydrology and basal processes.
It will also consider using higher-resolution grids and realistic
treatments of sub-shelf melting, iceberg calving and grounding-line
motion. In addition, it will couple to global climate models such as the
atmospheric and oceanic in order to be able to simulate the rapid dynamic
changes currently observed and contributing to global sea level.

Computationally, SEACISM will rely on Trilinos for efficiency and
scalability. Trilinos has capabilities of performing partitioning and
load balancing to achieve good performance and scalability of parallel
codes, i.e., to guarantee that all processors have approximately the same
amount of data (or computation) while minimizing inter-processor
communication. Trilinos hosts a wide range of different linear solvers,
preconditioning techniques and nonlinear solvers. It also provides users
with strategies for optimizing the problem solving process.

This talk will focus on the computational aspect and software development
side of SEACISM. I will present an overview of how Trilinos will be used
in SEACISM and what computational impacts this will result in. In
particular, some preliminary results will be presented to illustrate
computational improvements acquiring from utilizing Trilinos in SEACISM.





* Note: I would like this abstract to be placed in the "Environmental
Science Applications" session.


\end{document}
