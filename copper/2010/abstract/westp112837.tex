\documentclass{report}
\usepackage{amsmath,amssymb}
\setlength{\parindent}{0mm}
\setlength{\parskip}{1em}
\begin{document}
\begin{center}
\rule{6in}{1pt} \
{\large Chad Westphal \\
{\bf An Adaptive Weighted Norm Least-Squares Approach to Convection Dominated Diffusion Problems }}

Wabash College \\ Dept of Math/CS \\ 301 W Wabash Ave \\ Crawfordsville \\ IN 47933
\\
{\tt westphac@wabash.edu}\end{center}

Convection dominated diffusion problems generally have solutions that are
difficult to approximate numerically. Solutions with boundary layers or
high gradients in directions normal to the convection are common and, as
such, adaptive mesh refinement is often used to efficiently resolve the
solution. This talk details current progress on an iterative approach
that works similar to (and in conjunction with) adaptive mesh refinement
to adaptively adjust the least-squares functional norm on which the
discrete variational problem is defined.

In the general least-squares framework the finite element discretization
scheme is defined by optimization principles, essentially minimizing the
norm of the residual of a suitable first-order formulation of the
problem. This approach is well known and has been applied to a wide class
of problems. In the adaptively weighted approach we consider, the norm in
which the minimization is defined is a globally weighted $L^2$ norm,
where the weight functions are adaptively determined by an approximate
coarse solution. In essence this process is one of changing the metric of
the approximation space in an optimal way as the solution is refined. In
this talk we'll give an overview of the details of the approach and
numerical results for convection dominated problems with boundary layers.


\end{document}
