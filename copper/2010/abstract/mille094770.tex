\documentclass{report}
\usepackage{amsmath,amssymb}
\setlength{\parindent}{0mm}
\setlength{\parskip}{1em}
\begin{document}
\begin{center}
\rule{6in}{1pt} \
{\large Christopher, W. Miller \\
{\bf Assessment of Collocation and Galerkin Approaches to Partial Differential Equations With Random Data}}

7521 Maple Ave \\ Apt 10 \\ Takoma Park \\ MD 20912
\\
{\tt cmiller@math.umd.edu}\\
Raymond, S. Tuminaro\\
Eric, T. Phipps\\
Howard, C. Elman\end{center}

We compare the performance of two methods, the stochastic Galerkin method and
the stochastic collocation method for solving partial differential
equations (PDEs) with random data. The stochastic Galerkin method
requires the solution of a single linear system that is several orders
larger than linear systems associated with deterministic PDEs. The
stochastic collocation method requires many solves of deterministic PDEs,
which allows the use of existing software. However, the number of systems
that need to be solved for the stochastic collocation method can be
several times larger than the number of unknowns in the stochastic
Galerkin system.

We implement both of the above methods using the Trilinos software
package and we assess their cost and performance. The implementations in
Trilinos are known to be efficient which, allows
for a realistic assessment of the computational complexity of the
methods. We also develop a cost model for both methods that allows us to
examine asymptotic behavior.


\end{document}
