\documentclass{report}
\usepackage{amsmath,amssymb}
\setlength{\parindent}{0mm}
\setlength{\parskip}{1em}
\begin{document}
\begin{center}
\rule{6in}{1pt} \
{\large Carl T Kelley \\
{\bf Rank-deficient and Ill-conditioned Nonlinear Least Squares Problems}}

Dept of Mathematics \\ Box 8205 \\ NC State University \\ Raleigh \\ NC 27695-8205
\\
{\tt tim\_kelley@ncsu.edu}\\
Ilse Ipsen\\
Scott Pope\end{center}

It is easy to construct models with nonlinearly dependent parameters,
parameters to which the model is insensitive, or redundant parameters.
When one uses conventional nonlinear least squares methods to solve
these problems, the iteration can perform poorly. A common remedy for
this is a combination of the Levenberg-Marquardt method and a truncated
singular value decomposition. We will show how this approach is
affected by ill-conditioning and errors in the evaluations of
the residual and Jacobian. We show how subset selection
can be applied to the Jacobian to diagnose problems of this type
and improve the quality of the results. These results are motivated by
applications to cardiovascular modeling.


\end{document}
