\documentclass{report}
\usepackage{amsmath,amssymb}
\setlength{\parindent}{0mm}
\setlength{\parskip}{1em}
\begin{document}
\begin{center}
\rule{6in}{1pt} \
{\large Lei Tang \\
{\bf Parallel Efficiency Based Adaptive Local Refinement}}

Dept of Applied Math \\ 526 UCB \\ Boulder \\ CO 80309
\\
{\tt tangl@colorado.edu}\\
Marian Brezina\\
Jose Garcia\\
Tom Manteuffel\\
Steve McCormick\\
John Ruge\end{center}

We propose a new adaptive local refinement (ALR) strategy, the goal of
which is to reach a given error
tolerance with the least amount of computational cost. This strategy is
especially attractive in the setting
of a first-order system least-squares (FOSLS) finite element formulation
in conjunction with algebraic multigrid
(AMG) methods in the context of nested iteration (NI). To accomplish
this, the refinement decisions are determined
based on minimizing the predicted `accuracy-per-computational-cost'
efficiency (ACE). The nested iteration approach
produces a sequence of refinement levels in which the error is equally
distributed across elements on a
relatively coarse grid. Once the solution is numerically resolved,
refinement becomes nearly uniform.

This talk will first describe the algorithm and demonstrate its
efficiency on a simple test problem. Then,
modifications that yield an efficient parallel algorithm will be
discussed. These involve a geometric binning
strategy to reduce communication cost. Load balancing begins at coarse
levels using a parallel quad-tree and a space filling curve. We show that
this automatically ameliorates load balancing issues at finer levels.


\end{document}
