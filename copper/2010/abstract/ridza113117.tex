\documentclass{report}
\usepackage{amsmath,amssymb}
\setlength{\parindent}{0mm}
\setlength{\parskip}{1em}
\begin{document}
\begin{center}
\rule{6in}{1pt} \
{\large Denis Ridzal \\
{\bf A Robust Matrix-Free SQP Method for Large-Scale Optimization}}

Optimization and UQ \\ Sandia National Laboratories \\ P O Box 5800 \\ MS 1320 \\ Albuquerque \\ NM 87185-1320
\\
{\tt dridzal@sandia.gov}\end{center}

Optimal design, optimal control, parameter estimation and inverse
problems are ubiquitous in science and engineering. Their mathematical
models often involve large-scale PDE-constrained optimization problems
that can be solved numerically using sequential quadratic programming
(SQP) methods. For this class of problems, however, the inexactness in
the iterative solution of linear systems, typically due to the need for
matrix-free linear algebra, severely limits the effectiveness of a
conventional SQP approach. Recently, several strategies have been
proposed that aim at rigorous control of inexactness via a dynamic
management of stopping tolerances for linear solvers, based on the
overall progress of the optimization algorithm.

In this talk, we revisit an inexact matrix-free trust-region SQP
algorithm for equality-constrained optimization, and demonstrate its
effectiveness on several classes of problems for which conventional SQP
methods fail to converge. In addition, we investigate subtle yet
important algorithmic differences between the inexact trust-region
approach and a recently introduced line-search technique and identify
possible areas of improvement in both cases. Finally, we briefly examine
the challenges in extending the inexactness-control mechanisms of the
trust-region approach to the case of inequality constraints.


\end{document}
