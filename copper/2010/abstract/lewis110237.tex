\documentclass{report}
\usepackage{amsmath,amssymb}
\setlength{\parindent}{0mm}
\setlength{\parskip}{1em}
\begin{document}
\begin{center}
\rule{6in}{1pt} \
{\large Robert M. Lewis \\
{\bf A squared dissimilarity approach to the molecular embedding problem}}

Department of Mathematics \\ College of William and Mary \\ P O Box 8795 \\ Williamsburg \\ VA 23187-8795
\\
{\tt rmlewi@wm.edu}\\
Stephen G Nash\end{center}

We discuss an efficient computational approach to the embedding problem
in structural molecular biology.  This is the determination of a
molecule's three-dimensional structure from information about its
interatomic distances.  In the work discussed here, this information is
conveyed by lower and upper bounds on the distances which are estimated
by nuclear magnetic resonance (NMR) spectroscopy and a priori knowledge
of chemical structure.

Given a molecule with n atoms, in our approach we treat the n(n-1)/2
interatomic distances (dissimilarities, more precisely) as independent
variables, rather than parameterizing the problem in terms of the 3n
Cartesian coordinates of the atoms.  A classical result from distance
geometry enables us to steer the set of dissimilarities to one that
corresponds to the squared interpoint distances of an actual
configuration of atoms in three dimensions.

This formulation leads to a large-scale bound constrained, nonconvex
rank-constrained matrix optimization problem with a spectral objective.
 At first glance our approach appears to result in a much
larger, and therefore more computationally expensive, problem than one
would encounter in a coordinate-parameterized approach.  However, as we
discuss, the computational costs are actually quite tractable. Moreover,
the formulation leads to an optimization problem that, though
large, is relatively easy to solve, as is demonstrated by the numerical
results we present.

We also comment on the difference between real NMR data sets and the
synthetic data sets that are frequently used in the mathematical
literature on molecular embedding.  In particular, real NMR data sets
are sparse and have other features (such as lack of stereospecificity
for some atoms) that make realistic molecular embedding problems more
difficult than synthetic problems.

\end{document}
