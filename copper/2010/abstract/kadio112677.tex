\documentclass{report}
\usepackage{amsmath,amssymb}
\setlength{\parindent}{0mm}
\setlength{\parskip}{1em}
\begin{document}
\begin{center}
\rule{6in}{1pt} \
{\large Samet Y. Kadioglu \\
{\bf ANALYSIS OF A SELF-CONSISTENT IMEX METHOD FOR TIGHTLY COUPLED NON-LINEAR SYSTEMS}}

Fuels Modeling and Simulation Department \\ Idaho National Laboratory \\ P O Box 1625 \\ MS 3840 \\ Idaho Falls \\ ID \\ 83415
\\
{\tt samet.kadioglu@inl.gov}\\
Dana  A. Knoll\\
Robert  B.  Lowrie\end{center}

We introduce a mathematical analysis for our self-consistent
Implicit/Explicit (IMEX) algorithm. This algorithm is designed to produce
second order time convergent solutions to multi-physics
and multiple time scale fluid problems. The algorithm is a combination of
an explicit block that solves the non-stiff
part and an implicit block that solves the stiff part of the problem. The
explicit block is always solved inside the implicit block as part of the
non-linear function evaluation making use of the Jacobian-Free Newton
Krylov (JFNK) method. In this way, there is a continuous interaction
between the implicit and explicit blocks meaning that the improved
solutions (in terms of time accuracy) at each non-linear iteration are
immediately felt by the explicit block and the improved explicit
solutions are
readily available to form the next set of non-linear residuals. This
continuous interaction between the two algorithm blocks results in an
implicitly balanced algorithm in that all the non-linearities due to
coupling of different time terms are converged. In other words, we obtain
a self-consistent
IMEX method that eliminates the order reduction in time accuracy for
certain type of problems that a classical IMEX method can suffer from. We
note that the classic IMEX method splits the operators such a way that
the implicit and explicit blocks are executed independent of each other,
and this may lead to non-converged non-linearities therefore time
inaccuracies for certain models. In this study, we provide a mathematical
analysis (modified equation analysis) that examines and compares the time
behavior of our self-consistent IMEX method versus the classic IMEX
method. We also provide computational results to verify our analytical
findings.


\end{document}
