\documentclass{report}
\usepackage{amsmath,amssymb}
\setlength{\parindent}{0mm}
\setlength{\parskip}{1em}
\begin{document}
\begin{center}
\rule{6in}{1pt} \
{\large Bj\"orn Gmeiner \\
{\bf Validation and Optimization of the Convergence Rate on Semi-structured Meshes using the LFA}}

Department of Computer Science \\ System Simulation \\ Cauerstr 6 \\ 91058 Erlangen \\ Germany
\\
{\tt bjoern.gmeiner@informatik.uni-erlangen.de}\\
Tobias Gradl\\
Francisco Gaspar\\
Ulrich R\"ude\end{center}

The local Fourier analysis (LFA) is an excellent tool to estimate
asymptotic convergence factors for iterative solvers. The assumptions for
the LFA incorporate structured grids, which makes the LFA hard to apply
for arbitrary Finite Element (FE) meshes. Semi-structured meshes at least
partly fulfill this requirement. In this talk we compare measured
convergence rates from the Finite Element program Hierarchical Hybrid
Grids (HHG) with predictions of the LFA for a two-grid multigrid cycle.
HHG operates on 15-point stencils resulting from a Finite Element
discretization on semi-structured three-dimensional tetrahedral meshes.
Using these stencils the accordance for four iterative smoothers between
a structured region of HHG and the LFA for differently shaped
tetrahedral elements will be discussed. Representatives for the
point-wise relaxation schemes are the Jacobi, a lexicographic and a
four-color Gauss-Seidel smoother. Also results of a lexicographic line
smoother are presented.

Further we want to have a look how predictions of structured regions can
improve the overall convergence by redistributing the smoothing steps
over the domain. The convergence rates of the structured regions will
control this redistribution procedure. To determine the convergence
rates, we compare two approaches, one based on a look-up table, the other
one based on a direct LFA prediction, in a numerical experiment.

Key words: multigrid, local fourier analysis, convergence rates, smoothers


\end{document}
