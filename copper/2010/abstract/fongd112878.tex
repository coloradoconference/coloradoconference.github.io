\documentclass{report}
\usepackage{amsmath,amssymb}
\setlength{\parindent}{0mm}
\setlength{\parskip}{1em}
\begin{document}
\begin{center}
\rule{6in}{1pt} \
{\large David C.-L. Fong \\
{\bf LSMR: An iterative algorithm for least-squares problems}}

121 Campus Dr  \\ Apt 1109B \\ Stanford \\ CA 94305
\\
{\tt clfong@stanford.edu}\\
Michael A. Saunders\end{center}

\newcommand{\T}{^T\!}
\newcommand{\norm}[1]{\|#1\|}


An iterative method is presented for solving linear systems $Ax=b$ and
$\min \norm{Ax-b}_2$, with $A$ being large and sparse, or a fast linear
operator. The method is based on the Golub-Kahan bidiagonalization
process. It is analytically equivalent to the standard method of MINRES
applied to the normal equation $A\T Ax = A\T b$, so that the quantities
$\norm{A\T r_k}$ are monotonically decreasing (where $r_k = b - Ax_k$ is
the residual for the current iterate $x_k$). In practice we observe that
$\norm{r_k}$ also decreases monotonically. Compared to LSQR, for which
only $\norm{r_k}$ is monotonic, it is safer to terminate LSMR early.


\end{document}
