\documentclass{report}
\usepackage{amsmath,amssymb}
\setlength{\parindent}{0mm}
\setlength{\parskip}{1em}
\begin{document}
\begin{center}
\rule{6in}{1pt} \
{\large Luis Chac\'on \\
{\bf On scalable solvers for extended magnetohydrodynamics}}

PO Box 2008 \\ MS 6169 \\ Oak Ridge \\ TN 37830
\\
{\tt chaconl@ornl.gov}\end{center}

Recently, a fully implicit Newton-Krylov-based solver for resistive
MHD has been developed {[}1,2{]}, which features remarkable properties
both in terms of algorithmic and parallel efficiency. Algorithmically,
the solver was shown to be optimal, demonstrating a computational
complexity which is linearly increasing with the number of degrees
of freedom. In parallel, the approach demonstrated excellent scaling
up to 4096 processors and 0.13 billion unknowns. The approach rests
on a novel preconditioning strategy (oft-called ``physics-based''),
which achieves, via a Schur-complement-based factorization of the
linearized system, an effective parabolization of the otherwise hyperbolic
MHD equations, thereby making them amenable to a classical multigrid
treatment.

While such a development represents an important milestone in long-frequency
computational MHD, it still suffers from various limitations which
need to be addressed, both from the physics and from the implementation
standpoints. From the physics point of view, the demonstration in
Ref. 1 was performed on the well-known visco-resistive MHD model,
which, while useful in many important applications, falls short of
the requirements in modern MHD simulation tools. In particular, ``extended
MHD'' effects such as electron Hall physics, parallel electron transport,
and ion gyroviscosities are known to play an important role in certain
regimes of interest (notably, hot, moderate-density plasmas, such
as those found in the solar corona, the Earth's magnetosphere, and
in thermonuclear magnetic-confinement fusion devices). However, numerically,
the inclusion of such physical effects is highly nontrivial. Both
electron Hall physics and ion gyroviscosities introduce strongly hyperbolic
couplings, resulting in dispersive normal modes (waves) with frequencies
$\omega$ scaling with the square of the wavenumber $k^{2}$. On the
other hand, parallel electron transport, while parabolic in nature,
suffers in the discrete from an array of numerical problems stemming
from the exceedingly large parallel-to-perpendicular transport coefficient
ratios, $\chi_{\parallel}/\chi_{\perp}\sim10^{8}$. Such problems
include the near-singularity of the resulting matrices, the lack of
positivity of the discrete formulation (even though the continuum
features a maximum principle), and the need for high-order spatial
discretizations to avoid numerical pollution of the perpendicular
dynamics by the parallel one.

Implementation-wise, and with some of the applications of interest
in mind, the multigrid strategy used in Ref. 1 needs to be extended
to handle general curvilinear geometries, and in particular those
featuring a pole (e.g., cylindrical and toroidal). Such singular geometries
stress our geometric multigrid implementation, and require special
care for a viable multigrid solver.

In this talk, we will describe our progress in addressing some of
these limitations. In particular, we will describe our efforts to
include electron Hall physics (partially described in {[}2{]}) and
parallel electron transport effects. We will also describe our strategy
for a viable multigrid solver in polar coordinate systems. We will
demonstrate the effectiveness of our approach with numerical examples.

\noindent {[}1{]} L. Chac\'on, ``An optimal, parallel, fully
implicit Newton-Krylov solver for three-dimensional visco-resistive
magnetohydrodynamics,'' \emph{Phys. Plasmas}, \textbf{15},
056103 (2008)

\noindent {[}2{]} L. Chac\'on, ``Scalable solvers for
3D magnetohydrodynamics,'' {\em J. Physics: Conf. Series},
{\bf 125}, 012041 (2008)


\end{document}
