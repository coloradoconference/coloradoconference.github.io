\documentclass{report}
\usepackage{amsmath,amssymb}
\setlength{\parindent}{0mm}
\setlength{\parskip}{1em}
\begin{document}
\begin{center}
\rule{6in}{1pt} \
{\large David Mokrauer \\
{\bf Generating a Feasible Path Between Stable Molecular Geometries with Continuous Steepest Descent}}

2431 Condor Court \\ Raleigh \\ NC \\ 27615
\\
{\tt mokrauerd@gmail.com}\\
C. T. Kelley\end{center}

The potential energy of a molecule is a function of its geometry. Many
molecules exist in multiple stable geometric conformations which
correspond to local minima on a potential energy surface (PES). These
molecules can be excited by the introduction of light to discrete higher
energy states from that initial "ground" state. These higher energy
states have completely different potential energy surfaces. A
conformation that was a local minimum on the ground state may no longer
be a local minimum once it's energy has increased to an excited state.
Having become unstable the molecule will relax to a new stable geometry
in this state. Either by emitting energy or by gaining more energy the
molecule can once again transfer states. By selecting a specific set of
excited states it is possible to find a path that will lead from one
stable minimum on the ground state to a different one.

A molecule is determined by 3N-6 coordinates where N is the number of
atoms thus the subspace in which we solve this problem is only a subset
of those coordinates. Each point on these level surfaces is a minimum
energy conformation with the values of the subset of chosen coordinates
as constraints. Evaluating the energy at gridpoints on the surface
requires interfacing with computational chemistry software. Function
evaluations using this software can take up to an hour with the test
molecules that we have been using. Generating portions (or all) of these
potential energy surfaces is a significant computational burden.
Unfortunately, the software can only adequately compute the energy with a
good initial guess at the geometry due to the function itself being a
constrained optimization. This necessity of good initial geometries for
function evaluation prevents us from casting a net and evaluating all of
the gridpoints at once in a trivially parallel scheme. We adapted to this
problem by using initial iterates from successfully evaluated neighboring
points on the grid. An algorithm which manages the desired gridpoints in
a dependant manner so as to ensure that the function evaluations recieve
good initial iterates and that takes advantage of as many computing
resources as are available was developed.

These surfaces are continuously differentiable and can be interpolated
accurately by polynomial splines through a set of gridpoints. Since we
are mimicking the natural process of relaxation on these interpolated
level surfaces, continuous steepest descent is the proper optimization
method. The transitions between states requires selection rules before
the optimization on the next state. There are many choices for selection
rules including user interface, largest gradient, or largest angle
between the gradient and the vector between the current and initial
point. By integrating the algorithm for generating surfaces with an ode
solver and a set of selection rules we can simulate the natural
transformation processes undergone by molecules.


\end{document}
