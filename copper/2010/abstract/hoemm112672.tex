\documentclass{report}
\usepackage{amsmath,amssymb}
\setlength{\parindent}{0mm}
\setlength{\parskip}{1em}
\begin{document}
\begin{center}
\rule{6in}{1pt} \
{\large Mark Hoemmen \\
{\bf Communication-avoiding Krylov subspace methods}}

Sandia National Laboratories \\ New Mexico \\ PO Box 5800 \\ Albuquerque \\ NM 87185-1320
\\
{\tt mhoemme@sandia.gov}\\
James Demmel\\
Marghoob Mohiyuddin\\
Kathy Yelick\end{center}

Krylov subspace methods (KSMs) are numerical algorithms for
solving large, sparse linear systems of equations and
eigenvalue problems. Although they are commonly used and
often highly optimized, most KSMs achieve only a small
fraction of peak arithmetic performance of the computers on
which they are run. This occurs on almost all computers,
from workstations to massively parallel supercomputers. The
cause is that the performance of most commonly used KSMs is
bound by the speed of communication -- moving data between
processors or between levels of the memory hierarchy --
rather than by the speed of arithmetic. Communication is
much slower than arithmetic, and is only getting slower
relative to arithmetic as hardware evolves.

In previous works, we proposed new "communication-avoiding"
Krylov methods to address this problem. These require much
less data movement between levels of the memory hierarchy,
and between processors in parallel, than standard KSMs. In
this talk, we will present new communication-avoiding Krylov
methods, and discuss new numerical experiments and
performance results for different platforms.


\end{document}
