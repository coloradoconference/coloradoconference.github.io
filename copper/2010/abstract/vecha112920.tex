\documentclass{report}
\usepackage{amsmath,amssymb}
\setlength{\parindent}{0mm}
\setlength{\parskip}{1em}
\begin{document}
\begin{center}
\rule{6in}{1pt} \
{\large Eugene Vecharynski \\
{\bf Preconditioned gradient-type methods for computing extreme singular values}}

Department of Mathematical and Statistical Sciences University of Colorado Denver \\ 1250 Fourteenth Street \\ Suite 600 \\ Denver \\ CO 80202
\\
{\tt yaugen.vecharynski@ucdenver.edu}\\
Andrew Knyazev\end{center}

We consider a problem of computing a number of extreme singular values
and the corresponding left and right singular vectors of
a large possibly sparse matrix. We, first, review several existing
methods which are currently available in the literature
and discuss the complications which appear, e.g., when the triple
corresponding to the smallest singular value is required. Next, we
describe how gradient-type methods can be applied to the problem and
suggest an iterative scheme for computing the extreme singular triples.
Finally, we discuss possible ways of introducing a preconditioner to
improve the convergence to the smallest singular values and the
corresponding singular vectors.


\end{document}
