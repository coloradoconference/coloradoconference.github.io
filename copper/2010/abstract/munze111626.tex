\documentclass{report}
\usepackage{amsmath,amssymb}
\setlength{\parindent}{0mm}
\setlength{\parskip}{1em}
\begin{document}
\begin{center}
\rule{6in}{1pt} \
{\large Steffen M\"unzenmaier \\
{\bf Treatment of Interface Conditions for Coupled Stokes-Darcy Flow in Least Squares Finite Element Methods}}

Institut f\"ur Angewandte Mathematik \\ Gottfried Wilhelm Leibniz Universit\"at Hannover \\ Welfengarten 1 \\ D-30167 Hannover \\ Germany
\\
{\tt muenzenm@ifam.uni-hannover.de}\\
Gerhard Starke\end{center}

In this talk we will present a least squares based finite element method
for the coupled Stokes-Darcy flow. These problems typically arise in
investigating the interaction between surface and groundwater flow. Both
subproblems are well understood and there are several approaches in
setting up an elliptic least squares functional. We choose a
velocity-stress for Stokes flow and a flux-pressure formulation for Darcy
flow where all process variables involved in interface conditions are
directly available. Nonetheless formulating the coupled problem using
appropriate boundary conditions on the interface is essential to
guarantee well-posedness and the property of the least squares functional
as an error estimator. Still, while ensuring the well-posedness of the
problem, the practicability of the method has to be considered.

Numerical experiments using Raviart-Thomas (for each stress component and
the flux) and standard conforming piecewise polynomials (for the
velocities and the hydraulic potential) will be presented. Using the
least squares functional as local error estimator an adaptive refinement
strategy will be investigated.


\end{document}
