\documentclass{report}
\usepackage{amsmath,amssymb}
\setlength{\parindent}{0mm}
\setlength{\parskip}{1em}
\begin{document}
\begin{center}
\rule{6in}{1pt} \
{\large Eric T Phipps \\
{\bf Analysis of Intrusive Stochastic Galerkin Methods for Uncertainty Quantification of Nonlinear Stochastic PDEs}}

Sandia National Laboratories \\ Optimization and Uncertainty Quantification Department \\ PO Box 5800 MS-1318 \\ Albuquerque \\ NM 87185
\\
{\tt etphipp@sandia.gov}\end{center}

A critical component of predictive computational simulation is the
ability to effectively characterize uncertainties in simulation input
data and quantify the effects of those uncertainties on simulation
results. Frequently the system of interest is modeled by the solution to
one or more partial differential equations (PDEs) where random variables
or fields with known probability distributions model data uncertainties.
In this setting, stochastic Galerkin methods are a powerful family of
methods for quantifying uncertainties in PDE solutions when they exhibit
a high-degree of regularity with respect to uncertain input data. However
implementing these methods in large-scale computational engineering codes
is hampered by the fact that they require formulating and solving a fully
coupled spatial-stochastic nonlinear system that is distinctly different
from the deterministic nonlinear system they were originally designed
for. Usually this requires rewriting the simulation code to compute the
stochastic Galerkin nonlinear residuals and employing specialized solver
methods for solving the resulting fully coupled systems, and thus these
methods are typically referred to as intrusive.

Even with these challenges, good performance has been obtained with these
methods for linear stochastic PDEs due to the many fewer stochastic
degrees-of-freedom required for a given level of accuracy compared to
non-intrusive methods such as stochastic collocation. Unfortunately for
nonlinear stochastic PDEs, the method becomes much more computationally
expensive for problems with large stochastic dimension. In this talk we
investigate the application of these methods to representative nonlinear
stochastic PDEs by quantifying the increased computational cost nonlinear
problems present for both evaluating nonlinear residual equations and
solving the linearized equations in a Newton-type nonlinear solver scheme
using iterative, Krylov-based linear solvers. For problems with large
stochastic dimension, we find the matrix-vector multiplies required by
the iterative solver dominate the total computational cost. We then
present approaches for reducing the cost of the matrix-vector multiplies
based on random field modeling techniques.


\end{document}
