\documentclass{report}
\usepackage{amsmath,amssymb}
\setlength{\parindent}{0mm}
\setlength{\parskip}{1em}
\begin{document}
\begin{center}
\rule{6in}{1pt} \
{\large Abdul Hanan Sheikh \\
{\bf An efficient iterative scheme for the Helmholtz equation with deflation.}}

Room # HB 07 050 \\ EWI Building Makelweg 4 \\ 2628 CD \\ Delft \\ The Netherlands
\\
{\tt hanangul12@yahoo.co.uk}\\
Kees Vuik\\
Domenico Lahaye\end{center}

The Helmholtz equation arises in many physical problems involving steady
state (mechanical, acoustical, thermal, electromagnetic) oscillations.
The complexity of the physical problem leaves no choices except to solve
the equation numerically. The finite difference discretization leads to
sparse, complex symmetric coefficient matrices that become indefinite for
sufficient large wave number. The size of the problem increases with the
wave number as a minimum number of grid points per wavelength is required
to represent the physics correctly. Krylov subspace and Multigrid
techniques have been succesfully applied in a wide range of applications.
For symmetric positive definite problems, the conjugate gradient (CG) is
the method of choice. For indefinite problems, more general Krylov
subspace solvers are to be applied. The fast convergence of these methods
requires some form of preconditioning. The straightforward application of
multigrid as a preconditioner is hampered by slow convergence causes by
eigenmodes corresponding to eigenvalues with a negative real part. To
overcome this difficulty, Shifted-Laplace preconditioners have been
developed by Yogi A. Erlangga [Tech. Report 03-18, DIAM Delft Univ. Tech.
The Netherlands, 2003].The idea of shifted-Laplace preconditioners is to
base the precondioner of the discrete Helmholtz operator with a modified
shift (coefficient before the first order term) for with the multigrid
preconditioner can be shown to satisfaction. It was shown that the
resulting preconditioned GMRES solver has favourable convergence
properties in the sense the required number of iterations only depends
mildly on the wavenumber. Allowing some damping in the shifted Laplace
operator in particalur proves to benefical for speeding up overall
convergence. In more recent work by Yogi and Nabben[On a ultilevel Krylov
method for the Helmholtz equation preconditioned by shifted Laplacian, In
Press], the combined use of deflation and the shifted-Laplace
precondioner was shown to result in a on more performant solver.In this
paper we study the combined use of the shifted-Laplace precondioner with
multigrid deflation. We show that deflation allows to remove unfavourable
modes in the spectrum of the precondioned operator. This allows us to
that the Krylov solver is close to optimal in the sense that the required
number of iterations is almost independent of the wavenumber. Also the
deflation matrix is sparse. Experimental results to support the claims
are presented for a 2-D Helmholtz problem.


\end{document}
