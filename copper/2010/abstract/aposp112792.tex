\documentclass{report}
\usepackage{amsmath,amssymb}
\setlength{\parindent}{0mm}
\setlength{\parskip}{1em}
\begin{document}
\begin{center}
\rule{6in}{1pt} \
{\large Alexis Aposporidis \\
{\bf A Primal-Dual Formulation for the Bingham Flow}}

Emory University \\ 400 Dowman Drive \\ W401 \\ Atlanta \\ GA30322 \\ USA
\\
{\tt aapospo@emory.edu}\\
Alessandro Veneziani\\
Eldad Haber\end{center}

\newcommand{\Du}{\textbf{Du}}
\newcommand{\bu}{\textbf{u}}
\newcommand{\f}{\textbf{f}}
\newcommand{\tr}{\textnormal{tr}}
\newcommand{\bW}{\textbf{W}}

The Bingham flow is an example of a Stokes-type equation with
shear-dependent viscosity. If $\Du=\frac{1}{2}(\nabla \bu + \nabla
\bu^T)$ and $|\Du|=\sqrt{\tr(\Du^2)}$, the equations read
\[ \left\{ \begin{array}{rl} -\nabla \cdot \tau +\nabla p &= \f, \\
-\nabla \cdot \bu &=0 \\ +B.C., \end{array} \right. \]
and
\[ \left\{ \begin{array}{rl} \tau=2\mu \Du+\tau_s \frac{\Du}{|\Du|},
&\textnormal{ if } |\Du| \not=0, \\ |\tau| \leq \tau_s, & \textnormal{ if
} |\Du|=0, \end{array} \right. \]
where the velocity $\bu \in \mathbb{R}^n$, $n=2,3$ and $p \in \mathbb{R}$
are the unknowns and $\mu$, $\tau_s$ are given constants. Due to its
non-differentiability for $\Du =0$, a regularization of the form $\Du =
\sqrt{\tr(\Du^2)+\varepsilon}$ ($\varepsilon >0$) is necessary. It is a
well-known fact that applying a nonlinear solver such as Newton or Picard
to these equations results in a high number of outer iterations,
especially for small choices of $\varepsilon$ \cite{1}. In this talk we
suggest an alternative approach inspired by \cite{6}: We introduce a dual
variable $\bW = \frac{\Du}{|\Du|}$, the equations for the Bingham flow
are then reformulated as
\[ \left\{ \begin{array}{rl} -\nabla \cdot \left( 2\mu \Du + \tau_s \bW
\right) +\nabla p &=\f, \\ -\nabla \cdot \bu &=0, \\ \bW |\Du| &=\Du \\
+B.C. \end{array} \right. \]
We address a few properties of this formulation and its numerical
solution. Moreover, we perform several numerical experiments for solving
the Bingham equations in this formulation, including the lid-driven
cavity test and an example where the analytical solution is known. These
experiments indicate a significant reduction in the number of nonlinear
iterations over the nonlinear solvers of the equations in primal
variables. \\[2mm]

\noindent \textbf{Acknowledgement:}\\
We thank M. Olshanskii for fruitful discussions.

\begin{thebibliography}{99}
\bibitem{1}P. Grinevich, M. Olshanskii, \textit{An Iterative Method for
the Stokes Type Problem with Variable Viscosity}, SIAM J. Sci. Comput.
Volume 31, Issue 5, pp. 3959-3978 (2009)
\bibitem{2}E. Mitsoulis, Th. Zisis, \textit{Flow of Bingham plasics in a
lid-driven square cavity}, J. Non-Newtonian Fluid Mech., 101 (2001), pp.
173-180
\bibitem{3} M. Olshanskii, \textit{Analysis of semi-staggered
finite-difference method with application to Bingham flows}, Comput.
Methods Appl. Mech. Engrg., 198 (2009) pp. 975-985
\bibitem{4}T.C. Papanastasiou, A.G. Boudouvis, \textit{Flows of
Viscoplastic Materials: Models and Computations}, Computers and
Structures Vol. 64, No. 1-4, pp. 677-694, 1997
\bibitem{5}E.A. Muravleva, M.A. Olshanskii, \textit{Two finite-difference
schemes for calculation of Bingham fluid flows in a cavity}, Russ. J.
Numer. Anal. Math. Modelling, Vol. 23, No. 6, pp. 615-634 (2008)
\bibitem{6}T. F. Chan, G. H. Golub, P. Mulet, \textit{A nonlinear primal dual method
for total variation based image restoration}, SIAM J. Sci. Comput., 20 (1999), 1964-1977.
\bibitem{7}E.J. Dean, R. Glowinski, G. Guidoboni, \textit{On the
numerical simulation of Bingham viscoplastic flow: Old and new results},
J. Non-Newtonian Fluid Mech., 142 (2007), pp. 36-62.
\end{thebibliography}


\end{document}
