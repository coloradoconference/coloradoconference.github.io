\documentclass{report}
\usepackage{amsmath,amssymb}
\setlength{\parindent}{0mm}
\setlength{\parskip}{1em}
\begin{document}
\begin{center}
\rule{6in}{1pt} \
{\large DU Lei \\
{\bf A block IDR(s) method for nonsymmetric linear equations with multiple right-hand sides}}

Department of Computational Science and Engineering \\ Graduate School of Engineering \\ Nagoya University \\ Furo-cho \\ Chikusa-ku \\ Nagoya 464-8603 \\ Japan
\\
{\tt lei-du@na.cse.nagoya-u.ac.jp}\\
SOGABE Tomohiro\\
YU Bo\\
YAMAMOTO Yusaku\\
ZHANG Shao-Liang\end{center}

The problem for solving several linear systems of equations with the same
coefficient matrix and different right-hand sides, which can be written
as
\begin{equation}
Ax_i=b_i, A\in R^{n\times n}, b_i\in R^n, i=1,...,m
\end{equation}
appears in many applications, such as in electromagnetic scattering and
structural mechanics.

Many methods have been proposed to solve this kind of problem. One main
class of these methods is the block solvers, including the block CG,
block QMR, block GMRES, block Bi-CGSTAB, etc.
Recently, Based on the induced dimension reduction theorem, the IDR(s)
was developed by Sonneveld and Gijzen[1]. This new family of algorithms
has some distinguishing features, for example, IDR(s) may converge the
solution using at most n+n/s matrix-vector products, IDR(1) is equivalent
to Bi-CGSTAB at the even residuals and IDR(s) with s>1 is competitive
with most Bi-CG based methods.

For these reasons, we proposed the block IDR(s): a block version of
IDR(s) for solving linear systems of equation with multiple right-hand
sides. To define our algorithm, we first gave the block IDR theorem,
which is a generalization of the IDR theorem and proved that the maximum
number of matrix-vector products for block IDR(s) to reach the exact
solution is n+n/s in exact arithmetic, the same as IDR(s) for single
right-hand side. Numerical experiments show the effectiveness of our
proposed method.

Reference:
[1]P. Sonneveld and M. van Gijzen, IDR(s): a family of simple and fast
algorithms for solving large nonsymmetric systems of linear equations,
SIAM J. Sci Comput., 31(2008), pp. 1035-1062.


\end{document}
