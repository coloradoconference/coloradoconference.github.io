\documentclass{report}
\usepackage{amsmath,amssymb}
\setlength{\parindent}{0mm}
\setlength{\parskip}{1em}
\begin{document}
\begin{center}
\rule{6in}{1pt} \
{\large Christophe Calvin \\
{\bf Iterative Eigenvalue Computation on multicore and hybrid architecture}}

CEA/DEN/DANS/DM2S/SERMA \\ CEA Saclay \\ 91191 Gif-sur-Yvette cedex \\ FRANCE
\\
{\tt christophe.calvin@cea.fr}\\
Serge Petiton\\
J\'er\^ome Dubois\\
Erell Jamelot\end{center}

Iterative solvers for eigenvalue computation are widely used in physics
and numerical applications. As for linear solvers, it usually represents
a major computing time in the overall application performance. Thus for
many years now, these kind of numerical kernels have been optimized on
different architectures like massively parallel computers. For a few
years now, new emerging technologies arised based on accelerators, like
CELL or GPGPU. But with these new multicore and hybrid architectures, new
problems emerged like arithmetic accuracy, efficient programming and
hybrid parallelization.
In this talk, we will study the numerical behavior of heterogeneous
systems such as CPU with GPU or IBM Cell processors for some
orthogonalization processes for the Arnoldi method, which are crucials
for eigensolvers. We focus on the influence of the different floating
arithmetic handling of these accelerators with Gram-Schmidt
orthogonalizations (classical, modified or re-orthogonalized versions)
using single and double precision. We will present performance results on
dense and sparse matrices and will discuss about hybrid parallelization
mixing classical message passing with GPGPU highly multithreaded
parallelization.


\end{document}
