\documentclass{report}
\usepackage{amsmath,amssymb}
\setlength{\parindent}{0mm}
\setlength{\parskip}{1em}
\begin{document}
\begin{center}
\rule{6in}{1pt} \
{\large Jok M. Tang \\
{\bf A probing method for computing the diagonal of a matrix inverse}}

University of Minnesota \\ Department of Computer Science and Engineering \\ 4-192 EE/CS Building \\ 200 Union Street SE \\ Minneapolis \\ MN 55455 \\ USA
\\
{\tt jtang@cs.umn.edu}\\
Yousef Saad\end{center}

A probing method is presented to find the diagonal of the inverse of a
sparse, diagonal dominant, matrix. First, some properties of the matrix
inverse are established. This suggests several ways to
determine effective probing vectors. An iterative method is then applied
to solve the resulting sequence of linear systems, from which the
diagonal of the matrix inverse is extracted.

The algorithm is applied to the solution of Dyson's equation in Dynamic
Mean Field Theory (DMFT). DMFT has recently emerged as a very powerful
and reliable tool in physics for the investigation of lattice models of
correlated electrons in a quantum many-body system.


\end{document}
