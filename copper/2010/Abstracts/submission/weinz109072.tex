\documentclass{report}
\usepackage{amsmath,amssymb}
\setlength{\parindent}{0mm}
\setlength{\parskip}{1em}
\begin{document}
\begin{center}
\rule{6in}{1pt} \
{\large Tobias Weinzierl \\
{\bf Peano---A PDE Framework on Octree-Like Adaptive Cartesian Multiscale Grids}}

Institut f�r Informatik 5 \\ Technische Universit�t M�nchen \\ Boltzmannstr 3 \\ 85748 Garching
\\
{\tt weinzier@in.tum.de}\\
Miriam Mehl\end{center}

Almost all approaches to solve a PDE are based upon a spatial discretisation of
the computational domain --- a grid. This paper presents an algorithm to
traverse a hierarchy of $d$--dimensional adaptive cartesian grids represented by
a spacetree. The algorithm uses $2d+2$ stacks as data structure, and the
storage requirements for the pure grid reduce to one bit per vertex for
both the complete grid structure and the multilevel grid relations. Since the
traversal algorithm uses
only stacks, the algorithm's cache hit rate is higher than 99.9 percent, and the
runtime costs per vertex remain constant not depending on the overall number of
vertices, even if the required memory exceeds the main memory available. The
algorithmic approach might be the basis to solve
any $d$--dimensional problem represented by a compact discrete $3^d$--point
operator as they occur e.g.~in image processing or in FEM codes. In
the latter case, one can implement a Jacobi smoother, a Krylov
solver, or a geometric multigrid scheme within the presented traversal scheme
according to literature.
Results for a matrix-free multigrid FAS scheme are presented.
Due to the matrix-free approach, this multigrid scheme inherits the very low
memory footprint and the good cache characteristics from the pure grid
traversal.


\end{document}
