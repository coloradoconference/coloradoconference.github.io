\documentclass{report}
\usepackage{amsmath,amssymb}
\setlength{\parindent}{0mm}
\setlength{\parskip}{1em}
\begin{document}
\begin{center}
\rule{6in}{1pt} \
{\large D. Lahaye \\
{\bf Solving Large Scale Imaging Problems Using Manifold-Mapping Combined With Adjoint Sensitivity }}

Delft Institute for Applied Mathematics \\ Delft University of Technology \\ Mekelweg 4 \\ 2628 CD Delft \\ The Netherlands
\\
{\tt d.j.p.lahaye@tudelft.nl}\\
W.  Mulckhuyse\end{center}

Measuring the electro-magnetic field caused by different
current paths in a work piece allows for a non-invasive
imaging of corrosion spots. Solving this inverse problem
on an industrial scale by a black box still represents
several computational challenges. We formulate the problem
as a PDE-constrained optimization in which one aims at
minimizing the discrepancy between measured and simulated
data in an appropriate norm. The straightforward inclusion
of a finite element procedure in a derivative-free optimization
is prohibitively expensive. The space-mapping technique is a family of
surrogate-based optimization methods that aim at alleviating this
bottleneck. The manifold-mapping technique in particular is an output
space-mapping technique in which the mapping between the fine and coarse
model is constructed in
such a way to guarantee the converge the convergence to a
(local) minimizer of the fine model. In this work we attempt to combine
the manifold-mapping algorithm with adjoint sensitivity hoping to combine
the best of both worlds. We envisage to arrive that an hybrid algorithm
that is faster to convergence (due to
the inclusion of gradient information) while at the same being able to
escape from local minimizers (inclusion of coarse model information).
Preliminary numerical results will illustrate some of the salient points
discussed.


\end{document}
