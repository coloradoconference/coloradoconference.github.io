\documentclass{report}
\usepackage{amsmath,amssymb}
\setlength{\parindent}{0mm}
\setlength{\parskip}{1em}
\begin{document}
\begin{center}
\rule{6in}{1pt} \
{\large E.R. Jessup \\
{\bf Taxonomy for the Automated Tuning of Matrix Algebra Software}}

University of Colorado \\ Department of Computer Science \\ 430 UCB \\ Boulder \\ CO 80309
\\
{\tt jessup@cs.colorado.edu}\\
B. Norris\end{center}

Linear algebra calculations constitute the most time-consuming part
of simulations in diverse fields ranging from atmospheric science to
quantum physics to structural engineering. Reducing the costs of those
computations can have a significant impact on overall routine performance
especially as the size and complexity of scientific computations push
the limits of processor technology. Programming scientific applications
is hard, however, and optimizing them for high performance is even
harder. There is often a large gap between the achieved performance of
applications and the peak available performance, with many applications
achieving 10\% or less of the peak.

The process of converting matrix algebra from algorithm to high-quality
implementation is a complex one. The code developer must create efficient
implementations from scratch or select the appropriate numerical
routines and then devise ways to make these routines run efficiently
on the architecture at hand. Once the numerical routines have been
identified, the process of including them into a larger application
can often be tedious and difficult. The tuning of the application
itself then presents a myriad of options generally centered around one
or more of the following three approaches: manually optimizing code
fragments; using tuned libraries for key numerical algorithms; and,
less frequently, using compiler-based source transformation tools for
loop-level optimizations. At each step, the code developer is thus
confronted with many possibilities, generally requiring expertise in
numerical computation, mathematical software, compilers, and computer
architecture.

We will present our work in progress on a taxonomy of software that can
be used to build highly-optimized matrix algebra software. The taxonomy
will provide an organized anthology of software components and programming
tools needed for that task. The taxonomy will serve as a guide to
practitioners seeking to learn what is available for their programming
tasks, how to use it, and how the various parts fit together. It will
build upon and improve existing collections of numerical software, adding
tools for the tuning of matrix algebra computations. Its interface will
take a high level description of a matrix algebra computation and produce
a customizable optimized template using the software in the taxonomy.
That template will aid the developer at all steps of the process--from
the initial construction of Basic Linear Algebra Subprogram (BLAS)-based
codes through the full optimization of that code. Initially, the tools
will accept a MATLAB prototype and produce optimized Fortran or C.

A number of taxonomies exist to aid the code developer in translation of
matrix algebra algorithms to numerical software. Perhaps the oldest one
is the Netlib Mathematical Software Repository, started in 1985, which
contains freely available software, documents, and databases pertaining
to numerical computing including linear algebra. Contents are provided
as lists of packages or routines, with or without some explanatory words.
In newer work, the Linear Algebra Software Survey lists over sixty items
categorized as support routines, dense direct solvers, sparse direct
solvers, preconditioners, sparse iterative solvers, and sparse eigenvalue
solvers together with a checklist specifying problem types for each entry.
A third example is NIST's Guide to Available Mathematical Software (GAMS)
which includes even more basic matrix algebra software along with software
for a variety of other numerical applications.

Numerical software taxonomies such as those just mentioned are general
and allow relatively stand-alone algorithms to be found, downloaded, and
compiled, or used through a domain-specific Web interface. Operations for
which no library implementation exists or more complex software packages,
however, cannot be accommodated by this function-level indexing and
query capability. They may also not have sufficiently simple common
interfaces that enable Web-based services as is possible for some
optimization problems. For example, the functionality of large toolkits,
such as Trilinos and PETSc, is difficult or impossible to represent and
maintain in existing taxonomies, which at present simply point the user
to the toolkits' home pages. In many cases this idiosyncrasy results in a
taxonomy user's being directed to simpler implementations, rather than to
potentially better options, both in terms of suitability and performance.

One goal of our work is to make a taxonomy that is easy to use.
While GAMS and Netlib are extensive and valuable resources, it is not
always easy to find the proper results from them. In addition, the
taxonomies provide little information about the routines they return.
Our objective is to build a taxonomy that will provide all of the
software needed to take a matrix algebra problem from algorithm
description to a high-performance implementation. The taxonomy will
thus include not only numerical routines but also interfaces to tools
for code tuning. It will not comprise all available software but rather
a collection of routines that allows for efficient programming of a wide
variety of matrix algebra operations. We will pay special attention to
identifying combinations of numerical software and tuning tools that can
be successfully used together. Members of the taxonomy will be annotated
with enough information to explain the purpose of the routines and how
they are used in the code construction process. This practical taxonomy
will thus serve as an educational tool for the computational scientist.

Existing taxonomies serve primarily as user interfaces to numerical
software. While such functionality is also one of our goals, we will
create a richer representation and a set of interfaces that allows
software tools to access the taxonomy information. For example, the same
Netlib server that hosts the numerical software taxonomy also hosts a
performance database for a variety of benchmark codes on an extensive
list of platforms. There are no links, however, from the code of some
of these benchmarks in the taxonomy to the performance results for the
corresponding library or function in the performance database. A natural
extension to the information stored for entries of the taxonomy is to
provide references to existing performance data for benchmarks of that
operation or for applications that use it.


\end{document}
