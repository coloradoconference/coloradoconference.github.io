\documentclass{report}
\usepackage{amsmath,amssymb}
\setlength{\parindent}{0mm}
\setlength{\parskip}{1em}
\begin{document}
\begin{center}
\rule{6in}{1pt} \
{\large Xiaoye S. Li \\
{\bf A Supernodal Approach to ILU with Partial Pivoting}}

Lawrence Berkeley National Laboratory \\ MS 50F-1650 \\ 1 Cyclotron Rd \\ Berkeley \\ CA 94720-8139
\\
{\tt xsli@lbl.gov}\\
Meiyue Shao\end{center}

We present a new supernode-based incomplete LU factorization method
to construct a preconditioner for solving sparse nonsymmetric linear
systems with iterative methods. The new algorithm is primarily based
on the ILUTP approach by Saad, and we incorporate a number of
techniques to improve the robustness and performance of the
traditional ILUTP method. We present numerical experiments to
demonstrate that our new method is competitive with the other ILU
approaches and is well suited for today's high performance architectures.

Our contributions can be summarized as follows. We adapt the classic
dropping strategies of ILUTP in order to incorporate supernode
structures and to accommodate dynamic supernodes due to partial
pivoting. For the secondary dropping strategy, we propos an
area-based fill control method, which is more flexible and numerically
robust than the traditional column-based scheme. Furthermore, we
incorporate several heuristics for adaptively modifying various
threshold parameters as the factorization proceeds, which improves
the robustness of the algorithm. Finally, the implementation of the
algorithm has already been incorporated in the SuperLU version 4.0
release, downloadable at
http://crd.lbl.gov/~xiaoye/SuperLU/superlu_4.0.tar.gz


\end{document}
