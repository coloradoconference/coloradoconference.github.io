\documentclass{report}
\usepackage{amsmath,amssymb}
\setlength{\parindent}{0mm}
\setlength{\parskip}{1em}
\begin{document}
\begin{center}
\rule{6in}{1pt} \
{\large Travis, M. Austin \\
{\bf A Comparison of Some Sparse Preconditioners for High-Order Finite Element Systems}}

26951 La Alameda Apt 1212 \\ Mission Viejo \\ CA 92691
\\
{\tt austin@txcorp.com}\\
Marian Brezina\\
Thomas, A.  Manteuffel\\
John Ruge\\
   \\
   \\
   \end{center}

High-order finite elements have been observed to achieve a level of
accuracy per degree of freedom that exceeds that achieved for low-order
linear finite elements. Furthermore, high-order finite elements have been
used in climate modeling for problems involving long time integrations
and in MHD simulations to capture physics that standard low-order finite
elements fail to capture. Solving these systems efficiently and with a
low-memory overhead is thus beneficial to a number of research fields. We
present results for preconditioning these systems with an approximate
inversion of a matrix with the same rank as the high-order finite element
matrix. The preconditioning matrix is sparse as it is derived from a
matrix with a stencil size that is equivalent to using low-order finite
elements. This sparse matrix is approximately inverted using a single
V-cycle of algebraic multigrid. We consider both 2D and 3D problems and
also various nodal arrangements on an element for the high-order degrees
of freedom.

This research is supported by US DOE SBIR Phase II DE-FG02-08ER85154.


\end{document}
