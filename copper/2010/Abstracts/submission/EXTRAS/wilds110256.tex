\documentclass{report}
\usepackage{amsmath,amssymb}
\setlength{\parindent}{0mm}
\setlength{\parskip}{1em}
\begin{document}
\begin{center}
\rule{6in}{1pt} \
{\large Stefan Wild \\
{\bf Estimating Computational Noise in Iterative Solvers}}

Mathematics and Computer Science Division \\ Argonne National Laboratory \\ 9700 S Cass Ave \\ Argonne \\ Illinois 60439-4844
\\
{\tt wild@mcs.anl.gov}\\
Mor\'{e} Jorge\end{center}


Computational noise in deterministic simulations is as ill-defined a
concept as can be found in scientific computing. The effects (see e.g.,
\cite{Higham2002}) of finite precision arithmetic, discretizations,
numerical solutions to systems of equations, and adaptive techniques are
typically swept under the rug of most modern simulation codes, the
outputs of which we tacitly assume are smooth.

We are motivated by simulation-based optimization problems of the form
\begin{equation}
\min\left\{ f(x)=F[s(x)] : x\in \Omega\subseteq \mathbb{R}^n\right\},
\label{eq:spopt}
\end{equation}
where the objective is determined by the output,
$s:\mathbb{R}^n\rightarrow \mathbb{R}^m$, of a numerical simulation.
While the function $F$ and the process approximated by $s$ are typically
smooth, the computed $f$ is often noisy. In addition to hampering
optimization techniques, this computational noise can complicate
sensitivity analysis and other applications, which depend on a smooth
simulation output.

In this talk we present an algorithm, \textsf{ECNoise}, for quantifying
computational noise based on the work of Hamming \cite{hamming1971ian}.
Our theoretical framework is based on a model of stochastic noise in
univariate functions,
but requires only relatively few function evaluations and relies on very
few assumptions. In particular, we do not assume any specific
distribution forms for the cumulative errors. Our numerical tests suggest
the algorithm is also effective for deterministic and multivariate
functions.

Given a univariate stochastic $f: \mathbb{R} \rightarrow \mathbb{R}$, we
estimate the \underline{noise level}
$\epsilon_f:=\left(\mbox{Var}\{f(t)\}\right)^{1/2}$ by a weighted
root-mean-square of $k$-th order differences:
\begin{equation}
\sigma_k = \left(\frac{1}{m-k+1}\frac{(k!)^2}{(2k)!} \sum_{i=0}^{m-k}
\left[\Delta^k f(t+ih) \right]^2\right)^{1/2}.
\label{eq:est}
\end{equation}
Given a set of $m+1$ function values, \textsf{ECNoise} determines whether
the sampling distance $h$ is sufficiently small, and an order $k\leq m$
to estimate $\epsilon_f \approx \sigma_k$. Numerical tests on stochastic
functions show that \textsf{ECNoise} generally produces consistent
results using as few as $m=6$ additional function evaluations,
independent of the dimension.

We illustrate the potential for using \textsf{ECNoise} to gain insight
into complex deterministic numerical simulations by considering the
fundamental problem of solving a sparse linear system.
\begin{itemize}
\item When is the computational noise more than simple round off?
\item Is the noise level a property of the solver's operations or the
underlying (continuous) function?
\item Does demanding a tighter tolerance reduce the noise?
\end{itemize}
Using the Krylov solvers in \texttt{MATLAB} \cite{linearsys} and the
symmetric positive definite matrices in the Florida Sparse Matrix
collection \cite{ufmatrices}, we find surprising answers to these and
other questions.

\begin{thebibliography}{1}
\bibitem{linearsys} R. Barrett, M. Berry, T. F. Chan, J. Demmel, J.
Donato, J. Dongarra, V. Eijkhout, R. Pozo, C. Romine, and H. Van der
Vorst. {\em Templates for the Solution of Linear Systems: Building Blocks
for Iterative Methods, 2nd Edition}. SIAM, Philadelphia, PA, 1994.

\bibitem{ufmatrices} T.A. Davis. {\em The University of Florida Sparse
Matrix Collection.} Available at
\url{http://www.cise.ufl.edu/research/sparse/matrices}, 2009.

\bibitem{hamming1971ian} R.W. Hamming. {\em Introduction to Applied
Numerical Analysis.} McGraw-Hill, 1971.

\bibitem{Higham2002} N.J. Higham. {\em Accuracy and Stability of
Numerical Algorithms}. SIAM, Philadelphia, PA, 1996.
\end{thebibliography}


\end{document}
