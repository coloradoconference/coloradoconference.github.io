\documentclass{report}
\usepackage{amsmath,amssymb}
\setlength{\parindent}{0mm}
\setlength{\parskip}{1em}
\begin{document}
\begin{center}
\rule{6in}{1pt} \
{\large Kengo NAKAJIMA \\
{\bf Parallel Multigrid Solvers using OpenMP/MPI Hybrid Programming Models on Multi-Core/Multi-Socket Clusters}}

Information Technology Center \\ The University of Tokyo \\ 2-11-16 Yayoi \\ Bunkyo-ku \\ Tokyo 113-8658 \\ JAPAN
\\
{\tt nakajima@cc.u-tokyo.ac.jp}\end{center}

OpenMP/MPI hybrid parallel programming models were implemented to 3D
finite-volume based simulation code for groundwater flow problems through
heterogeneous porous media using parallel iterative solvers with
multigrid preconditioning by IC(0) smoothing. Performance and robustness
of the developed code has been evaluated on �gT2K Open Supercomputer
(Tokyo)�h and �gCray-XT4�h using up to 1,024 cores through both of weak
and strong scaling computations. Optimization procedures for OpenMP/MPI
hybrid parallel programming models originally developed for 3D FEM
applications [1], such as appropriate command lines for NUMA control,
first touch data placement and reordering of the mesh data for contiguous
access to memory, provided excellent improvement of performance on
multigrid preconditioners with OpenMP/MPI hybrid parallel programming
models. OpenMP/MPI hybrid demonstrated better performance and robustness
than flat MPI, especially with large number of cores for ill-conditioned
problems. Thus, hybrid parallel programming model could be a reasonable
choice for large-scale computing on multi-core/multi-socket clusters.


References

[1] Nakajima, K.: Flat MPI vs. Hybrid: Evaluation of Parallel Programming
Models for Preconditioned Iterative Solvers on �gT2K Open
Supercomputer�h, IEEE Proceedings of the 38th International Conference on
Parallel Processing (ICPP-09), pp.73-80 (2009)


\end{document}
