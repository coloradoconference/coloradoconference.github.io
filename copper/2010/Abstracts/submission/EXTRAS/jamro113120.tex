\documentclass{report}
\usepackage{amsmath,amssymb}
\setlength{\parindent}{0mm}
\setlength{\parskip}{1em}
\begin{document}
\begin{center}
\rule{6in}{1pt} \
{\large Ben Jamroz \\
{\bf  Jacobian-free Newton-Krylov methods in Production Level Magnetohydrodynamic Codes}}

Tech-X Corporation \\ 5621 Arapahoe Ave \\ Suite A \\ Boulder \\ CO 80303
\\
{\tt jamroz@txcorp.com}\\
Travis M. Austin\\
Scott Kruger\\
   \\
   \\
   \\
   \end{center}

NIMROD is one of the key MHD simulation tools supported by the DOE's
Office of Fusion Energy Sciences through the Center for Extended
Magnetohydrodynamic Modeling (CEMM) SciDAC. NIMROD has successfully
explained the key heat loss physics mechanisms occurring during tokamak
disruptions, the onset of tearing modes during tokamak discharges, and
the confinement times of spheromaks. Although NIMROD is a
production-level code producing realistic physics results, improvements
that would enable greater accuracy and faster execution are needed to
reduce the turnaround time of a simulation.

Although, semi-implicit methods are generally successful for enabling
many of the simulations that NIMROD performs, NIMROD still has a
time-step limitation due to the nonlinear advection terms in the
equations. Although there are implicit operators for these terms which
are often comparably small for simulated cases, there still exists a
CFL-like condition that depends on the flow velocities of the MHD fluid
in the plasma. Simulation time constraints can arise from this condition,
thus overcoming these constraints would extend the capability of NIMROD.
In addition, the operator splitting introduces discretization errors in
the NIMROD algorithms. Moving to an algorithm that is fully-centered
would improve the accuracy of NIMROD, especially for long-time
simulations.

Using the PETSc computational framework, we examine the use of
fully-centered Jacobian-free Newton-Krylov (JFNK) methods in the NIMROD
code via the SNES package. Previous approaches in NIMROD to incorporate
nonlinear terms only considered their addition in an ad hoc manner. We
report on the first fully nonlinear solve of the velocity equations and
present current progress on solving the fully nonlinear system for
velocity, temperature, magnetic field, and density. We will also discuss
the preconditioning strategy that will be required to efficiently solve
the linear systems within the JFNK iterations.


\end{document}
