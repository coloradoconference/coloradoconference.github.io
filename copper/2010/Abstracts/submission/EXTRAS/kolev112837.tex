\documentclass{report}
\usepackage{amsmath,amssymb}
\setlength{\parindent}{0mm}
\setlength{\parskip}{1em}
\begin{document}
\begin{center}
\rule{6in}{1pt} \
{\large Tzanio Kolev \\
{\bf AMG for Linear Systems Obtained by Local Elimination}}

Center for Applied Scientific Computing \\ Lawrence Livermore National Laboratory
\\
{\tt tzanio@llnl.gov}\\
Thomas Brunner\end{center}

We consider algebraic solvers for linear systems arising in finite
element simulations of scalar and electromagnetic diffusion applications,
where a set of ``interior'' degrees of freedom have been eliminated to
reduce the problem size. This elimination is assumed to be local, so the
``interior'' principal sub-matrix is block-diagonal, and the resulting
Schur complement is still sparse. In order to be feasible, the
elimination process should not only result in using less memory to store
the matrix and vector, but also lead to an algebraic problem that can
still be solved efficiently.

In this talk we investigate AMG-type solution algorithms applied to the
assembled reduced problem, and we discuss the influence of the local
elimination to solver-related properties such us element aspect ratios,
operator complexity and near-nullspace preservation. For Nedelec
discretizations of definite Maxwell problems, the reduction extends to
the nodal variables and generalizes the concepts of discrete gradient and
node-to-edge Nedelec interpolation matrices [1]. We also consider a
modification of the reduction process that targets singular problems,
which arise naturally in electromagnetic diffusion simulations with pure
void (zero conductivity) regions. This case needs a special treatment,
since the local ``interior'' matrices in the void elements have
non-trivial kernel components.

We conclude the presentation with a number of 2D, 3D and axi-symmetric
numerical simulations in the framework of [2]. The results demonstrate
that the combination of an appropriately chosen local elimination with
the use of the BoomerAMG and AMS solvers from [3] can lead to significant
improvements in the overall solution time.

\bigskip

\begin{itemize}

\item[][1]
{\sc T. Kolev and P. Vassilevski},
{\it Parallel Auxiliary Space AMG for H(curl) Problems}, Journal of
Computational Mathematics, (27) 2009, pp. 604-623.

\item[][2]
{\sc P. Bochev and A. Robinson},
{\it Matching algorithms with physics: Exact sequences of finite element spaces}, in
Collected Lectures on the Preservation of Stability under Discretization,
D. Estep and S. Tavener, eds., SIAM, Philadelphia, 2001, pp. 145-165.

\item[][3]
{\em hypre}: a library of high performance preconditioners, {\it
http://www.llnl.gov/CASC/hypre/}

\end{itemize}


\end{document}
