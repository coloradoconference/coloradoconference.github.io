\documentclass{report}
\usepackage{amsmath,amssymb}
\setlength{\parindent}{0mm}
\setlength{\parskip}{1em}
\begin{document}
\begin{center}
\rule{6in}{1pt} \
{\large Karthik Mani \\
{\bf Time-implicit approach using spatially non-uniform time-step adaptation driven by a-posteriori adjoint error estimates}}

1000 E University Ave \\ Dept 3295 \\ Laramie WY 82071
\\
{\tt kmani@uwyo.edu}\\
Dimitri Mavriplis\end{center}

We present a space-time finite-volume formulation for the unsteady Euler
equations, which allows the use of spatially non-uniform time-step sizes.
The space and time dimensions are treated in a unified manner by
integrating the equations over control volumes spanning both space and
time. The space-time control volumes for the two-dimensional example
problems presented in this work are prismatic in nature and are
constructed by connecting the vertices of triangular spatial elements at
two different time-steps. The triangular spatial elements that form the
top and bottom bounding surfaces of the prismatic space-time element are
termed temporal faces and have normal vector components that are zero in
the spatial dimensions. The space-time element is bounded on the sides by
space-time faces and can have either zero or non-zero temporal normal
vector components. The normal vectors of the space-time faces have
non-zero temporal components in the presence of spatial mesh motion and
have zero temporal components when spatial mesh motion is absent. The
space-time finite-volume formulation inherently accounts for the effect
of dynamically deforming computational meshes. It is shown that
discretizing the Euler equations in space-time finite-volume form is
identical to discretizing the Euler equations written in the
Arbitrary-Lagrangian-Eulerian form.

The primary goal of using the space-time framework is to maintain
solution accuracy while reducing the number of unknowns in the overall
solution process and potentially lower computational expense. While the
formulation presented is capable of simultaneously handling non conformal
meshes in both space and time, the scope of this work is limited to
conformal spatial meshes with non conformal temporal meshes. At any slice
in time, the number of spatial elements remains the same, but across any
slice in space, the number of time-steps is allowed to vary. In
traditional terms, this translates to non-uniform temporal advancement of
spatial elements in an unsteady problem.

The solution process involves first dividing the time domain of interest
into slabs of some predetermined temporal thickness that is sufficient to
capture the essential nonlinearities in the problem. Then, an
a-posteriori error indicator is employed to identify spatial elements
within each temporal slab that have to be advanced with more time-steps
of smaller size. In contrast to traditional implicit time-stepping
methods where the solution for all spatial elements at a time-step is
unknown and solved for implicitly, the implicit system in the space-time
framework assumes an unknown solution for all space-time elements within
each temporal slab. Compared to advancing on a time-step by time-step
basis, the space-time framework is advanced in time on a slab-by-slab
basis. A single implicit system of equations within a slab has to be
solved for before proceeding to the next temporal slab. It should be
noted that the proposed solution method is distinctly different from
adapting the time-step size on-the-fly during the nonlinear solution
process for each spatial element. While there have been attempts to solve
the governing equations in space-time integral form [1-3], advancing
spatial elements non-uniformly in time has mostly been limited to
explicit time-integration schemes.

For the work here, both local error-based and goal-based a-posteriori
error indicators are investigated for use in the space-time framework.
The local error-based indicator is constructed by re-evaluating the time
derivative term in the governing equations based on a higher order
discretization but using the obtained solution. Goal-based error
estimates are achieved by solving the adjoint equations in the space-time
framework and weighting the non-zero implicit residual with it. Previous
work on adapting the time-step size in different parts of the time domain
but uniformly for all spatially elements has shown savings in
computational expense and total degrees-of-freedom [4]. The proposed
method attempts to take this one step further by adapting not only the
time-step size in the time domain but additionally adapt the time-step
size locally for each spatial element.


Two unsteady problems, one where an isentropic vortex is convected
through a rectangular domain and one of a pitching NACA64A010 airfoil in
transonic conditions are presented to demonstrate the algorithm. In the
vortex convection problem, the local temporal error indicator is used to
identify space-time elements which require higher resolution in the time
dimension, thus marking them for temporal refinement. In the case of the
transonic pitching airfoil, the adjoint-weighted residual method
targeting the lift is used as the error indicator. The results indicate
that significant reduction in the overall degrees-of-freedom required to
solve an unsteady problem can be achieved using the proposed algorithm.
Modest improvements in computational expense for specific problems are
also observed.


References:
[1] van der Vegt, J. J. W. and van der Ven, H., �Space�time discontinuous
Galerkin finite element method with dynamic grid motion for inviscid
compressible flows: I. general formulation,� Journal of Computational
Physics, Vol. 182-2, 2002, pp. 546�585.

[2] Barth, T. J., �Space-Time Error Representation and Estimation in
Navier-Stokes Calculations,� Lecture Notes in Computational Science and
Engineering, Vol. 56, 2007.

[3] Abedi, R., Chung, S.-H., Erickson, J., Fan, Y., Garland, M., Guoy,
D., Haber, R., Sullivan, J. M., Thite, S., and Zhou, Y., �Spacetime
meshing with adaptive refinement and coarsening,� Annual Symposium on
Computational Geometry, Proceedings of the twentieth annual symposium on
Computational geometry Brooklyn, New York, USA, 2004.

[4] Mani, K., Mavriplis, D.J., "Error estimation and adaptation for
functional outputs in time-dependent flow problems", Journal of
Computational Physics. Vol. 229-2, January 2010, pp. 415-440,
doi:10.1016/j.jcp.2009.09.034


\end{document}
