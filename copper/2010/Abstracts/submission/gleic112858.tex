\documentclass{report}
\usepackage{amsmath,amssymb}
\setlength{\parindent}{0mm}
\setlength{\parskip}{1em}
\begin{document}
\begin{center}
\rule{6in}{1pt} \
{\large David, F. Gleich \\
{\bf Weakly intrusive methods for uncertainty quantification and parameterized matrix problems}}

74 Barnes Ct #100 \\ Stanford \\ CA 94305
\\
{\tt dgleich@stanford.edu}\\
Paul G. Constantine\\
Gianluca Iaccarino\end{center}

A recurring debate exists amongst practitioners of uncertainty
quantification over whether black-box techniques are sufficient for
meaningful uncertainty quantification or whether existing solvers should
be modified to apply so-called intrusive methods. In the context of
spectral methods, advocates of intrusive methodologies favor techniques
such as Galerkin for their flexibility and smaller number of unknowns,
while non-intrusive proponents prefer collocation and pseudo-spectral
methods because they require only existing solvers and exhibit comparable
asymptotic convergence rates. In this talk, we propose a middle ground: a
weakly intrusive paradigm. From a parameterized matrix perspective, a
weakly intrusive method needs a matrix-vector product at a point in the
parameter space. Methods in this paradigm are easy to implement and
promote software reuse and algorithmic design. We show how the spectral
Galerkin technique fits into the weakly intrusive framework in a method
dubbed Galerkin with Numerical Integration (G-NI). This method depends
crucially on a factorization of the Galerkin matrix. This factorization
also yields bounds on the eigenvalues and suggests preconditioning
strategies including the popular mean based preconditioner.


\end{document}
