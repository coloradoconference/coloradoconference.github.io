\documentclass{report}
\usepackage{amsmath,amssymb}
\setlength{\parindent}{0mm}
\setlength{\parskip}{1em}
\begin{document}
\begin{center}
\rule{6in}{1pt} \
{\large Haw-ren Fang \\
{\bf Two Classes of Multisecant Methods for Nonlinear Acceleration}}

Department of Computer Science and Engineering \\ University of Minnesota \\ 4-192 EE/CS Building \\ 200 Union St SE \\ Minneapolis \\ MN 55455
\\
{\tt hrfang@cs.umn.edu}\\
Yousef Saad\end{center}

Many applications in science and engineering lead to models which require
solving large-scale fixed point problems, or equivalently, systems of
nonlinear equations. Several successful techniques for handling such
problems are based on quasi-Newton methods that implicitly update the
approximate Jacobian or inverse Jacobian to satisfy a certain secant
condition.

We present two classes of multisecant methods which allows to take
into account a variable number of secant equations at each iteration. The
first is the Broyden-like class, of which Broyden's family is a subclass,
and Anderson mixing is a particular member. The second class is that of
the nonlinear Eirola-Nevanlinna-type methods.

This work was motivated by a problem in electronic structure
calculations, whereby a fixed point iteration, known as the
self-consistent field (SCF) iteration, is accelerated by various
strategies termed `mixing'.


\end{document}
