\documentclass{report}
\usepackage{amsmath,amssymb}
\setlength{\parindent}{0mm}
\setlength{\parskip}{1em}
\begin{document}
\begin{center}
\rule{6in}{1pt} \
{\large HyeongKae Park \\
{\bf Development and Application of Nonlinear Methods in Computational Nuclear Engineering}}

Idaho National Laboratory \\ P O Box 1625 \\ M S 3840 \\ Idaho Falls \\ ID 83415
\\
{\tt Ryosuke.Park@inl.gov}\\
Dana A. Knoll\end{center}

There is a growing trend in nuclear reactor simulation to consider
multiphysics problems. For example, analysts are interested in coupled
flow, heat transfer and neutronics in reactor analysis. Many of the
coupling efforts to date have been based on the some form of
linearization such as simple code coupling or first-order operator
splitting. This approach is often referred to as loose coupling. While
these approaches can produce answers, they usually leave questions of
accuracy and stability unanswered. Additionally, the different physics
often reside on separate grids which are coupled via simple
interpolation, again leaving open questions of stability and accuracy.

Recent advancement of Newton-based solution schemes, specifically
Jacobian-free Newton Krylov method, enables a robust solution scheme of
tightly coupled problems without operator-splitting. Consequently, this
approach can accurately capture the nonlinear coupling among physics and
achieve second-order convergence in time. Convergence of the system can
be further accelerated in two ways. First, convergence of a Krylov method
can be achieved via utilization of physics-based preconditioning (PBP).
The efficiency of PBP has been demonstrated in many field of
computational physics. Second, the size of Jacobian matrix can be reduced
using the nonlinear elimination method. With the nonlinear elimination
method, we redefine the nonlinear system with fewer dependent variables,
then advance the removed variables with new function evaluation. This
reduces complexity of system Jacobian and increases complexity of
function evaluations.

In this talk, we present our recent development of advanced applications
of Newton-based nonlinear solver in computational nuclear engineering
context. This includes efficient physics-based preconditioning of coupled
flow, heat and neutronics transient simulation, and examples of nonlinear
elimination method.


\end{document}
