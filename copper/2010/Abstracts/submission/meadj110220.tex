\documentclass{report}
\usepackage{amsmath,amssymb}
\setlength{\parindent}{0mm}
\setlength{\parskip}{1em}
\begin{document}
\begin{center}
\rule{6in}{1pt} \
{\large Jodi Mead \\
{\bf Solution of a nonlinear system for uncertainty quantification in inverse problems}}

Department of Mathematics \\ Boise State University \\ 1910 University Dr \\ Boise \\ ID 83725-1555
\\
{\tt jmead@boisestate.edu}\\
Rosemary Renaut\end{center}

We will describe an algorithm for uncertainty quantification in inverse
problems, and explore the solution of the resulting coupled nonlinear
system of equations. Statistical interpretation of inverse problems allow
for uncertainty quantification of parameter and state estimates. However,
this can be expensive when the probability distribution of the parameters
or state is estimated. Rather than estimating or assuming a specific
probability distribution, the $\chi^2$ method developed by the PI and
colleagues only estimates the second moment, or variance. The challenge
with this approach is that a nonlinear coupled system of equations, which
includes calculation of the square root of a symmetric positive definite
matrix, must be solved. The dimension of the system depends on the
coarsness of the uncertainty estimates. For example, in one dimension the
$\chi^2$ method accurately extends the Morov discrepancy principle which
is often used to find a single regularization parameter. As the dimension
of the nonlinear system increases, the regularization parameter becomes a
matrix weight for the parameter, and is an estimate of its {\em a priori}
inverse error covariance matrix.
Results from the $\chi^2$ method on a hydrological application will be
shown. We model spatial variability of soil moisture as
measurement uncertainty in order to determine the scales over which one
parameter set can be used.


\end{document}
