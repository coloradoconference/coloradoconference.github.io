\documentclass{report}
\usepackage{amsmath,amssymb}
\setlength{\parindent}{0mm}
\setlength{\parskip}{1em}
\begin{document}
\begin{center}
\rule{6in}{1pt} \
{\large Daniil Svyatskiy \\
{\bf Adaptive strategies in the multilevel multiscale mimetic (M$^3$) method for two-phase flows in porous media.}}

Mail Stop B284 \\ Los Alamos National Laboratory \\ Los Alamos \\ NM 87545 \\ U S A
\\
{\tt dasvyat@lanl.gov}\\
Konstantin Lipnikov\\
David Moulton\end{center}

Efficient modeling of multiscale system is a {\it must} for numerous subsurface
applications such as contaminant flow in aquifers, enhanced oil recovery, power
generation in fission reactors, carbon dioxide sequestration in complex geological
formations, and fluid flow and heat transfer in geothermal systems.
In all these applications, the key complicating factor is that system
response is measured at engineering length-scales (meters to kilometers) yet depend
crucially on information about microscopic length-scales (e.g. the rock pore geometry
at sub-micron length-scales). On one hand, a simulation fully resolving all microscopic
details of three-dimensional problems is beyond the capability of modern
supercomputers. On the other hand, a naive simulation that does not address the
influence of the fine-scales is terribly inaccurate.

In this work we study the simulation of two-phase flows in porous media
driven by the well production.
The governing equations for this model are
the elliptic equation for the reservoir pressure and the hyperbolic equation for
the water saturation. Many different model upscaling approaches have been
proposed to address this problem.
The methods using multiscale basis functions are two-level methods with unnecessary
rigid distinction between the resolved and unresolved scales, i.e. fine and coarse
scales. This domain decomposition style partitioning is what places the demands
on accurate boundary conditions for local problems, because as the coarsening factor
increases these artificial internal boundary conditions become increasingly important.
Most two-level methods achieve a coarsening factor of approximately
10 in each coordinate direction, while the trends in fine-scale realizations of
large reservoirs requires a coarsening factor more than 100.

In M$^3$ method the multigrid ideas are used to build recursively a problem-dependent
\textbf{multilevel hierarchy of models}. Each model preserves
important physical properties of the system, such as local mass
conservation. In contrast with classical two-level methods that
the multilevel hierarchal approach facilitates very large
total coarsening factors, such 100 or more.
The method combines two subgrid modeling technique.
The first technique is the
algebraic coarsening developed by Y.Kuznetsov that reduces the degrees
of freedom inside a coarse-grid cell. The
second is a problem dependent approach to conservative coarsening of
velocities on the edges of a coarse-grid cell. These complimentary
strategies ensure that the coarse-scale system has the same sparsity
structure as the fine-scale system, which naturally leads to a
multilevel algorithm.

The mimetic finite difference discretization on the fine scale is
a basis of multilevel hierarchy of discrete models. This allow us to handle
unstructured polyhedral meshes, including locally refined meshes, and
full tensor permeability fields.
Due to the algebraic nature of the method it can bee easily adapted to
other mixed-type discretizations,
such as mixed finite elements method, finite volume method, {\it etc}.


A robust and reliable error control is active research area in multiscale modeling.
The flexibility of the hierarchical approach allow us to incorporate different
adaptive strategies in a very efficient manner and to be closer to the
optimal trade-off between accuracy of the method and the computation cost.
In this work we present several adaptive techniques that reduce the cost
of the M$^3$ method,
including solution-based mesh coarsening and adaptive space and time
updates of the multilevel hierarchy of models.
Maintenance of the hierarchy of models incurs only a modest
computational overhead due to the efficiency of recursive
coarsening and adaptive update strategies.
Numerical experiments show that the new adaptive strategy allows us to
control the error in
water saturation. Compared uniform updates in time, the adaptive strategy gives
three times more accurate solution for the same computation cost.


The $M^3$ method was developed and tested in two-dimensions; however, it
can be readily extended
to polyhedral meshes by leveraging recent developments in mimetic finite difference
methods.


\end{document}
