\documentclass{report}
\usepackage{amsmath,amssymb}
\setlength{\parindent}{0mm}
\setlength{\parskip}{1em}
\begin{document}
\begin{center}
\rule{6in}{1pt} \
{\large Eric C. Cyr \\
{\bf Approximate Block Factorization and Physics-based Preconditioning: Application to CFD and MHD}}

Sandia National Laboratories \\ P O Box 5800 \\ MS 1320 \\ Albuquerque \\ NM 87185-1320
\\
{\tt eccyr@sandia.gov}\\
John N. Shadid\\
Ray S. Tuminaro\end{center}

Multiphysics applications are characterized by strongly-nonlinear coupled
physical mechanisms that produce a solution with a wide-range of length
and time scales. As a result, developing effective preconditioners for
these systems can be a formidable challenge. One approach, which has
demonstrated some success in a range of multiphysics applications (low
Mach number CFD, reacting flow, drift-diffusion simulations and low Mach
number resistive MHD), is a multigrid method with incomplete
factorization smoothing. The coarsening is an aggressive graph-based
aggregation applied to the nonzero block structure of the Jacobian
matrix. These techniques have been shown to have favorable scaling
properties for a number of nearly-elliptic systems. However, it is an
open question on how robust and efficient these methods will be as the
physics increases in complexity or progresses towards the hyperbolic
limit.

An alternative approach is to use approximate block factorizations for
preconditioning of multiphysics systems. These methods segregate the
linear operator into different sub-matrices based on the components of
the physics. These individual components are typically more amenable to
black-box AMG technology. The difficulty with this approach is that an
effective approximation of the physical coupling embodied in the Schur
complement operator is required. For the Navier-Stokes equations this has
been a very active area of study. The approaches have included the basic
physics-based SIMPLE solution method to more sophisticated techniques
based on commuting arguments, such as the pressure-convection diffusion
or least-squares commutator preconditioners.

In this talk performance results for both preconditioning techniques are
compared. Several applications are considered, including Navier-Stokes
and compressible and incompressible Magnetohydrodyanamics. The specific
Schur complement approximations as well as the implementation of these
methods in the recently developed Trilinos software package ``Teko'' will
be discussed.


\end{document}
