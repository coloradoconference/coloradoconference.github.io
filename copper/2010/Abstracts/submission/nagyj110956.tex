\documentclass{report}
\usepackage{amsmath,amssymb}
\setlength{\parindent}{0mm}
\setlength{\parskip}{1em}
\begin{document}
\begin{center}
\rule{6in}{1pt} \
{\large James Nagy \\
{\bf Efficient Iterative Solvers for Nonlinear Least Squares Problems in Imaging Applications}}

Mathematics and Computer Science Department \\ Emory University \\ 400 Dowman Drive \\ Suite W401 \\ Atlanta \\ GA 30322
\\
{\tt nagy@mathcs.emory.edu}\end{center}

In this talk we consider large-scale separable nonlinear least squares
problems that arise in image processing applications. We use a
Gauss-Newton method for the nonlinear solver, and LSQR for the
associated linear systems. Two important issues need to be
considered: how to incorporate regularization (the underlying
continuous problem is ill-posed), and how to efficiently handle large
scale linear solvers with the Jacobian matrix. This latter issue is
essential to making it worth using a Gauss-Newton method. We discuss
these issues in the context of a specific imaging application called
multi-frame blind deconvolution. We show that the Jacobian,
though dense, has an interesting structure that allows for very
efficient matrix-vector multiplications, as well as for construction
of effective preconditioners.


\end{document}
