\documentclass{report}
\usepackage{amsmath,amssymb}
\setlength{\parindent}{0mm}
\setlength{\parskip}{1em}
\begin{document}
\begin{center}
\rule{6in}{1pt} \
{\large P. A. Lott \\
{\bf Fast Solvers for Models of Fluid Flow with Spectral Elements}}

National Institute of Standards & Technology \\ 100 Bureau Dr Stop 8910 \\ Gaithersburg \\ MD 20899
\\
{\tt aaron.lott@nist.gov}\\
H. C. Elman\end{center}

Numerical simulation provides a mechanism for predicting the behavior and
aiding in the design of fluid systems. Computational methods based on
explicit or semi-implicit time stepping can fail to provide a scalable
means of performing simulations because of a stability-based time step
constraint. Implicit and steady state solvers, however, are not limited
by this constraint and can provide an efficient alternative for
performing flow studies. The expense of implicit and steady-state solvers
depends primarily on the cost of solving associated large sparse linear
systems. These linear systems are solved using iterative methods, and the
convergence can often be accelerated via preconditioning. We will
introduce a new technique for solving convection-diffusion systems that
leverages properties of the spectral element discretization and
demonstrate how this technique can be used as a component of a block
preconditioning strategy for solving a range of fluid flow problems.


\end{document}
