\documentclass{report}
\usepackage{amsmath,amssymb}
\setlength{\parindent}{0mm}
\setlength{\parskip}{1em}
\begin{document}
\begin{center}
\rule{6in}{1pt} \
{\large Eric de Sturler \\
{\bf A Convergence Analysis for Recycling Krylov Methods}}

Department of Mathematics \\ 460 McBryde Hall \\ Virginia Tech \\ Blacksburg \\ VA 24061-0123 \\ USA
\\
{\tt sturler@vt.edu}\end{center}

In many computational science and engineering problems, we have
to solve a sequence of large, sparse, linear systems, in which
the matrix changes slowly from one system to the next or
changes in an algebraically structured way. The right hand side
can change more drastically, although in many applications this
is not the case. We have developed several methods that
significantly improve the convergence of iterative solvers by
recycling from one system to the next an approximate invariant
subspace. For efficiency and because the system matrix changes
continually, we do not expect the approximate invariant
subspaces to be accurate. Nevertheless, this technique has
proved very successful in reducing total iteration counts for a
range of applications. In this presentation, we provide a
convergence analysis for iterative linear solvers recycling an
approximate invariant subspace. An important result of the
analysis is that significant improvement of convergence results
for very modest accuracy of the invariant subspace
approximation.

This is joint work with Michael L. Parks


\end{document}
