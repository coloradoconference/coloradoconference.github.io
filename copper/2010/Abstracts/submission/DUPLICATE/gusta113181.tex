\documentclass{report}
\usepackage{amsmath,amssymb}
\setlength{\parindent}{0mm}
\setlength{\parskip}{1em}
\begin{document}
\begin{center}
\rule{6in}{1pt} \
{\large Magnus Gustafsson \\
{\bf Parallel Lanczos-based exponential integrators for quantum dynamics}}

Div of Scientific Computing \\ Dept of Information Technology \\ Uppsala University \\ Box 337 \\ 751 05 Uppsala \\ Sweden
\\
{\tt magnus.gustafsson@it.uu.se}\\
Katharina Kormann\\
Sverker Holmgren\end{center}

The time-dependent Schr{\"o}dinger equation (TDSE) describes the quantum
dynamical nature of molecular processes. Simulations are, however,
computationally very demanding due to the curse of dimensionality. With
our MPI and OpenMP parallelized code, HAParaNDA (cf.[2]), we are able to
accurately solve the full Schr{\"o}dinger equation, currently in up to
five dimensions using a medium-size cluster. We report numerical
experiments which show that we can efficiently solve large-scale problems
using up 1024 computing cores.

The $d$-dimensional TDSE reads
\begin{equation*}
\mathrm{i}\hbar \psi(x,t) = \left(-
\sum_{i=1}^d\frac{\hbar^2}{2m_i}\frac{\partial^2}{\partial x_i^2} +
V(x,t)\right)\psi(x,t), \quad \psi(x,0), x \in \mathbb{R}^d = \psi_0,
\end{equation*}
where $\psi(x,t)$ denotes the wave packet and the potential $V(x,t)$
describes interactions within the system. Currently, we only consider
localized molecules. We take sufficiently large $d$-orthotopes as
computational domain and close the system by periodic boundary
conditions. Later, transparent boundary conditions will be included to
model dissociative problems. Our implementation is based on the method of
lines. We first introduce a spatial grid and compute the derivatives with
an $p$th order finite difference (FD) method (8th order in the reported
experiments). For a time-independent potential, the solution of the
semi-discrete system is given by
\begin{equation*}
u(t) = \mathrm{e}^{-\frac{\mathrm{i}}{\hbar} H}u(0),
\end{equation*}
where $H$ is the spatially discretized differential operator and $u$ is
the discrete wave packet. For time-dependent potentials, the exponential
form can still be exploited on small time intervals but the Hamiltonian
has to be time-averaged using e.g.~the Magnus expansion (see [4] for
details). The matrix exponential can be computed efficiently in serial
using the Lanczos algorithm. In a parallel environment, the fact that two
inner products have to be computed in each iteration hampers scalability.
Kim and Chronopoulos [3] discuss two modifications of the Lanczos
algorithm with reduced communication. Rearranging the computations makes
it possible to compute both inner products at the same point, reducing
the number of synchronizations to one per iteration. In the $s$-step
Lanczos method, blocks of $s$ consecutive steps are executed with only
one synchronization step required for each block. We have transferred
these ideas to the Lanczos propagator method.

A second hassle with using the Lanczos procedure is that the number of
iterations has to be chosen with care. Numerical round-off errors cause
instabilities. We therefore propose to choose the number of steps
adaptively and to make sure that the iterations are stopped once the
residual is small enough. For a discussion of error estimation in the
propagation of the TDSE, we refer to [5].

In order to demonstrate the performance of a massively parallel
simulation of the TDSE based on a FD-Lanczos discretization, we have
conducted several simulations on a medium-size cluster, consisting of 316
nodes. Each node is equipped with dual quad-core Intel Nehalem CPU and 24
GB of DRAM. The nodes are interconnected by an InfiniBand fabric.

For the experiments, we have considered the harmonic oscillator and the
Henon--Heiles potential. The analytical solution is known for the
harmonic oscillator, so we are able to verify the correctness and the
accuracy of our numerical results. Table \ref{tab:1} shows the
scalability of the results for the 4D harmonic oscillator. Comparing the
variants of the Lanczos algorithm, we see that the performance can indeed
be improved by reducing the communication. We are currently working on an
implementation of the $s$-step method and hope to further improve the
scalability in that way.

The Henon--Heiles potential is a common test case for high-dimensional
simulations (cf., e.g., [1]). We have simulated this problem in five
dimensions on a grid with $60^5$ points over $10000$ time steps. We used
the standard Lanczos algorithm and have chosen the size of the Krylov
space adaptively. The simulation was performed on 1024 cores in 19 hours.

\begin{table}[htdp]
\caption{\small{Scalability of standard Lanczos (L1) and the
few-synchronizations variant (L2). The wall time is reported in units of
1000 $s$. Each node solves a $60^{4}$ problem and the largest problem has
$240 \times 240 \times 120 \times 120$ unknowns.}}
\begin{center}
\begin{tabular}{|c|c|c|c|c|c|c|c|}
\hline
\# Cores & 8 & 16 & 32 & 64 & 128 & 256 & 512 \\
\hline
L1& 3.92& 4.31& 4.37& 4.55& 4.72& 4.87& 5.16 \\
L2& 3.79& 4.06& 4.26& 4.33& 4.43& 4.72& 4.83 \\
\hline
\end{tabular}
\end{center}
\label{tab:1}
\end{table}%


\begin{center}
{\bf REFERENCES}\\[-3mm]
\end{center}

\begin{tabular}{lp{0.9\textwidth}}
\hspace{-3mm}{[1]}&
M.L.~Brewer. On the scaling of semiclassical initial value methods,
\emph{J. Chem. Phys.} {\bf 111} (1999)
\\[1mm]
\hspace{-3mm}{[2]}&
M.~{Gustafsson} and S.~Holmgren. An implementation framework for solving
high-dimensional PDEs on massively parallel computers, to appear in:
\emph{Proceedings of ENUMATH 2009}, Uppsala, Sweden
\\[1mm]
\hspace{-3mm}{[3]}&
S.K.~{Kim} and A.T.~Chronopoulos. A class of Lanczos-like algorithms
implemented on parallel computers, \emph{Parallel Comput.} {\bf 17}
(1991)
\\[1mm]
\hspace{-3mm}{[4]}&
K.~{Kormann}, S.~{Holmgren}, and H.O.~{Karlsson}. Accurate time
propagation for the Schr\"odinger equation with an explicitly
time-dependent Hamiltonian, \emph{J. Chem. Phys.} {\bf 128} (2008)
\\[1mm]
\hspace{-3mm}{[5]}&
K.~{Kormann} and A.~Nissen. Error Control for Simulations of a
Dissociative Quantum System, to appear in: \emph{Proceedings of ENUMATH
2009}, Uppsala, Sweden
\\[1mm]
\end{tabular}


\end{document}
