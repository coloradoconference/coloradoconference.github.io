\documentclass{report}
\usepackage{amsmath,amssymb}
\setlength{\parindent}{0mm}
\setlength{\parskip}{1em}
\begin{document}
\begin{center}
\rule{6in}{1pt} \
{\large Miriam Mehl \\
{\bf Parallel Multigrid -- a Scalable Alternative to Parallel CG?}}

Department of Computer Science \\ Technische Universit\"at M\"unchen \\ Boltzmannstr 3 \\ 85748 Garching
\\
{\tt mehl@in.tum.de}\end{center}

From a computer science point of view, the performance evaluation of
iterative solvers for large systems of equations
does not only include numerical efficiency but also parallel
dcalability. Whereas multigrid methods are clearly superior from
a pure numerical point of view, the majority of parallel solvers
available in software packages are
based on cg-like methods.
These solvers achieve very good scalability on high performance
computing platforms even for large processors numbers. In contrast,
multigrid methods seem to perform much worse in the parallel context.
This arises the question whether they are only a nice theoretical
concept but not applicable in an efficient way in supercomputing
practice.

Considerations that might lead to such a conclusion often only include
so-called strong-scalability results --- constant problem size and
growing number of processors.
This point of view does not take into account the advantage of
multigrid methods --- the constant number of iterations with
increasing problem size.
Thus, the correct setting must be to consider weak-scalability, that
is problem sizes growing according to the processor number, and, in
addition, time until convergence instead of only single solver
iterations.

This approach leads to a new definition of an optimal solver.
Whereas an optimal sequential
solver computes the solution of a system in a computational
time growing proportional to the number of unknowns,
parallelisation even allows to look for solvers with
constant computational time.
So the question is: Is it possible
to implement a solver that solves a system of equations in constant
time independent from the size of the system or the mesh resolution,
respectively, by suitably increasing the number of processors?
In the optimal case, the number of processors increaes only proportional
to the problem size.

We will present first a theoretical
comparison of cg-like and multigrid methods for parallel computations
under simplifying assumptions showing the general potential of
multigrid and cg-like solvers on supercomputers and also
answering the question on the possibility to develop an optimal
parallel solver. In addition, we will relate various parallel
multigrid approaches from literature (see for example
[1, 2, 3, 4, 6, 7]) to this theoretical analysis
and, finally, present results from an own parallel multigrid
implementation on adaptive Cartesian grids with tree-structure
(see [5] for a decription of the corresponding
sequential code).
In contrast to the purely theoretical analysis that only takes into
account computational and communication costs, this will also give
hints on ways to improve other performance aspects such as
cache-efficiency that can be an additional stumbling block
for high performance multigrid implementations due to the more
complex data dependencies compared to cg-like solvers.

\begin{itemize}
\item[[1]] M. Adams and J. W. Demmel. Parallel multigrid solver for 3d
unstructured finite element problems. In {\em Supercomputing '99:
Proceedings of the 1999
ACM/IEEE conference on Supercomputing}, page 27, New York,
NY, USA, 1999. ACM.
\item[[2]] M. Berger, M. Aftosmis, and G. Adomavicius. Parallel multigrid
on cartesian meshes with complex geometry. In {\em Proc. of the
8th Intl. Conf. on Parallel CFD. Trondhiem}, 2000.
\item[[3]] T. Gradl and U. R\"ude. High performance multigrid on current
large scale parallel computers. In {\em 9th Workshop on Parallel
Systems and Algorithms (PASA), Dresden, 26.02.2008}, volume 124
of {\em GI Edition: Lecture Notes in Informatics},
pages 37--45. Gesellschaft f\"ur Informatik, 2008.
\item[[4]] H. van Emden and U.Meier Yang. Boomeramg: A parallel
algebraic multigrid solver and preconditioner. {\em Applied
Numerical
Mathematics}, 41(1):155--177, 2002.
\item[[5]] M. Mehl, T. Weinzierl, and Ch. Zenger. A cache-oblivious
selfadaptive full multigrid method. {\em Numerical Linear Algebra
with Applications}, 13(2-3):275--291, 2006.
\item[[6]] W. F. Mitchell. A refinement-tree based partitioning method for
dynamic load balancing with adaptively refined grids. {\em Journal
of Parallel and Distributed Computing}, 67(4):417--429, 2007.
\item[[7]] G. W. Zumbusch. Data parallel iterators for hierarchical grid
and tree algorithms. In W. E. Nagel, W. V. Walter,
and W. Lehner, editors, Euro-Par, volume 4128 of {\em Lecture Notes
in Computer Science}, pages 625--634. Springer, 2006.
\end{itemize}


\end{document}
