\documentclass{report}
\usepackage{amsmath,amssymb}
\setlength{\parindent}{0mm}
\setlength{\parskip}{1em}
\begin{document}
\begin{center}
\rule{6in}{1pt} \
{\large David J. Baker \\
{\bf The method of long characteristics for solving the neutron transport equation}}

Pope Building \\ School of Mathematical Sciences \\ University of Nottingham \\ University Park \\ Nottingham \\ NG7 2RD \\ United Kingdom
\\
{\tt pmxdb4@nottingham.ac.uk}\\
Andrew Cliffe\\
Paul Houston\\
Paul Smith\end{center}

The neutron transport equation is used to describe the behaviour of
neutrons in a heterogeneous, fission-capable medium as they move in a
range of directions and energies. The efficient and accurate solution of
this equation is vital in the field of nuclear reactor design and
operation, amongst other areas. To this end, a variety of numerical
methods have been developed over the past several decades. In particular,
the use of the method of characteristics in various applications of the
neutron transport equation has been common for some years now, and is now
an established method in several leading commercial codes, including the
CACTUS module in Serco Assurance's WIMS software. However, to date, no
comprehensive study of the mathematical properties of this method has
been undertaken.

In this talk we present an \emph{a priori} error analysis for a fully
discrete neutron transport problem on a two-dimensional plane, based on
exploiting the method of characteristics in the spatial domain and the
discrete ordinates technique for the discretization of the angular
dimension. Here, the error in the scalar flux is measured in terms of the
$L_{p}$-norm, with $p = 2+\epsilon$ and $epsilon>0$. The underlying proof
is based on a combination of the work undertaken by Johnson and
Pitk{\"a}ranta and general operator theory. Numerical examples
demonstrating the practical performance of the proposed discretization
method will be presented.


\end{document}
