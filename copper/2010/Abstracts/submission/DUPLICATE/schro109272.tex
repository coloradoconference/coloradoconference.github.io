\documentclass{report}
\usepackage{amsmath,amssymb}
\setlength{\parindent}{0mm}
\setlength{\parskip}{1em}
\begin{document}
\begin{center}
\rule{6in}{1pt} \
{\large Jacob B. Schroder \\
{\bf A General Interpolation Strategy for Algebraic Multigrid Using Energy-Minimization}}

Siebel Center for Computer Science \\ 4328 Siebel Center \\ 201 N Goodwin Ave \\ Urbana \\ IL 61801 USA
\\
{\tt jacob.bb.schroder@gmail.com}\\
Raymond S. Tuminaro\\
Luke N. Olson\end{center}

A general interpolation strategy for algebraic multigrid that uses an
energy-minimization principle is presented. The proposed strategy is
applicable to symmetric and nonsymmetric problems in both a classical and
smoothed-aggregation-based algebraic multigrid framework. Each column of
the interpolation operator, $P$, is minimized in an energy-based norm,
while enforcing two constraints. A sparsity pattern is enforced on $P$
and important user-defined modes, such as the constant, are preserved in
the span of $P$. To minimize energy, we use a Krylov-based strategy that
is equivalent to solving $A P = 0$, where the constraints ensure a
nonzero answer. For the symmetric case, a CG-based strategy is utilized,
while for the nonsymmetric case, GMRES and CGNR are explored. The
approach is flexible allowing for arbitrary coarsenings and sparsity
patterns, easy long distance interpolation and general constraints, which
we choose to be the preservation of important user-defined modes. We
conclude with encouraging numerical results.


\end{document}
