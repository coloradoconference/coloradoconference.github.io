\documentclass{report}
\usepackage{amsmath,amssymb}
\setlength{\parindent}{0mm}
\setlength{\parskip}{1em}
\begin{document}
\begin{center}
\rule{6in}{1pt} \
{\large Andrew Canning \\
{\bf Accelerating Parallel Iterative Eigensolvers for Large Scale First-Principles Materials Science Calculations }}

Lawrence Berkeley National Laboratory \\ One Cyclotron Road \\ Berkeley CA94720
\\
{\tt ACanning@lbl.gov}\end{center}

The most widely used approach in first-principles materials science
calculations involves the solution of some form of the Schrodinger
Equation, typically the Kohn-Sham form based on density functional
theory. The resulting eigenfunction problem involves the solution of the
lowest eigenpairs of a very large dense matrix which is typically solved
using an iterative approach such as conjugate gradient. In this talk we
will look at different methods to speedup the solution on large parallel
computers, including the investigation of collective compared to
point-to-point methods for reducing the communication costs as well as
algorithms to block the communications and reduce latency. We have also
investigated mixed MPI and OpenMP programming models to improve
performance on multicore computer architectures. We are also studying the
use of multilevel methods in Fourier space whereby the eigenvalue problem
is solved initially on a coarser grained Fourier grid which is then used
as a starting point for the solution on a finer grid.


\end{document}
