\documentclass{report}
\usepackage{amsmath,amssymb}
\setlength{\parindent}{0mm}
\setlength{\parskip}{1em}
\begin{document}
\begin{center}
\rule{6in}{1pt} \
{\large Steven Hamilton \\
{\bf Eigensolvers for radiation transport applications}}

Emory University \\ 400 Dowman Dr  \\ W401 \\ Atlanta \\ GA 30322
\\
{\tt sphamil@emory.edu}\\
Michele Benzi\end{center}

An area of active interest in nuclear reactor analysis is the solution of
the generalized eigenvalue problem corresponding to the discretized
Boltzmann radiation transport equation. In particular, the dominant
eigenvalue has a physical interpretation as the multiplication factor of
the system. We are interested in identifying this dominant eigenvalue and
the corresponding eigenvector (the neutron flux). The most commonly used
method for approaching this task is a straightforward power iteration,
possibly with some form of acceleration (Chebyshev, diffusion-based,
etc.). For difficult problems, however, convergence can be unacceptably
slow, motivating the desire for more effective techniques. In this study
we provide a comparison of the performance of several eigensolvers
including Rayleigh quotient iteration, Newton's method, the implicitly
restarted Arnoldi method, and a Jacobi-Davidson method. Difficulties in
solving linear sub-problems will be addressed, including a discussion of
potential preconditioners.


\end{document}
