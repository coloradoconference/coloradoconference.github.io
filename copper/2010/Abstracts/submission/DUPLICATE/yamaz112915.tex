\documentclass{report}
\usepackage{amsmath,amssymb}
\setlength{\parindent}{0mm}
\setlength{\parskip}{1em}
\begin{document}
\begin{center}
\rule{6in}{1pt} \
{\large Wataru YAMAZAKI \\
{\bf Design Optimization and Uncertainty Analysis by Using Gradient/Hessian Enhanced Surrogate Model Approaches}}

Department of Mechanical Engineering \\ 1000 E University Ave  \\ University of Wyoming \\ Laramie \\ WY 82072
\\
{\tt wyamazak@uwyo.edu}\\
Markus P. RUMPFKEIL\\
Dimitri J. MAVRIPLIS\end{center}

Numerical design optimization in the field of aerospace engineering has
attracted attention these days thanks to the maturity of Computational
Fluid Dynamics (CFD) and the increase in computer performance. The major
bottleneck of aerodynamic design optimization with global search methods,
such as genetic algorithms (GA), is the huge number of required CFD
function evaluations which can exceed several thousand. Recently,
therefore, optimization approaches based on surrogate models have
attracted attention because of the capability to reduce the number of CFD
function evaluations within design optimization.

Recently, efficient gradient and approximate Hessian calculation methods
using adjoint method and automatic differentiation (AD) have been
developed in our group [1]. In these approaches, the computational costs
for a gradient vector and Hessian matrix with respect to geometrical
variables are respectively comparable to one CFD function evaluation.
Thus it is promising to utilize the gradient and Hessian information
within surrogate models to obtain more accurate response surface. Such
enhanced response surface will promote the efficiency of design
optimization process.

Surrogate models can be constructed using a variety of methods [2]. In
this work, polynomial regression, radial basis function (RBF) and Kriging
methods have been extended to be enhanced by gradient and Hessian
information. Two gradient enhanced Kriging (direct and indirect
cokriging) approaches have been developed and applied to design
optimizations in the literature [3]. The indirect cokriging approach has
been extended in this work to make use of gradient/Hessian information.
This is because the number of sample points which have gradient/Hessian
information can be easily limited in this approach, which results in the
reduction of computational cost to construct surrogate model.

The developed gradient/Hessian enhanced surrogate model approaches have
been firstly investigated for multi-dimensional analytical function
fitting problems. The results showed that surrogate models enhanced by
gradient/Hessian information provided better fitting on the actual
functions. These methods showed better efficiencies even if taking the
computational cost for gradient/Hessian evaluation into consideration.
Then, these methods were applied to a drag minimization problem of a
two-dimensional transonic airfoil. A faster reduction of the objective
function value, in other words faster convergence towards the global
optimal design was confirmed with the gradient/Hessian enhanced model
approaches. Since the computational costs for CFD gradient and Hessian
evaluation are independent from the number of dimensionality of problem,
the developed methods should be more capable in higher-dimensional
complex problems.

Uncertainty analysis is also watched with keen interests these days [4-5]
since high-fidelity CFD computations typically assume perfect knowledge
of all parameters. In reality, however, there is much uncertainty due to
manufacturing tolerances, in-service wear-and-tear, approximate modeling
parameters and so on. The most straightforward and accurate method for
uncertainty analysis is a full nonlinear Monte Carlo (MC) simulation.
Although this method is easy to implement, it is still prohibitively
expensive for high-fidelity CFD computations. Moment methods based on
Taylor series expansions are alternative ways to assess uncertainty
although these methods only give the mean and standard deviation of the
output function. To apply surrogate model approaches is another positive
way for uncertainty analysis which can reduce the computational cost
dramatically. The computational advantage is evident because the
surrogate model is an analytic representation of the design space. Its
estimated function values can be used for an inexpensive Monte Carlo
(IMC) simulation to obtain not only mean and standard deviation, but also
an approximate probability density function (PDF) of the output function.
Our accurate response surfaces enhanced by gradient and Hessian
information can increase the accuracy of uncertainty analysis by using
IMC. The results of uncertainty analysis will be shown in the
presentation to compare the performances with full nonlinear MC
simulation. This approach is promising since it can be easily extended to
�gRobust Design�h approach because of the lower computational cost of the
uncertainty analysis.

[1] Rumpfkeil, M. P., and Mavriplis, D. J., �gEfficient Hessian
Calculations using Automatic Differentiation and the Adjoint Method,�h
AIAA-2010-1268, 2010.
[2] Peter, J., and Marcelet, M., �gComparison of Surrogate Models for
Turbomachinery Design,�h WSEAS Transactions on Fluid Mechanics, Issue.1,
Vol.3, 2008.
[3] Laurenceau, J., and Sagaut, P., �gBuilding Efficient Response
Surfaces of Aerodynamic Functions with Kriging and Cokriging,�h AIAA
Journal, Vol. 46, No. 2, pp.498-507, 2008.
[4] Ghate, D., and Giles, M. B., �gInexpensive Monte Carlo Uncertainty
Analysis,�h Recent Trends in Aerospace Design and Optimization,
pp.203-210, 2006.
[5] Chalot, F., Dinh, Q. V., Herbin, E., Martin, L., Ravachol, M., and
Roge, G., �gEstimation of the Impact of Geometrical Uncertainties on
Aerodynamic Coefficients Using CFD,�h AIAA-2008- 2068, 2008.


\end{document}
