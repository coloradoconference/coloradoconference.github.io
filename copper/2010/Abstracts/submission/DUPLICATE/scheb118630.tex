\documentclass{report}
\usepackage{amsmath,amssymb}
\setlength{\parindent}{0mm}
\setlength{\parskip}{1em}
\begin{document}
\begin{center}
\rule{6in}{1pt} \
{\large Fynn Scheben \\
{\bf Iterative methods for neutron transport eigenvalue problems}}

Department of Mathematical Sciences \\ University of Bath \\ Bath \\ BA2 7AY \\ United Kingdom
\\
{\tt F.Scheben@bath.ac.uk}\\
Ivan G. Graham\end{center}

We discuss iterative methods for computing criticality in nuclear
reactors. In general this requires the solution of a generalised
eigenvalue problem for an unsymmetric integro-differential operator in 6
independent variables, modelling transport, scattering and fission, where
the dependent variable is the neutron angular flux. In engineering
practice this problem is often solved iteratively, using some variant of
the inverse power method. Because of the high dimension, matrix
representations for the operators are often not available and the inner
solves needed for the eigenvalue iteration are implemented by matrix-free
inner iterations. This leads to technically complicated inexact iterative
methods, for which there appears to be no published rigorous convergence
theory.
For the monoenergetic homogeneous model case with isotropic scattering
and vacuum boundary conditions, we show that the general nonsymmetric
eigenproblem for the angular flux is equivalent to a certain related
eigenproblem for the scalar flux, involving a symmetric positive definite
weakly singular integral operator (in space only). This correspondence to
a symmetric problem (in a space of reduced dimension) permits us to give
a convergence theory for inexact inverse iteration and related methods.
In particular this theory provides rather precise criteria on how
accurate the inner solves need to be in order for the whole iterative
method to converge. The theory is illustrated with numerical computations
on a homogeneous benchmark problem from the Los Alamos test set and also
on an inhomogeneous bienergetic control rod problem, using GMRES as the
inner solver.


\end{document}
