\documentclass{report}
\usepackage{amsmath,amssymb}
\setlength{\parindent}{0mm}
\setlength{\parskip}{1em}
\begin{document}
\begin{center}
\rule{6in}{1pt} \
{\large Joshua Hughes \\
{\bf Leveraging the conjugate gradient method to texture mid-frequency active sonar.}}

Metron Inc \\ 11911 Library St \\ Suite 600 \\ Reston \\ VA 20190
\\
{\tt hughes@metsci.com}\\
Robert, E Zarnich\\
Margaret Stout\end{center}

Texture synthesis is used to replicate and construct larger images from
smaller sample images. Leveraging the framework of [1] we texture
synthesize MFA sonar returns where the pixel space is taken to be range
and bearing. In [1] blocks are tiled in raster scan order so that their
edge intersections satisfy an overlap constraint. To accomplish this, a
method of evaluating a set of blocks and randomly choosing a block
satisfying the overlap constraint is performed. We speed up this process
by replacing this method with [2] where the input function to the
conjugate gradient method is the overlapping constraint result.

References:

[1] Alexei A. Efros , William T. Freeman, Image quilting for texture
synthesis and transfer, Proceedings of the 28th annual conference on
Computer graphics and interactive techniques, p.341-346, August 2001

[2] Hestenes, Magnus R.; Stiefel, Eduard (December 1952). "Methods of
Conjugate Gradients for Solving Linear Systems" (PDF). Journal of
Research of the National Bureau of Standards 49 (6).


\end{document}
