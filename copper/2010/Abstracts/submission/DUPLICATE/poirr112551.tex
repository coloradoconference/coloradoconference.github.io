\documentclass{report}
\usepackage{amsmath,amssymb}
\setlength{\parindent}{0mm}
\setlength{\parskip}{1em}
\begin{document}
\begin{center}
\rule{6in}{1pt} \
{\large Baptiste Poirriez \\
{\bf Domain decomposition methods applied to flow simulation in 3D Discrete Fracture Networks}}

IRISA \\ Campus de Beaulieu \\ 35042 Rennes Cedex \\ France
\\
{\tt baptiste.poirriez@irisa.fr}\\
Geraldine Pichot\\
Jocelyne Erhel\end{center}

Numerical simulations in fractured media are an essential tool for
studying hydraulic properties. Discrete Fracture Networks are composed of
many multiscale plane fractures intersecting each other, leading to
complex geometries. We assume that the rock surrounding the fractures is
impervious and aim at simulating the flow in the fractures. Governing
equations are Darcy's law and mass continuity, with continuity conditions
at the intersections. Mesh generation is rather difficult in this
context, because of the geometry, and requires a specific method
\cite{erhe09b}. We apply a Mixed Hybrid Finite element method and get a
large sparse symmetric positive definite (spd) linear system to solve.
Direct methods are efficient but become time- and memory- consuming for
large systems. Algebraic multigrid method combined with Preconditioned
Conjugate Gradient (PCG) converges in few iterations but is also
time-consuming. The linear system has a specific structure since the
problem is elliptic, but the geometry is neither 2D nor 3D. Domain
decomposition methods are attractive because there is a natural
geometrical decomposition. In this paper, we explore a preconditioned
Schur complement approach.

A first step is to define a partition of the domain. A subdomain is here
a set of fractures, and the interface is the set of intersections with
fractures of other subdomains. With $n$ subdomains, the matrix $A$ has a
block-structure with non singular diagonal blocks $A_{ii},i=1,\ldots,n+1$
and nonzero blocks $A_{i,n+1},i=1,\ldots,n$. We define the local Schur
complement matrix $S_{ii}=A_{n+1,n+1}^{(i)}-A_{i,n+1}^T A_{ii}^{-1}
A_{i,n+1}$, the Schur matrix $S=\sum_{i=1}^{n} S_{ii}$ and the associated
right-hand side $b$. Since the matrix $S$ is spd, we apply PCG to solve
the linear system $Sx=b$.

We use the classical Neumann-Neumann (NN) preconditioner
\cite{carv01,tos05}. Since the partition generates floating subdomains,
some local Schur complements $S_{ii}$ are rank-deficient. In order to
ensure a kernel of dimension 1, we build a partition such that a
subdomain is composed of connected fractures. Then we apply an artificial
Dirichlet boundary condition at one edge of the interface to get a non
singular matrix $\tilde{S}_{ii}$. The NN preconditioner is then defined
by $M^{-1}=\frac{1}{n}\sum_{i=1}^{n} \tilde{S}_{ii}^{-1}$. To gain in
efficiency, we use the Cholesky factorization of $A_{ii}$ to complete the
Cholesky factorization of $\tilde{S}_{ii}$, by adapting algorithms aiming
at reducing fill-in.

We run numerical experiments with two fracture networks, containing
respectively 32 and 128 fractures. These networks are randomly generated,
with some large fractures, by using software MP-FRAC-D3 \cite{erhe09b};
the domain is partitioned into connected subdomains using the software
SCOTCH \cite{pell08}. The Schur method is implemented in Matlab.
In Tables \ref{tab:NN1} and \ref{tab:NN2}, we give convergence results
for CG and PCG(NN) applied to the Schur system. Clearly, NN
preconditioner reduces drastically the number of iterations Niter for a
small number of subdomains Ndom but is not efficient with many
subdomains, as expected.

We use coarse-grid and deflation preconditioners to overcome this
difficulty \cite{erhe00b,nab04}. However, currently, these
preconditioners are not efficient. This might be due to the specific
structure of the linear system. We further investigate this approach.

\begin{table}[h]
\begin{tabular}{|l|c|c|}
\hline
Ndom & Niter using CG & Niter using PCG(NN) \\ \hline
2 & 108 & 18 \\ \hline
4 & 122 & 52 \\ \hline
8 & 126 & 80 \\ \hline
16 & 129 & 108 \\ \hline
\end{tabular}
\caption{\label{tab:NN1} Number of iterations, solving the Schur system
by CG and PCG with Neumann-Neumann preconditioner. The network has 32
fractures.}
\end{table}
\begin{table}[h]
\begin{tabular}{|l|l|l|}
\hline
Ndom & Niter using CG & Niter using PCG(NN) \\ \hline
2 & 153 & 37 \\
\hline
4 & 164 & 75 \\
\hline
8 & 166 & 96 \\
\hline
16 & 165 & 149 \\
\hline
32 & 165 & 211 \\
\hline
56 & 165 & 275 \\
\hline
128 & 165 & 373 \\
\hline
\end{tabular}
\caption{\label{tab:NN2}Number of iterations, solving the Schur system by
CG and PCG with Neumann-Neumann preconditioner. The network has 128
fractures.}
\end{table}

\begin{thebibliography}{1}

\bibitem{carv01}
L.~Carvalho, L.~Giraud, and P.~Le Tallec.
\newblock Algebraic two-level preconditioners for the schur complement method.
\newblock {\em SIAM J. Sci. Comput.}, 22:1987--2005, 2001.

\bibitem{erhe09b}
J.~Erhel, J.-R. de~Dreuzy, and B.~Poirriez.
\newblock flow simulations in three-dimensional discrete fracture networks.
\newblock {\em SIAM Journal on Scientific Computing}, 31(4):2688--2705, 2009.

\bibitem{nab04}
R.~Nabben and C.~Vuik.
\newblock A comparison of {D}eflation and {C}oarse {G}rid {C}orrection applied
to porous media flow.
\newblock {\em SIAM J. Numer. Anal.}, 42:1631--1647, 2004.

\bibitem{pell08}
F.~Pellegrini.
\newblock Scotch 5.1 user's guide.
\newblock Technical report, LaBRI, 2008.

\bibitem{erhe00b}
Y.~Saad, M.~Yeung, J.~Erhel, and F.~Guyomarc'h.
\newblock A deflated version of the conjugate gradient algorithm.
\newblock {\em SIAM Journal on Scientific Computing}, 21(5):1909--1926, 2000.

\bibitem{tos05}
A.~Toseli and O.~Widlund.
\newblock {\em Domain decomposition methods-algorithms and theory}.
\newblock springer series in computational mathematics, 2005.

\end{thebibliography}


\end{document}
