\documentclass{report}
\usepackage{amsmath,amssymb}
\setlength{\parindent}{0mm}
\setlength{\parskip}{1em}
\begin{document}
\begin{center}
\rule{6in}{1pt} \
{\large Kengo NAKAJIMA \\
{\bf Parallel Multigrid Solvers using OpenMP/MPI Hybrid Programming Models on Multi-Core/Multi-Socket Clusters}}

Information Technology Center \\ The University of Tokyo \\ 2-11-16 Yayoi \\ Bunkyo-ku \\ Tokyo 113-8658 \\ JAPAN
\\
{\tt nakajima@cc.u-tokyo.ac.jp}\end{center}

In order to achieve minimal parallelization overheads on SMP (symmetric
multiprocessors) and multi-core clusters, a multi-level hybrid parallel
programming model is often employed. In this method, coarse-grained
parallelism is achieved through domain decomposition by message passing
among nodes, while fine-grained parallelism is obtained via loop-level
parallelism inside each node using compiler-based thread parallelization
techniques, such as OpenMP. Another often�@used programming model is the
single-level flat MPI model, in which separate single-threaded MPI
processes are executed on each core.

In the previous work [1], author applied OpenMP/MPI hybrid parallel
programming models to finite-element based simulations of linear
elasticity problems. The developed code has been tested on the �gT2K Open
Supercomputer�h [2] using up to 512 cores. Performance of OpenMP/MPI
hybrid parallel programming model is competitive with that of flat MPI
using appropriate command lines for NUMA control. Furthermore, reordering
of the mesh data for contiguous access to memory with first touch data
placement provides excellent improvement on performance of OpenMP/MPI
hybrid parallel programming models, especially if the problem size for
each core is relatively small. Generally speaking, OpenMP/MPI hybrid
parallel programming model provides excellent performance for strong
scaling cases where problems are less memory-bound.

In the present work, OpenMP/MPI hybrid parallel programming models were
implemented to 3D finite-volume based simulation code for groundwater
flow problems through heterogeneous porous media using parallel iterative
solvers with multigrid preconditioning, which was developed in [3].
Multigrid is a scalable method and expected to be a promising approach
for large-scale computations, but there are no detailed research works
where multigrid methods are evaluated on multi-core/multi-socket clusters
using OpenMP/MPI hybrid parallel programming models. In this work,
developed code has been evaluated on the �gT2K Open Super Computer (Todai
Combined Cluster) (T2K/Tokyo)�h at the University of Tokyo, and
�gCray-XT4�h at National Energy Research Scientific Computing Center
(NERSC) of Lawrence Berkeley National Laboratory [4] using up to 1,024
cores, and performance of flat MPI and three kinds of OpenMP/MPI hybrid
parallel programming models are evaluated.

Optimization procedures for OpenMP/MPI hybrid parallel programming models
originally developed for 3D FEM applications, such as appropriate command
lines for NUMA control, first touch data placement and reordering of the
mesh data for contiguous access to memory, provided excellent improvement
of performance on multigrid preconditioners with OpenMP/MPI hybrid
parallel programming models. OpenMP/MPI hybrid demonstrated better
performance and robustness than flat MPI, especially with large number of
cores for ill-conditioned problems. Thus, hybrid parallel programming
model could be a reasonable choice for large-scale computing on
multi-core/multi-socket clusters.

References
\begin{enumerate}
\item Nakajima, K.: Flat MPI vs. Hybrid: Evaluation of Parallel
Programming Models for Preconditioned Iterative Solvers on �gT2K Open
Supercomputer�h, IEEE Proceedings of the 38th International Conference on
Parallel Processing (ICPP-09), pp.73-80 (2009)

\item Information Technology Center, The University of Tokyo:
http://www.cc.u-tokyo.ac.jp/

\item Nakajima, K.: Parallel Multilevel Method for Heterogeneous Field,
IPSJ Proceedings of HPCS 2006, pp.95-102 (2006) (in Japanese)

\item NERSC, Lawrence Berkeley National Laboratory: http://www.nersc.gov/

\end{enumerate}


\end{document}
