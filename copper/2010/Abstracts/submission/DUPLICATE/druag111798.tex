\documentclass{report}
\usepackage{amsmath,amssymb}
\setlength{\parindent}{0mm}
\setlength{\parskip}{1em}
\begin{document}
\begin{center}
\rule{6in}{1pt} \
{\large Andrei Dr{\u a}g{\u a}nescu \\
{\bf Multigrid preconditioning of linear systems for interior point methods applied to a class of box-constrained optimal control problems}}

Department of Mathematics and Statistics \\ University of Maryland Baltimore County \\ Baltimore MD 21250
\\
{\tt draga@umbc.edu}\\
Cosmin Petra\end{center}

In this work we construct and analyze multigrid preconditioners for
operators of the form \mbox{${\mathcal D}_{\lambda}+{\mathcal
K}^*{\mathcal K}$}, where $D_{\lambda}$ is the multiplication with a
relatively ``smooth'' function $\lambda>0$ and ${\mathcal K}$ is a
discretization of a compact linear operator. These systems arise when
applying interior point methods to the distributed optimal control
problem \mbox{$\min_u\frac{1}{2}(|\!|{\mathcal K} u-f|\!|^2
+\beta|\!|u|\!|^2)$} with box constraints $\underline{u}\le
u\le\overline{u}$ on the control $u$. The presented preconditioning
technique is related to the one developed by Dr{\u a}g{\u a}nescu and
Dupont in~\cite{dradup} for the associated unconstrained problem, and is
intended for large-scale problems. As in~\cite{dradup}, the quality of
the resulting preconditioners is shown to increase as $h\downarrow 0$ at
a rate that is optimal with respect to $h$ if the meshes are uniform, but
decreases as the smoothness of $\lambda$ declines. We test this algorithm
first on a Tikhnov-regularized backward parabolic equation with $[0,1]$
constraints and then on the elliptic-constrained optimization problem
\begin{eqnarray*}
\begin{array}{cc}\vspace{7pt}
\textnormal{minimize}& \frac{1}{2}|\!|y-f|\!|^2 +
\frac{\beta}{2}|\!|u|\!|^2 \\ \textnormal{subj. to\ \ } &\Delta
y=u\ ,\ y\in H_0^1(\Omega),\ \underline{u} \le u \le \overline{u}\
\ a.e.
\end{array}
\end{eqnarray*}
In both cases it is shown that the number of linear iterations per
optimization step, as well as the total number of fine-scale
matrix-vector multiplications is decreasing with increasing resolution,
thus showing the method to be potentially very efficient for truly
large-scale problems.

This is joint work with Cosmin Petra from the Argonne National Laboratory.

\begin{thebibliography}{99}
\bibitem{dradup} Andrei Dr{\u a}g{\u a}nescu and Todd F.~Dupont.
\emph{Optimal order multilevel preconditioners for regularized ill-posed
problems}, Math. Comp., 77 (2008), pp. 2001--2038.


\end{document}
