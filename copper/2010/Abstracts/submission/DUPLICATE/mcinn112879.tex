\documentclass{report}
\usepackage{amsmath,amssymb}
\setlength{\parindent}{0mm}
\setlength{\parskip}{1em}
\begin{document}
\begin{center}
\rule{6in}{1pt} \
{\large Lois Curfman McInnes \\
{\bf Coupled Core-Edge Fusion Simulations}}

Mathematics and Computer Science Division \\ Argonne National Laboratory \\ 9700 South Cass Avenue \\ Argonne \\ IL 60439
\\
{\tt mcinnes@mcs.anl.gov}\\
John Cary\end{center}

Construction of large test fusion reactors such as the International
Thermonuclear Experimental Reactor (ITER) will require comprehensive
full-device modeling to answer key design questions that are far too
expensive to solve through direct experimentation. Full-device modeling
will require the self-consistent coupling of codes simulating different
physical processes. The regions that develop are distinct in time scales
and the physics needed to model them: (1) the core region is a
high-temperature, magnetically con&#64257;ned plasma where the fusion
reactions take place; (2) the edge region is a colder plasma where the
parallel transport competes with the perpendicular transport and the
material wall interactions cause atomic and chemical processes to be
important; and (3) the material wall contains approximately 90 percent of
the fuel in the entire system because of absorption of the plasma fueling
gas by the material lattice. Researchers have developed models with
varying degrees of accuracy for describing the physics of a region. These
different mathematical models subsequently have led to different codes
being written, each with many varying implementations and numerical
techniques.

To enable modeling from the material wall to the plasma core, the
Framework Application for Core-Edge Transport Simulations (FACETS)
project is developing a multiphysics, parallel application. An integral
aspect of this work is the coupling of core and edge simulations, along
with transport and wall interactions. The simplest model for the core
eliminates the fast parallel time scale analytically and models only the
transport of the plasma across the &#64257;eldlines, a slow process. The
models for this transport are stiff because the &#64258;ux transport
increases rapidly past a critical threshold. Modern tokamaks operate in
the regime where the boundary is near this threshold and thus displays a
sensitivity to the core-edge boundary for the regimes of interest.

This presentation will discuss the approach used in FACETS for core-edge
coupling, including issues involved in managing concurrent component
parallelism. We will provide an overview of the components that model the
core and edge regions, and we will explain the role of preconditioned
Newton-Krylov methods in this context. We will present preliminary
results using the FACETS framework for coupled core-edge simulations of
H-mode pedestal buildup in the DIII-D tokamak.

In addition, we will explain how this fusion research is motivating new
capabilities in the PETSc library to better handle strong coupling
between two or more distinct mathematical models based on partial
differential equations. Two complementary aspects of research are
multimodel algebraic system solution, or how to solve the resulting
coupled nonlinear algebraic systems, using Newton-based methods with
multiphysics-based linear and/or nonlinear preconditioning, and
multimodel algebraic system specification, or how to provide the
application programmer the flexibility to specify subsets of physics from
which the subsolvers may be efficiently constructed.


\end{document}
