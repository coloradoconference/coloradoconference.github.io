\documentclass{report}
\usepackage{amsmath,amssymb}
\setlength{\parindent}{0mm}
\setlength{\parskip}{1em}
\begin{document}
\begin{center}
\rule{6in}{1pt} \
{\large Homer F. Walker \\
{\bf Anderson Acceleration for Fixed-Point Iterations}}

Mathematical Sciences Dept \\ Worcester Polytechnic Institute \\ 100 Institute Road \\ Worcester \\ MA 01609-2280
\\
{\tt walker@wpi.edu}\\
Peng Ni\end{center}

Fixed-point iterations occur naturally and are commonly used in a broad
variety of computational science and engineering applications. In
practice, fixed-point iterates often converge undesirably slowly, if at
all, and procedures for accelerating the convergence are desirable. This
talk will focus on a particular acceleration method that originated in
work of D. G. Anderson [J. Assoc. Comput. Machinery, 12 (1965), 547-560].
This method has enjoyed considerable success in electronic-structure
computations but seems to have been untried or underexploited in many
other important applications. Moreover, while other acceleration methods
have been extensively studied by mathematicians and numerical analysts,
Anderson acceleration has received relatively little attention from them,
despite there being many significant unanswered mathematical questions.
In this talk, I will outline Anderson acceleration, discuss some of its
theoretical properties, and demonstrate its performance in several
applications.


\end{document}
