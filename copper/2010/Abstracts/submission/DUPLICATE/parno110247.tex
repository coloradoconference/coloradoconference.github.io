\documentclass{report}
\usepackage{amsmath,amssymb}
\setlength{\parindent}{0mm}
\setlength{\parskip}{1em}
\begin{document}
\begin{center}
\rule{6in}{1pt} \
{\large Matthew Parno \\
{\bf Population based optimization of mixed-variable problems with applications in hydrology}}

235 Albany St \\ Apartment 3073 \\ Cambridge \\ MA 02139
\\
{\tt mparno@mit.edu}\end{center}

Many difficult engineering optimization problems involve both continuous
and discrete variables. Additionally, some problems may have discrete
variables that alter the dimension of the continuous problem. Formally,
this work considers problems of the form
\begin{eqnarray*}
\mbox{min}& & f(x,y,z)\\
& & z\in F_z \\
& & y \in F_y(z)\\
& & x \in F_x(y,z)
\end{eqnarray*}
where $F_z\subseteq Z^{n_z}$, $F_y(z)\subseteq Z^{n_y(z)}$, and
$F_x(y,z)\subseteq \Re^{n_x(z)}$. The variables in $F_z$ and $F_y(z)$ do
not need to be ordered and can be categorical. In general, the continuous
problem $f(z,y,\cdot )$ will be nonsmooth and derivatives will not be
available. Thus, current algorithms alternate local searches with
derivative free optimization algorithms on the continuous variables with
a local search of the discrete parameters However, the continuous problem
can often be noisy with many local minima.

To this end, we review the existing framework and demonstrate that a
population based algorithm can also be used as a more robust continuous
optimization algorithm. Numerical results are given for several academic
problems as well as a hydraulic capture problem from hydrology.


\end{document}
