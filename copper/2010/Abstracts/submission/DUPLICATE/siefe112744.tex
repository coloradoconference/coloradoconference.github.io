\documentclass{report}
\usepackage{amsmath,amssymb}
\setlength{\parindent}{0mm}
\setlength{\parskip}{1em}
\begin{document}
\begin{center}
\rule{6in}{1pt} \
{\large Chris Siefert \\
{\bf Block Preconditioning for Implicit Ocean Models}}

Sandia National Laboratories \\ P O Box 5800 \\ MS 0378 \\ Albuquerque \\ NM 87185-0378
\\
{\tt csiefer@sandia.gov}\\
Andy Salinger\end{center}

Ocean modeling is a demanding multi-scale application, involving the
coupling of phenomena at various time and spatial scales. One
problem that scalable ocean models need to address is the spin-up
problem, which requires time integration over intervals measured in
centuries. Implicit models allow us to bypass CFL-induced timestep
limits, but leave us with the challenge of developing effective
preconditioners. We focus our attention on the implicit variant of the
Parallel Ocean Program (POP), which solves a thin stratified fluid
problem utilizing the hydrostatic and Boussinesq approximations. We
consider block preconditioners to address the coupling between the
velocity, salinity-temperature and pressure blocks. We present modest
theoretical results which guide the choice of preconditioners as well as
numerical results illustrating the effect of those choices in practice.


\end{document}
