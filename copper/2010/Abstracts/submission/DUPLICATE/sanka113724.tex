\documentclass{report}
\usepackage{amsmath,amssymb}
\setlength{\parindent}{0mm}
\setlength{\parskip}{1em}
\begin{document}
\begin{center}
\rule{6in}{1pt} \
{\large Sethuraman Sankaran \\
{\bf Robust design of cardiovascular surgeries - a derivative-free stochastic optimization approach}}

3922 Camino Calma \\ San Diego \\ CA 92122
\\
{\tt sesankar@ucsd.edu}\\
Alison Marsden\\
Charles Audet\end{center}

Recent advances in coupling novel optimization methods to large-scale
computing problems have opened the door to tackling a diverse set of
physically realistic engineering design problems. A large computational
overhead is associated with computing the objective function for most
practical problems involving complex physical phenomena. Such problems
are also plagued with uncertainties in a diverse set of parameters. We
present a computational framework for solving stochastic optimization
problems with multiple constraints. We demonstrate the use of this method
by coupling the optimization routine with a customized finite element
flow solver for cardiovascular surgery design.

A novel and non-intrusive adaptive stochastic collocation [1] technique
is used to account for uncertainties. Randomness is viewed as a separate
dimension (akin to space and time) which is referred to as stochastic
space. The stochastic space is numerically discretized using Chebyshev
points and Lagrange interpolates. Function adaptive Smolyak sparse grids
are computed using a novel algorithm that relies on hierarchical error
indicators. Optimization is performed using a derivative-free technique
called the Surrogate Management Framework (SMF), in which Kriging
interpolation functions are used to approximate the objective function.
The SMF method with mesh adaptive direct search polling strategy [2,3] is
extended to stochastic optimization problems. The filter approach is used
to incorporate non-linear probabilistic constraints. We show that the
iterates generated by the algorithm are stationary in the stochastic
sense using a moment-based approach [4]. The optimization algorithm is
validated on systems characterized by partial differential equations such
as thermal (heat-conduction) and solid mechanics problems with
reliability constraints. In each optimization iteration, a set of PDE�s
are solved. The numerical efficiency of this method is compared with
Monte-Carlo techniques. To further reduce the computational effort, a
stochastic response surface (SRS) technique is developed to approximate
higher order solution statistics. This results in a significant reduction
in the number of simulations and hence, the computational time.

The stochastic optimization technique is applied to cardiovascular bypass
graft shape optimization problems of potential clinical interest. We
optimize shape of BG surgeries to minimize adverse hemodynamic flow
conditions such as low wall shear stresses. Numerical simulations of
blood flow in the cardiovascular system are performed using custom finite
element methods and specialized outflow boundary conditions tailored for
cardiovascular applications. We account for uncertainties in
cardiovascular simulations which occur due to factors such as: (a)
reproducing simulation models based on MRI images, and (b) boundary
conditions including velocity, resistance and lumped parameter models. We
compare and contrast the robust design solutions with results obtained
using deterministic optimization. The SRS technique shows an order of
magnitude improvement in the computational efficiency over the
conventional stochastic collocation technique.

References:

[1] S.Sankaran and A.L.Marsden, �A stochastic collocation method for
uncertainty quantification and propagation in cardiovascular
simulations�, Journal of Biomechanical Engineering, in review.
[2] C.Audet, J.E.Dennis Jr., �Mesh adaptive direct search algorithms for
constrained optimization�, SIAM J. Optim. 2006;17 (1), 2�11.
[3] A.L.Marsden, J.A.Feinstein, and C.A. Taylor, �A computational
framework for derivative-free optimization of cardiovascular geometries�,
Computer Methods in Applied Mechanics and Engineering, 2008; 197 (21-24)
1890-1905.
[4] S.Sankaran, C.Audet, and A.L.Marsden, �A method for stochastic
constrained optimization using derivative-free surrogate pattern search
and collocation�, Journal of Computational Physics, in review.


\end{document}
