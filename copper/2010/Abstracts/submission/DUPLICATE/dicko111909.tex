\documentclass{report}
\usepackage{amsmath,amssymb}
\setlength{\parindent}{0mm}
\setlength{\parskip}{1em}
\begin{document}
\begin{center}
\rule{6in}{1pt} \
{\large Thomas Dickopf \\
{\bf Semi-geometric monotone multigrid methods using non-nested meshes}}

Institute for Numerical Simulation \\ University of Bonn \\ Wegelerstr 6 \\ 53115 Bonn \\ Germany
\\
{\tt dickopf@ins.uni-bonn.de}\\
Rolf Krause\end{center}

This talk is about multilevel discretization and solution strategies for
variational equalities and inequalities arising from, e.\,g., contact
problems in elasticity. Many important applications in computational
engineering, especially involving complicated geometries in three
dimensions, do not allow for a straightforward multilevel hierarchy.
Moreover, it may be a challenging task to obtain a proper multilevel
representation of the (contact) constraints. As fast and efficient
preconditioners for a broad set of linear problems, we introduce a new
class of semi-geometric multigrid methods that is based on non-nested
meshes. On top of that, a monotone method is developed by employing
appropriate (nonlinear) smoothers and local modifications of the coarse
level problems.

We compare existing and new strategies serving the above purposes. The
analysis of the semi-geometric method using non-nested meshes can be
carried out in a natural way by looking at the associated non-nested
spaces and the connecting operators. Going beyond previous studies in
context of, e.\,g., the auxiliary space method, we examine traditional
and novel operators for the implementation of an efficient and stable
information transfer between non-nested finite element spaces. The
treatment of boundary data for the coarse level problems is also
reconsidered. A particular advantage of our method is that the coarse
meshes can be chosen quite freely, e.\,g., generated independently by
standard mesh generators. Thus, the presented approach allows for a more
direct control of the additional coarse degrees of freedom; the little
geometric information entering the setup leads to a very efficient
multilevel hierarchy and both grid and operator complexity are
particularly small.

The convergence properties of both the proposed linear multilevel methods
and the (nonlinear) monotone methods as well as their complexities are
discussed in detail. We show various numerical results for three
dimensional applications and present a numerical comparison of existing
strategies and our new approach. Several selected prolongation and
restriction operators are studied and hints about the efficient
computation of inner products are given. We demonstrate the robustness of
the proposed semi-geometric methods with respect to the mesh size for
scalar problems and systems of PDEs.


\end{document}
