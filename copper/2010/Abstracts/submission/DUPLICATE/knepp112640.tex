\documentclass{report}
\usepackage{amsmath,amssymb}
\setlength{\parindent}{0mm}
\setlength{\parskip}{1em}
\begin{document}
\begin{center}
\rule{6in}{1pt} \
{\large Sarah M. Knepper \\
{\bf Iterative Restoration of Motion Blurred Brain Images}}

Dept of Math & CS \\ Emory University \\ 400 Dowman Dr  \\ W401 \\ Atlanta \\ GA 30322
\\
{\tt smknepp@emory.edu}\\
James G. Nagy\end{center}

Due to the long duration time of a positron emission tomography (PET)
scan, a patient may move. These movements degrade the reconstructed
image, especially as the resolution of scanners increases. Removing these
degradations through computational post-processing requires solving a
large-scale linear inverse problem. The quality of the model, and hence
the quality of the post-processing, depends on how accurately the patient
motion information can be measured. When imaging a rigid object, such as
the brain, the patient movements may be tracked and recorded with fairly
high precision. In this talk we describe the model problem, and in
particular how the matrix is generated from a series of interpolation
matrices based on the given patient motion. We discuss how sparsity
changes as we incorporate more motion information, and how to exploit
structure of the problem for parallel implementations. Results from
simulations as well as real patient data are presented to illustrate the
effectiveness of our approach.


\end{document}
