\documentclass{report}
\usepackage{amsmath,amssymb}
\setlength{\parindent}{0mm}
\setlength{\parskip}{1em}
\begin{document}
\begin{center}
\rule{6in}{1pt} \
{\large Chao Yang \\
{\bf Heuristics for Accelerating Electronic Structure Calculations}}

Lawrence Berkeley National Lab \\ MS-50F \\ 1 Cyclotron Rd \\ Berkeley \\ CA 94720
\\
{\tt cyang@lbl.gov}\\
Chao Yang\\
Juan Meza\end{center}

Under the Kohn-Sham density functional theory framework, the electron
density (rho) associated with a ground state atomistic system can be
obtained by solving the nonlinear equation rho=diag(f(H(rho))) where f is
the Fermi-Dirac distribution function and H is the Kohn-Sham Hamiltonian.
One way to solve this nonlinear equation is to apply a Broyden type of
method. I will discuss a number of heuristics for constructing an
effective Jacobian approximation and efficient ways to evaluate
diag(f(H(rho))).


\end{document}
