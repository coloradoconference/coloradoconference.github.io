\documentclass{report}
\usepackage{amsmath,amssymb}
\setlength{\parindent}{0mm}
\setlength{\parskip}{1em}
\begin{document}
\begin{center}
\rule{6in}{1pt} \
{\large Andrew Yeckel \\
{\bf Multiphysics challenges for the modeling of melt crystal growth systems}}

Department of Chemical Engineering and Materials Science \\ University of Minnesota \\ 421 Washington Ave SE \\ Minneapolis \\ MN 55406-0132
\\
{\tt yecke003@umn.edu}\\
Lisa Lun\\
Hui Xie\\
Yousef Saad\\
Jeffrey J. Derby\end{center}

Melt crystal growth presents several major challenges in numerical
analysis of continuum mechanics, particularly in three-dimensional
formulations. Problem complexity presents a significant technical
challenge, requiring the integration of a large-scale furnace model with
a strongly-coupled multiphysics transport problem of the Stefan
moving-boundary type. Problem nonlinearity can be severe, due to
high-temperature radiation heat transfer effects and strong, richly
structured laminar flows of a transitional nature. Robust computing of
steady-state solutions under these conditions can be achieved by
Newton-Raphson iteration, but its desirable quadratic convergence
property relies on our ability to compute a sufficiently accurate
approximation to the inverse of a Jacobian matrix of the global system of
equations. In addition, problem scaling in three dimensions weighs
heavily in favor of iterative over direct solvers for this task, but
developing effective solution methods is faced with several distinct
challenges, three of which will be discussed here. All relate to
reductions of the global, linearized system that are either advantageous,
or forced by circumstance, in high-performance computing situations.

Due to time constraints, the first two challenges will only briefly be
discussed. The first is finding an effective preconditioner for the
incompressible Navier-Stokes equations for a closed, buoyancy-driven flow
at high Rayleigh number. In mathematical terms, the Jacobian is made
strongly skew by the temperature coupling in the Boussinesq term in the
flow equations, an effect that habitually causes the iterative solver to
stall. In physical terms, it is difficult to design physics-based
preconditioners for these flows, which lack any obvious geometric
simplification due to their tendency towards flow separation and steep
internal layers. A block Jacobi/ILU(0)-based approach is tested and found
to be limited in use to Rayleigh numbers well below those encountered in
crystal growth practice, forcing us to seek alternatives.

The next challenge is the effect of the implicit constraint between
temperature and problem geometry at the moving boundary of the Stefan
problem. This too causes severe stalling of the iterative solver. Here
the problem coupling is isolated to a few equations that determine the
position of the crystal-melt interface, and it is possible to take
advantage of a Schur complement-based reduction of the problem to develop
a sequential iteration between geometry and temperature that results in
linear systems much easier to solve than the full system. Since this
iteration formulation is based on the full system of equations, the
robustness of a strongly coupled approach can be attained that improves
upon a simple fixed-point iteration.

The third challenge, which will be the primary focus of this
presentation, is the complexity of the global simulation problem, for
which it is costly to develop monolithic software that simultaneously
represents all chosen phenomena at all scales in a single model. From a
practical standpoint, problems of this scope favor a partitioned
approach, in which a few major subdomains of the problem are tackled
independently by existing software best suited to the task. The most
powerful techniques are those that can be applied when the equations are
solved by black-box computer codes that allow no intervention in their
algorithms. Such methods can be used to link together existing
best-in-class tools to tackle complex multiphysics and multiscale
problems, without requiring extraordinary programming effort.

Towards this end, we have developed an approximate block Newton method to
couple arbitrary black-box nonlinear solvers. This ABN method preserves
the quadratic approximation properties of exact Newton iteration. The
notion of a solver is abstract, encompassing any interpolations or other
transformations of data exchanged between solvers. It is shown that the
method behaves like a Newton iteration preconditioned by an inexact
Newton solver derived from subproblem Jacobians. The method is
demonstrated on several conjugate heat transfer problems modeled after
melt crystal growth processes. Whereas a typical block Gauss--Seidel
iteration fails about half the time for the model problems, quadratic
convergence is achieved by the ABN method under all conditions studied.

------------------

This work has been supported in part by a seed grant from the Minnesota
Supercomputing Institute and by the Department of Energy, National
Nuclear Security Administration, under Award Numbers DE-FG52-06NA27498
and DE-FG52-08NA28768, the content of which does not necessarily reflect
the position or policy of the United States Government, and no official
endorsement should be inferred.


\end{document}
