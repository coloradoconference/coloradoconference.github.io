\documentclass{report}
\usepackage{amsmath,amssymb}
\setlength{\parindent}{0mm}
\setlength{\parskip}{1em}
\begin{document}
\begin{center}
\rule{6in}{1pt} \
{\large Victoria Howle \\
{\bf Investigation of Soft Errors as Integrated Components of a Simulation}}

Texas Tech University \\ Department of Mathematics and Statistics \\ Broadway and Boston \\ Lubbock \\ TX 79409-1042
\\
{\tt victoria.howle@ttu.edu}\\
Patricia Hough\\
Michael Heroux\\
Ellen Durant\end{center}

As architectures become more and more complex, failures come increasingly
into play and simulation behavior becomes less predictable. Especially
worrisome is the emerging prevalence of soft errors, i.e. bit
&#64258;ips, due to systems operating at such low voltages. Soft errors
are especially insidious because they may not even be detected. In this
work, we investigate the sensitivity of several linear solvers to soft
errors.

In many simulations, linear algebra accounts for more than 80\% of the
computational time, making it a critical component in over-all fault
tolerance. However, the linear solver does not function in isolation and
must be considered as an integrated component of the algorithmic stack in
a simulation. We investigate the effects of soft errors in the linear
solver on the overall simulation.


\end{document}
