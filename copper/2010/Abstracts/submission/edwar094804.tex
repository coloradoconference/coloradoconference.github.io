\documentclass{report}
\usepackage{amsmath,amssymb}
\setlength{\parindent}{0mm}
\setlength{\parskip}{1em}
\begin{document}
\begin{center}
\rule{6in}{1pt} \
{\large H. Carter Edwards \\
{\bf Performance Concerns when Iterating Hybrid-Parallel Kernels}}

Sandia National Laboratories \\ P O Box 5800 / MS 0382 \\ Albuquerque \\ NM 87185
\\
{\tt hcedwar@sandia.gov}\end{center}

Iterative algorithms are typically implemented by a sequence of calls to
simple computational kernels, such as the BLAS or their sparse
equivalent. Hybrid-parallelization of these kernels on clusters of nodes
with multicore CPUs or GPGPUs has demonstrated performance gains for
individual kernels. An iterative algorithm can realize a similar
performance gain only if the programming model for calling a sequence of
these kernels does not introduce significant overhead. Such a programming
model for hybrid-parallel kernels has been implemented in Trilinos�
ThreadPool library. A simple CG iterative solver is implemented using the
ThreadPool library and its hybrid-parallel performance is assessed.


\end{document}
