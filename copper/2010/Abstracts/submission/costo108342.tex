\documentclass{report}
\usepackage{amsmath,amssymb}
\setlength{\parindent}{0mm}
\setlength{\parskip}{1em}
\begin{document}
\begin{center}
\rule{6in}{1pt} \
{\large Anne S. Costolanski \\
{\bf EFFICIENT IMPLEMENTATION OF THE WIGNER-POISSON FORMULATION FOR MODELING A RESONANT TUNNELING DIODE }}

920 Marilyn Drive \\ Raleigh \\ NC 27607
\\
{\tt ascostol@ncsu.edu}\\
C. T. Kelley\end{center}

Using the Wigner-Poisson equations to model the behavior of a resonant
tunneling diode, we have developed a more efficient algorithm that is
capable of simulating a larger variety of device structures than was
previously available. Several improvements have been made to increase
accuracy and reduce computation time, including the use of fourth-order
numerical methods, the use of non-uniform grids, and the incorporation of
analytical rather than numerical solution techniques. These improvements
allowed the number of processors to be reduced from 20 to one without
impacting run times.

Using our new code, we were able to show that using longer device lengths
reduces the numerical inconsistencies present when modeling shorter
devices. We also studied the impact of the correlation length parameter
in producing solutions that correspond with those typically expected from
experimental measurement.


\end{document}
