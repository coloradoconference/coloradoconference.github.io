\documentclass{report}
\usepackage{amsmath,amssymb}
\setlength{\parindent}{0mm}
\setlength{\parskip}{1em}
\begin{document}
\begin{center}
\rule{6in}{1pt} \
{\large Elisabeth Ullmann \\
{\bf Iterative solvers for Stochastic Galerkin discretizations of PDEs with random data}}

Institut fuer Numerische Mathematik und Optimierung \\ TU Bergakademie Freiberg \\ 09596 Freiberg \\ Germany
\\
{\tt ullmann@math.tu-freiberg.de}\end{center}

We study efficient iterative solvers for Galerkin equations associated
with finite element discretizations of second-order elliptic partial
differential equations (PDEs) with random coefficient functions. Such
systems arise, for example, from discretized stochastic diffusion
problems. The Galerkin matrix is a sum of Kronecker products of pairs of
matrices associated with the physical and stochastic discretization,
respectively. Due to the coupling of (standard) finite element
discretizations in the physical space and global polynomial chaos
approximations on a probability space, the number of unknowns is huge.
Moreover, depending on the models employed for the random coefficient
function, the Galerkin matrix can be block-dense, and the cost of a
matrix-vector product is non-trivial.

We review a recently proposed Kronecker product preconditioner for the
Galerkin matrix which - in contrast to the popular mean-based
preconditioner - makes use of the entire information contained in the
Galerkin matrix. Furthermore, we extend the idea of Kronecker product
preconditioning to the discretized mixed formulation of the stochastic
diffusion equation. We demonstrate numerically the improved robustness of
Kronecker product preconditioners compared to the mean-based approach
with respect to key statistical parameters of lognormal diffusion
coefficients.

Finally we discuss perspectives for multilevel techniques for the
Galerkin equations. In contrast to previous works, where many researchers
have applied multilevel methods exclusively to the deterministic finite
element spaces, we focus on multilevel decompositions for the stochastic
variational space and combined deterministic/stochastic multilevel
approaches.


\end{document}
