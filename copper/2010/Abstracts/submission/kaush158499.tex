\documentclass{report}
\usepackage{amsmath,amssymb}
\setlength{\parindent}{0mm}
\setlength{\parskip}{1em}
\begin{document}
\begin{center}
\rule{6in}{1pt} \
{\large Dinesh Kaushik \\
{\bf Computational Challenges in Coupled Multiphysics Simulations of Nuclear Reactor Cores}}

1728 Trevino Cir \\ Bolingbrook \\ IL 60490
\\
{\tt kaushik@mcs.anl.gov}\\
Micheal Smith\\
Allan Wollaber\\
Timothy Tautges\\
Barry Smith\\
Andrew Siegel\\
Won Sik Yang\end{center}

As part of the DOE�s Nuclear Energy Advanced Modeling and Simulation
(NEAMS) program, SHARP framework is being developed at Argonne National
Laboratory to carry out coupled multiphysics reactor core simulations in
high fidelity. The goal of this simulation effort is to reduce the
uncertainties and biases in reactor design calculations by progressively
replacing existing multi-level averaging (homogenization) techniques with
more direct solution methods. This talk will discuss the architectural
and algorithmic challenges encountered in these simulations on the
leadership platforms (such as Blue Gene/P at Argonne and XT5 at ORNL). We
will also present detailed, high-resolution simulations of neutron
transport in fast reactor cores using our code UNIC. This code implements
a scalable solution methodology for the discrete ordinates, even-parity
form of the neutron transport equation. For high-fidelity descriptions of
complex reactor geometries (respecting spatial heterogeneities and large
number of energy groups), the memory requirements are huge. We will
discuss an approach (using p-multigrid) to contain the memory requirement
in the context of Zero Power Reactor (ZPR) Experiment 6/6a simulations on
Blue Gene/P and XT5. This talk will also highlight the importance of
developing memory-saving algorithms for multiphysics simulations � a
crucial requirement for the future petascale (and exascale) machines
where the current trends indicate significantly reduced memory per thread
of execution.


\end{document}
