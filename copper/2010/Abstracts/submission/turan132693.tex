\documentclass{report}
\usepackage{amsmath,amssymb}
\setlength{\parindent}{0mm}
\setlength{\parskip}{1em}
\begin{document}
\begin{center}
\rule{6in}{1pt} \
{\large Erhan Turan \\
{\bf Set Reduction In Nonlinear Equations}}

Bogazici University \\ Department of Mechanical Engineering \\ Istanbul \\ Turkey
\\
{\tt turane@boun.edu.tr}\\
Erhan Turan\\
Ali Ecder\end{center}

In this paper, an idea to solve nonlinear equations is presented. During
the solution of any problem with Newton's Method, it might happen that
some of the unknowns satisfy the convergence criteria where the others
fail. The convergence happens only when all variables reach to the
convergence limit. A method to reduce the dimension of the overall system
by excluding some of the unknowns that satisfy an intermediate tolerance
is introduced. In this approach, a smaller system is solved in less
amount of time and already established local solutions are preserved and
kept as constants while the other variables that belong to the ``set"
will be relaxed. To realize the idea, an algorithm is given that utilizes
applications of pointers to reduce and evaluate the sets. Matrix-free
Newton-Krylov Techniques are used on a test problem and it is shown that
proposed idea improves the overall convergence.


\end{document}
