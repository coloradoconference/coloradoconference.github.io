\documentclass{report}
\usepackage{amsmath,amssymb}
\setlength{\parindent}{0mm}
\setlength{\parskip}{1em}
\begin{document}
\begin{center}
\rule{6in}{1pt} \
{\large Joel E. Dendy \\
{\bf A Cell-Structure-Preserving Multigrid Method for the Diffusion Equation}}

Los Alamos National Laboratory \\ MS-B284 \\ P O Box 1663 \\ Los Alamos \\ NM 87545
\\
{\tt jed@lanl.gov}\\
Joel E. Dendy\\
J. David Moulton\end{center}

We consider the standard cell-centered discretization of the
diffusion equation on a rectangle with discontinuous diffusion coefficient.
Previously we considered an extension of black box multigrid with a
coarsening factor of three. For a coarsening factor of two, the
standard application of black box multigrid has been to coarsen the
logically rectangular grid. This approach does not preserve the cell
structure on coarser grids, as one would wish to do for local grid
refinement and other applications. If variational coarsening is
employed, with interpolation being a generalization of bilinear
interpolation, it is well known that the difference operator on the
next coarser grid has a twenty-five point stencil. In this work we
extend the difference operator on the cell-centers of the finest grid
to be a difference operator on the underlying grid consisting of cell-
centers, cell-vertices and cell-edges. For this extended operator we
derive an operator-induced interpolation, which is used to coarsen
variationally to a difference operator on the grid of cell-vertices.
This second operator is coarsened in the standard black box multigrid
fashion to the grid of coarse-grid cell centers. The resulting
operator has a nine point stencil, and the resulting interpolation
operator from coarse-grid cell-centers to fine grid cell-centers has
the same stencil as bilinear interpolation while the restriction
operator from fine grid cell-centers to coarse-grid cell-centers has
the same stencil as the transpose of bilinear interpolation.


\end{document}
