\documentclass{report}
\usepackage{amsmath,amssymb}
\setlength{\parindent}{0mm}
\setlength{\parskip}{1em}
\begin{document}
\begin{center}
\rule{6in}{1pt} \
{\large Glenn Hammond \\
{\bf Parallel Newton-Krylov Methods for Ultrascale Subsurface Reactive Multiphase Flow}}

P O Box 999; MSIN K9-36 \\ Richland \\ WA 99352
\\
{\tt glenn.hammond@pnl.gov}\\
Barry Lee\end{center}

Reactive multiphase flow and transport codes play a vital role in
predicting the migration of subsurface contaminants and evaluating the
viability of alternatives for mitigating global climate change (e.g.
geologic sequestration of anthropogenic carbon). To perform such
analyses, scientists incorporate sophisticated physical and chemical
process models within their simulators to assess the impact of reactive
fluids on subsurface porous media. For large 3D field-scale models, these
simulators employ preconditioned Newton-Krylov methods to solve the
tightly-coupled systems of equations governing multiscale physical and
chemical processes in the subsurface.

Due to the extreme stiffness of the systems arising from the strong
feedback mechanisms between processes and the wide range of timescales
involved, significant challenges exist in the development of scalable
solver/preconditioner algorithms for these coupled process models. In the
case of carbon sequestration, simulation requires at a minimum the
coupling of multiphase fluid flow (i.e. supercritical CO$_2$ and water),
energy, and salt (solute transport), which results in large and complex
Jacobian matrices with irregular blocks of coefficients representing the
coupling between processes within a grid cell or matrix element. For
conventional multilevel algorithms, this unpredictable coupling of system
PDEs presents new challenges, especially at the extreme scale (i.e.
10,000+ processor cores).

This presentation focuses on research in the development of multilevel
preconditioners for ultrascale subsurface problems. In particular,
results are presented on the performance of parallel Newton-Krylov
solvers employed within codes executed on leadership class computing
platforms such as Oak Ridge National Laboratory�s Jaguar for simulation
of variably-saturated, single phase flow with reactive transport and
multiphase carbon sequestration.


\end{document}
