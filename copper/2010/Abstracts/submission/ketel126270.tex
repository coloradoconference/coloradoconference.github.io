\documentclass{report}
\usepackage{amsmath,amssymb}
\setlength{\parindent}{0mm}
\setlength{\parskip}{1em}
\begin{document}
\begin{center}
\rule{6in}{1pt} \
{\large Christian Ketelsen \\
{\bf A Gauge Invariant Discretization of Quantum Electrodynamics by Least-Squares Finite Elements.}}

Department of Applied Mathematics \\ 526 UCB \\ University of Colorado \\ Boulder \\ CO 80309-0526
\\
{\tt ketelsen@colorado.edu}\\
Tom Manteuffel\\
Steve McCormick\end{center}

The two-dimensional Dirac equation of quantum electrodynamics (QED)
describes the interaction between electrons and photons. Large scale
numerical simulations of the theory require repeated solution of the
two-dimensional Dirac equation, a system of two &#64257;rst-order partial
differential equations coupled to a background gauge &#64257;eld.
Traditional discretizations of this system are sparse and highly
structured, but contain random complex entries introduced by the
background &#64257;eld. For even mildly disordered gauge &#64257;elds the
near kernel components of the system are highly oscillatory, rendering
standard multilevel methods ineffective.

Recently, least-squares finite element methods have been used to
discretize the governing equations of QED. The resulting linear systems
are ideal because they agree spectrally with the continuum equations, are
amenable to solution by multilevel iterative methods, and satisfy
important properties of the continuum theory, including chiral symmetry,
the absence of species doubling, and gauge invariance. The most important
of these properties is gauge invariance. It states simply that an
arbitrary transformation of the unknown, $\psi$, by a member of the gauge
group (U(1) in the case of QED) does not change the physics of the model.
Traditional covariant finite-difference discretizations of the Dirac
equation capture the property of gauge invariance in a trivial manner.
While previous least-squares discretizations do retain gauge invariance,
they do so through a complicated gauge fixing process.

The objective of this paper is to present a new least-squares
discretization that naturally avoids the need for gauge fixing. This is
done by assuming that the finite element solution takes the polar form
$\psi^h = r^h \hspace{1mm} e^{i \theta^h}$ , where $r^h$ and $\theta^h$
are chosen from an appropriate bilinear finite element space. In the
finite element setting, a U(1) gauge transformation of QED can be written
as $\Omega^h = e^{i \omega^h}$. Choosing $\omega^h$ from the same
bilinear finite element space ensures that the transformed unknown,
$\Omega^h \psi^h$, has the assumed polar form. Then, applying the
least-squares finite element methodology results in a discretization for
which gauge invariance is straight forward.

The trivial gauge invariance of the method comes at a price, however,
since the polar representation on the unknown makes the governing
equations inherently nonlinear. To carry out the discretization we first
linearize the governing equations and apply the least-squares methodology
to the resulting linear system. The subsequent matrix equation is then
solved via an algebraic multilevel method. To obtain a good initial guess
on the finest grid a nested-iteration approach is used. The computational
performance and physical properties of resulting solution methodology are
discussed.


\end{document}
