\documentclass{report}
\usepackage{amsmath,amssymb}
\setlength{\parindent}{0mm}
\setlength{\parskip}{1em}
\begin{document}
\begin{center}
\rule{6in}{1pt} \
{\large Roscoe Bartlett \\
{\bf Technical and Software Challenges for Integrating Embedded Optimization Algorithms in Production Application Codes}}

PO Box 5800 \\ Albuquerque \\ NM 87185-1318
\\
{\tt rabartl@sandia.gov}\\
Roscoe Bartlett\end{center}

Embedded gradient-based optimization algorithms can out perform
non-invasive black-box optimization algorithms by large margins in speed,
accuracy and robustness in many cases. However, incorporating an embedded
optimization algorithm in a complex production application code is very
difficult to achieve, especially after a large simulation code already
exists. Different heroic efforts to get embedded optimization algorithms
into production codes have ultimately failed to achieve any real impact
because the interface and capabilities were not maintained as the code
was continued to be developed. There are many obstacles to getting
embedded optimization into production applications and maintaining the
capability (so that I can be used when it is really needed). Production
codes may not use sufficiently smooth mathematical formulations or
numerical approaches to provide smooth function derivatives. Even if the
functions are smooth, the production code may not be able to efficiently
compute accurate derivatives (e.g. using an automatic differentiation
tool). The structure of the application code may be very inflexible and
not allow the exposure of basic model functions and derivative operators
or allow control by the embedded optimization algorithm. Related to this,
when an embedded optimization algorithm is incorporated into an
production application, if it is not intimately integrated into the basic
infrastructure of the application, then the �hooks� will be lost through
atrophy. Lastly, if the software for the complex embedded optimization
algorithms and the production application are not kept integrated on a
frequent schedule, then the integrated embedded optimization capability
can disappear as the two codes continue to be developed independently.

In this presentation, I will describe experience at Sandia National Labs
in trying to incorporate embedded optimization algorithms in the MOOCHO
(once called rSQP++) package into several different semi-production
application codes. Early failures of the form described above will be
mentioned as well as current efforts to address the problems. The core
problems are being addressed on all fronts and there are now examples of
integrated embedded optimization capabilities that are being maintained
over many years, even over long periods where active embedded
optimization work is being done. The primary approaches that will be
discussed are the concept of Abstract Numerical Algorithms and the Thyra
interface layer, automatic differentiation of C++ codes using
Trilinos/Sacado, the Trilinos/Thyra ModelEvaluator interface to define
the application infrastructure, and formalized frequent software
integration strategies to maintain integrated embedded optimization
capabilities.


\end{document}
