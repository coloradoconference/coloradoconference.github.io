\documentclass{report}
\usepackage{amsmath,amssymb}
\setlength{\parindent}{0mm}
\setlength{\parskip}{1em}
\begin{document}
\begin{center}
\rule{6in}{1pt} \
{\large Barry Lee \\
{\bf A Space-Angle-Energy Multigrid Method for Sn Discretizations of the Multi-Energetic Boltzmann Equation}}

Computational Sciences and Mathematics Division \\ Pacific Northwest National Laboratory \\ Richland \\ WA 99354 \\ USA
\\
{\tt barry.lee@pnl.gov}\end{center}

The multi-energetic Boltzmann equation is used to model neutron/photon
transport. It is 7-dimensional in spatial
3-dimensions, and 6-dimensional in spatial 2-dimensions (1 in time, 1 in
energy, 2 in angle, and 3 or 2 in space).
Development of a fast multigrid solver for this multi-energetic equation
has been rather slow, particularly because
the energy coupling is described through scattering cross-sections that
are highly oscillatory in energy. Thus, robust and
efficient homogenization techniques for coarsening these cross-sections
have been difficult, if not impossible,
to develop. In this talk, we describe a multi-energetic extension of the
space-angle semi-coarsening method for the
mono-energetic Boltzmann equation. The resulting method is also a
semi-coarsening scheme: semi-coarsening
first in energy, then in space or angle. Coarsening in energy is
algebraically based (i.e., it is based on the
energy features of the near nullspace component of the multi-energetic
Boltzmann operator), which permits simple
and efficient handling of the highly oscillatory scattering
cross-sections. In fact, the energy restriction
operator is constant, and the coarse energy-grid selection is based on a
simple bining of the original energy groups.
At each energy level of the energy hierarchy, relaxation
consists of a few sweeps of a space-angle semi-coarsening V-cycle, with
the angles considered collectively over
all the energy groups at that energy level. Spectral analysis and
numerical results will be given.


\end{document}
