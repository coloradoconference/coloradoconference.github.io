\documentclass{report}
\usepackage{amsmath,amssymb}
\setlength{\parindent}{0mm}
\setlength{\parskip}{1em}
\begin{document}
\begin{center}
\rule{6in}{1pt} \
{\large Virginia Torczon \\
{\bf A generating set search approach to nonlinear programming problems using an augmented Lagrangian method with explicit treatment of linear constraints}}

College of William & Mary \\ Department of Computer Science \\ P O Box 8795 \\ Williamsburg VA 23187-8795
\\
{\tt va@cs.wm.edu}\\
Robert Michael Lewis\end{center}

We consider solving nonlinear programming problems using an augmented
Lagrangian method that makes use of derivative-free generating set search
to solve the subproblems. Our approach is based on the augmented
Lagrangian framework of Andreani, Birgin, Mart\'{i}nez, and Schuverdt
which allows one to partition the set of constraints so that one subset
can be left explicit, and thus treated directly when solving the
subproblems, while the remaining constraints are incorporated in the
augmented Lagrangian. Our goal in this paper is to show that using a
generating set search method for solving problems with linear
constraints, we can solve the linearly constrained subproblems with
sufficient accuracy to satisfy the analytic requirements of the general
framework, even though we do not have explicit recourse to the gradient
of the augmented Lagrangian function. Thus we inherit the analytical
features of the original approach (global convergence and bounded penalty
parameters) while making use of our ability to solve linearly constrained
problems effectively using generating set search methods. We need no
assumption of nondegeneracy for the linear constraints. Furthermore, our
preliminary numerical results demonstrate the benefits of treating the
linear constraints directly, rather than folding them into the augmented
Lagrangian.


\end{document}
