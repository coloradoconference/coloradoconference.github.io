\documentclass{report}
\usepackage{amsmath,amssymb}
\setlength{\parindent}{0mm}
\setlength{\parskip}{1em}
\begin{document}
\begin{center}
\rule{6in}{1pt} \
{\large James Adler \\
{\bf Nested Iteration and Adaptive Local Refinement for Resistive Magnetohydrodynamic (MHD) Equations}}

255 S Corl St \#10 \\ State College \\ PA 16801
\\
{\tt adler@math.psu.edu}\\
Tom Manteuffel\\
Steve McCormick\\
John Ruge\\
Lei Tang\end{center}

In this talk, we propose new adaptive local refinement (ALR) strategies
for first-order system least-squares (FOSLS) finite element in
conjunction with algebraic multigrid (AMG) methods in the context of
nested iteration (NI). The goal is to reach a certain error tolerance
with the least amount of computational cost and nearly uniform
distribution of the error over all elements. This talk will develop
theory that supports this argument, as well as show experiments to
confirm that the algorithm can be efficient for MHD problems. These
methods are applied to a 2D reduced model of the incompressible,
resistive magnetohydrodynamic (MHD) equations. These equations are used
to simulate instabilities in a large aspect-ratio tokamak. We show that,
by using the new ALR strategies on this system, we are able to resolve
the physics using only 10 percent of the computational cost used to
approximate the solutions on a uniformly refined mesh within the same
error tolerance.


\end{document}
