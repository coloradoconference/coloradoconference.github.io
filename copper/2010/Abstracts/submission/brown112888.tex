\documentclass{report}
\usepackage{amsmath,amssymb}
\setlength{\parindent}{0mm}
\setlength{\parskip}{1em}
\begin{document}
\begin{center}
\rule{6in}{1pt} \
{\large Jed Brown \\
{\bf Implicit integration of 3D ice sheet flow using hybrid factorization/relaxation block preconditioning}}

Versuchsanstalt f�r Wasserbau \\ Hydrologie und Glaziologie \\ ETH Zentrum \\ 8092 Z�rich \\ Switzerland
\\
{\tt brown@vaw.baug.ethz.ch}\end{center}

Ice sheets are non-Newtonian free-surface flows with nonlinear slip
boundary conditions. Their
most dynamic coupling to other climate components is the point where ice
becomes floating as it
flows into the ocean, known as the grounding line. Grounding line
stability is unknown for many
important regions, thus introducing great uncertainty in sea level
forecasts. Since this regime
is fundamentally non-shallow, we present an ALE formulation in which the
velocity is governed by
Stokes equations with power-law rheology, and the mesh satisfies surface
kinematic equations with
elasticity in the interior to prevent tangling.

Semidiscretizaton using a high-order finite element method produces
differential algebraic
equations where only surface location and enthalpy are differential. Integrating this DAE
implicitly using general linear methods, and solving the resulting algebraic system with
Jacobian-free Newton-Krylov, necessitates an effective preconditioner for
the coupled indefinite
system. Since this system contains physics operating on multiple scales
with very different
spectral properties, we apply relaxation field splitting to separate the
loosely coupled parts,
and a factorization split for the heterogeneous Stokes problem, with
preconditioners for the Schur
complement derived from approximate commutators.


\end{document}
