\documentclass{report}
\usepackage{amsmath,amssymb}
\setlength{\parindent}{0mm}
\setlength{\parskip}{1em}
\begin{document}
\begin{center}
\rule{6in}{1pt} \
{\large Genetha Gray \\
{\bf Determining the Impact of Computational Resolution Using a Mixed-Integer Hybrid Optimization Technique }}

Sandia National Labs \\ P O Box 969 \\ MS 9159 \\ Livermore \\ CA 94551-0969
\\
{\tt gagray@sandia.gov}\\
Katie Fowler\\
Matthew Farthing\\
Joshua Griffin\end{center}

Tackling water resources management problems routinely requires the
pairing of subsurface simulators and optimization algorithm. Inherent
challenges lie in choosing the appropriate realization of the subsurface,
formulating the objective function and constraints, and applying a
suitable optimization algorithm. The objective function and constraints
rely on output from the simulator, and the simulator often requires the
numerical solution to a system of nonlinear partial differential
equations thus derivative-free methods have emerged as the optimization
algorithms of choice.

Various assumptions can be made to simplify either the objective function
or the physical system including the use of coarsely discretized grids to
improve the computational efficiency the underlying simulation tool.
Previous studies have shown that solutions obtained using a course grid
simulation become suboptimal as the grid resolution is improved.
Moreover, optimization algorithms that previously succeeded in
identifying those are no longer suitable as the physical domain
approaches reality either due to the computational burden or the impact
of increased constraint violations.

In this work, we describe a derivative-free hybrid approach, which allows
us to combine the beneficial elements of multiple methods, and more
efficiently search the design space. We will use this method to
demonstrate a study of a groundwater supply problem on a series of
increasingly fine grids. Specifically, we will show that solutions on the
coarse grid are infeasible on a finer grid and subsequently show how the
hybrid approach can be applied to find a new solution on the fine grid.


\end{document}
