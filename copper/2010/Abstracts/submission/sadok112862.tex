\documentclass{report}
\usepackage{amsmath,amssymb}
\setlength{\parindent}{0mm}
\setlength{\parskip}{1em}
\begin{document}
\begin{center}
\rule{6in}{1pt} \
{\large HASSANE SADOK \\
{\bf A new look at CMRH and its relation to GMRES}}

Laboratoire de Mathematiques Pures et Appliquees \\ Universite du Littoral \\ Centre Universitaire de la Mi-Voix \\ B P 699 \\ 62228 Calais Cedex \\ France
\\
{\tt sadok@lmpa.univ-littoral.fr}\\
DANIEL SZYLD\end{center}

CMRH (Changing Minimal Residual method based on the Hessenberg process)
is a Krylov subspace method which uses the Hessenberg process to produce
a basis of a Krylov method, and minimizes a quasi-residual. The CMRH
method shares many of the computational properties of the well-known
GMRES method.

Hence this method produces convergence curves which are very close
to those of GMRES, but using fewer operations and storage.
In this paper we present new analysis which explains
why CMRH has this good convergence behavior.


\end{document}
