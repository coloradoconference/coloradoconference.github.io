\documentclass{report}
\usepackage{amsmath,amssymb}
\setlength{\parindent}{0mm}
\setlength{\parskip}{1em}
\begin{document}
\begin{center}
\rule{6in}{1pt} \
{\large Christoph Schwarzbach \\
{\bf The Discontinuous Galerkin method for highly inhomogeneous media}}

University of British Columbia \\ Department of Earth and Ocean Sciences \\ 6339 Stores Road \\ Vancouver \\ British Columbia \\ Canada \\ V6T 1Z4
\\
{\tt cschwarz@eos.ubc.ca}\\
Eldad Haber\end{center}

The simulation of geophysical measurements involves a physical
description of the earth's interior by spatially varying constitutive
parameters. In particular, these parameters may be discontinuous and have
jumps by orders of magnitude. We address the problem of solving Maxwell's
equations for that case by a discontinuous approach also for the fields.
The Discontinuous Galerkin (DG) method allows us to enforce the interface
conditions explicitly by adding appropriate penalty or Lagrange
multiplier terms to the variational formulation. We restrict ourselves to
the case of time-harmonic fields and derive a symmetric discretization
for the first order Maxwell system. In spite of potential savings in
storage for a second order formulation we prefer to discretize the first
order form as it results in accuracy for both the electric and magnetic
fields; both needed for geophysical applications. The main challenge in
the DG method is the solution of the linear system. We propose and
experiment with a number of preconditioning techniques. In particular, we
use a preconditioner based on the potentials formulation to Maxwell's
equations. We show that using our approach we are able to obtain
significant accuracy even for very challenging problems.


\end{document}
