\documentclass{report}
\usepackage{amsmath,amssymb}
\setlength{\parindent}{0mm}
\setlength{\parskip}{1em}
\begin{document}
\begin{center}
\rule{6in}{1pt} \
{\large F. J. Gaspar \\
{\bf Local Fourier Analysis of Multigrid Methods for Mimetic Discretizations on Triangular Grids}}

Department of Applied Mathematics \\
%-mb: Mar�a de Luna s/n 50018 Zaragoza (Spain)
Mar\'{i}a de Luna s/n 50018 Zaragoza (Spain)
\\
{\tt fjgaspar@unizar.es}\\
J.L. Gracia\\
F.J. Lisbona\\
	Rodrigo C.\end{center}

The differential operators divergence, gradient and rotor are
often used to formulate mathematical physics problems. A natural
way to discretize the differential problem is to define, on a
given grid, the corresponding discrete operators. When the
discrete operators satisfy the main properties of the continuous
operators and also some compatibility relations between them, the
associated methods are often called mimetic methods. Staggered grid
finite differences used in fluid dynamics, the finite difference time
domain in electromagnetic and the
support--operator method of Samarskii, A. A. are early examples of them.

To approximate solutions of mathematical physics problems defined on
irregular domains, it is convenient to use rough grids to fit better the
spatial domain. In a recent paper [2], a specially clear approach, called
VAGO (Vector Analysis Grid Operators) method, has been proposed on
Delaunay triangulations and the dual Voronoi grids. For this
discretization method is not necessary to use a concrete coordinate
system which is specially interesting for computations on irregular
grids.

A special issue is the efficient solution of the corresponding
algebraic system of equations which result after the
discretization process. Although the algebraic multigrid is a
useful tool to solve problems on unstructured grids, we consider
in this talk an efficient and robust geometric multigrid method, in a
free--matrix version, to solve mimetic finite difference discretizations
on triangular grids.

To design these geometric multigrid methods, a Local Fourier
Analysis is proposed [1]. This local mode analysis is based on an
expression of the Fourier transform in new coordinate systems for space
variables and for frequencies. The previous tool permits to study
different components of the multigrid method in a very similar way to the
rectangular grids case. Different smoothers for scalar and vector
problems are studied depending on the shape of the triangles. Numerical
test calculations validate the theoretical predictions.

\

\noindent {References:}


\begin{itemize}
\item[{[1]}] F.J. Gaspar, J.L. Gracia, F.J. Lisbona. {\em Fourier
analysis for multigrid methods on triangular
grids}. Submitted.
\item[{[2]}] P.N. Vabishchevich. {\em Finite difference approximation of
mathematical physics problems on irregular grids}, Comp. Meth. Appl.
Math. {\bf 5} (2005) pp. 294--330.
\end{itemize}


\end{document}
