\documentclass{report}
\usepackage{amsmath,amssymb}
\setlength{\parindent}{0mm}
\setlength{\parskip}{1em}
\begin{document}
\begin{center}
\rule{6in}{1pt} \
{\large Merico E. Argentati \\
{\bf Majorization-based convergence rate bounds for subspace iterations and the block Lanczos method}}

Department of Mathematical Sciences \\ University of Colorado Denver \\ P O Box 173364 \\ Campus Box 170 \\ Denver \\ CO 80217-3364
\\
{\tt merico.argentati@cudenver.edu}\\
Andrew V.  Knyazev\end{center}

The Rayleigh-Ritz method finds the stationary values, called Ritz values,
of the Rayleigh quotient on a given trial subspace as approximations to
eigenvalues of a Hermitian operator A. If the trial subspace is
A-invariant, the Ritz values are some of the eigenvalues of A. Given two
subspaces X and Y of the same finite dimension, such that X is
A-invariant, the absolute changes in the Ritz values of A with respect to
X compared to the Ritz values with respect to Y represent the absolute
eigenvalue approximation error. A recent paper by M. Argentati et al.
bounds the error in terms of the principal angles between X and Y using
weak majorization. In this talk we derive a new majorization-type
convergence rate bound for subspace iterations and combine it with the
previous result to obtain a similar bound for the block Lanczos method.
These majorization results imply a very general set of inequalities that
involve all of the principal angles, not just the largest principal
angle.

We proceed in two main steps. First, we prove new majorization-type
convergence rate bounds for subspace iterations in terms of the principal
angles between subspaces. Second, we assume that the Rayleigh-Ritz method
is applied in subspace iterations, so we combine the subspace iterations
convergence rate bounds for angles with our Rayleigh-Ritz method error
bounds for eigenvalues . We consider polynomial-type subspace iterations
and show that the Chebyshev polynomials are optimal in this context.
Finally, we apply the convergence rate bound of the Chebyshev subspace
iterations to the block Lanczos method. Weak majorization appears to be
novel in this context.


\end{document}
