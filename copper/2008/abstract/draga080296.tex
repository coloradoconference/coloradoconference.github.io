\documentclass{report}
\usepackage{amsmath,amssymb}
\setlength{\parindent}{0mm}
\setlength{\parskip}{1em}
\begin{document}
\begin{center}
\rule{6in}{1pt} \
{\large Andrei Draganescu \\
{\bf A multilevel algorithm for bound-constrained inverse problems}}

Department of Mathematics and Statistics \\ University of Maryland \\ Baltimore County \\ 1000 Hilltop Circle \\ Baltimore \\ MD 21250
\\
{\tt draga@math.umbc.edu}\end{center}

\noindent In this work we present some recent results on multilevel
(ML) algorithms for a class of inverse problems (IPs) with explicit
non-negativity constraints imposed on the control. The IPs under
scrutiny are formulated as regularized least-squares minimization
problems
$$\mathrm{find}\ \ \min_{u\in X}\ \frac{1}{2}|\!| Ku - f |\!|^2 +
\frac{\beta}{2}|\!|u|\!|^2,\ \ \ u\ge 0\ ,\ \hspace{2cm}(\dagger)\ $$
where $K:X\rightarrow Y$ is a compact operator between two Hilbert
spaces $X$ and $Y$, $f\in Y$, and $\beta>0$ is a regularization
parameter. If the non-negativity constraints on the control $u$ were
absent from the IP $(\dagger)$, then the solution would be given by
the linear system
$$(\beta I+ K^* K)u = K^* f\ . \hspace{5.4cm}(\ddagger)$$ It is
shown in~\cite{dradup} that ($\ddagger$) can be solved efficiently by
using specially designed ML preconditioners. More precisely, it is
shown that the quality of the ML preconditioning increases with
resolution $h\downarrow 0$, which results in a number of preconditioned conjugate
gradient iterations that decreases with $h\downarrow 0$.

${ }$

\noindent For a large number of applications, the explicit presence of
bound-constraints in the IP-formulation can be critical. For example,
in the inverse contamination problem~\cite{sc05}, where the control
represents the initial concentration of an air pollutant,
non-negativity constraints not only render a physically meaningful
solution, but are essential for the correct recovery of spatially
localized solutions. The question addressed in the presented research
is whether the qualitative behavior established for the unconstrained
problem ($\ddagger$) can be reproduced in the case of the constrained
IP ($\dagger$). The problem ($\dagger$) is solved via the semi-smooth Newton
method described in~\cite{hint_ito_kun} as an outer iteration with an
ML-preconditioned CG algorithm being used for solving the linear systems
arising at each outer iteration.

\begin{thebibliography}{99}
\bibitem{dradup} Andrei Dr{\u a}g{\u a}nescu and Todd F.~Dupont.
\emph{Optimal order multilevel preconditioners for regularized ill-posed problems}.
To appear in Math.~Comp.
\bibitem{sc05} Volkan Ak{\c c}elik, George Biros, Andrei
Dr{\u{a}}g{\u{a}}nescu, Omar Ghattas, Judith~C. Hill, and Bart~G. van
Bloemen Waanders.
Dynamic data driven inversion for terascale simulations: real-time
identification of airborne contaminants.
In \emph{Proceedings of SC2005}, Seattle, WA, November 2005. IEEE/ACM.
\bibitem{hint_ito_kun} M.~Hinterm{\"u}ller, K.~Ito, and K.~Kunisch. The
primal-dual active set strategy as a semismooth {N}ewton method.
\emph{SIAM J. Optim.}, 13:865--888, 2002.
\end{thebibliography}


\end{document}
