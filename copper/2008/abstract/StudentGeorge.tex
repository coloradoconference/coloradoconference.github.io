\documentclass{report}
\usepackage{amsmath,amssymb}
\setlength{\parindent}{0mm}
\setlength{\parskip}{1em}
\begin{document}
\begin{center}
\rule{6in}{1pt} \
{\large Thomas George \\
{\bf An Experimental Evaluation of Iterative Solvers for Large SPD Systems
of Linear Equations.
}}

Department of Department of Computer Science \\ 
Texas A&M University  
\\
{\tt tgeorge@cs.tamu.edu}\\
Anshul Gupta \\
Vivek Sarin
\end{center}

Direct methods for solving sparse systems of linear equations are
fast and robust, but can consume an impractical amount of memory,
particularly for large three-dimensional problems. Preconditioned
iterative solvers have the potential to solve very large systems with
a fraction of the memory used by direct methods. The diversity of
preconditioners makes it difficult to analyze them in a unified
theoretical model. Hence, a systematic evaluation of existing
preconditioned iterative solvers is necessary to identify the relative
advantages of iterative methods and to guide future efforts.
We present the results of a comprehensive experimental 
study of the most popular preconditioner and iterative solver
combinations for symmetric positive-definite systems. A detailed
comparison of the preconditioners, the iterative solver packages, and a
state-of-the-art direct solver gives interesting insights into their
strengths and weaknesses. We believe that these results would be
useful to researchers developing preconditioners and iterative solvers
as well as practitioners looking for appropriate sparse solvers for
their applications.
\end{document}
