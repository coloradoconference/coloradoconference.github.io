\documentclass{report}
\usepackage{amsmath,amssymb}
\setlength{\parindent}{0mm}
\setlength{\parskip}{1em}
\begin{document}
\begin{center}
\rule{6in}{1pt} \
{\large Richard L. Muddle \\
{\bf Efficient Block Preconditioning for a $C^1$ Finite Element Discretisation of the Biharmonic Problem}}

Department of Mathematics \\ University of Manchester \\ Oxford Road \\ Manchester \\ M13 9PL \\ UK
\\
{\tt rlm@cs.man.ac.uk}\\
Matthias Heil\\
Milan D. Mihajlovic\end{center}

The approximation of the biharmonic operator by the finite element method
requires either the use of $C^1$ continuous elements, or the
reformulation of the original problem in mixed form to allow the
discretisation by Lagrangian elements. We discuss an effective block
preconditioner for the $C^1$ discretisation of the 2D biharmonic problem
using bicubic Hermite finite elements. In this formulation each node is
associated with four different types of degree of freedom.

The resulting linear system is symmetric positive definite with a
condition number that behaves as $O(h^4)$ where $h$ is the mesh spacing
parameter. Therefore, the efficient solution of this linear system depend
on the development of an effective preconditioner. Standard
preconditioning techniques, including the "black-box" application of AMG
to the whole coefficient matrix, do not produce an efficient solution.
Reordering the linear system to group together the different types of
degree of freedom leads to a natural blocking of the coefficient matrix.
We present a block preconditioner, based on a form of block Jacobi
preconditioning which is spectrally equivalent to the coefficient matrix.
Some analytical results and computations of the spectrum of a
preconditioned discrete operator reveal that its condition number remains
small and bounded under mesh refinement.

We present an efficient implementation of this block preconditioner based
on approximate inversion techniques (using matrix lumping and algebraic
multigrid) for each of the diagonal blocks. Case studies demonstrate the
robustness (with respect to changes in domain shape, mesh deformation and
types of boundary conditions) and efficiency of this preconditioner for a
variety of non-trivial test problems.


\end{document}
