\documentclass{report}
\usepackage{amsmath,amssymb}
\setlength{\parindent}{0mm}
\setlength{\parskip}{1em}
\begin{document}
\begin{center}
\rule{6in}{1pt} \
{\large Andrew Canning \\
{\bf New Eigensolvers for Large Scale Nanoscience Simulations }}

Lawrence Berkeley National Laboratory \\ One Cyclotron road \\ Berkeley CA94720 \\ USA
\\
{\tt acanning@lbl.gov}\\
%%	Christof V�mel, Osni Marques, and Lin-Wang Wang(CRD, LBNL)
	Christof V\"{o}mel, Osni Marques, and Lin-Wang Wang(CRD, LBNL)
	Stanimire Tomov and Jack Dongarra (ICL, University of Tennessee at Knoxville) 
	\end{center}

We present results for state-of-the-art iterative eigensolvers based on
conjugate gradients and variants of Davidson in the context of
semi-empirical plane wave electronic structure calculations. These new
methods give significant speedup over existing conjugate gradient methods
used in electronic structure calculations. The new methods are
demonstrated for CdSe quantum dots as well as quantum wires (single
electron devices) constructed from layers of InP and InAs. These systems
are studied in the context of a semi-empirical potential where we
typically solve for a few states around the gap allowing us to study
large scale nanosystems. The parallelization of this approach is
discussed as well as scaling results to large processor counts.


\end{document}
