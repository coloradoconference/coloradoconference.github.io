\documentclass{report}
\usepackage{amsmath,amssymb}
\setlength{\parindent}{0mm}
\setlength{\parskip}{1em}
\begin{document}
\begin{center}
\rule{6in}{1pt} \
{\large Andrew Knyazev \\
{\bf Eigensolvers for analysis of microarray gene expression data}}

Department of Mathematical Sciences \\ University of Colorado Denver \\ P O Box 173364 \\ Campus Box 170 \\ Denver \\ CO 80217-3364
\\
{\tt andrew.knyazev@cudenver.edu}\end{center}

Microarray data analysis (MDA) has become an important tool in molecular
biology. Modern microarray data provide vast amounts of useful biological
information, but their analysis is computationally challenging. Molecular
biologists need fast, reliable, and advanced MDA software, e.g., to
locate clusters
of genes responsible for specific biological processes. Novel
mathematical algorithms for MDA, including the recently available
GeneChip tiling arrays, are necessary to widen the bottlenecks of
existing software, which primarily are slow performance and low accuracy.
Our core expertise is in development of iterative methods and software
for numerical solution of eigenvalue problems [1-4]. In this work, we
develop eigenvalue solvers tailored for MDA, specifically for spectral
gene clustering.

REFERENCES

[1] A. V. Knyazev. Preconditioned eigensolvers - an oxymoron? ETNA, 7,
104--123, 1998.
http://etna.mcs.kent.edu/vol.7.1998/pp104-123.dir/pp104-123.pdf

[2] A. V. Knyazev. Toward the Optimal Preconditioned Eigensolver: Locally
Optimal Block Preconditioned Conjugate Gradient Method. SISC, 23(2),
517--541, 2001. http://dx.doi.org/10.1137/S1064827500366124

[3] A. V. Knyazev, I. Lashuk, M. E. Argentati, and E. Ovchinnikov. Block
Locally Optimal Preconditioned Eigenvalue Xolvers (BLOPEX) in hypre and
PETSc. SISC 25(5), 2224--2239, 2007. http://dx.doi.org/10.1137/060661624

[4] F. Bottin, S. Leroux, A. Knyazev, and G. Zerah. “Large scale ab
initio calculations based on three levels of parallelization.”
Computational Material Science. In print, 2007.
http://dx.doi.org/10.1016/j.commatsci.2007.07.019


\end{document}
