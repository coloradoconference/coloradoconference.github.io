\documentclass{report}
\usepackage{amsmath,amssymb}
\setlength{\parindent}{0mm}
\setlength{\parskip}{1em}
\begin{document}
\begin{center}
\rule{6in}{1pt} \
{\large Lei-Hong Zhang \\
{\bf Fast Algorithms for the Generalized Foley-Sammon Discriminant Analysis.
}}

Department of Mathematics \\
Hong Kong Baptist University \\
Kowloon Tong, Kowloon \\
Hong Kong, P. R. China \\ 
{\tt lhzhang@math.hkbu.edu.hk}\\
Li-Zhi Liao \\ 
Michael K. Ng
\end{center}

%%-mb: Linear Discriminant Analysis (LDA) is one of the most popular approaches for feature extraction and dimension reduction to overcome the curse of the dimensionality of the high-dimensional data in many applications of data mining, machine learning, and bioinformatics. The undersampled problem, which arises frequently in many modern applications, involves small samples size n with high number of features N (N > n) and limits the application of the linear discriminant analysis. In this paper, we investigate the generalized Foley-Sammon transform (GFST, [7, 12]) and its regularization (RGFST) for undersampled problems. The optimal linear transformations of RGFST are characterized completely and an equivalent reduced RGFST is established, based on which a global and superlinear convergence algorithm is proposed. Practical implementations including computational complexity and storage of our method are discussed and experimental results on several real world data sets indicate the efficiency of the algorithm and the advantages of RGFST in classification.

Linear Discriminant Analysis (LDA) is one of the most popular
approaches for feature extraction and dimension reduction to
overcome the curse of the dimensionality of the high-dimensional
data in many applications of data mining, machine learning, and
bioinformatics. The undersampled problem, which arises frequently
in many modern applications, involves small samples size $n$ with
high number of features $N$ $(N> n)$ and limits the application of
the linear discriminant analysis. In this paper, we investigate
the generalized Foley-Sammon transform (GFST,
\cite{Foley&Sammon75,Guo&Li03}) and its regularization (RGFST) for
undersampled problems. The optimal linear transformations of RGFST
are characterized completely and an equivalent reduced RGFST is
established, based on which a global and superlinear convergence
algorithm is proposed. Practical implementations including
computational complexity and storage of our method are discussed
and experimental results on several real world data sets indicate
the efficiency of the algorithm and the advantages of RGFST in
classification.

\begin{thebibliography}{1}%% {\hspace{0.2in}}

\bibitem{Foley&Sammon75}
D. Foley and J. Sammon, ``An optimal set of discriminant vectors,"
\emph{IEEE Trans Computers,} vol. 24, pp. 281-289, 1975.

\bibitem{Guo&Li03}
Y. Guo, S. Li,  J. Yang, T. Shu and L. Wu, ``A Generalized
Foley-Sammon Transform (GFST) Based on Generalized Fisher
Discriminant Criterion and Its Application to Face Recognition,''
\emph{Pattern Recognition Letter,} vol. 24(1-3), pp. 147-158,
2003.

\end{thebibliography}
\end{document}
