\documentclass{report}
\usepackage{amsmath,amssymb}
\setlength{\parindent}{0mm}
\setlength{\parskip}{1em}
\begin{document}
\begin{center}
\rule{6in}{1pt} \
{\large Khosro, KS Shahbazi \\
{\bf A Multigrid Solver for High-Order Discontinuous Galerkin Discretizations of the Compressible Navier-Stokes Equations}}

Department of Mechanical Engineering \\ University of Wyoming \\ 1000 E University Ave \\ Laramie \\ WY 82071-3295 USA
\\
{\tt kshahbaz@uwyo.edu}\\
Dimitri Mavriplis\end{center}

We present a multigrid solver for the system arising from high-order
discontinuous Galerkin (DG) discretizations of the compressible
Navier-Stokes equations. We use the interior
penalty method (D. Arnold, SIAM J. Numer. Anal. 19 (1982), pp. 742-760)
for the discretization of the diffusion term and the local Lax-Fredriches
fluxes for the convective term. In the interior penalty method, the
penalty parameter is chosen based on an explicit expression derived in
(K. Shahbazi, J. Comput. Phys.
205 (2005), pp. 401-407). A Hierarchical basis consisting of high-order
polynomials is used for spatial discretizations.

The coarse grid approximations and smoothing schemes of our multigrid
solver are chosen as follows.
The coarse grid operators are direct discretizations of the governing
equations at lower approximation orders.
For smoothing schemes, we consider both block Jacobi and block
Gauss-Seidle iterations, where block corresponds to the matrix arising
from the restriction of DG discretizations to a single element. Since we
use hierarchical basis, the restriction operator from a high order to a
low order vectors is simply an identity matrix with zero columns.

We first verify the performance of the proposed multigrid solver for the
two-dimensional Poisson equation on a square domain for different
approximation orders and different mesh sizes.
We choose the coarsest grid to be the dicretization at approximation
order p=0. This allows fast solution of the
the coarsest grid system using the geometric multigrid. Mesh- and
order-independent convergence rates are achieved if the number of
smoothing iterations are chosen large enough.

We then examine the behavior of the scheme in simulating viscous flow
over NACA0012 airfoil at zero degree incidence, and with a freestream
Mach number of 0.5, and a Reynolds number of 5000.
Unlike the Poisson case, adopting p=0 for the coarsest grid,
does not yield a robust multigrid solver. Except for the p=1
approximation where fast convergence is obtained, consistent improvements
in convergence rates are not observed at higher
orders. We attribute this to the inconsistent discretization of the
diffusion operator at p=0. Thus, instead of the p=0, we propose to use
the p=1 as the coarsest level. The p=1 (coarsest grid) system itself is
efficiently solved using the proposed multigrid scheme.
Numerical results verifies the robustness and fast convergence of this scheme.


\end{document}
