\documentclass{report}
\usepackage{amsmath,amssymb}
\setlength{\parindent}{0mm}
\setlength{\parskip}{1em}
\begin{document}
\begin{center}
\rule{6in}{1pt} \
{\large Serkan Gugercin \\
{\bf Perturbation theory for inexact Krylov-based model reduction}}

Department of Mathematics \\ Virginia Tech \\ 460 McBryde Hall \\ Blacksburg \\ VA \\ 24061-0123
\\
{\tt gugercin@math.vt.edu}\\
Chris Beattie\end{center}


In this talk, we analyze the use of inexact solves in a Krylov-based
model reduction setting and present the resulting structured perturbation
effects on the underlying model reduction problem. We first show that for
a selection of interpolation points that satisfy first-order necessary
$H_{2}$-optimality conditions, a primitive basis remains remarkably
well-conditioned and errors due to inexact solves do not tend to degrade
the reduced order models. Conversely, for poorly selected interpolation
points, errors can be greatly magnified through the model reduction
process.


We prove that when inexact solves are performed within a Petrov-Galerkin
framework, the resulting reduced order models are backward stable with
respect to the approximating transfer function. As a consequence,
Krylov-based model reduction with well chosen interpolation points is
robust with respect to the structured perturbations due to inexact
solves. General bounds on the $H_{2}$ system error associated with an
inexact reduced order model are introduced that provide a new tool to
understand the structured backward error and stopping criteria. Several
numerical examples will be presented to support the theoretical
discussion.


\end{document}
