\documentclass{report}
\usepackage{amsmath,amssymb}
\setlength{\parindent}{0mm}
\setlength{\parskip}{1em}
\begin{document}
\begin{center}
\rule{6in}{1pt} \
{\large Tim Wildey \\
{\bf A Low Fidelity Approach for Efficient Uncertainty Quantification of High Fidelity Models}}

1 University Station C0200 \\ Austin \\ Texas 78712
\\
{\tt twildey@ices.utexas.edu}\\
Hector Klie\\
Mary F. Wheeler\end{center}

A fundamental difficulty in understanding and predicting large-scale
fluid movements in porous media is that these movements depend upon
phenomena occuring on small scales in space and/or time. The differences
in scale can be staggering. Aquifers and reservoirs extend for thousands
of meters, while their transport properties can vary across centimeters,
reflecting the depositional and diagenetic processes that formed the
rocks. In turn, transport properties depend on the distribution,
correlation and connectivity of micron sized geometric features such as
pore throats, and on molecular chemical reactions. Seepage and even
pumped velocities can be extremely small compared to the rates of phase
changes and chemical reactions. Thus, in subsurface modeling, it is
physically impossible to incorporate the exact values of certain
parameters at every point in the reservoir into the numerical model. To
account for this uncertainty, these parameters are often treated as
random variables rather than deterministic quantities. As a result, the
statistics of the flow variables over a large number of simulations
becomes of primary interest. We explore the use of low fidelity models to
quantify uncertainty in more complicated, high fidelity models and
present an efficient uncertainty quantification approach for nonlinear
compressible multiphase flow in porous media. We illustrate our approach
with numerical examples.


\end{document}
