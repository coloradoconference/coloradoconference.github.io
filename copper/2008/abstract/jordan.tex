\documentclass{report}
\usepackage{amsmath,amssymb}
\setlength{\parindent}{0mm}
\setlength{\parskip}{1em}
\begin{document}
\begin{center}
\rule{6in}{1pt} \
{\large Kirk E. Jordan \\
{\bf Parallelism in the large and the small:  Parallel Computing is becoming pervasive – are we ready?}}

IBM Systems and Technology Group \\ 
http://www-3.ibm.com/software/info/university/people/kjordan.html 
\\
{\tt kjordan@us.ibm.com}\end{center}

We can no longer rely on computational speed up through increases in clock speed.  In order to continue the speed ups users have come to expect from computer companies, the chip manufactures are turning to multi/many-core chips.  Are we ready for such systems composed of multi/many-cores on a chip?   Much of what we have learned on large distributed systems in principle will need to be applied to these multi/many-core efforts. I will describe briefly why computer manufactures are turning to these multi/many core chips.  I will give some details of one of IBM’s offerings in the multi/many core chips, the PowerXCell8i, and describe the system that this chip is part of as well as IBM’s new Blue Gene/P that uses a multi-core chip.  I will point out that lessons learn on large scale parallel systems will help us with developing algorithms for parallelism at the chip level.   While the computational power may be accessible, most scientists are ill prepared to fully take advantage of this.  New approaches to computational problems that will take full advantage of multi/many core computer systems need to be investigated.  Iterative methods that are often part of the computational intensive simulations being addressed by these are definitely among the important techniques.  I will show some performance that will illustrate the power of systems composed with multi/many-core chips/systems.  While progress is being made, there remain many challenges for the computational science community to apply multi/many-core systems to “Big” science problems with impact on society that until now or in current implementations have fallen short of the mark.  

\end{document}
