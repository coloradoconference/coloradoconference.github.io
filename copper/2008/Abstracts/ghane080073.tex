\documentclass{report}
\usepackage{amsmath,amssymb}
\setlength{\parindent}{0mm}
\setlength{\parskip}{1em}
\begin{document}
\begin{center}
\rule{6in}{1pt} \
{\large Roger G Ghanem \\
{\bf Parallellization and preconditioning of iterative solvers for linear systems arising in the stochastic finite element method}}

Civil and Environmental Engineering \\ University of Southern California \\ 3620 S Vermont Street KAP 254C \\ Los Angeles CA 90089-2531
\\
{\tt ghanem@usc.edu}\\
Ramakrishna Tipireddy\\
Maarten Arnst\end{center}

This communication investigates the use of preconditioning and
parallellization methods to improve the computational efficiency of
iterative solvers for linear systems arising in the Stochastic Finite
Element Method~(SFEM).

Linear matrix systems obtained in the SFEM are typically of much larger
dimension than those obtained in the classical deterministic FEM. Indeed,
SFEM systems are typically~\cite{ghanem1991} of form:
\begin{equation}
\sum_{i=0}^{N}\sum_{j=0}^{L}c_{ijk}\boldsymbol{K}_{j}\boldsymbol{u}_{i}=\boldsymbol{f}_{k}\;,\quad\text{for~$k
= 0,\ldots,N$}\quad,\label{eq:sfemsystem}
\end{equation}
where~$N$ is the number of terms retained in the polynomial chaos
expansion of the random response, $L$ is the number of terms in the
Karhunen-Loeve expansion of the random material properties, and the
dimension of all submatrices~$\boldsymbol{K}_{i}$ is equal to the number
of spatial degrees of freedom. Since the coefficients~$c_{ijk}$ vanish
for certain combinations of the indices~$i$, $j$ and~$k$, the system
matrix derived from the above formulation has a particular block-sparsity
structure, see e.g.~\cite{pellisetti2000}. Other important properties
include the dominance of~$\boldsymbol{K}_0$ over the
other~$\boldsymbol{K}_i$'s, and the fact that all
matrices~$\boldsymbol{K}_i$ have the same sparsity pattern and are
usually symmetric.

The objective of this communication is to compare several preconditioning
and parallellization methods which capitalize on the aforementioned
properties to improve the computational efficiency of iterative solvers
for SFEM systems of form~(\ref{eq:sfemsystem}). Based on the dominance
of~$\boldsymbol{K}_0$ over the other~$\boldsymbol{K}_i$'s, a first
preconditioning method consists in reformulating the problem as a system
of linear equations with multiple right-hand sides:
\begin{equation}
c_{k0k}\boldsymbol{K}_{0}\boldsymbol{u}_{k}=\boldsymbol{f}_{k}-\sum_{i\not=k}^{N}\sum_{j=0}^{L}c_{ijk}\boldsymbol{K}_{j}\boldsymbol{u}_{i}-\left(\sum_{j\not=0}^{L}c_{kjk}\boldsymbol{K}_{j}\right)\boldsymbol{u}_{k}\;,\quad\text{for~$k
= 0,\ldots,N$}.
\end{equation}
A second preconditioning method consists in keeping only the diagonal
block matrices on the left-hand side and moving the remaining blocks to
the right-hand side:
\begin{equation}
\sum_{j=0}^{L}c_{kjk}\boldsymbol{K}_{j}\boldsymbol{u}_{k}=\boldsymbol{f}_{k}-\sum_{i\not=k}^{N}\sum_{j=0}^{L}c_{ijk}\boldsymbol{K}_{j}\boldsymbol{u}_{i}\;,\quad\text{for~$k
= 0,\ldots,N$}.
\end{equation}
This method is expected to need fewer iterations to converge, but entails
a higher computational effort at each iteration since the diagonal blocks
are not identical.

The SANDIA developed Trilinos library~\cite{heroux2007} is used to
implement the framework. The package Epetra for mat-vec operations, and
the AztecOO, IFPACK and Belos preconditioner packages are used in
particular. The Belos package provides a solver manager for solving
linear systems simultaneously on multiple right-hand sides. All packages
use the Message Passing Interface~(MPI) to allow for execution on
parallel platforms. At the conference, the methods discussed above will
be presented and then compared based on their application to a case
history in stochastic structural mechanics.

\begin{thebibliography}{99}
\bibitem[1]{ghanem1991}
{R}. {G}hanem and {P}. {S}panos.
\newblock {\em {S}tochastic {F}inite {El}ements:{A} {S}pectral {A}pproach}.
\newblock Springer, 1991.

\bibitem[2]{heroux2007}
M.~Heroux and J.M. Willenbring.
\newblock Trilinos users guide.
\newblock Technical report, Sandia National Laboratories, 2007.
\newblock SAND2003-2952.

\bibitem[3]{pellisetti2000}
M.F. Pellissetti and R.G. Ghanem.
\newblock Iterative solution of systems of linear equations arising in the
context of stochastic finite elements.
\newblock {\em Advances in Engineering Software}, 31:\penalty0 607--616, 2000.
\end{thebibliography}


\end{document}
