\documentclass{report}
\usepackage{amsmath,amssymb}
\setlength{\parindent}{0mm}
\setlength{\parskip}{1em}
\begin{document}
\begin{center}
\rule{6in}{1pt} \
{\large Oliver G. Ernst \\
{\bf Iterative Solvers for Stochastic Galerkin Systems }}

Institut f�r Numerische Mathematik und Optimierung \\ TU Bergakademie Freiberg \\ 09596 Freiberg \\ Germany
\\
{\tt ernst@math.tu-freiberg.de}\\
Catherine Powell\\
David Silvester\\
	Ullmann, Elisabeth\end{center}

In recent years the technique of uncertainty quantification by solving
partial differential equations with random data has received increasing
attention.
A particularly popular solution approach for such problems is the
\emph{Stochastic Galerkin Method}, also known as the \emph{Stochastic
Finite Element Method}.
Stochastic Galerkin discretization combines a standard finite element
discretization of the deterministic variant of the underlying problem
with a discretization of the dependence of the solution on the uncertain
variables, in the form of a tensor product space. It has recently been
shown that, under certain weak stochastic regularity assumptions on the
uncertain variables, that stochastic Galerkin discretizations converge
faster than the more well-known Monte Carlo simulations.

Aside from many modeling and discretization issues, the task of solving
the extremely large linear systems of equations which arise in stochastic
Galerkin discretizations poses a substantial challenge.
In this talk we discuss the influence of various stochastic Galerkin
formulations on the resulting linear system of equations and present
recent theoretical and numerical results [1] based on Krylov subspace
solvers using a preconditioner based on the mean problem -- that which
results when the random coefficients are replaced by their mean values.

References

[1] Oliver G. Ernst, Catherine E. Powell, David Silvester, and Elisabeth Ullmann. Effi-
cient solvers for a linear stochastic Galerkin mixed formulation of diffusion problems
with random data. EPrint 2007.126, Manchester Institute for Mathematical Sciences,
University of Manchester, Manchester, UK, 2007.


\end{document}
