\documentclass{report}
\usepackage{amsmath,amssymb}
\setlength{\parindent}{0mm}
\setlength{\parskip}{1em}
\begin{document}
\begin{center}
\rule{6in}{1pt} \
{\large Jacob Schroder \\
{\bf Defining a General Strength-of-Connection Concept in AMG}}

Jacob Schroder \\ Siebel Center for Computer Science \\ 4328 Siebel Center \\ 201 N Goodwin Ave \\ Urbana IL 61801 USA
\\
{\tt jacob.bb.schroder@gmail.com}\\
Raymond Tuminaro\\
Luke Olson\end{center}

Algebraic Multigrid (AMG) is often optimal, in that \emph{O(n)}
complexity is realized by efficiently reducing significant residual
components on carefully constructed coarse spaces. A crucial part of
constructing coarse spaces is determining how algebraically smooth error
at one point depends on algebraically smooth error at a neighboring
point. If dependence is strong, the two points are considered strongly
connected. Accurate strength of connection information allows AMG to
correctly pick coarse spaces and to construct the corresponding
interpolation operators.

An accurate strength-of-connection measure exists for M-matrices, but
strength-of-connection is not well-understood for more general classes of
matrices. The purpose of this talk is to present a new general
strength-of-connection measure grounded in the relaxation-based evolution
of an initial Dirac $\delta$-function. Motivated by the relationship
between weighted-Jacobi relaxation and the time marching of ordinary
differential equations (ODEs), an ODE perspective is presented for
understanding the new strength-of-connection measure. The ODE perspective
is also very enlightening when examining strengths and weaknesses of
classical strength-of-connection measures and recent attempts to define a
general strength-of-connection concept, such as compatible relaxation and
an energy based approach.

We also present encouraging numerical results for the new
strength-of-connection measure used in conjunction with smoothed
aggregation. The modified smoothed aggregation solver is used to
accelerate Krylov-subspace methods applied to diffusion,
convection-diffusion and linearized elasticity problems. Last, the
effectiveness of the new strength-of-connection measure when applied to
higher-order discretizations is explored.


\end{document}
