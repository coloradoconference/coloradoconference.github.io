\documentclass{report}
\usepackage{amsmath,amssymb}
\setlength{\parindent}{0mm}
\setlength{\parskip}{1em}
\begin{document}
\begin{center}
\rule{6in}{1pt} \
{\large Li Wang \\
{\bf Goal-oriented \textit{hp}-adaptive Discontinuous Galerkin Methods for the Compressible Euler Equations on Unstructured Meshes}}

Department of Mechanical Engineering \\ University of Wyoming \\ Laramie \\ Wyoming 82071-3295
\\
{\tt wangli@uwyo.edu}\\
Dimitri J. Mavriplis\end{center}

High-order accurate Discontinuous Galerkin (DG) methods
\cite{Cockburn::98_1,Cockburn::01_1,Nastase::05_1,Wang::07_1}
have become a popular approach for solving hyperbolic conservation
systems over the last decade. The use of adaptive solution strategies via
the finite element method has also become a more accepted technique
to guarantee and improve numerical accuracy, particularly for cases
with complex geometries. The main idea of the adaptive strategies used in
this work is to start the computation with a relatively coarse mesh associated
with a uniformly low order of discretization on the computation domain.
Thereafter, a local or cellwise error indicator arising from an objective
functional of interest with engineering properties, such as Lift or Drag,
is estimated using an adjoint-based sensitivity analysis technique.
This approach results in a spatial error distribution, which is then used
to drive local mesh subdivision (\textit{h}-refinement) or
local modification of discretization orders (\textit{p}-enrichment),
in order to improve the accuracy of the objective function and the approximation.
One of the main benefits of the given method from a finite
element point of view is that the number of unknowns and thus the computational expense
are optimized by avoiding excessive resolution
in areas of little influence on the quantity of interest.

The \textit{a posteriori} error estimates employed in this work to predict the error
distribution involve the inner product of the finite element residuals with
the adjoint solution variables.
In order to approximate the adjoint solution,
we make use of the discrete adjoint strategy \cite{Mavriplis::06_1}
where the governing equations are first
discretized by the Discontinuous Galerkin method and then linearized,
thus reproducing the
exact sensitivity derivatives of the original discretization of the governing equations.
This approach requires one adjoint solution at each adaptive step, which
is equivalent to the cost
of one flow solution, since the coefficient matrix of the linear system of equations
constructed for the adjoint problem corresponds to the transpose
of the full-Jacobian matrix of the flow equations.
The solution of this system is relatively
straight-forward since many of the entries in the full Jacobian matrix
are already computed for the implicit element Jacobi scheme used to solve
the flow equations.
Moreover,
a linearized element Gauss-Seidel smoother and an \textit{hp}-multigrid
approach \cite{Nastase::05_1,Wang::07_1} are exclusively
used to accelerate the convergence of both the flow and adjoint solvers.

The adaptive mesh approach used in this paper consists of a purely \textit{h}-refinement
technique, locally subdividing one triangle into four self-similar
children triangles, while keeping
the discretization order fixed, as well as a purely \textit{p}-enrichment
technique, locally increasing
the order of interpolation polynomials while keeping the underlying triangulation fixed,
and a combined scheme known as \textit{hp}-adaptive refinement.
However, the principal obstacle towards effective use of \textit{hp}-adaptation
lies in making a decision between \textit{h}- and \textit{p}-refinement.
In this work, the choice is made by utilizing a smoothness
indicator to isolate regions with smooth solution behavior where local
\textit{p}-enrichment
performs more effectively, from areas with singularities or discontinuities where local
\textit{h}-refinement is more suitable.

In this paper, we show the convergence properties for both the flow
(primal) and the discrete adjoint (dual)
problems using \textit{hp}-multigrid solver driven by the element Gauss-Seidel smoother.
Then, the overall efficiency and accuracy of the purely
\textit{h}-refinement and the purely
\textit{p}-enrichment approaches are compared with those of the scheme
with uniform mesh or order refinement, respectively.
Next, the performance of the \textit{hp}-adaptive approach is compared with the purely
\textit{h}- and \textit{p}-refinement. Finally, a case with strong shocks
or discontinuities is
employed to demonstrate the shock-capturing properties of both \textit{h}- and
combined \textit{hp}-adaptive strategies. Future work will focus on
extending this approach
to unsteady problems.


\begin{thebibliography}{}

\bibitem[1]{Cockburn::98_1}

{B. Cockburn and C.-W. Shu},
{The Local Discontinuous Galerkin Method for Time-Dependent
Convection-diffusion Systems},
{SIAM J. Numer. Appl. Mech. Engrg.},
{1998},
{35},
{2440--2463}

\bibitem[2]{Barter::07_1}

{Garrett E. Barter and David L. Darmofal},
{Shock Capturing with Higher-Order, PDE-Based Arti&#64257;cial Viscosity},
{June},
{2007},
{{AIAA} Paper 2007-3823}

\bibitem[3]{Nastase::05_1}

{C. R. Nastase and D. J. Mavriplis},
{High-Order Discontinuous Galerkin Methods using a Spectral Multigrid Approach},
{Jan.},
{2005},
{{AIAA} Paper 2005-1268}

\bibitem[4]{Wang::07_1}

{Li Wang and Dimitri J. Mavriplis},
{Implicit Solution of the Unsteady {E}uler Equations for High-order
Accurate Discontinuous {G}alerkin Discretizations},
{J. Comput. Phys.},
{2007},
{225},
{1994--2015}

\bibitem[5]{Mavriplis::06_1}

{Dimitri J. Mavriplis},
{A Discrete Adjoint-Based Approach for Optimization Problems on
Three-Dimensional Unstructured Meshes},
{{J}an},
{2006},
{{AIAA} Paper 2006-0050}

\end{thebibliography}


\end{document}
