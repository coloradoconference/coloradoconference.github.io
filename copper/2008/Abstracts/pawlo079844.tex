\documentclass{report}
\usepackage{amsmath,amssymb}
\setlength{\parindent}{0mm}
\setlength{\parskip}{1em}
\begin{document}
\begin{center}
\rule{6in}{1pt} \
{\large Roger P. Pawlowski \\
{\bf Fully-Implicit / Direct-to-steady-state Solution of FE Formulations for Resistive Magneto-Hydrodynamic Systems*}}

Sandia National Labs \\ P O Box 5800 \\ MS-1318 \\ Albuquerque \\ NM 87185-1318
\\
{\tt rppawlo@sandia.gov}\\
John N. Shadid\\
Jeffery W. Banks\end{center}

Ionized fluids with strong electromagnetic effects occur frequently in
nature and are critical for many important technological applications.
Examples include stellar interiors, gaseous nebula, the earth's
magnetoshpere, and Tokamak and Z-pinch physics. These systems are
described by a set of partial differential equations that conserve
momentum, mass, charge, and energy along with magnetic flux for the
electric and magnetic fields (Maxwell's equations). The resulting
magnetohydrodynamics (MHD) equations are strongly coupled, highly
nonlinear, and span a large range of time and length scales, making the
scalable, robust, and accurate solution of such systems extremely
challenging.

In this presentation, we will briefly discuss the development of multiple
MHD formulations based on unstructured stabilized finite element methods.
The formulations are designed for weak enforcement of the solenoidal
constraint, $\nabla \cdot {\mathbf B}=0$. The formulations include a 2D
vector potential formulation, a 3D ${\mathbf B}$-field formulation using
projection and a 3D ${\mathbf B}$-field formulation using variational
multi-scale stabilization. The resulting set of nonlinear equations are
solved using a fully-coupled Newton-Krylov solver with nonlinear
globalization techniques. Linear systems are solved using a multi-level
preconditioned GMRES iterative technique. We will present numerical
performance, accuracy, and initial scalability studies of the
formulations. Additionally we will present an application of our solvers
to perform a stability and bifurcation analysis of the hydromagnetic
Rayleigh-Bernard problem.

*This work was partially funded by the DOE Office of Science AMR Program,
and was carried out at Sandia National Laboratories operated for the U.S.
Department of Energy under contract no. DE-ACO4-94AL85000


\end{document}
