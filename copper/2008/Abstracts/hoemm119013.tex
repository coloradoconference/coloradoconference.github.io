\documentclass{report}
\usepackage{amsmath,amssymb}
\setlength{\parindent}{0mm}
\setlength{\parskip}{1em}
\begin{document}
\begin{center}
\rule{6in}{1pt} \
{\large Mark Hoemmen \\
{\bf Communication-Avoiding Krylov Subspace Methods}}

567 Soda Hall \\ UC Berkeley \\ Berkeley \\ CA 94720
\\
{\tt mhoemmen@cs.berkeley.edu}\\
James Demmel\\
Marghoob Mohiyuddin\end{center}

The exponential growth of communication costs relative to computation on
modern computers motivates revisiting a previously dismissed set of
algorithms: $s$-step Krylov subspace methods. One iteration of an
$s$-step method has almost the same communication cost as one iteration
of its related standard Krylov method, but accomplishes the same work as
$s$ of these iterations. We address concerns which hindered the earlier
acceptance of these algorithms: performance, numerical stability, and
preconditioning.


\end{document}
