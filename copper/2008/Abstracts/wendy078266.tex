\documentclass{report}
\usepackage{amsmath,amssymb}
\setlength{\parindent}{0mm}
\setlength{\parskip}{1em}
\begin{document}
\begin{center}
\rule{6in}{1pt} \
{\large P. Wendykier \\
{\bf Large-Scale Iterative Image Deblurring in Java}}

Emory University \\ Math and Science Center \\ 400 Dowman Drive \\ W401 \\ Atlanta \\ GA 30322
\\
{\tt piotr.wendykier@emory.edu}\\
J. Nagy\end{center}

Image deblurring, or deconvolution, is the process of using computational
methods to reconstruct an image that has been degraded by blurring
and noise. I will describe \emph{Parallel Iterative Deconvolution}
Java software for image deblurring. It is a fully multithreaded implementation
of three iterative methods: Modified Residual Norm Steepest Descent
(MRNSD), Conjugate Gradient for Least Squares (CGLS) and Hybrid Bidiagonalization
Regularization (HyBR). Two key components of the software are \emph{JTransforms}
- the first, open source, multithreaded FFT library written in Java,
and \emph{Parallel Colt} - a multithreaded version of a Java library
for high performance scientific computing. Benchmarks show that Parallel
Iterative Deconvolution is highly scalable and efficient on SMP machines.
Image deblurring examples, including performance comparisons with
other existing software, will also be given.


\end{document}
