\documentclass{report}
\usepackage{amsmath,amssymb}
\setlength{\parindent}{0mm}
\setlength{\parskip}{1em}
\begin{document}
\begin{center}
\rule{6in}{1pt} \
{\large Misha Kilmer \\
{\bf Edge Preserving Projection-based Regularization}}

Department of Mathematics \\ Tufts University \\ 503 Boston Ave \\ Medford \\ MA 02155
\\
{\tt misha.kilmer@tufts.edu}\\
Per Christian Hansen\end{center}

We present a projection-based regularization strategy and algorithm for
retaining edges in a regularized solution. Our algorithm is suitable for
large-scale discrete ill-posed problems arising from the discretization
of Fredholm integral equations of the first kind. Such problems arise in
many technical and scientific applications such as astronomical or
medical imaging, geoscience, and non-destructive testing. Discretizations
of these problems often lead to large structured or sparse systems of
linear equations with highly ill-conditioned coefficient matrices. In
this talk, we focus on image deblurring applications.

Many of the edge-preserving regularization algorithms that have been
developed in the context of image de-noising and deblurring to date
involve ideas and techniques from partial differential equations and/or
level sets. The difficulty is that efficient implementation of these
algorithms is a complicated, computationally intensive task due to the
nonlinear computational problems involved.

Our strategy avoids some of these pitfalls by making use of orthogonal
decompositions/transforms, for example the DCT, in which components in
the so-called noise and signal subspaces can be generated quickly. The
edge-preserving algorithm is implemented by use of iteratively reweighted
least squares (O�Leary, 1990). The efficiency of the approach relies in
part on the choice of preconditioner for each least squares system, as
well as on some more subtle issues of formulation and tolerance levels.
Some preconditioner options for specific choices of orthogonal transform
and choice of regularization operator will be discussed. Numerical
results based on two-dimensional image deblurring problems show the
promise of our approach.


\end{document}
