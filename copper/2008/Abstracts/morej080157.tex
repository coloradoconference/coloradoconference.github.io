\documentclass{report}
\usepackage{amsmath,amssymb}
\setlength{\parindent}{0mm}
\setlength{\parskip}{1em}
\begin{document}
\begin{center}
\rule{6in}{1pt} \
{\large Jorge Mor\'e \\
{\bf Derivative-Free Optimization Solvers: A Shootout}}

Mathematics and Computer Science Division \\ Argonne National Laboratory \\ 9700 S Cass Ave \\ Argonne \\ Illinois 60439-4844
\\
{\tt more@mcs.anl.gov}\\
Stefan Wild\end{center}


\newcommand{\Ref}[1]{\mbox{\rm{(\ref{#1})}}}
\newcommand{\R}{\mbox{${\mathbb R}$}}
\newcommand{\NA}{\textsf{\small NEWUOA}}
\newcommand{\NM}{\textsf{\small NMSMAX}} \newcommand{\Apps}{\textsf{\small APPSPACK}}

\begin {center}
{\Large\textbf{Derivative-Free Optimization Solvers: A Shootout}}

\bigskip

\textbf{Jorge J. Mor\'e\footnotemark [1] and Stefan M. Wild\footnotemark [2]}

\end {center}

\footnotetext [1] { Mathematics and Computer Science Division, Argonne
National Laboratory, Argonne, IL 60439. This work was supported by the
Mathematical, Information, and Computational Sciences Division subprogram
of the Office of Advanced Scientific Computing Research, Office of
Science, U.S. Department of Energy, under Contract DE-AC02-06CH11357. }


\footnotetext [2] { School of Operations Research and Information
Engineering, Cornell University, Ithaca, NY 14853. Research supported by
a DOE Computational Science Graduate Fellowship under grant number
DE-FG02-97ER25308}

Derivative-free optimization has experienced a renewed interest over the
past decade that has encouraged a new wave of theory and
algorithms. While this research includes computational experiments that
compare and explore the properties of these algorithms, most of these
experiments do not provide useful information for users of
derivative-free algorithms with computationally expensive problems. In
our experience, these users want solvers that deliver the most reduction
in function value within a given computational budget.

We explore benchmarking procedures for derivative-free optimization
algorithms when there is a limited computational budget. The focus of our
work is the unconstrained optimization problem
\[
\min \left \{ f(x) : x \in \R^n \right \} ,
\]
where $ f : \R^n \to \R $ may be noisy or non-differentiable and, in
particular, in the case where the evaluation
of $f$ is computationally expensive. These expensive optimization
problems arise in science and engineering because evaluation of the
function $ f$ often requires a complex deterministic simulation based on
solving the equations (for example, nonlinear eigenvalue problems,
ordinary or partial differential equations) that describe the underlying
physical phenomena. The
computational noise associated with these complex simulations means that
obtaining derivatives is difficult and unreliable. Moreover, these
simulations often rely on legacy or proprietary codes and hence must be
treated as black-box functions, necessitating a derivative-free
optimization algorithm.


Several comparisons have been made of derivative-free algorithms on noisy
optimization problems that arise in applications. In particular, we
mention
\cite{Shootout07,Gray2004,Torczon2003b,Oeuvray2007,RGRCAS2007}. The most
ambitious work in this direction \cite{Shootout07} is a
comparison of six derivative-free optimization algorithms on two
variations of a groundwater problem specified by a simulator. This work
compares algorithms by their trajectories (plot of the best function
value against the number of evaluations) until the solver satisfies a
convergence test based on the resolution of the simulator.

Benchmarking derivative-free algorithms on selected applications with
trajectory plots provides useful information to users with related
applications. In particular, users can find the solver that delivers the
largest reduction within a given computational budget. However, the
conclusions in these computational studies do not readily extend to other
applications.

We use data \cite{JJMSW07} and performance profiles \cite{EDD01} to
analyze the performance of six derivative-free solvers on benchmark sets
of smooth, noisy, and piecewise-smooth problems. We use performance
measures that evaluate the progress of the algorithm in terms of the
function value, and thus avoid unrealistic convergence tests. We consider
geometry-based solvers (\Apps, for example) and model-based solvers (\NA,
for example). Our results show that on these problems, model-based
solvers performs better than geometry-based solvers, even for noisy and
piecewise-smooth problems.

Our results also show that data and performance profiles provide
complementary information that measures the strengths and weaknesses of
optimization solvers as a function of the computational budget. Data
profiles are useful, in particular, to assess the short-term behavior of
the algorithms.

\begin{thebibliography}{1}

\bibitem{EDD01}
{\sc E.~D. Dolan and J.~J. Mor\'e}, {\em Benchmarking optimization
software with performance profiles}, Math. Programming, 91 (2002),
pp.~201--213.

\bibitem{Shootout07}
{\sc K.~Fowler, J.~Reese, C.~Kees, J.~J.E.~Dennis, C.~Kelley, C.~Miller,
C.~Audet, A.~Booker, G.~Couture, R.~Darwin, M.~Farthing, D.~Finkel,
J.~Gablonsky, G.~Gray, and T.~Kolda}, {\em A comparison of
derivative-free optimization methods for water supply and hydraulic
capture community problems}, preprint, 2007. \newblock Submitted to
Advances in Water Resources, June 2007.

\bibitem{Gray2004}
{\sc G.~A. Gray, T.~G. Kolda, K.~Sale, and M.~M. Young}, {\em Optimizing
an empirical scoring function for transmembrane protein structure
determination}, INFORMS J. on Computing, 16 (2004), pp.~406--418.

\bibitem{Torczon2003b}
{\sc P.~D. Hough, T.~G. Kolda, and V.~J. Torczon}, {\em Asynchronous
parallel pattern search for nonlinear optimization}, SIAM J. Sci. Comp.,
23 (2001),
pp.~134&#150;--156.

\bibitem{JJMSW07}
{\sc J.~J. Mor\'e and S.~Wild}, {\em Benchmarking derivative-free
optimization algorithms}, Preprint ANL/MCS-P1471-1207, Argonne National
Laboratory, Argonne, Illinois, 2007.

\bibitem{Oeuvray2007}
{\sc R.~Oeuvray and M.~Bierlaire}, {\em A new derivative-free algorithm
for the medical image registration problem}, International Journal of
Modelling and Simulation, 27 (2007), pp.~115--124.

\bibitem{RGRCAS2007}
{\sc R.~G. Regis and C.~A. Shoemaker}, {\em A stochastic radial basis
function method for the global optimization of expensive functions},
INFORMS Journal of Computing, 19 (2007), pp.~457--509.

\end{thebibliography}


\end{document}
