\documentclass{report}
\usepackage{amsmath,amssymb}
\setlength{\parindent}{0mm}
\setlength{\parskip}{1em}
\begin{document}
\begin{center}
\rule{6in}{1pt} \
{\large Olaf Schenk \\
{\bf Inertia Revealing Preconditioning For Large-Scale Nonconvex PDE-Constrained Optimization }}

Department of Computer Science \\ University of Basel \\ Switzerland
\\
{\tt olaf.schenk@unibas.ch}\\
Andreas W\"achter\\
Martin Weiser\end{center}

Fast nonlinear programming methods following the all-at-once approach
usually employ Newton's method for solving linearized Karush-Kuhn-Tucker
(KKT) systems. In nonconvex problems, the Newton direction is only
guaranteed to be a descent direction if the Hessian of the Lagrange function
is positive definite on the nullspace of the active constraints, otherwise
some modifications to Newton's method are necessary. This condition can be
verified using the signs of the KKT's eigenvalues (inertia), which are
usually available from direct solvers for the arising linear saddle point
problems. Iterative solvers are mandatory for very large-scale problems,
but in general do not provide the inertia. Here we present a preconditioner
based on a multilevel incomplete $LBL^T$ factorization, from which an
approximation of the inertia can be obtained [1]. The suitability of the
heuristics for application in optimization methods is verified on an
interior point method applied to the CUTE and COPS test problems, on
large-scale 3D PDE-constrained optimal control problems, as well as 3D
PDE-constrained optimization in biomedical cancer hyperthermia treatment
planning. The efficiency of the preconditioner is demonstrated on very
large-scale convex and nonconvex problems with millions of state and control
variables, both subject to bound constraints.

We propose an inertia revealing preconditioning approach for the solution of
nonconvex PDE-constrained optimization problems. If interior methods with
second-derivative information are used for these optimization problems, a
linearized Karush-Kuhn-Tucker system of the optimality conditions has to be
solved. The main issue addressed is how to ensure that the Hessian is
positive definite in the null-space of the constraints while neither
adversely affecting the convergence of Newton's method or incurring a
significant computational overhead. In the nonconvex case, it is of
interest to find out the inertia of the current iteration system so that the
matrix may be modified a posteriori to obtain convergence to a minimum.
However, in order to not destroy the rapid convergence rate of the interior
method, the modification has only be performed in the cases where the
inertia is not correct and factorization methods are very often used in
order to compute the inertia information [2]. However, in this work we
propose a new inertia revealing preconditioned Krylov iteration to solve the
linearized Karush-Kuhn-Tucker system of optimality conditions.

For general nonconvex nonlinear programming problems it is therefore of
utmost importance to modify the method in such a way that no ascent steps
are taken and finally to check that the delivered solution is indeed a local
minimum. For both tasks it is necessary to compute the inertia of the
Karush-Kuhn-Tucker system of the optimality conditions, which is relatively
easy using direct factorization methods. For very large scale problems, in
particular discretized 3D partial differential equations, direct
factorizations are prohibitively expensive both in terms of computing time
and storage requirements. How to reliably obtain the inertia from iterative
methods, however, is essentially an open problem.

We propose an algebraic multilevel preconditioning technique using maximum
weighted matchings [3,4] for revealing the inertia to be used in a interior
point method [2]. Our preconditioning approaches for the symmetric
indefinite Karush-Kuhn-Tucker systems are based on maximum weighted
matchings and algebraic multi-level inverse-based incomplete $LBL^T$
factorizations [5]. We present numerical results on several large-scale
three-dimensional examples of PDE-constrained optimizations in the full
space of states, control and adjoint variables with equality and
non-equality constraints and test them with artificial as well as clinical
data from biomedical cancer hyperthermia treatment planning. The largest
nonconvex optimization problem from three-dimensional PDE-constrained
optimization with the inertia revealing preconditioning approach has more
than 30 million state variables and hundred of thousands million control
variables with both lower and upper bound.

[1] {\sc O. Schenk, A. W\"achter, and M. Weiser}, {\em Inertia revealing
preconditioning for large-scale nonconvex constrained optimization},
Technical Report CS-2007-12 (2007), Computer Science Department, University
of Basel, Switzerland, submitted.

[2] {\sc A.~W\"achter and L.~T. Biegler}, {\em On the implementation of a
primal-dual interior point filter line search algorithm for large-scale
nonlinear programming}, Mathematical Programming, 106 (2006), pp.~25--57.

[3] {\sc I.~S. Duff and S.~Pralet}, {\em Strategies for scaling and pivoting
for sparse symmetric indefinite problems}, SIAM J. Matrix Analysis and
Applications, 27 (2005), pp.~313--340.

[4] {\sc O.~Schenk and K. ~G\"artner}, {\em On fast factorization pivoting
methods for symmetric indefinite systems}, Elec. Trans. Numer. Anal., 23
(2006), pp.~158--179.

[5] {\sc O.~Schenk, M.~Bollh\"ofer, and R.~A. R\"omer}, {\em On large scale
diagonalization techniques for the {Anderson} model of localization}, SIAM
Review, (2008).


\end{document}
