\documentclass{report}
\usepackage{amsmath,amssymb}
\setlength{\parindent}{0mm}
\setlength{\parskip}{1em}
\begin{document}
\begin{center}
\rule{6in}{1pt} \
{\large Richard Tran Mills \\
{\bf Experiences using PETSc-based solvers in ultrascale simulations of subsurface flow and transport using PFLOTRAN}}

Oak Ridge National Laboratory \\ 1 Bethel Valley Road \\ P O Box 2008 MS 6015 \\ Oak Ridge TN 37830
\\
{\tt rmills@ornl.gov}\\
Barry Smith\\
Glenn Hammond\\
	Peter Lichtner\end{center}

Detailed modeling of reactive flows in geologic media is necessary to
understand a number of environmental problems of national importance;
examples are migration of radionuclides in groundwater and geologic
sequestration of CO$_2$ in deep reservoirs. Such problems generally
require three dimensional simulations of non-isothermal systems
consisting of multiple phases and possibly dozens of chemical components,
and often must incorporate processes operating at different spatial and
temporal scales ranging over orders of magnitude. In many cases, detailed
understanding
will require extremely computationally demanding simulations that will
require not only the coming advances in hardware but commensurate
advances in algorithms and software.

We will discuss our experiences developing and using PFLOTRAN, a new code
for the fully implicit simulation of coupled thermal-hydrologic-chemical
processes in variably saturated, nonisothermal, porous media. PFLOTRAN is
built on top of PETSc, the Portable, Extensible Toolkit for Scientific
Computation developed at Argonne National Laboratory. Leveraging PETSc
has allowed us to develop---with a relatively modest investment in
development effort---a code that exhibits excellent performance on
machines ranging from laptops to the very largest scale supercomputers.

We have recently begun using PFLOTRAN to simulate problems involving on
the order of one billion total degrees of freedom using tens of thousands
of processors on Jaguar, the Cray XT4 at Oak Ridge National Laboratory.
Running problems at such massive scales presents unique and considerable
challenges for the underlying algebraic solvers (in which the great
majority of our wall-clock time is spent). We will discuss some of the
solver challenges we have faced and the progress that has been made in
addressing them, and we will outline future challenges
that we foresee as we continue to add new physics and capabilities to PFLOTRAN.


\end{document}
