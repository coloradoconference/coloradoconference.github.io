\documentclass{report}
\usepackage{amsmath,amssymb}
\setlength{\parindent}{0mm}
\setlength{\parskip}{1em}
\begin{document}
\begin{center}
\rule{6in}{1pt} \
{\large Jing Li \\
{\bf A dual-primal FETI method for a type of fluid-structure interaction problem }}

Department of Mathematical Sciences \\ Kent State University \\ Kent \\ OH 44242
\\
{\tt li@math.kent.edu}\\
Charbel  Farhat \\
Philip  Avery \\
	Radek Tezaur\end{center}

The Dual-Primal Finite Element Tearing and Interconnecting method
(FETI-DP) is extended to solving a type of fluid-structure interaction
problem which simulates the time harmonic vibration of an elastic
structure immersed in a fluid in response to an incident acoustic wave
and/or external forces, as well as scattering of the acoustic wave by the
obstacle. We show that straightforward application of the FETI-DP
algorithm to fluid-structure interaction problems does not perform
satisfactorily, in particular for three-dimensional problems. Appropriate
choices of coarse level primal degrees of freedom on the fluid-structure
interface have to be made to improve both the iteration count and the
overall CPU time greatly. Such choices and their implementations are
discussed in this paper and they are able to make the convergence rate of
the proposed algorithm scalable with respect to the problem size, the
number of subdomains, and the wave number. Numerical experiments for
solving several three-dimensional fluid-structure interaction problems
are implemented on cluster computers and demonstrate the satisfactory
performance of the proposed algorithm.


\end{document}
