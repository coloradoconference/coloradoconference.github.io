\documentclass{report}
\usepackage{amsmath,amssymb}
\setlength{\parindent}{0mm}
\setlength{\parskip}{1em}
\begin{document}
\begin{center}
\rule{6in}{1pt} \
{\large Dywayne Nicely \\
{\bf Restarted Lanczos Algorithms}}

Math Department \\ Baylor University \\ Waco \\ TX 76798-7328
\\
{\tt Dywayne{\_}Nicely@baylor.edu}\\
Ron Morgan\end{center}

Restarted versions of the symmetric and nonsymmetric Lanczos algorithms
are given for computing eigenvalues. Approximate eigenvectors are
retained at the restart as with implicitly restarted Arnoldi. In the
nonsymmetric case, both the right and left eigenvectors are computed.
Three-term recurrences are used to reduce expense, and the restarting
limits the storage needs. Some reorthogonalization is needed, and several
approaches will be discussed.

These restarted Lanczos methods can also be used to solve large systems
of linear equations. In particular, a restarted BiCG algorithm will be
given that solves both the linear equations and the eigenvalue problem.
The inclusion of eigenvectors in the subspaces causes deflation of small
eigenvalues that improves the convergence compared to other restarted
approaches. For multiple right-hand sides, a deflated BiCGStab approach
uses both the right and left eigenvectors that were generated by the
restarted BiCG. Applications to quantum chromodynamics will be given.

We also investigate a restarted two-sided Arnoldi algorithm. We compare
expense and stability of this approach with restarted nonsymmetric
Lanczos.


\end{document}
