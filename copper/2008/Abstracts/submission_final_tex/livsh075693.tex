\documentclass{report}
\usepackage{amsmath,amssymb}
\setlength{\parindent}{0mm}
\setlength{\parskip}{1em}
\begin{document}
\begin{center}
\rule{6in}{1pt} \
{\large Ira Livshits \\
{\bf Multiscale method for finding many eigenfunctions of partial differential operators }}

Department of Mathematical Sciences \\ Ball State University
\\
{\tt ilivshits@bsu.edu}\end{center}

In this talk we introduce a new approach of finding a full eigenbasis of
discretized partial differential equations. The developed method allows
to calculate $O(N)$ eigenfunctions in $O(N\log N)$ operations, $N$ being
the size of the problem on the finest scale. The approach allows an
extremely efficient use of all intermediate scales,
in particular the coarsest one. There not only all eigenfunctions are
calculated but also the entire orthogonalization process takes place.

The algorithm determines dynamically when a new set of operators needs to
be calculated (to guarantee a required accuracy) as well as a vicinity of
eigenvalues for which this set should be accurate. The algorithm is
performed in the AMG framework, using prolongation operators each built
to accurately interpolate, in the least square sense, a set of test
functions, approximations to eigenfunctions with close eigenvalues.

Numerical experiments for the one-dimensional problems will conclude the
talk, along with a brief
discussion of the approach' extension to two dimensions. Joint work with Achi Brandt.


\end{document}
