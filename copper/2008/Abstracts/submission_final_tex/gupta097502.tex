\documentclass{report}
\usepackage{amsmath,amssymb}
\setlength{\parindent}{0mm}
\setlength{\parskip}{1em}
\begin{document}
\begin{center}
\rule{6in}{1pt} \
{\large Anshul Gupta \\
{\bf An Incomplete Factorization Based Robust and High-Performance Preconditioner for SPD Systems}}

IBM T J Watson Research Center \\ 1101 Kitchawan Rd \\ Yorktown Heights \\ NY 10598
\\
{\tt anshul@watson.ibm.com}\end{center}

Incomplete factorization methods have long been used to derive
preconditioners for Krylov subspace methods to solve large sparse systems
of linear equations.
Like most preconditioning methods, incomplete factorization has
its share of drawbacks. First, a computer program
implementing a non-static pattern incomplete factorization, which drops
entries based on their magnitude and the density of their row or column
in the incomplete factor, usually runs quite
inefficiently due to dropping related bookkeeping tasks. As a result,
for many problems, a direct factorization turns out to be mush faster,
even if it consumes significantly more memory than incomplete
factorization.
Secondly, for symmetric
positive-definite (SPD) coefficient matrices, Cholesky factorization
may encounter a negative pivot. This would result in a breakdown if incomplete
$LL^T$ factorization is used or an indefinite preconditioner if incomplete
$LDL^T$ factorization is used.
Finally, the performance and effectiveness of incomplete factorization
is not only problem dependent, but is
also highly sensitive to parameters such as drop tolerance,
fill factor, or level of fill.
We describe our work
that attempts to address the three aforementioned shortcomings
of preconditioning based on incomplete factorization for SPD systems.

The first problem we address is that of performance. A typical
implementation of complete sparse Cholesky factorization
can realize a fairly respectable fraction of the peak performance of a
machine. There are two main reasons for his. First, the numerical
factorization is preceded by a symbolic factorization phase that
computes the static structure of the factors.
Secondly, supernodal and multifrontal
techniques ensure that practically all numerical computation is performed by
cache-friendly Level 2 and Level 3 basic linear algebra subprograms
(BLAS). Block variants of incomplete
factorization have been successfully used to enhance its performance
through the use of higher level BLAS.
These block algorithms, however, apply only to matrices that
have a naturally occurring block structure. We have developed a sparse
incomplete Cholesky factorization procedure that relies on dense or nearly
dense blocks within the factors as they emerge. Therefore, this
algorithm is able to extract higher efficiencies even for those matrices
that have small or no blocks to begin with.
The algorithm follows an elimination
tree and proceeds much like a direct multifrontal
algorithm. It inexpensively detects and uses supernodes along the
portions of the elimination tree that consist of a chain of single-child
parent nodes. We experimentally demonstrate a dramatic improvement in
the speed of incomplete factorization at the cost of a small increase
in the memory required to store the factors and intermediate data-structures.

The second problem we address is that of breakdown of the Cholesky
process due to negative or near zero pivots. Traditional, methods
to address this problem fall into two categories:
{\em Preemptive methods}, which either modify the matrix or the
factorization method to ensure that breakdown does not occur, and
{\em Reactive methods}, which apply some sort of local correction or roll
back the computation upon encountering a zero or negative pivot
to incrementally increase fill-in until the factorization succeeds.
Preemptive methods unnecessarily increase the cost and error of incomplete
factorization, even for those matrices for which the standard
incomplete factorization process may have succeeded. Reactive methods
that roll back the computation are too costly and the ones that apply
a local correction are too error prone. Our first approach, which we
show to work well in practice, is to simply drop the onerous requirement
that the preconditioner be positive-definite. In our elimination-tree
based incomplete Cholesky factorization, we simply stop at a point along
a branch of the tree where a negative pivot is encountered, and proceed
with eliminating other parts of the tree where the Cholesky process can
continue. When Cholesky cannot proceed any further, we compute the
Schur complement, use threshold-based dropping to further sparsify it,
and use a direct $LDL^T$ factorization on the remaining submatrix.
During the solution phase, the conjugate gradient method is used if the
preconditioner is positive-definite and GMRES or symmetric QMR is used
if it is indefinite.

Finally, we build in features in both the incomplete factorization
algorithm and its software implementation that minimize the effort
required by the user to select appropriate parameters. The software
automatically selects two thresholds {\em droptol} and {\em fillfact}
based on the diagonal dominance of the coefficient matrix. Based on
these thresholds, it selects RCM ordering when it expects
low fill-in and nested dissection ordering when it
expects high fill-in. In order to minimize the impact of initial parameter
selection, our software takes
advantage of the fact that many applications require the solution of a
series of systems with the coefficient matrices changing gradually.
Based in the relative cost of the factorization and iterative solution
phases of the computation, our solver selects a direction and step size
to automatically modify {\em droptol} and {\em fillfact} to apply to
the next system. If at any point, the total time for preconditioner
generation and solution exceeds that of the preceding iteration, then
the thresholds are reset to their previous value and the step size is
reduced. Our experiments indicate that this search converges to
a reasonably good set of thresholds in 3 to 5 iterations. This relatively
simple search built into the solver not only automates the process
of threshold selection, but it also continuously adapts the thresholds
if the characteristics of the coefficient matrices change while
solving a sequence of systems.

We experimentally demonstrate that the techniques described above
result in a robust and high-performance sparse linear solver.


\end{document}
