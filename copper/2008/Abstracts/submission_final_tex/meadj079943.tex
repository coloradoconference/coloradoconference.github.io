\documentclass{report}
\usepackage{amsmath,amssymb}
\setlength{\parindent}{0mm}
\setlength{\parskip}{1em}
\begin{document}
\begin{center}
\rule{6in}{1pt} \
{\large Jodi Mead \\
{\bf The chi-squared method for constrained parameter estimation, and calculation of data weights.}}

1910 University Dr \\ Boise \\ ID 83725-1555
\\
{\tt jmead@boisestate.edu}\\
Rosemary Renaut\\
Molly Gribb\\
	McNamara, Jim\end{center}

The chi-squared method is a parameter estimation method for ill-posed
problems. We form it as a weighted least squares problem, where the
weights are found by ensuring the parameter estimates satisfy the
chi-squared test. Data and parameters are assumed to be random, but not
necessarily normally distributed. The method was introduced by Mead
(2007) and made efficient by Mead and Renaut (submitted). It is
considerably more efficient, and as accurate as traditional L-curve and
cross-correlation methods for parameter estimation. In this work we
develop the chi-squared method for parameters with constraints, and use
it to find data weights. Results from Hydrology will be shown, and
include soil moisture parameter estimates. The data error is calculated
by the chi-squared method, and parameter estimates are found within a
priori data uncertainty ranges.


\end{document}
