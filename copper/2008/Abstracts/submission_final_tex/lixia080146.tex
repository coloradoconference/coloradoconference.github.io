\documentclass{report}
\usepackage{amsmath,amssymb}
\setlength{\parindent}{0mm}
\setlength{\parskip}{1em}
\begin{document}
\begin{center}
\rule{6in}{1pt} \
{\large Xiaoye, S. Li \\
{\bf Evaluation of sparse factorization and triangular solution on multicore architectures}}

Lawrence Berkeley National Laboratory \\ MS 50F-1650 \\ 1 Cyclotron Rd \\ Berkeley \\ CA 94720-8139
\\
{\tt xsli@lbl.gov}\end{center}

Multicore processors will be the basic building blocks for
computer systems ranging from laptops to supercomputers.
New software developments at all levels are needed to fully
utilize these systems.
We are conducting performance evaluation of different parallel
algorithms for sparse LU factorization and triangular solution
on representative multicore machines, including an eight-core
Intel Clovertown and an eight-core Sun Niagara2 with
64-way threading.

In this study, we include both pthreads and MPI implementations,
and both left-looking (implemented in SuperLU\_MT) and
right-looking (implemented in SuperLU\_DIST) algorithm variants.
The preliminary results showed that the pthreads implementation
consistently delivers nearly linear speedups for most problems,
and a left-looking algorithm is usually superior.
We will present quantitative assessment of performance with
various algorithmic components. We believe our findings are
also relevant to the class of preconditioners which are based
on incomplete factorizations.


\end{document}
