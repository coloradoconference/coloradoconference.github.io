\documentclass{report}
\usepackage{amsmath,amssymb}
\setlength{\parindent}{0mm}
\setlength{\parskip}{1em}
\begin{document}
\begin{center}
\rule{6in}{1pt} \
{\large Ulrike Meier Yang \\
{\bf Parallel algebraic multigrid for systems of PDEs }}

Lawrence Livermore National Laboratory \\ Center for Applied Scientific Computing \\ Box 808 - L560 \\ Livermore \\ CA 94551
\\
{\tt umyang@llnl.gov}\\
Allison H. Baker\end{center}

Algebraic multigrid (AMG) methods are popular for solving large sparse
linear systems, particularly those resulting from the discretization of a
scalar elliptic PDE. When solving linear systems derived from systems of
PDEs involving multiple unknowns, modifications to classical AMG are
typically required. In particular, two accepted approaches are treating
variables corresponding to the same unknown separately (the "unknown"
approach) and treating variables corresponding to the same physical node
together (the "nodal" approach). We discuss the applicability and
parallel performance of these two approaches as well that of a "hybrid"
approach that combines aspects of each. In addition, because we are
interested in the efficient parallel solution of large systems of PDEs
arising from elasticity applications, we also discuss improving
interpolation of the rigid body modes.


\end{document}
