\documentclass{report}
\usepackage{amsmath,amssymb}
\setlength{\parindent}{0mm}
\setlength{\parskip}{1em}
\begin{document}
\begin{center}
\rule{6in}{1pt} \
{\large Peter Lucas \\
{\bf Efficient nonlinear solver for unsteady {\sc cfd} problems solved with higher order implicit time integration schemes}}

Kluyverweg 1 \\ 2629 HS Delft \\ The Netherlands
\\
{\tt p.lucas@tudelft.nl}\\
Hester Bijl\end{center}

Over the last decade(s) there has been an increasing demand to predict
unsteady flows. Knowledge of unsteady loads on, for example, wind turbine
blades have proved to be of vital importance for an efficient life cycle
design. Due to the unsteadiness of these problems and the large stiffness
associated with large Reynold number flows, these computations can take
months on a series of processors. To speed up these computations we use
higher order multistage implicit time integrations schemes, because they
were shown to be more computationally efficient than the standard
backward difference schemes, also for engineering orders of
accuracies$^1$. However, specially for larger time steps the performance
of standard nonlinear multigrid deteriorates. The goal of this research
is therefore to further speed up the computations.

Because of the potential speed up of a Jacobian-free Newton-Krylov ({\sc
jfnk}) method over standard nonlinear multigrid for these problems$^2$ we
have implemented a {\sc jfnk} method in our aerodynamic production code.
Preconditioning of the linear systems that arise after Newton
linearization is, however, of crucial importance for the computational
efficiency of the {\sc jfnk} method$^3$. This paper therefore seeks for
an optimal preconditioner to compute unsteady, large Reynolds' number
flows solved with higher order implicit time integration schemes. Because
higher order implicit time integrations schemes are not yet often used,
not much is known on how to create an efficient preconditioner.

Possible preconditioners are: (non)linear multigrid, a recursive variant
of {\sc gmres} and (matrix-free) approximate factorizations ({\sc af}) of
the Jacobian. (Non)linear multigrid, matrix-free {\sc af}'s and {\sc
gmresr} have the advantage of a low memory consumption. However, they
require more computational time per iteration and the low-frequency
errors may be poorly damped. {\sc af}'s of (an approximation of) the
Jacobian can be very powerful because errors in the whole frequency
domain are damped. Furthermore, once the factorization has been computed
it can be reused for next linear solves, which can make them relatively
cheap to apply. Finally, this type of preconditioner is relatively
straightforward to implement in our aerodynamic production code. We have
therefore chosen to precondition the linear systems with a Jacobian based
Incomplete Lower Upper factorization based on the footprint of the
Jacobian ({\sc ilu}(k)). Preliminary results showed that these {\sc af}'s
greatly outperformed multilevel {\sc af}'s and {\sc af}'s based on dual
thresholds.

Disadvantages of {\sc ilu}(k) preconditioners can be lack of
robustness$^4$ and large memory consumption. The lack of robustness is
easily circumvented by slightly increasing the diagonal dominance of the
Jacobian. A common approach to reduce memory usage is to neglect
contributions further away than nearest neighbors in the Jacobian$^5$.
However, for unsteady two-dimensional flows around wind turbine profiles
on unstructured grids, we find a much better linear convergence with an
{\sc ilu}(k) preconditioner that is based on a lumped Jacobian. For the
lumped Jacobian all contributions from neighbors of neighbors are
included into the Jacobian. Hereafter the contributions from neighbors of
neighbors are lumped to nearest neighbors and the {\sc af} is computed.

To further enhance the efficiency of the iterative solver we have
investigated the successive combination of nonlinear multigrid and the
{\sc jfnk} method. Furthermore, we are investigating recycling of Krylov
subspace to speed up the linear solves. In our paper results for a two
and three dimensional unsteady test case are discussed.

\vspace{1cm}

\begin{tabular}{lp{110mm}}
\hspace{-3mm}{[1]}&
A. H. van Zuijlen, and H. Bijl. {\em Implicit and explicit higher order
time integration schemes for structural dynamics and fluid-structure
interaction computations}, Computers $\&$ Structures, Vol. 83, 93-105,
2005.
\\[1mm]
\hspace{-3mm}{[2]}&
G. Jothiprasad, D.J. Mavriplis, D.J. and D. A. Caughey. {\em Higher-order
time integration schemes for the unsteady Navier-Stokes equations on
unstructured meshes}, Journal of Computational Physics, Vol. 191,
542-566, 2003.
\\[1mm]
\hspace{-3mm}{[3]}&
D.A. Knoll and D.E. Keyes. {\em Jacobian-free Newton-Krylov methods: a
survey of approaches and applications}, Journal of Computational Physics,
Vol. 193, 357-397 2004.
\\[1mm]
\hspace{-3mm}{[4]}&
E. Chow and Y. Saad. {\em Experimental study of ilu preconditioners for
indefinite matrices}, Journal of computational and applied mathematics,
Vol. 86, 387-414 1997.
\\[1mm]
\hspace{-3mm}{[5]}&
P. Wong and D.W. Zingg. ``Three-Dimensional Aerodynamic Computations on
Unstructured Grids Using a Newton-Krylov Approach''. {\em 17$^{th}$ AIAA
Computational Fluid Dynamics Conference}, Toronto, Canada, June 2005,
AIAA 2005-5231.
\\[1mm]

\end{tabular}


\end{document}
