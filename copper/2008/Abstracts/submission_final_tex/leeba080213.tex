\documentclass{report}
\usepackage{amsmath,amssymb}
\setlength{\parindent}{0mm}
\setlength{\parskip}{1em}
\begin{document}
\begin{center}
\rule{6in}{1pt} \
{\large Barry Lee \\
{\bf Multigrid for a Class of System Elliptic Partial Differential Equations}}

Center for Applied Scientific Computing \\ Lawrence Livermore National Laboratory \\ P O ~Box~808 \\ L-561 \\ Livermore \\ CA 94551 \\ USA
\\
{\tt lee123@llnl.gov}\end{center}
%\documentclass[10pt]{siamltex}
%%
%\begin{document}
%
%\title{Multigrid for a Class of System Elliptic Partial Differential Equations}
%\author{B. Lee
%\thanks{Center for Applied Scientific Computing, Lawrence Livermore National Laboratory,
%Livermore, CA. {\it email}: lee123@llnl.gov. This work performed under the auspices of the U.S. Department
%of Energy by Lawrence Livermore National Laboratory under Contract DE-AC52-07NA27344.}}
%\date{}
%\maketitle
%
%\begin{abstract}
This talk presents a new approach for solving a class of strongly coupled system of elliptic partial
differential equations (pdes) using multigrid strategies. Unlike existing
multigrid solvers which are constructed directly from the whole system
operator, this approach builds the solver using an approximate or exact factorization of the
system operator. This factorization has an algebraic coupling term and a diagonal 
(decoupled) differential operator. Exploiting the factorization,
this approach produces decoupled systems on the coarse levels. The corresponding coarse-grid
operators are in fact the Galerkin variational coarsening of the diagonal differential operator.
Thus, rather than performing delicate coarse-grid selection and interpolation weight procedures on the
original strongly coupled system as in some of the existing methods, these procedures are isolated 
to the diagonal differential operator. Moreover, as the coarse systems are decoupled,
simple smoothers can be used on the coarse levels even for original systems
that are strongly non-diagonally dominant. The small payoff for this approach is
a violation of the Galerkin variational principle on the first coarse grid:
this method can be viewed somewhat as a re-discretization of the system of
pdes for the first coarse level, although additional coarser-grid problems are derived
by applying the Galerkin variational principle to this first coarse-grid equation. This approach is, 
of course, viable only for systems of pdes that afford a realizable algebraic-differential factorization.
%\end{abstract}

\end{document}

