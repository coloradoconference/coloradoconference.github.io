\documentclass{report}
\usepackage{amsmath,amssymb}
\setlength{\parindent}{0mm}
\setlength{\parskip}{1em}
\begin{document}
\begin{center}
\rule{6in}{1pt} \
{\large Ray Tuminaro \\
{\bf Boundary Conditions for Block Preconditioners Based on Approximate Commutators}}

MS 9159 \\ Sandia National Laboratories \\ PO Box 969 \\ Livermore \\ Ca 94551
\\
{\tt rstumin@sandia.gov}\\
Howard Elman\end{center}

Pressure convection-diffusion preconditioners are based on an approximation to the
inverse Schur complement where the true Schur complement is obtained by eliminating
velocity variables from the incompressible Navier-Stokes equations. This
inverse Schur complement is essentially approximated by the product of two
matrices: a convection-diffusion operator and an approximation to the inverse
Laplacian. These operators follow from commuting properties of certain
differential operators away from boundaries. While these operators are now fairly
well understood for the domain interior, appropriate boundary conditions for
defining the preconditioning operators are significantly less clear.

In this talk, we look closely at boundary conditions. We show in particular that
for problems with velocities satisfying inflow boundary conditions, the
preconditioner can be defined so that discrete commutativity holds exactly at
the inflow boundary. That is, in certain situations the inverse Schur complement
approximation is exact, even at domain boundaries. This exact relationship
motivates new boundary conditions which give rise to good preconditioners in more
general settings. In addition to examining standard pressure convection-diffusion
preconditioners, the analysis motivates a simple and practical modification to the
related least squares commutator method (LSC). We demonstrate that the new boundary
conditions have a significant impact on the convergence behavior of both the
pressure convection-diffusion method and the LSC method. Further, mesh independent
convergence rates are now obtained for both methods in cases where previously some
convergence deterioration occurred as the mesh was refined.


\end{document}
