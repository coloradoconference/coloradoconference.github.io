\documentclass{report}
\usepackage{amsmath,amssymb}
\setlength{\parindent}{0mm}
\setlength{\parskip}{1em}
\begin{document}
\begin{center}
\rule{6in}{1pt} \
{\large Milan Mihajlovic \\
{\bf Efficient AMG block preconditioning of the Boussinesq problem}}

School of Computer Science \\ University of Manchester \\ Oxford Road \\ Manchester M13 9PL \\ United Kingdom
\\
{\tt milan@cs.man.ac.uk}\\
Howard Elman\\
David Silvester\end{center}

In this presentation we describe the design and examine the effectiveness
of a block preconditioner, based on a robust AMG solver, for the
iterative solution of linear systems that arise in the finite element
discretisation of the Boussinesq problem. The Boussinesq problem
represents a model of a thermally-driven flow, i.e. the flow of a
Newtonian fluid under a temperature
gradient. Such a model is valid under the assumption that the temperature
gradient is small, which allows fluid density to change linearly under
the action of gravity. The applications of this model are considerable:
various heat exchange systems (such as nuclear reactor cooling system,
forced cooling of
electronic equipment), semiconductor fabrication (crystal growth),
superconductivity (e.g. Rayleigh-Benard flow of liquid helium), etc.

The Boussinesq flow represents a coupled system of PDEs for the unknown
velocity, pressure and temperature fields. It consists of the
Navier-Stokes equations describing the fluid flow coupled with the
convection-diffusion equation for energy balance. The two classes of
problem that are the most frequently studied in this context are
time-dependent problems (describing the
transient behaviour of the system), and steady-state problems (essential
in the linear stability analysis).

The finite element discretisation of both classes of problem can be
performed using the same combinations of velocity-pressure approximation
spaces (as in the case of Navier-Stokes equations), while the usual
choice for the temperature
approximation space is the piecewise quadratic finite element space. The
time domain in transient problems is usually discretised using modern
adaptive fully implicit time-stepping algorithms (such as the trapesoidal
rule), resulting in a
sequence of large non-linear problems that need to be solved. Similar
situations arise in linear stability analysis if some of the continuation
algorithms (e.g. Keller's pseudo-arclength method) are used to compute
the solution paths. Standard linearisation procedures (e.g. Picard's or
Newton's method) are used to solve these non-linear systems, resulting in
a sequence of linear problems that need to be solved. From this argument
it is obvious that accurate solution of the Boussinesq problem (both
transient and steady-state) requires the solution of potentially large
number of linear systems, with different coefficient matrices. The size
of such linear systems needed for
accurate prediction of certain effects (e.g. critical values of the
bifurcation parameters) can be very large, especially in 3D. Thus, robust
and efficient preconditioned iterative solvers are absolutely essential
in this context.

In this presentation we discuss the development and study the efficiency
of a preconditioner for a fully coupled Boussinesq system. The
preconditioner is based on existing efficient preconditioners for the
Navier-Stokes equations (we
choose the scaled commutator preconditioner in this case). In the case of
Picard's linearisation, the Boussinesq coefficient matrix has an inherent
block triangular form, which allows the natural extension of the
Navier-Stokes preconditioner. When Newton's linearisation is used, the
block triangular form
of the coefficient matrix is lost, however the use of the same
preconditioner as in Picard's case (with the necessary modifications in
the treatment of the momentum block, as in the Navier-Stokes case) proved
to be the effective choice.
We test our preconditioning strategy on a range of non-trivial Boussinesq
problems recently studied in literature in various contexts
(differentially heated cavities, Rayleigh-Benard convection). We consider
both the transient and the steady-state problems, and demonstrate
numerically robustness of our
preconditioner with respect to the time step size, diffusivity
parameters, spatial discretisation parameter, grid stretching and the
shape of the domain.


\end{document}
