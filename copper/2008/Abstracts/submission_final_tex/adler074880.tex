\documentclass{report}
\usepackage{amsmath,amssymb}
\setlength{\parindent}{0mm}
\setlength{\parskip}{1em}
\begin{document}
\begin{center}
\rule{6in}{1pt} \
{\large James Adler \\
{\bf Nested Iteration FOSLS-AMG for Resistive Magnetohydrodynamics}}

2075 Goss St #11 \\ Boulder \\ CO 80302
\\
{\tt james.adler@colorado.edu}\end{center}

Magnetohydrodynamics (MHD) is a single-fluid theory that describes Plasma
Physics. MHD treats the plasma as one fluid
of charged particles. Hence, the equations that describe the plasma form
a nonlinear system that couples Navier-Stokes
with Maxwell's equations. To solve this system, a
nested-iteration-Newton-FOSLS-AMG approach is taken. The system is
linearized on a coarse grid using a Newton step and is then discretized
in a FOSLS functional upon which several AMG
V-cycles are performed. If necessary, another Newton step is taken and
more V-cycles are done. When the linear functional
has converged "enough," the approximation is interpolated to a finer grid
where it is again linearized, FOSLized, and
solved for. The goal is to determine the most efficient algorithm in this
context. One would like to do as much work as
possible on the coarse grid including most of the linearization. Ideally,
it would be good to show that at most one
Newton step and a few V-cycles are all that is needed on the finest grid.
This talk will develop theory that supports
this argument, as well as show experiments to confirm that the algorithm
can be efficient for MHD problems. Currently,
two test problems have been studied, both with the use of FOSPACK: a 3D
steady state MHD test problem with a manufactured
solution, and, for a more realistic problem, a reduced 2D time-dependent
formulation. The latter equations can simulate
a "large aspect-ratio" tokamak, with non-circular cross-sections. Here,
the problem was reformulated in a way that is
suitable for FOSLS and FOSPACK. This talk will discuss two stopping
criteria. First, on each refinement level, when
should one stop solving the linear system and re-linearize the problem.
Secondly, how does one choose whether to do
another Newton step or move to a finer grid. In addition, different types
of h and p refinement will be tested, as well
as adaptive mesh refinement. The goal is to resolve as much physics from
the test problem with the least amount of work.


\end{document}
