\documentclass{report}
\usepackage{amsmath,amssymb}
\setlength{\parindent}{0mm}
\setlength{\parskip}{1em}
\begin{document}
\begin{center}
\rule{6in}{1pt} \
{\large Genetha Gray \\
{\bf Hybrid Optimization: Combining the Best to Overcome the Worst}}

Sandia National Labs \\ P O Box 969 \\ MS 9159 \\ Livermore \\ CA 94550
\\
{\tt gagray@sandia.gov}\\
Katie Fowler\\
Josh Griffin\\
	Taddy, Matt\end{center}

Every optimization technique has inherent strengths and weaknesses.
Moreover, some optimization algorithms contain characteristics which make
them better suited to solve particular kinds of problems. Hybridization,
or the combining of two or more complementary, but distinct approaches,
allows the user to take advantage of the beneficial elements of multiple
methods. For example, consider two methods A and B where method A is
capable of handling noise and undefined points and method B excels in
smooth regions with small amounts of noise. In this case, method A may be
unacceptably slow to find a solution while method B may fail in noisy or
discontinuous regions of the domain. By forming a hybrid,
method A can help overcome difficult regions of the domain and method B
can be applied for fast convergence and efficiency.

In this talk, we will examine the application of hybrid optimization to
water resources management problems. Hybrids are particularly suited to
these problems as the objective function is often non-smooth or
discontinuous and the feasible region is usually disconnected. We will
describe an algorithm which combines statistical emulation via treed
Gaussian process with pattern search optimization. We will demonstrate
the applicability of our hybrid method to a plume containment problem
that was proposed in the literature specifically for benchmarking
purposes, and has
already been used for the comparison of a variety of derivative-free
optimization algorithms. In addition, we will describe how the treed
Gaussian process can also be used as a post-processing tool to increase
insight into the problem.


\end{document}
