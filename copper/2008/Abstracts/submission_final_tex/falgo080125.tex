\documentclass{report}
\usepackage{amsmath,amssymb}
\setlength{\parindent}{0mm}
\setlength{\parskip}{1em}
\begin{document}
\begin{center}
\rule{6in}{1pt} \
{\large Robert D. Falgout \\
{\bf Parallel Sweeping Algorithms in SN Transport}}

Center for Applied Scientific Computing \\ Lawrence Livermore National Laboratory \\ P O ~Box~808 \\ L-561 \\ Livermore \\ CA 94551 \\ USA
\\
{\tt rfalgout@llnl.gov}\end{center}

A potential bottleneck when solving Boltzmann transport equations in parallel is
the inversion of the streaming operator. The discretized form of this operator
is a lower triangular matrix or block lower triangular matrix with small blocks.
The solution of these triangular systems by direct methods is inherently
sequential. Although various overloading techniques have been used to amortize
the costs of these lower triangular solves or ``sweeps'', the practicality of
scaling to massively parallel machines with tens of thousands of processors is
unclear.

In this talk, we will present new theoretical scaling models for sweeping
algorithms and compare with experiment. In theory, these algorithms have the
potential to scale like $O(dP^{1/d} + M)$, where $d$ is the spatial dimension of
the problem, $M$ is the number of directions, and $P$ is the number of
processors. When $M$ is fairly large, it masks the effect of the $P$ term,
whereby delaying the poor asymptotic scaling behavior. This delay may be
adequate in some cases to get practical performance, even up to tens of
thousands of processors. However, some popular parallel sweep algorithms may
scale worse than this best-case theoretical model. This will also be discussed
in the talk.


\end{document}
