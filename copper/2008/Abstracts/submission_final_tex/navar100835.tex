\documentclass{report}
\usepackage{amsmath,amssymb}
\setlength{\parindent}{0mm}
\setlength{\parskip}{1em}
\begin{document}
\begin{center}
\rule{6in}{1pt} \
{\large Pablo Navarrete Michelini \\
{\bf Design of Multigrid Methods for Traffic Models based on Correlated Random Walks}}

Center for Wireless Systems and Applications \\ Purdue University \\ 465 Northwestern Ave \\ West Lafayette \\ IN 47907-2035 \\ USA
\\
{\tt pnavarre@purdue.edu}\\
Edward J. Coyle\end{center}

We study the configuration of multigrid methods for computing statistics
of traffic models. For this purpose we consider the available models for
the mobility of vehicles based on different types of continuous-time
Markov chains. Of these models we concentrate on the subset represented
by Correlated Random Walks because we will show that they are amenable to
solution by efficient multigrid methods.

The goal of the multigrid method is to compute statistics of ``absorbing
times'' that represent the time needed for a vehicle to arrive at a
certain destination from either a random or a fixed starting point.
Within this framework we apply our recent results on the convergence of
multigrid methods to design efficient inter-grid operators [1]. Thus, we
are able to obtain an optimal method, with efficient parameters, to
compute statistics on traffic models based on correlated random walks. We
are also able to provide a complete understanding of the convergence
issues in this setting.

A mobility model based on continuous-time Markov chains models the
streets of a city as a finite grid where each point on the grid
corresponds to an intersection of streets. Therefore, a vehicle moving in
the city belongs to one point in the grid at a time and it jumps to other
grid nodes at random times. The time instant at which the vehicle takes
its steps is governed by a Poisson process, meaning that it moves from
one intersection to a neighboring intersection with a randomly chosen
velocity. In the simplest scenario, the transitions occur only between
neighboring grid nodes, which represents a random walk mobility model. In
the correlated random walk mobility model, the probability of a vehicle
moving between grid nodes in the same direction as its previous step is
different than the probability of moving in the opposite direction. Thus,
correlated random walks are both more flexible and more accurate models
of traffic because they can account for time dependency, geographical
restrictions, and nonzero drift.

The statistics associated with absorbing times are important for the
evaluation of mobility models because they give an accurate description
of the motion of vehicles on a grid. For example, knowledge of the times
between a vehicle's contacts with other vehicles is important in mobile
ad-hoc networks that create a communication network based only on
wireless communications between vehicles. An important goal is to scale
the network and maintain reliable communication. In this scenario, it is
known that the mobility patterns have a significant impact on the network
coverage and throughput-delay characteristics [2]. The statistics of
absorbing times are thus important for the design of large mobile ad-hoc
networks.

The statistics of absorbing times are difficult to obtain since they
require the solution of sparse linear systems. The multigrid algorithm
represents an optimal choice for this purpose because its complexity
scales optimally. Nevertheless, the configuration of parameters needs
special care to guarantee and optimize convergence. Although the
structure of the systems is unusual for classic multigrid theory, we
prove that it fits the assumptions of our previous results on convergence
analysis. Using this analysis, we can design smoothing and inter-grid
operators with precise adjustment of convergence properties. Thus, we
obtain an efficient and robust algorithm for the calculation of absorbing
times.

To apply our analysis, we first need to check some assumptions on the
system. The strongest assumption is condensed in a single algebraic
property, called the Harmonic Aliasing Property, that contains all the
information needed from the geometry of the discretization and the modal
structure of the system. Then, the starting point is to check this
property on the correlated random walk mobility model. After proving this
property holds, we apply our analysis to design efficient inter-grid
operators and check their performance. By this means we obtain the exact
rates at which groups of modal components of the error evolve and
interact for systems of different sizes. This provides a complete
understanding of the algorithm in terms of its convergence properties and
how they scale with the system.

The multigrid configuration that we propose is not the only choice for
this application. The most common choice for these type of problems is
algebraic multigrid methods (AMGs), as they are designed to avoid almost
any assumption on the system. Unfortunately, the convergence results
obtained so far in the theory of AMG are not as strong as the
well-established results for linear operators with constant stencil
coefficients, known as Local Fourier Analysis (LFA). On the other hand,
the use of LFA for this application is not possible because the linear
systems needed to be solved have variable stencil coefficients. Thus,
this application shows how our convergence analysis is more flexible than
LFA as it allows us to obtain exact convergence rates for models where
LFA is not applicable.

Finally, in our previous work it remained unknown how flexible our
convergence analysis could be. We have shown that it is more general than
LFA but at the same time it is not as general as AMG. The last seems
reasonable, as our analysis gives convergence rates which are as precise
as the ones obtained by LFA. We have also provided examples of systems
where our analysis works but LFA does not. These early examples were not,
unlike the one in this paper, motivated by real applications. Thus, the
application of our convergence analysis in traffic models gives new
insight on the nature of the systems where our analysis is applicable.
This application is therefore an important step towards understanding the
flexibility and limitations of our new analytical approach.

[1]P. Navarrete, and E.J. Coyle, ``A semi-algebraic approach that enables
the design of inter-grid operators to optimize multigrid convergence,''
Numerical Linear Algebra with Applications, Vol. 15, No. 2-3, pp.
219-247, March 2008.

[2]S. Bandyopadhyay, E.J. Coyle, and T. Falck, ``Stochastic Properties of
Mobility Models in Mobile Ad-Hoc Networks,'' IEEE Transactions on Mobile
Computing, Vol. 6, No. 11, pp. 1218-1229, November 2007.


\end{document}
