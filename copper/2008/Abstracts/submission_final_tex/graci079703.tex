\documentclass{report}
\usepackage{amsmath,amssymb}
\setlength{\parindent}{0mm}
\setlength{\parskip}{1em}
\begin{document}
\begin{center}
\rule{6in}{1pt} \
{\large Jose Luis Gracia \\
{\bf Distributive Smoother for Dominating Grad--Div Problems}}

Department of Applied Mathematics \\ Escuela Universitaria de Ingenier�a T�cnica Industrial \\ Campus R�o Ebro Edificio Torres Quevedo \\ Mar�a de Luna s/n 50018 Zaragoza (Spain)
\\
{\tt jlgracia@unizar.es}\\
Cornelis Oosterlee\\
Francisco Lisbona\\
	Gaspar Francisco\end{center}

Some problems in applied mathematics related, for example, to fluid flow,
solid mechanics, magnetohydrodynamics or electromagnetism problems are
governed by equations in which a grad--div term is dominant. This
circumstance makes the corresponding algebraic equations appearing in the
discrete problem difficult to solve efficiently. In particular, standard
multigrid algorithms are not effective when applied to grad-div
dominating problems because the eigenspace associated to the minimal
eigenvalue of the discrete operator contains many arbitrary oscillatory
eigenvectors. These can neither be reduced by standard smoothing
procedures, nor be well represented on coarse grids


In the finite element context, multigrid methods for this kind of
problems were analyzed by Arnold et al. [1], Hiptmair and Hoppe [2] and
Vassilevski and Wang [3]. In this talk robust distributive smoothers are
proposed for discrete systems of equations that arise when discretizating
with finite differences on staggered grids. This kind of discretization
preserves the main properties of the differential operator in its
discrete form. Similarly to the continuous case, this property permits to
reformulate the discrete problems as a decoupled system where the
Laplacian is the dominant operator.


In particular we study the model problem $-\lambda \, {\rm grad \ div\,}
u +u= f$, where $\lambda$ is a large positive number as well as more
realistic problems as nearly incompressible elasticity and secondary
consolidation Biot's model. Some numerical experiments show that
distributive smoothing methods give small multigrid convergence factors
that are independent of problem parameters and of the mesh sizes in space
and time.


\

\noindent {References:}


\begin{itemize}
\item[{[1]}] D.N. Arnold, R.S. Falk and R. Winther. Preconditioning in
H(div) and applications.
Math. Comput. 1997; 66: 957�984.
\item[{[2]}] R. Hiptmair, R.H.W. Hoppe. Multilevel methods for mixed
finite elements in three
dimensions. Numer. Math. 1999 82: 253�279.
\item[{[3]}] P.S. Vassilevski, J. Wang. Multilevel iterative methods for
mixed finite element
discretizations for elliptic problems. Num. Mathematik 1992; 63; 503-520.
\end{itemize}


\end{document}
