\documentclass{report}
\usepackage{amsmath,amssymb}
\setlength{\parindent}{0mm}
\setlength{\parskip}{1em}
\begin{document}
\begin{center}
\rule{6in}{1pt} \
{\large Csaba Vigh \\
{\bf Efficient 2D Grid Generation for Dynamically Adaptive Problems}}

Institut fuer Informatik V \\ TU Muenchen \\ Boltzmannstr 3 \\ 85748 Garching \\ Germany
\\
{\tt vigh@in.tum.de}\\
Michael Bader\end{center}

This paper presents an approach to efficiently generate and dynamically
refine and coarsen triangular grids for the numerical simulation of
dynamically adaptive problems. The intended application is the simulation
of oceanic wave propagation
(Tsunami simulation, e.g.) based on the shallow water equations. This
special problem requires strong refinement of the grid along the
propagating wavefront and along the domain boundaries (coastline). Our
grid generation approach for this 2D problem is based on the recursive
bisection of triangles along marked edges. The recursive scheme may be
described with a binary refinement tree, which is then traversed
(sequentialized) in depth-first order, such that the traversal follows
(geometrically) a Sierpinski space-filling curve. Refinement and
coarsening is implemented with the use of stack- and stream-like data
structures and can be performed by accessing the linearized refinement
tree, only. Even after refinement and coarsening, the traversal order of
the updated grid still matches the order implied by the Sierpinski curve.
This allows a storage scheme that requires only a minimal amount of
memory (less than 10 bytes per grid cell), and leads to a simulation
process that is inherently cache-efficient. We demonstrate the
computational efficiency of the approach by showing performance results
for several test scenarios. For the numerical simulation we have
implemented explicit and implicit time-stepping schemes, as well as
efficient multilevel solvers for linear systems
arising from implicit time discretization. We will also present an
approach for parallelization and load distribution with space-filling
curves, and show first results of parallel refinement and coarsening for
our test problems.


\end{document}
