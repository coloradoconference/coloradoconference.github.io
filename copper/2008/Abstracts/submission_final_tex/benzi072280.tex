\documentclass{report}
\usepackage{amsmath,amssymb}
\setlength{\parindent}{0mm}
\setlength{\parskip}{1em}
\begin{document}
\begin{center}
\rule{6in}{1pt} \
{\large Michele Benzi \\
{\bf Modified Augmented Lagrangian Methods for the Navier-Stokes Equations}}

Department of Mathematics and Computer Science \\ Emory University \\ 400 Dowman Drive \\ Atlanta \\ GA 30322 \\ USA
\\
{\tt benzi@mathcs.emory.edu}\end{center}

Augmented Lagrangian-based preconditioners have been shown to be highly
effective and robust in the solution of the discrete Stokes and Oseen problem
and in solving the linear systems arising from the stability analysis of
(Newton) linearized solution to incompressible flow problems. So far, however,
the use of this methodology has been limited to discretizations using
structured meshes in simple, two-dimensional geometries. The main challenge
is the (approximate) solution of the velocity subproblem in the augmented
Lagrangian formulation: while effective geometric multigrid methods have been
developed for 2D structured meshes, the question remains open for the case of
3D unstructured meshes. One possible solution to this problem is to approximate
the coefficient matrix of the velocity subproblem in the preconditioner
with a more manageable one, in particular, one for which there exist
efficient algebraic solvers. The problem is how to do this in such a way
that the overall effectiveness of the preconditioner is preserved.
In this talk I will present some variants of the augmented Lagrangian
preconditioner that are applicable to rather general discretizations
and geometries and retain excellent overall robustness and effectiveness.

This is joint work with Alessandro Veneziani (Emory University).


\end{document}
