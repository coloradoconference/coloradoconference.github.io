\documentclass{report}
\usepackage{amsmath,amssymb}
\setlength{\parindent}{0mm}
\setlength{\parskip}{1em}
\begin{document}
\begin{center}
\rule{6in}{1pt} \
{\large Joshua Griffin \\
{\bf Hybrid optimization parallel search package}}

SAS Institute Inc \\ Numerical Optimization R&D \\ 100 SAS Campus Drive \\ Cary \\ NC 27513
\\
{\tt Joshua.Griffin@sas.com}\\
Genetha Gray\\
Tamara Kolda\\
	Herbie Lee, Monica Martinez-Canales, Matt Taddy
	\end{center}

Hybrid optimization in a parallel computing environment is attractive for
several reasons. First, excellent serial algorithms exist that may suffer
from load imbalance when parallelized; this load imbalance can be
exploited by other routines if concurrently running. Second, rather than
stringing a sequence of optimization routines together (in a head to tail
fashion), by running in parallel, individual routines may dynamically
exploit new information immediately, and self-correct when desirable; for
example, a new minimum substantially better than the current best is
encountered. Third, different routines can simultaneously work on
different aspects of the optimization problem itself; a local search
routine may refine the current best point, while a global routine
continues to explore for promising regions.

This talk will focus on the new software package HOPSPACK, developed to
facilitate hybrid optimization in parallel. HOPSPACK manages a queue of
optimization routines ran concurrently while sharing a pool of evaluation
processors. The design is such that existing individual optimization
routines can easily be incorporated using native source code. Because
HOPSPACK is dedicated to solving small-dimensional simulation-base
problems (where evaluations are assumed computationally expensive and
time consuming), an efficient lexicographically ordered function value
cache is utilized to avoid redundant evaluations. The evaluation queue
may be dynamically ordered according to user assigned preferences.
Completed evaluations are ubiquitous and it is the individual solver's
prerogative whether or not to incorporate externally generated
trial-points.


\end{document}
