\documentclass{report}
\usepackage{amsmath,amssymb}
\setlength{\parindent}{0mm}
\setlength{\parskip}{1em}
\begin{document}
\begin{center}
\rule{6in}{1pt} \
{\large Howard C. Elman \\
{\bf Fast Solvers for Models of ICEO Microfluidic Flows}}

Department of Computer Science \\ University of Maryland \\ College Park \\ MD 20817 \\ USA
\\
{\tt elman@cs.umd.edu}\\
Robert R. Shuttleworth\\
Kevin R. Long\\
	Templeton, Jeremy A.\end{center}

We demonstrate the performance of a fast computational algorithm
for modeling the design of a microfluidic mixing device. The mixing
device uses an electrokinetic process, induced charge
electroosmosis, by which a flow through the device is driven by a
set of charged obstacles in it. The device's design is realized by
manipulating the shape and orientation of the obstacles in order to
maximize the amount of fluid mixing within the device. The
computations required in the modeling of the electrokinetic process
entails the solution of a constrained optimization problem in which
function evaluations require the numerical solution of a set of
partial differential equations: a potential equation, the
incompressible Navier-Stokes equations, and a mass transport
equation. The most expensive component of the function evaluation
(which must be performed at every step of an iteration for the
optimization) is the solution of the Navier-Stokes equations. We
show that by using some new robust algorithms for this task, based
on certain preconditioners that take advantage of the structure of
the linearized problem and approximate the Schur complement, this
computation can be done efficiently. Using this computational
strategy, in conjunction with a derivative-free pattern search
algorithm for the optimization, applied to a finite element
discretization of the problem, we are able to determine optimal
configurations of microfluidic devices.


\end{document}
