\documentclass{report}
\usepackage{amsmath,amssymb}
\setlength{\parindent}{0mm}
\setlength{\parskip}{1em}
\begin{document}
\begin{center}
\rule{6in}{1pt} \
{\large Wilbert Weijer \\
{\bf Implicit time stepping and the Ocean General Circulation}}

CCS-2 \\ MS B296 \\ Los Alamos National Laboratory \\ Los Alamos \\ NM 87545
\\
{\tt wilbert@lanl.gov}\\
Katherine Evans\end{center}

The global ocean circulation is a prototypical multi-scale phenomenon:
the large scale circulation is controlled by mesoscale turbulence on
the order of tens of kilometers. Resolving these scales is prohibitive
for performing realistic simulations on time scales that are
climatologically relevant (centuries, millennia).

Recent advances in iterative methods show promise to revolutionize
ocean modeling. However, whether or not implicit techniques will be
able to compete with conventional (semi-)explicit methods strongly
depends on the efficiency of preconditioners.

Here we will report on plans to apply implicit time stepping
techniques to the Parallel Ocean Program (POP), the state-of-the-art
ocean model of Los Alamos National Laboratory.


\end{document}
