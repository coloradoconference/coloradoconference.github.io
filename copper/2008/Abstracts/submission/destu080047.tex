\documentclass{report}
\usepackage{amsmath,amssymb}
\setlength{\parindent}{0mm}
\setlength{\parskip}{1em}
\begin{document}
\begin{center}
\rule{6in}{1pt} \
{\large Eric de Sturler \\
{\bf Recycling Preconditioners and Krylov Subspaces for Quasi-Newton Methods}}

Department of Mathematics \\ 544 McBryde Hall \\ Virginia Tech \\ Blacksburg \\ VA 24061 \\ USA \\ http://www math vt edu/people/sturler/index html
\\
{\tt sturler@vt.edu}\\
Hector Klie\end{center}

For many large systems of nonlinear equations or optimization problems
the computation of the Jacobian (Hessian) may be very expensive. In
addition, potentially large jumps in the coefficients of a system of
partial differential equations and the highly nonlinear nature of some
problems, for example, porous media flow, require robust Newton methods.
As a result, such problems are often solved by Broyden-type methods or
other quasi-Newton algorithms, and, for large problems, the resulting
linear systems are solved using Krylov subspace methods. This leads to a
sequence of related linear systems that change in a structured way
although the changes need not be norm-wise small.

In this presentation we discuss how to solve such problems efficiently
using preconditioned Krylov subspace methods. The key to efficient
solvers is to update and reuse, hence {\em recycle}, preconditioners and
the generated search spaces, and modify the Newton scheme such as to
exploit this optimally.

Updating preconditioners reduces the often high computational cost of
computing the preconditioner while maintaining or even improving the
convergence rate.
We consider very simple right preconditioner updates that can be combined
directly with the recycling of an existing Krylov space or with the
generation of a new Krylov space. Furthermore, we consider various types
of recycling of search spaces to make the solution methods more
efficient. In particular, we focus on an algorithm that combines two
complementary types of recycling. The first type are the Krylov-Secant
solvers proposed by Klie and Wheeler [1]. These methods force secant type
updates to the
Jacobian such that the range of the operator (over a given number of
steps) remains in the Krylov space that already exists. The second type
are the recycling solvers proposed by Parks, de Sturler et al. [2] and
Wang, de Sturler, and Paulino [3] that significantly improve convergence
rates for sequences of linear systems where the matrix changes slowly.

By modifying the (quasi-)Newton method we are able to make multiple
nonlinear iterations with a very modest number of (preconditioned)
matrix-vector products in the linear solver, sometimes none. In
experimental results it also appears that the number of nonlinear steps
where a new accurate Jacobian must be computed is reduced.

\noindent
{\bf REFERENCES}\\

\begin{enumerate}
\item[1] H. Klie and M.F. Wheeler, {\em Nonlinear Krylov-Secant Solvers},
Technical Report ICES 06-02, Institute for Computational Engineering
and Sciences, University of Texas at Austin, January 2006

\item[2] M.L. Parks, E. de Sturler, G. Mackey, D.D. Johnson, S.
Maiti. {\em Recycling Krylov Subspaces for Sequences of Linear Systems}.
SIAM Journal on Scientific Computing, Vol. {\bf 28}, 1651-1674, 2006.

\item[3] S. Wang, E. de Sturler, and G.H. Paulino. {\em Large-Scale
Topology Optimization using Preconditioned Krylov Subspace Methods with
Recycling}. International Journal for Numerical Methods in Engineering,
Vol. {\bf 69}, 2441--2468, 2007.

\end{enumerate}


\end{document}
