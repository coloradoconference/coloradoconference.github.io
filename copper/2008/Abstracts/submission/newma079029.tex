\documentclass{report}
\usepackage{amsmath,amssymb}
\setlength{\parindent}{0mm}
\setlength{\parskip}{1em}
\begin{document}
\begin{center}
\rule{6in}{1pt} \
{\large Christopher Newman \\
{\bf Parallel Preconditioned Newton-Krylov solutions of the Laplace--Beltrami Target Metric (LBTM) smoothing method applied to three-dimensional unstructured meshes}}

Idaho National Laboratory \\ PO Box 1625 \\ Idaho Falls \\ ID 83415-3840
\\
{\tt christopher.newman@inl.gov}\\
Glen Hansen\\
Benoit Forget\end{center}

This presentation examines the performance of selected parallel,
"physics-based" preconditioning methods applied
to the Laplace--Beltrami Target Metric (LBTM) mesh smoothing equations.
These equations constitute a nonlinear elliptic
system of partial differential equations and are primarly used for unstructured mesh
generation on complex geometry. When a target metric mesh improvement
method is used, the three coordinate
equations are coupled through the metric. The weak form of this system
is solved using a standard Jacobian-free Newton--Krylov approach,
employing the NOX, EPETRA
and AZTECOO packages
contained in Sandia National Laboratories TRILINOS project.

Historically the use of Laplace--Beltrami mesh generation methods have been
limited by the scalability and efficiency of linear solvers.
This work focuses on a study
of the effectiveness and parallel scalability of selected "physics-based"
preconditioners applied to this
problem, and presents numerical examples to support this study
and to provide for comparison of the preconditioning strategies.

INL/CON-08-13810 Rev. 0
Work supported by the U.S. Department of Energy Office of Nuclear Energy,
under DOE Idaho Operations Office Contract DE-AC07-05ID14517.


\end{document}
