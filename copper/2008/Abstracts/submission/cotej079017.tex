\documentclass{report}
\usepackage{amsmath,amssymb}
\setlength{\parindent}{0mm}
\setlength{\parskip}{1em}
\begin{document}
\begin{center}
\rule{6in}{1pt} \
{\large Jean C\^ot\'e \\
{\bf Time stepping in atmospheric models from explicit to implicit }}

Recherche en pr\'evision num\'erique (RPN) \\ Environment Canada \\ 2121 Trans-Canada Highway \\ Dorval \\ QC \\ CANADA \\ H9P 1J3
\\
{\tt jean.cote@ec.gc.ca}\end{center}

This talk will present a review of the evolution of the time stepping
methods in atmospheric models from the early operational models to the
current unified models that are used both for operations and research.
The primary motivation for change has been efficiency i.e. the computer
time taken to complete a forecast or simulation at a given level of
accuracy. Operations in national weather centers have strict time
constraints and simulations for research must be done in finite time to
be useful. The available computers at these centers are also part of the
equation, and the future material is likely to influence the time
stepping method of choice of the future.


\end{document}
