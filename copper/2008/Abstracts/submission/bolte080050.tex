\documentclass{report}
\usepackage{amsmath,amssymb}
\setlength{\parindent}{0mm}
\setlength{\parskip}{1em}
\begin{document}
\begin{center}
\rule{6in}{1pt} \
{\large Matthias Bolten \\
{\bf Algebraic considerations regarding the replacement of the Galerkin operator in multigrid methods}}

Institute for Advanced Simulation (IAS) \\ IAS-1: Juelich Supercomputing Centre (JSC) \\ Research Centre Juelich \\ 52425 Juelich \\ Germany
\\
{\tt m.bolten@fz-juelich.de}\end{center}

Most algebraic multigrid methods use the Galerkin operator
\begin{equation*}
A_{C} = R A R^{H}
\end{equation*}
as approximation of the linear system on the coarse level. This
originates from finite element discretizations of the PDEs, where it is
natural to demand this property from the coarse level approximation. The
choice of the Galerkin operator is not mandatory, but it is optimal in
the sense that it fulfills a variational property and a lot of the theory
depends on this fact, which results in the coarse grid correction being
an orthogonal projector.

Nevertheless the use of the Galerkin operator has one big downside. As it
is essentially formed by applying the fine level operator to the
prolongated vector and restricting the result back to the coarse level,
the stencil describing the operator grows. Consider the 5-point
approximation of the Laplacian on a uniform grid in two dimensions. In
that case the Galerkin operator on the coarser level will be represented
by 9 points, resulting in a higher complexity per unknown on that level.
While this is not a big problem for stencils involving only nearest
neighbors, the problem gets more severe for very large and sparse
stencils. We were interested in replacing the Galerkin operator by some
similar operator, which does not posses this behaviour.

For that purpose we analyzed the convergence proofs for algebraic
multigrid methods by Ruge and St\"uben and derived some sufficient
conditions for convergence of multigrid methods not using the Galerkin
operator on the coarse level. It turns out that it is actually possible
to use another operator, that is close to the Galerkin operator and
fulffills some other purely algebraic conditions. We replace the original
coarse grid correction
\begin{equation*}
T = I - R^{H} A_{C}^{-1} R A
\end{equation*}
by another one, that is using an approximation to the Galerkin operator, namely
\begin{equation*}
\hat{T} = I - R^{H} \hat{A}_{C}^{-1} R A.
\end{equation*}
While it is relatively simple to derive sufficient conditions that
$\hat{A}_{C}$ has to fulfill in order to achieve W-cycle convergence,
more attention is required in the V-cycle case. We modified the general
estimate of the V-cycle convergence factor in the work of Ruge and
St\"uben \cite{inc:RUGE87}, which goes back to the work of Mandel
\cite{art:MAND88} and McCormick \cite{art:MCCO85}.

In the the talk an outline over the altered convergence analysis
including a discussion of the sufficient conditions will be given and
some examples will be presented.

\begin{thebibliography}{9}

\bibitem{art:MAND88}
J.~Mandel.
\newblock Algebraic study of multigrid methods for symmetric, definite
problems.
\newblock {\em Appl. Math. Comput.}, 25:39--56, 1988.

\bibitem{art:MCCO85}
S.~F. McCormick.
\newblock Multigrid methods for variational problems: General theory for the
v-cycle.
\newblock {\em SIAM J. Numer. Anal.}, 22(4):634--643, August 1985.

\bibitem{inc:RUGE87}
J.~W. Ruge and K.~St{\"u}ben.
\newblock Algebraic multigrid.
\newblock In S.~F. McCormick, editor, {\em Multigrid methods}, volume~3 of {\em
Frontiers Appl. Math.}, pages 73--130. SIAM, Philadelphia, 1987.

\end{thebibliography}


\end{document}
