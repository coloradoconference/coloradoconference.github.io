\documentclass{report}
\usepackage{amsmath,amssymb}
\setlength{\parindent}{0mm}
\setlength{\parskip}{1em}
\begin{document}
\begin{center}
\rule{6in}{1pt} \
{\large James Nagy \\
{\bf Preconditioning HyBR for Inverse Problems}}

Emory University \\ Math/CS Department \\ 400 Dowman Drive \\ Suite W401 \\ Atlanta \\ GA 30322
\\
{\tt nagy@mathcs.emory.edu}\end{center}

In this talk we consider large scale severely ill-conditioned linear
systems that arise when solving ill-posed inverse problems. Computed
solutions are very sensitive to errors in the data, and regularization
is needed to stabilize the inversion process. There are many forms or
regularization. In this talk we consider an iterative hybrid
bidiagonalization regularization (HyBR) scheme that efficiently
applies Tikhonov regularization. An advantage of the HyBR algorithm
is that it can automatically estimate and refine regularization
parameters at each iteration, and it can estimate a stopping iteration
(both of these issues are very difficult for inverse problems). HyBR
is based on Lanczos bidiagonalization, where the large scale problem
is projected onto a small Krylov subspace. The ``hybrid" part refers
to the fact that regularization is not applied a priori to the large
scale problem, but instead to the small projected problem.
Preconditioning to accelerate convergence of HyBR is very difficult;
standard approaches are likely to magnify data errors in the early
iterations, making it impossible to recover a good approximation of
the desired solution. In this talk we illustrate the difficulties of
preconditioning HyBR, and describe an approach that can be used for
certain classes of inverse problems. Computational examples include
the inverse heat equation, as well as examples from image processing.


\end{document}
