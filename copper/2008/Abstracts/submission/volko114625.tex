\documentclass{report}
\usepackage{amsmath,amssymb}
\setlength{\parindent}{0mm}
\setlength{\parskip}{1em}
\begin{document}
\begin{center}
\rule{6in}{1pt} \
{\large Darko Volkov \\
{\bf Recovery of active faults from surface displacement fields.}}

Department of Mathematical Sciences \\ Worcester Polytechnic Institute \\ 100 Institute Road \\ Worcester \\ MA 01609-2280
\\
{\tt darko@WPI.EDU}\\
I. R. Ionescu\end{center}

Earth science is seeking to understand the physics involved in seismic
activity. It has established that most earthquakes occur near faults. If
we knew the approximate geometry and location of faults, we could perform
numerical simulations of seismic episodes, thus assessing risk in given
areas. Unfortunately only the very rough location of faults is known.
Scientists have speculated that measurements of Earth's surface
displacements could in theory, after adequate data processing, signal the
possible imminence of earthquakes. With the exception of dislocations
resulting from strong earthquakes, these surface displacements are minute.
However, due to recent developments in GPS technology, they can now be
measured with remarkable accuracy.

The goal of this research project is to process measurements of surface
displacements in such a way to use them as data for the inverse problem
consisting of locating faults and portraying their geometry.

We have already entirely solved a two dimensional problem associated to
the strike slip model, which essentially reduces displacement fields to
two dimensional scalar fields. Deriving the inversion method involved a
rigorous mathematical eigenvalue asymptotic analysis, leading to closed
form inversion formulas. Those formulas were then tested for robustness in
numerical simulations. As the strike slip model is limited in scope (it
captures only one of the textbook examples of faults), we have worked on
extending our results to fully three dimensional fault problems. In this
much more difficult case, we have already obtained very promising closed
form formulas (valid for the dominant part of the asymptotic behavior),
and we have tested their use on numerical data.


\end{document}
