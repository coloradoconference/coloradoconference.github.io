\documentclass{report}
\usepackage{amsmath,amssymb}
\setlength{\parindent}{0mm}
\setlength{\parskip}{1em}
\begin{document}
\begin{center}
\rule{6in}{1pt} \
{\large Hillel Tal-Ezer \\
{\bf Computing Matrix Exponentials Arising in Markov Chains by Restarted Krylov Algorithm}}

4 Antokolsky st \\ Tel-Aviv 64044 \\ Israel
\\
{\tt hillel@mta.ac.il}\end{center}

It has been shown that Krylov space approach can result in an efficient
algorithm for computing the transient solution of Markov Chains.
Implementing Krylov algorithm, all the vectors which span the Krylov
space have to be stored. Due to storage restrictions, the common approach
is based on marching in time-steps. The size of the time step is dictated
by the the accuracy required and the amount of Krylov vectors which can
be stored. In this talk we will present an algorithm which eliminates the
need to divide the time interval into time steps. It treats the time
domain as one unit even though the Krylov space is of low dimension. The
solution vector, at any intermediate time level, can be computed for
free. The new algorithm is based on restart approach similar to the one
used for solving linear systems. Using the restarted Krylov algorithm,
the total number of matrix-vector multiplications is reduced
significantly.


\end{document}
