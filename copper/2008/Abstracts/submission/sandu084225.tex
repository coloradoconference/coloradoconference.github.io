\documentclass{report}
\usepackage{amsmath,amssymb}
\setlength{\parindent}{0mm}
\setlength{\parskip}{1em}
\begin{document}
\begin{center}
\rule{6in}{1pt} \
{\large Adrian Sandu \\
{\bf Multirate integration methods for PDEs}}

Computer Science Department \\ Virginia Tech \\ Blacksburg \\ VA 24061
\\
{\tt sandu@cs.vt.edu}\\
Emil Constantinescu\end{center}



Conservative high resolutionmethods with explicit time discretization
have gained widespread popularity for numerically solving conservation
laws. Stability requirements limit the temporal step size, with the upper
bound being determined by the ratio of the temporal and spatial
meshes and the magnitude of the wave speed. The timestep for the entire
domain is restricted by the finest (spatial)mesh resolution or by the highest
wave velocity, and is typically (much) smaller than necessary to accurately
represent other variables in the computational domain.
Implicit, unconditionally stable timestepping algorithms allow large
global timesteps; however, this approach requires the solution of large
(non)linear systems of equations.

We discuss new multirate methods for the solution of conservation laws.In
multirate time integrationmethods, the timestep can vary across the
spatial domain and has to satisfy the CFL condition only locally, resulting
in substantiallymore efficient overall computations. We discuss the
construction of multirate methods that are conservative and preserve the
linear and nonlinear stability.


\end{document}
