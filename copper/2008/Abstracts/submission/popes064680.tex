\documentclass{report}
\usepackage{amsmath,amssymb}
\setlength{\parindent}{0mm}
\setlength{\parskip}{1em}
\begin{document}
\begin{center}
\rule{6in}{1pt} \
{\large Scott R. Pope \\
{\bf Parameter Estimation with a Blood Flow, Blood Pressure Model}}

North Carolina State University \\ Dept of Mathematics \\ 255 Harrelson Hall \\ Box 8205 \\ Raleigh \\ NC 27695
\\
{\tt srpope@ncsu.edu}\\
C. T. Kelley\\
Mette Olufsen\\
	Laura Ellwein, Cheryl Zapata\end{center}

There are many aspects of the human circulatory system that affect blood
flow and blood pressure, but which cannot be measured directly. For
instance, vascular resistance and compliance affect blood flow and
pressure, but neither can be accurately measured without full access to
the veins and arteries. Further complications arise because these
resistances and compliances vary throughout the body. To get estimates
for these parameters, we have developed a lumped parameter model of the
human circulatory system. Our model treats the circulatory system as a
collection of interconnected compartments exchanging blood. For example,
the entire collection of veins in the brain is treated as a single
compartment. A compartment for the heart at the center of the system
drives blood flow. Each compartment has an associated compliance
parameter. Connections between compartments have an associated resistance
parameter. Treating the system as analogous to an electric circuit, we
derive a corresponding ordinary differential equation system dependent
upon our set of unknown parameter values. Finding parameter values that
cause the solution to the ODE system to best fit cerebral blood flow and
arterial blood pressure data collected non-evasively is a nonlinear least
squares problem. Various optimization techniques are used to find
parameter values that minimize the least squares error. We handle several
issues that make our problem difficult. These issues include finding
which parameter values that can be reliably predicted, handling
non-smooth aspects of the model, and various methods of obtaining a
Jacobian.


\end{document}
