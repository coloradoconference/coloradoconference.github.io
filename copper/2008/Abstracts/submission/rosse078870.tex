\documentclass{report}
\usepackage{amsmath,amssymb}
\setlength{\parindent}{0mm}
\setlength{\parskip}{1em}
\begin{document}
\begin{center}
\rule{6in}{1pt} \
{\large Eveline Rosseel \\
{\bf A comparison of iterative solvers for the stochastic finite element method}}

K U Leuven \\ Department of Computer Science \\ Celestijnenlaan 200A \\ B-3001 Leuven \\ Belgium
\\
{\tt Eveline.Rosseel@cs.kuleuven.be}\\
Stefan Vandewalle\end{center}

The stochastic finite element method is an important technique for
solving stochastic partial differential equations (PDEs). This method
approximates the solution of the PDE by a generalized polynomial chaos
expansion. By using a Galerkin projection in the stochastic dimension,
the stochastic PDE is transformed into a coupled set of deterministic
PDEs. A finite element discretization converts the deterministic PDEs
into a high dimensional algebraic system. Specialized iterative solvers
are required to solve this system.

A number of specialized solvers have already been proposed, for example a
conjugate gradient solver with multigrid preconditioning based on
coarsening the spatial domain, or conjugate gradients combined with a
Jacobi-based preconditioner. Here, we shall present an overview of
solution approaches. We start from iterative methods based on a block
splitting of the algebraic system matrices. Next, we extend these methods
for use as preconditioner or for use in a multilevel context. Then, the
various solvers are compared based on their convergence properties,
computational cost and implementation effort.

Our findings are illustrated on two numerical problems. The first is a
steady-state diffusion problem with a discontinuous random field as
diffusion coefficient. The second is a deterministic diffusion problem
defined on a random domain.


\end{document}
