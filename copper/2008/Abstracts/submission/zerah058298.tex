\documentclass{report}
\usepackage{amsmath,amssymb}
\setlength{\parindent}{0mm}
\setlength{\parskip}{1em}
\begin{document}
\begin{center}
\rule{6in}{1pt} \
{\large Gilles Zerah \\
{\bf Large scale parallel ab initio electronic structure calculations with the LOBPCG method.}}

CEA-DAM Ile de France \\ Bruy{\`e}res le Ch{\^a}tel-91297 Arpajon Cedex \\ France
\\
{\tt gilles.zerah@cea.fr}\\
Fran{\,c}ois Bottin\\
St{\'e}phane Le Roux\\
	Knyazev Andrew\end{center}

We present an implementation in the ab initio plane-wave code ABINIT, of
a parallelization scheme based on the locally optimal block
preconditioned conjugate gradient LOBPCG method, and using an optimized
three-dimensional (3D) fast Fourier transform (FFT).

We will first compare, for various systems, the performance of the method
with the more standard eigensolvers currently used in ABINIT.

Next, we present the parallelization scheme, which, in addition to the
standard data partitioning over processors corresponding to different
k-points, relies upon data partitioning with respect to blocks of bands
and Fourier coefficients.

Finally we analyze the performances of the whole scheme on multiprocessor
machines in terms of scalability and convergence speed.


\end{document}
