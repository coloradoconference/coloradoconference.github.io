\documentclass{report}
\usepackage{amsmath,amssymb}
\setlength{\parindent}{0mm}
\setlength{\parskip}{1em}
\begin{document}
\begin{center}
\rule{6in}{1pt} \
{\large Lior Horesh \\
%%-orig: {\bf A multilevel method for $\ell_1$ minimization}
{\bf Optimal Experimental Design of Non-linear Ill-posed Problems Based on
Sparsity Constraints.}
}

400 Dowman Dr \\ Atlanta, GA 30322 \\
{\tt horesh@mathcs.emory.edu}\\
Eldad Haber\end{center}

%-orig: Solving inverse problems using sparsity as a regularization
%-orig: is a new and promising frontier.
%-orig: So far a fixed, single level approach is taken for
%-orig: the solution of the problem.
%-orig: 
%-orig: In this work we present a multilevel approach for the
%-orig: solution of the problem. We generate a multilevel dictionary
%-orig: which adopts to the images at different levels.

The field of optimal experimental design of over-determined
problems is well established, covering a diverse of optimality criteria.
Yet, despite the practical necessity, optimal experimental design of
ill-posed problems and in particular non-linear problems has hardly been
addressed. In this talk we discuss the intrinsic differences between
over-determined and ill-posed experimental design, the complications arising
due to non-linearity, and finally we propose a generic framework for optimal
experimental design of non-linear ill-posed problems. We demonstrate the
effectiveness of our algorithm for several common model problems.


\end{document}
