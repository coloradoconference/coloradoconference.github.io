\documentclass{report}
\usepackage{amsmath,amssymb}
\setlength{\parindent}{0mm}
\setlength{\parskip}{1em}
\begin{document}
\begin{center}
\rule{6in}{1pt} \
{\large Johnathan M. Bardsley \\
{\bf An Efficient Iterative Method for Regularized Poisson Imaging Problems}}

Mathematical Sciences \\ University of Montana \\ Missoula \\ MT 59812-0864 \\ USA
\\
{\tt bardsleyj@mso.umt.edu}\end{center}

Approximating non-Gaussian noise processes with Gaussian models is
standard in data analysis. This is due in large part to the fact that
Gaussian models yield parameter estimation problems of least squares
form, which have been extensively studied both from the theoretical and
computational points of view. In image processing applications, for
example, data is often collected by a CCD camera, in which case the noise
is a Guassian/Poisson mixture with the Poisson noise dominating for a
sufficiently strong signal.

In this talk, we will present an efficient computational method for
minimizing the Poisson negative-log likelihood function resulting from an
accurate CCD camera noise model with a regularization term and a
nonnegativity constraint. We will prove convergence of the method for a
general, convex regularization function. We will then present numerical
results when Tikhonov, total variation, and Laplacian regularization
terms are used. Our results suggest that the iterative method is highly
efficient and yields a computationally viable�and even
advantageous�alternative to standard approaches for unconstrained,
regularized least squares minimization for image reconstruction.

Given the fact images corrupted by Poisson noise are myriad in
applications, e.g. astronomical imaging, and positron emission tomography
(PET) imaging, the use of the Poisson negative-log likelihood function in
image reconstruction is important. Moreover, the addition of a
regularization term both stabilizes the problem and allows for the
natural incorporation of prior information about an image. However this
approach has seen little attention in the literature. The iterative
method that is the subject of this talk gives researchers the ability to
implement this approach in a computationally efficient manner.


\end{document}
