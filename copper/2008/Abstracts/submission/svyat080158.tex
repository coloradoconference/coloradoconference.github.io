\documentclass{report}
\usepackage{amsmath,amssymb}
\setlength{\parindent}{0mm}
\setlength{\parskip}{1em}
\begin{document}
\begin{center}
\rule{6in}{1pt} \
{\large Daniil Svyatskiy \\
{\bf Nonlinear monotone finite volume method for anisotropic diffusion equation}}

Mathematical Modeling and Analysis \\ Theoretical Division \\ Mail Stop B284 \\ Los Alamos National Laboratory \\ Los Alamos \\ NM 87545 \\ U S A
\\
{\tt dasvyat@lanl.gov}\\
Lipnikov Konstantin\\
Shashkov Mikhail\\
	Yuri Vassilevski\end{center}

Predictive numerical simulations of subsurface processes require
not only more sophisticated physical models but also more accurate
and reliable discretization methods for these models.
We study a new monotone finite volume method
for diffusion problems with a heterogeneous anisotropic material tensor. Examples of
anisotropic diffusion includes diffusion in geological formations,
head conduction in structured materials and crystals, image processing,
biological systems, and plasma physics.
Development of a new discretization scheme should be based on the
requirements motivated by both practical implementation and physical
background. This scheme must

\begin{itemize}
\item[-] be locally conservative;
\item[-] be monotone, i.e. preserve positivity of the differential solution;
\item[-] be applicable to unstructured, anisotropic, and severely distorted meshes;
\item[-] allow arbitrary diffusion tensors;
\item[-] result in sparse systems with a minimal number of non-zero entries;
\item[-] have higher than the first-order accuracy for smooth solutions.
\end{itemize}

The discretization
methods used in existing simulations, such as Mixed Finite Element (MFE)
method, Finite Volume (FV) method,
Mimetic Finite Difference (MFD) method, Multi Point Flux Approximation (MPFA) method,
satisfy most of these requirements except the {\bf monotonicity}. They
fail to preserve positivity of a
continuum solution when the diffusion tensor is heterogeneous and
anisotropic or the computational mesh is strongly
perturbed.
Monotonicity is a very important as well as a difficult requirement to satisfy.

The mixed form of the diffusion equation includes the mass conservation equation
and the constitutive equation:
\begin{equation}
{\rm div} \, {\bf q} = Q, \qquad {\bf q} = - D\, {\rm grad}\, C,
\nonumber
\end{equation}
where $ D$ is the diffusion tensor, $Q$ is the source term, and $\bf q$ is the
flux of concentration $C$.


All the methods mentioned above use the same discretization of the mass
conservation equation
and differ by their approximation of the flux (constitutive) equation.
In the nonlinear finite volume scheme a {\it reference} point ${\bf x}_T$ is defined for
each mesh cell $T$ to approximate the concentration $C$. The position of
the reference point
depends on the geometry of $T$ and value of the diffusion tensor.
For isotropic diffusion tensors and triangular cell $T$, the center of
the inscribed circle
is the acceptable position for the reference point.

The flux $\bf q$ is approximated at the middle of each mesh
edge using a weighted difference of concentrations in two neighboring
cells. Nonlinearity comes from the fact that these weights
depend on a solution. Therefore the nonlinear finite volume method
results in a nonlinear algebraic system.
This system is very sparse and the dimension is equal to the number of mesh cells $T$.
For triangular meshes, the matrix of this system has at most four
non-zero elements in each row.
To solve the nonlinear algebraic problem we use the Picard iterative
method which guarantees monotonicity of the
discrete solution for all iterative steps. The convergence of nonlinear
iterations is a challenge problem in the case of highly anisotropic
diffusion.

The computational results demonstrate the flexibility and accuracy of the scheme.
For sufficiently smooth solutions, we achieve the
second-order convergence for concentration $C$ and at least the first-order for flux
${\bf q}$ in a mesh-dependent $L_2$-norm. For non-smooth, highly anisotropic solutions
we observe at least the first-order convergence for both unknowns.
The solution remains {\bf non-negative} on different types of meshes and for different
directions of anisotropy.
Both $RT_0$ and $P_1$ methods produce {\bf negative} values.


\end{document}
