\documentclass{report}
\usepackage{amsmath,amssymb}
\setlength{\parindent}{0mm}
\setlength{\parskip}{1em}
\begin{document}
\begin{center}
\rule{6in}{1pt} \
{\large Ron Morgan \\
{\bf Linear Equations with Multiple Right-hand Sides}}

Math Department \\ Baylor University \\ Waco \\ TX 76798-7328
\\
{\tt Ronald{\_}Morgan@baylor.edu}\\
Dywayne Nicely\end{center}

This talk will concentrate on symmetric and Hermitian systems with
multiple right-hand sides, but work on nonsymmetric problems may also be
mentioned. We first give the Lan-DR method. It is a restarted Lanczos
algorithm that solves a system of linear equations and simultaneously
computes both eigenvalues and eigenvectors. We will discuss the
relationship of Lan-DR to other methods such as GMRES-DR (a deflated
GMRES approach), implicitly restarted Arnoldi, and thick restart Lanczos.
We will give some reorthogonalization approaches, including a combination
of Parlett and Scott�s selective orthogonalization and Simon�s partial
orthogonalization.

For systems of linear equations with multiple right-hand sides, one
approach is to solve the first system with Lan-DR and then use the
eigenvectors generated to assist solving the other right-hand sides. A
deflated conjugate gradient method can be implemented that has a
projection over the eigenvectors followed by CG. We also consider seed
methods for solving multiple right-hand sides and suggest some
improvements. Both deflated CG and the new seed CG will be tested on
Hermitian problems from QCD.


\end{document}
