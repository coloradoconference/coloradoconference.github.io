\documentclass{report}
\usepackage{amsmath,amssymb}
\setlength{\parindent}{0mm}
\setlength{\parskip}{1em}
\begin{document}
\begin{center}
\rule{6in}{1pt} \
{\large Matthew Parno \\
{\bf A Framework for Particle Swarm Optimization with Surrogate Functions}}

Box 6996 \\ Clarkson University \\ Potsdam \\ NY 13699
\\
{\tt parnomd@clarkson.edu}\\
K.R. Fowler\\
T. Hemker\end{center}

Particle swarm optimization (PSO) is a population-based, heuristic
minimization technique that is based on social behavior. The
method has been shown to perform well on a variety of problems
including those with nonconvex, nonsmooth objective functions with
multiple local minima. However, the method can be computationally
expensive since many function calls are required to
advance the swarm at each optimization iteration. This is a
significant drawback when function evaluations depend on output
from an off-the-shelf simulation program, which is often the case
in engineering applications. To this end, we propose a hybrid algorithm
incorporating low fidelity surrogate functions that serve as a more
efficient information sharing medium. Numerical results are given that
show the hybrid approach can improve algorithmic efficiency in a number
of test problems including a difficult hydraulic capture problem.


\end{document}
