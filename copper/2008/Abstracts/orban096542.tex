\documentclass{report}
\usepackage{amsmath,amssymb}
\setlength{\parindent}{0mm}
\setlength{\parskip}{1em}
\begin{document}
\begin{center}
\rule{6in}{1pt} \
{\large Dominique Orban \\
{\bf Projected Krylov Methods for the Solution of Unsymmetric Augmented Systems}}

GERAD and Ecole Polytechnique de Montreal \\ Department of Mathematics & Industrial Engineering \\ CP 6079 \\ Succ Centre-Ville \\ Montreal H3C 3A7 \\ Quebec \\ Canada
\\
{\tt dominique.orban@gerad.ca}\end{center}

\newcommand{\eqn}[2]{%
\begin{equation}
\label{#1}
{#2}
\end{equation}
}
\newcommand{\req}[1]{(\ref{#1})}
\newcommand{\sub}[1]{_{{\mbox {\tiny #1}}}}
\newcommand{\half}{{\textstyle \frac{1}{2}}}
\newcommand{\T}{^T\!}
\newcommand{\R}{{\mathbb R}}
\newcommand{\minim}{\mathop{\hbox{minimize}}}
\newcommand{\minimize}[1]{\displaystyle\minim_{#1}}
\newcommand{\st}{\mathop{\hbox{subject to}}}
\newcommand{\vect}[1]{
\left[
\begin{array}{c}
#1
\end{array}
\right]
}

We consider the iterative solution of systems of the form
\eqn{eq:augmented}{
\begin{bmatrix}
A & B^T \\
B & 0
\end{bmatrix}
\vect{u \\ p} = \vect{b \\ d},
}
where $A \in \R^{n \times n}$ is square but not necessarily symmetric, $B \in
\R^{m \times n}$, $b \in \R^n$ and $d \in \R^m$. In a fluid flow context,
systems such as \req{eq:augmented} arise in Navier-Stokes
iterations when considering the flow of two or more immiscible fluids through a
cavity and must be solved to compute a correction $(u,p)$ in the velocity and
pressure fields. The large size of such problems preclude a direct
factorization of the coefficient matrix of \req{eq:augmented}. On the other
hand, iterative methods applied to \req{eq:augmented} usually perform very
poorly. We are particularly interested in the case where $A$ may not be
assembled explicitly but rather, matrix-vector products with $A$ may be
obtained by calling a function. We assume that $B$ is available explicitly.

Whenever $A$ is symmetric and positive definite,
\req{eq:augmented} represents the first-order optimality conditions of the
equality-constrained quadratic program
\eqn{eq:qp}{
\setlength{\arraycolsep}{.75em}
\begin{array}{ll}
\minimize{u \in \R^n} & -b\T u + \half u\T A u \\
\st & B u = d.
\end{array}
}
In this type of application, the density of the coefficient matrix is mostly
due to $A$. Assuming that $B$ has full row rank, it is often feasible to
compute a factorization of the {\em projection matrix}
\eqn{eq:projector}{
\begin{bmatrix}
G & B^T \\
B & 0
\end{bmatrix},
}
where $G$ is a sparse symmetric approximation to $A$ that is positive definite
over the nullspace of $B$. A particularly efficient iterative method for
\req{eq:qp} is then the projected preconditioned conjugate gradient algorithm
often used in
nonlinear optimization contexts. This method typically requires the
factorization of a single projection matrix and one matrix-vector product with
$A$ per iteration.

We examine similar Krylov-type iterations for the case where $A$
is unsymmetric and present a methodology by which to derive projected Krylov
methods for systems of the form \req{eq:augmented}. We concentrate on the {\sc
b}i-{\sc cgstab} and {\sc tfqmr} families of methods. The methods we consider
are akin to so-called projection methods,
which are
sometimes regarded as being too expensive and only effective on systems in
which $A$ is diagonally dominant. We hope that this
paper will correct that reputation by showing that efficient projections
combined with the appropriate Krylov iteration make for a very competitive
numerical method.

Let $Z \in \R^{q \times n}$ be a matrix whose rows form a basis for the
nullspace of $B$. Any solution $u^*$ to \req{eq:augmented} may be written $u^*
= Z u^*\sub{Z} + B\T u^*\sub{B}$, so that \req{eq:augmented} yields $B B\T
u^*\sub{B} = d$, which uniquely determines $u^*\sub{B}$ and leaves $u^*\sub{Z}$
as a solution to $Z\T A Z u^*\sub{Z} = Z^T (b - A B\T u^*\sub{B})$. Applying
any Krylov method to the latter system with a preconditioner of the form $M =
Z\T G Z$, where $G$ is such that $M$ is positive definite, is equivalent to
applying the same Krylov method with $A$ as coefficient matrix, with
preconditioning steps replaced by projections computed via \req{eq:projector},
and without recourse to computing $Z$. Special care must be observed in an
implementation of this scheme as severe numerical cancellation likely occurs,
especially in the projection steps. We will present a remedy to this difficulty.

In our implementation, factorization of the projection matrix is performed by the
multi-frontal symmetric indefinite {\sc ma57} from the Harwell Subroutine
Library. The stopping test implemented by default is triggered when the
projected {\sc b}i-{\sc cg} residual vector $s_k$ is small or when the residual
vector $r_k$ satisfies a Galerkin-type condition, or if the total number of
matrix-vector products exceeds $2 n_A$, where $n_A$ is the order of the $(1,1)$
block matrix $A$. We compare the projected {\sc b}i-{\sc cgstab} approach with
a direct LU factorization of \req{eq:augmented}. The comparison is based on the
total solution time and memory requirements. The LU factorization is realized
by means of the {\sc umfpack} package.

We present results on systems arising from the discretization of
Navier-Stokes equations for the flow of one or more immiscible
incompressible fuilds. Some test cases involve a potentially moving
obstacle taken into account by the fictitious domains method.


\end{document}
